\documentclass{article}

\usepackage[T1]{fontenc}
\usepackage[osf]{libertine}
\usepackage[scaled=0.8]{beramono}
\usepackage[margin=1.5in]{geometry}
\usepackage{url}
\usepackage{amsmath}
\usepackage{amssymb}
\usepackage{nicefrac}
\usepackage{microtype}
\usepackage{subcaption}
\usepackage{bm}

\usepackage{sectsty}
\sectionfont{\large}
\subsectionfont{\normalsize}

\usepackage{titlesec}
\titlespacing{\section}{0pt}{10pt plus 2pt minus 2pt}{0pt plus 2pt minus 0pt}
\titlespacing{\subsection}{0pt}{5pt plus 2pt minus 2pt}{0pt plus 2pt minus 0pt}

\usepackage{pgfplots}
\pgfplotsset{
  compat=newest,
  plot coordinates/math parser=false,
  tick label style={font=\footnotesize, /pgf/number format/fixed},
  label style={font=\small},
  legend style={font=\small},
  every axis/.append style={
    tick align=outside,
    clip mode=individual,
    scaled ticks=false,
    thick,
    tick style={semithick, black}
  }
}

\pgfkeys{/pgf/number format/.cd, set thousands separator={\,}}

\usepgfplotslibrary{external}
\tikzexternalize[prefix=tikz/]

\newlength\figurewidth
\newlength\figureheight

\setlength{\figurewidth}{12cm}
\setlength{\figureheight}{6cm}

\newlength\squarefigurewidth
\newlength\squarefigureheight

\setlength{\squarefigurewidth}{4cm}
\setlength{\squarefigureheight}{4cm}

\newlength\smallsquarefigurewidth
\newlength\smallsquarefigureheight

\setlength{\smallsquarefigurewidth}{3.25cm}
\setlength{\smallsquarefigureheight}{3.25cm}

\newlength\smallfigurewidth
\newlength\smallfigureheight

\setlength{\smallfigurewidth}{6.25cm}
\setlength{\smallfigureheight}{4cm}

\setlength{\parindent}{0pt}
\setlength{\parskip}{1ex}

\newcommand{\acro}[1]{\textsc{\MakeLowercase{#1}}}
\newcommand{\given}{\mid}
\newcommand{\mc}[1]{\mathcal{#1}}
\newcommand{\data}{\mc{D}}
\newcommand{\intd}[1]{\,\mathrm{d}{#1}}
\newcommand{\inv}{^{-1}}
\newcommand{\trans}{^\top}
\newcommand{\mat}[1]{\bm{\mathrm{#1}}}
\renewcommand{\vec}[1]{\bm{\mathrm{#1}}}
\newcommand{\R}{\mathbb{R}}
\renewcommand{\epsilon}{\varepsilon}

\DeclareMathOperator{\var}{var}
\DeclareMathOperator{\cov}{cov}
\DeclareMathOperator{\diag}{diag}
\DeclareMathOperator*{\argmin}{arg\,min}
\DeclareMathOperator*{\argmax}{arg\,max}

\begin{document}

\section*{The Kernel Trick}

Consider the assumption of linear regression with an explicit feature
expansion $\phi\colon \vec{x} \mapsto \phi(\vec{x})$:
\begin{equation*}
  y(\vec{x}) = \phi(\vec{x})\trans \vec{w} + \epsilon(\vec{x}).
\end{equation*}
Given training data $\data = \bigl\{ (\vec{x}, y) \bigr\} = (\mat{X},
\vec{y})$, recall we modeled the residuals $\epsilon(\vec{x})$ as
zero-mean independent, identically distributed Gaussians with variance
$\sigma^2$:
\begin{equation*}
  p(\vec{\epsilon})
  =
  \mc{N}(\vec{\epsilon}; \vec{0}, \sigma^2 \mat{I}),
\end{equation*}
giving rise to the following likelihood:
\begin{equation*}
  p(\vec{y} \given \mat{X}, \vec{w}, \sigma^2)
  =
  \mc{N}(\vec{y}; \mat{X}\vec{w}, \sigma^2 \mat{I}).
\end{equation*}

In Bayesian linear regression, we further chose a multivariate
Gaussian prior for $\vec{w}$:
\begin{equation*}
  p(\vec{w}) = \mc{N}(\vec{w}; \vec{\mu}, \mat{\Sigma}).
\end{equation*}
For simplicity, in the below we assume the prior mean for $\vec{w}$ is
$\vec{\mu} = \vec{0}$.

Given these assumptions, we were able to derive the posterior
distribution of $\vec{w}$ given the data $\data$:
\begin{equation*}
  p(\vec{w} \given \data, \sigma^2)
  =
  \mc{N}(\vec{w};
  \vec{\mu}_{\vec{w}\given\data},
  \mat{\Sigma}_{\vec{w}\given\data}
  ),
\end{equation*}
where
\begin{align*}
  \vec{\mu}_{\vec{w}\given\data}
  &=
  \mat{\Sigma}
  \mat{\Phi}\trans
  (\mat{\Phi}\mat{\Sigma}\mat{\Phi}\trans + \sigma^2 \mat{I})\inv
  \vec{y};
  \\
  \mat{\Sigma}_{\vec{w}\given\data}
  &=
  \mat{\Sigma}
  -
  \mat{\Sigma}
  \mat{\Phi}\trans
  (\mat{\Phi}\mat{\Sigma}\mat{\Phi}\trans + \sigma^2 \mat{I})\inv
  \mat{\Phi}
  \mat{\Sigma},
\end{align*}
where we have defined $\mat{\Phi} = \phi(\mat{X})$.

If we wish to use our model to predict the outputs $\vec{y}_\ast$
associated with a set of inputs $\mat{X}_\ast$, we previously derived:
\begin{equation*}
  p(\vec{y}_\ast \given \mat{X}_\ast, \data, \sigma^2)
  =
  \mc{N}(
  \vec{y}_\ast;
  \mat{\Phi}_\ast \vec{\mu}_{\vec{w}\given\data},
  \mat{\Phi}_\ast \mat{\Sigma}_{\vec{w}\given\data} \mat{\Phi}_\ast\trans + \sigma^2 \mat{I}).
\end{equation*}
Examining the forms of these expressions, we see that the feature
expansion $\phi$ always appears in one of the following expressions:
\begin{equation*}
  \mat{\Phi}\mat{\Sigma}\mat{\Phi}\trans
  \qquad
  \mat{\Phi}_\ast\mat{\Sigma}\mat{\Phi}\trans
  \qquad
  \mat{\Phi}\mat{\Sigma}\mat{\Phi}_\ast\trans
  \qquad
  \mat{\Phi}_\ast\mat{\Sigma}\mat{\Phi}_\ast\trans.
\end{equation*}
The entries of these matrices are always of the form
$\phi(\vec{x})\trans \mat{\Sigma} \phi(\vec{x}')$, where $\vec{x}$ and
$\vec{x}'$ are two arbitrary inputs.  To simply our expressions, we
define a function
\begin{equation*}
  K(\vec{x}, \vec{x}')
  =
  \phi(\vec{x})\trans\mat{\Sigma}\phi(\vec{x}').
\end{equation*}
Because $\mat{\Sigma}$ is positive definite, it has a ``matrix square
root,'' $\mat{\Sigma}^{\nicefrac{1}{2}}$ with the property
$(\mat{\Sigma}^{\nicefrac{1}{2}})^2 = \mat{\Sigma}$.\footnote{You can
  prove this via the singular value decomposition (\acro{SVD}): write
  $\mat{\Sigma} = \mat{U}\mat{D}\mat{U}\trans$, where $\mat{U}$ is
  unitary and $\mat{D}$ is diagonal with positive entries (because
  $\mat{\Sigma}$ is positive definite), then define
  $\mat{\Sigma}^{\nicefrac{1}{2}} =
  \mat{U}\mat{D}^{\nicefrac{1}{2}}\mat{U}\trans$.} If we define the
function $\psi(\vec{x}) =
\mat{\Sigma}^{\nicefrac{1}{2}}\phi(\vec{x})$, we can see that
$K$ is simply an inner product:
\begin{equation*}
  K(\vec{x}, \vec{x}')
  =
  \psi(\vec{x})\trans \psi(\vec{x}').
\end{equation*}
Such a function $K$ is guaranteed to always produce positive-definite
Gram matrices (a \emph{Gram matrix} is a square matrix of inner
products between pairs of elements), and is called a \emph{kernel} or
\emph{covariance function.}

Sometimes it is possible to specify a covariance function $K$ directly
without ever computing the feature map explicitly.  With such a
function, we could perform efficient Bayesian linear regression even
with a high-dimensional (or even infinite dimensional!) feature
expansion $\phi$ \emph{implicitly.} This idea of computing inner
products in a feature space directly is called the \emph{kernel trick}
and has been the basis of a large amount of work in the
machine-learning community.  Effectively, any algorithm that operates
purely in terms of inner products between input vectors can be made
nonlinear by replacing normal inner products with the evaluation of a
kernel.

With this definition, we may rewrite the predictive distribution
for $\vec{y}_\ast$:
\begin{equation*}
  p(\vec{y}_\ast \given \mat{X}_\ast, \data, \sigma^2)
  =
  \mc{N}(
  \vec{y}_\ast;
  \mu_{\vec{y}_\ast \given \data},
  K_{\vec{y}_\ast \given \data}),
\end{equation*}
where
\begin{align*}
  \vec{\mu}_{\vec{y}_\ast\given\data}
  &=
  K(\mat{X}_\ast, \mat{X})
  \bigl(K(\mat{X}, \mat{X}) + \sigma^2 \mat{I}\bigr)\inv
  \vec{y};
  \\
  K_{\vec{y}_\ast\given\data}
  &=
  K(\vec{X}_\ast, \vec{X}_\ast)
  -
  K(\mat{X}_\ast, \mat{X})
  \bigl(K(\mat{X}, \mat{X}) + \sigma^2 \mat{I}\bigr)\inv
  K(\mat{X}, \mat{X}_\ast).
\end{align*}

\subsection*{Examples}

Perhaps the most-commonly used kernel is the \emph{squared exponential
  covariance function:}
\begin{equation*}
  K(\vec{x}, \vec{x}'; \lambda, \ell)
  =
  \lambda^2
  \exp\biggl(-\frac{\lVert \vec{x} - \vec{x}' \rVert^2}{2\ell^2}\biggr),
\end{equation*}
where $\lambda$ and $\ell$ are parameters.  The former is simply a
multiplicative scaling constant (you can think of this as an implicit
scalar multiplication in the implicit feature map $\phi$).  The latter
takes the role of a \emph{length scale;} vectors separated by more
than a couple length scales will have a kernel value near zero.

An example of Bayesian linear regression using this kernel function is
shown in Figure \ref{kernel_example}.  We see that the use of this
kernel function allowed us to achieve nice nonlinear regression
without computing explicit basis expansions.  In fact, you can show
that the squared exponential kernel corresponds to a
\emph{infinite-dimensional} basis expansion, where we use a Gaussian
basis function \emph{centered on every point.}  Such a feature
expansion would be impossible to use if we attempted to use explicit
feature computation.

\begin{figure}
  \centering
  % This file was created by matlab2tikz.
% Minimal pgfplots version: 1.3
%
\tikzsetnextfilename{kernel_example}
\definecolor{mycolor1}{rgb}{0.65098,0.80784,0.89020}%
\definecolor{mycolor2}{rgb}{0.12157,0.47059,0.70588}%
\definecolor{mycolor3}{rgb}{0.89020,0.10196,0.10980}%
%
\begin{tikzpicture}

\begin{axis}[%
width=0.95092\figurewidth,
height=\figureheight,
at={(0\figurewidth,0\figureheight)},
scale only axis,
xmin=-4,
xmax=4,
xlabel={$x$},
ymin=-2.5,
ymax=2,
ylabel={$y$},
axis x line*=bottom,
axis y line*=left,
legend style={at={(0.97,0.03)},anchor=south east,legend cell align=left,align=left,fill=none,draw=none},
reverse legend
]

\addplot[area legend,solid,fill=mycolor1,opacity=8.000000e-01,draw=none]
table[row sep=crcr] {%
x	y\\
-4	-2.00484005152404\\
-3.99199199199199	-2.00384092509804\\
-3.98398398398398	-2.0028049933207\\
-3.97597597597598	-2.0017315591492\\
-3.96796796796797	-2.00061992294648\\
-3.95995995995996	-1.999469382749\\
-3.95195195195195	-1.99827923454117\\
-3.94394394394394	-1.9970487725365\\
-3.93593593593594	-1.99577728946535\\
-3.92792792792793	-1.99446407686919\\
-3.91991991991992	-1.99310842540142\\
-3.91191191191191	-1.99170962513453\\
-3.9039039039039	-1.99026696587367\\
-3.8958958958959	-1.98877973747645\\
-3.88788788788789	-1.9872472301789\\
-3.87987987987988	-1.98566873492753\\
-3.87187187187187	-1.98404354371737\\
-3.86386386386386	-1.98237094993589\\
-3.85585585585586	-1.98065024871274\\
-3.84784784784785	-1.97888073727502\\
-3.83983983983984	-1.97706171530826\\
-3.83183183183183	-1.97519248532272\\
-3.82382382382382	-1.97327235302496\\
-3.81581581581582	-1.97130062769466\\
-3.80780780780781	-1.96927662256635\\
-3.7997997997998	-1.96719965521609\\
-3.79179179179179	-1.96506904795285\\
-3.78378378378378	-1.96288412821444\\
-3.77577577577578	-1.96064422896783\\
-3.76776776776777	-1.95834868911381\\
-3.75975975975976	-1.95599685389561\\
-3.75175175175175	-1.95358807531149\\
-3.74374374374374	-1.95112171253106\\
-3.73573573573574	-1.94859713231507\\
-3.72772772772773	-1.94601370943865\\
-3.71971971971972	-1.94337082711766\\
-3.71171171171171	-1.94066787743798\\
-3.7037037037037	-1.93790426178764\\
-3.6956956956957	-1.93507939129149\\
-3.68768768768769	-1.93219268724814\\
-3.67967967967968	-1.9292435815692\\
-3.67167167167167	-1.92623151722031\\
-3.66366366366366	-1.92315594866395\\
-3.65565565565566	-1.92001634230373\\
-3.64764764764765	-1.91681217692992\\
-3.63963963963964	-1.91354294416602\\
-3.63163163163163	-1.91020814891615\\
-3.62362362362362	-1.90680730981298\\
-3.61561561561562	-1.903339959666\\
-3.60760760760761	-1.89980564590986\\
-3.5995995995996	-1.89620393105257\\
-3.59159159159159	-1.89253439312329\\
-3.58358358358358	-1.88879662611948\\
-3.57557557557558	-1.88499024045312\\
-3.56756756756757	-1.88111486339582\\
-3.55955955955956	-1.87717013952249\\
-3.55155155155155	-1.87315573115336\\
-3.54354354354354	-1.86907131879409\\
-3.53553553553554	-1.86491660157372\\
-3.52752752752753	-1.86069129768012\\
-3.51951951951952	-1.85639514479286\\
-3.51151151151151	-1.85202790051304\\
-3.5035035035035	-1.8475893427899\\
-3.4954954954955	-1.84307927034406\\
-3.48748748748749	-1.83849750308692\\
-3.47947947947948	-1.83384388253613\\
-3.47147147147147	-1.82911827222677\\
-3.46346346346346	-1.82432055811809\\
-3.45545545545546	-1.81945064899535\\
-3.44744744744745	-1.81450847686681\\
-3.43943943943944	-1.80949399735532\\
-3.43143143143143	-1.80440719008438\\
-3.42342342342342	-1.7992480590585\\
-3.41541541541542	-1.79401663303744\\
-3.40740740740741	-1.78871296590424\\
-3.3993993993994	-1.7833371370267\\
-3.39139139139139	-1.77788925161205\\
-3.38338338338338	-1.77236944105474\\
-3.37537537537538	-1.76677786327688\\
-3.36736736736737	-1.76111470306133\\
-3.35935935935936	-1.75538017237696\\
-3.35135135135135	-1.74957451069619\\
-3.34334334334334	-1.74369798530425\\
-3.33533533533534	-1.73775089160024\\
-3.32732732732733	-1.73173355338954\\
-3.31931931931932	-1.72564632316764\\
-3.31131131131131	-1.71948958239491\\
-3.3033033033033	-1.71326374176236\\
-3.2952952952953	-1.70696924144814\\
-3.28728728728729	-1.70060655136453\\
-3.27927927927928	-1.69417617139543\\
-3.27127127127127	-1.687678631624\\
-3.26326326326326	-1.68111449255052\\
-3.25525525525526	-1.67448434530017\\
-3.24724724724725	-1.66778881182073\\
-3.23923923923924	-1.66102854506999\\
-3.23123123123123	-1.65420422919291\\
-3.22322322322322	-1.64731657968831\\
-3.21521521521522	-1.64036634356508\\
-3.20720720720721	-1.63335429948793\\
-3.1991991991992	-1.62628125791251\\
-3.19119119119119	-1.61914806120995\\
-3.18318318318318	-1.61195558378083\\
-3.17517517517518	-1.60470473215858\\
-3.16716716716717	-1.59739644510225\\
-3.15915915915916	-1.5900316936789\\
-3.15115115115115	-1.58261148133549\\
-3.14314314314314	-1.57513684396042\\
-3.13513513513514	-1.56760884993495\\
-3.12712712712713	-1.56002860017447\\
-3.11911911911912	-1.55239722815988\\
-3.11111111111111	-1.54471589995928\\
-3.1031031031031	-1.53698581424022\\
-3.0950950950951	-1.52920820227266\\
-3.08708708708709	-1.52138432792302\\
-3.07907907907908	-1.51351548763968\\
-3.07107107107107	-1.50560301043016\\
-3.06306306306306	-1.4976482578305\\
-3.05505505505506	-1.48965262386729\\
-3.04704704704705	-1.48161753501265\\
-3.03903903903904	-1.47354445013304\\
-3.03103103103103	-1.46543486043209\\
-3.02302302302302	-1.45729028938843\\
-3.01501501501502	-1.44911229268898\\
-3.00700700700701	-1.44090245815863\\
-2.998998998999	-1.432662405687\\
-2.99099099099099	-1.42439378715324\\
-2.98298298298298	-1.41609828634984\\
-2.97497497497497	-1.40777761890654\\
-2.96696696696697	-1.39943353221529\\
-2.95895895895896	-1.39106780535779\\
-2.95095095095095	-1.38268224903663\\
-2.94294294294294	-1.37427870551166\\
-2.93493493493493	-1.36585904854305\\
-2.92692692692693	-1.3574251833427\\
-2.91891891891892	-1.34897904653587\\
-2.91091091091091	-1.34052260613484\\
-2.9029029029029	-1.33205786152682\\
-2.89489489489489	-1.32358684347821\\
-2.88688688688689	-1.31511161415781\\
-2.87887887887888	-1.30663426718138\\
-2.87087087087087	-1.29815692768058\\
-2.86286286286286	-1.28968175239912\\
-2.85485485485485	-1.28121092981952\\
-2.84684684684685	-1.27274668032393\\
-2.83883883883884	-1.26429125639283\\
-2.83083083083083	-1.25584694284562\\
-2.82282282282282	-1.24741605712762\\
-2.81481481481481	-1.23900094964811\\
-2.80680680680681	-1.23060400417448\\
-2.7987987987988	-1.22222763828817\\
-2.79079079079079	-1.21387430390804\\
-2.78278278278278	-1.20554648788775\\
-2.77477477477477	-1.19724671269398\\
-2.76676676676677	-1.18897753717263\\
-2.75875875875876	-1.18074155741126\\
-2.75075075075075	-1.17254140770609\\
-2.74274274274274	-1.16437976164257\\
-2.73473473473473	-1.15625933329966\\
-2.72672672672673	-1.14818287858808\\
-2.71871871871872	-1.14015319673375\\
-2.71071071071071	-1.13217313191861\\
-2.7027027027027	-1.12424557509149\\
-2.69469469469469	-1.11637346596273\\
-2.68668668668669	-1.10855979519697\\
-2.67867867867868	-1.10080760681944\\
-2.67067067067067	-1.09312000085196\\
-2.66266266266266	-1.08550013619541\\
-2.65465465465465	-1.07795123377659\\
-2.64664664664665	-1.07047657997748\\
-2.63863863863864	-1.06307953036599\\
-2.63063063063063	-1.05576351374723\\
-2.62262262262262	-1.04853203655445\\
-2.61461461461461	-1.0413886875987\\
-2.60660660660661	-1.03433714319535\\
-2.5985985985986	-1.02738117268443\\
-2.59059059059059	-1.02052464435966\\
-2.58258258258258	-1.01377153181823\\
-2.57457457457457	-1.00712592073923\\
-2.56656656656657	-1.000592016093\\
-2.55855855855856	-0.994174149776913\\
-2.55055055055055	-0.987876788663024\\
-2.54254254254254	-0.98170454303171\\
-2.53453453453453	-0.975662175349901\\
-2.52652652652653	-0.969754609333881\\
-2.51851851851852	-0.963986939213197\\
-2.51051051051051	-0.958364439083921\\
-2.5025025025025	-0.952892572204989\\
-2.49449449449449	-0.947577000049529\\
-2.48648648648649	-0.942423590874422\\
-2.47847847847848	-0.937438427512949\\
-2.47047047047047	-0.932627814028061\\
-2.46246246246246	-0.927998280787036\\
-2.45445445445445	-0.92355658743108\\
-2.44644644644645	-0.919309723117084\\
-2.43843843843844	-0.915264903305611\\
-2.43043043043043	-0.911429562260117\\
-2.42242242242242	-0.907811340313891\\
-2.41441441441441	-0.904418064858976\\
-2.40640640640641	-0.901257723924578\\
-2.3983983983984	-0.898338431152827\\
-2.39039039039039	-0.895668380964006\\
-2.38238238238238	-0.893255792746496\\
-2.37437437437437	-0.891108843032176\\
-2.36636636636637	-0.889235584843922\\
-2.35835835835836	-0.887643853750888\\
-2.35035035035035	-0.886341160652424\\
-2.34234234234234	-0.885334571941287\\
-2.33433433433433	-0.884630578465691\\
-2.32632632632633	-0.884234955592305\\
-2.31831831831832	-0.884152617624983\\
-2.31031031031031	-0.884387470788029\\
-2.3023023023023	-0.884942269847953\\
-2.29429429429429	-0.885818484120293\\
-2.28628628628629	-0.887016178980388\\
-2.27827827827828	-0.888533918973085\\
-2.27027027027027	-0.890368698130822\\
-2.26226226226226	-0.892515902144442\\
-2.25425425425425	-0.894969305624137\\
-2.24624624624625	-0.897721105937935\\
-2.23823823823824	-0.90076199317439\\
-2.23023023023023	-0.904081253823064\\
-2.22222222222222	-0.907666903993171\\
-2.21421421421421	-0.911505846560968\\
-2.20620620620621	-0.915584045670436\\
-2.1981981981982	-0.919886711568211\\
-2.19019019019019	-0.924398488827857\\
-2.18218218218218	-0.929103641544802\\
-2.17417417417417	-0.933986229960949\\
-2.16616616616617	-0.939030274081272\\
-2.15815815815816	-0.944219901046533\\
-2.15015015015015	-0.949539474217049\\
-2.14214214214214	-0.95497370301358\\
-2.13413413413413	-0.960507733494481\\
-2.12612612612613	-0.96612722039319\\
-2.11811811811812	-0.971818381890183\\
-2.11011011011011	-0.97756803875941\\
-2.1021021021021	-0.983363639734272\\
-2.09409409409409	-0.989193275010955\\
-2.08608608608609	-0.995045679779012\\
-2.07807807807808	-1.00091022956623\\
-2.07007007007007	-1.00677692903682\\
-2.06206206206206	-1.01263639570491\\
-2.05405405405405	-1.01847983983932\\
-2.04604604604605	-1.02429904164914\\
-2.03803803803804	-1.03008632666424\\
-2.03003003003003	-1.03583454006243\\
-2.02202202202202	-1.04153702055046\\
-2.01401401401401	-1.04718757427945\\
-2.00600600600601	-1.05278044916721\\
-1.997997997998	-1.05831030990747\\
-1.98998998998999	-1.06377221387081\\
-1.98198198198198	-1.06916158803914\\
-1.97397397397397	-1.07447420706531\\
-1.96596596596597	-1.07970617250966\\
-1.95795795795796	-1.08485389327359\\
-1.94994994994995	-1.08991406722627\\
-1.94194194194194	-1.09488366400246\\
-1.93393393393393	-1.09975990893616\\
-1.92592592592593	-1.10454026808508\\
-1.91791791791792	-1.10922243429509\\
-1.90990990990991	-1.11380431424955\\
-1.9019019019019	-1.11828401644682\\
-1.89389389389389	-1.12265984004836\\
-1.88588588588589	-1.12693026454077\\
-1.87787787787788	-1.13109394015638\\
-1.86986986986987	-1.13514967899864\\
-1.86186186186186	-1.13909644682155\\
-1.85385385385385	-1.14293335541424\\
-1.84584584584585	-1.14665965554477\\
-1.83783783783784	-1.15027473042023\\
-1.82982982982983	-1.15377808962225\\
-1.82182182182182	-1.15716936348048\\
-1.81381381381381	-1.16044829784832\\
-1.80580580580581	-1.16361474924832\\
-1.7977977977978	-1.16666868035629\\
-1.78978978978979	-1.16961015579586\\
-1.78178178178178	-1.17243933821669\\
-1.77377377377377	-1.17515648463154\\
-1.76576576576577	-1.17776194298926\\
-1.75775775775776	-1.1802561489617\\
-1.74974974974975	-1.18263962292458\\
-1.74174174174174	-1.18491296711277\\
-1.73373373373373	-1.18707686293236\\
-1.72572572572573	-1.18913206841187\\
-1.71771771771772	-1.1910794157766\\
-1.70970970970971	-1.19291980913031\\
-1.7017017017017	-1.19465422222881\\
-1.69369369369369	-1.19628369633133\\
-1.68568568568569	-1.19780933811468\\
-1.67767767767768	-1.19923231763652\\
-1.66966966966967	-1.20055386633364\\
-1.66166166166166	-1.2017752750413\\
-1.65365365365365	-1.20289789201991\\
-1.64564564564565	-1.20392312097499\\
-1.63763763763764	-1.2048524190563\\
-1.62962962962963	-1.20568729482208\\
-1.62162162162162	-1.20642930615376\\
-1.61361361361361	-1.20708005810648\\
-1.60560560560561	-1.20764120068063\\
-1.5975975975976	-1.20811442649907\\
-1.58958958958959	-1.20850146837474\\
-1.58158158158158	-1.20880409675307\\
-1.57357357357357	-1.20902411701358\\
-1.56556556556557	-1.20916336661493\\
-1.55755755755756	-1.20922371206794\\
-1.54954954954955	-1.20920704572127\\
-1.54154154154154	-1.20911528234478\\
-1.53353353353353	-1.2089503554965\\
-1.52552552552553	-1.20871421365986\\
-1.51751751751752	-1.20840881613888\\
-1.50950950950951	-1.20803612870088\\
-1.5015015015015	-1.20759811895779\\
-1.49349349349349	-1.20709675147945\\
-1.48548548548549	-1.20653398263504\\
-1.47747747747748	-1.20591175516167\\
-1.46946946946947	-1.20523199246283\\
-1.46146146146146	-1.20449659264322\\
-1.45345345345345	-1.20370742229093\\
-1.44544544544545	-1.20286631002275\\
-1.43743743743744	-1.20197503981362\\
-1.42942942942943	-1.20103534413654\\
-1.42142142142142	-1.20004889694538\\
-1.41341341341341	-1.19901730653845\\
-1.40540540540541	-1.19794210834718\\
-1.3973973973974	-1.19682475769973\\
-1.38938938938939	-1.19566662261501\\
-1.38138138138138	-1.19446897668805\\
-1.37337337337337	-1.19323299213213\\
-1.36536536536537	-1.19195973304676\\
-1.35735735735736	-1.19065014898402\\
-1.34934934934935	-1.18930506888721\\
-1.34134134134134	-1.18792519547645\\
-1.33333333333333	-1.18651110015535\\
-1.32532532532533	-1.18506321851008\\
-1.31731731731732	-1.18358184646897\\
-1.30930930930931	-1.18206713718506\\
-1.3013013013013	-1.18051909869729\\
-1.29329329329329	-1.1789375924179\\
-1.28528528528529	-1.17732233248391\\
-1.27727727727728	-1.17567288600018\\
-1.26926926926927	-1.17398867419008\\
-1.26126126126126	-1.17226897445768\\
-1.25325325325325	-1.17051292335332\\
-1.24524524524525	-1.1687195204221\\
-1.23723723723724	-1.16688763290293\\
-1.22922922922923	-1.16501600123458\\
-1.22122122122122	-1.16310324531486\\
-1.21321321321321	-1.16114787144993\\
-1.20520520520521	-1.15914827992279\\
-1.1971971971972	-1.15710277310415\\
-1.18918918918919	-1.15500956402357\\
-1.18118118118118	-1.15286678531654\\
-1.17317317317317	-1.15067249846116\\
-1.16516516516517	-1.14842470321864\\
-1.15715715715716	-1.14612134719374\\
-1.14914914914915	-1.14376033543379\\
-1.14114114114114	-1.14133953998984\\
-1.13313313313313	-1.13885680936824\\
-1.12512512512513	-1.13630997780713\\
-1.11711711711712	-1.13369687431907\\
-1.10910910910911	-1.13101533144764\\
-1.1011011011011	-1.12826319369357\\
-1.09309309309309	-1.12543832557253\\
-1.08508508508509	-1.12253861927428\\
-1.07707707707708	-1.11956200189949\\
-1.06906906906907	-1.11650644225692\\
-1.06106106106106	-1.11336995720968\\
-1.05305305305305	-1.11015061756493\\
-1.04504504504505	-1.10684655350603\\
-1.03703703703704	-1.10345595957057\\
-1.02902902902903	-1.0999770991817\\
-1.02102102102102	-1.09640830874306\\
-1.01301301301301	-1.09274800131045\\
-1.00500500500501	-1.08899466985546\\
-0.996996996996997	-1.08514689013791\\
-0.988988988988989	-1.08120332320526\\
-0.980980980980981	-1.07716271753789\\
-0.972972972972973	-1.07302391085947\\
-0.964964964964965	-1.06878583163232\\
-0.956956956956957	-1.06444750025685\\
-0.948948948948949	-1.0600080299945\\
-0.940940940940941	-1.05546662763274\\
-0.932932932932933	-1.0508225939101\\
-0.924924924924925	-1.04607532371871\\
-0.916916916916917	-1.04122430610053\\
-0.908908908908909	-1.03626912405302\\
-0.900900900900901	-1.03120945415883\\
-0.892892892892893	-1.02604506605314\\
-0.884884884884885	-1.02077582174139\\
-0.876876876876877	-1.01540167477928\\
-0.868868868868869	-1.00992266932567\\
-0.860860860860861	-1.00433893907867\\
-0.852852852852853	-0.998650706103697\\
-0.844844844844845	-0.99285827956207\\
-0.836836836836837	-0.986962054347499\\
-0.828828828828829	-0.980962509637323\\
-0.820820820820821	-0.974860207364634\\
-0.812812812812813	-0.968655790616735\\
-0.804804804804805	-0.962349981964902\\
-0.796796796796797	-0.955943581729744\\
-0.788788788788789	-0.94943746618617\\
-0.780780780780781	-0.942832585711287\\
-0.772772772772773	-0.936129962878412\\
-0.764764764764765	-0.929330690499789\\
-0.756756756756757	-0.922435929620506\\
-0.748748748748749	-0.915446907465662\\
-0.740740740740741	-0.908364915342676\\
-0.732732732732733	-0.901191306500481\\
-0.724724724724725	-0.893927493947051\\
-0.716716716716717	-0.88657494822672\\
-0.708708708708709	-0.87913519515851\\
-0.700700700700701	-0.871609813536771\\
-0.692692692692693	-0.864000432795258\\
-0.684684684684685	-0.856308730635835\\
-0.676676676676677	-0.848536430622994\\
-0.668668668668669	-0.840685299745394\\
-0.660660660660661	-0.832757145945747\\
-0.652652652652653	-0.824753815620377\\
-0.644644644644645	-0.816677191089972\\
-0.636636636636636	-0.808529188043174\\
-0.628628628628629	-0.800311752954713\\
-0.62062062062062	-0.792026860480111\\
-0.612612612612613	-0.783676510829001\\
-0.604604604604605	-0.775262727119504\\
-0.596596596596596	-0.766787552716172\\
-0.588588588588589	-0.758253048554266\\
-0.58058058058058	-0.749661290453494\\
-0.572572572572573	-0.74101436642445\\
-0.564564564564565	-0.73231437397133\\
-0.556556556556556	-0.723563417394768\\
-0.548548548548549	-0.714763605098861\\
-0.54054054054054	-0.705917046906774\\
-0.532532532532533	-0.697025851389553\\
-0.524524524524525	-0.68809212321298\\
-0.516516516516516	-0.67911796050767\\
-0.508508508508509	-0.670105452267704\\
-0.5005005005005	-0.661056675783384\\
-0.492492492492492	-0.651973694113856\\
-0.484484484484485	-0.642858553605556\\
-0.476476476476476	-0.633713281462451\\
-0.468468468468469	-0.624539883374343\\
-0.46046046046046	-0.615340341209282\\
-0.452452452452452	-0.606116610776499\\
-0.444444444444445	-0.596870619665986\\
-0.436436436436436	-0.587604265170854\\
-0.428428428428429	-0.578319412298613\\
-0.42042042042042	-0.569017891877136\\
-0.412412412412412	-0.559701498761091\\
-0.404404404404405	-0.550371990144198\\
-0.396396396396396	-0.541031083982429\\
-0.388388388388389	-0.531680457532917\\
-0.38038038038038	-0.522321746012918\\
-0.372372372372372	-0.512956541382737\\
-0.364364364364364	-0.503586391256005\\
-0.356356356356356	-0.494212797940241\\
-0.348348348348348	-0.484837217609974\\
-0.34034034034034	-0.475461059614252\\
-0.332332332332332	-0.466085685919564\\
-0.324324324324324	-0.45671241068883\\
-0.316316316316316	-0.4473424999962\\
-0.308308308308308	-0.43797717167704\\
-0.3003003003003	-0.428617595311642\\
-0.292292292292292	-0.419264892340686\\
-0.284284284284284	-0.409920136309875\\
-0.276276276276276	-0.400584353240507\\
-0.268268268268268	-0.391258522122332\\
-0.26026026026026	-0.381943575524411\\
-0.252252252252252	-0.372640400319258\\
-0.244244244244244	-0.363349838515214\\
-0.236236236236236	-0.354072688191443\\
-0.228228228228228	-0.344809704529804\\
-0.22022022022022	-0.335561600937482\\
-0.212212212212212	-0.326329050254082\\
-0.204204204204204	-0.317112686036717\\
-0.196196196196196	-0.307913103916593\\
-0.188188188188188	-0.298730863020461\\
-0.18018018018018	-0.289566487450413\\
-0.172172172172172	-0.280420467815523\\
-0.164164164164164	-0.271293262808969\\
-0.156156156156156	-0.262185300824465\\
-0.148148148148148	-0.253096981606048\\
-0.14014014014014	-0.244028677925441\\
-0.132132132132132	-0.23498073728167\\
-0.124124124124124	-0.225953483617798\\
-0.116116116116116	-0.216947219050008\\
-0.108108108108108	-0.207962225604767\\
-0.1001001001001	-0.198998766960004\\
-0.0920920920920922	-0.190057090186839\\
-0.084084084084084	-0.181137427488615\\
-0.0760760760760761	-0.172239997934651\\
-0.0680680680680679	-0.163365009186297\\
-0.06006006006006	-0.154512659213464\\
-0.0520520520520522	-0.145683138000199\\
-0.0440440440440439	-0.136876629238145\\
-0.0360360360360361	-0.128093312007366\\
-0.0280280280280278	-0.119333362444044\\
-0.02002002002002	-0.110596955395309\\
-0.0120120120120122	-0.101884266061421\\
-0.00400400400400391	-0.0931954716261246\\
0.00400400400400436	-0.0845307528760619\\
0.0120120120120122	-0.0758902958105759\\
0.02002002002002	-0.0672742932433228\\
0.0280280280280278	-0.0586829463972988\\
0.0360360360360357	-0.0501164664952495\\
0.0440440440440444	-0.0415750763471857\\
0.0520520520520522	-0.0330590119372544\\
0.06006006006006	-0.0245685240117994\\
0.0680680680680679	-0.0161038796707958\\
0.0760760760760757	-0.00766536396456008\\
0.0840840840840844	0.000746718502356691\\
0.0920920920920922	0.00913204195853695\\
0.1001001001001	0.0174902578427316\\
0.108108108108108	0.0258209931696276\\
0.116116116116116	0.0341238488836283\\
0.124124124124124	0.0423983982003887\\
0.132132132132132	0.0506441849366329\\
0.14014014014014	0.0588607218293157\\
0.148148148148148	0.0670474888461572\\
0.156156156156156	0.0752039314904616\\
0.164164164164164	0.0833294591043014\\
0.172172172172172	0.0914234431754423\\
0.18018018018018	0.0994852156548243\\
0.188188188188188	0.107514067293039\\
0.196196196196196	0.115509246006211\\
0.204204204204204	0.123469955283608\\
0.212212212212212	0.131395352651718\\
0.22022022022022	0.139284548212025\\
0.228228228228228	0.147136603272493\\
0.236236236236236	0.154950529095686\\
0.244244244244245	0.162725285789802\\
0.252252252252252	0.170459781372126\\
0.26026026026026	0.178152871038201\\
0.268268268268268	0.185803356673384\\
0.276276276276277	0.193409986647614\\
0.284284284284285	0.200971455937364\\
0.292292292292292	0.208486406622871\\
0.3003003003003	0.215953428811425\\
0.308308308308308	0.223371062040845\\
0.316316316316317	0.230737797218867\\
0.324324324324325	0.238052079156013\\
0.332332332332332	0.245312309749304\\
0.34034034034034	0.252516851873175\\
0.348348348348348	0.259664034031105\\
0.356356356356357	0.266752155816729\\
0.364364364364365	0.273779494226052\\
0.372372372372372	0.280744310853205\\
0.38038038038038	0.287644859989852\\
0.388388388388388	0.294479397633615\\
0.396396396396397	0.30124619139317\\
0.404404404404405	0.307943531257424\\
0.412412412412412	0.314569741173289\\
0.42042042042042	0.321123191352036\\
0.428428428428428	0.327602311198152\\
0.436436436436437	0.33400560272802\\
0.444444444444445	0.340331654319806\\
0.452452452452452	0.346579154611663\\
0.46046046046046	0.352746906343729\\
0.468468468468468	0.358833839922305\\
0.476476476476477	0.364839026472625\\
0.484484484484485	0.370761690141411\\
0.492492492492492	0.376601219412353\\
0.5005005005005	0.382357177207656\\
0.508508508508508	0.388029309566827\\
0.516516516516517	0.393617552719614\\
0.524524524524525	0.399122038402858\\
0.532532532532533	0.404543097309825\\
0.54054054054054	0.409881260603475\\
0.548548548548548	0.415137259470812\\
0.556556556556557	0.42031202274159\\
0.564564564564565	0.425406672639385\\
0.572572572572573	0.430422518774437\\
0.58058058058058	0.435361050524255\\
0.588588588588588	0.440223927978097\\
0.596596596596597	0.445012971644724\\
0.604604604604605	0.449730151138062\\
0.612612612612613	0.454377573063412\\
0.62062062062062	0.458957468327119\\
0.628628628628628	0.463472179086568\\
0.636636636636637	0.467924145545318\\
0.644644644644645	0.47231589278172\\
0.652652652652653	0.47665001777937\\
0.66066066066066	0.48092917680532\\
0.668668668668668	0.485156073258741\\
0.676676676676677	0.489333446088841\\
0.684684684684685	0.493464058858006\\
0.692692692692693	0.497550689504413\\
0.7007007007007	0.501596120838503\\
0.708708708708708	0.505603131790048\\
0.716716716716717	0.509574489407156\\
0.724724724724725	0.513512941595706\\
0.732732732732733	0.517421210577091\\
0.74074074074074	0.521301987033797\\
0.748748748748748	0.525157924906207\\
0.756756756756757	0.528991636799508\\
0.764764764764765	0.532805689956767\\
0.772772772772773	0.536602602753057\\
0.780780780780781	0.540384841664935\\
0.788788788788789	0.544154818670482\\
0.796796796796797	0.547914889036448\\
0.804804804804805	0.551667349450865\\
0.812812812812813	0.555414436462078\\
0.820820820820821	0.559158325187404\\
0.828828828828829	0.562901128257686\\
0.836836836836837	0.566644894966648\\
0.844844844844845	0.570391610596737\\
0.852852852852853	0.574143195895951\\
0.860860860860861	0.577901506682727\\
0.868868868868869	0.581668333558478\\
0.876876876876877	0.585445401709638\\
0.884884884884885	0.589234370783306\\
0.892892892892893	0.593036834822553\\
0.900900900900901	0.596854322249237\\
0.908908908908909	0.600688295883912\\
0.916916916916917	0.604540152993819\\
0.924924924924925	0.608411225361301\\
0.932932932932933	0.612302779366264\\
0.940940940940941	0.616216016077221\\
0.948948948948949	0.620152071346523\\
0.956956956956957	0.62411201590614\\
0.964964964964965	0.628096855461055\\
0.972972972972973	0.63210753077802\\
0.980980980980981	0.636144917767992\\
0.988988988988989	0.640209827560854\\
0.996996996996997	0.644303006571712\\
1.00500500500501	0.648425136558161\\
1.01301301301301	0.652576834668386\\
1.02102102102102	0.656758653480007\\
1.02902902902903	0.660971081030082\\
1.03703703703704	0.665214540836541\\
1.04504504504505	0.669489391911676\\
1.05305305305305	0.673795928768416\\
1.06106106106106	0.678134381420052\\
1.06906906906907	0.682504915374376\\
1.07707707707708	0.686907631623099\\
1.08508508508509	0.69134256662757\\
1.09309309309309	0.695809692301686\\
1.1011011011011	0.700308915993159\\
1.10910910910911	0.704840080464133\\
1.11711711711712	0.709402963872193\\
1.12512512512513	0.713997279752879\\
1.13313313313313	0.718622677004801\\
1.14114114114114	0.723278739878436\\
1.14914914914915	0.727964987969644\\
1.15715715715716	0.732680876219119\\
1.16516516516517	0.737425794918814\\
1.17317317317317	0.742199069726454\\
1.18118118118118	0.746999961689277\\
1.18918918918919	0.751827667278172\\
1.1971971971972	0.75668131843337\\
1.20520520520521	0.761559982622851\\
1.21321321321321	0.766462662914701\\
1.22122122122122	0.771388298064686\\
1.22922922922923	0.776335762620349\\
1.23723723723724	0.781303867042956\\
1.24524524524525	0.786291357848714\\
1.25325325325325	0.791296917770767\\
1.26126126126126	0.796319165943467\\
1.26926926926927	0.801356658110641\\
1.27727727727728	0.806407886859554\\
1.28528528528529	0.811471281882439\\
1.29329329329329	0.816545210267589\\
1.3013013013013	0.821627976822094\\
1.30930930930931	0.826717824428526\\
1.31731731731732	0.831812934438046\\
1.32532532532533	0.836911427102372\\
1.33333333333333	0.842011362047621\\
1.34134134134134	0.847110738792945\\
1.34934934934935	0.852207497317063\\
1.35735735735736	0.85729951867641\\
1.36536536536537	0.862384625678384\\
1.37337337337337	0.86746058361375\\
1.38138138138138	0.872525101052514\\
1.38938938938939	0.877575830707662\\
1.3973973973974	0.882610370371732\\
1.40540540540541	0.887626263931227\\
1.41341341341341	0.89262100246443\\
1.42142142142142	0.897592025428252\\
1.42942942942943	0.902536721940202\\
1.43743743743744	0.907452432161911\\
1.44544544544545	0.912336448790658\\
1.45345345345345	0.917186018665961\\
1.46146146146146	0.921998344498278\\
1.46946946946947	0.926770586727254\\
1.47747747747748	0.931499865516749\\
1.48548548548549	0.936183262894692\\
1.49349349349349	0.940817825044948\\
1.5015015015015	0.94540056475915\\
1.50950950950951	0.949928464055821\\
1.51751751751752	0.954398476974119\\
1.52552552552553	0.958807532549102\\
1.53353353353353	0.963152537975102\\
1.54154154154154	0.967430381962835\\
1.54954954954955	0.971637938295551\\
1.55755755755756	0.975772069588382\\
1.56556556556557	0.979829631253663\\
1.57357357357357	0.983807475674302\\
1.58158158158158	0.98770245658515\\
1.58958958958959	0.99151143366103\\
1.5975975975976	0.995231277307826\\
1.60560560560561	0.998858873651425\\
1.61361361361361	1.00239112971617\\
1.62162162162162	1.00582497878297\\
1.62962962962963	1.00915738591378\\
1.63763763763764	1.01238535362659\\
1.64564564564565	1.01550592770262\\
1.65365365365365	1.0185162031036\\
1.66166166166166	1.02141332997498\\
1.66966966966967	1.02419451970722\\
1.67767767767768	1.02685705102514\\
1.68568568568569	1.02939827607208\\
1.69369369369369	1.03181562645352\\
1.7017017017017	1.03410661920292\\
1.70970970970971	1.03626886262995\\
1.71771771771772	1.03830006201104\\
1.72572572572573	1.04019802508051\\
1.73373373373373	1.04196066728057\\
1.74174174174174	1.04358601672911\\
1.74974974974975	1.04507221886442\\
1.75775775775776	1.04641754072837\\
1.76576576576577	1.04762037485126\\
1.77377377377377	1.04867924270469\\
1.78178178178178	1.04959279769216\\
1.78978978978979	1.05035982765134\\
1.7977977977978	1.05097925684617\\
1.80580580580581	1.05145014743213\\
1.81381381381381	1.05177170038324\\
1.82182182182182	1.05194325587474\\
1.82982982982983	1.05196429312097\\
1.83783783783784	1.05183442967375\\
1.84584584584585	1.05155342019176\\
1.85385385385385	1.05112115469671\\
1.86186186186186	1.05053765633717\\
1.86986986986987	1.0498030786853\\
1.87787787787788	1.04891770259548\\
1.88588588588589	1.04788193265782\\
1.89389389389389	1.04669629328188\\
1.9019019019019	1.04536142444811\\
1.90990990990991	1.04387807716665\\
1.91791791791792	1.04224710868341\\
1.92592592592593	1.04046947747366\\
1.93393393393393	1.0385462380638\\
1.94194194194194	1.03647853572022\\
1.94994994994995	1.03426760104326\\
1.95795795795796	1.03191474450284\\
1.96596596596597	1.02942135094951\\
1.97397397397397	1.02678887413297\\
1.98198198198198	1.02401883125691\\
1.98998998998999	1.02111279759697\\
1.997997997998	1.01807240120477\\
2.00600600600601	1.01489931771909\\
2.01401401401401	1.01159526530165\\
2.02202202202202	1.00816199971255\\
2.03003003003003	1.00460130953748\\
2.03803803803804	1.00091501157644\\
2.04604604604605	0.997104946401064\\
2.05405405405405	0.99317297408569\\
2.06206206206206	0.989120970115119\\
2.07007007007007	0.984950821470043\\
2.07807807807808	0.980664422889865\\
2.08608608608609	0.976263673310893\\
2.09409409409409	0.971750472476757\\
2.1021021021021	0.967126717716882\\
2.11011011011011	0.962394300888126\\
2.11811811811812	0.957555105473608\\
2.12612612612613	0.952611003832759\\
2.13413413413413	0.947563854595851\\
2.14214214214214	0.942415500196181\\
2.15015015015015	0.937167764532837\\
2.15815815815816	0.931822450757225\\
2.16616616616617	0.92638133917608\\
2.17417417417417	0.920846185264444\\
2.18218218218218	0.915218717781847\\
2.19019019019019	0.909500636985401\\
2.1981981981982	0.903693612933709\\
2.20620620620621	0.897799283876159\\
2.21421421421421	0.891819254722131\\
2.22222222222222	0.88575509558541\\
2.23023023023023	0.879608340399571\\
2.23823823823824	0.873380485600109\\
2.24624624624625	0.867072988870516\\
2.25425425425425	0.860687267948822\\
2.26226226226226	0.85422469949272\\
2.27027027027027	0.847686618001265\\
2.27827827827828	0.841074314792046\\
2.28628628628629	0.834389037032913\\
2.29429429429429	0.827631986828488\\
2.3023023023023	0.820804320361575\\
2.31031031031031	0.813907147090778\\
2.31831831831832	0.806941529005953\\
2.32632632632633	0.799908479943667\\
2.33433433433433	0.792808964965761\\
2.34234234234234	0.785643899804352\\
2.35035035035035	0.778414150377807\\
2.35835835835836	0.771120532382276\\
2.36636636636637	0.763763810964632\\
2.37437437437437	0.756344700482783\\
2.38238238238238	0.748863864360749\\
2.39039039039039	0.74132191504565\\
2.3983983983984	0.733719414075646\\
2.40640640640641	0.726056872267338\\
2.41441441441441	0.718334750032783\\
2.42242242242242	0.710553457836482\\
2.43043043043043	0.702713356803431\\
2.43843843843844	0.694814759490068\\
2.44644644644645	0.686857930830349\\
2.45445445445445	0.678843089269562\\
2.46246246246246	0.670770408099389\\
2.47047047047047	0.66264001700734\\
2.47847847847848	0.654452003854348\\
2.48648648648649	0.646206416694224\\
2.49449449449449	0.637903266048196\\
2.5025025025025	0.629542527447866\\
2.51051051051051	0.621124144259131\\
2.51851851851852	0.612648030798458\\
2.52652652652653	0.604114075752214\\
2.53453453453453	0.595522145908222\\
2.54254254254254	0.586872090206438\\
2.55055055055055	0.578163744114297\\
2.55855855855856	0.569396934328969\\
2.56656656656657	0.560571483806388\\
2.57457457457457	0.551687217113343\\
2.58258258258258	0.542743966095265\\
2.59059059059059	0.533741575848654\\
2.5985985985986	0.524679910981776\\
2.60660660660661	0.5155588621439\\
2.61461461461461	0.506378352797068\\
2.62262262262262	0.49713834620037\\
2.63063063063063	0.487838852570824\\
2.63863863863864	0.478479936380212\\
2.64664664664665	0.469061723741719\\
2.65465465465465	0.459584409836066\\
2.66266266266266	0.450048266321709\\
2.67067067067067	0.440453648670518\\
2.67867867867868	0.430801003366211\\
2.68668668668669	0.421090874901135\\
2.69469469469469	0.411323912505216\\
2.7027027027027	0.401500876539793\\
2.71071071071071	0.391622644490422\\
2.71871871871872	0.381690216493697\\
2.72672672672673	0.371704720336485\\
2.73473473473473	0.361667415869737\\
2.74274274274274	0.351579698784148\\
2.75075075075075	0.341443103701554\\
2.75875875875876	0.331259306542683\\
2.76676676676677	0.321030126140305\\
2.77477477477477	0.310757525075056\\
2.78278278278278	0.30044360972094\\
2.79079079079079	0.290090629496011\\
2.7987987987988	0.279700975324137\\
2.80680680680681	0.269277177321916\\
2.81481481481481	0.258821901734581\\
2.82282282282282	0.248337947152294\\
2.83083083083083	0.237828240045832\\
2.83883883883884	0.22729582966744\\
2.84684684684685	0.216743882367867\\
2.85485485485485	0.20617567538521\\
2.86286286286286	0.195594590164249\\
2.87087087087087	0.185004105267439\\
2.87887887887888	0.174407788939141\\
2.88688688688689	0.163809291385093\\
2.89489489489489	0.153212336827743\\
2.9029029029029	0.142620715396022\\
2.91091091091091	0.13203827490518\\
2.91891891891892	0.12146891257931\\
2.92692692692693	0.110916566764481\\
2.93493493493493	0.100385208676143\\
2.94294294294294	0.0898788342198663\\
2.95095095095095	0.0794014559196364\\
2.95895895895896	0.0689570949819875\\
2.96696696696697	0.0585497735206459\\
2.97497497497497	0.048183506960126\\
2.98298298298298	0.0378622966329166\\
2.99099099099099	0.0275901225802367\\
2.998998998999	0.0173709365614506\\
3.00700700700701	0.00720865527430642\\
3.01501501501502	-0.00289284621618111\\
3.02302302302302	-0.0129297408454433\\
3.03103103103103	-0.0228982557214227\\
3.03903903903904	-0.0327946781124144\\
3.04704704704705	-0.0426153612633704\\
3.05505505505506	-0.0523567300563066\\
3.06306306306306	-0.0620152865315455\\
3.07107107107107	-0.0715876152885729\\
3.07907907907908	-0.0810703887869038\\
3.08708708708709	-0.0904603725678852\\
3.0950950950951	-0.0997544304202853\\
3.1031031031031	-0.108949529512521\\
3.11111111111111	-0.118042745516463\\
3.11911911911912	-0.127031267747656\\
3.12712712712713	-0.135912404347605\\
3.13513513513514	-0.144683587535309\\
3.14314314314314	-0.153342378954763\\
3.15115115115115	-0.16188647514696\\
3.15915915915916	-0.170313713174862\\
3.16716716716717	-0.17862207643064\\
3.17517517517518	-0.186809700655562\\
3.18318318318318	-0.194874880202835\\
3.19119119119119	-0.202816074574852\\
3.1991991991992	-0.210631915266485\\
3.20720720720721	-0.218321212946571\\
3.21521521521522	-0.225882965009856\\
3.22322322322322	-0.233316363531284\\
3.23123123123123	-0.240620803654949\\
3.23923923923924	-0.247795892448186\\
3.24724724724725	-0.254841458250452\\
3.25525525525526	-0.261757560544583\\
3.26326326326326	-0.268544500374394\\
3.27127127127127	-0.275202831329176\\
3.27927927927928	-0.281733371109603\\
3.28728728728729	-0.288137213682479\\
3.2952952952953	-0.294415742022555\\
3.3033033033033	-0.300570641428156\\
3.31131131131131	-0.306603913382473\\
3.31931931931932	-0.312517889915207\\
3.32732732732733	-0.318315248397046\\
3.33533533533534	-0.32399902667382\\
3.34334334334334	-0.329572638416334\\
3.35135135135135	-0.335039888525353\\
3.35935935935936	-0.340404988389099\\
3.36736736736737	-0.345672570742826\\
3.37537537537538	-0.350847703824314\\
3.38338338338338	-0.355935904459896\\
3.39139139139139	-0.360943149648315\\
3.3993993993994	-0.365875886139974\\
3.40740740740741	-0.370741037436305\\
3.41541541541542	-0.375546007561282\\
3.42342342342342	-0.380298680889723\\
3.43143143143143	-0.385007417256794\\
3.43943943943944	-0.389681041530561\\
3.44744744744745	-0.394328826808115\\
3.45545545545546	-0.3989604704044\\
3.46346346346346	-0.403586061854854\\
3.47147147147147	-0.408216042249168\\
3.47947947947948	-0.412861154369892\\
3.48748748748749	-0.417532383326557\\
3.4954954954955	-0.422240887658907\\
3.5035035035035	-0.426997921227486\\
3.51151151151151	-0.431814746609221\\
3.51951951951952	-0.43670254115263\\
3.52752752752753	-0.441672297299821\\
3.53553553553554	-0.446734719220447\\
3.54354354354354	-0.451900118190931\\
3.55155155155155	-0.457178309454891\\
3.55955955955956	-0.46257851348246\\
3.56756756756757	-0.46810926457843\\
3.57557557557558	-0.473778329655736\\
3.58358358358358	-0.479592639686862\\
3.59159159159159	-0.485558235885411\\
3.5995995995996	-0.491680232081734\\
3.60760760760761	-0.497962794081133\\
3.61561561561562	-0.50440913608229\\
3.62362362362362	-0.511021533539658\\
3.63163163163163	-0.517801351226288\\
3.63963963963964	-0.5247490847337\\
3.64764764764765	-0.531864413262046\\
3.65565565565566	-0.539146261318982\\
3.66366366366366	-0.546592866858964\\
3.67167167167167	-0.554201853444534\\
3.67967967967968	-0.561970304171472\\
3.68768768768769	-0.56989483534717\\
3.6956956956957	-0.577971668212702\\
3.7037037037037	-0.586196697328821\\
3.71171171171171	-0.59456555457811\\
3.71971971971972	-0.60307366805142\\
3.72772772772773	-0.611716315372681\\
3.73573573573574	-0.620488671262434\\
3.74374374374374	-0.62938584934487\\
3.75175175175175	-0.638402938363505\\
3.75975975975976	-0.647535033090787\\
3.76776776776777	-0.656777260300407\\
3.77577577577578	-0.666124800223069\\
3.78378378378378	-0.675572903932821\\
3.79179179179179	-0.685116907116678\\
3.7997997997998	-0.694752240670632\\
3.80780780780781	-0.704474438544181\\
3.81581581581582	-0.714279143227174\\
3.82382382382382	-0.724162109239917\\
3.83183183183183	-0.734119204951977\\
3.83983983983984	-0.744146413020748\\
3.84784784784785	-0.754239829704856\\
3.85585585585586	-0.764395663275813\\
3.86386386386386	-0.774610231720302\\
3.87187187187187	-0.784879959897792\\
3.87987987987988	-0.795201376293524\\
3.88788788788789	-0.805571109483525\\
3.8958958958959	-0.815985884410266\\
3.9039039039039	-0.826442518549122\\
3.91191191191191	-0.836937918031801\\
3.91991991991992	-0.847469073779796\\
3.92792792792793	-0.858033057690333\\
3.93593593593594	-0.86862701890801\\
3.94394394394394	-0.879248180207556\\
3.95195195195195	-0.889893834506821\\
3.95995995995996	-0.900561341523682\\
3.96796796796797	-0.911248124586104\\
3.97597597597598	-0.921951667601129\\
3.98398398398398	-0.93266951218539\\
3.99199199199199	-0.943399254957772\\
4	-0.954138544992529\\
4	0.204183246471276\\
3.99199199199199	0.193465373389055\\
3.98398398398398	0.182874017925989\\
3.97597597597598	0.172413102865484\\
3.96796796796797	0.162086596207371\\
3.95995995995996	0.151898513167\\
3.95195195195195	0.141852918329831\\
3.94394394394394	0.131953927963412\\
3.93593593593594	0.122205712486225\\
3.92792792792793	0.112612499090992\\
3.91991991991992	0.10317857451682\\
3.91191191191191	0.0939082879610996\\
3.9039039039039	0.0848060541176526\\
3.8958958958959	0.0758763563221881\\
3.88788788788789	0.0671237497798096\\
3.87987987987988	0.0585528648415807\\
3.87187187187187	0.0501684102878296\\
3.86386386386386	0.0419751765653624\\
3.85585585585586	0.0339780389125824\\
3.84784784784785	0.0261819602925959\\
3.83983983983984	0.0185919940358468\\
3.83183183183183	0.0112132860758645\\
3.82382382382382	0.00405107663828613\\
3.81581581581582	-0.00288929878133842\\
3.80780780780781	-0.00960240934310824\\
3.7997997997998	-0.0160827299148929\\
3.79179179179179	-0.0223246435058631\\
3.78378378378378	-0.028322445372244\\
3.77577577577578	-0.0340703491205381\\
3.76776776776777	-0.039562495168864\\
3.75975975975976	-0.0447929619600092\\
3.75175175175175	-0.0497557803480547\\
3.74374374374374	-0.054444951601445\\
3.73573573573574	-0.0588544694749474\\
3.72772772772773	-0.0629783467973559\\
3.71971971971972	-0.0668106469954529\\
3.71171171171171	-0.0703455209227109\\
3.7037037037037	-0.0735772492777721\\
3.6956956956957	-0.0765002907775693\\
3.68768768768769	-0.07910933608959\\
3.67967967967968	-0.081399367323363\\
3.67167167167167	-0.083365722634972\\
3.66366366366366	-0.085004165212445\\
3.65565565565566	-0.0863109555939701\\
3.64764764764765	-0.0872829259388606\\
3.63963963963964	-0.0879175545414739\\
3.63163163163163	-0.0882130385771492\\
3.62362362362362	-0.0881683628217277\\
3.61561561561562	-0.0877833619259432\\
3.60760760760761	-0.0870587737760824\\
3.5995995995996	-0.0859962815590498\\
3.59159159159159	-0.0845985423847756\\
3.58358358358358	-0.0828692007022692\\
3.57557557557558	-0.0808128852654914\\
3.56756756756757	-0.0784351890324683\\
3.55955955955956	-0.0757426320750047\\
3.55155155155155	-0.0727426082871982\\
3.54354354354354	-0.06944331735633\\
3.53553553553554	-0.0658536840480929\\
3.52752752752753	-0.0619832673183853\\
3.51951951951952	-0.0578421620679236\\
3.51151151151151	-0.0534408964892585\\
3.5035035035035	-0.0487903279235977\\
3.4954954954955	-0.0439015399630518\\
3.48748748748749	-0.0387857432312858\\
3.47947947947948	-0.0334541818874764\\
3.47147147147147	-0.0279180474603876\\
3.46346346346346	-0.0221884011669134\\
3.45545545545546	-0.0162761054324161\\
3.44744744744745	-0.0101917649307796\\
3.43943943943944	-0.0039456771175011\\
3.43143143143143	0.00245220805383645\\
3.42342342342342	0.00899231975826795\\
3.41541541541542	0.015665489952353\\
3.40740740740741	0.0224629698324155\\
3.3993993993994	0.029376439929334\\
3.39139139139139	0.0363980146796379\\
3.38338338338338	0.0435202422913092\\
3.37537537537538	0.0507361006802428\\
3.36736736736737	0.0580389901929351\\
3.35935935935936	0.0654227237636818\\
3.35135135135135	0.0728815150818494\\
3.34334334334334	0.080409965271899\\
3.33533533533534	0.0880030485191671\\
3.32732732732733	0.0956560970070374\\
3.31931931931932	0.103364785471926\\
3.31131131131131	0.111125115626763\\
3.3033033033033	0.118933400655884\\
3.2952952952953	0.126786249942081\\
3.28728728728729	0.134680554150077\\
3.27927927927928	0.142613470759788\\
3.27127127127127	0.150582410117164\\
3.26326326326326	0.158585022048086\\
3.25525525525526	0.166619183063757\\
3.24724724724725	0.174682984171012\\
3.23923923923924	0.18277471928953\\
3.23123123123123	0.190892874268731\\
3.22322322322322	0.199036116490043\\
3.21521521521522	0.207203285034154\\
3.20720720720721	0.215393381389479\\
3.1991991991992	0.223605560674433\\
3.19119119119119	0.231839123343989\\
3.18318318318318	0.24009350735026\\
3.17517517517518	0.248368280724907\\
3.16716716716717	0.256663134551607\\
3.15915915915916	0.264977876296501\\
3.15115115115115	0.273312423464548\\
3.14314314314314	0.281666797550321\\
3.13513513513514	0.290041118251851\\
3.12712712712713	0.298435597917339\\
3.11911911911912	0.306850536194432\\
3.11111111111111	0.315286314852886\\
3.1031031031031	0.323743392752156\\
3.0950950950951	0.332222300925514\\
3.08708708708709	0.340723637753963\\
3.07907907907908	0.349248064202875\\
3.07107107107107	0.357796299095667\\
3.06306306306306	0.366369114399766\\
3.05505505505506	0.374967330499908\\
3.04704704704705	0.383591811435903\\
3.03903903903904	0.392243460082064\\
3.03103103103103	0.400923213247298\\
3.02302302302302	0.409632036675476\\
3.01501501501502	0.418370919927248\\
3.00700700700701	0.427140871126536\\
2.998998998999	0.435942911555961\\
2.99099099099099	0.444778070088525\\
2.98298298298298	0.45364737744396\\
2.97497497497497	0.462551860261943\\
2.96696696696697	0.471492534986814\\
2.95895895895896	0.480470401561527\\
2.95095095095095	0.489486436932815\\
2.94294294294294	0.498541588372581\\
2.93493493493493	0.507636766625315\\
2.92692692692693	0.516772838896007\\
2.91891891891892	0.525950621696835\\
2.91091091091091	0.53517087357711\\
2.9029029029029	0.544434287764579\\
2.89489489489489	0.553741484752063\\
2.88688688688689	0.563093004868326\\
2.87887887887888	0.572489300876524\\
2.87087087087087	0.581930730648024\\
2.86286286286286	0.591417549963835\\
2.85485485485485	0.600949905499119\\
2.84684684684685	0.610527828049019\\
2.83883883883884	0.620151226056141\\
2.83083083083083	0.629819879501392\\
2.82282282282282	0.639533434219471\\
2.81481481481481	0.649291396699827\\
2.80680680680681	0.6590931294314\\
2.7987987987988	0.668937846846513\\
2.79079079079079	0.67882461191445\\
2.78278278278278	0.688752333430253\\
2.77477477477477	0.698719764037264\\
2.76676676676677	0.708725499014457\\
2.75875875875876	0.718767975851983\\
2.75075075075075	0.728845474628621\\
2.74274274274274	0.738956119196544\\
2.73473473473473	0.74909787916854\\
2.72672672672673	0.759268572694195\\
2.71871871871872	0.769465870001893\\
2.71071071071071	0.779687297675168\\
2.7027027027027	0.789930243623608\\
2.69469469469469	0.800191962701692\\
2.68668668668669	0.810469582922294\\
2.67867867867868	0.820760112206597\\
2.67067067067067	0.831060445608202\\
2.66266266266266	0.841367372946088\\
2.65465465465465	0.851677586779844\\
2.64664664664665	0.861987690659433\\
2.63863863863864	0.872294207582733\\
2.63063063063063	0.882593588595921\\
2.62262262262262	0.892882221473379\\
2.61461461461461	0.903156439417982\\
2.60660660660661	0.91341252972578\\
2.5985985985986	0.923646742364147\\
2.59059059059059	0.933855298416679\\
2.58258258258258	0.944034398353616\\
2.57457457457457	0.954180230091242\\
2.56656656656657	0.964288976809622\\
2.55855855855856	0.974356824502153\\
2.55055055055055	0.984379969236436\\
2.54254254254254	0.994354624109601\\
2.53453453453453	1.00427702588633\\
2.52652652652653	1.01414344131163\\
2.51851851851852	1.02395017309391\\
2.51051051051051	1.03369356555779\\
2.5025025025025	1.04337000996797\\
2.49449449449449	1.05297594952933\\
2.48648648648649	1.06250788406929\\
2.47847847847848	1.07196237441111\\
2.47047047047047	1.08133604644809\\
2.46246246246246	1.09062559492946\\
2.45445445445445	1.09982778696984\\
2.44644644644645	1.10893946529488\\
2.43843843843844	1.11795755123583\\
2.43043043043043	1.12687904748587\\
2.42242242242242	1.1357010406315\\
2.41441441441441	1.1444207034714\\
2.40640640640641	1.15303529713562\\
2.3983983983984	1.16154217301709\\
2.39039039039039	1.16993877452696\\
2.38238238238238	1.17822263868513\\
2.37437437437437	1.18639139755618\\
2.36636636636637	1.1944427795407\\
2.35835835835836	1.20237461053128\\
2.35035035035035	1.21018481494123\\
2.34234234234234	1.21787141661436\\
2.33433433433433	1.22543253962238\\
2.32632632632633	1.23286640895669\\
2.31831831831832	1.24017135111997\\
2.31031031031031	1.24734579462266\\
2.3023023023023	1.25438827038855\\
2.29429429429429	1.26129741207334\\
2.28628628628629	1.26807195629876\\
2.27827827827828	1.27471074280513\\
2.27027027027027	1.28121271452357\\
2.26226226226226	1.28757691756934\\
2.25425425425425	1.29380250115667\\
2.24624624624625	1.29988871743509\\
2.23823823823824	1.30583492124653\\
2.23023023023023	1.31164056980214\\
2.22222222222222	1.31730522227701\\
2.21421421421421	1.32282853932037\\
2.20620620620621	1.32821028247881\\
2.1981981981982	1.3334503135288\\
2.19019019019019	1.33854859371505\\
2.18218218218218	1.34350518289004\\
2.17417417417417	1.34832023855024\\
2.16616616616617	1.35299401476378\\
2.15815815815816	1.3575268609836\\
2.15015015015015	1.36191922074058\\
2.14214214214214	1.36617163020973\\
2.13413413413413	1.37028471664328\\
2.12612612612613	1.37425919666322\\
2.11811811811812	1.37809587440649\\
2.11011011011011	1.38179563951553\\
2.1021021021021	1.38535946496659\\
2.09409409409409	1.38878840472892\\
2.08608608608609	1.39208359124747\\
2.07807807807808	1.39524623274219\\
2.07007007007007	1.39827761031777\\
2.06206206206206	1.40117907487766\\
2.05405405405405	1.40395204383713\\
2.04604604604605	1.40659799763129\\
2.03803803803804	1.40911847601451\\
2.03003003003003	1.41151507414928\\
2.02202202202202	1.413789438484\\
2.01401401401401	1.41594326242068\\
2.00600600600601	1.41797828177522\\
1.997997997998	1.41989627003572\\
1.98998998998999	1.42169903342544\\
1.98198198198198	1.42338840578044\\
1.97397397397397	1.42496624325391\\
1.96596596596597	1.42643441886225\\
1.95795795795796	1.42779481689041\\
1.94994994994995	1.42904932717762\\
1.94194194194194	1.43019983930661\\
1.93393393393393	1.43124823672303\\
1.92592592592593	1.43219639081437\\
1.91791791791792	1.43304615498011\\
1.90990990990991	1.43379935872751\\
1.9019019019019	1.43445780182945\\
1.89389389389389	1.43502324858254\\
1.88588588588589	1.43549742220491\\
1.87787787787788	1.43588199941428\\
1.86986986986987	1.43617860522688\\
1.86186186186186	1.43638880801764\\
1.85385385385385	1.43651411488123\\
1.84584584584585	1.43655596733209\\
1.83783783783784	1.43651573737893\\
1.82982982982983	1.43639472400714\\
1.82182182182182	1.43619415009847\\
1.81381381381381	1.43591515981359\\
1.80580580580581	1.43555881645883\\
1.7977977977978	1.43512610085339\\
1.78978978978979	1.43461791020781\\
1.78178178178178	1.43403505751972\\
1.77377377377377	1.43337827148672\\
1.76576576576577	1.43264819693079\\
1.75775775775776	1.43184539572354\\
1.74974974974975	1.43097034819602\\
1.74174174174174	1.43002345501183\\
1.73373373373373	1.42900503947812\\
1.72572572572573	1.42791535026476\\
1.71771771771772	1.42675456449862\\
1.70970970970971	1.42552279119683\\
1.7017017017017	1.42422007500119\\
1.69369369369369	1.42284640017355\\
1.68568568568569	1.42140169481175\\
1.67767767767768	1.41988583524506\\
1.66966966966967	1.41829865056823\\
1.66166166166166	1.41663992727453\\
1.65365365365365	1.41490941394909\\
1.64564564564565	1.41310682598566\\
1.63763763763764	1.41123185029236\\
1.62962962962963	1.40928414995409\\
1.62162162162162	1.40726336882204\\
1.61361361361361	1.40516913600366\\
1.60560560560561	1.40300107022926\\
1.5975975975976	1.40075878407443\\
1.58958958958959	1.3984418880205\\
1.58158158158158	1.39604999433802\\
1.57357357357357	1.393582720781\\
1.56556556556557	1.3910396940828\\
1.55755755755756	1.38842055324615\\
1.54954954954955	1.3857249526231\\
1.54154154154154	1.382952564782\\
1.53353353353353	1.38010308316126\\
1.52552552552553	1.3771762245106\\
1.51751751751752	1.37417173112269\\
1.50950950950951	1.37108937285916\\
1.5015015015015	1.36792894897566\\
1.49349349349349	1.3646902897524\\
1.48548548548549	1.36137325793678\\
1.47747747747748	1.35797775000535\\
1.46946946946947	1.35450369725318\\
1.46146146146146	1.3509510667187\\
1.45345345345345	1.34731986195238\\
1.44544544544545	1.34361012363806\\
1.43743743743744	1.33982193007496\\
1.42942942942943	1.33595539752952\\
1.42142142142142	1.33201068046502\\
1.41341341341341	1.32798797165743\\
1.40540540540541	1.32388750220542\\
1.3973973973974	1.31970954144251\\
1.38938938938939	1.31545439675864\\
1.38138138138138	1.31112241333855\\
1.37337337337337	1.306713973824\\
1.36536536536537	1.30222949790613\\
1.35735735735736	1.29766944185462\\
1.34934934934935	1.29303429798946\\
1.34134134134134	1.28832459410105\\
1.33333333333333	1.28354089282396\\
1.32532532532533	1.27868379096961\\
1.31731731731732	1.27375391882246\\
1.30930930930931	1.26875193940434\\
1.3013013013013	1.26367854771122\\
1.29329329329329	1.2585344699264\\
1.28528528528529	1.25332046261389\\
1.27727727727728	1.24803731189555\\
1.26926926926927	1.24268583261548\\
1.26126126126126	1.23726686749461\\
1.25325325325325	1.23178128627866\\
1.24524524524525	1.22622998488229\\
1.23723723723724	1.22061388453197\\
1.22922922922923	1.21493393091037\\
1.22122122122122	1.20919109330435\\
1.21321321321321	1.20338636375916\\
1.20520520520521	1.19752075624087\\
1.1971971971972	1.19159530580909\\
1.18918918918919	1.18561106780212\\
1.18118118118118	1.17956911703637\\
1.17317317317317	1.17347054702187\\
1.16516516516517	1.16731646919576\\
1.15715715715716	1.16110801217541\\
1.14914914914915	1.1548463210329\\
1.14114114114114	1.14853255659247\\
1.13313313313313	1.14216789475257\\
1.12512512512513	1.13575352583398\\
1.11711711711712	1.12929065395574\\
1.10910910910911	1.12278049644001\\
1.1011011011011	1.11622428324777\\
1.09309309309309	1.10962325644644\\
1.08508508508509	1.10297866971102\\
1.07707707707708	1.09629178786004\\
1.06906906906907	1.08956388642774\\
1.06106106106106	1.0827962512737\\
1.05305305305305	1.07599017823128\\
1.04504504504505	1.06914697279602\\
1.03703703703704	1.06226794985526\\
1.02902902902903	1.05535443346006\\
1.02102102102102	1.04840775664036\\
1.01301301301301	1.04142926126449\\
1.00500500500501	1.03442029794384\\
0.996996996996997	1.02738222598334\\
0.988988988988989	1.02031641337841\\
0.980980980980981	1.01322423685874\\
0.972972972972973	1.00610708197924\\
0.964964964964965	0.998966343257903\\
0.956956956956957	0.991803424360602\\
0.948948948948949	0.98461973833183\\
0.940940940940941	0.977416707870804\\
0.932932932932933	0.970195765651184\\
0.924924924924925	0.96295835468262\\
0.916916916916917	0.955705928711702\\
0.908908908908909	0.948439952659102\\
0.900900900900901	0.941161903088971\\
0.892892892892893	0.933873268706057\\
0.884884884884885	0.926575550874584\\
0.876876876876877	0.91927026415235\\
0.868868868868869	0.911958936831925\\
0.860860860860861	0.904643111479629\\
0.852852852852853	0.897324345461409\\
0.844844844844845	0.890004211443105\\
0.836836836836837	0.882684297850677\\
0.828828828828829	0.875366209273985\\
0.820820820820821	0.868051566795578\\
0.812812812812813	0.860742008223482\\
0.804804804804805	0.853439188204598\\
0.796796796796797	0.846144778192512\\
0.788788788788789	0.838860466241049\\
0.780780780780781	0.831587956591656\\
0.772772772772773	0.824328969020402\\
0.764764764764765	0.817085237907073\\
0.756756756756757	0.809858510986774\\
0.748748748748748	0.802650547741554\\
0.74074074074074	0.795463117388085\\
0.732732732732733	0.788297996415729\\
0.724724724724725	0.781156965628697\\
0.716716716716717	0.774041806646325\\
0.708708708708708	0.766954297816845\\
0.7007007007007	0.759896209502724\\
0.692692692692693	0.752869298700123\\
0.684684684684685	0.745875302961296\\
0.676676676676677	0.738915933596861\\
0.668668668668668	0.731992868145689\\
0.66066066066066	0.725107742112866\\
0.652652652652653	0.718262139991614\\
0.644644644644645	0.711457585602649\\
0.636636636636637	0.704695531804405\\
0.628628628628628	0.697977349649135\\
0.62062062062062	0.691304317082823\\
0.612612612612613	0.684677607310665\\
0.604604604604605	0.678098276973127\\
0.596596596596597	0.671567254300016\\
0.588588588588588	0.665085327429942\\
0.58058058058058	0.658653133099057\\
0.572572572572573	0.652271145915008\\
0.564564564564565	0.645939668437986\\
0.556556556556557	0.639658822290637\\
0.548548548548548	0.633428540510496\\
0.54054054054054	0.627248561343276\\
0.532532532532533	0.621118423652307\\
0.524524524524525	0.615037464089012\\
0.516516516516517	0.609004816132959\\
0.508508508508508	0.60301941106843\\
0.5005005005005	0.597079980919912\\
0.492492492492492	0.591185063322582\\
0.484484484484485	0.585333008258347\\
0.476476476476477	0.579521986545041\\
0.468468468468468	0.573749999927498\\
0.46046046046046	0.56801489258656\\
0.452452452452452	0.562314363856156\\
0.444444444444445	0.556645981920678\\
0.436436436436437	0.551007198254904\\
0.428428428428428	0.545395362566577\\
0.42042042042042	0.5398077380073\\
0.412412412412412	0.534241516429058\\
0.404404404404405	0.528693833481072\\
0.396396396396397	0.523161783363082\\
0.388388388388388	0.51764243307556\\
0.38038038038038	0.512132836033377\\
0.372372372372372	0.506630044935869\\
0.364364364364365	0.501131123812501\\
0.356356356356357	0.49563315918773\\
0.348348348348348	0.490133270331725\\
0.34034034034034	0.484628618583704\\
0.332332332332332	0.479116415752445\\
0.324324324324325	0.473593931613368\\
0.316316316316317	0.468058500533756\\
0.308308308308308	0.462507527267027\\
0.3003003003003	0.45693849196406\\
0.292292292292292	0.451348954454379\\
0.284284284284285	0.445736557852769\\
0.276276276276277	0.440099031548135\\
0.268268268268268	0.434434193631329\\
0.26026026026026	0.428739952817139\\
0.252252252252252	0.423014309913866\\
0.244244244244245	0.417255358890688\\
0.236236236236236	0.411461287590229\\
0.228228228228228	0.405630378129883\\
0.22022022022022	0.399761007031953\\
0.212212212212212	0.393851645118939\\
0.204204204204204	0.387900857206585\\
0.196196196196196	0.381907301623837\\
0.188188188188188	0.375869729585416\\
0.18018018018018	0.369786984439534\\
0.172172172172172	0.363658000810323\\
0.164164164164164	0.357481803651845\\
0.156156156156156	0.351257507227916\\
0.148148148148148	0.344984314029872\\
0.14014014014014	0.338661513642257\\
0.132132132132132	0.332288481564534\\
0.124124124124124	0.325864677995478\\
0.116116116116116	0.319389646585182\\
0.108108108108108	0.312863013158698\\
0.1001001001001	0.306284484413857\\
0.0920920920920922	0.299653846595232\\
0.0840840840840844	0.292970964145045\\
0.0760760760760757	0.286235778331456\\
0.0680680680680679	0.279448305853837\\
0.06006006006006	0.27260863742427\\
0.0520520520520522	0.265716936324116\\
0.0440440440440444	0.258773436934272\\
0.0360360360360357	0.251778443237259\\
0.0280280280280278	0.244732327289465\\
0.02002002002002	0.237635527661517\\
0.0120120120120122	0.230488547844772\\
0.00400400400400436	0.223291954622042\\
-0.00400400400400391	0.216046376400548\\
-0.0120120120120122	0.208752501505379\\
-0.02002002002002	0.201411076431701\\
-0.0280280280280278	0.194022904054318\\
-0.0360360360360361	0.186588841793219\\
-0.0440440440440439	0.179109799734221\\
-0.0520520520520522	0.171586738703825\\
-0.06006006006006	0.164020668297924\\
-0.0680680680680679	0.156412644864289\\
-0.0760760760760761	0.148763769438998\\
-0.084084084084084	0.141075185637512\\
-0.0920920920920922	0.133348077501399\\
-0.1001001001001	0.125583667302105\\
-0.108108108108108	0.117783213303714\\
-0.116116116116116	0.109948007486861\\
-0.124124124124124	0.102079373236678\\
-0.132132132132132	0.0941786629978177\\
-0.14014014014014	0.0862472559002377\\
-0.148148148148148	0.0782865553597683\\
-0.156156156156156	0.0702979866579497\\
-0.164164164164164	0.0622829945060197\\
-0.172172172172172	0.0542430405983225\\
-0.18018018018018	0.0461796011608125\\
-0.188188188188188	0.0380941645006298\\
-0.196196196196196	0.0299882285630527\\
-0.204204204204204	0.0218632985024008\\
-0.212212212212212	0.0137208842737504\\
-0.22022022022022	0.00556249825234451\\
-0.228228228228228	-0.00261034711203126\\
-0.236236236236236	-0.010796141598534\\
-0.244244244244244	-0.0189933794694357\\
-0.252252252252252	-0.0272005616675791\\
-0.26026026026026	-0.0354161979500858\\
-0.268268268268268	-0.0436388089513002\\
-0.276276276276276	-0.0518669281680796\\
-0.284284284284284	-0.0600991038609274\\
-0.292292292292292	-0.0683339008647192\\
-0.3003003003003	-0.0765699023030724\\
-0.308308308308308	-0.0848057112010136\\
-0.316316316316316	-0.0930399519908108\\
-0.324324324324324	-0.101271271906508\\
-0.332332332332332	-0.109498342263284\\
-0.34034034034034	-0.117719859618065\\
-0.348348348348348	-0.125934546808822\\
-0.356356356356356	-0.134141153870134\\
-0.364364364364364	-0.142338458823656\\
-0.372372372372372	-0.150525268342481\\
-0.38038038038038	-0.158700418289071\\
-0.388388388388389	-0.166862774127173\\
-0.396396396396396	-0.175011231208566\\
-0.404404404404405	-0.183144714936211\\
-0.412412412412412	-0.191262180805814\\
-0.42042042042042	-0.199362614328483\\
-0.428428428428429	-0.207445030837492\\
-0.436436436436436	-0.215508475182692\\
-0.444444444444445	-0.223552021316575\\
-0.452452452452452	-0.231574771776198\\
-0.46046046046046	-0.239575857065629\\
-0.468468468468469	-0.247554434943785\\
-0.476476476476476	-0.255509689622728\\
-0.484484484484485	-0.263440830881642\\
-0.492492492492492	-0.271347093101944\\
-0.5005005005005	-0.279227734228853\\
-0.508508508508509	-0.287082034664921\\
-0.516516516516516	-0.294909296101004\\
-0.524524524524525	-0.302708840289976\\
-0.532532532532533	-0.310480007768557\\
-0.54054054054054	-0.318222156532363\\
-0.548548548548549	-0.325934660669206\\
-0.556556556556556	-0.33361690895551\\
-0.564564564564565	-0.34126830342042\\
-0.572572572572573	-0.348888257882072\\
-0.58058058058058	-0.356476196460133\\
-0.588588588588589	-0.364031552068645\\
-0.596596596596596	-0.371553764892748\\
-0.604604604604605	-0.379042280852796\\
-0.612612612612613	-0.386496550059022\\
-0.62062062062062	-0.393916025259609\\
-0.628628628628629	-0.401300160284951\\
-0.636636636636636	-0.408648408490396\\
-0.644644644644645	-0.41596022119983\\
-0.652652652652653	-0.423235046151907\\
-0.660660660660661	-0.43047232595084\\
-0.668668668668669	-0.437671496523219\\
-0.676676676676677	-0.444831985582366\\
-0.684684684684685	-0.451953211101362\\
-0.692692692692693	-0.459034579795929\\
-0.700700700700701	-0.466075485618159\\
-0.708708708708709	-0.473075308261929\\
-0.716716716716717	-0.480033411680897\\
-0.724724724724725	-0.486949142619808\\
-0.732732732732733	-0.493821829159897\\
-0.740740740740741	-0.500650779279104\\
-0.748748748748749	-0.507435279427928\\
-0.756756756756757	-0.514174593121755\\
-0.764764764764765	-0.520867959550564\\
-0.772772772772773	-0.527514592207101\\
-0.780780780780781	-0.534113677534672\\
-0.788788788788789	-0.54066437359597\\
-0.796796796796797	-0.547165808764534\\
-0.804804804804805	-0.55361708044063\\
-0.812812812812813	-0.560017253793798\\
-0.820820820820821	-0.566365360534429\\
-0.828828828828829	-0.572660397717268\\
-0.836836836836837	-0.578901326580105\\
-0.844844844844845	-0.58508707142133\\
-0.852852852852853	-0.5912165185206\\
-0.860860860860861	-0.597288515107386\\
-0.868868868868869	-0.60330186838271\\
-0.876876876876877	-0.609255344600182\\
-0.884884884884885	-0.615147668212968\\
-0.892892892892893	-0.620977521094157\\
-0.900900900900901	-0.626743541838788\\
-0.908908908908909	-0.632444325156611\\
-0.916916916916917	-0.638078421365512\\
-0.924924924924925	-0.643644335996458\\
-0.932932932932933	-0.649140529521802\\
-0.940940940940941	-0.654565417219684\\
-0.948948948948949	-0.659917369188211\\
-0.956956956956957	-0.665194710524175\\
-0.964964964964965	-0.670395721681934\\
-0.972972972972973	-0.675518639028945\\
-0.980980980980981	-0.680561655615439\\
-0.988988988988989	-0.685522922176313\\
-0.996996996996997	-0.690400548384107\\
-1.00500500500501	-0.695192604372385\\
-1.01301301301301	-0.699897122549083\\
-1.02102102102102	-0.70451209971974\\
-1.02902902902903	-0.709035499540005\\
-1.03703703703704	-0.71346525531674\\
-1.04504504504505	-0.717799273175938\\
-1.05305305305305	-0.72203543561462\\
-1.06106106106106	-0.726171605452218\\
-1.06906906906907	-0.73020563019482\\
-1.07707707707708	-0.734135346822938\\
-1.08508508508509	-0.737958587010438\\
-1.09309309309309	-0.741673182778455\\
-1.1011011011011	-0.745276972583548\\
-1.10910910910911	-0.748767807834943\\
-1.11711711711712	-0.75214355982975\\
-1.12512512512513	-0.755402127089341\\
-1.13313313313313	-0.758541443073591\\
-1.14114114114114	-0.761559484242865\\
-1.14914914914915	-0.764454278430606\\
-1.15715715715716	-0.76722391348214\\
-1.16516516516517	-0.769866546108238\\
-1.17317317317317	-0.772380410894852\\
-1.18118118118118	-0.774763829404087\\
-1.18918918918919	-0.777015219295137\\
-1.1971971971972	-0.779133103389039\\
-1.20520520520521	-0.781116118596545\\
-1.21321321321321	-0.782963024625575\\
-1.22122122122122	-0.784672712382852\\
-1.22922922922923	-0.786244211984127\\
-1.23723723723724	-0.787676700288784\\
-1.24524524524525	-0.788969507877536\\
-1.25325325325325	-0.790122125396655\\
-1.26126126126126	-0.79113420919839\\
-1.26926926926927	-0.792005586215086\\
-1.27727727727728	-0.792736258013546\\
-1.28528528528529	-0.793326403986651\\
-1.29329329329329	-0.793776383650406\\
-1.3013013013013	-0.794086738026373\\
-1.30930930930931	-0.794258190101911\\
-1.31731731731732	-0.79429164437267\\
-1.32532532532533	-0.794188185483858\\
-1.33333333333333	-0.79394907599826\\
-1.34134134134134	-0.79357575332948\\
-1.34934934934935	-0.793069825888404\\
-1.35735735735736	-0.792433068499113\\
-1.36536536536537	-0.791667417147284\\
-1.37337337337337	-0.790774963129509\\
-1.38138138138138	-0.7897579466756\\
-1.38938938938939	-0.788618750118366\\
-1.3973973973974	-0.787359890686017\\
-1.40540540540541	-0.785984012991813\\
-1.41341341341341	-0.784493881293717\\
-1.42142142142142	-0.78289237159381\\
-1.42942942942943	-0.781182463643411\\
-1.43743743743744	-0.779367232915198\\
-1.44544544544545	-0.777449842598339\\
-1.45345345345345	-0.775433535667077\\
-1.46146146146146	-0.773321627067351\\
-1.46946946946947	-0.77111749606008\\
-1.47747747747748	-0.768824578753722\\
-1.48548548548549	-0.766446360853148\\
-1.49349349349349	-0.763986370646146\\
-1.5015015015015	-0.761448172243879\\
-1.50950950950951	-0.758835359086564\\
-1.51751751751752	-0.756151547721644\\
-1.52552552552553	-0.753400371857157\\
-1.53353353353353	-0.750585476690165\\
-1.54154154154154	-0.74771051350654\\
-1.54954954954955	-0.744779134546038\\
-1.55755755755756	-0.741794988124066\\
-1.56556556556557	-0.738761714000204\\
-1.57357357357357	-0.735682938981459\\
-1.58158158158158	-0.732562272747333\\
-1.58958958958959	-0.729403303883064\\
-1.5975975975976	-0.72620959610629\\
-1.60560560560561	-0.722984684672341\\
-1.61361361361361	-0.719732072942885\\
-1.62162162162162	-0.716455229102634\\
-1.62962962962963	-0.713157583008763\\
-1.63763763763764	-0.709842523157817\\
-1.64564564564565	-0.706513393755042\\
-1.65365365365365	-0.703173491871261\\
-1.66166166166166	-0.699826064672646\\
-1.66966966966967	-0.696474306709028\\
-1.67767767767768	-0.693121357246495\\
-1.68568568568569	-0.689770297630347\\
-1.69369369369369	-0.686424148664605\\
-1.7017017017017	-0.683085867994273\\
-1.70970970970971	-0.679758347476837\\
-1.71771771771772	-0.676444410529286\\
-1.72572572572573	-0.67314680943682\\
-1.73373373373373	-0.669868222609574\\
-1.74174174174174	-0.666611251772917\\
-1.74974974974975	-0.663378419076987\\
-1.75775775775776	-0.660172164110462\\
-1.76576576576577	-0.656994840802915\\
-1.77377377377377	-0.653848714199683\\
-1.78178178178178	-0.650735957092145\\
-1.78978978978979	-0.647658646485405\\
-1.7977977977978	-0.644618759884498\\
-1.80580580580581	-0.641618171378804\\
-1.81381381381381	-0.638658647503082\\
-1.82182182182182	-0.63574184285217\\
-1.82982982982983	-0.632869295424567\\
-1.83783783783784	-0.630042421668379\\
-1.84584584584585	-0.62726251120114\\
-1.85385385385385	-0.624530721173004\\
-1.86186186186186	-0.621848070240252\\
-1.86986986986987	-0.619215432114001\\
-1.87787787787788	-0.616633528646206\\
-1.88588588588589	-0.614102922412443\\
-1.89389389389389	-0.611624008748339\\
-1.9019019019019	-0.609197007193801\\
-1.90990990990991	-0.606821952296228\\
-1.91791791791792	-0.604498683721743\\
-1.92592592592593	-0.602226835620838\\
-1.93393393393393	-0.600005825192938\\
-1.94194194194194	-0.597834840393161\\
-1.94994994994995	-0.59571282672374\\
-1.95795795795796	-0.593638473053207\\
-1.96596596596597	-0.591610196408355\\
-1.97397397397397	-0.589626125687838\\
-1.98198198198198	-0.587684084252538\\
-1.98998998998999	-0.585781571357191\\
-1.997997997998	-0.58391574240124\\
-2.00600600600601	-0.582083387994824\\
-2.01401401401401	-0.580280911859983\\
-2.02202202202202	-0.578504307618732\\
-2.03003003003003	-0.576749134559451\\
-2.03803803803804	-0.57501049252336\\
-2.04604604604605	-0.573282996115607\\
-2.05405405405405	-0.571560748521105\\
-2.06206206206206	-0.569837315297146\\
-2.07007007007007	-0.568105698623392\\
-2.07807807807808	-0.566358312616303\\
-2.08608608608609	-0.564586960459476\\
-2.09409409409409	-0.56278281426385\\
-2.1021021021021	-0.560936398747246\\
-2.11011011011011	-0.559037580008925\\
-2.11811811811812	-0.557075560861028\\
-2.12612612612613	-0.555038884355743\\
-2.13413413413413	-0.552915447295038\\
-2.14214214214214	-0.550692525612689\\
-2.15015015015015	-0.548356813546196\\
-2.15815815815816	-0.545894478443476\\
-2.16616616616617	-0.543291232844127\\
-2.17417417417417	-0.540532425109213\\
-2.18218218218218	-0.537603149323421\\
-2.19019019019019	-0.534488374448539\\
-2.1981981981982	-0.531173091774102\\
-2.20620620620621	-0.527642478619945\\
-2.21421421421421	-0.523882075054567\\
-2.22222222222222	-0.519877969191402\\
-2.23023023023023	-0.51561698552182\\
-2.23823823823824	-0.511086869865911\\
-2.24624624624625	-0.506276463995978\\
-2.25425425425425	-0.501175862913456\\
-2.26226226226226	-0.49577654820356\\
-2.27027027027027	-0.49007149185806\\
-2.27827827827828	-0.484055226386327\\
-2.28628628628629	-0.47772387880798\\
-2.29429429429429	-0.471075168073625\\
-2.3023023023023	-0.464108367400838\\
-2.31031031031031	-0.456824234762586\\
-2.31831831831832	-0.449224916172221\\
-2.32632632632633	-0.441313827374258\\
-2.33433433433433	-0.433095520035706\\
-2.34234234234234	-0.424575538556631\\
-2.35035035035035	-0.415760273246275\\
-2.35835835835836	-0.406656814938508\\
-2.36636636636637	-0.397272815255173\\
-2.37437437437437	-0.387616355771908\\
-2.38238238238238	-0.377695828388335\\
-2.39039039039039	-0.36751982832195\\
-2.3983983983984	-0.357097060376177\\
-2.40640640640641	-0.346436258503204\\
-2.41441441441441	-0.335546118197108\\
-2.42242242242242	-0.324435240903692\\
-2.43043043043043	-0.31311208940759\\
-2.43843843843844	-0.301584953031657\\
-2.44644644644645	-0.289861921440615\\
-2.45445445445445	-0.277950865856571\\
-2.46246246246246	-0.265859426553775\\
-2.47047047047047	-0.253595005586641\\
-2.47847847847848	-0.241164763807389\\
-2.48648648648649	-0.228575621338072\\
-2.49449449449449	-0.215834260770936\\
-2.5025025025025	-0.202947132474113\\
-2.51051051051051	-0.189920461476047\\
-2.51851851851852	-0.17676025548925\\
-2.52652652652653	-0.163472313710761\\
-2.53453453453453	-0.150062236103981\\
-2.54254254254254	-0.136535432925001\\
-2.55055055055055	-0.122897134305156\\
-2.55855855855856	-0.109152399743386\\
-2.56656656656657	-0.0953061273964994\\
-2.57457457457457	-0.081363063083739\\
-2.58258258258258	-0.0673278089454969\\
-2.59059059059059	-0.0532048317146044\\
-2.5985985985986	-0.0389984705742463\\
-2.60660660660661	-0.0247129445877524\\
-2.61461461461461	-0.0103523596957187\\
-2.62262262262262	0.00407928471751073\\
-2.63063063063063	0.0185780896800452\\
-2.63863863863864	0.0331402508995546\\
-2.64664664664665	0.0477620534263963\\
-2.65465465465465	0.0624398666959819\\
-2.66266266266266	0.077170139932308\\
-2.67067067067067	0.0919493978937194\\
-2.67867867867868	0.106774236941824\\
-2.68668668668669	0.121641321414602\\
-2.69469469469469	0.136547380284816\\
-2.7027027027027	0.151489204085573\\
-2.71071071071071	0.166463642085488\\
-2.71871871871872	0.181467599696493\\
-2.72672672672673	0.196498036098261\\
-2.73473473473473	0.211551962064052\\
-2.74274274274274	0.226626437973495\\
-2.75075075075075	0.241718571998743\\
-2.75875875875876	0.256825518451318\\
-2.76676676676677	0.271944476277598\\
-2.77477477477477	0.287072687691726\\
-2.78278278278278	0.302207436935634\\
-2.79079079079079	0.317346049156258\\
-2.7987987987988	0.332485889390859\\
-2.80680680680681	0.347624361652103\\
-2.81481481481481	0.362758908104842\\
-2.82282282282282	0.377887008327334\\
-2.83083083083083	0.393006178650093\\
-2.83883883883884	0.408113971565983\\
-2.84684684684685	0.423207975205615\\
-2.85485485485485	0.438285812872732\\
-2.86286286286286	0.453345142634272\\
-2.87087087087087	0.468383656960529\\
-2.87887887887888	0.483399082411009\\
-2.88688688688689	0.498389179361803\\
-2.89489489489489	0.513351741770788\\
-2.9029029029029	0.528284596977101\\
-2.91091091091091	0.543185605531595\\
-2.91891891891892	0.558052661055225\\
-2.92692692692693	0.572883690122595\\
-2.93493493493493	0.587676652167997\\
-2.94294294294294	0.602429539411483\\
-2.95095095095095	0.617140376802721\\
-2.95895895895896	0.631807221980555\\
-2.96696696696697	0.646428165246259\\
-2.97497497497497	0.661001329548631\\
-2.98298298298298	0.675524870479367\\
-2.99099099099099	0.689996976276964\\
-2.998998998999	0.704415867837768\\
-3.00700700700701	0.71877979873287\\
-3.01501501501502	0.733087055229433\\
-3.02302302302302	0.747335956315374\\
-3.03103103103103	0.761524853726326\\
-3.03903903903904	0.775652131973758\\
-3.04704704704705	0.789716208373398\\
-3.05505505505506	0.80371553307309\\
-3.06306306306306	0.817648589079196\\
-3.07107107107107	0.831513892280861\\
-3.07907907907908	0.845309991471423\\
-3.08708708708709	0.859035468366342\\
-3.0950950950951	0.872688937617\\
-3.1031031031031	0.886269046819897\\
-3.11111111111111	0.899774476520705\\
-3.11911911911912	0.913203940212703\\
-3.12712712712713	0.926556184329197\\
-3.13513513513514	0.939829988229519\\
-3.14314314314314	0.953024164178274\\
-3.15115115115115	0.966137557317475\\
-3.15915915915916	0.979169045631318\\
-3.16716716716717	0.992117539903309\\
-3.17517517517518	1.0049819836655\\
-3.18318318318318	1.01776135313969\\
-3.19119119119119	1.03045465717028\\
-3.1991991991992	1.04306093714877\\
-3.20720720720721	1.05557926692966\\
-3.21521521521522	1.06800875273763\\
-3.22322322322322	1.080348533066\\
-3.23123123123123	1.09259777856626\\
-3.23923923923924	1.10475569192871\\
-3.24724724724725	1.11682150775409\\
-3.25525525525526	1.12879449241626\\
-3.26326326326326	1.14067394391583\\
-3.27127127127127	1.1524591917247\\
-3.27927927927928	1.16414959662176\\
-3.28728728728729	1.17574455051946\\
-3.2952952952953	1.18724347628151\\
-3.3033033033033	1.1986458275317\\
-3.31131131131131	1.20995108845387\\
-3.31931931931932	1.22115877358314\\
-3.32732732732733	1.23226842758851\\
-3.33533533533534	1.24327962504682\\
-3.34334334334334	1.25419197020832\\
-3.35135135135135	1.26500509675379\\
-3.35935935935936	1.27571866754357\\
-3.36736736736737	1.2863323743583\\
-3.37537537537538	1.29684593763187\\
-3.38338338338338	1.30725910617637\\
-3.39139139139139	1.31757165689954\\
-3.3993993993994	1.32778339451453\\
-3.40740740740741	1.33789415124243\\
-3.41541541541542	1.34790378650754\\
-3.42342342342342	1.35781218662565\\
-3.43143143143143	1.36761926448546\\
-3.43943943943944	1.37732495922338\\
-3.44744744744745	1.38692923589183\\
-3.45545545545546	1.39643208512127\\
-3.46346346346346	1.40583352277622\\
-3.47147147147147	1.41513358960528\\
-3.47947947947948	1.42433235088556\\
-3.48748748748749	1.43342989606165\\
-3.4954954954955	1.4424263383793\\
-3.5035035035035	1.45132181451404\\
-3.51151151151151	1.46011648419501\\
-3.51951951951952	1.46881052982417\\
-3.52752752752753	1.47740415609108\\
-3.53553553553554	1.48589758958352\\
-3.54354354354354	1.49429107839419\\
-3.55155155155155	1.50258489172364\\
-3.55955955955956	1.51077931947972\\
-3.56756756756757	1.51887467187372\\
-3.57557557557558	1.52687127901348\\
-3.58358358358358	1.53476949049367\\
-3.59159159159159	1.54256967498347\\
-3.5995995995996	1.55027221981184\\
-3.60760760760761	1.55787753055072\\
-3.61561561561562	1.56538603059619\\
-3.62362362362362	1.57279816074808\\
-3.63163163163163	1.58011437878795\\
-3.63963963963964	1.58733515905598\\
-3.64764764764765	1.59446099202669\\
-3.65565565565566	1.601492383884\\
-3.66366366366366	1.60842985609556\\
-3.67167167167167	1.61527394498689\\
-3.67967967967968	1.62202520131519\\
-3.68768768768769	1.6286841898434\\
-3.6956956956957	1.63525148891441\\
-3.7037037037037	1.64172769002587\\
-3.71171171171171	1.64811339740562\\
-3.71971971971972	1.65440922758813\\
-3.72772772772773	1.66061580899198\\
-3.73573573573574	1.66673378149873\\
-3.74374374374374	1.67276379603323\\
-3.75175175175175	1.67870651414579\\
-3.75975975975976	1.68456260759609\\
-3.76776776776777	1.69033275793936\\
-3.77577577577578	1.69601765611473\\
-3.78378378378378	1.70161800203613\\
-3.79179179179179	1.70713450418583\\
-3.7997997997998	1.7125678792108\\
-3.80780780780781	1.71791885152208\\
-3.81581581581582	1.72318815289737\\
-3.82382382382382	1.7283765220869\\
-3.83183183183183	1.73348470442288\\
-3.83983983983984	1.73851345143253\\
-3.84784784784785	1.74346352045501\\
-3.85585585585586	1.74833567426227\\
-3.86386386386386	1.75313068068403\\
-3.87187187187187	1.75784931223699\\
-3.87987987987988	1.76249234575852\\
-3.88788788788789	1.76706056204472\\
-3.8958958958959	1.77155474549329\\
-3.9039039039039	1.77597568375108\\
-3.91191191191191	1.78032416736656\\
-3.91991991991992	1.7846009894473\\
-3.92792792792793	1.78880694532258\\
-3.93593593593594	1.79294283221118\\
-3.94394394394394	1.79700944889454\\
-3.95195195195195	1.80100759539531\\
-3.95995995995996	1.80493807266142\\
-3.96796796796797	1.80880168225571\\
-3.97597597597598	1.81259922605129\\
-3.98398398398398	1.81633150593259\\
-3.99199199199199	1.81999932350225\\
-4	1.82360347979392\\
}--cycle;

\addlegendentry{$\pm 2\sqrt{\diag K_{\vec{y}_\ast\given\data}}$};

\addplot [color=mycolor2,solid]
  table[row sep=crcr]{%
-4	-0.0906182858650599\\
-3.99199199199199	-0.0919208007978954\\
-3.98398398398398	-0.0932367436940547\\
-3.97597597597598	-0.0945661665489521\\
-3.96796796796797	-0.0959091203453838\\
-3.95995995995996	-0.0972656550437897\\
-3.95195195195195	-0.0986358195729289\\
-3.94394394394394	-0.100019661820982\\
-3.93593593593594	-0.101417228627087\\
-3.92792792792793	-0.102828565773307\\
-3.91991991991992	-0.104253717977058\\
-3.91191191191191	-0.105692728883984\\
-3.9039039039039	-0.107145641061296\\
-3.8958958958959	-0.108612495991582\\
-3.88788788788789	-0.110093334067093\\
-3.87987987987988	-0.111588194584506\\
-3.87187187187187	-0.113097115740187\\
-3.86386386386386	-0.114620134625935\\
-3.85585585585586	-0.116157287225233\\
-3.84784784784785	-0.117708608410002\\
-3.83983983983984	-0.119274131937865\\
-3.83183183183183	-0.120853890449919\\
-3.82382382382382	-0.12244791546903\\
-3.81581581581582	-0.124056237398645\\
-3.80780780780781	-0.125678885522134\\
-3.7997997997998	-0.127315888002648\\
-3.79179179179179	-0.128967271883513\\
-3.78378378378378	-0.130633063089154\\
-3.77577577577578	-0.132313286426552\\
-3.76776776776777	-0.134007965587226\\
-3.75975975975976	-0.135717123149759\\
-3.75175175175175	-0.137440780582853\\
-3.74374374374374	-0.139178958248913\\
-3.73573573573574	-0.140931675408171\\
-3.72772772772773	-0.142698950223334\\
-3.71971971971972	-0.144480799764763\\
-3.71171171171171	-0.146277240016178\\
-3.7037037037037	-0.148088285880888\\
-3.6956956956957	-0.149913951188538\\
-3.68768768768769	-0.151754248702371\\
-3.67967967967968	-0.153609190127006\\
-3.67167167167167	-0.155478786116711\\
-3.66366366366366	-0.157363046284194\\
-3.65565565565566	-0.159261979209868\\
-3.64764764764765	-0.161175592451614\\
-3.63963963963964	-0.163103892555021\\
-3.63163163163163	-0.1650468850641\\
-3.62362362362362	-0.167004574532454\\
-3.61561561561562	-0.168976964534907\\
-3.60760760760761	-0.170964057679574\\
-3.5995995995996	-0.172965855620365\\
-3.59159159159159	-0.174982359069913\\
-3.58358358358358	-0.177013567812906\\
-3.57557557557558	-0.179059480719822\\
-3.56756756756757	-0.181120095761052\\
-3.55955955955956	-0.183195410021384\\
-3.55155155155155	-0.185285419714858\\
-3.54354354354354	-0.187390120199952\\
-3.53553553553554	-0.189509505995099\\
-3.52752752752753	-0.19164357079452\\
-3.51951951951952	-0.193792307484345\\
-3.51151151151151	-0.195955708159012\\
-3.5035035035035	-0.198133764137931\\
-3.4954954954955	-0.200326465982381\\
-3.48748748748749	-0.202533803512635\\
-3.47947947947948	-0.204755765825286\\
-3.47147147147147	-0.206992341310749\\
-3.46346346346346	-0.20924351767093\\
-3.45545545545546	-0.211509281937036\\
-3.44744744744745	-0.213789620487493\\
-3.43943943943944	-0.216084519065968\\
-3.43143143143143	-0.218393962799457\\
-3.42342342342342	-0.220717936216422\\
-3.41541541541542	-0.223056423264948\\
-3.40740740740741	-0.225409407330908\\
-3.3993993993994	-0.227776871256086\\
-3.39139139139139	-0.230158797356256\\
-3.38338338338338	-0.232555167439183\\
-3.37537537537538	-0.23496596282251\\
-3.36736736736737	-0.237391164351513\\
-3.35935935935936	-0.239830752416696\\
-3.35135135135135	-0.242284706971198\\
-3.34334334334334	-0.244753007547969\\
-3.33533533533534	-0.247235633276707\\
-3.32732732732733	-0.249732562900516\\
-3.31931931931932	-0.25224377479225\\
-3.31131131131131	-0.254769246970519\\
-3.3033033033033	-0.257308957115328\\
-3.2952952952953	-0.259862882583312\\
-3.28728728728729	-0.262431000422535\\
-3.27927927927928	-0.265013287386834\\
-3.27127127127127	-0.267609719949651\\
-3.26326326326326	-0.270220274317348\\
-3.25525525525526	-0.272844926441955\\
-3.24724724724725	-0.275483652033321\\
-3.23923923923924	-0.278136426570643\\
-3.23123123123123	-0.280803225313326\\
-3.22322322322322	-0.283484023311151\\
-3.21521521521522	-0.28617879541372\\
-3.20720720720721	-0.288887516279132\\
-3.1991991991992	-0.291610160381866\\
-3.19119119119119	-0.294346702019833\\
-3.18318318318318	-0.297097115320572\\
-3.17517517517518	-0.299861374246539\\
-3.16716716716717	-0.302639452599472\\
-3.15915915915916	-0.305431324023793\\
-3.15115115115115	-0.308236962009006\\
-3.14314314314314	-0.311056339891072\\
-3.13513513513514	-0.313889430852717\\
-3.12712712712713	-0.316736207922637\\
-3.11911911911912	-0.319596643973586\\
-3.11111111111111	-0.322470711719287\\
-3.1031031031031	-0.325358383710163\\
-3.0950950950951	-0.32825963232783\\
-3.08708708708709	-0.33117442977834\\
-3.07907907907908	-0.334102748084128\\
-3.07107107107107	-0.337044559074648\\
-3.06306306306306	-0.339999834375654\\
-3.05505505505506	-0.342968545397098\\
-3.04704704704705	-0.345950663319628\\
-3.03903903903904	-0.348946159079642\\
-3.03103103103103	-0.351955003352882\\
-3.02302302302302	-0.354977166536526\\
-3.01501501501502	-0.358012618729774\\
-3.00700700700701	-0.361061329712882\\
-2.998998998999	-0.364123268924618\\
-2.99099099099099	-0.367198405438138\\
-2.98298298298298	-0.370286707935238\\
-2.97497497497497	-0.373388144678953\\
-2.96696696696697	-0.376502683484516\\
-2.95895895895896	-0.379630291688617\\
-2.95095095095095	-0.382770936116956\\
-2.94294294294294	-0.385924583050091\\
-2.93493493493493	-0.389091198187525\\
-2.92692692692693	-0.392270746610052\\
-2.91891891891892	-0.395463192740322\\
-2.91091091091091	-0.398668500301623\\
-2.9029029029029	-0.401886632274857\\
-2.89489489489489	-0.405117550853712\\
-2.88688688688689	-0.408361217398002\\
-2.87887887887888	-0.411617592385183\\
-2.87087087087087	-0.414886635360023\\
-2.86286286286286	-0.418168304882423\\
-2.85485485485485	-0.421462558473393\\
-2.84684684684685	-0.424769352559159\\
-2.83883883883884	-0.428088642413424\\
-2.83083083083083	-0.431420382097761\\
-2.82282282282282	-0.434764524400145\\
-2.81481481481481	-0.438121020771634\\
-2.80680680680681	-0.441489821261191\\
-2.7987987987988	-0.444870874448657\\
-2.79079079079079	-0.44826412737589\\
-2.78278278278278	-0.45166952547606\\
-2.77477477477477	-0.455087012501127\\
-2.76676676676677	-0.458516530447514\\
-2.75875875875876	-0.461958019479973\\
-2.75075075075075	-0.465411417853674\\
-2.74274274274274	-0.468876661834537\\
-2.73473473473473	-0.472353685617805\\
-2.72672672672673	-0.475842421244911\\
-2.71871871871872	-0.479342798518629\\
-2.71071071071071	-0.482854744916559\\
-2.7027027027027	-0.486378185502959\\
-2.69469469469469	-0.489913042838959\\
-2.68668668668669	-0.493459236891183\\
-2.67867867867868	-0.497016684938809\\
-2.67067067067067	-0.50058530147912\\
-2.66266266266266	-0.504164998131553\\
-2.65465465465465	-0.507755683540306\\
-2.64664664664665	-0.511357263275543\\
-2.63863863863864	-0.514969639733217\\
-2.63063063063063	-0.518592712033591\\
-2.62262262262262	-0.522226375918469\\
-2.61461461461461	-0.525870523647208\\
-2.60660660660661	-0.529525043891552\\
-2.5985985985986	-0.533189821629337\\
-2.59059059059059	-0.536864738037132\\
-2.58258258258258	-0.540549670381864\\
-2.57457457457457	-0.544244491911485\\
-2.56656656656657	-0.547949071744749\\
-2.55855855855856	-0.551663274760149\\
-2.55055055055055	-0.55538696148409\\
-2.54254254254254	-0.559119987978355\\
-2.53453453453453	-0.562862205726941\\
-2.52652652652653	-0.566613461522321\\
-2.51851851851852	-0.570373597351224\\
-2.51051051051051	-0.574142450279984\\
-2.5025025025025	-0.577919852339551\\
-2.49449449449449	-0.581705630410232\\
-2.48648648648649	-0.585499606106247\\
-2.47847847847848	-0.589301595660169\\
-2.47047047047047	-0.593111409807351\\
-2.46246246246246	-0.596928853670405\\
-2.45445445445445	-0.600753726643826\\
-2.44644644644645	-0.604585822278849\\
-2.43843843843844	-0.608424928168634\\
-2.43043043043043	-0.612270825833853\\
-2.42242242242242	-0.616123290608791\\
-2.41441441441441	-0.619982091528042\\
-2.40640640640641	-0.623846991213891\\
-2.3983983983984	-0.627717745764502\\
-2.39039039039039	-0.631594104642978\\
-2.38238238238238	-0.635475810567415\\
-2.37437437437437	-0.639362599402042\\
-2.36636636636637	-0.643254200049547\\
-2.35835835835836	-0.647150334344698\\
-2.35035035035035	-0.651050716949349\\
-2.34234234234234	-0.654955055248959\\
-2.33433433433433	-0.658863049250698\\
-2.32632632632633	-0.662774391483281\\
-2.31831831831832	-0.666688766898602\\
-2.31031031031031	-0.670605852775308\\
-2.3023023023023	-0.674525318624396\\
-2.29429429429429	-0.678446826096959\\
-2.28628628628629	-0.682370028894184\\
-2.27827827827828	-0.686294572679706\\
-2.27027027027027	-0.690220094994441\\
-2.26226226226226	-0.694146225174001\\
-2.25425425425425	-0.698072584268796\\
-2.24624624624625	-0.701998784966956\\
-2.23823823823824	-0.70592443152015\\
-2.23023023023023	-0.709849119672442\\
-2.22222222222222	-0.713772436592287\\
-2.21421421421421	-0.717693960807767\\
-2.20620620620621	-0.72161326214519\\
-2.1981981981982	-0.725529901671156\\
-2.19019019019019	-0.729443431638198\\
-2.18218218218218	-0.733353395434112\\
-2.17417417417417	-0.737259327535081\\
-2.16616616616617	-0.7411607534627\\
-2.15815815815816	-0.745057189745005\\
-2.15015015015015	-0.748948143881622\\
-2.14214214214214	-0.752833114313134\\
-2.13413413413413	-0.75671159039476\\
-2.12612612612613	-0.760583052374466\\
-2.11811811811812	-0.764446971375606\\
-2.11011011011011	-0.768302809384168\\
-2.1021021021021	-0.772150019240759\\
-2.09409409409409	-0.775988044637402\\
-2.08608608608609	-0.779816320119244\\
-2.07807807807808	-0.783634271091268\\
-2.07007007007007	-0.787441313830104\\
-2.06206206206206	-0.791236855501028\\
-2.05405405405405	-0.795020294180215\\
-2.04604604604605	-0.798791018882373\\
-2.03803803803804	-0.802548409593798\\
-2.03003003003003	-0.806291837310942\\
-2.02202202202202	-0.810020664084594\\
-2.01401401401401	-0.813734243069715\\
-2.00600600600601	-0.817431918581016\\
-1.997997997998	-0.821113026154355\\
-1.98998998998999	-0.824776892614001\\
-1.98198198198198	-0.828422836145837\\
-1.97397397397397	-0.832050166376573\\
-1.96596596596597	-0.835658184459009\\
-1.95795795795796	-0.839246183163401\\
-1.94994994994995	-0.842813446975004\\
-1.94194194194194	-0.846359252197813\\
-1.93393393393393	-0.849882867064551\\
-1.92592592592593	-0.853383551852961\\
-1.91791791791792	-0.856860559008416\\
-1.90990990990991	-0.860313133272887\\
-1.9019019019019	-0.863740511820308\\
-1.89389389389389	-0.867141924398347\\
-1.88588588588589	-0.870516593476608\\
-1.87787787787788	-0.873863734401294\\
-1.86986986986987	-0.87718255555632\\
-1.86186186186186	-0.880472258530902\\
-1.85385385385385	-0.88373203829362\\
-1.84584584584585	-0.886961083372957\\
-1.83783783783784	-0.890158576044302\\
-1.82982982982983	-0.893323692523411\\
-1.82182182182182	-0.896455603166325\\
-1.81381381381381	-0.899553472675703\\
-1.80580580580581	-0.90261646031356\\
-1.7977977977978	-0.905643720120393\\
-1.78978978978979	-0.908634401140632\\
-1.78178178178178	-0.911587647654416\\
-1.77377377377377	-0.914502599415613\\
-1.76576576576577	-0.917378391896085\\
-1.75775775775776	-0.920214156536084\\
-1.74974974974975	-0.923009021000782\\
-1.74174174174174	-0.925762109442844\\
-1.73373373373373	-0.928472542770966\\
-1.72572572572573	-0.931139438924343\\
-1.71771771771772	-0.933761913152945\\
-1.70970970970971	-0.936339078303572\\
-1.7017017017017	-0.938870045111541\\
-1.69369369369369	-0.941353922497968\\
-1.68568568568569	-0.943789817872515\\
-1.67767767767768	-0.946176837441509\\
-1.66966966966967	-0.948514086521333\\
-1.66166166166166	-0.950800669856972\\
-1.65365365365365	-0.953035691945587\\
-1.64564564564565	-0.955218257365014\\
-1.63763763763764	-0.957347471107057\\
-1.62962962962963	-0.959422438915423\\
-1.62162162162162	-0.961442267628198\\
-1.61361361361361	-0.963406065524681\\
-1.60560560560561	-0.965312942676485\\
-1.5975975975976	-0.967162011302681\\
-1.58958958958959	-0.968952386128902\\
-1.58158158158158	-0.970683184750203\\
-1.57357357357357	-0.97235352799752\\
-1.56556556556557	-0.973962540307568\\
-1.55755755755756	-0.975509350096002\\
-1.54954954954955	-0.976993090133653\\
-1.54154154154154	-0.97841289792566\\
-1.53353353353353	-0.979767916093331\\
-1.52552552552553	-0.981057292758508\\
-1.51751751751752	-0.982280181930261\\
-1.50950950950951	-0.983435743893722\\
-1.5015015015015	-0.984523145600833\\
-1.49349349349349	-0.985541561062798\\
-1.48548548548549	-0.986490171744092\\
-1.47747747747748	-0.987368166957695\\
-1.46946946946947	-0.988174744261455\\
-1.46146146146146	-0.988909109855287\\
-1.45345345345345	-0.989570478979002\\
-1.44544544544545	-0.990158076310546\\
-1.43743743743744	-0.990671136364408\\
-1.42942942942943	-0.991108903889977\\
-1.42142142142142	-0.991470634269594\\
-1.41341341341341	-0.991755593916082\\
-1.40540540540541	-0.991963060669498\\
-1.3973973973974	-0.992092324192875\\
-1.38938938938939	-0.992142686366687\\
-1.38138138138138	-0.992113461681827\\
-1.37337337337337	-0.992003977630819\\
-1.36536536536537	-0.991813575097022\\
-1.35735735735736	-0.991541608741566\\
-1.34934934934935	-0.991187447387805\\
-1.34134134134134	-0.990750474402967\\
-1.33333333333333	-0.990230088076808\\
-1.32532532532533	-0.989625701996971\\
-1.31731731731732	-0.988936745420822\\
-1.30930930930931	-0.988162663643486\\
-1.3013013013013	-0.987302918361832\\
-1.29329329329329	-0.986356988034155\\
-1.28528528528529	-0.985324368235279\\
-1.27727727727728	-0.984204572006861\\
-1.26926926926927	-0.982997130202584\\
-1.26126126126126	-0.981701591828036\\
-1.25325325325325	-0.980317524374987\\
-1.24524524524525	-0.978844514149818\\
-1.23723723723724	-0.977282166595857\\
-1.22922922922923	-0.975630106609352\\
-1.22122122122122	-0.973887978848858\\
-1.21321321321321	-0.972055448037753\\
-1.20520520520521	-0.97013219925967\\
-1.1971971971972	-0.968117938246593\\
-1.18918918918919	-0.966012391659353\\
-1.18118118118118	-0.963815307360314\\
-1.17317317317317	-0.961526454678004\\
-1.16516516516517	-0.959145624663439\\
-1.15715715715716	-0.956672630337939\\
-1.14914914914915	-0.954107306932196\\
-1.14114114114114	-0.951449512116352\\
-1.13313313313313	-0.948699126220915\\
-1.12512512512513	-0.945856052448237\\
-1.11711711711712	-0.942920217074408\\
-1.10910910910911	-0.939891569641293\\
-1.1011011011011	-0.936770083138561\\
-1.09309309309309	-0.933555754175492\\
-1.08508508508509	-0.93024860314236\\
-1.07707707707708	-0.926848674361214\\
-1.06906906906907	-0.923356036225869\\
-1.06106106106106	-0.919770781330947\\
-1.05305305305305	-0.916093026589776\\
-1.04504504504505	-0.912322913340985\\
-1.03703703703704	-0.908460607443655\\
-1.02902902902903	-0.90450629936085\\
-1.02102102102102	-0.900460204231398\\
-1.01301301301301	-0.896322561929766\\
-1.00500500500501	-0.892093637113921\\
-0.996996996996997	-0.887773719261007\\
-0.988988988988989	-0.883363122690789\\
-0.980980980980981	-0.878862186576664\\
-0.972972972972973	-0.87427127494421\\
-0.964964964964965	-0.869590776657129\\
-0.956956956956957	-0.864821105390514\\
-0.948948948948949	-0.859962699591357\\
-0.940940940940941	-0.85501602242621\\
-0.932932932932933	-0.849981561715952\\
-0.924924924924925	-0.844859829857585\\
-0.916916916916917	-0.839651363733021\\
-0.908908908908909	-0.834356724604817\\
-0.900900900900901	-0.828976497998809\\
-0.892892892892893	-0.823511293573646\\
-0.884884884884885	-0.81796174497718\\
-0.876876876876877	-0.812328509689729\\
-0.868868868868869	-0.80661226885419\\
-0.860860860860861	-0.800813727093027\\
-0.852852852852853	-0.794933612312148\\
-0.844844844844845	-0.7889726754917\\
-0.836836836836837	-0.782931690463802\\
-0.828828828828829	-0.776811453677296\\
-0.820820820820821	-0.770612783949532\\
-0.812812812812813	-0.764336522205266\\
-0.804804804804805	-0.757983531202766\\
-0.796796796796797	-0.751554695247139\\
-0.788788788788789	-0.74505091989107\\
-0.780780780780781	-0.738473131622979\\
-0.772772772772773	-0.731822277542756\\
-0.764764764764765	-0.725099325025176\\
-0.756756756756757	-0.71830526137113\\
-0.748748748748749	-0.711441093446795\\
-0.740740740740741	-0.70450784731089\\
-0.732732732732733	-0.697506567830189\\
-0.724724724724725	-0.690438318283429\\
-0.716716716716717	-0.683304179953808\\
-0.708708708708709	-0.67610525171022\\
-0.700700700700701	-0.668842649577465\\
-0.692692692692693	-0.661517506295594\\
-0.684684684684685	-0.654130970868599\\
-0.676676676676677	-0.64668420810268\\
-0.668668668668669	-0.639178398134306\\
-0.660660660660661	-0.631614735948294\\
-0.652652652652653	-0.623994430886142\\
-0.644644644644645	-0.616318706144901\\
-0.636636636636636	-0.608588798266785\\
-0.628628628628629	-0.600805956619832\\
-0.62062062062062	-0.59297144286986\\
-0.612612612612613	-0.585086530444011\\
-0.604604604604605	-0.57715250398615\\
-0.596596596596596	-0.56917065880446\\
-0.588588588588589	-0.561142300311455\\
-0.58058058058058	-0.553068743456814\\
-0.572572572572573	-0.544951312153261\\
-0.564564564564565	-0.536791338695875\\
-0.556556556556556	-0.528590163175139\\
-0.548548548548549	-0.520349132884033\\
-0.54054054054054	-0.512069601719568\\
-0.532532532532533	-0.503752929579055\\
-0.524524524524525	-0.495400481751478\\
-0.516516516516516	-0.487013628304337\\
-0.508508508508509	-0.478593743466313\\
-0.5005005005005	-0.470142205006118\\
-0.492492492492492	-0.4616603936079\\
-0.484484484484485	-0.453149692243599\\
-0.476476476476476	-0.444611485542589\\
-0.468468468468469	-0.436047159159064\\
-0.46046046046046	-0.427458099137455\\
-0.452452452452452	-0.418845691276348\\
-0.444444444444445	-0.41021132049128\\
-0.436436436436436	-0.401556370176773\\
-0.428428428428429	-0.392882221568053\\
-0.42042042042042	-0.38419025310281\\
-0.412412412412412	-0.375481839783453\\
-0.404404404404405	-0.366758352540205\\
-0.396396396396396	-0.358021157595497\\
-0.388388388388389	-0.349271615830045\\
-0.38038038038038	-0.340511082150994\\
-0.372372372372372	-0.331740904862609\\
-0.364364364364364	-0.322962425039831\\
-0.356356356356356	-0.314176975905187\\
-0.348348348348348	-0.305385882209398\\
-0.34034034034034	-0.296590459616158\\
-0.332332332332332	-0.287792014091424\\
-0.324324324324324	-0.278991841297669\\
-0.316316316316316	-0.270191225993505\\
-0.308308308308308	-0.261391441439027\\
-0.3003003003003	-0.252593748807357\\
-0.292292292292292	-0.243799396602702\\
-0.284284284284284	-0.235009620085401\\
-0.276276276276276	-0.226225640704293\\
-0.268268268268268	-0.217448665536816\\
-0.26026026026026	-0.208679886737248\\
-0.252252252252252	-0.199920480993418\\
-0.244244244244244	-0.191171608992325\\
-0.236236236236236	-0.182434414894988\\
-0.228228228228228	-0.173710025820918\\
-0.22022022022022	-0.164999551342569\\
-0.212212212212212	-0.156304082990166\\
-0.204204204204204	-0.147624693767158\\
-0.196196196196196	-0.13896243767677\\
-0.188188188188188	-0.130318349259916\\
-0.18018018018018	-0.1216934431448\\
-0.172172172172172	-0.1130887136086\\
-0.164164164164164	-0.104505134151475\\
-0.156156156156156	-0.0959436570832577\\
-0.148148148148148	-0.0874052131231396\\
-0.14014014014014	-0.0788907110126015\\
-0.132132132132132	-0.0704010371419259\\
-0.124124124124124	-0.0619370551905602\\
-0.116116116116116	-0.0534996057815735\\
-0.108108108108108	-0.0450895061505268\\
-0.1001001001001	-0.0367075498289495\\
-0.0920920920920922	-0.0283545063427201\\
-0.084084084084084	-0.0200311209255513\\
-0.0760760760760761	-0.0117381142478262\\
-0.0680680680680679	-0.00347618216100439\\
-0.06006006006006	0.00475400454223\\
-0.0520520520520522	0.0129518003518129\\
-0.0440440440440439	0.0211165852480378\\
-0.0360360360360361	0.0292477648929266\\
-0.0280280280280278	0.0373447708051367\\
-0.02002002002002	0.045407060518196\\
-0.0120120120120122	0.0534341177219789\\
-0.00400400400400391	0.0614254523872116\\
0.00400400400400436	0.0693806008729899\\
0.0120120120120122	0.077299126017098\\
0.02002002002002	0.0851806172090973\\
0.0280280280280278	0.093024690446083\\
0.0360360360360357	0.100830988371004\\
0.0440440440440444	0.108599180293543\\
0.0520520520520522	0.116328962193431\\
0.06006006006006	0.124020056706235\\
0.0680680680680679	0.13167221309152\\
0.0760760760760757	0.139285207183448\\
0.0840840840840844	0.146858841323701\\
0.0920920920920922	0.154392944276884\\
0.1001001001001	0.161887371128294\\
0.108108108108108	0.169342003164163\\
0.116116116116116	0.176756747734405\\
0.124124124124124	0.184131538097933\\
0.132132132132132	0.191466333250583\\
0.14014014014014	0.198761117735786\\
0.148148148148148	0.206015901438014\\
0.156156156156156	0.213230719359189\\
0.164164164164164	0.220405631378073\\
0.172172172172172	0.227540721992883\\
0.18018018018018	0.234636100047179\\
0.188188188188188	0.241691898439228\\
0.196196196196196	0.248708273815024\\
0.204204204204204	0.255685406245097\\
0.212212212212212	0.262623498885329\\
0.22022022022022	0.269522777621989\\
0.228228228228228	0.276383490701188\\
0.236236236236236	0.283205908342958\\
0.244244244244245	0.289990322340245\\
0.252252252252252	0.296737045642996\\
0.26026026026026	0.30344641192767\\
0.268268268268268	0.310118775152356\\
0.276276276276277	0.316754509097875\\
0.284284284284285	0.323354006895067\\
0.292292292292292	0.329917680538625\\
0.3003003003003	0.336445960387743\\
0.308308308308308	0.342939294653936\\
0.316316316316317	0.349398148876311\\
0.324324324324325	0.35582300538469\\
0.332332332332332	0.362214362750875\\
0.34034034034034	0.36857273522844\\
0.348348348348348	0.374898652181415\\
0.356356356356357	0.38119265750223\\
0.364364364364365	0.387455309019276\\
0.372372372372372	0.393687177894537\\
0.38038038038038	0.399888848011615\\
0.388388388388388	0.406060915354587\\
0.396396396396397	0.412203987378126\\
0.404404404404405	0.418318682369248\\
0.412412412412412	0.424405628801173\\
0.42042042042042	0.430465464679668\\
0.428428428428428	0.436498836882364\\
0.436436436436437	0.442506400491462\\
0.444444444444445	0.448488818120242\\
0.452452452452452	0.454446759233909\\
0.46046046046046	0.460380899465145\\
0.468468468468468	0.466291919924901\\
0.476476476476477	0.472180506508833\\
0.484484484484485	0.478047349199879\\
0.492492492492492	0.483893141367468\\
0.5005005005005	0.489718579063784\\
0.508508508508508	0.495524360317629\\
0.516516516516517	0.501311184426286\\
0.524524524524525	0.507079751245935\\
0.532532532532533	0.512830760481066\\
0.54054054054054	0.518564910973376\\
0.548548548548548	0.524282899990654\\
0.556556556556557	0.529985422516113\\
0.564564564564565	0.535673170538686\\
0.572572572572573	0.541346832344723\\
0.58058058058058	0.547007091811656\\
0.588588588588588	0.552654627704019\\
0.596596596596597	0.55829011297237\\
0.604604604604605	0.563914214055594\\
0.612612612612613	0.569527590187039\\
0.62062062062062	0.575130892704971\\
0.628628628628628	0.580724764367852\\
0.636636636636637	0.586309838674862\\
0.644644644644645	0.591886739192185\\
0.652652652652653	0.597456078885492\\
0.66066066066066	0.603018459459093\\
0.668668668668668	0.608574470702215\\
0.676676676676677	0.614124689842851\\
0.684684684684685	0.619669680909651\\
0.692692692692693	0.625209994102268\\
0.7007007007007	0.630746165170613\\
0.708708708708708	0.636278714803446\\
0.716716716716717	0.64180814802674\\
0.724724724724725	0.647334953612201\\
0.732732732732733	0.65285960349641\\
0.74074074074074	0.658382552210941\\
0.748748748748748	0.663904236323881\\
0.756756756756757	0.669425073893141\\
0.764764764764765	0.67494546393192\\
0.772772772772773	0.680465785886729\\
0.780780780780781	0.685986399128296\\
0.788788788788789	0.691507642455765\\
0.796796796796797	0.69702983361448\\
0.804804804804805	0.702553268827732\\
0.812812812812813	0.70807822234278\\
0.820820820820821	0.713604945991491\\
0.828828828828829	0.719133668765836\\
0.836836836836837	0.724664596408663\\
0.844844844844845	0.730197911019921\\
0.852852852852853	0.73573377067868\\
0.860860860860861	0.741272309081178\\
0.868868868868869	0.746813635195201\\
0.876876876876877	0.752357832930994\\
0.884884884884885	0.757904960828945\\
0.892892892892893	0.763455051764305\\
0.900900900900901	0.769008112669104\\
0.908908908908909	0.774564124271507\\
0.916916916916917	0.780123040852761\\
0.924924924924925	0.78568479002196\\
0.932932932932933	0.791249272508724\\
0.940940940940941	0.796816361974013\\
0.948948948948949	0.802385904839177\\
0.956956956956957	0.807957720133371\\
0.964964964964965	0.813531599359479\\
0.972972972972973	0.819107306378628\\
0.980980980980981	0.824684577313367\\
0.988988988988989	0.830263120469631\\
0.996996996996997	0.835842616277527\\
1.00500500500501	0.841422717250999\\
1.01301301301301	0.847003047966436\\
1.02102102102102	0.852583205060183\\
1.02902902902903	0.858162757245074\\
1.03703703703704	0.863741245345901\\
1.04504504504505	0.869318182353846\\
1.05305305305305	0.874893053499847\\
1.06106106106106	0.880465316346877\\
1.06906906906907	0.886034400901058\\
1.07707707707708	0.891599709741571\\
1.08508508508509	0.897160618169294\\
1.09309309309309	0.902716474374064\\
1.1011011011011	0.908266599620467\\
1.10910910910911	0.913810288452072\\
1.11711711711712	0.919346808913967\\
1.12512512512513	0.924875402793429\\
1.13313313313313	0.930395285878683\\
1.14114114114114	0.935905648235453\\
1.14914914914915	0.941405654501271\\
1.15715715715716	0.946894444197267\\
1.16516516516517	0.952371132057288\\
1.17317317317317	0.957834808374164\\
1.18118118118118	0.963284539362824\\
1.18918918918919	0.968719367540144\\
1.1971971971972	0.974138312121228\\
1.20520520520521	0.979540369431858\\
1.21321321321321	0.984924513336929\\
1.22122122122122	0.990289695684516\\
1.22922922922923	0.995634846765357\\
1.23723723723724	1.00095887578746\\
1.24524524524525	1.0062606713655\\
1.25325325325325	1.01153910202471\\
1.26126126126126	1.01679301671904\\
1.26926926926927	1.02202124536306\\
1.27727727727728	1.02722259937755\\
1.28528528528529	1.03239587224816\\
1.29329329329329	1.03753984009699\\
1.3013013013013	1.04265326226666\\
1.30930930930931	1.04773488191643\\
1.31731731731732	1.05278342663025\\
1.32532532532533	1.05779760903599\\
1.33333333333333	1.06277612743579\\
1.34134134134134	1.067717666447\\
1.34934934934935	1.07262089765326\\
1.35735735735736	1.07748448026551\\
1.36536536536537	1.08230706179226\\
1.37337337337337	1.08708727871888\\
1.38138138138138	1.09182375719553\\
1.38938938938939	1.09651511373315\\
1.3973973973974	1.10115995590712\\
1.40540540540541	1.10575688306832\\
1.41341341341341	1.11030448706093\\
1.42142142142142	1.11480135294664\\
1.42942942942943	1.11924605973486\\
1.43743743743744	1.12363718111844\\
1.44544544544545	1.12797328621436\\
1.45345345345345	1.13225294030917\\
1.46146146146146	1.13647470560849\\
1.46946946946947	1.14063714199022\\
1.47747747747748	1.14473880776105\\
1.48548548548549	1.14877826041574\\
1.49349349349349	1.15275405739868\\
1.5015015015015	1.1566647568674\\
1.50950950950951	1.16050891845749\\
1.51751751751752	1.16428510404841\\
1.52552552552553	1.16799187852985\\
1.53353353353353	1.17162781056818\\
1.54154154154154	1.17519147337242\\
1.54954954954955	1.17868144545933\\
1.55755755755756	1.18209631141727\\
1.56556556556557	1.18543466266823\\
1.57357357357357	1.18869509822765\\
1.58158158158158	1.19187622546158\\
1.58958958958959	1.19497666084077\\
1.5975975975976	1.19799503069113\\
1.60560560560561	1.20092997194034\\
1.61361361361361	1.20378013285991\\
1.62162162162162	1.20654417380251\\
1.62962962962963	1.20922076793393\\
1.63763763763764	1.21180860195948\\
1.64564564564565	1.21430637684414\\
1.65365365365365	1.21671280852635\\
1.66166166166166	1.21902662862475\\
1.66966966966967	1.22124658513772\\
1.67767767767768	1.2233714431351\\
1.68568568568569	1.22539998544191\\
1.69369369369369	1.22733101331353\\
1.7017017017017	1.22916334710205\\
1.70970970970971	1.23089582691339\\
1.71771771771772	1.23252731325483\\
1.72572572572573	1.23405668767264\\
1.73373373373373	1.23548285337935\\
1.74174174174174	1.23680473587047\\
1.74974974974975	1.23802128353022\\
1.75775775775776	1.23913146822596\\
1.76576576576577	1.24013428589103\\
1.77377377377377	1.24102875709571\\
1.78178178178178	1.24181392760594\\
1.78978978978979	1.24248886892957\\
1.7977977977978	1.24305267884978\\
1.80580580580581	1.24350448194548\\
1.81381381381381	1.24384343009841\\
1.82182182182182	1.24406870298661\\
1.82982982982983	1.24417950856406\\
1.83783783783784	1.24417508352634\\
1.84584584584585	1.24405469376193\\
1.85385385385385	1.24381763478897\\
1.86186186186186	1.2434632321774\\
1.86986986986987	1.24299084195609\\
1.87787787787788	1.24239985100488\\
1.88588588588589	1.24168967743136\\
1.89389389389389	1.24085977093221\\
1.9019019019019	1.23990961313878\\
1.90990990990991	1.23883871794708\\
1.91791791791792	1.23764663183176\\
1.92592592592593	1.23633293414402\\
1.93393393393393	1.23489723739342\\
1.94194194194194	1.23333918751341\\
1.94994994994995	1.23165846411044\\
1.95795795795796	1.22985478069662\\
1.96596596596597	1.22792788490588\\
1.97397397397397	1.22587755869344\\
1.98198198198198	1.22370361851867\\
1.98998998998999	1.2214059155112\\
1.997997997998	1.21898433562025\\
2.00600600600601	1.21643879974716\\
2.01401401401401	1.21376926386116\\
2.02202202202202	1.21097571909828\\
2.03003003003003	1.20805819184338\\
2.03803803803804	1.20501674379547\\
2.04604604604605	1.20185147201618\\
2.05405405405405	1.19856250896141\\
2.06206206206206	1.19515002249639\\
2.07007007007007	1.19161421589391\\
2.07807807807808	1.18795532781603\\
2.08608608608609	1.18417363227918\\
2.09409409409409	1.18026943860284\\
2.1021021021021	1.17624309134173\\
2.11011011011011	1.17209497020183\\
2.11811811811812	1.16782548994005\\
2.12612612612613	1.16343510024799\\
2.13413413413413	1.15892428561957\\
2.14214214214214	1.15429356520295\\
2.15015015015015	1.14954349263671\\
2.15815815815816	1.14467465587041\\
2.16616616616617	1.13968767696993\\
2.17417417417417	1.13458321190734\\
2.18218218218218	1.12936195033594\\
2.19019019019019	1.12402461535023\\
2.1981981981982	1.11857196323126\\
2.20620620620621	1.11300478317748\\
2.21421421421421	1.10732389702125\\
2.22222222222222	1.10153015893121\\
2.23023023023023	1.09562445510086\\
2.23823823823824	1.08960770342332\\
2.24624624624625	1.0834808531528\\
2.25425425425425	1.07724488455275\\
2.26226226226226	1.07090080853103\\
2.27027027027027	1.06444966626242\\
2.27827827827828	1.05789252879859\\
2.28628628628629	1.05123049666584\\
2.29429429429429	1.04446469945091\\
2.3023023023023	1.03759629537506\\
2.31031031031031	1.03062647085672\\
2.31831831831832	1.02355644006296\\
2.32632632632633	1.01638744445018\\
2.33433433433433	1.00912075229407\\
2.34234234234234	1.00175765820935\\
2.35035035035035	0.99429948265952\\
2.35835835835836	0.986747571456779\\
2.36636636636637	0.979103295252667\\
2.37437437437437	0.97136804901948\\
2.38238238238238	0.963543251522937\\
2.39039039039039	0.955630344786306\\
2.3983983983984	0.947630793546367\\
2.40640640640641	0.939546084701479\\
2.41441441441441	0.931377726752089\\
2.42242242242242	0.92312724923399\\
2.43043043043043	0.914796202144652\\
2.43843843843844	0.906386155362951\\
2.44644644644645	0.897898698062617\\
2.45445445445445	0.889335438119703\\
2.46246246246246	0.880698001514426\\
2.47047047047047	0.871988031727716\\
2.47847847847848	0.863207189132728\\
2.48648648648649	0.854357150381756\\
2.49449449449449	0.845439607788762\\
2.5025025025025	0.836456268707919\\
2.51051051051051	0.827408854908459\\
2.51851851851852	0.818299101946186\\
2.52652652652653	0.809128758531921\\
2.53453453453453	0.799899585897277\\
2.54254254254254	0.790613357158019\\
2.55055055055055	0.781271856675366\\
2.55855855855856	0.771876879415561\\
2.56656656656657	0.762430230308005\\
2.57457457457457	0.752933723602292\\
2.58258258258258	0.743389182224441\\
2.59059059059059	0.733798437132667\\
2.5985985985986	0.724163326672962\\
2.60660660660661	0.71448569593484\\
2.61461461461461	0.704767396107525\\
2.62262262262262	0.695010283836875\\
2.63063063063063	0.685216220583373\\
2.63863863863864	0.675387071981472\\
2.64664664664665	0.665524707200576\\
2.65465465465465	0.655630998307955\\
2.66266266266266	0.645707819633898\\
2.67067067067067	0.63575704713936\\
2.67867867867868	0.625780557786404\\
2.68668668668669	0.615780228911715\\
2.69469469469469	0.605757937603454\\
2.7027027027027	0.5957155600817\\
2.71071071071071	0.585654971082795\\
2.71871871871872	0.575578043247795\\
2.72672672672673	0.56548664651534\\
2.73473473473473	0.555382647519138\\
2.74274274274274	0.545267908990346\\
2.75075075075075	0.535144289165088\\
2.75875875875876	0.525013641197333\\
2.76676676676677	0.514877812577381\\
2.77477477477477	0.50473864455616\\
2.78278278278278	0.494597971575597\\
2.79079079079079	0.484457620705231\\
2.7987987987988	0.474319411085325\\
2.80680680680681	0.464185153376658\\
2.81481481481481	0.454056649217204\\
2.82282282282282	0.443935690685882\\
2.83083083083083	0.433824059773612\\
2.83883883883884	0.42372352786179\\
2.84684684684685	0.413635855208443\\
2.85485485485485	0.403562790442165\\
2.86286286286286	0.393506070064042\\
2.87087087087087	0.383467417957731\\
2.87887887887888	0.373448544907832\\
2.88688688688689	0.36345114812671\\
2.89489489489489	0.353476910789903\\
2.9029029029029	0.3435275015803\\
2.91091091091091	0.333604574241145\\
2.91891891891892	0.323709767138072\\
2.92692692692693	0.313844702830244\\
2.93493493493493	0.304010987650729\\
2.94294294294294	0.294210211296223\\
2.95095095095095	0.284443946426226\\
2.95895895895896	0.274713748271757\\
2.96696696696697	0.26502115425373\\
2.97497497497497	0.255367683611034\\
2.98298298298298	0.245754837038438\\
2.99099099099099	0.236184096334381\\
2.998998998999	0.226656924058706\\
3.00700700700701	0.217174763200421\\
3.01501501501502	0.207739036855533\\
3.02302302302302	0.198351147915016\\
3.03103103103103	0.189012478762938\\
3.03903903903904	0.179724390984825\\
3.04704704704705	0.170488225086266\\
3.05505505505506	0.161305300221801\\
3.06306306306306	0.15217691393411\\
3.07107107107107	0.143104341903547\\
3.07907907907908	0.134088837707986\\
3.08708708708709	0.125131632593039\\
3.0950950950951	0.116233935252614\\
3.1031031031031	0.107396931619818\\
3.11111111111111	0.0986217846682116\\
3.11911911911912	0.0899096342233877\\
3.12712712712713	0.0812615967848666\\
3.13513513513514	0.072678765358271\\
3.14314314314314	0.0641622092977793\\
3.15115115115115	0.055712974158794\\
3.15915915915916	0.0473320815608194\\
3.16716716716717	0.0390205290604839\\
3.17517517517518	0.0307792900346726\\
3.18318318318318	0.0226093135737124\\
3.19119119119119	0.0145115243845683\\
3.1991991991992	0.00648682270397377\\
3.20720720720721	-0.00146391577854599\\
3.21521521521522	-0.00933983998785093\\
3.22322322322322	-0.0171401235206203\\
3.23123123123123	-0.0248639646931089\\
3.23923923923924	-0.032510586579328\\
3.24724724724725	-0.0400792370397202\\
3.25525525525526	-0.0475691887404132\\
3.26326326326326	-0.0549797391631543\\
3.27127127127127	-0.062310210606006\\
3.27927927927928	-0.0695599501749072\\
3.28728728728729	-0.0767283297662012\\
3.2952952952953	-0.083814746040237\\
3.3033033033033	-0.090818620386136\\
3.31131131131131	-0.0977393988778549\\
3.31931931931932	-0.10457655222164\\
3.32732732732733	-0.111329575695004\\
3.33533533533534	-0.117997989077327\\
3.34334334334334	-0.124581336572218\\
3.35135135135135	-0.131079186721752\\
3.35935935935936	-0.137491132312709\\
3.36736736736737	-0.143816790274945\\
3.37537537537538	-0.150055801572035\\
3.38338338338338	-0.156207831084293\\
3.39139139139139	-0.162272567484339\\
3.3993993993994	-0.16824972310532\\
3.40740740740741	-0.174139033801945\\
3.41541541541542	-0.179940258804464\\
3.42342342342342	-0.185653180565727\\
3.43143143143143	-0.191277604601479\\
3.43943943943944	-0.196813359324031\\
3.44744744744745	-0.202260295869447\\
3.45545545545546	-0.207618287918408\\
3.46346346346346	-0.212887231510884\\
3.47147147147147	-0.218067044854778\\
3.47947947947948	-0.223157668128684\\
3.48748748748749	-0.228159063278921\\
3.4954954954955	-0.23307121381098\\
3.5035035035035	-0.237894124575542\\
3.51151151151151	-0.24262782154924\\
3.51951951951952	-0.247272351610277\\
3.52752752752753	-0.251827782309103\\
3.53553553553554	-0.25629420163427\\
3.54354354354354	-0.260671717773631\\
3.55155155155155	-0.264960458871045\\
3.55955955955956	-0.269160572778732\\
3.56756756756757	-0.273272226805449\\
3.57557557557558	-0.277295607460614\\
3.58358358358358	-0.281230920194565\\
3.59159159159159	-0.285078389135093\\
3.5995995995996	-0.288838256820392\\
3.60760760760761	-0.292510783928607\\
3.61561561561562	-0.296096249004117\\
3.62362362362362	-0.299594948180693\\
3.63163163163163	-0.303007194901719\\
3.63963963963964	-0.306333319637587\\
3.64764764764765	-0.309573669600453\\
3.65565565565566	-0.312728608456476\\
3.66366366366366	-0.315798516035705\\
3.67167167167167	-0.318783788039753\\
3.67967967967968	-0.321684835747417\\
3.68768768768769	-0.32450208571838\\
3.6956956956957	-0.327235979495136\\
3.7037037037037	-0.329886973303296\\
3.71171171171171	-0.332455537750411\\
3.71971971971972	-0.334942157523436\\
3.72772772772773	-0.337347331085018\\
3.73573573573574	-0.339671570368691\\
3.74374374374374	-0.341915400473158\\
3.75175175175175	-0.34407935935578\\
3.75975975975976	-0.346163997525398\\
3.76776776776777	-0.348169877734636\\
3.77577577577578	-0.350097574671804\\
3.78378378378378	-0.351947674652533\\
3.79179179179179	-0.353720775311271\\
3.7997997997998	-0.355417485292763\\
3.80780780780781	-0.357038423943645\\
3.81581581581582	-0.358584221004256\\
3.82382382382382	-0.360055516300815\\
3.83183183183183	-0.361452959438056\\
3.83983983983984	-0.362777209492451\\
3.84784784784785	-0.36402893470613\\
3.85585585585586	-0.365208812181615\\
3.86386386386386	-0.36631752757747\\
3.87187187187187	-0.367355774804981\\
3.87987987987988	-0.368324255725972\\
3.88788788788789	-0.369223679851858\\
3.8958958958959	-0.370054764044039\\
3.9039039039039	-0.370818232215735\\
3.91191191191191	-0.371514815035351\\
3.91991991991992	-0.372145249631488\\
3.92792792792793	-0.372710279299671\\
3.93593593593594	-0.373210653210893\\
3.94394394394394	-0.373647126122072\\
3.95195195195195	-0.374020458088495\\
3.95995995995996	-0.374331414178341\\
3.96796796796797	-0.374580764189367\\
3.97597597597598	-0.374769282367823\\
3.98398398398398	-0.374897747129701\\
3.99199199199199	-0.374966940784358\\
4	-0.374977649260626\\
};
\addlegendentry{$\mu_{\vec{y}_\ast\given\data}$};

\addplot [color=mycolor3,solid]
  table[row sep=crcr]{%
-4	0.756802495307928\\
-3.99199199199199	0.751543901844466\\
-3.98398398398398	0.746237113486732\\
-3.97597597597598	0.740882470547652\\
-3.96796796796797	0.735480316408962\\
-3.95995995995996	0.730030997499192\\
-3.95195195195195	0.724534863271444\\
-3.94394394394394	0.718992266180986\\
-3.93593593593594	0.713403561662651\\
-3.92792792792793	0.707769108108042\\
-3.91991991991992	0.702089266842549\\
-3.91191191191191	0.696364402102178\\
-3.9039039039039	0.690594881010193\\
-3.8958958958959	0.684781073553575\\
-3.88788788788789	0.678923352559294\\
-3.87987987987988	0.673022093670401\\
-3.87187187187187	0.667077675321938\\
-3.86386386386386	0.66109047871667\\
-3.85585585585586	0.655060887800641\\
-3.84784784784785	0.64898928923855\\
-3.83983983983984	0.642876072388956\\
-3.83183183183183	0.636721629279309\\
-3.82382382382382	0.630526354580811\\
-3.81581581581582	0.624290645583107\\
-3.80780780780781	0.618014902168803\\
-3.7997997997998	0.611699526787831\\
-3.79179179179179	0.605344924431632\\
-3.78378378378378	0.59895150260719\\
-3.77577577577578	0.592519671310899\\
-3.76776776776777	0.586049843002266\\
-3.75975975975976	0.579542432577471\\
-3.75175175175175	0.572997857342748\\
-3.74374374374374	0.566416536987635\\
-3.73573573573574	0.559798893558052\\
-3.72772772772773	0.553145351429241\\
-3.71971971971972	0.546456337278553\\
-3.71171171171171	0.539732280058078\\
-3.7037037037037	0.532973610967149\\
-3.6956956956957	0.526180763424678\\
-3.68768768768769	0.519354173041373\\
-3.67967967967968	0.512494277591792\\
-3.67167167167167	0.50560151698628\\
-3.66366366366366	0.498676333242752\\
-3.65565565565566	0.49171917045835\\
-3.64764764764765	0.484730474780961\\
-3.63963963963964	0.47771069438061\\
-3.63163163163163	0.470660279420719\\
-3.62362362362362	0.463579682029239\\
-3.61561561561562	0.456469356269652\\
-3.60760760760761	0.44932975811186\\
-3.5995995995996	0.442161345402939\\
-3.59159159159159	0.434964577837781\\
-3.58358358358358	0.427739916929614\\
-3.57557557557558	0.420487825980406\\
-3.56756756756757	0.413208770051153\\
-3.55955955955956	0.40590321593206\\
-3.55155155155155	0.398571632112601\\
-3.54354354354354	0.391214488751482\\
-3.53553553553554	0.383832257646483\\
-3.52752752752753	0.376425412204213\\
-3.51951951951952	0.36899442740974\\
-3.51151151151151	0.361539779796139\\
-3.5035035035035	0.354061947413931\\
-3.4954954954955	0.346561409800427\\
-3.48748748748749	0.339038647948973\\
-3.47947947947948	0.331494144278111\\
-3.47147147147147	0.323928382600635\\
-3.46346346346346	0.316341848092574\\
-3.45545545545546	0.30873502726207\\
-3.44744744744745	0.301108407918185\\
-3.43943943943944	0.293462479139619\\
-3.43143143143143	0.285797731243339\\
-3.42342342342342	0.278114655753147\\
-3.41541541541542	0.270413745368152\\
-3.40740740740741	0.262695493931177\\
-3.3993993993994	0.254960396397088\\
-3.39139139139139	0.247208948801056\\
-3.38338338338338	0.239441648226747\\
-3.37537537537538	0.231658992774443\\
-3.36736736736737	0.223861481529103\\
-3.35935935935936	0.216049614528354\\
-3.35135135135135	0.208223892730428\\
-3.34334334334334	0.200384817982036\\
-3.33533533533534	0.192532892986182\\
-3.32732732732733	0.184668621269933\\
-3.31931931931932	0.176792507152122\\
-3.31131131131131	0.168905055711009\\
-3.3033033033033	0.161006772751895\\
-3.2952952952953	0.15309816477468\\
-3.28728728728729	0.145179738941388\\
-3.27927927927928	0.137252003043638\\
-3.27127127127127	0.129315465470086\\
-3.26326326326326	0.121370635173819\\
-3.25525525525526	0.113418021639719\\
-3.24724724724725	0.105458134851792\\
-3.23923923923924	0.0974914852604574\\
-3.23123123123123	0.0895185837498232\\
-3.22322322322322	0.0815399416049179\\
-3.21521521521522	0.0735560704789051\\
-3.20720720720721	0.0655674823602708\\
-3.1991991991992	0.0575746895399915\\
-3.19119119119119	0.0495782045786837\\
-3.18318318318318	0.041578540273731\\
-3.17517517517518	0.0335762096264033\\
-3.16716716716717	0.0255717258089561\\
-3.15915915915916	0.017565602131723\\
-3.15115115115115	0.00955835201019938\\
-3.14314314314314	0.00155048893211566\\
-3.13513513513514	-0.0064574735754886\\
-3.12712712712713	-0.0144650219791989\\
-3.11911911911912	-0.0224716427721563\\
-3.11111111111111	-0.0304768225069864\\
-3.1031031031031	-0.0384800478287283\\
-3.0950950950951	-0.0464808055077517\\
-3.08708708708709	-0.0544785824726723\\
-3.07907907907908	-0.0624728658432525\\
-3.07107107107107	-0.0704631429632903\\
-3.06306306306306	-0.0784489014334974\\
-3.05505505505506	-0.0864296291443551\\
-3.04704704704705	-0.094404814308958\\
-3.03903903903904	-0.102373945495832\\
-3.03103103103103	-0.110336511661731\\
-3.02302302302302	-0.118292002184409\\
-3.01501501501502	-0.126239906895368\\
-3.00700700700701	-0.134179716112569\\
-2.998998998999	-0.142110920673123\\
-2.99099099099099	-0.150033011965935\\
-2.98298298298298	-0.157945481964328\\
-2.97497497497497	-0.165847823258615\\
-2.96696696696697	-0.173739529088644\\
-2.95895895895896	-0.181620093376289\\
-2.95095095095095	-0.18948901075791\\
-2.94294294294294	-0.197345776616757\\
-2.93493493493493	-0.205189887115331\\
-2.92692692692693	-0.213020839227693\\
-2.91891891891892	-0.220838130771726\\
-2.91091091091091	-0.228641260441333\\
-2.9029029029029	-0.236429727838587\\
-2.89489489489489	-0.244203033505825\\
-2.88688688688689	-0.251960678957669\\
-2.87887887887888	-0.259702166712999\\
-2.87087087087087	-0.267427000326854\\
-2.86286286286286	-0.275134684422264\\
-2.85485485485485	-0.282824724722024\\
-2.84684684684685	-0.290496628080386\\
-2.83883883883884	-0.298149902514687\\
-2.83083083083083	-0.305784057236895\\
-2.82282282282282	-0.313398602685084\\
-2.81481481481481	-0.320993050554829\\
-2.80680680680681	-0.32856691383052\\
-2.7987987987988	-0.336119706816592\\
-2.79079079079079	-0.343650945168674\\
-2.78278278278278	-0.351160145924643\\
-2.77477477477477	-0.358646827535605\\
-2.76676676676677	-0.366110509896768\\
-2.75875875875876	-0.373550714378232\\
-2.75075075075075	-0.380966963855683\\
-2.74274274274274	-0.38835878274099\\
-2.73473473473473	-0.395725697012704\\
-2.72672672672673	-0.403067234246455\\
-2.71871871871872	-0.410382923645248\\
-2.71071071071071	-0.417672296069653\\
-2.7027027027027	-0.424934884067893\\
-2.69469469469469	-0.432170221905816\\
-2.68668668668669	-0.439377845596767\\
-2.67867867867868	-0.446557292931338\\
-2.67067067067067	-0.453708103507011\\
-2.66266266266266	-0.460829818757679\\
-2.65465465465465	-0.46792198198306\\
-2.64664664664665	-0.474984138377976\\
-2.63863863863864	-0.482015835061526\\
-2.63063063063063	-0.489016621106122\\
-2.62262262262262	-0.49598604756641\\
-2.61461461461461	-0.502923667508058\\
-2.60660660660661	-0.509829036036418\\
-2.5985985985986	-0.516701710325057\\
-2.59059059059059	-0.523541249644151\\
-2.58258258258258	-0.530347215388752\\
-2.57457457457457	-0.537119171106913\\
-2.56656656656657	-0.543856682527676\\
-2.55855855855856	-0.550559317588923\\
-2.55055055055055	-0.55722664646508\\
-2.54254254254254	-0.563858241594684\\
-2.53453453453453	-0.570453677707801\\
-2.52652652652653	-0.577012531853296\\
-2.51851851851852	-0.583534383425957\\
-2.51051051051051	-0.590018814193468\\
-2.5025025025025	-0.596465408323227\\
-2.49449449449449	-0.602873752409017\\
-2.48648648648649	-0.609243435497511\\
-2.47847847847848	-0.615574049114632\\
-2.47047047047047	-0.621865187291741\\
-2.46246246246246	-0.628116446591676\\
-2.45445445445445	-0.634327426134621\\
-2.44644644644645	-0.640497727623814\\
-2.43843843843844	-0.64662695537109\\
-2.43043043043043	-0.652714716322253\\
-2.42242242242242	-0.658760620082286\\
-2.41441441441441	-0.664764278940382\\
-2.40640640640641	-0.670725307894808\\
-2.3983983983984	-0.676643324677597\\
-2.39039039039039	-0.682517949779058\\
-2.38238238238238	-0.688348806472117\\
-2.37437437437437	-0.694135520836474\\
-2.36636636636637	-0.69987772178258\\
-2.35835835835836	-0.705575041075436\\
-2.35035035035035	-0.711227113358209\\
-2.34234234234234	-0.716833576175656\\
-2.33433433433433	-0.722394069997375\\
-2.32632632632633	-0.727908238240854\\
-2.31831831831832	-0.733375727294341\\
-2.31031031031031	-0.73879618653952\\
-2.3023023023023	-0.744169268373998\\
-2.29429429429429	-0.749494628233589\\
-2.28628628628629	-0.75477192461442\\
-2.27827827827828	-0.76000081909482\\
-2.27027027027027	-0.765180976357031\\
-2.26226226226226	-0.770312064208709\\
-2.25425425425425	-0.775393753604222\\
-2.24624624624625	-0.780425718665759\\
-2.23823823823824	-0.78540763670422\\
-2.23023023023023	-0.790339188239915\\
-2.22222222222222	-0.795220057023049\\
-2.21421421421421	-0.800049930054003\\
-2.20620620620621	-0.804828497603407\\
-2.1981981981982	-0.809555453231999\\
-2.19019019019019	-0.814230493810279\\
-2.18218218218218	-0.818853319537949\\
-2.17417417417417	-0.823423633963136\\
-2.16616616616617	-0.827941144001405\\
-2.15815815815816	-0.832405559954551\\
-2.15015015015015	-0.836816595529179\\
-2.14214214214214	-0.841173967855064\\
-2.13413413413413	-0.845477397503289\\
-2.12612612612613	-0.849726608504164\\
-2.11811811811812	-0.853921328364927\\
-2.11011011011011	-0.858061288087212\\
-2.1021021021021	-0.862146222184306\\
-2.09409409409409	-0.86617586869817\\
-2.08608608608609	-0.870149969216237\\
-2.07807807807808	-0.874068268887988\\
-2.07007007007007	-0.877930516441291\\
-2.06206206206206	-0.881736464198517\\
-2.05405405405405	-0.885485868092419\\
-2.04604604604605	-0.889178487681788\\
-2.03803803803804	-0.892814086166873\\
-2.03003003003003	-0.896392430404559\\
-2.02202202202202	-0.899913290923325\\
-2.01401401401401	-0.903376441937959\\
-2.00600600600601	-0.906781661364033\\
-1.997997997998	-0.910128730832148\\
-1.98998998998999	-0.913417435701936\\
-1.98198198198198	-0.916647565075827\\
-1.97397397397397	-0.919818911812569\\
-1.96596596596597	-0.922931272540517\\
-1.95795795795796	-0.92598444767067\\
-1.94994994994995	-0.928978241409473\\
-1.94194194194194	-0.93191246177137\\
-1.93393393393393	-0.934786920591121\\
-1.92592592592593	-0.937601433535862\\
-1.91791791791792	-0.940355820116931\\
-1.90990990990991	-0.943049903701439\\
-1.9019019019019	-0.9456835115236\\
-1.89389389389389	-0.948256474695807\\
-1.88588588588589	-0.950768628219465\\
-1.87787787787788	-0.953219810995572\\
-1.86986986986987	-0.955609865835046\\
-1.86186186186186	-0.95793863946881\\
-1.85385385385385	-0.96020598255762\\
-1.84584584584585	-0.962411749701638\\
-1.83783783783784	-0.964555799449762\\
-1.82982982982983	-0.966637994308691\\
-1.82182182182182	-0.968658200751747\\
-1.81381381381381	-0.970616289227435\\
-1.80580580580581	-0.972512134167752\\
-1.7977977977978	-0.974345613996239\\
-1.78978978978979	-0.976116611135777\\
-1.78178178178178	-0.977825012016127\\
-1.77377377377377	-0.979470707081216\\
-1.76576576576577	-0.981053590796158\\
-1.75775775775776	-0.982573561654024\\
-1.74974974974975	-0.984030522182352\\
-1.74174174174174	-0.985424378949395\\
-1.73373373373373	-0.986755042570117\\
-1.72572572572573	-0.988022427711922\\
-1.71771771771772	-0.989226453100126\\
-1.70970970970971	-0.990367041523169\\
-1.7017017017017	-0.99144411983757\\
-1.69369369369369	-0.992457618972611\\
-1.68568568568569	-0.993407473934773\\
-1.67767767767768	-0.9942936238119\\
-1.66966966966967	-0.995116011777106\\
-1.66166166166166	-0.995874585092419\\
-1.65365365365365	-0.996569295112163\\
-1.64564564564565	-0.997200097286079\\
-1.63763763763764	-0.99776695116218\\
-1.62962962962963	-0.998269820389346\\
-1.62162162162162	-0.998708672719655\\
-1.61361361361361	-0.999083480010451\\
-1.60560560560561	-0.999394218226148\\
-1.5975975975976	-0.999640867439772\\
-1.58958958958959	-0.99982341183424\\
-1.58158158158158	-0.999941839703372\\
-1.57357357357357	-0.999996143452644\\
-1.56556556556557	-0.999986319599673\\
-1.55755755755756	-0.999912368774442\\
-1.54954954954955	-0.999774295719258\\
-1.54154154154154	-0.999572109288449\\
-1.53353353353353	-0.999305822447797\\
-1.52552552552553	-0.998975452273703\\
-1.51751751751752	-0.998581019952096\\
-1.50950950950951	-0.998122550777074\\
-1.5015015015015	-0.997600074149278\\
-1.49349349349349	-0.997013623574011\\
-1.48548548548549	-0.996363236659087\\
-1.47747747747748	-0.99564895511242\\
-1.46946946946947	-0.994870824739351\\
-1.46146146146146	-0.994028895439706\\
-1.45345345345345	-0.993123221204601\\
-1.44544544544545	-0.992153860112977\\
-1.43743743743744	-0.991120874327876\\
-1.42942942942943	-0.990024330092456\\
-1.42142142142142	-0.988864297725739\\
-1.41341341341341	-0.987640851618108\\
-1.40540540540541	-0.98635407022653\\
-1.3973973973974	-0.985004036069529\\
-1.38938938938939	-0.983590835721891\\
-1.38138138138138	-0.982114559809117\\
-1.37337337337337	-0.980575303001605\\
-1.36536536536537	-0.978973164008585\\
-1.35735735735736	-0.977308245571786\\
-1.34934934934935	-0.975580654458845\\
-1.34134134134134	-0.973790501456467\\
-1.33333333333333	-0.971937901363313\\
-1.32532532532533	-0.970022972982644\\
-1.31731731731732	-0.968045839114698\\
-1.30930930930931	-0.966006626548819\\
-1.3013013013013	-0.963905466055324\\
-1.29329329329329	-0.961742492377117\\
-1.28528528528529	-0.959517844221048\\
-1.27727727727728	-0.957231664249019\\
-1.26926926926927	-0.954884099068836\\
-1.26126126126126	-0.952475299224806\\
-1.25325325325325	-0.950005419188082\\
-1.24524524524525	-0.94747461734676\\
-1.23723723723724	-0.944883055995719\\
-1.22922922922923	-0.942230901326217\\
-1.22122122122122	-0.939518323415228\\
-1.21321321321321	-0.936745496214542\\
-1.20520520520521	-0.933912597539602\\
-1.1971971971972	-0.931019809058111\\
-1.18918918918919	-0.928067316278371\\
-1.18118118118118	-0.925055308537396\\
-1.17317317317317	-0.921983978988765\\
-1.16516516516517	-0.918853524590237\\
-1.15715715715716	-0.915664146091122\\
-1.14914914914915	-0.912416048019403\\
-1.14114114114114	-0.909109438668625\\
-1.13313313313313	-0.905744530084535\\
-1.12512512512513	-0.902321538051485\\
-1.11711711711712	-0.898840682078593\\
-1.10910910910911	-0.895302185385666\\
-1.1011011011011	-0.891706274888888\\
-1.09309309309309	-0.888053181186267\\
-1.08508508508509	-0.884343138542847\\
-1.07707707707708	-0.880576384875683\\
-1.06906906906907	-0.876753161738587\\
-1.06106106106106	-0.87287371430664\\
-1.05305305305305	-0.86893829136046\\
-1.04504504504505	-0.86494714527026\\
-1.03703703703704	-0.860900531979654\\
-1.02902902902903	-0.856798710989252\\
-1.02102102102102	-0.852641945340013\\
-1.01301301301301	-0.848430501596379\\
-1.00500500500501	-0.844164649829181\\
-0.996996996996997	-0.839844663598319\\
-0.988988988988989	-0.83547081993522\\
-0.980980980980981	-0.831043399325073\\
-0.972972972972973	-0.82656268568884\\
-0.964964964964965	-0.822028966365051\\
-0.956956956956957	-0.817442532091378\\
-0.948948948948949	-0.812803676985987\\
-0.940940940940941	-0.80811269852868\\
-0.932932932932933	-0.803369897541819\\
-0.924924924924925	-0.798575578171031\\
-0.916916916916917	-0.793730047865708\\
-0.908908908908909	-0.788833617359288\\
-0.900900900900901	-0.78388660064933\\
-0.892892892892893	-0.778889314977378\\
-0.884884884884885	-0.773842080808614\\
-0.876876876876877	-0.768745221811314\\
-0.868868868868869	-0.763599064836083\\
-0.860860860860861	-0.758403939894902\\
-0.852852852852853	-0.753160180139961\\
-0.844844844844845	-0.747868121842297\\
-0.836836836836837	-0.742528104370229\\
-0.828828828828829	-0.737140470167594\\
-0.820820820820821	-0.731705564731787\\
-0.812812812812813	-0.726223736591606\\
-0.804804804804805	-0.720695337284902\\
-0.796796796796797	-0.715120721336034\\
-0.788788788788789	-0.709500246233134\\
-0.780780780780781	-0.703834272405184\\
-0.772772772772773	-0.698123163198902\\
-0.764764764764765	-0.69236728485544\\
-0.756756756756757	-0.686567006486897\\
-0.748748748748749	-0.680722700052653\\
-0.740740740740741	-0.674834740335512\\
-0.732732732732733	-0.668903504917667\\
-0.724724724724725	-0.662929374156493\\
-0.716716716716717	-0.656912731160147\\
-0.708708708708709	-0.650853961763007\\
-0.700700700700701	-0.644753454500924\\
-0.692692692692693	-0.638611600586311\\
-0.684684684684685	-0.632428793883052\\
-0.676676676676677	-0.626205430881244\\
-0.668668668668669	-0.619941910671774\\
-0.660660660660661	-0.613638634920724\\
-0.652652652652653	-0.607296007843613\\
-0.644644644644645	-0.600914436179475\\
-0.636636636636636	-0.594494329164778\\
-0.628628628628629	-0.58803609850718\\
-0.62062062062062	-0.581540158359123\\
-0.612612612612613	-0.575006925291282\\
-0.604604604604605	-0.568436818265842\\
-0.596596596596596	-0.561830258609638\\
-0.588588588588589	-0.555187669987136\\
-0.58058058058058	-0.548509478373257\\
-0.572572572572573	-0.54179611202607\\
-0.564564564564565	-0.535048001459322\\
-0.556556556556556	-0.528265579414831\\
-0.548548548548549	-0.521449280834738\\
-0.54054054054054	-0.514599542833613\\
-0.532532532532533	-0.507716804670425\\
-0.524524524524525	-0.50080150772037\\
-0.516516516516516	-0.493854095446571\\
-0.508508508508509	-0.486875013371638\\
-0.5005005005005	-0.479864709049094\\
-0.492492492492492	-0.472823632034682\\
-0.484484484484485	-0.465752233857529\\
-0.476476476476476	-0.458650967991193\\
-0.468468468468469	-0.451520289824584\\
-0.46046046046046	-0.444360656632758\\
-0.452452452452452	-0.437172527547595\\
-0.444444444444445	-0.429956363528356\\
-0.436436436436436	-0.422712627332121\\
-0.428428428428429	-0.415441783484116\\
-0.42042042042042	-0.40814429824792\\
-0.412412412412412	-0.40082063959557\\
-0.404404404404405	-0.393471277177546\\
-0.396396396396396	-0.386096682292654\\
-0.388388388388389	-0.378697327857807\\
-0.38038038038038	-0.371273688377692\\
-0.372372372372372	-0.363826239914345\\
-0.364364364364364	-0.35635546005662\\
-0.356356356356356	-0.348861827889565\\
-0.348348348348348	-0.341345823963696\\
-0.34034034034034	-0.333807930264181\\
-0.332332332332332	-0.326248630179934\\
-0.324324324324324	-0.318668408472613\\
-0.316316316316316	-0.311067751245536\\
-0.308308308308308	-0.303447145912505\\
-0.3003003003003	-0.295807081166554\\
-0.292292292292292	-0.288148046948604\\
-0.284284284284284	-0.28047053441605\\
-0.276276276276276	-0.27277503591126\\
-0.268268268268268	-0.265062044930006\\
-0.26026026026026	-0.25733205608981\\
-0.252252252252252	-0.249585565098237\\
-0.244244244244244	-0.241823068721094\\
-0.236236236236236	-0.234045064750581\\
-0.228228228228228	-0.226252051973366\\
-0.22022022022022	-0.218444530138601\\
-0.212212212212212	-0.210622999925871\\
-0.204204204204204	-0.202787962913089\\
-0.196196196196196	-0.19493992154433\\
-0.188188188188188	-0.18707937909761\\
-0.18018018018018	-0.179206839652612\\
-0.172172172172172	-0.171322808058362\\
-0.164164164164164	-0.163427789900851\\
-0.156156156156156	-0.155522291470617\\
-0.148148148148148	-0.147606819730272\\
-0.14014014014014	-0.139681882281999\\
-0.132132132132132	-0.131747987334993\\
-0.124124124124124	-0.123805643672876\\
-0.116116116116116	-0.115855360621067\\
-0.108108108108108	-0.107897648014123\\
-0.1001001001001	-0.0999330161630392\\
-0.0920920920920922	-0.0919619758225296\\
-0.084084084084084	-0.0839850381582693\\
-0.0760760760760761	-0.0760027147141175\\
-0.0680680680680679	-0.0680155173793103\\
-0.06006006006006	-0.0600239583556376\\
-0.0520520520520522	-0.0520285501245938\\
-0.0440440440440439	-0.0440298054145147\\
-0.0360360360360361	-0.0360282371676984\\
-0.0280280280280278	-0.0280243585075088\\
-0.02002002002002	-0.0200186827054735\\
-0.0120120120120122	-0.0120117231483653\\
-0.00400400400400391	-0.00400399330528172\\
0.00400400400400436	0.00400399330528216\\
0.0120120120120122	0.0120117231483653\\
0.02002002002002	0.0200186827054735\\
0.0280280280280278	0.0280243585075088\\
0.0360360360360357	0.0360282371676979\\
0.0440440440440444	0.0440298054145151\\
0.0520520520520522	0.0520285501245938\\
0.06006006006006	0.0600239583556376\\
0.0680680680680679	0.0680155173793103\\
0.0760760760760757	0.076002714714117\\
0.0840840840840844	0.0839850381582697\\
0.0920920920920922	0.0919619758225296\\
0.1001001001001	0.0999330161630392\\
0.108108108108108	0.107897648014123\\
0.116116116116116	0.115855360621067\\
0.124124124124124	0.123805643672876\\
0.132132132132132	0.131747987334993\\
0.14014014014014	0.139681882281999\\
0.148148148148148	0.147606819730272\\
0.156156156156156	0.155522291470616\\
0.164164164164164	0.163427789900852\\
0.172172172172172	0.171322808058362\\
0.18018018018018	0.179206839652612\\
0.188188188188188	0.18707937909761\\
0.196196196196196	0.19493992154433\\
0.204204204204204	0.20278796291309\\
0.212212212212212	0.210622999925871\\
0.22022022022022	0.218444530138601\\
0.228228228228228	0.226252051973366\\
0.236236236236236	0.23404506475058\\
0.244244244244245	0.241823068721094\\
0.252252252252252	0.249585565098237\\
0.26026026026026	0.25733205608981\\
0.268268268268268	0.265062044930005\\
0.276276276276277	0.272775035911261\\
0.284284284284285	0.28047053441605\\
0.292292292292292	0.288148046948604\\
0.3003003003003	0.295807081166554\\
0.308308308308308	0.303447145912505\\
0.316316316316317	0.311067751245536\\
0.324324324324325	0.318668408472613\\
0.332332332332332	0.326248630179934\\
0.34034034034034	0.333807930264181\\
0.348348348348348	0.341345823963695\\
0.356356356356357	0.348861827889565\\
0.364364364364365	0.35635546005662\\
0.372372372372372	0.363826239914345\\
0.38038038038038	0.371273688377692\\
0.388388388388388	0.378697327857806\\
0.396396396396397	0.386096682292654\\
0.404404404404405	0.393471277177546\\
0.412412412412412	0.40082063959557\\
0.42042042042042	0.40814429824792\\
0.428428428428428	0.415441783484115\\
0.436436436436437	0.422712627332121\\
0.444444444444445	0.429956363528356\\
0.452452452452452	0.437172527547595\\
0.46046046046046	0.444360656632758\\
0.468468468468468	0.451520289824584\\
0.476476476476477	0.458650967991194\\
0.484484484484485	0.465752233857529\\
0.492492492492492	0.472823632034682\\
0.5005005005005	0.479864709049094\\
0.508508508508508	0.486875013371637\\
0.516516516516517	0.493854095446572\\
0.524524524524525	0.50080150772037\\
0.532532532532533	0.507716804670425\\
0.54054054054054	0.514599542833613\\
0.548548548548548	0.521449280834738\\
0.556556556556557	0.528265579414831\\
0.564564564564565	0.535048001459322\\
0.572572572572573	0.54179611202607\\
0.58058058058058	0.548509478373257\\
0.588588588588588	0.555187669987135\\
0.596596596596597	0.561830258609639\\
0.604604604604605	0.568436818265842\\
0.612612612612613	0.575006925291282\\
0.62062062062062	0.581540158359123\\
0.628628628628628	0.58803609850718\\
0.636636636636637	0.594494329164779\\
0.644644644644645	0.600914436179475\\
0.652652652652653	0.607296007843613\\
0.66066066066066	0.613638634920724\\
0.668668668668668	0.619941910671774\\
0.676676676676677	0.626205430881244\\
0.684684684684685	0.632428793883052\\
0.692692692692693	0.638611600586311\\
0.7007007007007	0.644753454500924\\
0.708708708708708	0.650853961763006\\
0.716716716716717	0.656912731160147\\
0.724724724724725	0.662929374156493\\
0.732732732732733	0.668903504917667\\
0.74074074074074	0.674834740335511\\
0.748748748748748	0.680722700052653\\
0.756756756756757	0.686567006486897\\
0.764764764764765	0.69236728485544\\
0.772772772772773	0.698123163198902\\
0.780780780780781	0.703834272405184\\
0.788788788788789	0.709500246233134\\
0.796796796796797	0.715120721336034\\
0.804804804804805	0.720695337284902\\
0.812812812812813	0.726223736591606\\
0.820820820820821	0.731705564731787\\
0.828828828828829	0.737140470167595\\
0.836836836836837	0.74252810437023\\
0.844844844844845	0.747868121842297\\
0.852852852852853	0.753160180139961\\
0.860860860860861	0.758403939894901\\
0.868868868868869	0.763599064836083\\
0.876876876876877	0.768745221811314\\
0.884884884884885	0.773842080808614\\
0.892892892892893	0.778889314977378\\
0.900900900900901	0.78388660064933\\
0.908908908908909	0.788833617359288\\
0.916916916916917	0.793730047865708\\
0.924924924924925	0.798575578171031\\
0.932932932932933	0.803369897541819\\
0.940940940940941	0.80811269852868\\
0.948948948948949	0.812803676985987\\
0.956956956956957	0.817442532091378\\
0.964964964964965	0.822028966365051\\
0.972972972972973	0.82656268568884\\
0.980980980980981	0.831043399325072\\
0.988988988988989	0.83547081993522\\
0.996996996996997	0.839844663598319\\
1.00500500500501	0.844164649829181\\
1.01301301301301	0.848430501596379\\
1.02102102102102	0.852641945340013\\
1.02902902902903	0.856798710989253\\
1.03703703703704	0.860900531979654\\
1.04504504504505	0.86494714527026\\
1.05305305305305	0.86893829136046\\
1.06106106106106	0.872873714306639\\
1.06906906906907	0.876753161738588\\
1.07707707707708	0.880576384875683\\
1.08508508508509	0.884343138542847\\
1.09309309309309	0.888053181186267\\
1.1011011011011	0.891706274888888\\
1.10910910910911	0.895302185385666\\
1.11711711711712	0.898840682078593\\
1.12512512512513	0.902321538051485\\
1.13313313313313	0.905744530084535\\
1.14114114114114	0.909109438668625\\
1.14914914914915	0.912416048019403\\
1.15715715715716	0.915664146091122\\
1.16516516516517	0.918853524590237\\
1.17317317317317	0.921983978988765\\
1.18118118118118	0.925055308537396\\
1.18918918918919	0.928067316278371\\
1.1971971971972	0.931019809058111\\
1.20520520520521	0.933912597539602\\
1.21321321321321	0.936745496214541\\
1.22122122122122	0.939518323415228\\
1.22922922922923	0.942230901326217\\
1.23723723723724	0.944883055995719\\
1.24524524524525	0.94747461734676\\
1.25325325325325	0.950005419188082\\
1.26126126126126	0.952475299224806\\
1.26926926926927	0.954884099068836\\
1.27727727727728	0.957231664249019\\
1.28528528528529	0.959517844221048\\
1.29329329329329	0.961742492377117\\
1.3013013013013	0.963905466055324\\
1.30930930930931	0.966006626548819\\
1.31731731731732	0.968045839114698\\
1.32532532532533	0.970022972982644\\
1.33333333333333	0.971937901363313\\
1.34134134134134	0.973790501456467\\
1.34934934934935	0.975580654458845\\
1.35735735735736	0.977308245571786\\
1.36536536536537	0.978973164008585\\
1.37337337337337	0.980575303001605\\
1.38138138138138	0.982114559809117\\
1.38938938938939	0.983590835721891\\
1.3973973973974	0.985004036069529\\
1.40540540540541	0.98635407022653\\
1.41341341341341	0.987640851618108\\
1.42142142142142	0.988864297725739\\
1.42942942942943	0.990024330092456\\
1.43743743743744	0.991120874327876\\
1.44544544544545	0.992153860112977\\
1.45345345345345	0.993123221204601\\
1.46146146146146	0.994028895439706\\
1.46946946946947	0.994870824739351\\
1.47747747747748	0.99564895511242\\
1.48548548548549	0.996363236659087\\
1.49349349349349	0.997013623574011\\
1.5015015015015	0.997600074149278\\
1.50950950950951	0.998122550777074\\
1.51751751751752	0.998581019952096\\
1.52552552552553	0.998975452273703\\
1.53353353353353	0.999305822447797\\
1.54154154154154	0.999572109288449\\
1.54954954954955	0.999774295719258\\
1.55755755755756	0.999912368774442\\
1.56556556556557	0.999986319599673\\
1.57357357357357	0.999996143452644\\
1.58158158158158	0.999941839703372\\
1.58958958958959	0.99982341183424\\
1.5975975975976	0.999640867439772\\
1.60560560560561	0.999394218226148\\
1.61361361361361	0.999083480010451\\
1.62162162162162	0.998708672719655\\
1.62962962962963	0.998269820389346\\
1.63763763763764	0.99776695116218\\
1.64564564564565	0.997200097286079\\
1.65365365365365	0.996569295112163\\
1.66166166166166	0.995874585092418\\
1.66966966966967	0.995116011777106\\
1.67767767767768	0.9942936238119\\
1.68568568568569	0.993407473934773\\
1.69369369369369	0.992457618972611\\
1.7017017017017	0.99144411983757\\
1.70970970970971	0.990367041523169\\
1.71771771771772	0.989226453100126\\
1.72572572572573	0.988022427711922\\
1.73373373373373	0.986755042570117\\
1.74174174174174	0.985424378949395\\
1.74974974974975	0.984030522182352\\
1.75775775775776	0.982573561654024\\
1.76576576576577	0.981053590796158\\
1.77377377377377	0.979470707081216\\
1.78178178178178	0.977825012016127\\
1.78978978978979	0.976116611135777\\
1.7977977977978	0.974345613996239\\
1.80580580580581	0.972512134167752\\
1.81381381381381	0.970616289227435\\
1.82182182182182	0.968658200751747\\
1.82982982982983	0.966637994308691\\
1.83783783783784	0.964555799449762\\
1.84584584584585	0.962411749701638\\
1.85385385385385	0.96020598255762\\
1.86186186186186	0.95793863946881\\
1.86986986986987	0.955609865835046\\
1.87787787787788	0.953219810995572\\
1.88588588588589	0.950768628219465\\
1.89389389389389	0.948256474695807\\
1.9019019019019	0.9456835115236\\
1.90990990990991	0.943049903701439\\
1.91791791791792	0.940355820116931\\
1.92592592592593	0.937601433535862\\
1.93393393393393	0.934786920591121\\
1.94194194194194	0.93191246177137\\
1.94994994994995	0.928978241409473\\
1.95795795795796	0.92598444767067\\
1.96596596596597	0.922931272540517\\
1.97397397397397	0.919818911812569\\
1.98198198198198	0.916647565075827\\
1.98998998998999	0.913417435701936\\
1.997997997998	0.910128730832148\\
2.00600600600601	0.906781661364033\\
2.01401401401401	0.903376441937959\\
2.02202202202202	0.899913290923325\\
2.03003003003003	0.896392430404559\\
2.03803803803804	0.892814086166873\\
2.04604604604605	0.889178487681789\\
2.05405405405405	0.885485868092418\\
2.06206206206206	0.881736464198517\\
2.07007007007007	0.877930516441291\\
2.07807807807808	0.874068268887988\\
2.08608608608609	0.870149969216237\\
2.09409409409409	0.86617586869817\\
2.1021021021021	0.862146222184306\\
2.11011011011011	0.858061288087212\\
2.11811811811812	0.853921328364927\\
2.12612612612613	0.849726608504164\\
2.13413413413413	0.845477397503289\\
2.14214214214214	0.841173967855064\\
2.15015015015015	0.836816595529179\\
2.15815815815816	0.832405559954551\\
2.16616616616617	0.827941144001405\\
2.17417417417417	0.823423633963136\\
2.18218218218218	0.818853319537949\\
2.19019019019019	0.814230493810279\\
2.1981981981982	0.809555453231999\\
2.20620620620621	0.804828497603407\\
2.21421421421421	0.800049930054003\\
2.22222222222222	0.795220057023049\\
2.23023023023023	0.790339188239915\\
2.23823823823824	0.78540763670422\\
2.24624624624625	0.780425718665759\\
2.25425425425425	0.775393753604222\\
2.26226226226226	0.770312064208709\\
2.27027027027027	0.765180976357031\\
2.27827827827828	0.76000081909482\\
2.28628628628629	0.754771924614419\\
2.29429429429429	0.749494628233589\\
2.3023023023023	0.744169268373998\\
2.31031031031031	0.73879618653952\\
2.31831831831832	0.733375727294341\\
2.32632632632633	0.727908238240853\\
2.33433433433433	0.722394069997375\\
2.34234234234234	0.716833576175656\\
2.35035035035035	0.711227113358209\\
2.35835835835836	0.705575041075436\\
2.36636636636637	0.69987772178258\\
2.37437437437437	0.694135520836474\\
2.38238238238238	0.688348806472117\\
2.39039039039039	0.682517949779058\\
2.3983983983984	0.676643324677597\\
2.40640640640641	0.670725307894808\\
2.41441441441441	0.664764278940382\\
2.42242242242242	0.658760620082286\\
2.43043043043043	0.652714716322253\\
2.43843843843844	0.64662695537109\\
2.44644644644645	0.640497727623814\\
2.45445445445445	0.634327426134621\\
2.46246246246246	0.628116446591676\\
2.47047047047047	0.621865187291741\\
2.47847847847848	0.615574049114632\\
2.48648648648649	0.609243435497511\\
2.49449449449449	0.602873752409017\\
2.5025025025025	0.596465408323227\\
2.51051051051051	0.590018814193468\\
2.51851851851852	0.583534383425958\\
2.52652652652653	0.577012531853296\\
2.53453453453453	0.570453677707801\\
2.54254254254254	0.563858241594684\\
2.55055055055055	0.55722664646508\\
2.55855855855856	0.550559317588923\\
2.56656656656657	0.543856682527676\\
2.57457457457457	0.537119171106913\\
2.58258258258258	0.530347215388752\\
2.59059059059059	0.523541249644151\\
2.5985985985986	0.516701710325057\\
2.60660660660661	0.509829036036418\\
2.61461461461461	0.502923667508058\\
2.62262262262262	0.49598604756641\\
2.63063063063063	0.489016621106122\\
2.63863863863864	0.482015835061527\\
2.64664664664665	0.474984138377976\\
2.65465465465465	0.46792198198306\\
2.66266266266266	0.460829818757679\\
2.67067067067067	0.453708103507011\\
2.67867867867868	0.446557292931339\\
2.68668668668669	0.439377845596767\\
2.69469469469469	0.432170221905816\\
2.7027027027027	0.424934884067893\\
2.71071071071071	0.417672296069653\\
2.71871871871872	0.410382923645248\\
2.72672672672673	0.403067234246455\\
2.73473473473473	0.395725697012704\\
2.74274274274274	0.38835878274099\\
2.75075075075075	0.380966963855683\\
2.75875875875876	0.373550714378231\\
2.76676676676677	0.366110509896768\\
2.77477477477477	0.358646827535605\\
2.78278278278278	0.351160145924643\\
2.79079079079079	0.343650945168674\\
2.7987987987988	0.336119706816592\\
2.80680680680681	0.32856691383052\\
2.81481481481481	0.320993050554829\\
2.82282282282282	0.313398602685084\\
2.83083083083083	0.305784057236895\\
2.83883883883884	0.298149902514687\\
2.84684684684685	0.290496628080386\\
2.85485485485485	0.282824724722024\\
2.86286286286286	0.275134684422264\\
2.87087087087087	0.267427000326854\\
2.87887887887888	0.259702166712999\\
2.88688688688689	0.251960678957669\\
2.89489489489489	0.244203033505825\\
2.9029029029029	0.236429727838587\\
2.91091091091091	0.228641260441333\\
2.91891891891892	0.220838130771726\\
2.92692692692693	0.213020839227693\\
2.93493493493493	0.205189887115331\\
2.94294294294294	0.197345776616757\\
2.95095095095095	0.189489010757911\\
2.95895895895896	0.181620093376289\\
2.96696696696697	0.173739529088644\\
2.97497497497497	0.165847823258615\\
2.98298298298298	0.157945481964328\\
2.99099099099099	0.150033011965936\\
2.998998998999	0.142110920673123\\
3.00700700700701	0.134179716112569\\
3.01501501501502	0.126239906895368\\
3.02302302302302	0.118292002184409\\
3.03103103103103	0.110336511661731\\
3.03903903903904	0.102373945495832\\
3.04704704704705	0.094404814308958\\
3.05505505505506	0.0864296291443551\\
3.06306306306306	0.0784489014334974\\
3.07107107107107	0.0704631429632907\\
3.07907907907908	0.0624728658432521\\
3.08708708708709	0.0544785824726723\\
3.0950950950951	0.0464808055077517\\
3.1031031031031	0.0384800478287283\\
3.11111111111111	0.0304768225069869\\
3.11911911911912	0.0224716427721559\\
3.12712712712713	0.0144650219791989\\
3.13513513513514	0.0064574735754886\\
3.14314314314314	-0.00155048893211566\\
3.15115115115115	-0.00955835201019893\\
3.15915915915916	-0.0175656021317234\\
3.16716716716717	-0.0255717258089561\\
3.17517517517518	-0.0335762096264033\\
3.18318318318318	-0.041578540273731\\
3.19119119119119	-0.0495782045786832\\
3.1991991991992	-0.0575746895399919\\
3.20720720720721	-0.0655674823602708\\
3.21521521521522	-0.0735560704789051\\
3.22322322322322	-0.0815399416049179\\
3.23123123123123	-0.0895185837498228\\
3.23923923923924	-0.0974914852604574\\
3.24724724724725	-0.105458134851792\\
3.25525525525526	-0.113418021639719\\
3.26326326326326	-0.121370635173819\\
3.27127127127127	-0.129315465470086\\
3.27927927927928	-0.137252003043638\\
3.28728728728729	-0.145179738941388\\
3.2952952952953	-0.15309816477468\\
3.3033033033033	-0.161006772751894\\
3.31131131131131	-0.168905055711009\\
3.31931931931932	-0.176792507152122\\
3.32732732732733	-0.184668621269933\\
3.33533533533534	-0.192532892986182\\
3.34334334334334	-0.200384817982035\\
3.35135135135135	-0.208223892730429\\
3.35935935935936	-0.216049614528354\\
3.36736736736737	-0.223861481529103\\
3.37537537537538	-0.231658992774443\\
3.38338338338338	-0.239441648226747\\
3.39139139139139	-0.247208948801057\\
3.3993993993994	-0.254960396397088\\
3.40740740740741	-0.262695493931177\\
3.41541541541542	-0.270413745368152\\
3.42342342342342	-0.278114655753147\\
3.43143143143143	-0.285797731243339\\
3.43943943943944	-0.293462479139619\\
3.44744744744745	-0.301108407918185\\
3.45545545545546	-0.30873502726207\\
3.46346346346346	-0.316341848092573\\
3.47147147147147	-0.323928382600635\\
3.47947947947948	-0.331494144278111\\
3.48748748748749	-0.339038647948973\\
3.4954954954955	-0.346561409800427\\
3.5035035035035	-0.354061947413931\\
3.51151151151151	-0.361539779796139\\
3.51951951951952	-0.36899442740974\\
3.52752752752753	-0.376425412204213\\
3.53553553553554	-0.383832257646483\\
3.54354354354354	-0.391214488751481\\
3.55155155155155	-0.398571632112601\\
3.55955955955956	-0.40590321593206\\
3.56756756756757	-0.413208770051153\\
3.57557557557558	-0.420487825980406\\
3.58358358358358	-0.427739916929614\\
3.59159159159159	-0.434964577837782\\
3.5995995995996	-0.442161345402939\\
3.60760760760761	-0.44932975811186\\
3.61561561561562	-0.456469356269652\\
3.62362362362362	-0.463579682029238\\
3.63163163163163	-0.47066027942072\\
3.63963963963964	-0.47771069438061\\
3.64764764764765	-0.484730474780961\\
3.65565565565566	-0.49171917045835\\
3.66366366366366	-0.498676333242752\\
3.67167167167167	-0.505601516986281\\
3.67967967967968	-0.512494277591792\\
3.68768768768769	-0.519354173041373\\
3.6956956956957	-0.526180763424678\\
3.7037037037037	-0.532973610967149\\
3.71171171171171	-0.539732280058079\\
3.71971971971972	-0.546456337278553\\
3.72772772772773	-0.553145351429241\\
3.73573573573574	-0.559798893558051\\
3.74374374374374	-0.566416536987634\\
3.75175175175175	-0.572997857342748\\
3.75975975975976	-0.579542432577471\\
3.76776776776777	-0.586049843002266\\
3.77577577577578	-0.592519671310898\\
3.78378378378378	-0.59895150260719\\
3.79179179179179	-0.605344924431632\\
3.7997997997998	-0.611699526787831\\
3.80780780780781	-0.618014902168803\\
3.81581581581582	-0.624290645583106\\
3.82382382382382	-0.630526354580811\\
3.83183183183183	-0.636721629279309\\
3.83983983983984	-0.642876072388956\\
3.84784784784785	-0.64898928923855\\
3.85585585585586	-0.655060887800641\\
3.86386386386386	-0.66109047871667\\
3.87187187187187	-0.667077675321938\\
3.87987987987988	-0.673022093670401\\
3.88788788788789	-0.678923352559294\\
3.8958958958959	-0.684781073553574\\
3.9039039039039	-0.690594881010193\\
3.91191191191191	-0.696364402102178\\
3.91991991991992	-0.702089266842549\\
3.92792792792793	-0.707769108108042\\
3.93593593593594	-0.713403561662651\\
3.94394394394394	-0.718992266180986\\
3.95195195195195	-0.724534863271444\\
3.95995995995996	-0.730030997499192\\
3.96796796796797	-0.735480316408962\\
3.97597597597598	-0.740882470547652\\
3.98398398398398	-0.746237113486732\\
3.99199199199199	-0.751543901844466\\
4	-0.756802495307928\\
};
\addlegendentry{true function $f(x)$};

\addplot [color=black,only marks,mark=*,mark options={solid}]
  table[row sep=crcr]{%
0.5376671395461	0.377144985764094\\
1.83388501459509	1.2690836747931\\
-2.25884686100365	-0.699945033650382\\
0.862173320368121	0.752953240332253\\
0.318765239858981	0.38486853343868\\
-1.30768829630527	-0.986082903468253\\
-0.433592022305684	-0.432547566116779\\
0.34262446653865	0.484929929653214\\
3.57839693972576	-0.28214251860352\\
2.76943702988488	0.505343578517061\\
};
\addlegendentry{observations $\data$};

\end{axis}
\end{tikzpicture}%
  \caption{Example of Bayesian linear regression using the squared
    exponential covariance function.  The true function is $f =
    \sin(x)$.  The kernel parameters are $\lambda = \ell = 1$, and the
    noise variance was set to $\sigma^2 = 0.1^2$.}
  \label{kernel_example}
\end{figure}

Sometimes thinking in terms of the kernel can help even when you have
an explicit feature expansion on hand.  As an example, imagine our
inputs are binary vectors of length $n$ (so each input $\vec{x}$ is a
subset, a member of the power set $\mc{P}(n)$).  One rather expensive
feature expansion we could try would be to enumerate every member of
$\mc{P}(n)$ and define $\phi(\vec{x})_i = \vec{s}_i \subset \vec{x}$,
where $\vec{s}_i$ is the $i$th element of the power set.  So we
represent our set $\vec{x}$ by a feature vector of length $2^n$
indicating every subset of $\vec{x}$.  This is a very expensive
feature expansion, requiring exponential space to store for each
input.  However, if we take $\mat{\Sigma} = \mat{I}$, we can compute
the dot product as:
\begin{equation}
  K(\vec{x}, \vec{x}') = 2^{\lvert x \cap x' \rvert},
\end{equation}
which only requires time and space linear in $n$!

\end{document}
