\documentclass{article}

\usepackage[T1]{fontenc}
\usepackage[osf]{libertine}
\usepackage[scaled=0.8]{beramono}
\usepackage[margin=1.5in]{geometry}
\usepackage{url}
\usepackage{booktabs}
\usepackage{amsmath}
\usepackage{amssymb}
\usepackage{nicefrac}
\usepackage{microtype}
\usepackage{bm}

\usepackage{sectsty}
\sectionfont{\large}
\subsectionfont{\normalsize}

\usepackage{titlesec}
\titlespacing{\section}{0pt}{10pt plus 2pt minus 2pt}{0pt plus 2pt minus 0pt}
\titlespacing{\subsection}{0pt}{5pt plus 2pt minus 2pt}{0pt plus 2pt minus 0pt}

\usepackage{pgfplots}
\pgfplotsset{
  compat=newest,
  plot coordinates/math parser=false,
  tick label style={font=\footnotesize, /pgf/number format/fixed},
  label style={font=\small},
  legend style={font=\small},
  every axis/.append style={
    tick align=outside,
    clip mode=individual,
    scaled ticks=false,
    thick,
    tick style={semithick, black}
  }
}

\pgfkeys{/pgf/number format/.cd, set thousands separator={\,}}

\usepgfplotslibrary{external}
\tikzexternalize[prefix=tikz/]

\newlength\figurewidth
\newlength\figureheight

\setlength{\figurewidth}{8cm}
\setlength{\figureheight}{6cm}

\newlength\squarefigurewidth
\newlength\squarefigureheight

\setlength{\squarefigurewidth}{7cm}
\setlength{\squarefigureheight}{6cm}

\setlength{\parindent}{0pt}
\setlength{\parskip}{1ex}

\newcommand{\acro}[1]{\textsc{\MakeLowercase{#1}}}
\newcommand{\given}{\mid}
\newcommand{\mc}[1]{\mathcal{#1}}
\newcommand{\data}{\mc{D}}
\newcommand{\trans}{^\top}
\newcommand{\inv}{^{-1}}
\newcommand{\intd}[1]{\,\mathrm{d}{#1}}
\newcommand{\mat}[1]{\bm{\mathrm{#1}}}
\renewcommand{\vec}[1]{\bm{\mathrm{#1}}}

\DeclareMathOperator{\var}{var}

\begin{document}

{\large \textbf{CSE 515T (Spring 2015) Midterm solutions}}
\begin{enumerate}
\item
  Consider two coins with unknown bias $\theta_1$ and $\theta_2$,
  respectively.  We place independent, identical beta priors on these
  quantities:
  \begin{equation*}
    p(\theta_1) = \mc{B}(\theta_1; 2, 2);
    \qquad
    p(\theta_2) = \mc{B}(\theta_2; 2, 2).
  \end{equation*}
  Imagine someone flips both coins and tells you that \emph{exactly
    one} of the outcomes (but not which) was a ``head.''  Thus the
  observation was either HT or TH, but you are not told which.  The
  below expressions are conditioned on ``H'' to indicate this
  observation.
  \begin{itemize}
  \item
    Give an expression for the posterior of the first coin's bias
    given this observation, $p(\theta_1 \given \text{H})$.  Simplify
    the result as much as you can.  Plot the prior and the posterior
    for $\theta_1$ over the interval $\theta_1 \in (0, 1)$.
  \item
    Give an expression for the joint posterior $p(\theta_1, \theta_2
    \given \text{H})$. Plot the joint prior, the likelihood, and the
    joint posterior as three separate heat maps over the unit square
    $(\theta_1, \theta_2) \in (0, 1)^2$.  Use a grid with at least 100
    values along each of the two $\theta$ axes.
  \item
    Summarize what the observation taught us about the bias of the
    coins.
  \end{itemize}
\end{enumerate}

\subsection*{Solution}

The outcome of the unseen experiment was either HT or TH.  These are
mutually exhaustive and independent events, so we may use the sum
rule to derive the desired posterior:
\begin{equation*}
  p(\theta_1 \given \text{H})
  =
  \Pr(\text{HT})p(\theta_1 \given \text{HT})
  +
  \Pr(\text{TH})p(\theta_1 \given \text{TH})
  .
\end{equation*}
Notice that both $p(\theta_1 \given \text{HT})$ and $p(\theta_1 \given
\text{TH})$ can be computed explicitly as updated beta distributions,
now that the coins in the outcomes have been identified:
\begin{align*}
  p(\theta_1 \given \text{HT})
  &=
  \mc{B}(\theta_1; 3, 2)
  \\
  p(\theta_1 \given \text{TH})
  &=
  \mc{B}(\theta_1; 2, 3).
\end{align*}

What is $\Pr(\text{HT})$?  It can be calculated explicitly, but a
simpler approach is to appeal to symmetry between the coins to
conclude $\Pr(\text{HT}) = \Pr(\text{TH}) = \nicefrac{1}{2}$.  Thus
\begin{equation*}
  p(\theta_1 \given \text{H})
  =
  \tfrac{1}{2}
  \bigl(
  p(\theta_1 \given \text{HT}) +
  p(\theta_1 \given \text{TH})
  \bigr)
  =
  \tfrac{1}{2}
  \bigl(
  \mc{B}(\theta_1; 3, 2) +
  \mc{B}(\theta_1; 2, 3)
  \bigr)
  .
\end{equation*}
We can simply this expression further.  The posterior is proportional
to
\begin{equation*}
  p(\theta_1 \given \text{H})
  \propto
  \theta_1^2 (1 - \theta_1)
  +
  \theta_1 (1 - \theta_1)^2
  =
  \theta_1(1 - \theta_1)
  \propto
  \mc{B}(\theta_1; 2, 2).
\end{equation*}
Therefore the distribution of $\theta_1$ has not changed given our
observation!  The prior (and posterior!) for $\theta_1$ is plotted
in Figure \ref{problem_1_prior_1}.

\begin{figure}
  \centering
  % This file was created by matlab2tikz.
% Minimal pgfplots version: 1.3
%
\tikzsetnextfilename{problem_1_prior_1}
\definecolor{mycolor1}{rgb}{0.12157,0.47059,0.70588}%
%
\begin{tikzpicture}

\begin{axis}[%
width=0.95092\figurewidth,
height=\figureheight,
at={(0\figurewidth,0\figureheight)},
scale only axis,
xmin=0,
xmax=1,
xlabel={$\theta_1$},
ymin=0,
ymax=1.75,
axis x line*=bottom,
axis y line*=left,
legend style={legend cell align=left,align=left,fill=none,draw=none}
]
\addplot [color=mycolor1,solid]
  table[row sep=crcr]{%
0	0\\
0.001001001001001	0.00599999398798197\\
0.002002002002002	0.0119879639399159\\
0.003003003003003	0.0179639098558017\\
0.004004004004004	0.0239278317356395\\
0.005005005005005	0.0298797295794293\\
0.00600600600600601	0.0358196033871709\\
0.00700700700700701	0.0417474531588646\\
0.00800800800800801	0.0476632788945101\\
0.00900900900900901	0.0535670805941076\\
0.01001001001001	0.059458858257657\\
0.011011011011011	0.0653386118851584\\
0.012012012012012	0.0712063414766117\\
0.013013013013013	0.077062047032017\\
0.014014014014014	0.0829057285513742\\
0.015015015015015	0.0887373860346834\\
0.016016016016016	0.0945570194819444\\
0.017017017017017	0.100364628893157\\
0.018018018018018	0.106160214268322\\
0.019019019019019	0.111943775607439\\
0.02002002002002	0.117715312910508\\
0.021021021021021	0.123474826177529\\
0.022022022022022	0.129222315408502\\
0.023023023023023	0.134957780603426\\
0.024024024024024	0.140681221762303\\
0.025025025025025	0.146392638885131\\
0.026026026026026	0.152092031971912\\
0.027027027027027	0.157779401022644\\
0.028028028028028	0.163454746037329\\
0.029029029029029	0.169118067015965\\
0.03003003003003	0.174769363958553\\
0.031031031031031	0.180408636865093\\
0.032032032032032	0.186035885735585\\
0.033033033033033	0.19165111057003\\
0.034034034034034	0.197254311368425\\
0.035035035035035	0.202845488130773\\
0.036036036036036	0.208424640857073\\
0.037037037037037	0.213991769547325\\
0.038038038038038	0.219546874201529\\
0.039039039039039	0.225089954819685\\
0.04004004004004	0.230621011401792\\
0.041041041041041	0.236140043947852\\
0.042042042042042	0.241647052457863\\
0.043043043043043	0.247142036931827\\
0.044044044044044	0.252624997369742\\
0.045045045045045	0.258095933771609\\
0.046046046046046	0.263554846137429\\
0.047047047047047	0.2690017344672\\
0.048048048048048	0.274436598760923\\
0.049049049049049	0.279859439018598\\
0.0500500500500501	0.285270255240225\\
0.0510510510510511	0.290669047425804\\
0.0520520520520521	0.296055815575335\\
0.0530530530530531	0.301430559688818\\
0.0540540540540541	0.306793279766253\\
0.0550550550550551	0.312143975807639\\
0.0560560560560561	0.317482647812978\\
0.0570570570570571	0.322809295782269\\
0.0580580580580581	0.328123919715511\\
0.0590590590590591	0.333426519612706\\
0.0600600600600601	0.338717095473852\\
0.0610610610610611	0.343995647298951\\
0.0620620620620621	0.349262175088001\\
0.0630630630630631	0.354516678841003\\
0.0640640640640641	0.359759158557957\\
0.0650650650650651	0.364989614238864\\
0.0660660660660661	0.370208045883722\\
0.0670670670670671	0.375414453492532\\
0.0680680680680681	0.380608837065294\\
0.0690690690690691	0.385791196602007\\
0.0700700700700701	0.390961532102673\\
0.0710710710710711	0.396119843567291\\
0.0720720720720721	0.401266130995861\\
0.0730730730730731	0.406400394388382\\
0.0740740740740741	0.411522633744856\\
0.0750750750750751	0.416632849065282\\
0.0760760760760761	0.421731040349659\\
0.0770770770770771	0.426817207597988\\
0.0780780780780781	0.43189135081027\\
0.0790790790790791	0.436953469986503\\
0.0800800800800801	0.442003565126688\\
0.0810810810810811	0.447041636230825\\
0.0820820820820821	0.452067683298914\\
0.0830830830830831	0.457081706330956\\
0.0840840840840841	0.462083705326949\\
0.0850850850850851	0.467073680286893\\
0.0860860860860861	0.47205163121079\\
0.0870870870870871	0.477017558098639\\
0.0880880880880881	0.48197146095044\\
0.0890890890890891	0.486913339766193\\
0.0900900900900901	0.491843194545897\\
0.0910910910910911	0.496761025289554\\
0.0920920920920921	0.501666831997162\\
0.0930930930930931	0.506560614668723\\
0.0940940940940941	0.511442373304235\\
0.0950950950950951	0.5163121079037\\
0.0960960960960961	0.521169818467116\\
0.0970970970970971	0.526015504994484\\
0.0980980980980981	0.530849167485804\\
0.0990990990990991	0.535670805941076\\
0.1001001001001	0.5404804203603\\
0.101101101101101	0.545278010743476\\
0.102102102102102	0.550063577090604\\
0.103103103103103	0.554837119401684\\
0.104104104104104	0.559598637676716\\
0.105105105105105	0.5643481319157\\
0.106106106106106	0.569085602118635\\
0.107107107107107	0.573811048285523\\
0.108108108108108	0.578524470416362\\
0.109109109109109	0.583225868511154\\
0.11011011011011	0.587915242569897\\
0.111111111111111	0.592592592592593\\
0.112112112112112	0.59725791857924\\
0.113113113113113	0.601911220529839\\
0.114114114114114	0.60655249844439\\
0.115115115115115	0.611181752322894\\
0.116116116116116	0.615798982165348\\
0.117117117117117	0.620404187971755\\
0.118118118118118	0.624997369742114\\
0.119119119119119	0.629578527476425\\
0.12012012012012	0.634147661174688\\
0.121121121121121	0.638704770836903\\
0.122122122122122	0.64324985646307\\
0.123123123123123	0.647782918053188\\
0.124124124124124	0.652303955607259\\
0.125125125125125	0.656812969125281\\
0.126126126126126	0.661309958607256\\
0.127127127127127	0.665794924053182\\
0.128128128128128	0.670267865463061\\
0.129129129129129	0.674728782836891\\
0.13013013013013	0.679177676174673\\
0.131131131131131	0.683614545476407\\
0.132132132132132	0.688039390742093\\
0.133133133133133	0.692452211971731\\
0.134134134134134	0.696853009165322\\
0.135135135135135	0.701241782322864\\
0.136136136136136	0.705618531444357\\
0.137137137137137	0.709983256529803\\
0.138138138138138	0.714335957579201\\
0.139139139139139	0.718676634592551\\
0.14014014014014	0.723005287569852\\
0.141141141141141	0.727321916511106\\
0.142142142142142	0.731626521416311\\
0.143143143143143	0.735919102285469\\
0.144144144144144	0.740199659118578\\
0.145145145145145	0.74446819191564\\
0.146146146146146	0.748724700676652\\
0.147147147147147	0.752969185401618\\
0.148148148148148	0.757201646090535\\
0.149149149149149	0.761422082743404\\
0.15015015015015	0.765630495360225\\
0.151151151151151	0.769826883940998\\
0.152152152152152	0.774011248485723\\
0.153153153153153	0.7781835889944\\
0.154154154154154	0.782343905467029\\
0.155155155155155	0.786492197903609\\
0.156156156156156	0.790628466304142\\
0.157157157157157	0.794752710668627\\
0.158158158158158	0.798864930997063\\
0.159159159159159	0.802965127289452\\
0.16016016016016	0.807053299545792\\
0.161161161161161	0.811129447766084\\
0.162162162162162	0.815193571950329\\
0.163163163163163	0.819245672098525\\
0.164164164164164	0.823285748210673\\
0.165165165165165	0.827313800286773\\
0.166166166166166	0.831329828326825\\
0.167167167167167	0.835333832330829\\
0.168168168168168	0.839325812298785\\
0.169169169169169	0.843305768230693\\
0.17017017017017	0.847273700126553\\
0.171171171171171	0.851229607986365\\
0.172172172172172	0.855173491810128\\
0.173173173173173	0.859105351597844\\
0.174174174174174	0.863025187349512\\
0.175175175175175	0.866932999065131\\
0.176176176176176	0.870828786744703\\
0.177177177177177	0.874712550388226\\
0.178178178178178	0.878584289995701\\
0.179179179179179	0.882444005567129\\
0.18018018018018	0.886291697102508\\
0.181181181181181	0.890127364601839\\
0.182182182182182	0.893951008065122\\
0.183183183183183	0.897762627492357\\
0.184184184184184	0.901562222883544\\
0.185185185185185	0.905349794238683\\
0.186186186186186	0.909125341557774\\
0.187187187187187	0.912888864840817\\
0.188188188188188	0.916640364087812\\
0.189189189189189	0.920379839298758\\
0.19019019019019	0.924107290473657\\
0.191191191191191	0.927822717612508\\
0.192192192192192	0.93152612071531\\
0.193193193193193	0.935217499782064\\
0.194194194194194	0.938896854812771\\
0.195195195195195	0.942564185807429\\
0.196196196196196	0.946219492766039\\
0.197197197197197	0.949862775688602\\
0.198198198198198	0.953494034575116\\
0.199199199199199	0.957113269425582\\
0.2002002002002	0.96072048024\\
0.201201201201201	0.96431566701837\\
0.202202202202202	0.967898829760692\\
0.203203203203203	0.971469968466966\\
0.204204204204204	0.975029083137191\\
0.205205205205205	0.978576173771369\\
0.206206206206206	0.982111240369498\\
0.207207207207207	0.98563428293158\\
0.208208208208208	0.989145301457614\\
0.209209209209209	0.992644295947599\\
0.21021021021021	0.996131266401537\\
0.211211211211211	0.999606212819426\\
0.212212212212212	1.00306913520127\\
0.213213213213213	1.00652003354706\\
0.214214214214214	1.00995890785681\\
0.215215215215215	1.0133857581305\\
0.216216216216216	1.01680058436815\\
0.217217217217217	1.02020338656975\\
0.218218218218218	1.02359416473531\\
0.219219219219219	1.02697291886481\\
0.22022022022022	1.03033964895827\\
0.221221221221221	1.03369435501568\\
0.222222222222222	1.03703703703704\\
0.223223223223223	1.04036769502235\\
0.224224224224224	1.04368632897161\\
0.225225225225225	1.04699293888483\\
0.226226226226226	1.050287524762\\
0.227227227227227	1.05357008660312\\
0.228228228228228	1.05684062440819\\
0.229229229229229	1.06009913817722\\
0.23023023023023	1.06334562791019\\
0.231231231231231	1.06658009360712\\
0.232232232232232	1.069802535268\\
0.233233233233233	1.07301295289283\\
0.234234234234234	1.07621134648162\\
0.235235235235235	1.07939771603435\\
0.236236236236236	1.08257206155104\\
0.237237237237237	1.08573438303168\\
0.238238238238238	1.08888468047627\\
0.239239239239239	1.09202295388482\\
0.24024024024024	1.09514920325731\\
0.241241241241241	1.09826342859376\\
0.242242242242242	1.10136562989416\\
0.243243243243243	1.10445580715851\\
0.244244244244244	1.10753396038681\\
0.245245245245245	1.11060008957907\\
0.246246246246246	1.11365419473528\\
0.247247247247247	1.11669627585544\\
0.248248248248248	1.11972633293955\\
0.249249249249249	1.12274436598761\\
0.25025025025025	1.12575037499962\\
0.251251251251251	1.12874435997559\\
0.252252252252252	1.13172632091551\\
0.253253253253253	1.13469625781938\\
0.254254254254254	1.1376541706872\\
0.255255255255255	1.14060005951898\\
0.256256256256256	1.14353392431471\\
0.257257257257257	1.14645576507438\\
0.258258258258258	1.14936558179801\\
0.259259259259259	1.1522633744856\\
0.26026026026026	1.15514914313713\\
0.261261261261261	1.15802288775262\\
0.262262262262262	1.16088460833206\\
0.263263263263263	1.16373430487545\\
0.264264264264264	1.16657197738279\\
0.265265265265265	1.16939762585408\\
0.266266266266266	1.17221125028933\\
0.267267267267267	1.17501285068853\\
0.268268268268268	1.17780242705168\\
0.269269269269269	1.18057997937878\\
0.27027027027027	1.18334550766983\\
0.271271271271271	1.18609901192484\\
0.272272272272272	1.1888404921438\\
0.273273273273273	1.19156994832671\\
0.274274274274274	1.19428738047357\\
0.275275275275275	1.19699278858438\\
0.276276276276276	1.19968617265915\\
0.277277277277277	1.20236753269786\\
0.278278278278278	1.20503686870053\\
0.279279279279279	1.20769418066715\\
0.28028028028028	1.21033946859773\\
0.281281281281281	1.21297273249225\\
0.282282282282282	1.21559397235073\\
0.283283283283283	1.21820318817316\\
0.284284284284284	1.22080037995954\\
0.285285285285285	1.22338554770987\\
0.286286286286286	1.22595869142416\\
0.287287287287287	1.22851981110239\\
0.288288288288288	1.23106890674458\\
0.289289289289289	1.23360597835072\\
0.29029029029029	1.23613102592082\\
0.291291291291291	1.23864404945486\\
0.292292292292292	1.24114504895286\\
0.293293293293293	1.24363402441481\\
0.294294294294294	1.24611097584071\\
0.295295295295295	1.24857590323056\\
0.296296296296296	1.25102880658436\\
0.297297297297297	1.25346968590212\\
0.298298298298298	1.25589854118383\\
0.299299299299299	1.25831537242949\\
0.3003003003003	1.2607201796391\\
0.301301301301301	1.26311296281266\\
0.302302302302302	1.26549372195018\\
0.303303303303303	1.26786245705165\\
0.304304304304304	1.27021916811707\\
0.305305305305305	1.27256385514644\\
0.306306306306306	1.27489651813976\\
0.307307307307307	1.27721715709704\\
0.308308308308308	1.27952577201826\\
0.309309309309309	1.28182236290344\\
0.31031031031031	1.28410692975258\\
0.311311311311311	1.28637947256566\\
0.312312312312312	1.28863999134269\\
0.313313313313313	1.29088848608368\\
0.314314314314314	1.29312495678862\\
0.315315315315315	1.29534940345751\\
0.316316316316316	1.29756182609035\\
0.317317317317317	1.29976222468715\\
0.318318318318318	1.3019505992479\\
0.319319319319319	1.3041269497726\\
0.32032032032032	1.30629127626125\\
0.321321321321321	1.30844357871385\\
0.322322322322322	1.3105838571304\\
0.323323323323323	1.31271211151091\\
0.324324324324324	1.31482834185537\\
0.325325325325325	1.31693254816378\\
0.326326326326326	1.31902473043614\\
0.327327327327327	1.32110488867246\\
0.328328328328328	1.32317302287272\\
0.329329329329329	1.32522913303694\\
0.33033033033033	1.32727321916511\\
0.331331331331331	1.32930528125723\\
0.332332332332332	1.33132531931331\\
0.333333333333333	1.33333333333333\\
0.334334334334334	1.33532932331731\\
0.335335335335335	1.33731328926524\\
0.336336336336336	1.33928523117712\\
0.337337337337337	1.34124514905296\\
0.338338338338338	1.34319304289274\\
0.339339339339339	1.34512891269648\\
0.34034034034034	1.34705275846417\\
0.341341341341341	1.34896458019581\\
0.342342342342342	1.35086437789141\\
0.343343343343343	1.35275215155095\\
0.344344344344344	1.35462790117445\\
0.345345345345345	1.3564916267619\\
0.346346346346346	1.3583433283133\\
0.347347347347347	1.36018300582865\\
0.348348348348348	1.36201065930796\\
0.349349349349349	1.36382628875121\\
0.35035035035035	1.36562989415842\\
0.351351351351351	1.36742147552958\\
0.352352352352352	1.3692010328647\\
0.353353353353353	1.37096856616376\\
0.354354354354354	1.37272407542678\\
0.355355355355355	1.37446756065375\\
0.356356356356356	1.37619902184467\\
0.357357357357357	1.37791845899954\\
0.358358358358358	1.37962587211836\\
0.359359359359359	1.38132126120114\\
0.36036036036036	1.38300462624787\\
0.361361361361361	1.38467596725855\\
0.362362362362362	1.38633528423318\\
0.363363363363363	1.38798257717177\\
0.364364364364364	1.3896178460743\\
0.365365365365365	1.39124109094079\\
0.366366366366366	1.39285231177123\\
0.367367367367367	1.39445150856562\\
0.368368368368368	1.39603868132397\\
0.369369369369369	1.39761383004626\\
0.37037037037037	1.39917695473251\\
0.371371371371371	1.40072805538271\\
0.372372372372372	1.40226713199686\\
0.373373373373373	1.40379418457497\\
0.374374374374374	1.40530921311702\\
0.375375375375375	1.40681221762303\\
0.376376376376376	1.40830319809299\\
0.377377377377377	1.4097821545269\\
0.378378378378378	1.41124908692476\\
0.379379379379379	1.41270399528658\\
0.38038038038038	1.41414687961234\\
0.381381381381381	1.41557773990206\\
0.382382382382382	1.41699657615574\\
0.383383383383383	1.41840338837336\\
0.384384384384384	1.41979817655493\\
0.385385385385385	1.42118094070046\\
0.386386386386386	1.42255168080994\\
0.387387387387387	1.42391039688337\\
0.388388388388388	1.42525708892075\\
0.389389389389389	1.42659175692209\\
0.39039039039039	1.42791440088737\\
0.391391391391391	1.42922502081661\\
0.392392392392392	1.4305236167098\\
0.393393393393393	1.43181018856695\\
0.394394394394394	1.43308473638804\\
0.395395395395395	1.43434726017309\\
0.396396396396396	1.43559775992208\\
0.397397397397397	1.43683623563503\\
0.398398398398398	1.43806268731194\\
0.399399399399399	1.43927711495279\\
0.4004004004004	1.4404795185576\\
0.401401401401401	1.44166989812635\\
0.402402402402402	1.44284825365906\\
0.403403403403403	1.44401458515573\\
0.404404404404404	1.44516889261634\\
0.405405405405405	1.44631117604091\\
0.406406406406406	1.44744143542942\\
0.407407407407407	1.44855967078189\\
0.408408408408408	1.44966588209831\\
0.409409409409409	1.45076006937869\\
0.41041041041041	1.45184223262301\\
0.411411411411411	1.45291237183129\\
0.412412412412412	1.45397048700352\\
0.413413413413413	1.4550165781397\\
0.414414414414414	1.45605064523983\\
0.415415415415415	1.45707268830392\\
0.416416416416416	1.45808270733196\\
0.417417417417417	1.45908070232395\\
0.418418418418418	1.46006667327989\\
0.419419419419419	1.46104062019978\\
0.42042042042042	1.46200254308362\\
0.421421421421421	1.46295244193142\\
0.422422422422422	1.46389031674317\\
0.423423423423423	1.46481616751887\\
0.424424424424424	1.46572999425852\\
0.425425425425425	1.46663179696213\\
0.426426426426426	1.46752157562968\\
0.427427427427427	1.46839933026119\\
0.428428428428428	1.46926506085665\\
0.429429429429429	1.47011876741606\\
0.43043043043043	1.47096044993943\\
0.431431431431431	1.47179010842675\\
0.432432432432432	1.47260774287801\\
0.433433433433433	1.47341335329323\\
0.434434434434434	1.4742069396724\\
0.435435435435435	1.47498850201553\\
0.436436436436436	1.47575804032261\\
0.437437437437437	1.47651555459363\\
0.438438438438438	1.47726104482861\\
0.439439439439439	1.47799451102754\\
0.44044044044044	1.47871595319043\\
0.441441441441441	1.47942537131726\\
0.442442442442442	1.48012276540805\\
0.443443443443443	1.48080813546279\\
0.444444444444444	1.48148148148148\\
0.445445445445445	1.48214280346412\\
0.446446446446446	1.48279210141072\\
0.447447447447447	1.48342937532127\\
0.448448448448448	1.48405462519577\\
0.449449449449449	1.48466785103422\\
0.45045045045045	1.48526905283662\\
0.451451451451451	1.48585823060298\\
0.452452452452452	1.48643538433328\\
0.453453453453453	1.48700051402754\\
0.454454454454454	1.48755361968575\\
0.455455455455455	1.48809470130791\\
0.456456456456456	1.48862375889403\\
0.457457457457457	1.4891407924441\\
0.458458458458458	1.48964580195811\\
0.459459459459459	1.49013878743608\\
0.46046046046046	1.49061974887801\\
0.461461461461461	1.49108868628388\\
0.462462462462462	1.49154559965371\\
0.463463463463463	1.49199048898749\\
0.464464464464464	1.49242335428522\\
0.465465465465465	1.4928441955469\\
0.466466466466466	1.49325301277253\\
0.467467467467467	1.49364980596212\\
0.468468468468468	1.49403457511566\\
0.469469469469469	1.49440732023315\\
0.47047047047047	1.49476804131459\\
0.471471471471471	1.49511673835998\\
0.472472472472472	1.49545341136933\\
0.473473473473473	1.49577806034262\\
0.474474474474474	1.49609068527987\\
0.475475475475475	1.49639128618108\\
0.476476476476476	1.49667986304623\\
0.477477477477477	1.49695641587533\\
0.478478478478478	1.49722094466839\\
0.479479479479479	1.4974734494254\\
0.48048048048048	1.49771393014636\\
0.481481481481481	1.49794238683128\\
0.482482482482482	1.49815881948014\\
0.483483483483483	1.49836322809296\\
0.484484484484485	1.49855561266973\\
0.485485485485485	1.49873597321045\\
0.486486486486487	1.49890430971512\\
0.487487487487487	1.49906062218375\\
0.488488488488488	1.49920491061632\\
0.48948948948949	1.49933717501285\\
0.49049049049049	1.49945741537333\\
0.491491491491492	1.49956563169776\\
0.492492492492492	1.49966182398615\\
0.493493493493493	1.49974599223848\\
0.494494494494495	1.49981813645477\\
0.495495495495495	1.49987825663501\\
0.496496496496497	1.49992635277921\\
0.497497497497497	1.49996242488735\\
0.498498498498498	1.49998647295945\\
0.4994994994995	1.49999849699549\\
0.500500500500501	1.49999849699549\\
0.501501501501502	1.49998647295945\\
0.502502502502503	1.49996242488735\\
0.503503503503503	1.49992635277921\\
0.504504504504504	1.49987825663501\\
0.505505505505506	1.49981813645477\\
0.506506506506507	1.49974599223848\\
0.507507507507508	1.49966182398615\\
0.508508508508508	1.49956563169776\\
0.509509509509509	1.49945741537333\\
0.510510510510511	1.49933717501285\\
0.511511511511512	1.49920491061632\\
0.512512512512513	1.49906062218375\\
0.513513513513513	1.49890430971512\\
0.514514514514514	1.49873597321045\\
0.515515515515516	1.49855561266973\\
0.516516516516517	1.49836322809296\\
0.517517517517518	1.49815881948014\\
0.518518518518518	1.49794238683128\\
0.519519519519519	1.49771393014636\\
0.520520520520521	1.4974734494254\\
0.521521521521522	1.49722094466839\\
0.522522522522523	1.49695641587533\\
0.523523523523523	1.49667986304623\\
0.524524524524524	1.49639128618108\\
0.525525525525526	1.49609068527987\\
0.526526526526527	1.49577806034262\\
0.527527527527528	1.49545341136933\\
0.528528528528528	1.49511673835998\\
0.529529529529529	1.49476804131459\\
0.530530530530531	1.49440732023315\\
0.531531531531532	1.49403457511566\\
0.532532532532533	1.49364980596212\\
0.533533533533533	1.49325301277253\\
0.534534534534535	1.4928441955469\\
0.535535535535536	1.49242335428522\\
0.536536536536537	1.49199048898749\\
0.537537537537538	1.49154559965371\\
0.538538538538539	1.49108868628388\\
0.53953953953954	1.49061974887801\\
0.540540540540541	1.49013878743608\\
0.541541541541542	1.48964580195811\\
0.542542542542543	1.4891407924441\\
0.543543543543544	1.48862375889403\\
0.544544544544545	1.48809470130791\\
0.545545545545546	1.48755361968575\\
0.546546546546547	1.48700051402754\\
0.547547547547548	1.48643538433328\\
0.548548548548549	1.48585823060298\\
0.54954954954955	1.48526905283662\\
0.550550550550551	1.48466785103422\\
0.551551551551552	1.48405462519577\\
0.552552552552553	1.48342937532127\\
0.553553553553554	1.48279210141072\\
0.554554554554555	1.48214280346412\\
0.555555555555556	1.48148148148148\\
0.556556556556557	1.48080813546279\\
0.557557557557558	1.48012276540805\\
0.558558558558559	1.47942537131726\\
0.55955955955956	1.47871595319043\\
0.560560560560561	1.47799451102754\\
0.561561561561562	1.47726104482861\\
0.562562562562563	1.47651555459363\\
0.563563563563564	1.4757580403226\\
0.564564564564565	1.47498850201553\\
0.565565565565566	1.4742069396724\\
0.566566566566567	1.47341335329323\\
0.567567567567568	1.47260774287801\\
0.568568568568569	1.47179010842675\\
0.56956956956957	1.47096044993943\\
0.570570570570571	1.47011876741606\\
0.571571571571572	1.46926506085665\\
0.572572572572573	1.46839933026119\\
0.573573573573574	1.46752157562968\\
0.574574574574575	1.46663179696213\\
0.575575575575576	1.46572999425852\\
0.576576576576577	1.46481616751887\\
0.577577577577578	1.46389031674317\\
0.578578578578579	1.46295244193142\\
0.57957957957958	1.46200254308362\\
0.580580580580581	1.46104062019978\\
0.581581581581582	1.46006667327989\\
0.582582582582583	1.45908070232395\\
0.583583583583584	1.45808270733196\\
0.584584584584585	1.45707268830392\\
0.585585585585586	1.45605064523983\\
0.586586586586587	1.4550165781397\\
0.587587587587588	1.45397048700352\\
0.588588588588589	1.45291237183129\\
0.58958958958959	1.45184223262301\\
0.590590590590591	1.45076006937869\\
0.591591591591592	1.44966588209831\\
0.592592592592593	1.44855967078189\\
0.593593593593594	1.44744143542942\\
0.594594594594595	1.44631117604091\\
0.595595595595596	1.44516889261634\\
0.596596596596597	1.44401458515573\\
0.597597597597598	1.44284825365906\\
0.598598598598599	1.44166989812635\\
0.5995995995996	1.4404795185576\\
0.600600600600601	1.43927711495279\\
0.601601601601602	1.43806268731194\\
0.602602602602603	1.43683623563503\\
0.603603603603604	1.43559775992208\\
0.604604604604605	1.43434726017309\\
0.605605605605606	1.43308473638804\\
0.606606606606607	1.43181018856695\\
0.607607607607608	1.4305236167098\\
0.608608608608609	1.42922502081661\\
0.60960960960961	1.42791440088737\\
0.610610610610611	1.42659175692209\\
0.611611611611612	1.42525708892075\\
0.612612612612613	1.42391039688337\\
0.613613613613614	1.42255168080994\\
0.614614614614615	1.42118094070046\\
0.615615615615616	1.41979817655493\\
0.616616616616617	1.41840338837336\\
0.617617617617618	1.41699657615574\\
0.618618618618619	1.41557773990206\\
0.61961961961962	1.41414687961234\\
0.620620620620621	1.41270399528658\\
0.621621621621622	1.41124908692476\\
0.622622622622623	1.4097821545269\\
0.623623623623624	1.40830319809299\\
0.624624624624625	1.40681221762303\\
0.625625625625626	1.40530921311702\\
0.626626626626627	1.40379418457497\\
0.627627627627628	1.40226713199686\\
0.628628628628629	1.40072805538271\\
0.62962962962963	1.39917695473251\\
0.630630630630631	1.39761383004626\\
0.631631631631632	1.39603868132397\\
0.632632632632633	1.39445150856562\\
0.633633633633634	1.39285231177123\\
0.634634634634635	1.39124109094079\\
0.635635635635636	1.3896178460743\\
0.636636636636637	1.38798257717177\\
0.637637637637638	1.38633528423318\\
0.638638638638639	1.38467596725855\\
0.63963963963964	1.38300462624787\\
0.640640640640641	1.38132126120114\\
0.641641641641642	1.37962587211836\\
0.642642642642643	1.37791845899954\\
0.643643643643644	1.37619902184467\\
0.644644644644645	1.37446756065375\\
0.645645645645646	1.37272407542678\\
0.646646646646647	1.37096856616376\\
0.647647647647648	1.3692010328647\\
0.648648648648649	1.36742147552958\\
0.64964964964965	1.36562989415842\\
0.650650650650651	1.36382628875121\\
0.651651651651652	1.36201065930796\\
0.652652652652653	1.36018300582865\\
0.653653653653654	1.3583433283133\\
0.654654654654655	1.3564916267619\\
0.655655655655656	1.35462790117445\\
0.656656656656657	1.35275215155095\\
0.657657657657658	1.3508643778914\\
0.658658658658659	1.34896458019581\\
0.65965965965966	1.34705275846417\\
0.660660660660661	1.34512891269648\\
0.661661661661662	1.34319304289274\\
0.662662662662663	1.34124514905296\\
0.663663663663664	1.33928523117712\\
0.664664664664665	1.33731328926524\\
0.665665665665666	1.33532932331731\\
0.666666666666667	1.33333333333333\\
0.667667667667668	1.33132531931331\\
0.668668668668669	1.32930528125723\\
0.66966966966967	1.32727321916511\\
0.670670670670671	1.32522913303694\\
0.671671671671672	1.32317302287272\\
0.672672672672673	1.32110488867246\\
0.673673673673674	1.31902473043614\\
0.674674674674675	1.31693254816378\\
0.675675675675676	1.31482834185537\\
0.676676676676677	1.31271211151091\\
0.677677677677678	1.3105838571304\\
0.678678678678679	1.30844357871385\\
0.67967967967968	1.30629127626125\\
0.680680680680681	1.3041269497726\\
0.681681681681682	1.3019505992479\\
0.682682682682683	1.29976222468715\\
0.683683683683684	1.29756182609035\\
0.684684684684685	1.29534940345751\\
0.685685685685686	1.29312495678862\\
0.686686686686687	1.29088848608368\\
0.687687687687688	1.28863999134269\\
0.688688688688689	1.28637947256566\\
0.68968968968969	1.28410692975258\\
0.690690690690691	1.28182236290344\\
0.691691691691692	1.27952577201826\\
0.692692692692693	1.27721715709704\\
0.693693693693694	1.27489651813976\\
0.694694694694695	1.27256385514644\\
0.695695695695696	1.27021916811707\\
0.696696696696697	1.26786245705165\\
0.697697697697698	1.26549372195018\\
0.698698698698699	1.26311296281266\\
0.6996996996997	1.2607201796391\\
0.700700700700701	1.25831537242949\\
0.701701701701702	1.25589854118383\\
0.702702702702703	1.25346968590212\\
0.703703703703704	1.25102880658436\\
0.704704704704705	1.24857590323056\\
0.705705705705706	1.24611097584071\\
0.706706706706707	1.24363402441481\\
0.707707707707708	1.24114504895286\\
0.708708708708709	1.23864404945486\\
0.70970970970971	1.23613102592082\\
0.710710710710711	1.23360597835072\\
0.711711711711712	1.23106890674458\\
0.712712712712713	1.22851981110239\\
0.713713713713714	1.22595869142416\\
0.714714714714715	1.22338554770987\\
0.715715715715716	1.22080037995954\\
0.716716716716717	1.21820318817316\\
0.717717717717718	1.21559397235073\\
0.718718718718719	1.21297273249225\\
0.71971971971972	1.21033946859773\\
0.720720720720721	1.20769418066715\\
0.721721721721722	1.20503686870053\\
0.722722722722723	1.20236753269786\\
0.723723723723724	1.19968617265915\\
0.724724724724725	1.19699278858438\\
0.725725725725726	1.19428738047357\\
0.726726726726727	1.1915699483267\\
0.727727727727728	1.1888404921438\\
0.728728728728729	1.18609901192484\\
0.72972972972973	1.18334550766983\\
0.730730730730731	1.18057997937878\\
0.731731731731732	1.17780242705168\\
0.732732732732733	1.17501285068853\\
0.733733733733734	1.17221125028933\\
0.734734734734735	1.16939762585408\\
0.735735735735736	1.16657197738279\\
0.736736736736737	1.16373430487545\\
0.737737737737738	1.16088460833206\\
0.738738738738739	1.15802288775262\\
0.73973973973974	1.15514914313713\\
0.740740740740741	1.1522633744856\\
0.741741741741742	1.14936558179801\\
0.742742742742743	1.14645576507438\\
0.743743743743744	1.14353392431471\\
0.744744744744745	1.14060005951898\\
0.745745745745746	1.1376541706872\\
0.746746746746747	1.13469625781938\\
0.747747747747748	1.13172632091551\\
0.748748748748749	1.12874435997559\\
0.74974974974975	1.12575037499962\\
0.750750750750751	1.12274436598761\\
0.751751751751752	1.11972633293955\\
0.752752752752753	1.11669627585543\\
0.753753753753754	1.11365419473528\\
0.754754754754755	1.11060008957907\\
0.755755755755756	1.10753396038681\\
0.756756756756757	1.10445580715851\\
0.757757757757758	1.10136562989416\\
0.758758758758759	1.09826342859376\\
0.75975975975976	1.09514920325731\\
0.760760760760761	1.09202295388482\\
0.761761761761762	1.08888468047627\\
0.762762762762763	1.08573438303168\\
0.763763763763764	1.08257206155104\\
0.764764764764765	1.07939771603435\\
0.765765765765766	1.07621134648162\\
0.766766766766767	1.07301295289283\\
0.767767767767768	1.069802535268\\
0.768768768768769	1.06658009360712\\
0.76976976976977	1.06334562791019\\
0.770770770770771	1.06009913817722\\
0.771771771771772	1.05684062440819\\
0.772772772772773	1.05357008660312\\
0.773773773773774	1.050287524762\\
0.774774774774775	1.04699293888483\\
0.775775775775776	1.04368632897161\\
0.776776776776777	1.04036769502235\\
0.777777777777778	1.03703703703704\\
0.778778778778779	1.03369435501568\\
0.77977977977978	1.03033964895827\\
0.780780780780781	1.02697291886481\\
0.781781781781782	1.02359416473531\\
0.782782782782783	1.02020338656975\\
0.783783783783784	1.01680058436815\\
0.784784784784785	1.0133857581305\\
0.785785785785786	1.00995890785681\\
0.786786786786787	1.00652003354706\\
0.787787787787788	1.00306913520127\\
0.788788788788789	0.999606212819426\\
0.78978978978979	0.996131266401537\\
0.790790790790791	0.992644295947599\\
0.791791791791792	0.989145301457614\\
0.792792792792793	0.98563428293158\\
0.793793793793794	0.982111240369498\\
0.794794794794795	0.978576173771369\\
0.795795795795796	0.975029083137191\\
0.796796796796797	0.971469968466965\\
0.797797797797798	0.967898829760692\\
0.798798798798799	0.96431566701837\\
0.7997997997998	0.96072048024\\
0.800800800800801	0.957113269425582\\
0.801801801801802	0.953494034575115\\
0.802802802802803	0.949862775688602\\
0.803803803803804	0.946219492766039\\
0.804804804804805	0.942564185807429\\
0.805805805805806	0.938896854812771\\
0.806806806806807	0.935217499782064\\
0.807807807807808	0.93152612071531\\
0.808808808808809	0.927822717612508\\
0.80980980980981	0.924107290473657\\
0.810810810810811	0.920379839298758\\
0.811811811811812	0.916640364087811\\
0.812812812812813	0.912888864840817\\
0.813813813813814	0.909125341557774\\
0.814814814814815	0.905349794238683\\
0.815815815815816	0.901562222883544\\
0.816816816816817	0.897762627492357\\
0.817817817817818	0.893951008065122\\
0.818818818818819	0.890127364601839\\
0.81981981981982	0.886291697102508\\
0.820820820820821	0.882444005567129\\
0.821821821821822	0.878584289995701\\
0.822822822822823	0.874712550388226\\
0.823823823823824	0.870828786744703\\
0.824824824824825	0.866932999065131\\
0.825825825825826	0.863025187349511\\
0.826826826826827	0.859105351597844\\
0.827827827827828	0.855173491810128\\
0.828828828828829	0.851229607986365\\
0.82982982982983	0.847273700126553\\
0.830830830830831	0.843305768230693\\
0.831831831831832	0.839325812298785\\
0.832832832832833	0.835333832330829\\
0.833833833833834	0.831329828326825\\
0.834834834834835	0.827313800286773\\
0.835835835835836	0.823285748210673\\
0.836836836836837	0.819245672098525\\
0.837837837837838	0.815193571950329\\
0.838838838838839	0.811129447766084\\
0.83983983983984	0.807053299545792\\
0.840840840840841	0.802965127289452\\
0.841841841841842	0.798864930997063\\
0.842842842842843	0.794752710668627\\
0.843843843843844	0.790628466304142\\
0.844844844844845	0.786492197903609\\
0.845845845845846	0.782343905467028\\
0.846846846846847	0.7781835889944\\
0.847847847847848	0.774011248485723\\
0.848848848848849	0.769826883940998\\
0.84984984984985	0.765630495360225\\
0.850850850850851	0.761422082743404\\
0.851851851851852	0.757201646090535\\
0.852852852852853	0.752969185401618\\
0.853853853853854	0.748724700676653\\
0.854854854854855	0.744468191915639\\
0.855855855855856	0.740199659118578\\
0.856856856856857	0.735919102285469\\
0.857857857857858	0.731626521416311\\
0.858858858858859	0.727321916511106\\
0.85985985985986	0.723005287569852\\
0.860860860860861	0.71867663459255\\
0.861861861861862	0.714335957579201\\
0.862862862862863	0.709983256529803\\
0.863863863863864	0.705618531444357\\
0.864864864864865	0.701241782322863\\
0.865865865865866	0.696853009165321\\
0.866866866866867	0.692452211971731\\
0.867867867867868	0.688039390742094\\
0.868868868868869	0.683614545476407\\
0.86986986986987	0.679177676174673\\
0.870870870870871	0.674728782836891\\
0.871871871871872	0.670267865463061\\
0.872872872872873	0.665794924053182\\
0.873873873873874	0.661309958607256\\
0.874874874874875	0.656812969125281\\
0.875875875875876	0.652303955607259\\
0.876876876876877	0.647782918053188\\
0.877877877877878	0.64324985646307\\
0.878878878878879	0.638704770836903\\
0.87987987987988	0.634147661174688\\
0.880880880880881	0.629578527476425\\
0.881881881881882	0.624997369742114\\
0.882882882882883	0.620404187971756\\
0.883883883883884	0.615798982165349\\
0.884884884884885	0.611181752322893\\
0.885885885885886	0.60655249844439\\
0.886886886886887	0.601911220529839\\
0.887887887887888	0.59725791857924\\
0.888888888888889	0.592592592592593\\
0.88988988988989	0.587915242569897\\
0.890890890890891	0.583225868511154\\
0.891891891891892	0.578524470416362\\
0.892892892892893	0.573811048285523\\
0.893893893893894	0.569085602118635\\
0.894894894894895	0.564348131915699\\
0.895895895895896	0.559598637676716\\
0.896896896896897	0.554837119401684\\
0.897897897897898	0.550063577090604\\
0.898898898898899	0.545278010743477\\
0.8998998998999	0.5404804203603\\
0.900900900900901	0.535670805941076\\
0.901901901901902	0.530849167485804\\
0.902902902902903	0.526015504994484\\
0.903903903903904	0.521169818467116\\
0.904904904904905	0.516312107903699\\
0.905905905905906	0.511442373304235\\
0.906906906906907	0.506560614668723\\
0.907907907907908	0.501666831997162\\
0.908908908908909	0.496761025289553\\
0.90990990990991	0.491843194545897\\
0.910910910910911	0.486913339766192\\
0.911911911911912	0.48197146095044\\
0.912912912912913	0.477017558098639\\
0.913913913913914	0.47205163121079\\
0.914914914914915	0.467073680286893\\
0.915915915915916	0.462083705326949\\
0.916916916916917	0.457081706330956\\
0.917917917917918	0.452067683298915\\
0.918918918918919	0.447041636230825\\
0.91991991991992	0.442003565126688\\
0.920920920920921	0.436953469986503\\
0.921921921921922	0.43189135081027\\
0.922922922922923	0.426817207597988\\
0.923923923923924	0.421731040349659\\
0.924924924924925	0.416632849065281\\
0.925925925925926	0.411522633744856\\
0.926926926926927	0.406400394388383\\
0.927927927927928	0.401266130995861\\
0.928928928928929	0.396119843567291\\
0.92992992992993	0.390961532102673\\
0.930930930930931	0.385791196602007\\
0.931931931931932	0.380608837065294\\
0.932932932932933	0.375414453492532\\
0.933933933933934	0.370208045883721\\
0.934934934934935	0.364989614238863\\
0.935935935935936	0.359759158557957\\
0.936936936936937	0.354516678841003\\
0.937937937937938	0.349262175088001\\
0.938938938938939	0.34399564729895\\
0.93993993993994	0.338717095473852\\
0.940940940940941	0.333426519612706\\
0.941941941941942	0.328123919715511\\
0.942942942942943	0.322809295782269\\
0.943943943943944	0.317482647812978\\
0.944944944944945	0.312143975807639\\
0.945945945945946	0.306793279766253\\
0.946946946946947	0.301430559688818\\
0.947947947947948	0.296055815575335\\
0.948948948948949	0.290669047425804\\
0.94994994994995	0.285270255240225\\
0.950950950950951	0.279859439018598\\
0.951951951951952	0.274436598760923\\
0.952952952952953	0.2690017344672\\
0.953953953953954	0.263554846137429\\
0.954954954954955	0.258095933771609\\
0.955955955955956	0.252624997369742\\
0.956956956956957	0.247142036931827\\
0.957957957957958	0.241647052457864\\
0.958958958958959	0.236140043947851\\
0.95995995995996	0.230621011401792\\
0.960960960960961	0.225089954819685\\
0.961961961961962	0.219546874201529\\
0.962962962962963	0.213991769547325\\
0.963963963963964	0.208424640857073\\
0.964964964964965	0.202845488130773\\
0.965965965965966	0.197254311368425\\
0.966966966966967	0.19165111057003\\
0.967967967967968	0.186035885735586\\
0.968968968968969	0.180408636865093\\
0.96996996996997	0.174769363958553\\
0.970970970970971	0.169118067015965\\
0.971971971971972	0.163454746037329\\
0.972972972972973	0.157779401022644\\
0.973973973973974	0.152092031971912\\
0.974974974974975	0.146392638885131\\
0.975975975975976	0.140681221762303\\
0.976976976976977	0.134957780603426\\
0.977977977977978	0.129222315408501\\
0.978978978978979	0.123474826177529\\
0.97997997997998	0.117715312910508\\
0.980980980980981	0.111943775607439\\
0.981981981981982	0.106160214268323\\
0.982982982982983	0.100364628893157\\
0.983983983983984	0.0945570194819443\\
0.984984984984985	0.0887373860346834\\
0.985985985985986	0.0829057285513743\\
0.986986986986987	0.0770620470320172\\
0.987987987987988	0.0712063414766114\\
0.988988988988989	0.0653386118851583\\
0.98998998998999	0.059458858257657\\
0.990990990990991	0.0535670805941077\\
0.991991991991992	0.0476632788945104\\
0.992992992992993	0.0417474531588643\\
0.993993993993994	0.0358196033871708\\
0.994994994994995	0.0298797295794293\\
0.995995995995996	0.0239278317356397\\
0.996996996996997	0.017963909855802\\
0.997997997997998	0.0119879639399156\\
0.998998998998999	0.00599999398798184\\
1	0\\
};
\addlegendentry{$p(\theta_1) = p(\theta_1 \given \text{H})$};

\end{axis}
\end{tikzpicture}%
  \caption{The prior and posterior (given the observation H) of
    $\theta_1$.}
  \label{problem_1_prior_1}
\end{figure}

There are two ways to compute the joint posterior over $(\theta_1,
\theta_2)$: the easy way and the hard way.  The easy way is to use the
sum rule again to write
\begin{align*}
  p(\theta_1, \theta_2 \given \text{H})
  &=
  \Pr(\text{HT})
  p(\theta_1, \theta_2 \given \text{HT})
  +
  \Pr(\text{TH})
  p(\theta_1, \theta_2 \given \text{TH})
  \\
  &=
  \tfrac{1}{2}
  \mc{B}(\theta_1; 3, 2)
  \mc{B}(\theta_2; 2, 3)
  +
  \tfrac{1}{2}
  \mc{B}(\theta_1; 2, 3)
  \mc{B}(\theta_2; 3, 2).
\end{align*}

If we go with the hard way, we begin by computing the joint prior.  By
independence, we have:
\begin{equation*}
  p(\theta_1, \theta_2)
  =
  \mc{B}(\theta_1; 2, 2)
  \mc{B}(\theta_2; 2, 2).
\end{equation*}
To derive the likelihood, we again note that our observation could
have been generated by two mutually independent events: HT or TH.
Given $\theta_1$ and $\theta_2$, the total probability of these
events is:
\begin{equation*}
  \Pr(\text{H} \given \theta_1, \theta_2)
  =
  \theta_1(1 - \theta_2)
  +
  (1 - \theta_1)\theta_2;
\end{equation*}
the first term accounts for HT and the second term for TH.  The
posterior is now:
\begin{align*}
  p(\theta_1, \theta_2 \given \text{H})
  &=
  \tfrac{1}{Z}
  \Pr(\text{H} \given \theta_1, \theta_2)
  p(\theta_1, \theta_2)
  \\
  &=
  \tfrac{1}{Z}
  \bigl(
  \theta_1(1 - \theta_2)
  +
  (1 - \theta_1)\theta_2
  \bigr)
  \mc{B}(\theta_1; 2, 2)
  \mc{B}(\theta_2; 2, 2).
\end{align*}
The normalization constant is
\begin{equation*}
  Z
  =
  \Pr(\text{H})
  =
  \int_0^1
  \int_0^1
  \bigl(
  \theta_1(1 - \theta_2)
  +
  (1 - \theta_1)\theta_2
  \bigr)
  \mc{B}(\theta_1; 2, 2)
  \mc{B}(\theta_2; 2, 2)
  \intd{\theta_1}
  \intd{\theta_2}.
\end{equation*}
This integral is tractable and equals
\nicefrac{1}{2}.\footnote{\url{http://goo.gl/wRofuX}} In fact, this is
always true for any arbitrary mean-\nicefrac{1}{2} beta priors on
$\theta_1, \theta_2$: if our best guess is that each coin is unbiased,
then the outcomes HT/TH always have equal combined probability as the
outcomes HH/TT.

The posterior is now
\begin{equation*}
  p(\theta_1, \theta_2 \given \text{H})
  =
  2
  \bigl(
  \theta_1(1 - \theta_2)
  +
  (1 - \theta_1)\theta_2
  \bigr)
  \mc{B}(\theta_1; 2, 2)
  \mc{B}(\theta_2; 2, 2).
\end{equation*}
This is equivalent to the expression we derived with ``the easy way.''

The prior, likelihood, and posterior are plotted below.  From the
posterior, we can see that joint probabilities corresponding to
jointly low or high values: these combinations would correspond to a
higher probability of seeing either HH or TT observations.  Despite
the marginals for $\theta_1$ and $\theta_2$ remaining unchanged, the H
observation has entangled the previously independent beliefs in the
anticorrelated posterior.

\begin{figure}
  \centering
  % This file was created by matlab2tikz.
% Minimal pgfplots version: 1.3
%
\tikzsetnextfilename{problem_1_joint_prior}
\begin{tikzpicture}

\begin{axis}[%
width=0.825873\squarefigurewidth,
height=\squarefigureheight,
at={(0\squarefigurewidth,0\squarefigureheight)},
scale only axis,
axis on top,
xmin=-0.0005005005005005,
xmax=1.0005005005005,
xlabel={$\theta_1$},
ymin=-0.0005005005005005,
ymax=1.0005005005005,
ytick={  0, 0.2, 0.4, 0.6, 0.8,   1},
ylabel={$\theta_2$},
axis x line*=bottom,
axis y line*=left,
colormap={mymap}{[1pt] rgb(0pt)=(0.0143,0.0143,0.0143); rgb(1pt)=(0.0244084,0.0168792,0.0197143); rgb(2pt)=(0.0343475,0.0194402,0.02529); rgb(3pt)=(0.0440851,0.0219827,0.031023); rgb(4pt)=(0.0535893,0.0245066,0.0369095); rgb(5pt)=(0.062828,0.0270114,0.0429454); rgb(6pt)=(0.071769,0.0294971,0.0491269); rgb(7pt)=(0.0803802,0.0319633,0.05545); rgb(8pt)=(0.0886297,0.0344097,0.0619107); rgb(9pt)=(0.0964852,0.0368363,0.0685051); rgb(10pt)=(0.103915,0.0392426,0.0752293); rgb(11pt)=(0.110886,0.0416284,0.0820793); rgb(12pt)=(0.117368,0.0439936,0.0890511); rgb(13pt)=(0.123327,0.0463378,0.0961409); rgb(14pt)=(0.128732,0.0486608,0.103345); rgb(15pt)=(0.13355,0.0509623,0.110658); rgb(16pt)=(0.13775,0.0532421,0.118078); rgb(17pt)=(0.1413,0.0555,0.1256); rgb(18pt)=(0.144494,0.0577222,0.133297); rgb(19pt)=(0.147635,0.0598991,0.141234); rgb(20pt)=(0.150707,0.0620365,0.149395); rgb(21pt)=(0.15369,0.0641403,0.157763); rgb(22pt)=(0.156566,0.0662164,0.166324); rgb(23pt)=(0.159318,0.0682706,0.175061); rgb(24pt)=(0.161926,0.0703088,0.183957); rgb(25pt)=(0.164372,0.0723369,0.192998); rgb(26pt)=(0.166638,0.0743606,0.202166); rgb(27pt)=(0.168707,0.076386,0.211447); rgb(28pt)=(0.170558,0.0784188,0.220823); rgb(29pt)=(0.172176,0.0804649,0.230279); rgb(30pt)=(0.17354,0.0825301,0.2398); rgb(31pt)=(0.174632,0.0846204,0.249368); rgb(32pt)=(0.175436,0.0867415,0.258968); rgb(33pt)=(0.175931,0.0888995,0.268584); rgb(34pt)=(0.1761,0.0911,0.2782); rgb(35pt)=(0.17608,0.0932893,0.288236); rgb(36pt)=(0.17602,0.0954203,0.299026); rgb(37pt)=(0.175922,0.0975092,0.310428); rgb(38pt)=(0.175786,0.0995723,0.322297); rgb(39pt)=(0.175613,0.101626,0.334492); rgb(40pt)=(0.175404,0.103685,0.346867); rgb(41pt)=(0.17516,0.105768,0.359281); rgb(42pt)=(0.174882,0.107889,0.37159); rgb(43pt)=(0.174571,0.110066,0.383651); rgb(44pt)=(0.174227,0.112313,0.39532); rgb(45pt)=(0.173853,0.114648,0.406454); rgb(46pt)=(0.173448,0.117087,0.41691); rgb(47pt)=(0.173013,0.119645,0.426544); rgb(48pt)=(0.172551,0.122339,0.435213); rgb(49pt)=(0.17206,0.125186,0.442775); rgb(50pt)=(0.171543,0.128201,0.449085); rgb(51pt)=(0.171,0.1314,0.454); rgb(52pt)=(0.170031,0.134957,0.458091); rgb(53pt)=(0.16829,0.139002,0.46202); rgb(54pt)=(0.165863,0.143483,0.465783); rgb(55pt)=(0.162835,0.148348,0.469374); rgb(56pt)=(0.15929,0.153547,0.47279); rgb(57pt)=(0.155314,0.159029,0.476026); rgb(58pt)=(0.150992,0.164743,0.479077); rgb(59pt)=(0.146408,0.170638,0.481938); rgb(60pt)=(0.141647,0.176662,0.484605); rgb(61pt)=(0.136795,0.182765,0.487073); rgb(62pt)=(0.131936,0.188896,0.489338); rgb(63pt)=(0.127156,0.195003,0.491395); rgb(64pt)=(0.122538,0.201036,0.49324); rgb(65pt)=(0.118169,0.206943,0.494867); rgb(66pt)=(0.114133,0.212673,0.496273); rgb(67pt)=(0.110515,0.218176,0.497452); rgb(68pt)=(0.1074,0.2234,0.4984); rgb(69pt)=(0.104662,0.228422,0.49922); rgb(70pt)=(0.102104,0.233358,0.500015); rgb(71pt)=(0.099699,0.238221,0.500783); rgb(72pt)=(0.0974245,0.24302,0.50152); rgb(73pt)=(0.0952557,0.247765,0.502223); rgb(74pt)=(0.0931681,0.252468,0.502888); rgb(75pt)=(0.0911374,0.257138,0.503513); rgb(76pt)=(0.0891393,0.261787,0.504093); rgb(77pt)=(0.0871493,0.266426,0.504626); rgb(78pt)=(0.0851432,0.271063,0.505109); rgb(79pt)=(0.0830965,0.275711,0.505538); rgb(80pt)=(0.0809849,0.280379,0.50591); rgb(81pt)=(0.0787839,0.285079,0.506222); rgb(82pt)=(0.0764694,0.28982,0.50647); rgb(83pt)=(0.0740168,0.294614,0.506651); rgb(84pt)=(0.0714018,0.29947,0.506762); rgb(85pt)=(0.0686,0.3044,0.5068); rgb(86pt)=(0.0653572,0.309404,0.506262); rgb(87pt)=(0.0615042,0.314469,0.504714); rgb(88pt)=(0.0571445,0.319589,0.502254); rgb(89pt)=(0.0523815,0.324755,0.498978); rgb(90pt)=(0.0473184,0.32996,0.494986); rgb(91pt)=(0.0420588,0.335197,0.490374); rgb(92pt)=(0.036706,0.340457,0.48524); rgb(93pt)=(0.0313634,0.345734,0.479682); rgb(94pt)=(0.0261344,0.351019,0.473798); rgb(95pt)=(0.0211225,0.356305,0.467685); rgb(96pt)=(0.0164309,0.361585,0.461442); rgb(97pt)=(0.0121631,0.366851,0.455165); rgb(98pt)=(0.00842253,0.372095,0.448952); rgb(99pt)=(0.00531253,0.377309,0.442902); rgb(100pt)=(0.00293651,0.382486,0.437111); rgb(101pt)=(0.00139787,0.387619,0.431678); rgb(102pt)=(0.0008,0.3927,0.4267); rgb(103pt)=(0.000709623,0.397742,0.42201); rgb(104pt)=(0.000624546,0.402766,0.417367); rgb(105pt)=(0.000544794,0.407771,0.412761); rgb(106pt)=(0.000470387,0.412759,0.408183); rgb(107pt)=(0.00040135,0.417729,0.403621); rgb(108pt)=(0.000337705,0.422684,0.399065); rgb(109pt)=(0.000279475,0.427622,0.394505); rgb(110pt)=(0.000226682,0.432546,0.38993); rgb(111pt)=(0.00017935,0.437456,0.385331); rgb(112pt)=(0.000137501,0.442352,0.380696); rgb(113pt)=(0.000101158,0.447235,0.376016); rgb(114pt)=(7.03435e-05,0.452106,0.37128); rgb(115pt)=(4.50806e-05,0.456965,0.366478); rgb(116pt)=(2.5392e-05,0.461813,0.361599); rgb(117pt)=(1.13005e-05,0.466652,0.356634); rgb(118pt)=(2.82893e-06,0.47148,0.351571); rgb(119pt)=(0,0.4763,0.3464); rgb(120pt)=(0,0.481102,0.341072); rgb(121pt)=(0,0.485879,0.335558); rgb(122pt)=(0,0.490634,0.329884); rgb(123pt)=(0,0.49537,0.324072); rgb(124pt)=(0,0.50009,0.318147); rgb(125pt)=(0,0.504796,0.312132); rgb(126pt)=(0,0.509492,0.306052); rgb(127pt)=(0,0.514179,0.29993); rgb(128pt)=(0,0.518862,0.293791); rgb(129pt)=(0,0.523544,0.287657); rgb(130pt)=(0,0.528226,0.281554); rgb(131pt)=(0,0.532912,0.275504); rgb(132pt)=(0,0.537604,0.269532); rgb(133pt)=(0,0.542307,0.263662); rgb(134pt)=(0,0.547021,0.257917); rgb(135pt)=(0,0.551752,0.252322); rgb(136pt)=(0,0.5565,0.2469); rgb(137pt)=(0,0.561307,0.241711); rgb(138pt)=(0,0.5662,0.236772); rgb(139pt)=(0,0.571161,0.232039); rgb(140pt)=(0,0.576172,0.22747); rgb(141pt)=(0,0.581218,0.223024); rgb(142pt)=(0,0.58628,0.218657); rgb(143pt)=(0,0.591341,0.214328); rgb(144pt)=(0,0.596384,0.209994); rgb(145pt)=(0,0.601392,0.205613); rgb(146pt)=(0,0.606347,0.201143); rgb(147pt)=(0,0.611231,0.196542); rgb(148pt)=(0,0.616029,0.191766); rgb(149pt)=(0,0.620721,0.186774); rgb(150pt)=(0,0.625292,0.181524); rgb(151pt)=(0,0.629724,0.175973); rgb(152pt)=(0,0.633999,0.170079); rgb(153pt)=(0,0.6381,0.1638); rgb(154pt)=(0.00153448,0.642042,0.156421); rgb(155pt)=(0.00594177,0.645859,0.147433); rgb(156pt)=(0.0129276,0.649563,0.137101); rgb(157pt)=(0.0221978,0.653167,0.125691); rgb(158pt)=(0.033458,0.656683,0.113468); rgb(159pt)=(0.046414,0.660125,0.100697); rgb(160pt)=(0.0607716,0.663503,0.0876444); rgb(161pt)=(0.0762365,0.666832,0.0745752); rgb(162pt)=(0.0925145,0.670123,0.0617548); rgb(163pt)=(0.109311,0.673389,0.0494487); rgb(164pt)=(0.126333,0.676643,0.0379222); rgb(165pt)=(0.143285,0.679896,0.0274408); rgb(166pt)=(0.159872,0.683162,0.0182699); rgb(167pt)=(0.175802,0.686453,0.010675); rgb(168pt)=(0.19078,0.689781,0.00492137); rgb(169pt)=(0.20451,0.693159,0.00127458); rgb(170pt)=(0.2167,0.6966,0); rgb(171pt)=(0.227739,0.700166,0); rgb(172pt)=(0.238267,0.703889,0); rgb(173pt)=(0.248364,0.707737,0); rgb(174pt)=(0.258112,0.711676,0); rgb(175pt)=(0.267591,0.715675,0); rgb(176pt)=(0.276883,0.7197,0); rgb(177pt)=(0.286067,0.723718,0); rgb(178pt)=(0.295225,0.727696,0); rgb(179pt)=(0.304438,0.731602,0); rgb(180pt)=(0.313786,0.735402,0); rgb(181pt)=(0.32335,0.739065,0); rgb(182pt)=(0.333211,0.742556,0); rgb(183pt)=(0.34345,0.745844,0); rgb(184pt)=(0.354148,0.748895,0); rgb(185pt)=(0.365385,0.751677,0); rgb(186pt)=(0.377242,0.754156,0); rgb(187pt)=(0.3898,0.7563,0); rgb(188pt)=(0.403574,0.758195,0); rgb(189pt)=(0.418885,0.759957,0); rgb(190pt)=(0.435525,0.761599,0); rgb(191pt)=(0.453286,0.763136,0); rgb(192pt)=(0.471958,0.764579,0); rgb(193pt)=(0.491333,0.765943,0); rgb(194pt)=(0.511203,0.767241,0); rgb(195pt)=(0.531358,0.768485,0); rgb(196pt)=(0.551591,0.76969,0); rgb(197pt)=(0.571692,0.770869,0); rgb(198pt)=(0.591453,0.772035,0); rgb(199pt)=(0.610666,0.773201,0); rgb(200pt)=(0.629122,0.77438,0); rgb(201pt)=(0.646612,0.775587,0); rgb(202pt)=(0.662927,0.776833,0); rgb(203pt)=(0.677859,0.778133,0); rgb(204pt)=(0.6912,0.7795,0); rgb(205pt)=(0.70344,0.780892,0.00343064); rgb(206pt)=(0.715224,0.782261,0.0132589); rgb(207pt)=(0.72658,0.783614,0.0287895); rgb(208pt)=(0.737533,0.784956,0.0493267); rgb(209pt)=(0.748107,0.786293,0.0741753); rgb(210pt)=(0.758328,0.787632,0.10264); rgb(211pt)=(0.768223,0.788977,0.134025); rgb(212pt)=(0.777816,0.790335,0.167635); rgb(213pt)=(0.787133,0.791712,0.202774); rgb(214pt)=(0.7962,0.793114,0.238748); rgb(215pt)=(0.805042,0.794545,0.27486); rgb(216pt)=(0.813684,0.796013,0.310415); rgb(217pt)=(0.822153,0.797523,0.344718); rgb(218pt)=(0.830473,0.799081,0.377074); rgb(219pt)=(0.838671,0.800692,0.406786); rgb(220pt)=(0.846771,0.802363,0.43316); rgb(221pt)=(0.8548,0.8041,0.4555); rgb(222pt)=(0.863049,0.805883,0.475346); rgb(223pt)=(0.871713,0.807693,0.494702); rgb(224pt)=(0.880675,0.809539,0.513574); rgb(225pt)=(0.889816,0.811426,0.531964); rgb(226pt)=(0.899019,0.813361,0.549875); rgb(227pt)=(0.908164,0.815352,0.567311); rgb(228pt)=(0.917134,0.817406,0.584275); rgb(229pt)=(0.925811,0.819529,0.600771); rgb(230pt)=(0.934077,0.821728,0.616801); rgb(231pt)=(0.941812,0.824011,0.63237); rgb(232pt)=(0.9489,0.826383,0.64748); rgb(233pt)=(0.955221,0.828853,0.662135); rgb(234pt)=(0.960658,0.831427,0.676338); rgb(235pt)=(0.965093,0.834111,0.690093); rgb(236pt)=(0.968407,0.836914,0.703402); rgb(237pt)=(0.970482,0.839841,0.71627); rgb(238pt)=(0.9712,0.8429,0.7287); rgb(239pt)=(0.9712,0.846163,0.740814); rgb(240pt)=(0.971197,0.84969,0.752722); rgb(241pt)=(0.971189,0.853471,0.764409); rgb(242pt)=(0.971174,0.857495,0.775859); rgb(243pt)=(0.971149,0.861753,0.787056); rgb(244pt)=(0.971112,0.866234,0.797986); rgb(245pt)=(0.97106,0.870928,0.808633); rgb(246pt)=(0.970992,0.875824,0.81898); rgb(247pt)=(0.970903,0.880913,0.829014); rgb(248pt)=(0.970793,0.886183,0.838717); rgb(249pt)=(0.970658,0.891625,0.848075); rgb(250pt)=(0.970497,0.897228,0.857072); rgb(251pt)=(0.970306,0.902981,0.865692); rgb(252pt)=(0.970083,0.908876,0.873921); rgb(253pt)=(0.969826,0.914901,0.881742); rgb(254pt)=(0.969533,0.921046,0.88914); rgb(255pt)=(0.9692,0.9273,0.8961)},
colorbar,
colorbar style={separate axis lines},
point meta min=0,
point meta max=2.24999549098874
]
\addplot [forget plot] graphics [xmin=-0.0005005005005005,xmax=1.0005005005005,ymin=-0.0005005005005005,ymax=1.0005005005005] {problem_1_joint_prior-1.png};
\end{axis}
\end{tikzpicture}%
  \caption{The joint prior $p(\theta_1, \theta_2)$.}
  \label{problem_joint_prior}
\end{figure}

\begin{figure}
  \centering
  % This file was created by matlab2tikz.
% Minimal pgfplots version: 1.3
%
\tikzsetnextfilename{problem_1_joint_likelihood}
\begin{tikzpicture}

\begin{axis}[%
width=0.825873\squarefigurewidth,
height=\squarefigureheight,
at={(0\squarefigurewidth,0\squarefigureheight)},
scale only axis,
axis on top,
xmin=-0.0005005005005005,
xmax=1.0005005005005,
xlabel={$\theta_1$},
ymin=-0.0005005005005005,
ymax=1.0005005005005,
ytick={  0, 0.2, 0.4, 0.6, 0.8,   1},
ylabel={$\theta_2$},
axis x line*=bottom,
axis y line*=left,
colormap={mymap}{[1pt] rgb(0pt)=(0.0143,0.0143,0.0143); rgb(1pt)=(0.0244084,0.0168792,0.0197143); rgb(2pt)=(0.0343475,0.0194402,0.02529); rgb(3pt)=(0.0440851,0.0219827,0.031023); rgb(4pt)=(0.0535893,0.0245066,0.0369095); rgb(5pt)=(0.062828,0.0270114,0.0429454); rgb(6pt)=(0.071769,0.0294971,0.0491269); rgb(7pt)=(0.0803802,0.0319633,0.05545); rgb(8pt)=(0.0886297,0.0344097,0.0619107); rgb(9pt)=(0.0964852,0.0368363,0.0685051); rgb(10pt)=(0.103915,0.0392426,0.0752293); rgb(11pt)=(0.110886,0.0416284,0.0820793); rgb(12pt)=(0.117368,0.0439936,0.0890511); rgb(13pt)=(0.123327,0.0463378,0.0961409); rgb(14pt)=(0.128732,0.0486608,0.103345); rgb(15pt)=(0.13355,0.0509623,0.110658); rgb(16pt)=(0.13775,0.0532421,0.118078); rgb(17pt)=(0.1413,0.0555,0.1256); rgb(18pt)=(0.144494,0.0577222,0.133297); rgb(19pt)=(0.147635,0.0598991,0.141234); rgb(20pt)=(0.150707,0.0620365,0.149395); rgb(21pt)=(0.15369,0.0641403,0.157763); rgb(22pt)=(0.156566,0.0662164,0.166324); rgb(23pt)=(0.159318,0.0682706,0.175061); rgb(24pt)=(0.161926,0.0703088,0.183957); rgb(25pt)=(0.164372,0.0723369,0.192998); rgb(26pt)=(0.166638,0.0743606,0.202166); rgb(27pt)=(0.168707,0.076386,0.211447); rgb(28pt)=(0.170558,0.0784188,0.220823); rgb(29pt)=(0.172176,0.0804649,0.230279); rgb(30pt)=(0.17354,0.0825301,0.2398); rgb(31pt)=(0.174632,0.0846204,0.249368); rgb(32pt)=(0.175436,0.0867415,0.258968); rgb(33pt)=(0.175931,0.0888995,0.268584); rgb(34pt)=(0.1761,0.0911,0.2782); rgb(35pt)=(0.17608,0.0932893,0.288236); rgb(36pt)=(0.17602,0.0954203,0.299026); rgb(37pt)=(0.175922,0.0975092,0.310428); rgb(38pt)=(0.175786,0.0995723,0.322297); rgb(39pt)=(0.175613,0.101626,0.334492); rgb(40pt)=(0.175404,0.103685,0.346867); rgb(41pt)=(0.17516,0.105768,0.359281); rgb(42pt)=(0.174882,0.107889,0.37159); rgb(43pt)=(0.174571,0.110066,0.383651); rgb(44pt)=(0.174227,0.112313,0.39532); rgb(45pt)=(0.173853,0.114648,0.406454); rgb(46pt)=(0.173448,0.117087,0.41691); rgb(47pt)=(0.173013,0.119645,0.426544); rgb(48pt)=(0.172551,0.122339,0.435213); rgb(49pt)=(0.17206,0.125186,0.442775); rgb(50pt)=(0.171543,0.128201,0.449085); rgb(51pt)=(0.171,0.1314,0.454); rgb(52pt)=(0.170031,0.134957,0.458091); rgb(53pt)=(0.16829,0.139002,0.46202); rgb(54pt)=(0.165863,0.143483,0.465783); rgb(55pt)=(0.162835,0.148348,0.469374); rgb(56pt)=(0.15929,0.153547,0.47279); rgb(57pt)=(0.155314,0.159029,0.476026); rgb(58pt)=(0.150992,0.164743,0.479077); rgb(59pt)=(0.146408,0.170638,0.481938); rgb(60pt)=(0.141647,0.176662,0.484605); rgb(61pt)=(0.136795,0.182765,0.487073); rgb(62pt)=(0.131936,0.188896,0.489338); rgb(63pt)=(0.127156,0.195003,0.491395); rgb(64pt)=(0.122538,0.201036,0.49324); rgb(65pt)=(0.118169,0.206943,0.494867); rgb(66pt)=(0.114133,0.212673,0.496273); rgb(67pt)=(0.110515,0.218176,0.497452); rgb(68pt)=(0.1074,0.2234,0.4984); rgb(69pt)=(0.104662,0.228422,0.49922); rgb(70pt)=(0.102104,0.233358,0.500015); rgb(71pt)=(0.099699,0.238221,0.500783); rgb(72pt)=(0.0974245,0.24302,0.50152); rgb(73pt)=(0.0952557,0.247765,0.502223); rgb(74pt)=(0.0931681,0.252468,0.502888); rgb(75pt)=(0.0911374,0.257138,0.503513); rgb(76pt)=(0.0891393,0.261787,0.504093); rgb(77pt)=(0.0871493,0.266426,0.504626); rgb(78pt)=(0.0851432,0.271063,0.505109); rgb(79pt)=(0.0830965,0.275711,0.505538); rgb(80pt)=(0.0809849,0.280379,0.50591); rgb(81pt)=(0.0787839,0.285079,0.506222); rgb(82pt)=(0.0764694,0.28982,0.50647); rgb(83pt)=(0.0740168,0.294614,0.506651); rgb(84pt)=(0.0714018,0.29947,0.506762); rgb(85pt)=(0.0686,0.3044,0.5068); rgb(86pt)=(0.0653572,0.309404,0.506262); rgb(87pt)=(0.0615042,0.314469,0.504714); rgb(88pt)=(0.0571445,0.319589,0.502254); rgb(89pt)=(0.0523815,0.324755,0.498978); rgb(90pt)=(0.0473184,0.32996,0.494986); rgb(91pt)=(0.0420588,0.335197,0.490374); rgb(92pt)=(0.036706,0.340457,0.48524); rgb(93pt)=(0.0313634,0.345734,0.479682); rgb(94pt)=(0.0261344,0.351019,0.473798); rgb(95pt)=(0.0211225,0.356305,0.467685); rgb(96pt)=(0.0164309,0.361585,0.461442); rgb(97pt)=(0.0121631,0.366851,0.455165); rgb(98pt)=(0.00842253,0.372095,0.448952); rgb(99pt)=(0.00531253,0.377309,0.442902); rgb(100pt)=(0.00293651,0.382486,0.437111); rgb(101pt)=(0.00139787,0.387619,0.431678); rgb(102pt)=(0.0008,0.3927,0.4267); rgb(103pt)=(0.000709623,0.397742,0.42201); rgb(104pt)=(0.000624546,0.402766,0.417367); rgb(105pt)=(0.000544794,0.407771,0.412761); rgb(106pt)=(0.000470387,0.412759,0.408183); rgb(107pt)=(0.00040135,0.417729,0.403621); rgb(108pt)=(0.000337705,0.422684,0.399065); rgb(109pt)=(0.000279475,0.427622,0.394505); rgb(110pt)=(0.000226682,0.432546,0.38993); rgb(111pt)=(0.00017935,0.437456,0.385331); rgb(112pt)=(0.000137501,0.442352,0.380696); rgb(113pt)=(0.000101158,0.447235,0.376016); rgb(114pt)=(7.03435e-05,0.452106,0.37128); rgb(115pt)=(4.50806e-05,0.456965,0.366478); rgb(116pt)=(2.5392e-05,0.461813,0.361599); rgb(117pt)=(1.13005e-05,0.466652,0.356634); rgb(118pt)=(2.82893e-06,0.47148,0.351571); rgb(119pt)=(0,0.4763,0.3464); rgb(120pt)=(0,0.481102,0.341072); rgb(121pt)=(0,0.485879,0.335558); rgb(122pt)=(0,0.490634,0.329884); rgb(123pt)=(0,0.49537,0.324072); rgb(124pt)=(0,0.50009,0.318147); rgb(125pt)=(0,0.504796,0.312132); rgb(126pt)=(0,0.509492,0.306052); rgb(127pt)=(0,0.514179,0.29993); rgb(128pt)=(0,0.518862,0.293791); rgb(129pt)=(0,0.523544,0.287657); rgb(130pt)=(0,0.528226,0.281554); rgb(131pt)=(0,0.532912,0.275504); rgb(132pt)=(0,0.537604,0.269532); rgb(133pt)=(0,0.542307,0.263662); rgb(134pt)=(0,0.547021,0.257917); rgb(135pt)=(0,0.551752,0.252322); rgb(136pt)=(0,0.5565,0.2469); rgb(137pt)=(0,0.561307,0.241711); rgb(138pt)=(0,0.5662,0.236772); rgb(139pt)=(0,0.571161,0.232039); rgb(140pt)=(0,0.576172,0.22747); rgb(141pt)=(0,0.581218,0.223024); rgb(142pt)=(0,0.58628,0.218657); rgb(143pt)=(0,0.591341,0.214328); rgb(144pt)=(0,0.596384,0.209994); rgb(145pt)=(0,0.601392,0.205613); rgb(146pt)=(0,0.606347,0.201143); rgb(147pt)=(0,0.611231,0.196542); rgb(148pt)=(0,0.616029,0.191766); rgb(149pt)=(0,0.620721,0.186774); rgb(150pt)=(0,0.625292,0.181524); rgb(151pt)=(0,0.629724,0.175973); rgb(152pt)=(0,0.633999,0.170079); rgb(153pt)=(0,0.6381,0.1638); rgb(154pt)=(0.00153448,0.642042,0.156421); rgb(155pt)=(0.00594177,0.645859,0.147433); rgb(156pt)=(0.0129276,0.649563,0.137101); rgb(157pt)=(0.0221978,0.653167,0.125691); rgb(158pt)=(0.033458,0.656683,0.113468); rgb(159pt)=(0.046414,0.660125,0.100697); rgb(160pt)=(0.0607716,0.663503,0.0876444); rgb(161pt)=(0.0762365,0.666832,0.0745752); rgb(162pt)=(0.0925145,0.670123,0.0617548); rgb(163pt)=(0.109311,0.673389,0.0494487); rgb(164pt)=(0.126333,0.676643,0.0379222); rgb(165pt)=(0.143285,0.679896,0.0274408); rgb(166pt)=(0.159872,0.683162,0.0182699); rgb(167pt)=(0.175802,0.686453,0.010675); rgb(168pt)=(0.19078,0.689781,0.00492137); rgb(169pt)=(0.20451,0.693159,0.00127458); rgb(170pt)=(0.2167,0.6966,0); rgb(171pt)=(0.227739,0.700166,0); rgb(172pt)=(0.238267,0.703889,0); rgb(173pt)=(0.248364,0.707737,0); rgb(174pt)=(0.258112,0.711676,0); rgb(175pt)=(0.267591,0.715675,0); rgb(176pt)=(0.276883,0.7197,0); rgb(177pt)=(0.286067,0.723718,0); rgb(178pt)=(0.295225,0.727696,0); rgb(179pt)=(0.304438,0.731602,0); rgb(180pt)=(0.313786,0.735402,0); rgb(181pt)=(0.32335,0.739065,0); rgb(182pt)=(0.333211,0.742556,0); rgb(183pt)=(0.34345,0.745844,0); rgb(184pt)=(0.354148,0.748895,0); rgb(185pt)=(0.365385,0.751677,0); rgb(186pt)=(0.377242,0.754156,0); rgb(187pt)=(0.3898,0.7563,0); rgb(188pt)=(0.403574,0.758195,0); rgb(189pt)=(0.418885,0.759957,0); rgb(190pt)=(0.435525,0.761599,0); rgb(191pt)=(0.453286,0.763136,0); rgb(192pt)=(0.471958,0.764579,0); rgb(193pt)=(0.491333,0.765943,0); rgb(194pt)=(0.511203,0.767241,0); rgb(195pt)=(0.531358,0.768485,0); rgb(196pt)=(0.551591,0.76969,0); rgb(197pt)=(0.571692,0.770869,0); rgb(198pt)=(0.591453,0.772035,0); rgb(199pt)=(0.610666,0.773201,0); rgb(200pt)=(0.629122,0.77438,0); rgb(201pt)=(0.646612,0.775587,0); rgb(202pt)=(0.662927,0.776833,0); rgb(203pt)=(0.677859,0.778133,0); rgb(204pt)=(0.6912,0.7795,0); rgb(205pt)=(0.70344,0.780892,0.00343064); rgb(206pt)=(0.715224,0.782261,0.0132589); rgb(207pt)=(0.72658,0.783614,0.0287895); rgb(208pt)=(0.737533,0.784956,0.0493267); rgb(209pt)=(0.748107,0.786293,0.0741753); rgb(210pt)=(0.758328,0.787632,0.10264); rgb(211pt)=(0.768223,0.788977,0.134025); rgb(212pt)=(0.777816,0.790335,0.167635); rgb(213pt)=(0.787133,0.791712,0.202774); rgb(214pt)=(0.7962,0.793114,0.238748); rgb(215pt)=(0.805042,0.794545,0.27486); rgb(216pt)=(0.813684,0.796013,0.310415); rgb(217pt)=(0.822153,0.797523,0.344718); rgb(218pt)=(0.830473,0.799081,0.377074); rgb(219pt)=(0.838671,0.800692,0.406786); rgb(220pt)=(0.846771,0.802363,0.43316); rgb(221pt)=(0.8548,0.8041,0.4555); rgb(222pt)=(0.863049,0.805883,0.475346); rgb(223pt)=(0.871713,0.807693,0.494702); rgb(224pt)=(0.880675,0.809539,0.513574); rgb(225pt)=(0.889816,0.811426,0.531964); rgb(226pt)=(0.899019,0.813361,0.549875); rgb(227pt)=(0.908164,0.815352,0.567311); rgb(228pt)=(0.917134,0.817406,0.584275); rgb(229pt)=(0.925811,0.819529,0.600771); rgb(230pt)=(0.934077,0.821728,0.616801); rgb(231pt)=(0.941812,0.824011,0.63237); rgb(232pt)=(0.9489,0.826383,0.64748); rgb(233pt)=(0.955221,0.828853,0.662135); rgb(234pt)=(0.960658,0.831427,0.676338); rgb(235pt)=(0.965093,0.834111,0.690093); rgb(236pt)=(0.968407,0.836914,0.703402); rgb(237pt)=(0.970482,0.839841,0.71627); rgb(238pt)=(0.9712,0.8429,0.7287); rgb(239pt)=(0.9712,0.846163,0.740814); rgb(240pt)=(0.971197,0.84969,0.752722); rgb(241pt)=(0.971189,0.853471,0.764409); rgb(242pt)=(0.971174,0.857495,0.775859); rgb(243pt)=(0.971149,0.861753,0.787056); rgb(244pt)=(0.971112,0.866234,0.797986); rgb(245pt)=(0.97106,0.870928,0.808633); rgb(246pt)=(0.970992,0.875824,0.81898); rgb(247pt)=(0.970903,0.880913,0.829014); rgb(248pt)=(0.970793,0.886183,0.838717); rgb(249pt)=(0.970658,0.891625,0.848075); rgb(250pt)=(0.970497,0.897228,0.857072); rgb(251pt)=(0.970306,0.902981,0.865692); rgb(252pt)=(0.970083,0.908876,0.873921); rgb(253pt)=(0.969826,0.914901,0.881742); rgb(254pt)=(0.969533,0.921046,0.88914); rgb(255pt)=(0.9692,0.9273,0.8961)},
colorbar,
colorbar style={separate axis lines},
point meta min=0,
point meta max=1
]
\addplot [forget plot] graphics [xmin=-0.0005005005005005,xmax=1.0005005005005,ymin=-0.0005005005005005,ymax=1.0005005005005] {problem_1_joint_likelihood-1.png};
\end{axis}
\end{tikzpicture}%
  \caption{The joint likelihood $p(\text{H} \given \theta_1, \theta_2)$.}
  \label{problem_joint_likelihood}
\end{figure}

\begin{figure}
  \centering
  % This file was created by matlab2tikz.
% Minimal pgfplots version: 1.3
%
\tikzsetnextfilename{problem_1_joint_posterior}
\begin{tikzpicture}

\begin{axis}[%
width=0.825873\squarefigurewidth,
height=\squarefigureheight,
at={(0\squarefigurewidth,0\squarefigureheight)},
scale only axis,
axis on top,
xmin=-0.0005005005005005,
xmax=1.0005005005005,
xlabel={$\theta_1$},
ymin=-0.0005005005005005,
ymax=1.0005005005005,
ytick={  0, 0.2, 0.4, 0.6, 0.8,   1},
ylabel={$\theta_2$},
axis x line*=bottom,
axis y line*=left,
colormap={mymap}{[1pt] rgb(0pt)=(0.0143,0.0143,0.0143); rgb(1pt)=(0.0244084,0.0168792,0.0197143); rgb(2pt)=(0.0343475,0.0194402,0.02529); rgb(3pt)=(0.0440851,0.0219827,0.031023); rgb(4pt)=(0.0535893,0.0245066,0.0369095); rgb(5pt)=(0.062828,0.0270114,0.0429454); rgb(6pt)=(0.071769,0.0294971,0.0491269); rgb(7pt)=(0.0803802,0.0319633,0.05545); rgb(8pt)=(0.0886297,0.0344097,0.0619107); rgb(9pt)=(0.0964852,0.0368363,0.0685051); rgb(10pt)=(0.103915,0.0392426,0.0752293); rgb(11pt)=(0.110886,0.0416284,0.0820793); rgb(12pt)=(0.117368,0.0439936,0.0890511); rgb(13pt)=(0.123327,0.0463378,0.0961409); rgb(14pt)=(0.128732,0.0486608,0.103345); rgb(15pt)=(0.13355,0.0509623,0.110658); rgb(16pt)=(0.13775,0.0532421,0.118078); rgb(17pt)=(0.1413,0.0555,0.1256); rgb(18pt)=(0.144494,0.0577222,0.133297); rgb(19pt)=(0.147635,0.0598991,0.141234); rgb(20pt)=(0.150707,0.0620365,0.149395); rgb(21pt)=(0.15369,0.0641403,0.157763); rgb(22pt)=(0.156566,0.0662164,0.166324); rgb(23pt)=(0.159318,0.0682706,0.175061); rgb(24pt)=(0.161926,0.0703088,0.183957); rgb(25pt)=(0.164372,0.0723369,0.192998); rgb(26pt)=(0.166638,0.0743606,0.202166); rgb(27pt)=(0.168707,0.076386,0.211447); rgb(28pt)=(0.170558,0.0784188,0.220823); rgb(29pt)=(0.172176,0.0804649,0.230279); rgb(30pt)=(0.17354,0.0825301,0.2398); rgb(31pt)=(0.174632,0.0846204,0.249368); rgb(32pt)=(0.175436,0.0867415,0.258968); rgb(33pt)=(0.175931,0.0888995,0.268584); rgb(34pt)=(0.1761,0.0911,0.2782); rgb(35pt)=(0.17608,0.0932893,0.288236); rgb(36pt)=(0.17602,0.0954203,0.299026); rgb(37pt)=(0.175922,0.0975092,0.310428); rgb(38pt)=(0.175786,0.0995723,0.322297); rgb(39pt)=(0.175613,0.101626,0.334492); rgb(40pt)=(0.175404,0.103685,0.346867); rgb(41pt)=(0.17516,0.105768,0.359281); rgb(42pt)=(0.174882,0.107889,0.37159); rgb(43pt)=(0.174571,0.110066,0.383651); rgb(44pt)=(0.174227,0.112313,0.39532); rgb(45pt)=(0.173853,0.114648,0.406454); rgb(46pt)=(0.173448,0.117087,0.41691); rgb(47pt)=(0.173013,0.119645,0.426544); rgb(48pt)=(0.172551,0.122339,0.435213); rgb(49pt)=(0.17206,0.125186,0.442775); rgb(50pt)=(0.171543,0.128201,0.449085); rgb(51pt)=(0.171,0.1314,0.454); rgb(52pt)=(0.170031,0.134957,0.458091); rgb(53pt)=(0.16829,0.139002,0.46202); rgb(54pt)=(0.165863,0.143483,0.465783); rgb(55pt)=(0.162835,0.148348,0.469374); rgb(56pt)=(0.15929,0.153547,0.47279); rgb(57pt)=(0.155314,0.159029,0.476026); rgb(58pt)=(0.150992,0.164743,0.479077); rgb(59pt)=(0.146408,0.170638,0.481938); rgb(60pt)=(0.141647,0.176662,0.484605); rgb(61pt)=(0.136795,0.182765,0.487073); rgb(62pt)=(0.131936,0.188896,0.489338); rgb(63pt)=(0.127156,0.195003,0.491395); rgb(64pt)=(0.122538,0.201036,0.49324); rgb(65pt)=(0.118169,0.206943,0.494867); rgb(66pt)=(0.114133,0.212673,0.496273); rgb(67pt)=(0.110515,0.218176,0.497452); rgb(68pt)=(0.1074,0.2234,0.4984); rgb(69pt)=(0.104662,0.228422,0.49922); rgb(70pt)=(0.102104,0.233358,0.500015); rgb(71pt)=(0.099699,0.238221,0.500783); rgb(72pt)=(0.0974245,0.24302,0.50152); rgb(73pt)=(0.0952557,0.247765,0.502223); rgb(74pt)=(0.0931681,0.252468,0.502888); rgb(75pt)=(0.0911374,0.257138,0.503513); rgb(76pt)=(0.0891393,0.261787,0.504093); rgb(77pt)=(0.0871493,0.266426,0.504626); rgb(78pt)=(0.0851432,0.271063,0.505109); rgb(79pt)=(0.0830965,0.275711,0.505538); rgb(80pt)=(0.0809849,0.280379,0.50591); rgb(81pt)=(0.0787839,0.285079,0.506222); rgb(82pt)=(0.0764694,0.28982,0.50647); rgb(83pt)=(0.0740168,0.294614,0.506651); rgb(84pt)=(0.0714018,0.29947,0.506762); rgb(85pt)=(0.0686,0.3044,0.5068); rgb(86pt)=(0.0653572,0.309404,0.506262); rgb(87pt)=(0.0615042,0.314469,0.504714); rgb(88pt)=(0.0571445,0.319589,0.502254); rgb(89pt)=(0.0523815,0.324755,0.498978); rgb(90pt)=(0.0473184,0.32996,0.494986); rgb(91pt)=(0.0420588,0.335197,0.490374); rgb(92pt)=(0.036706,0.340457,0.48524); rgb(93pt)=(0.0313634,0.345734,0.479682); rgb(94pt)=(0.0261344,0.351019,0.473798); rgb(95pt)=(0.0211225,0.356305,0.467685); rgb(96pt)=(0.0164309,0.361585,0.461442); rgb(97pt)=(0.0121631,0.366851,0.455165); rgb(98pt)=(0.00842253,0.372095,0.448952); rgb(99pt)=(0.00531253,0.377309,0.442902); rgb(100pt)=(0.00293651,0.382486,0.437111); rgb(101pt)=(0.00139787,0.387619,0.431678); rgb(102pt)=(0.0008,0.3927,0.4267); rgb(103pt)=(0.000709623,0.397742,0.42201); rgb(104pt)=(0.000624546,0.402766,0.417367); rgb(105pt)=(0.000544794,0.407771,0.412761); rgb(106pt)=(0.000470387,0.412759,0.408183); rgb(107pt)=(0.00040135,0.417729,0.403621); rgb(108pt)=(0.000337705,0.422684,0.399065); rgb(109pt)=(0.000279475,0.427622,0.394505); rgb(110pt)=(0.000226682,0.432546,0.38993); rgb(111pt)=(0.00017935,0.437456,0.385331); rgb(112pt)=(0.000137501,0.442352,0.380696); rgb(113pt)=(0.000101158,0.447235,0.376016); rgb(114pt)=(7.03435e-05,0.452106,0.37128); rgb(115pt)=(4.50806e-05,0.456965,0.366478); rgb(116pt)=(2.5392e-05,0.461813,0.361599); rgb(117pt)=(1.13005e-05,0.466652,0.356634); rgb(118pt)=(2.82893e-06,0.47148,0.351571); rgb(119pt)=(0,0.4763,0.3464); rgb(120pt)=(0,0.481102,0.341072); rgb(121pt)=(0,0.485879,0.335558); rgb(122pt)=(0,0.490634,0.329884); rgb(123pt)=(0,0.49537,0.324072); rgb(124pt)=(0,0.50009,0.318147); rgb(125pt)=(0,0.504796,0.312132); rgb(126pt)=(0,0.509492,0.306052); rgb(127pt)=(0,0.514179,0.29993); rgb(128pt)=(0,0.518862,0.293791); rgb(129pt)=(0,0.523544,0.287657); rgb(130pt)=(0,0.528226,0.281554); rgb(131pt)=(0,0.532912,0.275504); rgb(132pt)=(0,0.537604,0.269532); rgb(133pt)=(0,0.542307,0.263662); rgb(134pt)=(0,0.547021,0.257917); rgb(135pt)=(0,0.551752,0.252322); rgb(136pt)=(0,0.5565,0.2469); rgb(137pt)=(0,0.561307,0.241711); rgb(138pt)=(0,0.5662,0.236772); rgb(139pt)=(0,0.571161,0.232039); rgb(140pt)=(0,0.576172,0.22747); rgb(141pt)=(0,0.581218,0.223024); rgb(142pt)=(0,0.58628,0.218657); rgb(143pt)=(0,0.591341,0.214328); rgb(144pt)=(0,0.596384,0.209994); rgb(145pt)=(0,0.601392,0.205613); rgb(146pt)=(0,0.606347,0.201143); rgb(147pt)=(0,0.611231,0.196542); rgb(148pt)=(0,0.616029,0.191766); rgb(149pt)=(0,0.620721,0.186774); rgb(150pt)=(0,0.625292,0.181524); rgb(151pt)=(0,0.629724,0.175973); rgb(152pt)=(0,0.633999,0.170079); rgb(153pt)=(0,0.6381,0.1638); rgb(154pt)=(0.00153448,0.642042,0.156421); rgb(155pt)=(0.00594177,0.645859,0.147433); rgb(156pt)=(0.0129276,0.649563,0.137101); rgb(157pt)=(0.0221978,0.653167,0.125691); rgb(158pt)=(0.033458,0.656683,0.113468); rgb(159pt)=(0.046414,0.660125,0.100697); rgb(160pt)=(0.0607716,0.663503,0.0876444); rgb(161pt)=(0.0762365,0.666832,0.0745752); rgb(162pt)=(0.0925145,0.670123,0.0617548); rgb(163pt)=(0.109311,0.673389,0.0494487); rgb(164pt)=(0.126333,0.676643,0.0379222); rgb(165pt)=(0.143285,0.679896,0.0274408); rgb(166pt)=(0.159872,0.683162,0.0182699); rgb(167pt)=(0.175802,0.686453,0.010675); rgb(168pt)=(0.19078,0.689781,0.00492137); rgb(169pt)=(0.20451,0.693159,0.00127458); rgb(170pt)=(0.2167,0.6966,0); rgb(171pt)=(0.227739,0.700166,0); rgb(172pt)=(0.238267,0.703889,0); rgb(173pt)=(0.248364,0.707737,0); rgb(174pt)=(0.258112,0.711676,0); rgb(175pt)=(0.267591,0.715675,0); rgb(176pt)=(0.276883,0.7197,0); rgb(177pt)=(0.286067,0.723718,0); rgb(178pt)=(0.295225,0.727696,0); rgb(179pt)=(0.304438,0.731602,0); rgb(180pt)=(0.313786,0.735402,0); rgb(181pt)=(0.32335,0.739065,0); rgb(182pt)=(0.333211,0.742556,0); rgb(183pt)=(0.34345,0.745844,0); rgb(184pt)=(0.354148,0.748895,0); rgb(185pt)=(0.365385,0.751677,0); rgb(186pt)=(0.377242,0.754156,0); rgb(187pt)=(0.3898,0.7563,0); rgb(188pt)=(0.403574,0.758195,0); rgb(189pt)=(0.418885,0.759957,0); rgb(190pt)=(0.435525,0.761599,0); rgb(191pt)=(0.453286,0.763136,0); rgb(192pt)=(0.471958,0.764579,0); rgb(193pt)=(0.491333,0.765943,0); rgb(194pt)=(0.511203,0.767241,0); rgb(195pt)=(0.531358,0.768485,0); rgb(196pt)=(0.551591,0.76969,0); rgb(197pt)=(0.571692,0.770869,0); rgb(198pt)=(0.591453,0.772035,0); rgb(199pt)=(0.610666,0.773201,0); rgb(200pt)=(0.629122,0.77438,0); rgb(201pt)=(0.646612,0.775587,0); rgb(202pt)=(0.662927,0.776833,0); rgb(203pt)=(0.677859,0.778133,0); rgb(204pt)=(0.6912,0.7795,0); rgb(205pt)=(0.70344,0.780892,0.00343064); rgb(206pt)=(0.715224,0.782261,0.0132589); rgb(207pt)=(0.72658,0.783614,0.0287895); rgb(208pt)=(0.737533,0.784956,0.0493267); rgb(209pt)=(0.748107,0.786293,0.0741753); rgb(210pt)=(0.758328,0.787632,0.10264); rgb(211pt)=(0.768223,0.788977,0.134025); rgb(212pt)=(0.777816,0.790335,0.167635); rgb(213pt)=(0.787133,0.791712,0.202774); rgb(214pt)=(0.7962,0.793114,0.238748); rgb(215pt)=(0.805042,0.794545,0.27486); rgb(216pt)=(0.813684,0.796013,0.310415); rgb(217pt)=(0.822153,0.797523,0.344718); rgb(218pt)=(0.830473,0.799081,0.377074); rgb(219pt)=(0.838671,0.800692,0.406786); rgb(220pt)=(0.846771,0.802363,0.43316); rgb(221pt)=(0.8548,0.8041,0.4555); rgb(222pt)=(0.863049,0.805883,0.475346); rgb(223pt)=(0.871713,0.807693,0.494702); rgb(224pt)=(0.880675,0.809539,0.513574); rgb(225pt)=(0.889816,0.811426,0.531964); rgb(226pt)=(0.899019,0.813361,0.549875); rgb(227pt)=(0.908164,0.815352,0.567311); rgb(228pt)=(0.917134,0.817406,0.584275); rgb(229pt)=(0.925811,0.819529,0.600771); rgb(230pt)=(0.934077,0.821728,0.616801); rgb(231pt)=(0.941812,0.824011,0.63237); rgb(232pt)=(0.9489,0.826383,0.64748); rgb(233pt)=(0.955221,0.828853,0.662135); rgb(234pt)=(0.960658,0.831427,0.676338); rgb(235pt)=(0.965093,0.834111,0.690093); rgb(236pt)=(0.968407,0.836914,0.703402); rgb(237pt)=(0.970482,0.839841,0.71627); rgb(238pt)=(0.9712,0.8429,0.7287); rgb(239pt)=(0.9712,0.846163,0.740814); rgb(240pt)=(0.971197,0.84969,0.752722); rgb(241pt)=(0.971189,0.853471,0.764409); rgb(242pt)=(0.971174,0.857495,0.775859); rgb(243pt)=(0.971149,0.861753,0.787056); rgb(244pt)=(0.971112,0.866234,0.797986); rgb(245pt)=(0.97106,0.870928,0.808633); rgb(246pt)=(0.970992,0.875824,0.81898); rgb(247pt)=(0.970903,0.880913,0.829014); rgb(248pt)=(0.970793,0.886183,0.838717); rgb(249pt)=(0.970658,0.891625,0.848075); rgb(250pt)=(0.970497,0.897228,0.857072); rgb(251pt)=(0.970306,0.902981,0.865692); rgb(252pt)=(0.970083,0.908876,0.873921); rgb(253pt)=(0.969826,0.914901,0.881742); rgb(254pt)=(0.969533,0.921046,0.88914); rgb(255pt)=(0.9692,0.9273,0.8961)},
colorbar,
colorbar style={separate axis lines},
point meta min=0,
point meta max=2.24999774549098
]
\addplot [forget plot] graphics [xmin=-0.0005005005005005,xmax=1.0005005005005,ymin=-0.0005005005005005,ymax=1.0005005005005] {problem_1_joint_posterior-1.png};
\end{axis}
\end{tikzpicture}%
  \caption{The joint posterior $p(\theta_1, \theta_2 \given \text{H})$.}
  \label{problem_joint_posterior}
\end{figure}

\clearpage
\begin{enumerate}
\setcounter{enumi}{1}
\item
  Consider the three-dimensional parameter vector $\vec{\theta} =
  [\theta_1, \theta_2, \theta_3]\trans$, with the following joint
  multivariate Gaussian prior:
  \begin{equation*}
    p(\vec{\theta})
    =
    \mc{N}(\vec{\theta}; \vec{\mu}, \mat{\Sigma})
    =
    \mc{N}
    \left(
    \begin{bmatrix}
      \theta_1 \\
      \theta_2 \\
      \theta_3
    \end{bmatrix}
    ;
    \begin{bmatrix}
      0 \\
      1 \\
      2
    \end{bmatrix},
    \begin{bmatrix}
      1 & 2 & 0   \\
      2 & 9 & 0 \\
      0 & 0 & 16
    \end{bmatrix}
    \right).
  \end{equation*}
  We are going to consider a decision problem with action space
  $\mc{A} = \{1, 2, 3\}$.  The result of choosing an action $a \in
  \mc{A}$ will be to observe the exact value of $\theta_a$, the $a$th
  element of $\vec{\theta}$.

  Consider the following loss functions, $\ell_1$ and $\ell_2$:
  \begin{equation*}
    \ell_1(\vec{\theta}, a) =
    \begin{cases}
      1 & \theta_a   >  0 \\
      0 & \theta_a \leq 0
    \end{cases}
    \qquad
    \ell_2(\vec{\theta}, a) = \min(0, \theta_a).
  \end{equation*}
  For each:
  \begin{itemize}
  \item
    Write a generic expression for the expected loss of action $a$ in
    terms of $\vec{\mu}$ and $\mat{\Sigma}$.  Evaluate any integrals
    you encounter.
  \item
    Give a numerical value for the expected loss of each action, using
    the values of $(\vec{\mu}, \mat{\Sigma})$ provided above.
  \item
    State the Bayes action.
  \end{itemize}
\end{enumerate}

\subsection*{Solution}

First, we note that the loss functions only depend on $\vec{\theta}$
through $\theta_a$, so we must only consider the marginal belief about
$\theta_a$ when contemplating action $a$.  By applying the marginalization
formula for multivariate Gaussians, this belief is:
\begin{equation*}
  p(\theta_a) = \mc{N}(\theta_a; \mu_a, \Sigma_{aa}).
\end{equation*}
For loss $\ell_1$, we may calculate the expected loss of each action:
\begin{equation*}
  \mathbb{E}
  \bigl[
    \ell_1(\vec{\theta}, a)
  \bigr]
  =
  \int_{-\infty}^\infty
  \ell_1(\vec{\theta}, a)
  p(\theta_a)
  \intd{\theta_a}
  =
  \int_0^\infty
  \mc{N}(\theta_a; \mu_a, \Sigma_{aa})
  \intd{\theta_a}
  =
  1 - \Phi(0; \mu_a, \Sigma_{aa}).
\end{equation*}
Using this result, we may numerically calculate the expected loss for
each action:
\begin{equation*}
  \mathbb{E}\bigl[\ell_1(\vec{\theta}, 1)\bigr]
  =
  0.5 \qquad
  \mathbb{E}\bigl[\ell_1(\vec{\theta}, 2)\bigr]
  =
  0.631 \qquad
  \mathbb{E}\bigl[\ell_1(\vec{\theta}, 3)\bigr]
  =
  0.691.
\end{equation*}
The Bayes action is $a = 1$, with the lowest expected loss.

For loss $\ell_2$, we proceed in the same way:
\begin{equation*}
  \mathbb{E}
  \bigl[
    \ell_2(\vec{\theta}, a)
  \bigr]
  =
  \int_{-\infty}^\infty
  \ell_2(\vec{\theta}, a)
  p(\theta_a)
  \intd{\theta_a}
  =
  \int_{-\infty}^0
  \theta_a
  \mc{N}(\theta_a; \mu_a, \Sigma_{aa})
  \intd{\theta_a}.
\end{equation*}
We may compute this definite integral; I used a table of Gaussian
integrals\footnote{\url{http://en.wikipedia.org/wiki/List_of_integrals_of_Gaussian_functions}}
and the identity
\begin{equation*}
  \phi\biggl(
  \frac{a - \mu}{\sigma}
  \biggr)
  =
  \sigma \mc{N}(a; \mu, \sigma^2)
\end{equation*}
to derive
\begin{equation*}
  \mathbb{E}
  \bigl[
    \ell_2(\vec{\theta}, a)
  \bigr]
  =
  \mu_a \Phi(0; \mu_a, \Sigma_{aa})
  -
  \Sigma_{aa}
  \mc{N}(0; \mu_a, \Sigma_{aa}).
\end{equation*}
Using this result, we may numerically calculate the expected loss for
each action:
\begin{equation*}
  \mathbb{E}\bigl[\ell_2(\vec{\theta}, 1)\bigr]
  =
  -0.399 \qquad
  \mathbb{E}\bigl[\ell_2(\vec{\theta}, 2)\bigr]
  =
  -0.763 \qquad
  \mathbb{E}\bigl[\ell_2(\vec{\theta}, 3)\bigr]
  =
  -0.791.
\end{equation*}
The Bayes action is now $a = 3$, with the lowest expected loss.

\clearpage
\begin{enumerate}
\setcounter{enumi}{2}
\item
  Consider a $d$-dimensional vector $\vec{\theta}$ with an arbitrary
  multivariate Gaussian distribution:
  \begin{equation*}
    p(\vec{\theta})
    =
    \mc{N}(\vec{\theta}; \vec{\mu}, \mat{\Sigma}).
  \end{equation*}
  \begin{itemize}
  \item
    Give a general expression for the distribution of the following
    (scalar) value $\tau$.
    \begin{equation*}
      \tau = \theta_1 + 2\theta_2 + \dotsb d\theta_d
    \end{equation*}
  \item
    Consider again the specific distribution of the three-dimensional
    vector $\vec{\theta}$ from the last problem, as well as the action
    space $\mc{A}$ with the same observation mechanism: after choosing
    $a \in \mc{A}$, we will observe the corresponding value
    $\theta_a$.  Suppose we may select one action and then must
    predict $\tau$ under a squared loss function:
    \begin{equation*}
      \ell(\tau, \hat{\tau}) = (\tau - \hat{\tau})^2.
    \end{equation*}
    Using the distribution from the last problem. what is the expected
    loss of each of the three available actions?  Which is the Bayes
    action?
  \end{itemize}
\end{enumerate}

\subsection*{Solution}

Define the (row) vector $\vec{d}\trans = [1, 2, \dotsc, d]$.  We first
notice that $\tau$ is simply a linear transformation of
$\vec{\theta}$:
\begin{equation*}
  \tau = \vec{d}\trans \vec{\theta};
\end{equation*}
therefore $\tau$ has a multivariate Gaussian distribution:
\begin{equation*}
  p(\tau)
  =
  p(\vec{d}\trans \vec{\theta})
  =
  \mc{N}(\tau; \vec{d}\trans \vec{\mu}, \vec{d}\trans \mat{\Sigma} \vec{d}).
\end{equation*}

In the second part of the question, we must consider estimating $\tau$
under a squared loss function $\ell(\tau, \hat{\tau}) = (\tau -
\hat{\tau})^2$.  A general result from Bayesian decision theory is
that the Bayes action is to estimate $\hat{\tau}$ as the (posterior)
mean of $\tau$.  For example, given the initial belief from the last
problem, we would predict $\hat{\tau} = \vec{d}\trans \vec{\mu} = 8$.
What is the \emph{expected} loss when predicting the mean $\hat{\tau}
= \mathbb{E}[\tau]$?
\begin{equation*}
  \mathbb{E}
  \Bigl[
    \ell\bigl(\tau, \mathbb{E}[\tau]\bigr)
  \Bigr]
  =
  \mathbb{E}
  \Bigl[
    \bigl(\tau - \mathbb{E}[\tau]\bigr)^2
  \Bigr]
  =
  \var [ \tau ]
  =
  \vec{d}\trans \mat{\Sigma} \vec{d}.
\end{equation*}
The expected squared loss is simply the variance of $\tau$!  Conveniently,
we have a closed-form expression for this variance.

The problem asks us to consider how we would proceed with estimating
$\tau$ if we could observe one of the entries of the vector
$\vec{\theta}$ before making our prediction $\hat{\tau}$.  If we wish
to minimize our expected loss, we should minimize the variance of
$\tau$ with our observation.  Observing an entry of $\vec{\theta}$ is
a conditioning observation of a multivariate Gaussian.  We have a
closed-form expression for the posterior covariance of $\vec{\theta}$
after observing any entry $\theta_a$.  Remarkably, the posterior
covariance of $\vec{\theta}$ does not depend on the actual value we
observe, only the index of the entry we choose, $a$.  The posterior
covariance matrices $\mat{\Sigma}_{\vec{\theta} \given \theta_a}$ for
each available action $a \in \mc{A}$ are:
\begin{align*}
  \mat{\Sigma}_{\vec{\theta} \given \theta_1}
  &=
  \begin{bmatrix}
    0 & 0 & 0   \\
    0 & 5 & 0   \\
    0 & 0 & 16
  \end{bmatrix}
  \\
  \mat{\Sigma}_{\vec{\theta} \given \theta_2}
  &=
  \begin{bmatrix}
    \frac{5}{9} & 0 & 0   \\
    0           & 0 & 0   \\
    0           & 0 & 16
  \end{bmatrix}
  \\
  \mat{\Sigma}_{\vec{\theta} \given \theta_3}
  &=
  \begin{bmatrix}
    1 & 2 & 0 \\
    2 & 9 & 0 \\
    0 & 0 & 0
  \end{bmatrix}.
\end{align*}
The expected losses of our final prediction of $\hat{\tau}$
given $\theta_a$ is now given by
\begin{equation*}
  \mathbb{E}\bigl[\ell(\tau, \hat{\tau}) \given \theta_a \bigr]
  =
  \vec{d}\trans \mat{\Sigma}_{\vec{\theta} \given \theta_a} \vec{d}.
\end{equation*}
For our particular problem, the expected final loss after each potential
action is:
\begin{equation*}
  \mathbb{E}\bigl[\ell(\tau, \hat{\tau}) \given \theta_1\bigr]
  =
  164 \qquad
  \mathbb{E}\bigl[\ell(\tau, \hat{\tau}) \given \theta_2\bigr]
  =
  144\,\nicefrac{5}{9} \qquad
  \mathbb{E}\bigl[\ell(\tau, \hat{\tau}) \given \theta_3\bigr]
  =
  45.
\end{equation*}
The Bayes action is $a = 3$.  Despite the fact that $\theta_3$ is
uncorrelated with the other two entries, collapsing its large variance
from $16$ to zero has the effect of reducing the variance of $\tau$
(and therefore our expected loss) by $3^2 \cdot 16 = 144$.

\clearpage
\begin{enumerate}
\setcounter{enumi}{3}
\item
  Consider the following data:
  \begin{align*}
    \vec{x} &= [0.54, 1.84, -2.26, 0.86, 0.32]\trans; \\
    \vec{y} &= [-1.31, -0.43, 0.34, 3.58, 2.77]\trans.
  \end{align*}
  Consider the Bayesian linear regression model with $\phi(x) = [1,
    x]\trans$.  Use the prior $p(\vec{w}) = \mc{N}(\vec{w}; \vec{0},
  \mat{I})$.

  Plot the posterior probability that the slope of the regression line
  is positive as a function of the standard deviation of the
  observation noise $\sigma$ (the noise variance is then $\sigma^2$).
  Use a grid of at least 100 points in the range $\sigma \in (0.01,
  10)$.
\end{enumerate}

\subsection*{Solution}

Using the given linear regression model, we assume
\begin{equation*}
  y
  =
  \phi(x)\trans \vec{w} + \varepsilon
  =
  w_1 + w_2 x + \varepsilon.
\end{equation*}
The second entry of the weight vector $\vec{w}$, $w_2$, therefore
serves as the slope of the regression line.

The Bayesian linear regression model gives the following posterior for
$\vec{w}$ given observations $\data$ and a specified noise variance
$\sigma^2$:
\begin{equation*}
  p(\vec{w} \given \data, \sigma^2)
  =
  \mc{N}(\vec{w};
  \vec{\mu}_{\vec{w}\given\data},
  \mat{\Sigma}_{\vec{w}\given\data}
  ),
\end{equation*}
where
\begin{align*}
  \vec{\mu}_{\vec{w}\given\data}
  &=
  \mat{X}\trans
  (\mat{X}\mat{X}\trans + \sigma^2 \mat{I})\inv
  \vec{y};
  \\
  \mat{\Sigma}_{\vec{w}\given\data}
  &=
  \mat{I}
  -
  \mat{X}\trans
  (\mat{X}\mat{X}\trans + \sigma^2 \mat{I})\inv
  \mat{X},
\end{align*}
where we have plugged the given prior $p(\vec{w}) = \mc{N}(\vec{w};
\vec{0}, \mat{I})$ into the general result.

Given a value of $\sigma$, the formulas above give the posterior over
$\vec{w}$.  To determine the probability that $w_2$ is positive, we
simply take the marginal posterior distribution and evaluate the
normal \acro{CDF}:
\begin{equation*}
  \Pr(w_2 > 0 \given \data, \sigma^2)
  =
  1 -
  \Phi\bigl(
  0;
  (\vec{\mu}_{\vec{w}\given\data})_2
  ,
  (\mat{\Sigma}_{\vec{w}\given\data})_{22}
  \bigr).
\end{equation*}

This quantity is plotted as a function of $\sigma$ in Figure
\ref{problem_4}.  The larger the noise, the less confident we become
about the sign of the slope.

\begin{figure}
  \centering
  % This file was created by matlab2tikz.
% Minimal pgfplots version: 1.3
%
\tikzsetnextfilename{problem_4}
\definecolor{mycolor1}{rgb}{0.12157,0.47059,0.70588}%
\definecolor{mycolor2}{rgb}{0.89020,0.10196,0.10980}%
%
\begin{tikzpicture}

\begin{axis}[%
width=0.95092\figurewidth,
height=\figureheight,
at={(0\figurewidth,0\figureheight)},
scale only axis,
unbounded coords=jump,
xmin=0,
xmax=1,
xlabel={$\theta$},
ymin=0,
ymax=70,
axis x line*=bottom,
axis y line*=left,
legend style={at={(0.03,0.97)},anchor=north west,legend cell align=left,align=left,fill=none,draw=none}
]
\addplot [color=mycolor1,solid]
  table[row sep=crcr]{%
0	inf\\
0.001001001001001	26.4367529085439\\
0.002002002002002	15.1961097601181\\
0.003003003003003	10.9953249913737\\
0.004004004004004	8.74190509398868\\
0.005005005005005	7.31859027032782\\
0.00600600600600601	6.33041447603642\\
0.00700700700700701	5.60047386889016\\
0.00800800800800801	5.03711228155502\\
0.00900900900900901	4.5878623760066\\
0.01001001001001	4.22041843769625\\
0.011011011011011	3.91374784189009\\
0.012012012012012	3.6535396591462\\
0.013013013013013	3.42970156130518\\
0.014014014014014	3.23490202658145\\
0.015015015015015	3.06368040554663\\
0.016016016016016	2.9118818172298\\
0.017017017017017	2.77628609293619\\
0.018018018018018	2.65435701036163\\
0.019019019019019	2.54406851729216\\
0.02002002002002	2.44378162491698\\
0.021021021021021	2.35215547806563\\
0.022022022022022	2.26808198626059\\
0.023023023023023	2.19063701684504\\
0.024024024024024	2.11904343637489\\
0.025025025025025	2.05264276357458\\
0.026026026026026	1.99087317222251\\
0.027027027027027	1.93325223831636\\
0.028028028028028	1.87936327488662\\
0.029029029029029	1.82884441008033\\
0.03003003003003	1.78137978446673\\
0.031031031031031	1.73669240108159\\
0.032032032032032	1.6945382758161\\
0.033033033033033	1.65470161932172\\
0.034034034034034	1.61699084347045\\
0.035035035035035	1.58123523167504\\
0.036036036036036	1.54728214729528\\
0.037037037037037	1.5149946809483\\
0.038038038038038	1.48424965795746\\
0.039039039039039	1.45493594297079\\
0.04004004004004	1.42695299109234\\
0.041041041041041	1.40020960453234\\
0.042042042042042	1.37462286141491\\
0.043043043043043	1.35011718944879\\
0.044044044044044	1.32662356201736\\
0.045045045045045	1.30407879814363\\
0.046046046046046	1.2824249509378\\
0.047047047047047	1.26160877169568\\
0.048048048048048	1.24158123890578\\
0.049049049049049	1.22229714313701\\
0.0500500500500501	1.20371472019014\\
0.0510510510510511	1.18579532606384\\
0.0520520520520521	1.16850314825537\\
0.0530530530530531	1.15180494872424\\
0.0540540540540541	1.13566983452327\\
0.0550550550550551	1.12006905266935\\
0.0560560560560561	1.10497580630456\\
0.0570570570570571	1.09036508960259\\
0.0580580580580581	1.07621353921849\\
0.0590590590590591	1.06249930037123\\
0.0600600600600601	1.04920190589769\\
0.0610610610610611	1.03630216682912\\
0.0620620620620621	1.02378207322417\\
0.0630630630630631	1.01162470414917\\
0.0640640640640641	0.999814145832239\\
0.0650650650650651	0.988335417134306\\
0.0660660660660661	0.977174401582041\\
0.0670670670670671	0.966317785295439\\
0.0680680680680681	0.955753000219582\\
0.0690690690690691	0.94546817213695\\
0.0700700700700701	0.935452072995093\\
0.0710710710710711	0.92569407713565\\
0.0720720720720721	0.916184121055662\\
0.0730730730730731	0.906912666371565\\
0.0740740740740741	0.897870665691108\\
0.0750750750750751	0.889049531129099\\
0.0760760760760761	0.880441105230045\\
0.0770770770770771	0.872037634084802\\
0.0780780780780781	0.863831742449633\\
0.0790790790790791	0.855816410695074\\
0.0800800800800801	0.847984953428811\\
0.0810810810810811	0.840330999651818\\
0.0820820820820821	0.832848474320428\\
0.0830830830830831	0.825531581198948\\
0.0840840840840841	0.818374786898217\\
0.0850850850850851	0.811372806005073\\
0.0860860860860861	0.804520587216334\\
0.0870870870870871	0.797813300398694\\
0.0880880880880881	0.791246324502835\\
0.0890890890890891	0.784815236266443\\
0.0900900900900901	0.778515799646391\\
0.0910910910910911	0.772343955925554\\
0.0920920920920921	0.766295814444307\\
0.0930930930930931	0.760367643910963\\
0.0940940940940941	0.754555864249217\\
0.0950950950950951	0.748857038944104\\
0.0960960960960961	0.7432678678511\\
0.0970970970970971	0.737785180435864\\
0.0980980980980981	0.732405929414694\\
0.0990990990990991	0.727127184768133\\
0.1001001001001	0.721946128102318\\
0.101101101101101	0.716860047334629\\
0.102102102102102	0.711866331681984\\
0.103103103103103	0.706962466931776\\
0.104104104104104	0.702146030976947\\
0.105105105105105	0.697414689598064\\
0.106106106106106	0.692766192476544\\
0.107107107107107	0.688198369424294\\
0.108108108108108	0.683709126816156\\
0.109109109109109	0.67929644421245\\
0.11011011011011	0.674958371159882\\
0.111111111111111	0.670693024159853\\
0.112112112112112	0.666498583794002\\
0.113113113113113	0.662373291997506\\
0.114114114114114	0.658315449471314\\
0.115115115115115	0.654323413225084\\
0.116116116116116	0.650395594243161\\
0.117117117117117	0.646530455266421\\
0.118118118118118	0.642726508683312\\
0.119119119119119	0.638982314523834\\
0.12012012012012	0.635296478550609\\
0.121121121121121	0.631667650441608\\
0.122122122122122	0.62809452205938\\
0.123123123123123	0.624575825802027\\
0.124124124124124	0.621110333031418\\
0.125125125125125	0.617696852574445\\
0.126126126126126	0.614334229293361\\
0.127127127127127	0.61102134272152\\
0.128128128128128	0.607757105761012\\
0.129129129129129	0.604540463438947\\
0.13013013013013	0.601370391719303\\
0.131131131131131	0.598245896367468\\
0.132132132132132	0.595166011864728\\
0.133133133133133	0.592129800370184\\
0.134134134134134	0.589136350727647\\
0.135135135135135	0.586184777515275\\
0.136136136136136	0.583274220135795\\
0.137137137137137	0.580403841945298\\
0.138138138138138	0.5775728294187\\
0.139139139139139	0.574780391350089\\
0.14014014014014	0.572025758086231\\
0.141141141141141	0.569308180791666\\
0.142142142142142	0.566626930743862\\
0.143143143143143	0.563981298656992\\
0.144144144144144	0.561370594033002\\
0.145145145145145	0.558794144538659\\
0.146146146146146	0.556251295407397\\
0.147147147147147	0.553741408864785\\
0.148148148148148	0.551263863576558\\
0.149149149149149	0.548818054118153\\
0.15015015015015	0.546403390464797\\
0.151151151151151	0.544019297501202\\
0.152152152152152	0.541665214550004\\
0.153153153153153	0.5393405949181\\
0.154154154154154	0.537044905460099\\
0.155155155155155	0.53477762615813\\
0.156156156156156	0.532538249717302\\
0.157157157157157	0.530326281176123\\
0.158158158158158	0.528141237531251\\
0.159159159159159	0.525982647375946\\
0.16016016016016	0.523850050551652\\
0.161161161161161	0.521742997812149\\
0.162162162162162	0.51966105049975\\
0.163163163163163	0.517603780233031\\
0.164164164164164	0.515570768605631\\
0.165165165165165	0.513561606895647\\
0.166166166166166	0.511575895785203\\
0.167167167167167	0.509613245089774\\
0.168168168168168	0.507673273496867\\
0.169169169169169	0.505755608313687\\
0.17017017017017	0.503859885223423\\
0.171171171171171	0.501985748049815\\
0.172172172172172	0.500132848529672\\
0.173173173173173	0.498300846093027\\
0.174174174174174	0.496489407650638\\
0.175175175175175	0.494698207388533\\
0.176176176176176	0.492926926569348\\
0.177177177177177	0.491175253340175\\
0.178178178178178	0.489442882546688\\
0.179179179179179	0.4877295155533\\
0.18018018018018	0.486034860069122\\
0.181181181181181	0.484358629979513\\
0.182182182182182	0.482700545183\\
0.183183183183183	0.481060331433384\\
0.184184184184184	0.479437720186826\\
0.185185185185185	0.477832448453731\\
0.186186186186186	0.476244258655271\\
0.187187187187187	0.474672898484354\\
0.188188188188188	0.473118120770894\\
0.189189189189189	0.471579683351224\\
0.19019019019019	0.470057348941499\\
0.191191191191191	0.468550885014955\\
0.192192192192192	0.467060063682875\\
0.193193193193193	0.465584661579146\\
0.194194194194194	0.46412445974827\\
0.195195195195195	0.462679243536708\\
0.196196196196196	0.461248802487459\\
0.197197197197197	0.459832930237726\\
0.198198198198198	0.45843142441961\\
0.199199199199199	0.457044086563685\\
0.2002002002002	0.455670722005381\\
0.201201201201201	0.45431113979408\\
0.202202202202202	0.452965152604813\\
0.203203203203203	0.451632576652498\\
0.204204204204204	0.450313231608614\\
0.205205205205205	0.449006940520239\\
0.206206206206206	0.447713529731373\\
0.207207207207207	0.446432828806467\\
0.208208208208208	0.445164670456091\\
0.209209209209209	0.44390889046467\\
0.21021021021021	0.442665327620216\\
0.211211211211211	0.441433823646001\\
0.212212212212212	0.440214223134097\\
0.213213213213213	0.439006373480734\\
0.214214214214214	0.437810124823415\\
0.215215215215215	0.436625329979724\\
0.216216216216216	0.435451844387789\\
0.217217217217217	0.434289526048336\\
0.218218218218218	0.433138235468287\\
0.219219219219219	0.431997835605855\\
0.22022022022022	0.430868191817096\\
0.221221221221221	0.429749171803858\\
0.222222222222222	0.428640645563101\\
0.223223223223223	0.427542485337532\\
0.224224224224224	0.426454565567524\\
0.225225225225225	0.42537676284428\\
0.226226226226226	0.424308955864198\\
0.227227227227227	0.423251025384405\\
0.228228228228228	0.422202854179429\\
0.229229229229229	0.421164326998969\\
0.23023023023023	0.420135330526729\\
0.231231231231231	0.419115753340295\\
0.232232232232232	0.418105485872015\\
0.233233233233233	0.417104420370852\\
0.234234234234234	0.416112450865192\\
0.235235235235235	0.415129473126566\\
0.236236236236236	0.414155384634272\\
0.237237237237237	0.413190084540858\\
0.238238238238238	0.412233473638459\\
0.239239239239239	0.411285454325942\\
0.24024024024024	0.410345930576856\\
0.241241241241241	0.409414807908152\\
0.242242242242242	0.408491993349655\\
0.243243243243243	0.407577395414269\\
0.244244244244244	0.406670924068891\\
0.245245245245245	0.405772490706016\\
0.246246246246246	0.404882008116014\\
0.247247247247247	0.40399939046006\\
0.248248248248248	0.403124553243701\\
0.249249249249249	0.40225741329104\\
0.25025025025025	0.401397888719515\\
0.251251251251251	0.400545898915279\\
0.252252252252252	0.399701364509128\\
0.253253253253253	0.398864207353001\\
0.254254254254254	0.398034350497007\\
0.255255255255255	0.397211718166985\\
0.256256256256256	0.396396235742565\\
0.257257257257257	0.395587829735737\\
0.258258258258258	0.394786427769896\\
0.259259259259259	0.39399195855936\\
0.26026026026026	0.393204351889351\\
0.261261261261261	0.392423538596416\\
0.262262262262262	0.391649450549291\\
0.263263263263263	0.390882020630182\\
0.264264264264264	0.390121182716465\\
0.265265265265265	0.389366871662783\\
0.266266266266266	0.388619023283544\\
0.267267267267267	0.387877574335788\\
0.268268268268268	0.387142462502443\\
0.269269269269269	0.386413626375926\\
0.27027027027027	0.385691005442115\\
0.271271271271271	0.384974540064656\\
0.272272272272272	0.38426417146961\\
0.273273273273273	0.383559841730424\\
0.274274274274274	0.38286149375323\\
0.275275275275275	0.382169071262449\\
0.276276276276276	0.381482518786701\\
0.277277277277277	0.380801781645015\\
0.278278278278278	0.380126805933325\\
0.279279279279279	0.379457538511251\\
0.28028028028028	0.378793926989153\\
0.281281281281281	0.378135919715458\\
0.282282282282282	0.377483465764246\\
0.283283283283283	0.376836514923095\\
0.284284284284284	0.376195017681175\\
0.285285285285285	0.375558925217584\\
0.286286286286286	0.374928189389928\\
0.287287287287287	0.374302762723129\\
0.288288288288288	0.373682598398458\\
0.289289289289289	0.3730676502428\\
0.29029029029029	0.372457872718127\\
0.291291291291291	0.371853220911181\\
0.292292292292292	0.371253650523372\\
0.293293293293293	0.370659117860876\\
0.294294294294294	0.37006957982492\\
0.295295295295295	0.369484993902276\\
0.296296296296296	0.368905318155931\\
0.297297297297297	0.368330511215945\\
0.298298298298298	0.367760532270492\\
0.299299299299299	0.367195341057069\\
0.3003003003003	0.366634897853888\\
0.301301301301301	0.366079163471424\\
0.302302302302302	0.365528099244135\\
0.303303303303303	0.364981667022342\\
0.304304304304304	0.36443982916426\\
0.305305305305305	0.363902548528189\\
0.306306306306306	0.363369788464855\\
0.307307307307307	0.36284151280989\\
0.308308308308308	0.362317685876465\\
0.309309309309309	0.361798272448055\\
0.31031031031031	0.361283237771352\\
0.311311311311311	0.360772547549298\\
0.312312312312312	0.360266167934263\\
0.313313313313313	0.359764065521343\\
0.314314314314314	0.359266207341785\\
0.315315315315315	0.35877256085654\\
0.316316316316316	0.35828309394993\\
0.317317317317317	0.357797774923436\\
0.318318318318318	0.357316572489603\\
0.319319319319319	0.356839455766052\\
0.32032032032032	0.356366394269612\\
0.321321321321321	0.35589735791055\\
0.322322322322322	0.355432316986913\\
0.323323323323323	0.354971242178973\\
0.324324324324324	0.354514104543773\\
0.325325325325325	0.35406087550977\\
0.326326326326326	0.353611526871581\\
0.327327327327327	0.353166030784818\\
0.328328328328328	0.352724359761023\\
0.329329329329329	0.352286486662687\\
0.33033033033033	0.351852384698364\\
0.331331331331331	0.351422027417875\\
0.332332332332332	0.350995388707588\\
0.333333333333333	0.350572442785794\\
0.334334334334334	0.350153164198158\\
0.335335335335335	0.349737527813254\\
0.336336336336336	0.349325508818178\\
0.337337337337337	0.34891708271424\\
0.338338338338338	0.348512225312731\\
0.339339339339339	0.348110912730766\\
0.34034034034034	0.347713121387198\\
0.341341341341341	0.347318827998606\\
0.342342342342342	0.346928009575353\\
0.343343343343343	0.346540643417712\\
0.344344344344344	0.346156707112061\\
0.345345345345345	0.345776178527141\\
0.346346346346346	0.345399035810384\\
0.347347347347347	0.345025257384304\\
0.348348348348348	0.344654821942941\\
0.349349349349349	0.344287708448387\\
0.35035035035035	0.343923896127345\\
0.351351351351351	0.343563364467775\\
0.352352352352352	0.343206093215575\\
0.353353353353353	0.342852062371332\\
0.354354354354354	0.342501252187125\\
0.355355355355355	0.342153643163384\\
0.356356356356356	0.341809216045799\\
0.357357357357357	0.341467951822284\\
0.358358358358358	0.341129831719998\\
0.359359359359359	0.340794837202406\\
0.36036036036036	0.340462949966398\\
0.361361361361361	0.340134151939455\\
0.362362362362362	0.339808425276861\\
0.363363363363363	0.339485752358967\\
0.364364364364364	0.339166115788494\\
0.365365365365365	0.338849498387888\\
0.366366366366366	0.338535883196718\\
0.367367367367367	0.338225253469113\\
0.368368368368368	0.337917592671253\\
0.369369369369369	0.337612884478887\\
0.37037037037037	0.337311112774908\\
0.371371371371371	0.337012261646959\\
0.372372372372372	0.33671631538508\\
0.373373373373373	0.336423258479402\\
0.374374374374374	0.336133075617866\\
0.375375375375375	0.335845751683997\\
0.376376376376376	0.335561271754698\\
0.377377377377377	0.335279621098097\\
0.378378378378378	0.335000785171414\\
0.379379379379379	0.334724749618879\\
0.38038038038038	0.334451500269676\\
0.381381381381381	0.334181023135919\\
0.382382382382382	0.33391330441067\\
0.383383383383383	0.333648330465984\\
0.384384384384384	0.33338608785099\\
0.385385385385385	0.333126563289999\\
0.386386386386386	0.332869743680651\\
0.387387387387387	0.332615616092087\\
0.388388388388388	0.332364167763154\\
0.389389389389389	0.332115386100637\\
0.39039039039039	0.331869258677529\\
0.391391391391391	0.331625773231317\\
0.392392392392392	0.331384917662309\\
0.393393393393393	0.33114668003198\\
0.394394394394394	0.33091104856135\\
0.395395395395395	0.330678011629391\\
0.396396396396396	0.330447557771456\\
0.397397397397397	0.330219675677737\\
0.398398398398398	0.329994354191749\\
0.399399399399399	0.329771582308837\\
0.4004004004004	0.329551349174716\\
0.401401401401401	0.329333644084021\\
0.402402402402402	0.329118456478901\\
0.403403403403403	0.328905775947617\\
0.404404404404404	0.328695592223179\\
0.405405405405405	0.328487895181998\\
0.406406406406406	0.328282674842562\\
0.407407407407407	0.328079921364139\\
0.408408408408408	0.327879625045499\\
0.409409409409409	0.327681776323654\\
0.41041041041041	0.327486365772628\\
0.411411411411411	0.327293384102241\\
0.412412412412412	0.327102822156919\\
0.413413413413413	0.326914670914523\\
0.414414414414414	0.326728921485195\\
0.415415415415415	0.32654556511023\\
0.416416416416416	0.326364593160964\\
0.417417417417417	0.326185997137682\\
0.418418418418418	0.326009768668548\\
0.419419419419419	0.325835899508548\\
0.42042042042042	0.325664381538458\\
0.421421421421421	0.325495206763828\\
0.422422422422422	0.325328367313986\\
0.423423423423423	0.325163855441052\\
0.424424424424424	0.325001663518985\\
0.425425425425425	0.324841784042629\\
0.426426426426426	0.324684209626795\\
0.427427427427427	0.324528933005344\\
0.428428428428428	0.324375947030297\\
0.429429429429429	0.324225244670958\\
0.43043043043043	0.324076819013055\\
0.431431431431431	0.323930663257893\\
0.432432432432432	0.32378677072153\\
0.433433433433433	0.323645134833961\\
0.434434434434434	0.323505749138325\\
0.435435435435435	0.323368607290123\\
0.436436436436436	0.323233703056451\\
0.437437437437437	0.323101030315252\\
0.438438438438438	0.322970583054578\\
0.439439439439439	0.322842355371871\\
0.44044044044044	0.322716341473254\\
0.441441441441441	0.322592535672842\\
0.442442442442442	0.322470932392064\\
0.443443443443443	0.322351526158997\\
0.444444444444444	0.322234311607719\\
0.445445445445445	0.322119283477673\\
0.446446446446446	0.322006436613042\\
0.447447447447447	0.321895765962145\\
0.448448448448448	0.321787266576838\\
0.449449449449449	0.321680933611933\\
0.45045045045045	0.321576762324629\\
0.451451451451451	0.321474748073958\\
0.452452452452452	0.321374886320239\\
0.453453453453453	0.321277172624548\\
0.454454454454454	0.321181602648203\\
0.455455455455455	0.321088172152256\\
0.456456456456456	0.320996876997\\
0.457457457457457	0.320907713141491\\
0.458458458458458	0.320820676643077\\
0.459459459459459	0.320735763656942\\
0.46046046046046	0.320652970435663\\
0.461461461461461	0.320572293328775\\
0.462462462462462	0.320493728782354\\
0.463463463463463	0.320417273338603\\
0.464464464464464	0.320342923635458\\
0.465465465465465	0.3202706764062\\
0.466466466466466	0.320200528479083\\
0.467467467467467	0.320132476776966\\
0.468468468468468	0.320066518316966\\
0.469469469469469	0.320002650210113\\
0.47047047047047	0.319940869661023\\
0.471471471471471	0.319881173967578\\
0.472472472472472	0.319823560520617\\
0.473473473473473	0.319768026803641\\
0.474474474474474	0.319714570392527\\
0.475475475475475	0.319663188955249\\
0.476476476476476	0.319613880251619\\
0.477477477477477	0.319566642133027\\
0.478478478478478	0.319521472542203\\
0.479479479479479	0.319478369512978\\
0.48048048048048	0.319437331170068\\
0.481481481481481	0.319398355728858\\
0.482482482482482	0.319361441495201\\
0.483483483483483	0.319326586865226\\
0.484484484484485	0.319293790325161\\
0.485485485485485	0.319263050451156\\
0.486486486486487	0.319234365909127\\
0.487487487487487	0.319207735454605\\
0.488488488488488	0.319183157932595\\
0.48948948948949	0.319160632277447\\
0.49049049049049	0.319140157512735\\
0.491491491491492	0.319121732751146\\
0.492492492492492	0.319105357194386\\
0.493493493493493	0.319091030133082\\
0.494494494494495	0.319078750946709\\
0.495495495495495	0.319068519103517\\
0.496496496496497	0.319060334160472\\
0.497497497497497	0.319054195763208\\
0.498498498498498	0.319050103645983\\
0.4994994994995	0.319048057631654\\
0.500500500500501	0.319048057631654\\
0.501501501501502	0.319050103645983\\
0.502502502502503	0.319054195763208\\
0.503503503503503	0.319060334160472\\
0.504504504504504	0.319068519103517\\
0.505505505505506	0.319078750946709\\
0.506506506506507	0.319091030133082\\
0.507507507507508	0.319105357194386\\
0.508508508508508	0.319121732751146\\
0.509509509509509	0.319140157512735\\
0.510510510510511	0.319160632277447\\
0.511511511511512	0.319183157932595\\
0.512512512512513	0.319207735454605\\
0.513513513513513	0.319234365909127\\
0.514514514514514	0.319263050451156\\
0.515515515515516	0.319293790325161\\
0.516516516516517	0.319326586865226\\
0.517517517517518	0.319361441495201\\
0.518518518518518	0.319398355728858\\
0.519519519519519	0.319437331170068\\
0.520520520520521	0.319478369512978\\
0.521521521521522	0.319521472542203\\
0.522522522522523	0.319566642133027\\
0.523523523523523	0.319613880251619\\
0.524524524524524	0.319663188955249\\
0.525525525525526	0.319714570392527\\
0.526526526526527	0.319768026803641\\
0.527527527527528	0.319823560520617\\
0.528528528528528	0.319881173967578\\
0.529529529529529	0.319940869661023\\
0.530530530530531	0.320002650210113\\
0.531531531531532	0.320066518316966\\
0.532532532532533	0.320132476776966\\
0.533533533533533	0.320200528479083\\
0.534534534534535	0.3202706764062\\
0.535535535535536	0.320342923635458\\
0.536536536536537	0.320417273338603\\
0.537537537537538	0.320493728782354\\
0.538538538538539	0.320572293328775\\
0.53953953953954	0.320652970435663\\
0.540540540540541	0.320735763656942\\
0.541541541541542	0.320820676643077\\
0.542542542542543	0.320907713141491\\
0.543543543543544	0.320996876997\\
0.544544544544545	0.321088172152256\\
0.545545545545546	0.321181602648203\\
0.546546546546547	0.321277172624548\\
0.547547547547548	0.321374886320239\\
0.548548548548549	0.321474748073958\\
0.54954954954955	0.321576762324629\\
0.550550550550551	0.321680933611933\\
0.551551551551552	0.321787266576838\\
0.552552552552553	0.321895765962145\\
0.553553553553554	0.322006436613042\\
0.554554554554555	0.322119283477673\\
0.555555555555556	0.322234311607719\\
0.556556556556557	0.322351526158997\\
0.557557557557558	0.322470932392064\\
0.558558558558559	0.322592535672842\\
0.55955955955956	0.322716341473254\\
0.560560560560561	0.322842355371871\\
0.561561561561562	0.322970583054578\\
0.562562562562563	0.323101030315252\\
0.563563563563564	0.323233703056451\\
0.564564564564565	0.323368607290123\\
0.565565565565566	0.323505749138325\\
0.566566566566567	0.323645134833961\\
0.567567567567568	0.32378677072153\\
0.568568568568569	0.323930663257893\\
0.56956956956957	0.324076819013055\\
0.570570570570571	0.324225244670958\\
0.571571571571572	0.324375947030297\\
0.572572572572573	0.324528933005344\\
0.573573573573574	0.324684209626795\\
0.574574574574575	0.324841784042629\\
0.575575575575576	0.325001663518985\\
0.576576576576577	0.325163855441052\\
0.577577577577578	0.325328367313986\\
0.578578578578579	0.325495206763828\\
0.57957957957958	0.325664381538458\\
0.580580580580581	0.325835899508548\\
0.581581581581582	0.326009768668548\\
0.582582582582583	0.326185997137682\\
0.583583583583584	0.326364593160964\\
0.584584584584585	0.32654556511023\\
0.585585585585586	0.326728921485195\\
0.586586586586587	0.326914670914523\\
0.587587587587588	0.327102822156919\\
0.588588588588589	0.327293384102241\\
0.58958958958959	0.327486365772628\\
0.590590590590591	0.327681776323654\\
0.591591591591592	0.327879625045499\\
0.592592592592593	0.328079921364139\\
0.593593593593594	0.328282674842562\\
0.594594594594595	0.328487895181998\\
0.595595595595596	0.328695592223179\\
0.596596596596597	0.328905775947617\\
0.597597597597598	0.329118456478901\\
0.598598598598599	0.329333644084021\\
0.5995995995996	0.329551349174716\\
0.600600600600601	0.329771582308837\\
0.601601601601602	0.329994354191749\\
0.602602602602603	0.330219675677737\\
0.603603603603604	0.330447557771456\\
0.604604604604605	0.330678011629391\\
0.605605605605606	0.33091104856135\\
0.606606606606607	0.33114668003198\\
0.607607607607608	0.331384917662309\\
0.608608608608609	0.331625773231317\\
0.60960960960961	0.331869258677529\\
0.610610610610611	0.332115386100637\\
0.611611611611612	0.332364167763154\\
0.612612612612613	0.332615616092087\\
0.613613613613614	0.332869743680651\\
0.614614614614615	0.333126563289999\\
0.615615615615616	0.33338608785099\\
0.616616616616617	0.333648330465984\\
0.617617617617618	0.33391330441067\\
0.618618618618619	0.334181023135919\\
0.61961961961962	0.334451500269676\\
0.620620620620621	0.334724749618879\\
0.621621621621622	0.335000785171414\\
0.622622622622623	0.335279621098097\\
0.623623623623624	0.335561271754698\\
0.624624624624625	0.335845751683997\\
0.625625625625626	0.336133075617866\\
0.626626626626627	0.336423258479402\\
0.627627627627628	0.33671631538508\\
0.628628628628629	0.337012261646959\\
0.62962962962963	0.337311112774908\\
0.630630630630631	0.337612884478887\\
0.631631631631632	0.337917592671253\\
0.632632632632633	0.338225253469113\\
0.633633633633634	0.338535883196718\\
0.634634634634635	0.338849498387888\\
0.635635635635636	0.339166115788494\\
0.636636636636637	0.339485752358967\\
0.637637637637638	0.339808425276861\\
0.638638638638639	0.340134151939455\\
0.63963963963964	0.340462949966398\\
0.640640640640641	0.340794837202406\\
0.641641641641642	0.341129831719999\\
0.642642642642643	0.341467951822284\\
0.643643643643644	0.341809216045799\\
0.644644644644645	0.342153643163384\\
0.645645645645646	0.342501252187125\\
0.646646646646647	0.342852062371332\\
0.647647647647648	0.343206093215575\\
0.648648648648649	0.343563364467775\\
0.64964964964965	0.343923896127345\\
0.650650650650651	0.344287708448387\\
0.651651651651652	0.344654821942941\\
0.652652652652653	0.345025257384304\\
0.653653653653654	0.345399035810384\\
0.654654654654655	0.345776178527141\\
0.655655655655656	0.346156707112061\\
0.656656656656657	0.346540643417712\\
0.657657657657658	0.346928009575353\\
0.658658658658659	0.347318827998606\\
0.65965965965966	0.347713121387198\\
0.660660660660661	0.348110912730766\\
0.661661661661662	0.348512225312731\\
0.662662662662663	0.34891708271424\\
0.663663663663664	0.349325508818178\\
0.664664664664665	0.349737527813254\\
0.665665665665666	0.350153164198158\\
0.666666666666667	0.350572442785794\\
0.667667667667668	0.350995388707588\\
0.668668668668669	0.351422027417875\\
0.66966966966967	0.351852384698364\\
0.670670670670671	0.352286486662687\\
0.671671671671672	0.352724359761023\\
0.672672672672673	0.353166030784818\\
0.673673673673674	0.353611526871581\\
0.674674674674675	0.35406087550977\\
0.675675675675676	0.354514104543773\\
0.676676676676677	0.354971242178973\\
0.677677677677678	0.355432316986913\\
0.678678678678679	0.35589735791055\\
0.67967967967968	0.356366394269612\\
0.680680680680681	0.356839455766052\\
0.681681681681682	0.357316572489603\\
0.682682682682683	0.357797774923436\\
0.683683683683684	0.35828309394993\\
0.684684684684685	0.35877256085654\\
0.685685685685686	0.359266207341785\\
0.686686686686687	0.359764065521343\\
0.687687687687688	0.360266167934263\\
0.688688688688689	0.360772547549298\\
0.68968968968969	0.361283237771352\\
0.690690690690691	0.361798272448055\\
0.691691691691692	0.362317685876465\\
0.692692692692693	0.36284151280989\\
0.693693693693694	0.363369788464855\\
0.694694694694695	0.363902548528189\\
0.695695695695696	0.36443982916426\\
0.696696696696697	0.364981667022342\\
0.697697697697698	0.365528099244135\\
0.698698698698699	0.366079163471424\\
0.6996996996997	0.366634897853888\\
0.700700700700701	0.367195341057069\\
0.701701701701702	0.367760532270491\\
0.702702702702703	0.368330511215945\\
0.703703703703704	0.368905318155931\\
0.704704704704705	0.369484993902276\\
0.705705705705706	0.37006957982492\\
0.706706706706707	0.370659117860876\\
0.707707707707708	0.371253650523372\\
0.708708708708709	0.371853220911181\\
0.70970970970971	0.372457872718127\\
0.710710710710711	0.3730676502428\\
0.711711711711712	0.373682598398458\\
0.712712712712713	0.374302762723129\\
0.713713713713714	0.374928189389928\\
0.714714714714715	0.375558925217584\\
0.715715715715716	0.376195017681175\\
0.716716716716717	0.376836514923095\\
0.717717717717718	0.377483465764246\\
0.718718718718719	0.378135919715458\\
0.71971971971972	0.378793926989153\\
0.720720720720721	0.379457538511251\\
0.721721721721722	0.380126805933325\\
0.722722722722723	0.380801781645015\\
0.723723723723724	0.381482518786701\\
0.724724724724725	0.382169071262449\\
0.725725725725726	0.38286149375323\\
0.726726726726727	0.383559841730424\\
0.727727727727728	0.38426417146961\\
0.728728728728729	0.384974540064656\\
0.72972972972973	0.385691005442115\\
0.730730730730731	0.386413626375926\\
0.731731731731732	0.387142462502443\\
0.732732732732733	0.387877574335788\\
0.733733733733734	0.388619023283544\\
0.734734734734735	0.389366871662783\\
0.735735735735736	0.390121182716464\\
0.736736736736737	0.390882020630182\\
0.737737737737738	0.391649450549291\\
0.738738738738739	0.392423538596416\\
0.73973973973974	0.393204351889351\\
0.740740740740741	0.39399195855936\\
0.741741741741742	0.394786427769896\\
0.742742742742743	0.395587829735737\\
0.743743743743744	0.396396235742565\\
0.744744744744745	0.397211718166985\\
0.745745745745746	0.398034350497007\\
0.746746746746747	0.398864207353001\\
0.747747747747748	0.399701364509128\\
0.748748748748749	0.400545898915279\\
0.74974974974975	0.401397888719515\\
0.750750750750751	0.402257413291039\\
0.751751751751752	0.403124553243701\\
0.752752752752753	0.40399939046006\\
0.753753753753754	0.404882008116014\\
0.754754754754755	0.405772490706016\\
0.755755755755756	0.406670924068891\\
0.756756756756757	0.407577395414269\\
0.757757757757758	0.408491993349655\\
0.758758758758759	0.409414807908152\\
0.75975975975976	0.410345930576856\\
0.760760760760761	0.411285454325942\\
0.761761761761762	0.412233473638459\\
0.762762762762763	0.413190084540858\\
0.763763763763764	0.414155384634272\\
0.764764764764765	0.415129473126566\\
0.765765765765766	0.416112450865192\\
0.766766766766767	0.417104420370852\\
0.767767767767768	0.418105485872015\\
0.768768768768769	0.419115753340295\\
0.76976976976977	0.420135330526729\\
0.770770770770771	0.421164326998969\\
0.771771771771772	0.422202854179429\\
0.772772772772773	0.423251025384405\\
0.773773773773774	0.424308955864198\\
0.774774774774775	0.42537676284428\\
0.775775775775776	0.426454565567524\\
0.776776776776777	0.427542485337532\\
0.777777777777778	0.428640645563101\\
0.778778778778779	0.429749171803858\\
0.77977977977978	0.430868191817096\\
0.780780780780781	0.431997835605855\\
0.781781781781782	0.433138235468287\\
0.782782782782783	0.434289526048336\\
0.783783783783784	0.435451844387789\\
0.784784784784785	0.436625329979724\\
0.785785785785786	0.437810124823415\\
0.786786786786787	0.439006373480734\\
0.787787787787788	0.440214223134097\\
0.788788788788789	0.441433823646001\\
0.78978978978979	0.442665327620216\\
0.790790790790791	0.44390889046467\\
0.791791791791792	0.445164670456091\\
0.792792792792793	0.446432828806467\\
0.793793793793794	0.447713529731373\\
0.794794794794795	0.449006940520239\\
0.795795795795796	0.450313231608614\\
0.796796796796797	0.451632576652498\\
0.797797797797798	0.452965152604813\\
0.798798798798799	0.45431113979408\\
0.7997997997998	0.455670722005381\\
0.800800800800801	0.457044086563685\\
0.801801801801802	0.45843142441961\\
0.802802802802803	0.459832930237726\\
0.803803803803804	0.461248802487459\\
0.804804804804805	0.462679243536708\\
0.805805805805806	0.46412445974827\\
0.806806806806807	0.465584661579146\\
0.807807807807808	0.467060063682875\\
0.808808808808809	0.468550885014955\\
0.80980980980981	0.470057348941499\\
0.810810810810811	0.471579683351224\\
0.811811811811812	0.473118120770894\\
0.812812812812813	0.474672898484354\\
0.813813813813814	0.476244258655271\\
0.814814814814815	0.477832448453731\\
0.815815815815816	0.479437720186826\\
0.816816816816817	0.481060331433384\\
0.817817817817818	0.482700545183\\
0.818818818818819	0.484358629979513\\
0.81981981981982	0.486034860069122\\
0.820820820820821	0.4877295155533\\
0.821821821821822	0.489442882546688\\
0.822822822822823	0.491175253340175\\
0.823823823823824	0.492926926569348\\
0.824824824824825	0.494698207388533\\
0.825825825825826	0.496489407650638\\
0.826826826826827	0.498300846093027\\
0.827827827827828	0.500132848529672\\
0.828828828828829	0.501985748049815\\
0.82982982982983	0.503859885223423\\
0.830830830830831	0.505755608313687\\
0.831831831831832	0.507673273496867\\
0.832832832832833	0.509613245089774\\
0.833833833833834	0.511575895785203\\
0.834834834834835	0.513561606895647\\
0.835835835835836	0.515570768605631\\
0.836836836836837	0.517603780233031\\
0.837837837837838	0.51966105049975\\
0.838838838838839	0.521742997812149\\
0.83983983983984	0.523850050551651\\
0.840840840840841	0.525982647375946\\
0.841841841841842	0.528141237531251\\
0.842842842842843	0.530326281176123\\
0.843843843843844	0.532538249717302\\
0.844844844844845	0.53477762615813\\
0.845845845845846	0.537044905460099\\
0.846846846846847	0.5393405949181\\
0.847847847847848	0.541665214550004\\
0.848848848848849	0.544019297501202\\
0.84984984984985	0.546403390464797\\
0.850850850850851	0.548818054118153\\
0.851851851851852	0.551263863576558\\
0.852852852852853	0.553741408864785\\
0.853853853853854	0.556251295407397\\
0.854854854854855	0.55879414453866\\
0.855855855855856	0.561370594033002\\
0.856856856856857	0.563981298656992\\
0.857857857857858	0.566626930743861\\
0.858858858858859	0.569308180791666\\
0.85985985985986	0.572025758086231\\
0.860860860860861	0.574780391350089\\
0.861861861861862	0.5775728294187\\
0.862862862862863	0.580403841945297\\
0.863863863863864	0.583274220135795\\
0.864864864864865	0.586184777515276\\
0.865865865865866	0.589136350727647\\
0.866866866866867	0.592129800370184\\
0.867867867867868	0.595166011864728\\
0.868868868868869	0.598245896367467\\
0.86986986986987	0.601370391719303\\
0.870870870870871	0.604540463438947\\
0.871871871871872	0.607757105761012\\
0.872872872872873	0.61102134272152\\
0.873873873873874	0.614334229293361\\
0.874874874874875	0.617696852574445\\
0.875875875875876	0.621110333031419\\
0.876876876876877	0.624575825802027\\
0.877877877877878	0.62809452205938\\
0.878878878878879	0.631667650441607\\
0.87987987987988	0.635296478550609\\
0.880880880880881	0.638982314523834\\
0.881881881881882	0.642726508683312\\
0.882882882882883	0.646530455266421\\
0.883883883883884	0.65039559424316\\
0.884884884884885	0.654323413225084\\
0.885885885885886	0.658315449471314\\
0.886886886886887	0.662373291997506\\
0.887887887887888	0.666498583794002\\
0.888888888888889	0.670693024159853\\
0.88988988988989	0.674958371159882\\
0.890890890890891	0.67929644421245\\
0.891891891891892	0.683709126816156\\
0.892892892892893	0.688198369424294\\
0.893893893893894	0.692766192476544\\
0.894894894894895	0.697414689598065\\
0.895895895895896	0.702146030976947\\
0.896896896896897	0.706962466931776\\
0.897897897897898	0.711866331681984\\
0.898898898898899	0.716860047334629\\
0.8998998998999	0.721946128102318\\
0.900900900900901	0.727127184768133\\
0.901901901901902	0.732405929414694\\
0.902902902902903	0.737785180435864\\
0.903903903903904	0.7432678678511\\
0.904904904904905	0.748857038944105\\
0.905905905905906	0.754555864249217\\
0.906906906906907	0.760367643910963\\
0.907907907907908	0.766295814444307\\
0.908908908908909	0.772343955925554\\
0.90990990990991	0.778515799646391\\
0.910910910910911	0.784815236266443\\
0.911911911911912	0.791246324502835\\
0.912912912912913	0.797813300398693\\
0.913913913913914	0.804520587216335\\
0.914914914914915	0.811372806005073\\
0.915915915915916	0.818374786898218\\
0.916916916916917	0.825531581198948\\
0.917917917917918	0.832848474320428\\
0.918918918918919	0.840330999651819\\
0.91991991991992	0.847984953428811\\
0.920920920920921	0.855816410695074\\
0.921921921921922	0.863831742449632\\
0.922922922922923	0.872037634084802\\
0.923923923923924	0.880441105230046\\
0.924924924924925	0.889049531129099\\
0.925925925925926	0.897870665691108\\
0.926926926926927	0.906912666371565\\
0.927927927927928	0.916184121055662\\
0.928928928928929	0.925694077135651\\
0.92992992992993	0.935452072995093\\
0.930930930930931	0.94546817213695\\
0.931931931931932	0.955753000219582\\
0.932932932932933	0.966317785295439\\
0.933933933933934	0.977174401582041\\
0.934934934934935	0.988335417134306\\
0.935935935935936	0.999814145832239\\
0.936936936936937	1.01162470414917\\
0.937937937937938	1.02378207322417\\
0.938938938938939	1.03630216682913\\
0.93993993993994	1.04920190589769\\
0.940940940940941	1.06249930037123\\
0.941941941941942	1.07621353921849\\
0.942942942942943	1.09036508960259\\
0.943943943943944	1.10497580630456\\
0.944944944944945	1.12006905266935\\
0.945945945945946	1.13566983452327\\
0.946946946946947	1.15180494872424\\
0.947947947947948	1.16850314825537\\
0.948948948948949	1.18579532606384\\
0.94994994994995	1.20371472019014\\
0.950950950950951	1.22229714313701\\
0.951951951951952	1.24158123890578\\
0.952952952952953	1.26160877169568\\
0.953953953953954	1.2824249509378\\
0.954954954954955	1.30407879814363\\
0.955955955955956	1.32662356201736\\
0.956956956956957	1.35011718944879\\
0.957957957957958	1.37462286141491\\
0.958958958958959	1.40020960453234\\
0.95995995995996	1.42695299109234\\
0.960960960960961	1.45493594297079\\
0.961961961961962	1.48424965795746\\
0.962962962962963	1.5149946809483\\
0.963963963963964	1.54728214729528\\
0.964964964964965	1.58123523167505\\
0.965965965965966	1.61699084347045\\
0.966966966966967	1.65470161932172\\
0.967967967967968	1.6945382758161\\
0.968968968968969	1.73669240108159\\
0.96996996996997	1.78137978446673\\
0.970970970970971	1.82884441008033\\
0.971971971971972	1.87936327488662\\
0.972972972972973	1.93325223831637\\
0.973973973973974	1.99087317222251\\
0.974974974974975	2.05264276357458\\
0.975975975975976	2.11904343637489\\
0.976976976976977	2.19063701684504\\
0.977977977977978	2.26808198626059\\
0.978978978978979	2.35215547806564\\
0.97997997997998	2.44378162491698\\
0.980980980980981	2.54406851729216\\
0.981981981981982	2.65435701036163\\
0.982982982982983	2.77628609293619\\
0.983983983983984	2.9118818172298\\
0.984984984984985	3.06368040554663\\
0.985985985985986	3.23490202658145\\
0.986986986986987	3.42970156130517\\
0.987987987987988	3.65353965914621\\
0.988988988988989	3.91374784189009\\
0.98998998998999	4.22041843769625\\
0.990990990990991	4.58786237600659\\
0.991991991991992	5.037112281555\\
0.992992992992993	5.60047386889019\\
0.993993993993994	6.33041447603644\\
0.994994994994995	7.31859027032782\\
0.995995995995996	8.74190509398864\\
0.996996996996997	10.9953249913736\\
0.997997997997998	15.1961097601184\\
0.998998998998999	26.4367529085444\\
1	inf\\
};
\addlegendentry{$p(\theta)$};

\addplot [color=mycolor2,solid]
  table[row sep=crcr]{%
0	0\\
0.001001001001001	6.74958379519495e-08\\
0.002002002002002	3.1037825724429e-07\\
0.003003003003003	7.57950277145697e-07\\
0.004004004004004	1.42841677818789e-06\\
0.005005005005005	2.33564228099244e-06\\
0.00600600600600601	3.49103962662187e-06\\
0.00700700700700701	4.90442137730138e-06\\
0.00800800800800801	6.58446470807573e-06\\
0.00900900900900901	8.53899563473328e-06\\
0.01001001001001	1.07751765107319e-05\\
0.011011011011011	1.32996367128517e-05\\
0.012012012012012	1.61185676240269e-05\\
0.013013013013013	1.9237793982827e-05\\
0.014014014014014	2.26628289306321e-05\\
0.015015015015015	2.6398917429586e-05\\
0.016016016016016	3.04510711501838e-05\\
0.017017017017017	3.48240969523607e-05\\
0.018018018018018	3.95226204570336e-05\\
0.019019019019019	4.45511057888726e-05\\
0.02002002002002	4.9913872287104e-05\\
0.021021021021021	5.56151087826589e-05\\
0.022022022022022	6.16588858983032e-05\\
0.023023023023023	6.80491667253342e-05\\
0.024024024024024	7.47898161542133e-05\\
0.025025025025025	8.18846090792844e-05\\
0.026026026026026	8.93372376541793e-05\\
0.027027027027027	9.71513177409545e-05\\
0.028028028028028	0.000105330394669857\\
0.029029029029029	0.000113877948406031\\
0.03003003003003	0.000122797398203107\\
0.031031031031031	0.000132092106810513\\
0.032032032032032	0.000141765384290701\\
0.033033033033033	0.000151820491493902\\
0.034034034034034	0.000162260643230873\\
0.035035035035035	0.000173089011178308\\
0.036036036036036	0.000184308726546664\\
0.037037037037037	0.000195922882536164\\
0.038038038038038	0.000207934536603259\\
0.039039039039039	0.000220346712557004\\
0.04004004004004	0.000233162402502323\\
0.041041041041041	0.000246384568645093\\
0.042042042042042	0.000260016144972155\\
0.043043043043043	0.000274060038817864\\
0.044044044044044	0.000288519132327453\\
0.045045045045045	0.000303396283826339\\
0.046046046046046	0.000318694329103517\\
0.047047047047047	0.000334416082616316\\
0.048048048048048	0.000350564338623046\\
0.049049049049049	0.000367141872249358\\
0.0500500500500501	0.000384151440493627\\
0.0510510510510511	0.000401595783176086\\
0.0520520520520521	0.00041947762383602\\
0.0530530530530531	0.000437799670580908\\
0.0540540540540541	0.000456564616891063\\
0.0550550550550551	0.000475775142382966\\
0.0560560560560561	0.000495433913534248\\
0.0570570570570571	0.000515543584372969\\
0.0580580580580581	0.000536106797133658\\
0.0590590590590591	0.000557126182882355\\
0.0600600600600601	0.000578604362112696\\
0.0610610610610611	0.000600543945314941\\
0.0620620620620621	0.000622947533519665\\
0.0630630630630631	0.000645817718817738\\
0.0640640640640641	0.000669157084858051\\
0.0650650650650651	0.000692968207324343\\
0.0660660660660661	0.000717253654392419\\
0.0670670670670671	0.000742015987168902\\
0.0680680680680681	0.000767257760112609\\
0.0690690690690691	0.00079298152143956\\
0.0700700700700701	0.000819189813512549\\
0.0710710710710711	0.000845885173216155\\
0.0720720720720721	0.000873070132317987\\
0.0730730730730731	0.000900747217816934\\
0.0740740740740741	0.000928918952279117\\
0.0750750750750751	0.000957587854162193\\
0.0760760760760761	0.000986756438128648\\
0.0770770770770771	0.00101642721534863\\
0.0780780780780781	0.00104660269379289\\
0.0790790790790791	0.00107728537851627\\
0.0800800800800801	0.00110847777193236\\
0.0810810810810811	0.00114018237407957\\
0.0820820820820821	0.00117240168287917\\
0.0830830830830831	0.00120513819438572\\
0.0840840840840841	0.00123839440303011\\
0.0850850850850851	0.00127217280185572\\
0.0860860860860861	0.00130647588274791\\
0.0870870870870871	0.00134130613665727\\
0.0880880880880881	0.00137666605381675\\
0.0890890890890891	0.00141255812395318\\
0.0900900900900901	0.00144898483649324\\
0.0910910910910911	0.00148594868076418\\
0.0920920920920921	0.00152345214618961\\
0.0930930930930931	0.00156149772248051\\
0.0940940940940941	0.00160008789982167\\
0.0950950950950951	0.00163922516905376\\
0.0960960960960961	0.00167891202185123\\
0.0970970970970971	0.00171915095089629\\
0.0980980980980981	0.001759944450049\\
0.0990990990990991	0.00180129501451374\\
0.1001001001001	0.00184320514100215\\
0.101101101101101	0.0018856773278928\\
0.102102102102102	0.00192871407538757\\
0.103103103103103	0.00197231788566501\\
0.104104104104104	0.00201649126303069\\
0.105105105105105	0.00206123671406482\\
0.106106106106106	0.00210655674776707\\
0.107107107107107	0.00215245387569884\\
0.108108108108108	0.00219893061212304\\
0.109109109109109	0.00224598947414146\\
0.11011011011011	0.00229363298182986\\
0.111111111111111	0.00234186365837085\\
0.112112112112112	0.00239068403018471\\
0.113113113113113	0.00244009662705804\\
0.114114114114114	0.00249010398227068\\
0.115115115115115	0.00254070863272057\\
0.116116116116116	0.00259191311904694\\
0.117117117117117	0.00264371998575168\\
0.118118118118118	0.00269613178131919\\
0.119119119119119	0.0027491510583345\\
0.12012012012012	0.00280278037359998\\
0.121121121121121	0.00285702228825051\\
0.122122122122122	0.00291187936786733\\
0.123123123123123	0.00296735418259047\\
0.124124124124124	0.0030234493072299\\
0.125125125125125	0.00308016732137551\\
0.126126126126126	0.00313751080950581\\
0.127127127127127	0.00319548236109553\\
0.128128128128128	0.00325408457072215\\
0.129129129129129	0.00331332003817137\\
0.13013013013013	0.0033731913685415\\
0.131131131131131	0.00343370117234703\\
0.132132132132132	0.00349485206562115\\
0.133133133133133	0.00355664667001744\\
0.134134134134134	0.00361908761291076\\
0.135135135135135	0.00368217752749719\\
0.136136136136136	0.00374591905289339\\
0.137137137137137	0.00381031483423506\\
0.138138138138138	0.00387536752277475\\
0.139139139139139	0.00394107977597902\\
0.14014014014014	0.00400745425762496\\
0.141141141141141	0.00407449363789599\\
0.142142142142142	0.00414220059347726\\
0.143143143143143	0.00421057780765031\\
0.144144144144144	0.00427962797038734\\
0.145145145145145	0.00434935377844488\\
0.146146146146146	0.00441975793545704\\
0.147147147147147	0.0044908431520283\\
0.148148148148148	0.00456261214582585\\
0.149149149149149	0.00463506764167157\\
0.15015015015015	0.00470821237163357\\
0.151151151151151	0.00478204907511743\\
0.152152152152152	0.00485658049895703\\
0.153153153153153	0.00493180939750519\\
0.154154154154154	0.00500773853272386\\
0.155155155155155	0.00508437067427418\\
0.156156156156156	0.00516170859960613\\
0.157157157157157	0.00523975509404811\\
0.158158158158158	0.0053185129508962\\
0.159159159159159	0.00539798497150322\\
0.16016016016016	0.00547817396536764\\
0.161161161161161	0.00555908275022237\\
0.162162162162162	0.00564071415212325\\
0.163163163163163	0.00572307100553759\\
0.164164164164164	0.00580615615343247\\
0.165165165165165	0.005889972447363\\
0.166166166166166	0.00597452274756049\\
0.167167167167167	0.00605980992302051\\
0.168168168168168	0.00614583685159096\\
0.169169169169169	0.00623260642006002\\
0.17017017017017	0.00632012152424419\\
0.171171171171171	0.00640838506907619\\
0.172172172172172	0.00649739996869295\\
0.173173173173173	0.00658716914652358\\
0.174174174174174	0.00667769553537737\\
0.175175175175175	0.00676898207753187\\
0.176176176176176	0.00686103172482096\\
0.177177177177177	0.00695384743872301\\
0.178178178178178	0.00704743219044918\\
0.179179179179179	0.0071417889610317\\
0.18018018018018	0.00723692074141234\\
0.181181181181181	0.00733283053253099\\
0.182182182182182	0.00742952134541429\\
0.183183183183183	0.00752699620126449\\
0.184184184184184	0.00762525813154842\\
0.185185185185185	0.00772431017808658\\
0.186186186186186	0.00782415539314248\\
0.187187187187187	0.00792479683951212\\
0.188188188188188	0.00802623759061362\\
0.189189189189189	0.00812848073057714\\
0.19019019019019	0.00823152935433493\\
0.191191191191191	0.00833538656771166\\
0.192192192192192	0.00844005548751499\\
0.193193193193193	0.00854553924162629\\
0.194194194194194	0.00865184096909172\\
0.195195195195195	0.00875896382021352\\
0.196196196196196	0.0088669109566416\\
0.197197197197197	0.00897568555146535\\
0.198198198198198	0.00908529078930584\\
0.199199199199199	0.00919572986640821\\
0.2002002002002	0.00930700599073447\\
0.201201201201201	0.00941912238205657\\
0.202202202202202	0.00953208227204976\\
0.203203203203203	0.00964588890438639\\
0.204204204204204	0.00976054553482997\\
0.205205205205205	0.00987605543132957\\
0.206206206206206	0.00999242187411473\\
0.207207207207207	0.0101096481557906\\
0.208208208208208	0.0102277375814333\\
0.209209209209209	0.0103466934686864\\
0.21021021021021	0.0104665191478566\\
0.211211211211211	0.010587217962011\\
0.212212212212212	0.0107087932670741\\
0.213213213213213	0.0108312484319254\\
0.214214214214214	0.0109545868384975\\
0.215215215215215	0.0110788118818745\\
0.216216216216216	0.011203926970391\\
0.217217217217217	0.0113299355257314\\
0.218218218218218	0.0114568409830301\\
0.219219219219219	0.0115846467909713\\
0.22022022022022	0.0117133564118904\\
0.221221221221221	0.0118429733218747\\
0.222222222222222	0.011973501010866\\
0.223223223223223	0.0121049429827623\\
0.224224224224224	0.0122373027555209\\
0.225225225225225	0.0123705838612617\\
0.226226226226226	0.0125047898463714\\
0.227227227227227	0.0126399242716076\\
0.228228228228228	0.0127759907122041\\
0.229229229229229	0.0129129927579764\\
0.23023023023023	0.0130509340134277\\
0.231231231231231	0.0131898180978561\\
0.232232232232232	0.0133296486454613\\
0.233233233233233	0.0134704293054531\\
0.234234234234234	0.0136121637421594\\
0.235235235235235	0.0137548556351359\\
0.236236236236236	0.0138985086792755\\
0.237237237237237	0.0140431265849188\\
0.238238238238238	0.014188713077965\\
0.239239239239239	0.0143352718999838\\
0.24024024024024	0.0144828068083278\\
0.241241241241241	0.0146313215762453\\
0.242242242242242	0.0147808199929938\\
0.243243243243243	0.0149313058639551\\
0.244244244244244	0.0150827830107494\\
0.245245245245245	0.0152352552713519\\
0.246246246246246	0.0153887265002088\\
0.247247247247247	0.0155432005683546\\
0.248248248248248	0.01569868136353\\
0.249249249249249	0.0158551727903008\\
0.25025025025025	0.0160126787701767\\
0.251251251251251	0.0161712032417323\\
0.252252252252252	0.0163307501607273\\
0.253253253253253	0.0164913235002286\\
0.254254254254254	0.0166529272507326\\
0.255255255255255	0.0168155654202884\\
0.256256256256256	0.0169792420346223\\
0.257257257257257	0.0171439611372622\\
0.258258258258258	0.0173097267896635\\
0.259259259259259	0.0174765430713357\\
0.26026026026026	0.0176444140799696\\
0.261261261261261	0.0178133439315658\\
0.262262262262262	0.0179833367605635\\
0.263263263263263	0.0181543967199707\\
0.264264264264264	0.0183265279814948\\
0.265265265265265	0.0184997347356744\\
0.266266266266266	0.0186740211920122\\
0.267267267267267	0.018849391579108\\
0.268268268268268	0.0190258501447937\\
0.269269269269269	0.0192034011562683\\
0.27027027027027	0.0193820489002343\\
0.271271271271271	0.0195617976830351\\
0.272272272272272	0.0197426518307932\\
0.273273273273273	0.0199246156895493\\
0.274274274274274	0.0201076936254025\\
0.275275275275275	0.0202918900246517\\
0.276276276276276	0.0204772092939373\\
0.277277277277277	0.0206636558603851\\
0.278278278278278	0.0208512341717497\\
0.279279279279279	0.0210399486965605\\
0.28028028028028	0.0212298039242677\\
0.281281281281281	0.0214208043653893\\
0.282282282282282	0.0216129545516605\\
0.283283283283283	0.021806259036182\\
0.284284284284284	0.0220007223935716\\
0.285285285285285	0.0221963492201154\\
0.286286286286286	0.022393144133921\\
0.287287287287287	0.022591111775071\\
0.288288288288288	0.0227902568057784\\
0.289289289289289	0.022990583910543\\
0.29029029029029	0.0231920977963083\\
0.291291291291291	0.0233948031926208\\
0.292292292292292	0.0235987048517892\\
0.293293293293293	0.0238038075490457\\
0.294294294294294	0.0240101160827082\\
0.295295295295295	0.0242176352743435\\
0.296296296296296	0.0244263699689323\\
0.297297297297297	0.0246363250350349\\
0.298298298298298	0.0248475053649586\\
0.299299299299299	0.025059915874926\\
0.3003003003003	0.0252735615052447\\
0.301301301301301	0.0254884472204788\\
0.302302302302302	0.0257045780096208\\
0.303303303303303	0.0259219588862658\\
0.304304304304304	0.026140594888786\\
0.305305305305305	0.0263604910805079\\
0.306306306306306	0.0265816525498894\\
0.307307307307307	0.0268040844106996\\
0.308308308308308	0.0270277918021992\\
0.309309309309309	0.0272527798893223\\
0.31031031031031	0.0274790538628604\\
0.311311311311311	0.0277066189396468\\
0.312312312312312	0.0279354803627434\\
0.313313313313313	0.0281656434016285\\
0.314314314314314	0.028397113352386\\
0.315315315315315	0.0286298955378964\\
0.316316316316316	0.0288639953080294\\
0.317317317317317	0.0290994180398375\\
0.318318318318318	0.029336169137752\\
0.319319319319319	0.0295742540337794\\
0.32032032032032	0.029813678187701\\
0.321321321321321	0.0300544470872724\\
0.322322322322322	0.030296566248426\\
0.323323323323323	0.0305400412154741\\
0.324324324324324	0.0307848775613146\\
0.325325325325325	0.0310310808876375\\
0.326326326326326	0.0312786568251338\\
0.327327327327327	0.0315276110337058\\
0.328328328328328	0.031777949202679\\
0.329329329329329	0.032029677051016\\
0.33033033033033	0.032282800327532\\
0.331331331331331	0.0325373248111124\\
0.332332332332332	0.0327932563109316\\
0.333333333333333	0.0330506006666742\\
0.334334334334334	0.033309363748758\\
0.335335335335335	0.0335695514585584\\
0.336336336336336	0.0338311697286349\\
0.337337337337337	0.0340942245229601\\
0.338338338338338	0.0343587218371495\\
0.339339339339339	0.0346246676986943\\
0.34034034034034	0.0348920681671954\\
0.341341341341341	0.0351609293345999\\
0.342342342342342	0.0354312573254393\\
0.343343343343343	0.0357030582970698\\
0.344344344344344	0.0359763384399147\\
0.345345345345345	0.0362511039777091\\
0.346346346346346	0.0365273611677457\\
0.347347347347347	0.0368051163011244\\
0.348348348348348	0.0370843757030023\\
0.349349349349349	0.0373651457328471\\
0.35035035035035	0.0376474327846921\\
0.351351351351351	0.0379312432873933\\
0.352352352352352	0.0382165837048894\\
0.353353353353353	0.038503460536463\\
0.354354354354354	0.0387918803170055\\
0.355355355355355	0.0390818496172826\\
0.356356356356356	0.0393733750442037\\
0.357357357357357	0.0396664632410924\\
0.358358358358358	0.0399611208879606\\
0.359359359359359	0.0402573547017836\\
0.36036036036036	0.0405551714367788\\
0.361361361361361	0.0408545778846865\\
0.362362362362362	0.0411555808750526\\
0.363363363363363	0.0414581872755149\\
0.364364364364364	0.0417624039920908\\
0.365365365365365	0.0420682379694686\\
0.366366366366366	0.0423756961913003\\
0.367367367367367	0.0426847856804981\\
0.368368368368368	0.0429955134995328\\
0.369369369369369	0.0433078867507353\\
0.37037037037037	0.0436219125766002\\
0.371371371371371	0.0439375981600931\\
0.372372372372372	0.0442549507249596\\
0.373373373373373	0.0445739775360381\\
0.374374374374374	0.0448946858995744\\
0.375375375375375	0.0452170831635402\\
0.376376376376376	0.0455411767179536\\
0.377377377377377	0.0458669739952028\\
0.378378378378378	0.0461944824703734\\
0.379379379379379	0.0465237096615775\\
0.38038038038038	0.0468546631302866\\
0.381381381381381	0.0471873504816676\\
0.382382382382382	0.0475217793649213\\
0.383383383383383	0.0478579574736247\\
0.384384384384384	0.0481958925460759\\
0.385385385385385	0.0485355923656424\\
0.386386386386386	0.0488770647611127\\
0.387387387387387	0.0492203176070513\\
0.388388388388388	0.0495653588241561\\
0.389389389389389	0.0499121963796206\\
0.39039039039039	0.0502608382874981\\
0.391391391391391	0.0506112926090702\\
0.392392392392392	0.050963567453218\\
0.393393393393393	0.0513176709767978\\
0.394394394394394	0.0516736113850191\\
0.395395395395395	0.0520313969318272\\
0.396396396396396	0.0523910359202888\\
0.397397397397397	0.0527525367029813\\
0.398398398398398	0.0531159076823862\\
0.399399399399399	0.0534811573112856\\
0.4004004004004	0.0538482940931629\\
0.401401401401401	0.0542173265826072\\
0.402402402402402	0.0545882633857216\\
0.403403403403403	0.0549611131605353\\
0.404404404404404	0.0553358846174194\\
0.405405405405405	0.0557125865195076\\
0.406406406406406	0.0560912276831195\\
0.407407407407407	0.0564718169781896\\
0.408408408408408	0.0568543633286986\\
0.409409409409409	0.057238875713111\\
0.41041041041041	0.0576253631648147\\
0.411411411411411	0.0580138347725669\\
0.412412412412412	0.0584042996809426\\
0.413413413413413	0.0587967670907886\\
0.414414414414414	0.0591912462596818\\
0.415415415415415	0.0595877465023911\\
0.416416416416416	0.0599862771913447\\
0.417417417417417	0.0603868477571023\\
0.418418418418418	0.0607894676888303\\
0.419419419419419	0.0611941465347836\\
0.42042042042042	0.0616008939027908\\
0.421421421421421	0.0620097194607448\\
0.422422422422422	0.0624206329370981\\
0.423423423423423	0.0628336441213629\\
0.424424424424424	0.0632487628646165\\
0.425425425425425	0.0636659990800115\\
0.426426426426426	0.0640853627432905\\
0.427427427427427	0.0645068638933074\\
0.428428428428428	0.0649305126325523\\
0.429429429429429	0.0653563191276827\\
0.43043043043043	0.0657842936100598\\
0.431431431431431	0.0662144463762897\\
0.432432432432432	0.0666467877887711\\
0.433433433433433	0.067081328276247\\
0.434434434434434	0.0675180783343638\\
0.435435435435435	0.0679570485262345\\
0.436436436436436	0.0683982494830083\\
0.437437437437437	0.0688416919044461\\
0.438438438438438	0.0692873865595016\\
0.439439439439439	0.0697353442869085\\
0.44044044044044	0.0701855759957736\\
0.441441441441441	0.0706380926661756\\
0.442442442442442	0.0710929053497712\\
0.443443443443443	0.0715500251704061\\
0.444444444444444	0.0720094633247329\\
0.445445445445445	0.0724712310828357\\
0.446446446446446	0.0729353397888604\\
0.447447447447447	0.0734018008616522\\
0.448448448448448	0.0738706257953992\\
0.449449449449449	0.0743418261602836\\
0.45045045045045	0.0748154136031379\\
0.451451451451451	0.0752913998481101\\
0.452452452452452	0.0757697966973345\\
0.453453453453453	0.0762506160316096\\
0.454454454454454	0.0767338698110839\\
0.455455455455455	0.0772195700759478\\
0.456456456456456	0.0777077289471341\\
0.457457457457457	0.0781983586270245\\
0.458458458458458	0.0786914714001647\\
0.459459459459459	0.0791870796339867\\
0.46046046046046	0.0796851957795383\\
0.461461461461461	0.0801858323722212\\
0.462462462462462	0.0806890020325363\\
0.463463463463463	0.0811947174668374\\
0.464464464464464	0.0817029914680928\\
0.465465465465465	0.0822138369166547\\
0.466466466466466	0.0827272667810377\\
0.467467467467467	0.083243294118705\\
0.468468468468468	0.0837619320768631\\
0.469469469469469	0.0842831938932655\\
0.47047047047047	0.084807092897025\\
0.471471471471471	0.0853336425094342\\
0.472472472472472	0.0858628562447962\\
0.473473473473473	0.0863947477112627\\
0.474474474474474	0.0869293306116827\\
0.475475475475475	0.0874666187444597\\
0.476476476476476	0.0880066260044184\\
0.477477477477477	0.0885493663836813\\
0.478478478478478	0.0890948539725543\\
0.479479479479479	0.0896431029604226\\
0.48048048048048	0.0901941276366563\\
0.481481481481481	0.0907479423915259\\
0.482482482482482	0.0913045617171284\\
0.483483483483483	0.091864000208323\\
0.484484484484485	0.0924262725636779\\
0.485485485485485	0.092991393586427\\
0.486486486486487	0.0935593781854379\\
0.487487487487487	0.0941302413761905\\
0.488488488488488	0.0947039982817666\\
0.48948948948949	0.0952806641338505\\
0.49049049049049	0.0958602542737407\\
0.491491491491492	0.0964427841533739\\
0.492492492492492	0.0970282693363596\\
0.493493493493493	0.0976167254990268\\
0.494494494494495	0.0982081684314829\\
0.495495495495495	0.098802614038684\\
0.496496496496497	0.0994000783415179\\
0.497497497497497	0.100000577477899\\
0.498498498498498	0.100604127703878\\
0.4994994994995	0.101210745394759\\
0.500500500500501	0.101820447046235\\
0.501501501501502	0.102433249275536\\
0.502502502502503	0.103049168822584\\
0.503503503503503	0.103668222551171\\
0.504504504504504	0.104290427450143\\
0.505505505505506	0.1049158006346\\
0.506506506506507	0.105544359347114\\
0.507507507507508	0.106176120958952\\
0.508508508508508	0.106811102971325\\
0.509509509509509	0.107449323016642\\
0.510510510510511	0.108090798859783\\
0.511511511511512	0.108735548399385\\
0.512512512512513	0.109383589669148\\
0.513513513513513	0.110034940839149\\
0.514514514514514	0.11068962021718\\
0.515515515515516	0.111347646250091\\
0.516516516516517	0.112009037525161\\
0.517517517517518	0.112673812771477\\
0.518518518518518	0.113341990861332\\
0.519519519519519	0.114013590811642\\
0.520520520520521	0.114688631785373\\
0.521521521521522	0.115367133092997\\
0.522522522522523	0.116049114193951\\
0.523523523523523	0.116734594698127\\
0.524524524524524	0.11742359436737\\
0.525525525525526	0.118116133117002\\
0.526526526526527	0.118812231017356\\
0.527527527527528	0.11951190829534\\
0.528528528528528	0.120215185336006\\
0.529529529529529	0.120922082684152\\
0.530530530530531	0.121632621045932\\
0.531531531531532	0.122346821290496\\
0.532532532532533	0.123064704451643\\
0.533533533533533	0.123786291729495\\
0.534534534534535	0.124511604492196\\
0.535535535535536	0.125240664277629\\
0.536536536536537	0.125973492795153\\
0.537537537537538	0.126710111927365\\
0.538538538538539	0.127450543731879\\
0.53953953953954	0.128194810443138\\
0.540540540540541	0.12894293447423\\
0.541541541541542	0.12969493841875\\
0.542542542542543	0.130450845052662\\
0.543543543543544	0.131210677336206\\
0.544544544544545	0.131974458415811\\
0.545545545545546	0.132742211626045\\
0.546546546546547	0.133513960491581\\
0.547547547547548	0.134289728729197\\
0.548548548548549	0.135069540249791\\
0.54954954954955	0.135853419160431\\
0.550550550550551	0.136641389766425\\
0.551551551551552	0.137433476573422\\
0.552552552552553	0.138229704289536\\
0.553553553553554	0.139030097827499\\
0.554554554554555	0.139834682306846\\
0.555555555555556	0.140643483056119\\
0.556556556556557	0.141456525615109\\
0.557557557557558	0.142273835737123\\
0.558558558558559	0.143095439391277\\
0.55955955955956	0.143921362764827\\
0.560560560560561	0.144751632265526\\
0.561561561561562	0.145586274524009\\
0.562562562562563	0.146425316396216\\
0.563563563563564	0.14726878496584\\
0.564564564564565	0.148116707546816\\
0.565565565565566	0.148969111685832\\
0.566566566566567	0.149826025164885\\
0.567567567567568	0.15068747600386\\
0.568568568568569	0.151553492463152\\
0.56956956956957	0.152424103046321\\
0.570570570570571	0.153299336502777\\
0.571571571571572	0.154179221830508\\
0.572572572572573	0.155063788278844\\
0.573573573573574	0.155953065351252\\
0.574574574574575	0.156847082808177\\
0.575575575575576	0.157745870669919\\
0.576576576576577	0.158649459219542\\
0.577577577577578	0.159557879005837\\
0.578578578578579	0.160471160846308\\
0.57957957957958	0.161389335830217\\
0.580580580580581	0.162312435321656\\
0.581581581581582	0.163240490962669\\
0.582582582582583	0.164173534676416\\
0.583583583583584	0.165111598670379\\
0.584584584584585	0.166054715439613\\
0.585585585585586	0.167002917770045\\
0.586586586586587	0.167956238741814\\
0.587587587587588	0.168914711732661\\
0.588588588588589	0.169878370421367\\
0.58958958958959	0.170847248791238\\
0.590590590590591	0.171821381133636\\
0.591591591591592	0.172800802051565\\
0.592592592592593	0.173785546463309\\
0.593593593593594	0.174775649606111\\
0.594594594594595	0.175771147039916\\
0.595595595595596	0.176772074651163\\
0.596596596596597	0.177778468656628\\
0.597597597597598	0.178790365607328\\
0.598598598598599	0.179807802392476\\
0.5995995995996	0.180830816243494\\
0.600600600600601	0.18185944473809\\
0.601601601601602	0.18289372580438\\
0.602602602602603	0.183933697725086\\
0.603603603603604	0.184979399141785\\
0.604604604604605	0.186030869059222\\
0.605605605605606	0.187088146849686\\
0.606606606606607	0.188151272257451\\
0.607607607607608	0.189220285403284\\
0.608608608608609	0.190295226789014\\
0.60960960960961	0.191376137302172\\
0.610610610610611	0.1924630582207\\
0.611611611611612	0.193556031217732\\
0.612612612612613	0.194655098366438\\
0.613613613613614	0.195760302144947\\
0.614614614614615	0.196871685441347\\
0.615615615615616	0.197989291558748\\
0.616616616616617	0.199113164220433\\
0.617617617617618	0.200243347575081\\
0.618618618618619	0.201379886202067\\
0.61961961961962	0.202522825116851\\
0.620620620620621	0.203672209776437\\
0.621621621621622	0.204828086084925\\
0.622622622622623	0.205990500399142\\
0.623623623623624	0.207159499534362\\
0.624624624624625	0.208335130770115\\
0.625625625625626	0.209517441856079\\
0.626626626626627	0.210706481018073\\
0.627627627627628	0.211902296964134\\
0.628628628628629	0.21310493889069\\
0.62962962962963	0.214314456488837\\
0.630630630630631	0.2155308999507\\
0.631631631631632	0.216754319975908\\
0.632632632632633	0.217984767778161\\
0.633633633633634	0.219222295091901\\
0.634634634634635	0.220466954179095\\
0.635635635635636	0.221718797836119\\
0.636636636636637	0.222977879400753\\
0.637637637637638	0.224244252759287\\
0.638638638638639	0.225517972353741\\
0.63963963963964	0.226799093189202\\
0.640640640640641	0.228087670841277\\
0.641641641641642	0.229383761463662\\
0.642642642642643	0.230687421795843\\
0.643643643643644	0.231998709170912\\
0.644644644644645	0.233317681523517\\
0.645645645645646	0.234644397397932\\
0.646646646646647	0.235978915956274\\
0.647647647647648	0.237321296986836\\
0.648648648648649	0.238671600912574\\
0.64964964964965	0.240029888799719\\
0.650650650650651	0.241396222366544\\
0.651651651651652	0.242770663992266\\
0.652652652652653	0.244153276726103\\
0.653653653653654	0.245544124296475\\
0.654654654654655	0.246943271120367\\
0.655655655655656	0.248350782312843\\
0.656656656656657	0.24976672369672\\
0.657657657657658	0.251191161812407\\
0.658658658658659	0.252624163927909\\
0.65965965965966	0.254065798049003\\
0.660660660660661	0.255516132929583\\
0.661661661661662	0.256975238082185\\
0.662662662662663	0.258443183788691\\
0.663663663663664	0.259920041111212\\
0.664664664664665	0.26140588190317\\
0.665665665665666	0.262900778820555\\
0.666666666666667	0.264404805333394\\
0.667667667667668	0.265918035737403\\
0.668668668668669	0.267440545165854\\
0.66966966966967	0.268972409601644\\
0.670670670670671	0.270513705889577\\
0.671671671671672	0.272064511748858\\
0.672672672672673	0.273624905785815\\
0.673673673673674	0.275194967506837\\
0.674674674674675	0.276774777331548\\
0.675675675675676	0.278364416606215\\
0.676676676676677	0.27996396761739\\
0.677677677677678	0.281573513605802\\
0.678678678678679	0.283193138780494\\
0.67967967967968	0.284822928333223\\
0.680680680680681	0.286462968453106\\
0.681681681681682	0.288113346341546\\
0.682682682682683	0.289774150227422\\
0.683683683683684	0.291445469382554\\
0.684684684684685	0.29312739413746\\
0.685685685685686	0.294820015897394\\
0.686686686686687	0.296523427158681\\
0.687687687687688	0.298237721525359\\
0.688688688688689	0.299962993726122\\
0.68968968968969	0.301699339631584\\
0.690690690690691	0.303446856271864\\
0.691691691691692	0.305205641854496\\
0.692692692692693	0.306975795782685\\
0.693693693693694	0.308757418673892\\
0.694694694694695	0.310550612378788\\
0.695695695695696	0.312355480000555\\
0.696696696696697	0.314172125914556\\
0.697697697697698	0.316000655788388\\
0.698698698698699	0.31784117660231\\
0.6996996996997	0.319693796670068\\
0.700700700700701	0.321558625660125\\
0.701701701701702	0.323435774617296\\
0.702702702702703	0.325325355984804\\
0.703703703703704	0.327227483626771\\
0.704704704704705	0.329142272851142\\
0.705705705705706	0.331069840433064\\
0.706706706706707	0.333010304638722\\
0.707707707707708	0.334963785249649\\
0.708708708708709	0.336930403587512\\
0.70970970970971	0.338910282539397\\
0.710710710710711	0.340903546583599\\
0.711711711711712	0.342910321815924\\
0.712712712712713	0.34493073597652\\
0.713713713713714	0.346964918477254\\
0.714714714714715	0.349013000429642\\
0.715715715715716	0.351075114673338\\
0.716716716716717	0.353151395805218\\
0.717717717717718	0.355241980209046\\
0.718718718718719	0.357347006085757\\
0.71971971971972	0.359466613484361\\
0.720720720720721	0.361600944333489\\
0.721721721721722	0.363750142473587\\
0.722722722722723	0.365914353689787\\
0.723723723723724	0.368093725745463\\
0.724724724724725	0.370288408416493\\
0.725725725725726	0.37249855352624\\
0.726726726726727	0.374724314981268\\
0.727727727727728	0.376965848807831\\
0.728728728728729	0.379223313189117\\
0.72972972972973	0.381496868503311\\
0.730730730730731	0.383786677362455\\
0.731731731731732	0.386092904652156\\
0.732732732732733	0.388415717572151\\
0.733733733733734	0.390755285677746\\
0.734734734734735	0.393111780922169\\
0.735735735735736	0.395485377699832\\
0.736736736736737	0.397876252890562\\
0.737737737737738	0.400284585904787\\
0.738738738738739	0.402710558729738\\
0.73973973973974	0.405154355976665\\
0.740740740740741	0.407616164929112\\
0.741741741741742	0.410096175592264\\
0.742742742742743	0.412594580743415\\
0.743743743743744	0.415111575983561\\
0.744744744744745	0.417647359790166\\
0.745745745745746	0.420202133571132\\
0.746746746746747	0.422776101719991\\
0.747747747747748	0.425369471672367\\
0.748748748748749	0.427982453963738\\
0.74974974974975	0.430615262288524\\
0.750750750750751	0.433268113560555\\
0.751751751751752	0.43594122797494\\
0.752752752752753	0.43863482907138\\
0.753753753753754	0.441349143798975\\
0.754754754754755	0.444084402582551\\
0.755755755755756	0.446840839390558\\
0.756756756756757	0.449618691804584\\
0.757757757757758	0.452418201090521\\
0.758758758758759	0.455239612271445\\
0.75975975975976	0.45808317420224\\
0.760760760760761	0.46094913964603\\
0.761761761761762	0.463837765352466\\
0.762762762762763	0.466749312137915\\
0.763763763763764	0.469684044967614\\
0.764764764764765	0.472642233039848\\
0.765765765765766	0.475624149872192\\
0.766766766766767	0.478630073389904\\
0.767767767767768	0.481660286016507\\
0.768768768768769	0.484715074766646\\
0.76976976976977	0.487794731341264\\
0.770770770770771	0.490899552225189\\
0.771771771771772	0.494029838787187\\
0.772772772772773	0.497185897382561\\
0.773773773773774	0.50036803945837\\
0.774774774774775	0.503576581661354\\
0.775775775775776	0.506811845948636\\
0.776776776776777	0.510074159701294\\
0.777777777777778	0.513363855840882\\
0.778778778778779	0.516681272949006\\
0.77977977977978	0.520026755390032\\
0.780780780780781	0.523400653437027\\
0.781781781781782	0.526803323401047\\
0.782782782782783	0.530235127763853\\
0.783783783783784	0.533696435314191\\
0.784784784784785	0.537187621287717\\
0.785785785785786	0.540709067510713\\
0.786786786786787	0.544261162547687\\
0.787787787787788	0.547844301853007\\
0.788788788788789	0.551458887926675\\
0.78978978978979	0.555105330474396\\
0.790790790790791	0.55878404657206\\
0.791791791791792	0.562495460834801\\
0.792792792792793	0.566240005590771\\
0.793793793793794	0.570018121059782\\
0.794794794794795	0.573830255536991\\
0.795795795795796	0.577676865581779\\
0.796796796796797	0.58155841621201\\
0.797797797797798	0.585475381103838\\
0.798798798798799	0.58942824279727\\
0.7997997997998	0.593417492907651\\
0.800800800800801	0.597443632343303\\
0.801801801801802	0.601507171529503\\
0.802802802802803	0.605608630639037\\
0.803803803803804	0.609748539829548\\
0.804804804804805	0.613927439487913\\
0.805805805805806	0.618145880481907\\
0.806806806806807	0.622404424419384\\
0.807807807807808	0.626703643915281\\
0.808808808808809	0.631044122866675\\
0.80980980980981	0.635426456736225\\
0.810810810810811	0.639851252844264\\
0.811811811811812	0.644319130669878\\
0.812812812812813	0.648830722161285\\
0.813813813813814	0.653386672055854\\
0.814814814814815	0.657987638210127\\
0.815815815815816	0.662634291940195\\
0.816816816816817	0.667327318372832\\
0.817817817817818	0.67206741680777\\
0.818818818818819	0.676855301091548\\
0.81981981981982	0.681691700003354\\
0.820820820820821	0.686577357653336\\
0.821821821821822	0.691513033893827\\
0.822822822822823	0.696499504744019\\
0.823823823823824	0.701537562828563\\
0.824824824824825	0.706628017830663\\
0.825825825825826	0.711771696960214\\
0.826826826826827	0.716969445437587\\
0.827827827827828	0.722222126993662\\
0.828828828828829	0.727530624386762\\
0.82982982982983	0.732895839937173\\
0.830830830830831	0.738318696079928\\
0.831831831831832	0.743800135936618\\
0.832832832832833	0.749341123906998\\
0.833833833833834	0.754942646281178\\
0.834834834834835	0.760605711873277\\
0.835835835835836	0.766331352677398\\
0.836836836836837	0.772120624546865\\
0.837837837837838	0.777974607897702\\
0.838838838838839	0.78389440843737\\
0.83983983983984	0.789881157919822\\
0.840840840840841	0.795936014928018\\
0.841841841841842	0.802060165685063\\
0.842842842842843	0.80825482489522\\
0.843843843843844	0.814521236616076\\
0.844844844844845	0.820860675163245\\
0.845845845845846	0.827274446049025\\
0.846846846846847	0.833763886956534\\
0.847847847847848	0.840330368750882\\
0.848848848848849	0.846975296529061\\
0.84984984984985	0.853700110710308\\
0.850850850850851	0.860506288168758\\
0.851851851851852	0.867395343410362\\
0.852852852852853	0.874368829796075\\
0.853853853853854	0.881428340813495\\
0.854854854854855	0.8885755113992\\
0.855855855855856	0.895812019314173\\
0.856856856856857	0.903139586574853\\
0.857857857857858	0.910559980942439\\
0.858858858858859	0.91807501747327\\
0.85985985985986	0.925686560133239\\
0.860860860860861	0.933396523479359\\
0.861861861861862	0.941206874411782\\
0.862862862862863	0.949119633999774\\
0.863863863863864	0.957136879385312\\
0.864864864864865	0.965260745768223\\
0.865865865865866	0.973493428476981\\
0.866866866866867	0.981837185129544\\
0.867867867867868	0.990294337888839\\
0.868868868868869	0.998867275817824\\
0.86986986986987	1.00755845733929\\
0.870870870870871	1.01637041280594\\
0.871871871871872	1.02530574718657\\
0.872872872872873	1.03436714287459\\
0.873873873873874	1.0435573626254\\
0.874874874874875	1.0528792526297\\
0.875875875875876	1.0623357457301\\
0.876876876876877	1.071929864789\\
0.877877877877878	1.08166472621609\\
0.878878878878879	1.09154354366443\\
0.87987987987988	1.10156963190474\\
0.880880880880881	1.11174641088783\\
0.881881881881882	1.12207741000635\\
0.882882882882883	1.13256627256695\\
0.883883883883884	1.14321676048566\\
0.884884884884885	1.1540327592192\\
0.885885885885886	1.16501828294661\\
0.886886886886887	1.17617748001603\\
0.887887887887888	1.18751463867278\\
0.888888888888889	1.19903419308588\\
0.88988988988989	1.21074072969145\\
0.890890890890891	1.22263899387275\\
0.891891891891892	1.2347338969979\\
0.892892892892893	1.24703052383811\\
0.893893893893894	1.25953414039059\\
0.894894894894895	1.27225020213253\\
0.895895895895896	1.28518436273398\\
0.896896896896897	1.29834248326018\\
0.897897897897898	1.31173064189557\\
0.898898898898899	1.32535514422485\\
0.8998998998999	1.3392225341086\\
0.900900900900901	1.35333960519439\\
0.901901901901902	1.36771341310732\\
0.902902902902903	1.3823512883675\\
0.903903903903904	1.3972608500859\\
0.904904904904905	1.41245002049407\\
0.905905905905906	1.42792704036805\\
0.906906906906907	1.44370048541157\\
0.907907907907908	1.45977928366937\\
0.908908908908909	1.47617273404749\\
0.90990990990991	1.49289052602382\\
0.910910910910911	1.50994276063986\\
0.911911911911912	1.5273399728723\\
0.912912912912913	1.54509315549209\\
0.913913913913914	1.56321378452848\\
0.914914914914915	1.58171384646589\\
0.915915915915916	1.60060586731379\\
0.916916916916917	1.61990294370266\\
0.917917917917918	1.63961877617358\\
0.918918918918919	1.65976770484531\\
0.91991991991992	1.68036474766045\\
0.920920920920921	1.70142564143217\\
0.921921921921922	1.72296688593518\\
0.922922922922923	1.74500579130905\\
0.923923923923924	1.76756052906983\\
0.924924924924925	1.79065018705604\\
0.925925925925926	1.81429482867015\\
0.926926926926927	1.83851555681443\\
0.927927927927928	1.86333458296374\\
0.928928928928929	1.88877530186608\\
0.92992992992993	1.91486237241627\\
0.930930930930931	1.94162180530993\\
0.931931931931932	1.96908105815397\\
0.932932932932933	1.99726913878893\\
0.933933933933934	2.02621671766699\\
0.934934934934935	2.05595625023106\\
0.935935935935936	2.08652211035491\\
0.936936936936937	2.11795073603556\\
0.937937937937938	2.15028078867843\\
0.938938938938939	2.18355332748704\\
0.93993993993994	2.21781200066491\\
0.940940940940941	2.2531032553628\\
0.941941941941942	2.28947656856358\\
0.942942942942943	2.32698470139654\\
0.943943943943944	2.36568397971917\\
0.944944944944945	2.40563460420593\\
0.945945945945946	2.44690099365053\\
0.946946946946947	2.48955216573221\\
0.947947947947948	2.53366216013281\\
0.948948948948949	2.57931050963702\\
0.94994994994995	2.62658276572433\\
0.950950950950951	2.67557108619532\\
0.951951951951952	2.72637489359706\\
0.952952952952953	2.77910161466344\\
0.953953953953954	2.8338675127143\\
0.954954954954955	2.89079862702169\\
0.955955955955956	2.95003183562879\\
0.956956956956957	3.01171606108913\\
0.957957957957958	3.07601364219939\\
0.958958958958959	3.14310189917673\\
0.95995995995996	3.21317492506896\\
0.960960960960961	3.28644564272549\\
0.961961961961962	3.36314817470517\\
0.962962962962963	3.44354058345561\\
0.963963963963964	3.52790805148289\\
0.964964964964965	3.61656658671968\\
0.965965965965966	3.70986735778614\\
0.966966966966967	3.80820178850284\\
0.967967967967968	3.91200757244213\\
0.968968968968969	4.0217758086224\\
0.96996996996997	4.13805951155364\\
0.970970970970971	4.26148381670331\\
0.971971971971972	4.39275829156048\\
0.972972972972973	4.53269188052198\\
0.973973973973974	4.6822111696684\\
0.974974974974975	4.84238287068313\\
0.975975975975976	5.01444171419698\\
0.976976976976977	5.1998253447465\\
0.977977977977978	5.40021837148537\\
0.978978978978979	5.61760852513525\\
0.97997997997998	5.85435901671014\\
0.980980980980981	6.11330286625463\\
0.981981981981982	6.39786745598145\\
0.982982982982983	6.71224132811675\\
0.983983983983984	7.0616010727781\\
0.984984984984985	7.45242537232179\\
0.985985985985986	7.89293824753433\\
0.986986986986987	8.39374859804435\\
0.987987987987988	8.96879640508227\\
0.988988988988989	9.63679357811927\\
0.98998998998999	10.4234927331912\\
0.990990990990991	11.36540318983\\
0.991991991991992	12.5161706754042\\
0.992992992992993	13.9581760993786\\
0.993993993993994	15.8251841660247\\
0.994994994994995	18.3508197184176\\
0.995995995995996	21.9858981128334\\
0.996996996996997	27.7367113675713\\
0.997997997997998	38.4491530952287\\
0.998998998998999	67.0916723343279\\
1	inf\\
};
\addlegendentry{$p(\theta \given \text{HHH})$};

\end{axis}
\end{tikzpicture}%
  \caption{The posterior probability that the slope of the regression
    line is positive as a function of the noise standard deviation
    $\sigma$.  Note the limits on the $y$-axis.}
  \label{problem_4}
\end{figure}

\end{document}
