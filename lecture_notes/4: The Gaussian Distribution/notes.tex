\documentclass{article}

\usepackage[T1]{fontenc}
\usepackage[osf]{libertine}
\usepackage[scaled=0.8]{beramono}
\usepackage[margin=1.5in]{geometry}
\usepackage{url}
\usepackage{booktabs}
\usepackage{amsmath}
\usepackage{amssymb}
\usepackage{nicefrac}
\usepackage{microtype}
\usepackage{subcaption}
\usepackage{bm}

\usepackage{sectsty}
\sectionfont{\large}
\subsectionfont{\normalsize}

\usepackage{titlesec}
\titlespacing{\section}{0pt}{10pt plus 2pt minus 2pt}{0pt plus 2pt minus 0pt}
\titlespacing{\subsection}{0pt}{5pt plus 2pt minus 2pt}{0pt plus 2pt minus 0pt}

\usepackage{pgfplots}
\pgfplotsset{
  compat=newest,
  plot coordinates/math parser=false,
  tick label style={font=\footnotesize, /pgf/number format/fixed},
  label style={font=\small},
  legend style={font=\small},
  every axis/.append style={
    tick align=outside,
    clip mode=individual,
    scaled ticks=false,
    thick,
    tick style={semithick, black}
  }
}

\pgfkeys{/pgf/number format/.cd, set thousands separator={\,}}

\usepgfplotslibrary{external}
\tikzexternalize[prefix=tikz/]

\newlength\figurewidth
\newlength\figureheight

\setlength{\figurewidth}{8cm}
\setlength{\figureheight}{6cm}

\newlength\squarefigurewidth
\newlength\squarefigureheight

\setlength{\squarefigurewidth}{4cm}
\setlength{\squarefigureheight}{4cm}

\newlength\smallsquarefigurewidth
\newlength\smallsquarefigureheight

\setlength{\smallsquarefigurewidth}{3.25cm}
\setlength{\smallsquarefigureheight}{3.25cm}

\newlength\smallfigurewidth
\newlength\smallfigureheight

\setlength{\smallfigurewidth}{6.25cm}
\setlength{\smallfigureheight}{4cm}

\setlength{\parindent}{0pt}
\setlength{\parskip}{1ex}

\newcommand{\acro}[1]{\textsc{\MakeLowercase{#1}}}
\newcommand{\given}{\mid}
\newcommand{\mc}[1]{\mathcal{#1}}
\newcommand{\data}{\mc{D}}
\newcommand{\intd}[1]{\,\mathrm{d}{#1}}
\newcommand{\inv}{^{-1}}
\newcommand{\trans}{^\top}
\newcommand{\mat}[1]{\bm{\mathrm{#1}}}
\renewcommand{\vec}[1]{\bm{\mathrm{#1}}}
\newcommand{\R}{\mathbb{R}}

\DeclareMathOperator{\var}{var}
\DeclareMathOperator{\cov}{cov}
\DeclareMathOperator{\diag}{diag}

\begin{document}

\section*{The Gaussian distribution}
Probably the most-important distribution in all of statistics is the
\emph{Gaussian distribution,} also called the \emph{normal
  distribution.}  The Gaussian distribution arises in many contexts
and is widely used for modeling continuous random variables.

The probability density function of the univariate (one-dimensional)
Gaussian distribution is
\begin{equation*}
  p(x \given \mu, \sigma^2)
  =
  \mc{N}(x; \mu, \sigma^2)
  =
  \frac{1}{Z}
  \exp\biggl(
  -\frac{(x - \mu)^2}{2 \sigma^2}
  \biggr).
\end{equation*}
The normalization constant $Z$ is
\begin{equation*}
  Z = \sqrt{2 \pi \sigma^2}.
\end{equation*}
The parameters $\mu$ and $\sigma^2$ specify the mean and variance
of the distribution, respectively:
\begin{equation*}
  \mu = \mathbb{E}[x]; \qquad
  \sigma^2 = \var[x].
\end{equation*}
Figure \ref{1d_examples} plots the probability density function for
several sets of parameters $(\mu, \sigma^2)$.  The distribution is
symmetric around the mean and most of the density ($\approx 99.7\%$)
is contained within $\pm 3 \sigma$ of the mean.

\begin{figure}
  \centering
  % This file was created by matlab2tikz.
% Minimal pgfplots version: 1.3
%
\tikzsetnextfilename{1d_gaussian_pdfs}
\definecolor{mycolor1}{rgb}{0.12157,0.47059,0.70588}%
\definecolor{mycolor2}{rgb}{0.20000,0.62745,0.17255}%
\definecolor{mycolor3}{rgb}{0.07843,0.07843,0.07843}%
%
\begin{tikzpicture}

\begin{axis}[%
width=0.95092\figurewidth,
height=\figureheight,
at={(0\figurewidth,0\figureheight)},
scale only axis,
xmin=-4,
xmax=4,
xlabel={$x$},
ymin=0,
ymax=0.8,
axis x line*=bottom,
axis y line*=left,
legend style={at={(0.03,0.97)},anchor=north west,legend cell align=left,align=left,fill=none,draw=none}
]
\addplot [color=mycolor1,solid]
  table[row sep=crcr]{%
-4	0.000133830225764885\\
-3.98396793587174	0.000142675349741047\\
-3.96793587174349	0.000152065976657072\\
-3.95190380761523	0.000162033024993012\\
-3.93587174348697	0.000172608985016727\\
-3.91983967935872	0.000183827987316515\\
-3.90380761523046	0.000195725873640051\\
-3.8877755511022	0.000208340270077416\\
-3.87174348697395	0.000221710662624091\\
-3.85571142284569	0.000235878475157708\\
-3.83967935871743	0.00025088714986005\\
-3.82364729458918	0.000266782230113281\\
-3.80761523046092	0.000283611445896645\\
-3.79158316633267	0.000301424801706908\\
-3.77555110220441	0.000320274667022632\\
-3.75951903807615	0.00034021586932888\\
-3.7434869739479	0.000361305789715299\\
-3.72745490981964	0.000383604461056511\\
-3.71142284569138	0.000407174668779565\\
-3.69539078156313	0.000432082054218673\\
-3.67935871743487	0.000458395220552653\\
-3.66332665330661	0.000486185841315487\\
-3.64729458917836	0.000515528771464996\\
-3.6312625250501	0.000546502160988996\\
-3.61523046092184	0.000579187571022355\\
-3.59919839679359	0.000613670092442086\\
-3.58316633266533	0.000650038466901082\\
-3.56713426853707	0.00068838521025417\\
-3.55110220440882	0.000728806738323016\\
-3.53507014028056	0.000771403494938878\\
-3.5190380761523	0.000816280082194388\\
-3.50300601202405	0.000863545392827412\\
-3.48697394789579	0.000913312744651563\\
-3.47094188376753	0.000965700016939215\\
-3.45490981963928	0.00102082978865373\\
-3.43887775551102	0.00107882947841831\\
-3.42284569138277	0.00113983148609902\\
-3.40681362725451	0.00120397333586992\\
-3.39078156312625	0.00127139782061737\\
-3.374749498998	0.00134225314753069\\
-3.35871743486974	0.00141669308471503\\
-3.34268537074148	0.00149487710865133\\
-3.32665330661323	0.00157697055231716\\
-3.31062124248497	0.00166314475377022\\
-3.29458917835671	0.00175357720498483\\
-3.27855711422846	0.0018484517007196\\
-3.2625250501002	0.0019479584871821\\
-3.24649298597194	0.00205229441024435\\
-3.23046092184369	0.00216166306294995\\
-3.21442885771543	0.00227627493204146\\
-3.19839679358717	0.00239634754322351\\
-3.18236472945892	0.00252210560486446\\
-3.16633266533066	0.00265378114982641\\
-3.1503006012024	0.00279161367510061\\
-3.13426853707415	0.002935850278912\\
-3.11823647294589	0.00308674579494412\\
-3.10220440881764	0.00324456292332248\\
-3.08617234468938	0.00340957235798186\\
-3.07014028056112	0.00358205291003031\\
-3.05410821643287	0.00376229162671013\\
-3.03807615230461	0.00395058390554404\\
-3.02204408817635	0.0041472336032425\\
-3.0060120240481	0.00435255313893657\\
-2.98997995991984	0.00456686359128916\\
-2.97394789579158	0.00479049478902667\\
-2.95791583166333	0.00502378539442237\\
-2.94188376753507	0.00526708297925248\\
-2.92585170340681	0.00552074409273662\\
-2.90981963927856	0.005785134320965\\
-2.8937875751503	0.00606062833730627\\
-2.87775551102204	0.00634760994328225\\
-2.86172344689379	0.00664647209938829\\
-2.84569138276553	0.00695761694533229\\
-2.82965931863727	0.00728145580915903\\
-2.81362725450902	0.00761840920472244\\
-2.79759519038076	0.00796890681696439\\
-2.7815631262525	0.00833338747445562\\
-2.76553106212425	0.00871229910865282\\
-2.74949899799599	0.0091060986993251\\
-2.73346693386774	0.00951525220560328\\
-2.71743486973948	0.00994023448210711\\
-2.70140280561122	0.0103815291796084\\
-2.68537074148297	0.0108396286296914\\
-2.66933867735471	0.0113150337128785\\
-2.65330661322645	0.0118082537096938\\
-2.6372745490982	0.0123198061341476\\
-2.62124248496994	0.0128502165491321\\
-2.60521042084168	0.0134000183632325\\
-2.58917835671343	0.013969752608466\\
-2.57314629258517	0.0145599676984796\\
-2.55711422845691	0.015171219166751\\
-2.54108216432866	0.0158040693843541\\
-2.5250501002004	0.016459087256871\\
-2.50901803607214	0.0171368479000506\\
-2.49298597194389	0.0178379322938401\\
-2.47695390781563	0.0185629269144352\\
-2.46092184368737	0.0193124233440244\\
-2.44488977955912	0.0200870178579282\\
-2.42885771543086	0.0208873109888638\\
-2.41282565130261	0.021713907068098\\
-2.39679358717435	0.0225674137432813\\
-2.38076152304609	0.0234484414727945\\
-2.36472945891784	0.0243576029964712\\
-2.34869739478958	0.0252955127826016\\
-2.33266533066132	0.0262627864511589\\
-2.31663326653307	0.0272600401732346\\
-2.30060120240481	0.0282878900467097\\
-2.28456913827655	0.0293469514482331\\
-2.2685370741483	0.0304378383616278\\
-2.25250501002004	0.031561162682888\\
-2.23647294589178	0.0327175335019856\\
-2.22044088176353	0.033907556361749\\
-2.20440881763527	0.0351318324941335\\
-2.18837675350701	0.0363909580342532\\
-2.17234468937876	0.0376855232125996\\
-2.1563126252505	0.0390161115259268\\
-2.14028056112224	0.0403832988873405\\
-2.12424849699399	0.0417876527561837\\
-2.10821643286573	0.0432297312483712\\
-2.09218436873747	0.0447100822278847\\
-2.07615230460922	0.0462292423801957\\
-2.06012024048096	0.0477877362684481\\
-2.04408817635271	0.0493860753732887\\
-2.02805611222445	0.0510247571172962\\
-2.01202404809619	0.052704263875018\\
-1.99599198396794	0.0544250619696859\\
-1.97995991983968	0.0561876006577405\\
-1.96392785571142	0.0579923111023547\\
-1.94789579158317	0.0598396053372053\\
-1.93186372745491	0.0617298752217999\\
-1.91583166332665	0.0636634913897248\\
-1.8997995991984	0.0656408021912354\\
-1.88376753507014	0.0676621326316657\\
-1.86773547094188	0.0697277833071892\\
-1.85170340681363	0.0718380293395149\\
-1.83567134268537	0.0739931193111512\\
-1.81963927855711	0.0761932742029252\\
-1.80360721442886	0.078438686335485\\
-1.7875751503006	0.0807295183165622\\
-1.77154308617234	0.0830659019958129\\
-1.75551102204409	0.0854479374290955\\
-1.73947895791583	0.0878756918540804\\
-1.72344689378758	0.0903491986791223\\
-1.70741482965932	0.0928684564873552\\
-1.69138276553106	0.0954334280580022\\
-1.67535070140281	0.098044039406911\\
-1.65931863727455	0.100700178848355\\
-1.64328657314629	0.103401696080149\\
-1.62725450901804	0.106148401294153\\
-1.61122244488978	0.108940064314234\\
-1.59519038076152	0.111776413763773\\
-1.57915831663327	0.114657136264811\\
-1.56312625250501	0.117581875670895\\
-1.54709418837675	0.120550232335723\\
-1.5310621242485	0.12356176241964\\
-1.51503006012024	0.126615977236029\\
-1.49899799599198	0.129712342639645\\
-1.48296593186373	0.132850278458862\\
-1.46693386773547	0.136029157973839\\
-1.45090180360721	0.139248307442514\\
-1.43486973947896	0.142507005676342\\
-1.4188376753507	0.145804483667619\\
-1.40280561122244	0.149139924270191\\
-1.38677354709419	0.152512461935312\\
-1.37074148296593	0.155921182504311\\
-1.35470941883768	0.159365123059726\\
-1.33867735470942	0.162843271836423\\
-1.32264529058116	0.166354568194204\\
-1.30661322645291	0.169897902653299\\
-1.29058116232465	0.173472116994052\\
-1.27454909819639	0.177076004422035\\
-1.25851703406814	0.180708309799727\\
-1.24248496993988	0.184367729945797\\
-1.22645290581162	0.188052914002922\\
-1.21042084168337	0.191762463874985\\
-1.19438877755511	0.195494934734372\\
-1.17835671342685	0.199248835599966\\
-1.1623246492986	0.203022629986358\\
-1.14629258517034	0.206814736624628\\
-1.13026052104208	0.210623530254964\\
-1.11422845691383	0.214447342491225\\
-1.09819639278557	0.218284462757478\\
-1.08216432865731	0.222133139296347\\
-1.06613226452906	0.225991580248921\\
-1.0501002004008	0.229857954805832\\
-1.03406813627255	0.23373039442894\\
-1.01803607214429	0.237606994142988\\
-1.00200400801603	0.241485813896379\\
-0.985971943887776	0.245364879990149\\
-0.969939879759519	0.249242186574038\\
-0.953907815631263	0.253115697208426\\
-0.937875751503006	0.256983346490764\\
-0.92184368737475	0.260843041745\\
-0.905811623246493	0.264692664772344\\
-0.889779559118236	0.268530073661598\\
-0.87374749498998	0.272353104657127\\
-0.857715430861723	0.276159574082418\\
-0.841683366733467	0.279947280317065\\
-0.82565130260521	0.283714005824839\\
-0.809619238476954	0.287457519230438\\
-0.793587174348697	0.291175577442345\\
-0.777555110220441	0.29486592781912\\
-0.761523046092184	0.29852631037635\\
-0.745490981963928	0.302154460031336\\
-0.729458917835671	0.305748108882535\\
-0.713426853707415	0.309304988520637\\
-0.697394789579158	0.31282283236809\\
-0.681362725450902	0.316299378043765\\
-0.665330661322646	0.319732369749404\\
-0.649298597194389	0.323119560674393\\
-0.633266533066132	0.326458715415327\\
-0.617234468937876	0.329747612406789\\
-0.601202404809619	0.332984046359693\\
-0.585170340681363	0.336165830703478\\
-0.569138276553106	0.339290800028423\\
-0.55310621242485	0.342356812524294\\
-0.537074148296593	0.345361752411504\\
-0.521042084168337	0.348303532360969\\
-0.50501002004008	0.351180095898786\\
-0.488977955911824	0.353989419791894\\
-0.472945891783567	0.356729516410858\\
-0.456913827655311	0.359398436065926\\
-0.440881763527054	0.361994269312532\\
-0.424849699398798	0.364515149222442\\
-0.408817635270541	0.366959253616761\\
-0.392785571142285	0.369324807257091\\
-0.376753507014028	0.371610083991139\\
-0.360721442885771	0.373813408849153\\
-0.344689378757515	0.375933160087644\\
-0.328657314629258	0.377967771176883\\
-0.312625250501002	0.379915732728785\\
-0.296593186372745	0.381775594361851\\
-0.280561122244489	0.38354596649995\\
-0.264529058116232	0.385225522101812\\
-0.248496993987976	0.386812998318237\\
-0.232464929859719	0.388307198074096\\
-0.216432865731463	0.389706991572382\\
-0.200400801603207	0.391011317717639\\
-0.18436873747495	0.392219185456281\\
-0.168336673346694	0.393329675031411\\
-0.152304609218437	0.394341939149931\\
-0.13627254509018	0.395255204059869\\
-0.120240480961924	0.396068770535991\\
-0.104208416833667	0.39678201477196\\
-0.0881763527054109	0.397394389177437\\
-0.0721442885771544	0.397905423078697\\
-0.0561122244488979	0.398314723321516\\
-0.0400801603206413	0.398621974775231\\
-0.0240480961923848	0.398826940737091\\
-0.00801603206412826	0.398929463236143\\
0.00801603206412782	0.398929463236143\\
0.0240480961923843	0.398826940737091\\
0.0400801603206409	0.398621974775231\\
0.0561122244488974	0.398314723321516\\
0.0721442885771539	0.397905423078697\\
0.0881763527054105	0.397394389177437\\
0.104208416833667	0.39678201477196\\
0.120240480961924	0.396068770535991\\
0.13627254509018	0.395255204059869\\
0.152304609218437	0.394341939149931\\
0.168336673346693	0.393329675031411\\
0.18436873747495	0.392219185456281\\
0.200400801603206	0.391011317717639\\
0.216432865731463	0.389706991572382\\
0.232464929859719	0.388307198074096\\
0.248496993987976	0.386812998318237\\
0.264529058116232	0.385225522101812\\
0.280561122244489	0.38354596649995\\
0.296593186372745	0.381775594361851\\
0.312625250501002	0.379915732728785\\
0.328657314629258	0.377967771176883\\
0.344689378757515	0.375933160087644\\
0.360721442885771	0.373813408849153\\
0.376753507014028	0.371610083991139\\
0.392785571142285	0.369324807257091\\
0.408817635270541	0.366959253616761\\
0.424849699398798	0.364515149222442\\
0.440881763527054	0.361994269312532\\
0.456913827655311	0.359398436065926\\
0.472945891783567	0.356729516410858\\
0.488977955911824	0.353989419791894\\
0.50501002004008	0.351180095898786\\
0.521042084168337	0.348303532360969\\
0.537074148296593	0.345361752411504\\
0.55310621242485	0.342356812524294\\
0.569138276553106	0.339290800028423\\
0.585170340681363	0.336165830703478\\
0.601202404809619	0.332984046359693\\
0.617234468937876	0.329747612406789\\
0.633266533066132	0.326458715415327\\
0.649298597194389	0.323119560674393\\
0.665330661322646	0.319732369749404\\
0.681362725450902	0.316299378043765\\
0.697394789579159	0.31282283236809\\
0.713426853707415	0.309304988520637\\
0.729458917835672	0.305748108882535\\
0.745490981963928	0.302154460031336\\
0.761523046092185	0.29852631037635\\
0.777555110220441	0.29486592781912\\
0.793587174348698	0.291175577442344\\
0.809619238476954	0.287457519230438\\
0.825651302605211	0.283714005824839\\
0.841683366733467	0.279947280317065\\
0.857715430861724	0.276159574082418\\
0.87374749498998	0.272353104657126\\
0.889779559118236	0.268530073661598\\
0.905811623246493	0.264692664772344\\
0.921843687374749	0.260843041745\\
0.937875751503006	0.256983346490764\\
0.953907815631262	0.253115697208426\\
0.969939879759519	0.249242186574039\\
0.985971943887775	0.245364879990149\\
1.00200400801603	0.241485813896379\\
1.01803607214429	0.237606994142988\\
1.03406813627254	0.23373039442894\\
1.0501002004008	0.229857954805832\\
1.06613226452906	0.225991580248922\\
1.08216432865731	0.222133139296347\\
1.09819639278557	0.218284462757478\\
1.11422845691383	0.214447342491225\\
1.13026052104208	0.210623530254964\\
1.14629258517034	0.206814736624628\\
1.1623246492986	0.203022629986358\\
1.17835671342685	0.199248835599966\\
1.19438877755511	0.195494934734372\\
1.21042084168337	0.191762463874985\\
1.22645290581162	0.188052914002922\\
1.24248496993988	0.184367729945797\\
1.25851703406814	0.180708309799727\\
1.27454909819639	0.177076004422035\\
1.29058116232465	0.173472116994052\\
1.30661322645291	0.169897902653299\\
1.32264529058116	0.166354568194204\\
1.33867735470942	0.162843271836423\\
1.35470941883768	0.159365123059726\\
1.37074148296593	0.155921182504311\\
1.38677354709419	0.152512461935312\\
1.40280561122244	0.149139924270191\\
1.4188376753507	0.145804483667619\\
1.43486973947896	0.142507005676342\\
1.45090180360721	0.139248307442514\\
1.46693386773547	0.136029157973839\\
1.48296593186373	0.132850278458862\\
1.49899799599198	0.129712342639645\\
1.51503006012024	0.126615977236029\\
1.5310621242485	0.12356176241964\\
1.54709418837675	0.120550232335723\\
1.56312625250501	0.117581875670895\\
1.57915831663327	0.114657136264811\\
1.59519038076152	0.111776413763773\\
1.61122244488978	0.108940064314234\\
1.62725450901804	0.106148401294153\\
1.64328657314629	0.103401696080149\\
1.65931863727455	0.100700178848355\\
1.67535070140281	0.098044039406911\\
1.69138276553106	0.0954334280580021\\
1.70741482965932	0.0928684564873551\\
1.72344689378758	0.0903491986791222\\
1.73947895791583	0.0878756918540804\\
1.75551102204409	0.0854479374290954\\
1.77154308617235	0.0830659019958128\\
1.7875751503006	0.0807295183165622\\
1.80360721442886	0.0784386863354851\\
1.81963927855711	0.0761932742029253\\
1.83567134268537	0.0739931193111512\\
1.85170340681363	0.0718380293395149\\
1.86773547094188	0.0697277833071893\\
1.88376753507014	0.0676621326316657\\
1.8997995991984	0.0656408021912355\\
1.91583166332665	0.0636634913897249\\
1.93186372745491	0.0617298752217999\\
1.94789579158317	0.0598396053372053\\
1.96392785571142	0.0579923111023548\\
1.97995991983968	0.0561876006577406\\
1.99599198396794	0.054425061969686\\
2.01202404809619	0.052704263875018\\
2.02805611222445	0.0510247571172962\\
2.04408817635271	0.0493860753732887\\
2.06012024048096	0.0477877362684481\\
2.07615230460922	0.0462292423801957\\
2.09218436873747	0.0447100822278847\\
2.10821643286573	0.0432297312483712\\
2.12424849699399	0.0417876527561837\\
2.14028056112224	0.0403832988873405\\
2.1563126252505	0.0390161115259268\\
2.17234468937876	0.0376855232125996\\
2.18837675350701	0.0363909580342532\\
2.20440881763527	0.0351318324941335\\
2.22044088176353	0.033907556361749\\
2.23647294589178	0.0327175335019856\\
2.25250501002004	0.031561162682888\\
2.2685370741483	0.0304378383616278\\
2.28456913827655	0.0293469514482331\\
2.30060120240481	0.0282878900467097\\
2.31663326653307	0.0272600401732346\\
2.33266533066132	0.0262627864511589\\
2.34869739478958	0.0252955127826016\\
2.36472945891784	0.0243576029964712\\
2.38076152304609	0.0234484414727945\\
2.39679358717435	0.0225674137432813\\
2.41282565130261	0.021713907068098\\
2.42885771543086	0.0208873109888638\\
2.44488977955912	0.0200870178579282\\
2.46092184368737	0.0193124233440244\\
2.47695390781563	0.0185629269144352\\
2.49298597194389	0.0178379322938401\\
2.50901803607214	0.0171368479000506\\
2.5250501002004	0.016459087256871\\
2.54108216432866	0.0158040693843541\\
2.55711422845691	0.015171219166751\\
2.57314629258517	0.0145599676984796\\
2.58917835671343	0.013969752608466\\
2.60521042084168	0.0134000183632325\\
2.62124248496994	0.0128502165491321\\
2.6372745490982	0.0123198061341475\\
2.65330661322645	0.0118082537096938\\
2.66933867735471	0.0113150337128785\\
2.68537074148297	0.0108396286296914\\
2.70140280561122	0.0103815291796084\\
2.71743486973948	0.00994023448210712\\
2.73346693386774	0.00951525220560329\\
2.74949899799599	0.00910609869932512\\
2.76553106212425	0.00871229910865283\\
2.7815631262525	0.00833338747445562\\
2.79759519038076	0.00796890681696439\\
2.81362725450902	0.00761840920472244\\
2.82965931863727	0.00728145580915903\\
2.84569138276553	0.00695761694533229\\
2.86172344689379	0.00664647209938829\\
2.87775551102204	0.00634760994328225\\
2.8937875751503	0.00606062833730627\\
2.90981963927856	0.005785134320965\\
2.92585170340681	0.00552074409273662\\
2.94188376753507	0.00526708297925248\\
2.95791583166333	0.00502378539442237\\
2.97394789579158	0.00479049478902667\\
2.98997995991984	0.00456686359128916\\
3.0060120240481	0.00435255313893657\\
3.02204408817635	0.0041472336032425\\
3.03807615230461	0.00395058390554404\\
3.05410821643287	0.00376229162671013\\
3.07014028056112	0.00358205291003031\\
3.08617234468938	0.00340957235798186\\
3.10220440881764	0.00324456292332248\\
3.11823647294589	0.00308674579494412\\
3.13426853707415	0.002935850278912\\
3.1503006012024	0.00279161367510061\\
3.16633266533066	0.00265378114982641\\
3.18236472945892	0.00252210560486446\\
3.19839679358717	0.00239634754322351\\
3.21442885771543	0.00227627493204146\\
3.23046092184369	0.00216166306294995\\
3.24649298597194	0.00205229441024435\\
3.2625250501002	0.0019479584871821\\
3.27855711422846	0.0018484517007196\\
3.29458917835671	0.00175357720498483\\
3.31062124248497	0.00166314475377022\\
3.32665330661323	0.00157697055231716\\
3.34268537074148	0.00149487710865133\\
3.35871743486974	0.00141669308471503\\
3.374749498998	0.00134225314753069\\
3.39078156312625	0.00127139782061736\\
3.40681362725451	0.00120397333586992\\
3.42284569138277	0.00113983148609902\\
3.43887775551102	0.00107882947841831\\
3.45490981963928	0.00102082978865373\\
3.47094188376754	0.000965700016939214\\
3.48697394789579	0.000913312744651562\\
3.50300601202405	0.00086354539282741\\
3.51903807615231	0.000816280082194387\\
3.53507014028056	0.000771403494938876\\
3.55110220440882	0.000728806738323014\\
3.56713426853707	0.000688385210254171\\
3.58316633266533	0.000650038466901083\\
3.59919839679359	0.000613670092442087\\
3.61523046092184	0.000579187571022356\\
3.6312625250501	0.000546502160988997\\
3.64729458917836	0.000515528771464997\\
3.66332665330661	0.000486185841315487\\
3.67935871743487	0.000458395220552653\\
3.69539078156313	0.000432082054218674\\
3.71142284569138	0.000407174668779566\\
3.72745490981964	0.000383604461056511\\
3.7434869739479	0.0003613057897153\\
3.75951903807615	0.000340215869328881\\
3.77555110220441	0.000320274667022632\\
3.79158316633267	0.000301424801706908\\
3.80761523046092	0.000283611445896645\\
3.82364729458918	0.000266782230113281\\
3.83967935871743	0.00025088714986005\\
3.85571142284569	0.000235878475157708\\
3.87174348697395	0.000221710662624091\\
3.8877755511022	0.000208340270077416\\
3.90380761523046	0.000195725873640051\\
3.91983967935872	0.000183827987316515\\
3.93587174348697	0.000172608985016727\\
3.95190380761523	0.000162033024993012\\
3.96793587174349	0.000152065976657072\\
3.98396793587174	0.000142675349741047\\
4	0.000133830225764885\\
};
\addlegendentry{$\mu = 0$, $\sigma = 1$};

\addplot [color=mycolor2,solid]
  table[row sep=crcr]{%
-4	1.53891972534128e-22\\
-3.98396793587174	2.11955873177388e-22\\
-3.96793587174349	2.91627478052945e-22\\
-3.95190380761523	4.0083434590059e-22\\
-3.93587174348697	5.50370200107088e-22\\
-3.91983967935872	7.54915587246906e-22\\
-3.90380761523046	1.03441634092881e-21\\
-3.8877755511022	1.41594326067698e-21\\
-3.87174348697395	1.93619826745075e-21\\
-3.85571142284569	2.64488809509329e-21\\
-3.83967935871743	3.60926088302852e-21\\
-3.82364729458918	4.92019987675253e-21\\
-3.80761523046092	6.70039977086776e-21\\
-3.79158316633267	9.11532552374123e-21\\
-3.77555110220441	1.23878845268837e-20\\
-3.75951903807615	1.68180489606468e-20\\
-3.7434869739479	2.280906982505e-20\\
-3.72745490981964	3.0902458735757e-20\\
-3.71142284569138	4.18246190408963e-20\\
-3.69539078156313	5.65489376895076e-20\\
-3.67935871743487	7.63783735484062e-20\\
-3.66332665330661	1.03055187875129e-19\\
-3.64729458917836	1.3890657752314e-19\\
-3.6312625250501	1.87037746301209e-19\\
-3.61523046092184	2.51587586503701e-19\\
-3.59919839679359	3.3806690311058e-19\\
-3.58316633266533	4.53805337686232e-19\\
-3.56713426853707	6.08541278456747e-19\\
-3.55110220440882	8.15199648137031e-19\\
-3.53507014028056	1.09091627972888e-18\\
-3.5190380761523	1.45838557567446e-18\\
-3.50300601202405	1.94763152658657e-18\\
-3.48697394789579	2.59833252581142e-18\\
-3.47094188376753	3.46286979286603e-18\\
-3.45490981963928	4.61032023095384e-18\\
-3.43887775551102	6.13168055256708e-18\\
-3.42284569138277	8.14669482197525e-18\\
-3.40681362725451	1.08127677021948e-17\\
-3.39078156312625	1.43365877140254e-17\\
-3.374749498998	1.8989267730426e-17\\
-3.35871743486974	2.51260452416897e-17\\
-3.34268537074148	3.32118872900924e-17\\
-3.32665330661323	4.38547330374651e-17\\
-3.31062124248497	5.784860440196e-17\\
-3.29458917835671	7.62294521774156e-17\\
-3.27855711422846	1.00347419691062e-16\\
-3.2625250501002	1.31960246689458e-16\\
-3.24649298597194	1.73353863893082e-16\\
-3.23046092184369	2.27497920742313e-16\\
-3.21442885771543	2.98246142692342e-16\\
-3.19839679358717	3.90594152761526e-16\\
-3.18236472945892	5.11010869392431e-16\\
-3.16633266533066	6.6786400323985e-16\\
-3.1503006012024	8.71965763037064e-16\\
-3.13426853707415	1.13727195086039e-15\\
-3.11823647294589	1.48177656264648e-15\\
-3.10220440881764	1.92865528936911e-15\\
-3.08617234468938	2.5077255357038e-15\\
-3.07014028056112	3.2573084820798e-15\\
-3.05410821643287	4.22660124187289e-15\\
-3.03807615230461	5.47869546420596e-15\\
-3.02204408817635	7.09441424101068e-15\\
-3.0060120240481	9.17718367759567e-15\\
-2.98997995991984	1.18592111888434e-14\\
-2.97394789579158	1.53093122238598e-14\\
-2.95791583166333	1.97428140822579e-14\\
-2.94188376753507	2.54340739406737e-14\\
-2.92585170340681	3.27322833070446e-14\\
-2.90981963927856	4.20813992026143e-14\\
-2.8937875751503	5.40452512346075e-14\\
-2.87775551102204	6.93391329423799e-14\\
-2.86172344689379	8.88695071227575e-14\\
-2.84569138276553	1.13783852654812e-13\\
-2.82965931863727	1.45533172252414e-13\\
-2.81362725450902	1.85950288078966e-13\\
-2.79759519038076	2.37347801499425e-13\\
-2.7815631262525	3.02640516387794e-13\\
-2.76553106212425	3.8549826125144e-13\\
-2.74949899799599	4.90536441243449e-13\\
-2.73346693386774	6.23553364268159e-13\\
-2.71743486973948	7.91825485351425e-13\\
-2.70140280561122	1.00447428436345e-12\\
-2.68537074148297	1.2729216357061e-12\\
-2.66933867735471	1.61145436739458e-12\\
-2.65330661322645	2.03792338893383e-12\\
-2.6372745490982	2.57460847046196e-12\\
-2.62124248496994	3.24928669366813e-12\\
-2.60521042084168	4.09655102465512e-12\\
-2.58917835671343	5.15943568780087e-12\\
-2.57314629258517	6.49141741589183e-12\\
-2.55711422845691	8.15887665525831e-12\\
-2.54108216432866	1.02441209402867e-11\\
-2.5250501002004	1.28490945450471e-11\\
-2.50901803607214	1.60999249151554e-11\\
-2.49298597194389	2.0152488164735e-11\\
-2.47695390781563	2.51992141406188e-11\\
-2.46092184368737	3.14773974516741e-11\\
-2.44488977955912	3.92793359055598e-11\\
-2.42885771543086	4.89646837249011e-11\\
-2.41282565130261	6.09754857777926e-11\\
-2.39679358717435	7.58544532653917e-11\\
-2.38076152304609	9.42671536977558e-11\\
-2.36472945891784	1.17028921886936e-10\\
-2.34869739478958	1.45137457995714e-10\\
-2.33266533066132	1.79812267967972e-10\\
-2.31663326653307	2.22542326268771e-10\\
-2.30060120240481	2.75143607004166e-10\\
-2.28456913827655	3.3982844442417e-10\\
-2.2685370741483	4.19289055969746e-10\\
-2.25250501002004	5.16798000196917e-10\\
-2.23647294589178	6.36328859446761e-10\\
-2.22044088176353	7.82701046161699e-10\\
-2.20440881763527	9.61753347447938e-10\\
-2.18837675350701	1.18055166234897e-09\\
-2.17234468937876	1.44763737047182e-09\\
-2.1563126252505	1.77332392225139e-09\\
-2.14028056112224	2.17005058681553e-09\\
-2.12424849699399	2.65280386355256e-09\\
-2.10821643286573	3.23961889287048e-09\\
-2.09218436873747	3.95217532997178e-09\\
-2.07615230460922	4.81650461797058e-09\\
-2.06012024048096	5.86382846476303e-09\\
-2.04408817635271	7.1315516500734e-09\\
-2.02805611222445	8.66443613111991e-09\\
-2.01202404809619	1.05159878520763e-08\\
-1.99599198396794	1.27500927782242e-08\\
-1.97995991983968	1.54429445652736e-08\\
-1.96392785571142	1.86853130443859e-08\\
-1.94789579158317	2.25852104734782e-08\\
-1.93186372745491	2.72710214091405e-08\\
-1.91583166332665	3.28951722402225e-08\\
-1.8997995991984	3.96384280600957e-08\\
-1.88376753507014	4.77149178245075e-08\\
-1.86773547094188	5.73780038506198e-08\\
-1.85170340681363	6.89271288867453e-08\\
-1.83567134268537	8.27157934713146e-08\\
-1.81963927855711	9.91608383785895e-08\\
-1.80360721442886	1.18753231919262e-07\\
-1.7875751503006	1.42070590056416e-07\\
-1.77154308617234	1.69791689071564e-07\\
-1.75551102204409	2.02713266263472e-07\\
-1.73947895791583	2.41769444309201e-07\\
-1.72344689378758	2.88054159921658e-07\\
-1.70741482965932	3.42847027795806e-07\\
-1.69138276553106	4.07643127077923e-07\\
-1.67535070140281	4.84187260286863e-07\\
-1.65931863727455	5.74513304368372e-07\\
-1.64328657314629	6.809893510147e-07\\
-1.62725450901804	8.06369419209278e-07\\
-1.61122244488978	9.53852617871475e-07\\
-1.59519038076152	1.12715074122227e-06\\
-1.57915831663327	1.33056539483964e-06\\
-1.56312625250501	1.56907587711805e-06\\
-1.54709418837675	1.84843917980459e-06\\
-1.5310621242485	2.17530362328145e-06\\
-1.51503006012024	2.55733780813272e-06\\
-1.49899799599198	3.00337674509901e-06\\
-1.48296593186373	3.52358722159695e-06\\
-1.46693386773547	4.12965467534403e-06\\
-1.45090180360721	4.83499407502103e-06\\
-1.43486973947896	5.6549875550032e-06\\
-1.4188376753507	6.60725181659572e-06\\
-1.40280561122244	7.71193859242009e-06\\
-1.38677354709419	8.99207177399041e-06\\
-1.37074148296593	1.04739251253205e-05\\
-1.35470941883768	1.21874448476637e-05\\
-1.33867735470942	1.41667216220499e-05\\
-1.32264529058116	1.64505171367515e-05\\
-1.30661322645291	1.90828505055048e-05\\
-1.29058116232465	2.2113650398253e-05\\
-1.27454909819639	2.55994791380216e-05\\
-1.25851703406814	2.96043354635482e-05\\
-1.24248496993988	3.42005431153078e-05\\
-1.22645290581162	3.9469732869947e-05\\
-1.21042084168337	4.55039261216934e-05\\
-1.19438877755511	5.2406728585307e-05\\
-1.17835671342685	6.02946431692348e-05\\
-1.1623246492986	6.92985115348048e-05\\
-1.14629258517034	7.95650943118743e-05\\
-1.13026052104208	9.12588003768694e-05\\
-1.11422845691383	0.000104563576008458\\
-1.09819639278557	0.00011968496511116\\
-1.08216432865731	0.000136852352029266\\
-1.06613226452906	0.000156321398752792\\
-1.0501002004008	0.00017837668853352\\
-1.03406813627255	0.000203334588070964\\
-1.01803607214429	0.000231546340483441\\
-1.00200400801603	0.00026340140123532\\
-0.985971943887776	0.000299331029034295\\
-0.969939879759519	0.000339812143427968\\
-0.953907815631263	0.000385371460402131\\
-0.937875751503006	0.000436589916698427\\
-0.92184368737475	0.000494107392810878\\
-0.905811623246493	0.000558627743672793\\
-0.889779559118236	0.000630924144891793\\
-0.87374749498998	0.000711844761014941\\
-0.857715430861723	0.000802318740692415\\
-0.841683366733467	0.000903362541741525\\
-0.82565130260521	0.00101608658697869\\
-0.809619238476954	0.0011417022492716\\
-0.793587174348697	0.00128152916155519\\
-0.777555110220441	0.00143700284454276\\
-0.761523046092184	0.00160968264153885\\
-0.745490981963928	0.00180125994611751\\
-0.729458917835671	0.00201356670446452\\
-0.713426853707415	0.00224858416989487\\
-0.697394789579158	0.00250845188245034\\
-0.681362725450902	0.00279547684156305\\
-0.665330661322646	0.00311214283455081\\
-0.649298597194389	0.0034611198782046\\
-0.633266533066132	0.00384527372495768\\
-0.617234468937876	0.00426767537911676\\
-0.601202404809619	0.00473161056241806\\
-0.585170340681363	0.00524058906178388\\
-0.569138276553106	0.00579835388564059\\
-0.55310621242485	0.00640889014856644\\
-0.537074148296593	0.00707643359742366\\
-0.521042084168337	0.00780547868555424\\
-0.50501002004008	0.0086007860951533\\
-0.488977955911824	0.00946738960164941\\
-0.472945891783567	0.0104106021679003\\
-0.456913827655311	0.0114360211503395\\
-0.440881763527054	0.0125495324939737\\
-0.424849699398798	0.0137573137884342\\
-0.408817635270541	0.0150658360532126\\
-0.392785571142285	0.0164818641168857\\
-0.376753507014028	0.0180124554526374\\
-0.360721442885771	0.0196649573308382\\
-0.344689378757515	0.0214470021489475\\
-0.328657314629258	0.0233665007996569\\
-0.312625250501002	0.0254316339401048\\
-0.296593186372745	0.0276508410282581\\
-0.280561122244489	0.0300328069972645\\
-0.264529058116232	0.0325864464448139\\
-0.248496993987976	0.0353208852223943\\
-0.232464929859719	0.0382454393188353\\
-0.216432865731463	0.0413695909437682\\
-0.200400801603207	0.0447029617296321\\
-0.18436873747495	0.0482552829856324\\
-0.168336673346694	0.0520363629536386\\
-0.152304609218437	0.0560560510343623\\
-0.13627254509018	0.060324198972258\\
-0.120240480961924	0.0648506190093916\\
-0.104208416833667	0.0696450390419384\\
-0.0881763527054109	0.074717054837911\\
-0.0721442885771544	0.0800760794010559\\
-0.0561122244488979	0.0857312895934465\\
-0.0400801603206413	0.0916915701579668\\
-0.0240480961923848	0.0979654553114391\\
-0.00801603206412826	0.104561068109363\\
0.00801603206412782	0.111486057813883\\
0.0240480961923843	0.118747535527408\\
0.0400801603206409	0.126352008384983\\
0.0561122244488974	0.13430531262879\\
0.0721442885771539	0.142612545917657\\
0.0881763527054105	0.151277999252933\\
0.104208416833667	0.160305088929087\\
0.120240480961924	0.169696288942724\\
0.13627254509018	0.17945306431684\\
0.152304609218437	0.189575805817889\\
0.168336673346693	0.200063766561132\\
0.18436873747495	0.210915001014523\\
0.200400801603206	0.222126306922707\\
0.216432865731463	0.233693170680243\\
0.232464929859719	0.245609716686683\\
0.248496993987976	0.257868661215341\\
0.264529058116232	0.270461271322234\\
0.280561122244489	0.283377329311692\\
0.296593186372745	0.296605103260183\\
0.312625250501002	0.310131324080083\\
0.328657314629258	0.323941169580213\\
0.344689378757515	0.338018255950053\\
0.360721442885771	0.352344637059683\\
0.376753507014028	0.366900811927684\\
0.392785571142285	0.381665740664756\\
0.408817635270541	0.396616869151784\\
0.424849699398798	0.411730162657837\\
0.440881763527054	0.426980148546367\\
0.456913827655311	0.442339968157193\\
0.472945891783567	0.457781437887988\\
0.488977955911824	0.473275119432533\\
0.50501002004008	0.488790399064445\\
0.521042084168337	0.504295575784972\\
0.537074148296593	0.519757958082404\\
0.55310621242485	0.535143968979375\\
0.569138276553106	0.5504192589733\\
0.585170340681363	0.565548826405331\\
0.601202404809619	0.580497144725009\\
0.617234468937876	0.59522829605203\\
0.633266533066132	0.609706110373886\\
0.649298597194389	0.623894309659292\\
0.665330661322646	0.637756656112871\\
0.681362725450902	0.651257103747191\\
0.697394789579159	0.664359952404539\\
0.713426853707415	0.677030003323295\\
0.729458917835672	0.689232715312901\\
0.745490981963928	0.700934360577719\\
0.761523046092185	0.71210217921379\\
0.777555110220441	0.72270453139408\\
0.793587174348698	0.732711046257316\\
0.809619238476954	0.742092766523206\\
0.825651302605211	0.750822287872756\\
0.841683366733467	0.758873892156466\\
0.857715430861724	0.766223673525392\\
0.87374749498998	0.772849656620095\\
0.889779559118236	0.77873190600022\\
0.905811623246493	0.783852626052358\\
0.921843687374749	0.788196250675633\\
0.937875751503006	0.791749522112534\\
0.953907815631262	0.794501558366304\\
0.969939879759519	0.796443908725124\\
0.985971943887775	0.797570596996603\\
1.00200400801603	0.797878152143009\\
1.01803607214429	0.797365626097458\\
1.03406813627254	0.79603459863309\\
1.0501002004008	0.793889169250212\\
1.06613226452906	0.790935936139739\\
1.08216432865731	0.787183962373994\\
1.09819639278557	0.782644729567323\\
1.11422845691383	0.777332079338119\\
1.13026052104208	0.77126214298993\\
1.14629258517034	0.764453259911588\\
1.1623246492986	0.756925885273992\\
1.17835671342685	0.74870248767362\\
1.19438877755511	0.739807437439367\\
1.21042084168337	0.730266886379461\\
1.22645290581162	0.720108639798332\\
1.24248496993988	0.709362021659139\\
1.25851703406814	0.698057733805791\\
1.27454909819639	0.686227710188436\\
1.29058116232465	0.673904967058468\\
1.30661322645291	0.661123450112955\\
1.32264529058116	0.647917879574002\\
1.33867735470942	0.634323594186128\\
1.35470941883768	0.62037639510427\\
1.37074148296593	0.606112390626898\\
1.38677354709419	0.591567842703173\\
1.40280561122244	0.576779016110513\\
1.4188376753507	0.561782031159821\\
1.43486973947896	0.546612720740437\\
1.45090180360721	0.531306492466181\\
1.46693386773547	0.515898196628267\\
1.48296593186373	0.500422000600979\\
1.49899799599198	0.484911270282515\\
1.51503006012024	0.469398459086987\\
1.5310621242485	0.453915004934845\\
1.54709418837675	0.438491235618799\\
1.56312625250501	0.423156282851161\\
1.57915831663327	0.407938005227241\\
1.59519038076152	0.392862920268557\\
1.61122244488978	0.377956145639853\\
1.62725450901804	0.363241349565829\\
1.64328657314629	0.34874071040763\\
1.65931863727455	0.334474885296029\\
1.67535070140281	0.32046298765839\\
1.69138276553106	0.306722573420217\\
1.70741482965932	0.293269635609921\\
1.72344689378758	0.280118607047465\\
1.73947895791583	0.267282370754269\\
1.75551102204409	0.254772277683146\\
1.77154308617235	0.242598171333402\\
1.7875751503006	0.230768418787584\\
1.80360721442886	0.21928994768273\\
1.81963927855711	0.208168288610362\\
1.83567134268537	0.197407622425787\\
1.85170340681363	0.187010831938446\\
1.86773547094188	0.176979557450868\\
1.88376753507014	0.167314255614076\\
1.8997995991984	0.158014261071853\\
1.91583166332665	0.149077850374779\\
1.93186372745491	0.140502307657168\\
1.94789579158317	0.132283991585632\\
1.96392785571142	0.124418403106633\\
1.97995991983968	0.116900253541763\\
1.99599198396794	0.109723532603221\\
2.01202404809619	0.102881575927714\\
2.02805611222445	0.0963671317544384\\
2.04408817635271	0.0901724264015365\\
2.06012024048096	0.0842892282251758\\
2.07615230460922	0.078708909775788\\
2.09218436873747	0.0734225078967435\\
2.10821643286573	0.0684207815415375\\
2.12424849699399	0.0636942671161181\\
2.14028056112224	0.0592333311830682\\
2.1563126252505	0.0550282203937032\\
2.17234468937876	0.051069108542562\\
2.18837675350701	0.0473461406660611\\
2.20440881763527	0.0438494741330822\\
2.22044088176353	0.0405693166998326\\
2.23647294589178	0.037495961524339\\
2.25250501002004	0.0346198191573197\\
2.2685370741483	0.0319314465458465\\
2.28456913827655	0.0294215731041248\\
2.30060120240481	0.0270811239218457\\
2.31663326653307	0.0249012401948979\\
2.33266533066132	0.0228732969757851\\
2.34869739478958	0.0209889183518896\\
2.36472945891784	0.0192399901688149\\
2.38076152304609	0.0176186704234743\\
2.39679358717435	0.0161173974574388\\
2.41282565130261	0.0147288960853972\\
2.42885771543086	0.0134461817965\\
2.44488977955912	0.0122625631679438\\
2.46092184368737	0.0111716426305139\\
2.47695390781563	0.01016731572503\\
2.49298597194389	0.00924376898684806\\
2.50901803607214	0.00839547659285776\\
2.5250501002004	0.00761719590188945\\
2.54108216432866	0.00690396201521116\\
2.55711422845691	0.00625108147895168\\
2.57314629258517	0.00565412524493391\\
2.58917835671343	0.0051089210006341\\
2.60521042084168	0.00461154497288758\\
2.62124248496994	0.00415831330362521\\
2.6372745490982	0.00374577308942436\\
2.65330661322645	0.00337069317006793\\
2.66933867735471	0.00303005474469047\\
2.68537074148297	0.00272104188751237\\
2.70140280561122	0.00244103202867636\\
2.71743486973948	0.00218758645935095\\
2.73346693386774	0.00195844091409799\\
2.74949899799599	0.00175149627754809\\
2.76553106212425	0.00156480945672187\\
2.7815631262525	0.00139658445489841\\
2.79759519038076	0.00124516367778677\\
2.81362725450902	0.00110901949791444\\
2.82965931863727	0.000986746098619404\\
2.84569138276553	0.000877051614825844\\
2.86172344689379	0.000778750583898959\\
2.87775551102204	0.000690756716311436\\
2.8937875751503	0.000612075992608232\\
2.90981963927856	0.000541800090220759\\
2.92585170340681	0.000479100141047358\\
2.94188376753507	0.000423220818373064\\
2.95791583166333	0.000373474749635655\\
2.97394789579158	0.000329237249743032\\
2.98997995991984	0.000289941368094298\\
3.0060120240481	0.000255073241138213\\
3.02204408817635	0.000224167741201799\\
3.03807615230461	0.000196804411422966\\
3.05410821643287	0.000172603675907591\\
3.07014028056112	0.000151223313687814\\
3.08617234468938	0.000132355184668333\\
3.10220440881764	0.000115722195496194\\
3.11823647294589	0.000101075493161917\\
3.13426853707415	8.81918741218101e-05\\
3.1503006012024	7.68713968094399e-05\\
3.16633266533066	6.69351855657903e-05\\
3.18236472945892	5.82234142507088e-05\\
3.19839679358717	5.05934580917472e-05\\
3.21442885771543	4.39182026701802e-05\\
3.23046092184369	3.80844993284059e-05\\
3.24649298597194	3.29917566995032e-05\\
3.2625250501002	2.85506585006592e-05\\
3.27855711422846	2.4681998190472e-05\\
3.29458917835671	2.13156215595545e-05\\
3.31062124248497	1.8389468798872e-05\\
3.32665330661323	1.58487080660222e-05\\
3.34268537074148	1.36449530419994e-05\\
3.35871743486974	1.17355574362959e-05\\
3.374749498998	1.008297985344e-05\\
3.39078156312625	8.65421287672211e-06\\
3.40681362725451	7.42027065288074e-06\\
3.42284569138277	6.35572967325306e-06\\
3.43887775551102	5.43831784111763e-06\\
3.45490981963928	4.64854729074684e-06\\
3.47094188376754	3.96938678045218e-06\\
3.48697394789579	3.38596981930739e-06\\
3.50300601202405	2.88533500517232e-06\\
3.51903807615231	2.45619535020949e-06\\
3.53507014028056	2.08873364955431e-06\\
3.55110220440882	1.77442120957702e-06\\
3.56713426853707	1.50585749478662e-06\\
3.58316633266533	1.27662847749564e-06\\
3.59919839679359	1.08118168258444e-06\\
3.61523046092184	9.14716111825402e-07\\
3.6312625250501	7.73085409041954e-07\\
3.64729458917836	6.52712789699897e-07\\
3.66332665330661	5.50516407188031e-07\\
3.67935871743487	4.63843963872923e-07\\
3.69539078156313	3.90415498833077e-07\\
3.71142284569138	3.28273396801489e-07\\
3.72745490981964	2.75738765062487e-07\\
3.7434869739479	2.31373417623563e-07\\
3.75951903807615	1.93946789651011e-07\\
3.77555110220441	1.62407180622433e-07\\
3.79158316633267	1.35856792578181e-07\\
3.80761523046092	1.13530090880004e-07\\
3.82364729458918	9.47750696044644e-08\\
3.83967935871743	7.9037052669979e-08\\
3.85571142284569	6.58447055416062e-08\\
3.87174348697395	5.47979713630939e-08\\
3.8877755511022	4.55576800816971e-08\\
3.90380761523046	3.78366099745828e-08\\
3.91983967935872	3.13918083399781e-08\\
3.93587174348697	2.60180023341224e-08\\
3.95190380761523	2.15419523393797e-08\\
3.96793587174349	1.7817619134337e-08\\
3.98396793587174	1.47220327718536e-08\\
4	1.21517656996466e-08\\
};
\addlegendentry{$\mu = 1$, $\sigma = \nicefrac{1}{2}$};

\addplot [color=mycolor3,solid]
  table[row sep=crcr]{%
-4	0.026995483256594\\
-3.98396793587174	0.0274308831508051\\
-3.96793587174349	0.0278715144534709\\
-3.95190380761523	0.0283174041183623\\
-3.93587174348697	0.0287685785096099\\
-3.91983967935872	0.0292250633854408\\
-3.90380761523046	0.0296868838818521\\
-3.8877755511022	0.030154064496227\\
-3.87174348697395	0.030626629070897\\
-3.85571142284569	0.0311046007766583\\
-3.83967935871743	0.0315880020962459\\
-3.82364729458918	0.0320768548077721\\
-3.80761523046092	0.032571179968136\\
-3.79158316633267	0.033070997896408\\
-3.77555110220441	0.0335763281571974\\
-3.75951903807615	0.0340871895440086\\
-3.7434869739479	0.0346036000625911\\
-3.72745490981964	0.0351255769142919\\
-3.71142284569138	0.0356531364794143\\
-3.69539078156313	0.0361862943005922\\
-3.67935871743487	0.0367250650661838\\
-3.66332665330661	0.0372694625936946\\
-3.64729458917836	0.0378194998132333\\
-3.6312625250501	0.0383751887510102\\
-3.61523046092184	0.038936540512884\\
-3.59919839679359	0.0395035652679632\\
-3.58316633266533	0.0400762722322718\\
-3.56713426853707	0.0406546696524835\\
-3.55110220440882	0.0412387647897342\\
-3.53507014028056	0.0418285639035189\\
-3.5190380761523	0.042424072235681\\
-3.50300601202405	0.0430252939945011\\
-3.48697394789579	0.0436322323388933\\
-3.47094188376753	0.0442448893627169\\
-3.45490981963928	0.0448632660792101\\
-3.43887775551102	0.0454873624055549\\
-3.42284569138277	0.0461171771475804\\
-3.40681362725451	0.0467527079846114\\
-3.39078156312625	0.0473939514544724\\
-3.374749498998	0.0480409029386528\\
-3.35871743486974	0.0486935566476427\\
-3.34268537074148	0.0493519056064466\\
-3.32665330661323	0.0500159416402831\\
-3.31062124248497	0.0506856553604792\\
-3.29458917835671	0.0513610361505671\\
-3.27855711422846	0.0520420721525905\\
-3.2625250501002	0.0527287502536309\\
-3.24649298597194	0.0534210560725596\\
-3.23046092184369	0.0541189739470249\\
-3.21442885771543	0.0548224869206824\\
-3.19839679358717	0.0555315767306768\\
-3.18236472945892	0.0562462237953822\\
-3.16633266533066	0.0569664072024107\\
-3.1503006012024	0.0576921046968958\\
-3.13426853707415	0.0584232926700607\\
-3.11823647294589	0.0591599461480766\\
-3.10220440881764	0.0599020387812222\\
-3.08617234468938	0.0606495428333505\\
-3.07014028056112	0.0614024291716709\\
-3.05410821643287	0.0621606672568556\\
-3.03807615230461	0.0629242251334775\\
-3.02204408817635	0.0636930694207867\\
-3.0060120240481	0.0644671653038353\\
-2.98997995991984	0.065246476524956\\
-2.97394789579158	0.0660309653756038\\
-2.95791583166333	0.0668205926885674\\
-2.94188376753507	0.0676153178305586\\
-2.92585170340681	0.068415098695186\\
-2.90981963927856	0.0692198916963215\\
-2.8937875751503	0.0700296517618662\\
-2.87775551102204	0.070844332327923\\
-2.86172344689379	0.0716638853333824\\
-2.84569138276553	0.0724882612149299\\
-2.82965931863727	0.0733174089024803\\
-2.81362725450902	0.0741512758150457\\
-2.79759519038076	0.0749898078570456\\
-2.7815631262525	0.0758329494150628\\
-2.76553106212425	0.0766806433550544\\
-2.74949899799599	0.0775328310200209\\
-2.73346693386774	0.0783894522281428\\
-2.71743486973948	0.0792504452713881\\
-2.70140280561122	0.0801157469145975\\
-2.68537074148297	0.0809852923950532\\
-2.66933867735471	0.0818590154225362\\
-2.65330661322645	0.0827368481798771\\
-2.6372745490982	0.0836187213240074\\
-2.62124248496994	0.0845045639875132\\
-2.60521042084168	0.0853943037806995\\
-2.58917835671343	0.0862878667941671\\
-2.57314629258517	0.0871851776019075\\
-2.55711422845691	0.0880861592649208\\
-2.54108216432866	0.088990733335359\\
-2.5250501002004	0.0898988198612005\\
-2.50901803607214	0.0908103373914574\\
-2.49298597194389	0.0917252029819211\\
-2.47695390781563	0.0926433322014479\\
-2.46092184368737	0.0935646391387883\\
-2.44488977955912	0.0944890364099633\\
-2.42885771543086	0.0954164351661889\\
-2.41282565130261	0.0963467451023527\\
-2.39679358717435	0.0972798744660439\\
-2.38076152304609	0.0982157300671394\\
-2.36472945891784	0.0991542172879459\\
-2.34869739478958	0.100095240093902\\
-2.33266533066132	0.101038701044841\\
-2.31663326653307	0.10198450130681\\
-2.30060120240481	0.102932540664459\\
-2.28456913827655	0.103882717533983\\
-2.2685370741483	0.104834928976634\\
-2.25250501002004	0.10578907071279\\
-2.23647294589178	0.106745037136592\\
-2.22044088176353	0.107702721331134\\
-2.20440881763527	0.108662015084225\\
-2.18837675350701	0.1096228089047\\
-2.17234468937876	0.110584992039298\\
-2.1563126252505	0.111548452490093\\
-2.14028056112224	0.112513077032479\\
-2.12424849699399	0.113478751233711\\
-2.10821643286573	0.114445359471997\\
-2.09218436873747	0.115412784956133\\
-2.07615230460922	0.116380909745693\\
-2.06012024048096	0.117349614771747\\
-2.04408817635271	0.118318779858133\\
-2.02805611222445	0.119288283743253\\
-2.01202404809619	0.120258004102402\\
-1.99599198396794	0.121227817570629\\
-1.97995991983968	0.122197599766111\\
-1.96392785571142	0.123167225314055\\
-1.94789579158317	0.124136567871102\\
-1.93186372745491	0.125105500150245\\
-1.91583166332665	0.126073893946243\\
-1.8997995991984	0.127041620161537\\
-1.88376753507014	0.12800854883265\\
-1.86773547094188	0.128974549157067\\
-1.85170340681363	0.129939489520601\\
-1.83567134268537	0.130903237525219\\
-1.81963927855711	0.131865660017332\\
-1.80360721442886	0.132826623116545\\
-1.7875751503006	0.133785992244844\\
-1.77154308617234	0.134743632156228\\
-1.75551102204409	0.135699406966772\\
-1.73947895791583	0.136653180185109\\
-1.72344689378758	0.137604814743325\\
-1.70741482965932	0.138554173028258\\
-1.69138276553106	0.139501116913197\\
-1.67535070140281	0.140445507789955\\
-1.65931863727455	0.141387206601333\\
-1.64328657314629	0.142326073873932\\
-1.62725450901804	0.143261969751335\\
-1.61122244488978	0.144194754027628\\
-1.59519038076152	0.145124286181252\\
-1.57915831663327	0.146050425409184\\
-1.56312625250501	0.146973030661424\\
-1.54709418837675	0.147891960675793\\
-1.5310621242485	0.148807074013008\\
-1.51503006012024	0.14971822909204\\
-1.49899799599198	0.150625284225742\\
-1.48296593186373	0.151528097656724\\
-1.46693386773547	0.152426527593471\\
-1.45090180360721	0.15332043224669\\
-1.43486973947896	0.154209669865874\\
-1.4188376753507	0.155094098776067\\
-1.40280561122244	0.155973577414823\\
-1.38677354709419	0.156847964369334\\
-1.37074148296593	0.157717118413732\\
-1.35470941883768	0.158580898546532\\
-1.33867735470942	0.159439164028218\\
-1.32264529058116	0.160291774418945\\
-1.30661322645291	0.161138589616359\\
-1.29058116232465	0.161979469893501\\
-1.27454909819639	0.162814275936798\\
-1.25851703406814	0.163642868884122\\
-1.24248496993988	0.164465110362896\\
-1.22645290581162	0.165280862528239\\
-1.21042084168337	0.166089988101135\\
-1.19438877755511	0.16689235040661\\
-1.17835671342685	0.167687813411904\\
-1.1623246492986	0.168476241764617\\
-1.14629258517034	0.169257500830824\\
-1.13026052104208	0.170031456733136\\
-1.11422845691383	0.1707979763887\\
-1.09819639278557	0.171556927547109\\
-1.08216432865731	0.172308178828225\\
-1.06613226452906	0.173051599759888\\
-1.0501002004008	0.173787060815496\\
-1.03406813627255	0.174514433451448\\
-1.01803607214429	0.17523359014443\\
-1.00200400801603	0.175944404428528\\
-0.985971943887776	0.176646750932156\\
-0.969939879759519	0.177340505414785\\
-0.953907815631263	0.178025544803447\\
-0.937875751503006	0.178701747229023\\
-0.92184368737475	0.179368992062267\\
-0.905811623246493	0.180027159949583\\
-0.889779559118236	0.18067613284852\\
-0.87374749498998	0.181315794062976\\
-0.857715430861723	0.181946028278099\\
-0.841683366733467	0.182566721594865\\
-0.82565130260521	0.183177761564329\\
-0.809619238476954	0.183779037221515\\
-0.793587174348697	0.184370439118954\\
-0.777555110220441	0.184951859359842\\
-0.761523046092184	0.185523191630802\\
-0.745490981963928	0.186084331234246\\
-0.729458917835671	0.18663517512032\\
-0.713426853707415	0.187175621918405\\
-0.697394789579158	0.187705571968189\\
-0.681362725450902	0.188224927350265\\
-0.665330661322646	0.188733591916263\\
-0.649298597194389	0.189231471318498\\
-0.633266533066132	0.189718473039116\\
-0.617234468937876	0.190194506418732\\
-0.601202404809619	0.19065948268454\\
-0.585170340681363	0.191113314977897\\
-0.569138276553106	0.191555918381348\\
-0.55310621242485	0.191987209945101\\
-0.537074148296593	0.192407108712929\\
-0.521042084168337	0.192815535747483\\
-0.50501002004008	0.193212414155024\\
-0.488977955911824	0.193597669109542\\
-0.472945891783567	0.193971227876262\\
-0.456913827655311	0.19433301983453\\
-0.440881763527054	0.194682976500055\\
-0.424849699398798	0.195021031546516\\
-0.408817635270541	0.19534712082651\\
-0.392785571142285	0.195661182391831\\
-0.376753507014028	0.195963156513089\\
-0.360721442885771	0.19625298569864\\
-0.344689378757515	0.19653061471282\\
-0.328657314629258	0.196795990593498\\
-0.312625250501002	0.197049062668908\\
-0.296593186372745	0.197289782573775\\
-0.280561122244489	0.197518104264719\\
-0.264529058116232	0.197733984034935\\
-0.248496993987976	0.197937380528134\\
-0.232464929859719	0.198128254751748\\
-0.216432865731463	0.198306570089388\\
-0.200400801603207	0.198472292312553\\
-0.18436873747495	0.198625389591576\\
-0.168336673346694	0.198765832505816\\
-0.152304609218437	0.198893594053079\\
-0.13627254509018	0.199008649658273\\
-0.120240480961924	0.199110977181281\\
-0.104208416833667	0.199200556924069\\
-0.0881763527054109	0.199277371636997\\
-0.0721442885771544	0.199341406524365\\
-0.0561122244488979	0.199392649249151\\
-0.0400801603206413	0.199431089936983\\
-0.0240480961923848	0.199456721179303\\
-0.00801603206412826	0.199469538035752\\
0.00801603206412782	0.199469538035752\\
0.0240480961923843	0.199456721179303\\
0.0400801603206409	0.199431089936983\\
0.0561122244488974	0.199392649249151\\
0.0721442885771539	0.199341406524365\\
0.0881763527054105	0.199277371636997\\
0.104208416833667	0.199200556924069\\
0.120240480961924	0.199110977181281\\
0.13627254509018	0.199008649658273\\
0.152304609218437	0.198893594053079\\
0.168336673346693	0.198765832505816\\
0.18436873747495	0.198625389591576\\
0.200400801603206	0.198472292312553\\
0.216432865731463	0.198306570089388\\
0.232464929859719	0.198128254751748\\
0.248496993987976	0.197937380528134\\
0.264529058116232	0.197733984034935\\
0.280561122244489	0.197518104264719\\
0.296593186372745	0.197289782573775\\
0.312625250501002	0.197049062668908\\
0.328657314629258	0.196795990593498\\
0.344689378757515	0.19653061471282\\
0.360721442885771	0.19625298569864\\
0.376753507014028	0.195963156513089\\
0.392785571142285	0.195661182391831\\
0.408817635270541	0.19534712082651\\
0.424849699398798	0.195021031546516\\
0.440881763527054	0.194682976500055\\
0.456913827655311	0.19433301983453\\
0.472945891783567	0.193971227876262\\
0.488977955911824	0.193597669109542\\
0.50501002004008	0.193212414155024\\
0.521042084168337	0.192815535747483\\
0.537074148296593	0.192407108712929\\
0.55310621242485	0.191987209945101\\
0.569138276553106	0.191555918381348\\
0.585170340681363	0.191113314977897\\
0.601202404809619	0.19065948268454\\
0.617234468937876	0.190194506418732\\
0.633266533066132	0.189718473039116\\
0.649298597194389	0.189231471318498\\
0.665330661322646	0.188733591916263\\
0.681362725450902	0.188224927350265\\
0.697394789579159	0.187705571968189\\
0.713426853707415	0.187175621918405\\
0.729458917835672	0.18663517512032\\
0.745490981963928	0.186084331234246\\
0.761523046092185	0.185523191630801\\
0.777555110220441	0.184951859359842\\
0.793587174348698	0.184370439118954\\
0.809619238476954	0.183779037221515\\
0.825651302605211	0.183177761564329\\
0.841683366733467	0.182566721594865\\
0.857715430861724	0.181946028278099\\
0.87374749498998	0.181315794062976\\
0.889779559118236	0.18067613284852\\
0.905811623246493	0.180027159949583\\
0.921843687374749	0.179368992062267\\
0.937875751503006	0.178701747229023\\
0.953907815631262	0.178025544803447\\
0.969939879759519	0.177340505414785\\
0.985971943887775	0.176646750932156\\
1.00200400801603	0.175944404428528\\
1.01803607214429	0.17523359014443\\
1.03406813627254	0.174514433451448\\
1.0501002004008	0.173787060815496\\
1.06613226452906	0.173051599759888\\
1.08216432865731	0.172308178828225\\
1.09819639278557	0.171556927547109\\
1.11422845691383	0.1707979763887\\
1.13026052104208	0.170031456733136\\
1.14629258517034	0.169257500830824\\
1.1623246492986	0.168476241764617\\
1.17835671342685	0.167687813411904\\
1.19438877755511	0.16689235040661\\
1.21042084168337	0.166089988101135\\
1.22645290581162	0.165280862528239\\
1.24248496993988	0.164465110362896\\
1.25851703406814	0.163642868884122\\
1.27454909819639	0.162814275936798\\
1.29058116232465	0.161979469893501\\
1.30661322645291	0.161138589616359\\
1.32264529058116	0.160291774418945\\
1.33867735470942	0.159439164028218\\
1.35470941883768	0.158580898546532\\
1.37074148296593	0.157717118413732\\
1.38677354709419	0.156847964369334\\
1.40280561122244	0.155973577414823\\
1.4188376753507	0.155094098776067\\
1.43486973947896	0.154209669865874\\
1.45090180360721	0.15332043224669\\
1.46693386773547	0.152426527593471\\
1.48296593186373	0.151528097656724\\
1.49899799599198	0.150625284225742\\
1.51503006012024	0.14971822909204\\
1.5310621242485	0.148807074013008\\
1.54709418837675	0.147891960675793\\
1.56312625250501	0.146973030661424\\
1.57915831663327	0.146050425409184\\
1.59519038076152	0.145124286181252\\
1.61122244488978	0.144194754027628\\
1.62725450901804	0.143261969751335\\
1.64328657314629	0.142326073873931\\
1.65931863727455	0.141387206601333\\
1.67535070140281	0.140445507789955\\
1.69138276553106	0.139501116913197\\
1.70741482965932	0.138554173028258\\
1.72344689378758	0.137604814743325\\
1.73947895791583	0.136653180185109\\
1.75551102204409	0.135699406966772\\
1.77154308617235	0.134743632156228\\
1.7875751503006	0.133785992244844\\
1.80360721442886	0.132826623116545\\
1.81963927855711	0.131865660017332\\
1.83567134268537	0.130903237525219\\
1.85170340681363	0.129939489520601\\
1.86773547094188	0.128974549157067\\
1.88376753507014	0.12800854883265\\
1.8997995991984	0.127041620161537\\
1.91583166332665	0.126073893946243\\
1.93186372745491	0.125105500150245\\
1.94789579158317	0.124136567871102\\
1.96392785571142	0.123167225314055\\
1.97995991983968	0.122197599766111\\
1.99599198396794	0.121227817570629\\
2.01202404809619	0.120258004102402\\
2.02805611222445	0.119288283743253\\
2.04408817635271	0.118318779858133\\
2.06012024048096	0.117349614771747\\
2.07615230460922	0.116380909745693\\
2.09218436873747	0.115412784956133\\
2.10821643286573	0.114445359471997\\
2.12424849699399	0.113478751233711\\
2.14028056112224	0.112513077032479\\
2.1563126252505	0.111548452490093\\
2.17234468937876	0.110584992039298\\
2.18837675350701	0.1096228089047\\
2.20440881763527	0.108662015084225\\
2.22044088176353	0.107702721331134\\
2.23647294589178	0.106745037136592\\
2.25250501002004	0.10578907071279\\
2.2685370741483	0.104834928976634\\
2.28456913827655	0.103882717533983\\
2.30060120240481	0.102932540664459\\
2.31663326653307	0.10198450130681\\
2.33266533066132	0.101038701044841\\
2.34869739478958	0.100095240093902\\
2.36472945891784	0.0991542172879459\\
2.38076152304609	0.0982157300671394\\
2.39679358717435	0.0972798744660439\\
2.41282565130261	0.0963467451023527\\
2.42885771543086	0.0954164351661889\\
2.44488977955912	0.0944890364099633\\
2.46092184368737	0.0935646391387883\\
2.47695390781563	0.0926433322014479\\
2.49298597194389	0.0917252029819211\\
2.50901803607214	0.0908103373914574\\
2.5250501002004	0.0898988198612005\\
2.54108216432866	0.088990733335359\\
2.55711422845691	0.0880861592649207\\
2.57314629258517	0.0871851776019075\\
2.58917835671343	0.0862878667941671\\
2.60521042084168	0.0853943037806995\\
2.62124248496994	0.0845045639875131\\
2.6372745490982	0.0836187213240073\\
2.65330661322645	0.0827368481798771\\
2.66933867735471	0.0818590154225362\\
2.68537074148297	0.0809852923950533\\
2.70140280561122	0.0801157469145975\\
2.71743486973948	0.0792504452713881\\
2.73346693386774	0.0783894522281428\\
2.74949899799599	0.0775328310200209\\
2.76553106212425	0.0766806433550544\\
2.7815631262525	0.0758329494150628\\
2.79759519038076	0.0749898078570456\\
2.81362725450902	0.0741512758150457\\
2.82965931863727	0.0733174089024803\\
2.84569138276553	0.0724882612149299\\
2.86172344689379	0.0716638853333824\\
2.87775551102204	0.070844332327923\\
2.8937875751503	0.0700296517618662\\
2.90981963927856	0.0692198916963215\\
2.92585170340681	0.068415098695186\\
2.94188376753507	0.0676153178305586\\
2.95791583166333	0.0668205926885674\\
2.97394789579158	0.0660309653756038\\
2.98997995991984	0.065246476524956\\
3.0060120240481	0.0644671653038353\\
3.02204408817635	0.0636930694207867\\
3.03807615230461	0.0629242251334775\\
3.05410821643287	0.0621606672568556\\
3.07014028056112	0.0614024291716709\\
3.08617234468938	0.0606495428333505\\
3.10220440881764	0.0599020387812222\\
3.11823647294589	0.0591599461480766\\
3.13426853707415	0.0584232926700607\\
3.1503006012024	0.0576921046968958\\
3.16633266533066	0.0569664072024107\\
3.18236472945892	0.0562462237953822\\
3.19839679358717	0.0555315767306768\\
3.21442885771543	0.0548224869206824\\
3.23046092184369	0.0541189739470249\\
3.24649298597194	0.0534210560725596\\
3.2625250501002	0.0527287502536309\\
3.27855711422846	0.0520420721525905\\
3.29458917835671	0.0513610361505671\\
3.31062124248497	0.0506856553604792\\
3.32665330661323	0.0500159416402831\\
3.34268537074148	0.0493519056064466\\
3.35871743486974	0.0486935566476427\\
3.374749498998	0.0480409029386528\\
3.39078156312625	0.0473939514544724\\
3.40681362725451	0.0467527079846114\\
3.42284569138277	0.0461171771475804\\
3.43887775551102	0.0454873624055549\\
3.45490981963928	0.0448632660792101\\
3.47094188376754	0.0442448893627169\\
3.48697394789579	0.0436322323388933\\
3.50300601202405	0.0430252939945011\\
3.51903807615231	0.042424072235681\\
3.53507014028056	0.0418285639035189\\
3.55110220440882	0.0412387647897342\\
3.56713426853707	0.0406546696524835\\
3.58316633266533	0.0400762722322718\\
3.59919839679359	0.0395035652679632\\
3.61523046092184	0.038936540512884\\
3.6312625250501	0.0383751887510102\\
3.64729458917836	0.0378194998132333\\
3.66332665330661	0.0372694625936946\\
3.67935871743487	0.0367250650661839\\
3.69539078156313	0.0361862943005922\\
3.71142284569138	0.0356531364794143\\
3.72745490981964	0.0351255769142919\\
3.7434869739479	0.0346036000625911\\
3.75951903807615	0.0340871895440086\\
3.77555110220441	0.0335763281571974\\
3.79158316633267	0.033070997896408\\
3.80761523046092	0.032571179968136\\
3.82364729458918	0.0320768548077721\\
3.83967935871743	0.0315880020962459\\
3.85571142284569	0.0311046007766583\\
3.87174348697395	0.030626629070897\\
3.8877755511022	0.030154064496227\\
3.90380761523046	0.0296868838818521\\
3.91983967935872	0.0292250633854408\\
3.93587174348697	0.0287685785096099\\
3.95190380761523	0.0283174041183623\\
3.96793587174349	0.0278715144534709\\
3.98396793587174	0.0274308831508051\\
4	0.026995483256594\\
};
\addlegendentry{$\mu = 0$, $\sigma = 2$};

\end{axis}
\end{tikzpicture}%
  \caption{Examples of univariate Gaussian \acro{PDF}s $\mc{N}(x; \mu,
    \sigma^2)$.}
  \label{1d_examples}
\end{figure}

We may extend the univariate Gaussian distribution to a distribution
over $d$-dimensional vectors, producing a multivariate analog.  The
probablity density function of the multivariate Gaussian distribution
is
\begin{equation*}
  p(\vec{x} \given \vec{\mu}, \mat{\Sigma})
  =
  \mc{N}(\vec{x}; \vec{\mu}, \mat{\Sigma})
  =
  \frac{1}{Z}
  \exp\biggl(
  -\frac{1}{2}
  (\vec{x} - \vec{\mu})\trans
  \mat{\Sigma}\inv
  (\vec{x} - \vec{\mu})
  \biggr).
\end{equation*}
The normalization constant $Z$ is
\begin{equation*}
  Z = \sqrt{\det (2 \pi \mat{\Sigma})}
    = (2 \pi)^{\nicefrac{d}{2}} (\det \mat{\Sigma})^{\nicefrac{1}{2}}.
\end{equation*}
Examining these equations, we can see that the multivariate density
coincides with the univariate density in the special case
when $\mat{\Sigma}$ is the scalar $\sigma^2$.

Again, the vector $\vec{\mu}$ specifies the mean of the multivariate
Gaussian distribution.  The matrix $\mat{\Sigma}$ specifies the
\emph{covariance} between each pair of variables in $\vec{x}$:
\begin{equation*}
  \mat{\Sigma}
  =
  \cov(\vec{x}, \vec{x})
  =
  \mathbb{E}\bigl[(\vec{x} - \vec{\mu})(\vec{x} - \vec{\mu})\trans\bigr].
\end{equation*}
Covariance matrices are necessarily symmetric and \emph{positive
  semidefinite,} which means their eigenvalues are nonnegative.  Note
that the density function above requires that $\mat{\Sigma}$ be
\emph{positive definite,} or have strictly positive eigenvalues.  A
zero eigenvalue would result in a determinant of zero, making the
normalization impossible.

The dependence of the multivariate Gaussian density on $\vec{x}$ is
entirely through the value of the quadratic form
\begin{equation*}
  \Delta^2
  =
  (\vec{x} - \vec{\mu})\trans
  \mat{\Sigma}\inv
  (\vec{x} - \vec{\mu}).
\end{equation*}
The value $\Delta$ (obtained via a square root) is called the
\emph{Mahalanobis distance,} and can be seen as a generalization of
the $Z$ score $\frac{(x - \mu)}{\sigma}$, often encountered in
statistics.

To understand the behavior of the density geometrically, we can set
the Mahalanobis distance to a constant. The set of points in $\R^d$
satisfying $\Delta = c$ for any given value $c > 0$ is an ellipsoid
with the eigenvectors of $\mat{\Sigma}$ defining the directions of the
principal axes.

\begin{figure}
  \begin{subfigure}{0.33\textwidth}
    % This file was created by matlab2tikz.
% Minimal pgfplots version: 1.3
%
\tikzsetnextfilename{2d_gaussian_pdf_1}
\definecolor{mycolor1}{rgb}{0.01430,0.01430,0.01430}%
\definecolor{mycolor2}{rgb}{0.17602,0.09542,0.29903}%
\definecolor{mycolor3}{rgb}{0.09526,0.24777,0.50222}%
\definecolor{mycolor4}{rgb}{0.00028,0.42762,0.39450}%
\definecolor{mycolor5}{rgb}{0.00000,0.60635,0.20114}%
\definecolor{mycolor6}{rgb}{0.33321,0.74256,0.00000}%
\definecolor{mycolor7}{rgb}{0.83867,0.80069,0.40679}%
\definecolor{mycolor8}{rgb}{0.96920,0.92730,0.89610}%
%
\begin{tikzpicture}

\begin{axis}[%
width=0.95092\smallsquarefigurewidth,
height=\smallsquarefigureheight,
at={(0\smallsquarefigurewidth,0\smallsquarefigureheight)},
scale only axis,
xmin=-4,
xmax=4,
xlabel={$x_1$},
ymin=-4,
ymax=4,
ylabel={$x_2$},
axis x line*=bottom,
axis y line*=left
]

\addplot[area legend,solid,fill=mycolor1,draw=black,forget plot]
table[row sep=crcr] {%
x	y\\
-4	4\\
-3.98396793587174	4.00000000000001\\
-3.96793587174349	4.00000000000002\\
-3.95190380761523	4.00000000000004\\
-3.93587174348697	4.00000000000005\\
-3.91983967935872	4.00000000000007\\
-3.90380761523046	4.00000000000008\\
-3.8877755511022	4.0000000000001\\
-3.87174348697395	4.00000000000012\\
-3.85571142284569	4.00000000000014\\
-3.83967935871743	4.00000000000016\\
-3.82364729458918	4.00000000000018\\
-3.80761523046092	4.0000000000002\\
-3.79158316633267	4.00000000000023\\
-3.77555110220441	4.00000000000025\\
-3.75951903807615	4.00000000000028\\
-3.7434869739479	4.00000000000031\\
-3.72745490981964	4.00000000000034\\
-3.71142284569138	4.00000000000037\\
-3.69539078156313	4.0000000000004\\
-3.67935871743487	4.00000000000044\\
-3.66332665330661	4.00000000000048\\
-3.64729458917836	4.00000000000052\\
-3.6312625250501	4.00000000000056\\
-3.61523046092184	4.0000000000006\\
-3.59919839679359	4.00000000000065\\
-3.58316633266533	4.0000000000007\\
-3.56713426853707	4.00000000000075\\
-3.55110220440882	4.0000000000008\\
-3.53507014028056	4.00000000000086\\
-3.5190380761523	4.00000000000092\\
-3.50300601202405	4.00000000000098\\
-3.48697394789579	4.00000000000105\\
-3.47094188376753	4.00000000000112\\
-3.45490981963928	4.0000000000012\\
-3.43887775551102	4.00000000000127\\
-3.42284569138277	4.00000000000136\\
-3.40681362725451	4.00000000000144\\
-3.39078156312625	4.00000000000153\\
-3.374749498998	4.00000000000163\\
-3.35871743486974	4.00000000000173\\
-3.34268537074148	4.00000000000183\\
-3.32665330661323	4.00000000000195\\
-3.31062124248497	4.00000000000206\\
-3.29458917835671	4.00000000000218\\
-3.27855711422846	4.00000000000231\\
-3.2625250501002	4.00000000000245\\
-3.24649298597194	4.00000000000259\\
-3.23046092184369	4.00000000000273\\
-3.21442885771543	4.00000000000289\\
-3.19839679358717	4.00000000000305\\
-3.18236472945892	4.00000000000322\\
-3.16633266533066	4.0000000000034\\
-3.1503006012024	4.00000000000358\\
-3.13426853707415	4.00000000000378\\
-3.11823647294589	4.00000000000398\\
-3.10220440881764	4.00000000000419\\
-3.08617234468938	4.00000000000442\\
-3.07014028056112	4.00000000000465\\
-3.05410821643287	4.00000000000489\\
-3.03807615230461	4.00000000000515\\
-3.02204408817635	4.00000000000541\\
-3.0060120240481	4.00000000000569\\
-2.98997995991984	4.00000000000598\\
-2.97394789579158	4.00000000000628\\
-2.95791583166333	4.00000000000659\\
-2.94188376753507	4.00000000000692\\
-2.92585170340681	4.00000000000726\\
-2.90981963927856	4.00000000000762\\
-2.8937875751503	4.00000000000799\\
-2.87775551102204	4.00000000000838\\
-2.86172344689379	4.00000000000878\\
-2.84569138276553	4.0000000000092\\
-2.82965931863727	4.00000000000964\\
-2.81362725450902	4.00000000001009\\
-2.79759519038076	4.00000000001056\\
-2.7815631262525	4.00000000001105\\
-2.76553106212425	4.00000000001157\\
-2.74949899799599	4.0000000000121\\
-2.73346693386774	4.00000000001265\\
-2.71743486973948	4.00000000001322\\
-2.70140280561122	4.00000000001382\\
-2.68537074148297	4.00000000001443\\
-2.66933867735471	4.00000000001507\\
-2.65330661322645	4.00000000001574\\
-2.6372745490982	4.00000000001643\\
-2.62124248496994	4.00000000001714\\
-2.60521042084168	4.00000000001789\\
-2.58917835671343	4.00000000001865\\
-2.57314629258517	4.00000000001945\\
-2.55711422845691	4.00000000002027\\
-2.54108216432866	4.00000000002113\\
-2.5250501002004	4.00000000002201\\
-2.50901803607214	4.00000000002292\\
-2.49298597194389	4.00000000002387\\
-2.47695390781563	4.00000000002485\\
-2.46092184368737	4.00000000002586\\
-2.44488977955912	4.0000000000269\\
-2.42885771543086	4.00000000002798\\
-2.41282565130261	4.00000000002909\\
-2.39679358717435	4.00000000003025\\
-2.38076152304609	4.00000000003143\\
-2.36472945891784	4.00000000003266\\
-2.34869739478958	4.00000000003392\\
-2.33266533066132	4.00000000003523\\
-2.31663326653307	4.00000000003657\\
-2.30060120240481	4.00000000003796\\
-2.28456913827655	4.00000000003939\\
-2.2685370741483	4.00000000004086\\
-2.25250501002004	4.00000000004237\\
-2.23647294589178	4.00000000004393\\
-2.22044088176353	4.00000000004553\\
-2.20440881763527	4.00000000004718\\
-2.18837675350701	4.00000000004888\\
-2.17234468937876	4.00000000005063\\
-2.1563126252505	4.00000000005242\\
-2.14028056112224	4.00000000005426\\
-2.12424849699399	4.00000000005616\\
-2.10821643286573	4.0000000000581\\
-2.09218436873747	4.0000000000601\\
-2.07615230460922	4.00000000006215\\
-2.06012024048096	4.00000000006425\\
-2.04408817635271	4.0000000000664\\
-2.02805611222445	4.00000000006861\\
-2.01202404809619	4.00000000007087\\
-1.99599198396794	4.00000000007319\\
-1.97995991983968	4.00000000007557\\
-1.96392785571142	4.000000000078\\
-1.94789579158317	4.00000000008049\\
-1.93186372745491	4.00000000008304\\
-1.91583166332665	4.00000000008565\\
-1.8997995991984	4.00000000008832\\
-1.88376753507014	4.00000000009104\\
-1.86773547094188	4.00000000009383\\
-1.85170340681363	4.00000000009667\\
-1.83567134268537	4.00000000009958\\
-1.81963927855711	4.00000000010254\\
-1.80360721442886	4.00000000010557\\
-1.7875751503006	4.00000000010866\\
-1.77154308617234	4.00000000011181\\
-1.75551102204409	4.00000000011502\\
-1.73947895791583	4.00000000011829\\
-1.72344689378758	4.00000000012163\\
-1.70741482965932	4.00000000012502\\
-1.69138276553106	4.00000000012848\\
-1.67535070140281	4.000000000132\\
-1.65931863727455	4.00000000013558\\
-1.64328657314629	4.00000000013922\\
-1.62725450901804	4.00000000014293\\
-1.61122244488978	4.00000000014669\\
-1.59519038076152	4.00000000015052\\
-1.57915831663327	4.0000000001544\\
-1.56312625250501	4.00000000015834\\
-1.54709418837675	4.00000000016234\\
-1.5310621242485	4.0000000001664\\
-1.51503006012024	4.00000000017052\\
-1.49899799599198	4.0000000001747\\
-1.48296593186373	4.00000000017893\\
-1.46693386773547	4.00000000018321\\
-1.45090180360721	4.00000000018755\\
-1.43486973947896	4.00000000019195\\
-1.4188376753507	4.00000000019639\\
-1.40280561122244	4.00000000020089\\
-1.38677354709419	4.00000000020544\\
-1.37074148296593	4.00000000021003\\
-1.35470941883768	4.00000000021467\\
-1.33867735470942	4.00000000021936\\
-1.32264529058116	4.0000000002241\\
-1.30661322645291	4.00000000022887\\
-1.29058116232465	4.00000000023369\\
-1.27454909819639	4.00000000023855\\
-1.25851703406814	4.00000000024345\\
-1.24248496993988	4.00000000024838\\
-1.22645290581162	4.00000000025335\\
-1.21042084168337	4.00000000025835\\
-1.19438877755511	4.00000000026338\\
-1.17835671342685	4.00000000026844\\
-1.1623246492986	4.00000000027353\\
-1.14629258517034	4.00000000027865\\
-1.13026052104208	4.00000000028378\\
-1.11422845691383	4.00000000028894\\
-1.09819639278557	4.00000000029411\\
-1.08216432865731	4.0000000002993\\
-1.06613226452906	4.0000000003045\\
-1.0501002004008	4.00000000030971\\
-1.03406813627255	4.00000000031493\\
-1.01803607214429	4.00000000032016\\
-1.00200400801603	4.00000000032539\\
-0.985971943887776	4.00000000033062\\
-0.969939879759519	4.00000000033585\\
-0.953907815631263	4.00000000034107\\
-0.937875751503006	4.00000000034628\\
-0.92184368737475	4.00000000035149\\
-0.905811623246493	4.00000000035668\\
-0.889779559118236	4.00000000036185\\
-0.87374749498998	4.000000000367\\
-0.857715430861723	4.00000000037214\\
-0.841683366733467	4.00000000037724\\
-0.82565130260521	4.00000000038232\\
-0.809619238476954	4.00000000038737\\
-0.793587174348697	4.00000000039238\\
-0.777555110220441	4.00000000039736\\
-0.761523046092184	4.00000000040229\\
-0.745490981963928	4.00000000040718\\
-0.729458917835671	4.00000000041203\\
-0.713426853707415	4.00000000041682\\
-0.697394789579158	4.00000000042156\\
-0.681362725450902	4.00000000042625\\
-0.665330661322646	4.00000000043088\\
-0.649298597194389	4.00000000043545\\
-0.633266533066132	4.00000000043995\\
-0.617234468937876	4.00000000044438\\
-0.601202404809619	4.00000000044875\\
-0.585170340681363	4.00000000045303\\
-0.569138276553106	4.00000000045725\\
-0.55310621242485	4.00000000046138\\
-0.537074148296593	4.00000000046543\\
-0.521042084168337	4.0000000004694\\
-0.50501002004008	4.00000000047328\\
-0.488977955911824	4.00000000047706\\
-0.472945891783567	4.00000000048076\\
-0.456913827655311	4.00000000048436\\
-0.440881763527054	4.00000000048786\\
-0.424849699398798	4.00000000049126\\
-0.408817635270541	4.00000000049455\\
-0.392785571142285	4.00000000049774\\
-0.376753507014028	4.00000000050082\\
-0.360721442885771	4.00000000050379\\
-0.344689378757515	4.00000000050665\\
-0.328657314629258	4.00000000050939\\
-0.312625250501002	4.00000000051202\\
-0.296593186372745	4.00000000051453\\
-0.280561122244489	4.00000000051691\\
-0.264529058116232	4.00000000051918\\
-0.248496993987976	4.00000000052132\\
-0.232464929859719	4.00000000052333\\
-0.216432865731463	4.00000000052522\\
-0.200400801603207	4.00000000052698\\
-0.18436873747495	4.00000000052861\\
-0.168336673346694	4.0000000005301\\
-0.152304609218437	4.00000000053147\\
-0.13627254509018	4.0000000005327\\
-0.120240480961924	4.0000000005338\\
-0.104208416833667	4.00000000053476\\
-0.0881763527054109	4.00000000053558\\
-0.0721442885771544	4.00000000053627\\
-0.0561122244488979	4.00000000053682\\
-0.0400801603206413	4.00000000053724\\
-0.0240480961923848	4.00000000053751\\
-0.00801603206412826	4.00000000053765\\
0.00801603206412782	4.00000000053765\\
0.0240480961923843	4.00000000053751\\
0.0400801603206409	4.00000000053724\\
0.0561122244488974	4.00000000053682\\
0.0721442885771539	4.00000000053627\\
0.0881763527054105	4.00000000053558\\
0.104208416833667	4.00000000053476\\
0.120240480961924	4.0000000005338\\
0.13627254509018	4.0000000005327\\
0.152304609218437	4.00000000053147\\
0.168336673346693	4.0000000005301\\
0.18436873747495	4.00000000052861\\
0.200400801603206	4.00000000052698\\
0.216432865731463	4.00000000052522\\
0.232464929859719	4.00000000052333\\
0.248496993987976	4.00000000052132\\
0.264529058116232	4.00000000051918\\
0.280561122244489	4.00000000051691\\
0.296593186372745	4.00000000051453\\
0.312625250501002	4.00000000051202\\
0.328657314629258	4.00000000050939\\
0.344689378757515	4.00000000050665\\
0.360721442885771	4.00000000050379\\
0.376753507014028	4.00000000050082\\
0.392785571142285	4.00000000049774\\
0.408817635270541	4.00000000049455\\
0.424849699398798	4.00000000049126\\
0.440881763527054	4.00000000048786\\
0.456913827655311	4.00000000048436\\
0.472945891783567	4.00000000048076\\
0.488977955911824	4.00000000047706\\
0.50501002004008	4.00000000047328\\
0.521042084168337	4.0000000004694\\
0.537074148296593	4.00000000046543\\
0.55310621242485	4.00000000046138\\
0.569138276553106	4.00000000045725\\
0.585170340681363	4.00000000045303\\
0.601202404809619	4.00000000044875\\
0.617234468937876	4.00000000044438\\
0.633266533066132	4.00000000043995\\
0.649298597194389	4.00000000043545\\
0.665330661322646	4.00000000043088\\
0.681362725450902	4.00000000042625\\
0.697394789579159	4.00000000042156\\
0.713426853707415	4.00000000041682\\
0.729458917835672	4.00000000041203\\
0.745490981963928	4.00000000040718\\
0.761523046092185	4.00000000040229\\
0.777555110220441	4.00000000039736\\
0.793587174348698	4.00000000039238\\
0.809619238476954	4.00000000038737\\
0.825651302605211	4.00000000038232\\
0.841683366733467	4.00000000037724\\
0.857715430861724	4.00000000037214\\
0.87374749498998	4.000000000367\\
0.889779559118236	4.00000000036185\\
0.905811623246493	4.00000000035668\\
0.921843687374749	4.00000000035149\\
0.937875751503006	4.00000000034628\\
0.953907815631262	4.00000000034107\\
0.969939879759519	4.00000000033585\\
0.985971943887775	4.00000000033062\\
1.00200400801603	4.00000000032539\\
1.01803607214429	4.00000000032016\\
1.03406813627254	4.00000000031493\\
1.0501002004008	4.00000000030971\\
1.06613226452906	4.0000000003045\\
1.08216432865731	4.0000000002993\\
1.09819639278557	4.00000000029411\\
1.11422845691383	4.00000000028894\\
1.13026052104208	4.00000000028378\\
1.14629258517034	4.00000000027865\\
1.1623246492986	4.00000000027353\\
1.17835671342685	4.00000000026844\\
1.19438877755511	4.00000000026338\\
1.21042084168337	4.00000000025835\\
1.22645290581162	4.00000000025335\\
1.24248496993988	4.00000000024838\\
1.25851703406814	4.00000000024345\\
1.27454909819639	4.00000000023855\\
1.29058116232465	4.00000000023369\\
1.30661322645291	4.00000000022887\\
1.32264529058116	4.0000000002241\\
1.33867735470942	4.00000000021936\\
1.35470941883768	4.00000000021467\\
1.37074148296593	4.00000000021003\\
1.38677354709419	4.00000000020544\\
1.40280561122244	4.00000000020089\\
1.4188376753507	4.00000000019639\\
1.43486973947896	4.00000000019195\\
1.45090180360721	4.00000000018755\\
1.46693386773547	4.00000000018321\\
1.48296593186373	4.00000000017893\\
1.49899799599198	4.0000000001747\\
1.51503006012024	4.00000000017052\\
1.5310621242485	4.0000000001664\\
1.54709418837675	4.00000000016234\\
1.56312625250501	4.00000000015834\\
1.57915831663327	4.0000000001544\\
1.59519038076152	4.00000000015052\\
1.61122244488978	4.00000000014669\\
1.62725450901804	4.00000000014293\\
1.64328657314629	4.00000000013922\\
1.65931863727455	4.00000000013558\\
1.67535070140281	4.000000000132\\
1.69138276553106	4.00000000012848\\
1.70741482965932	4.00000000012502\\
1.72344689378758	4.00000000012163\\
1.73947895791583	4.00000000011829\\
1.75551102204409	4.00000000011502\\
1.77154308617235	4.00000000011181\\
1.7875751503006	4.00000000010866\\
1.80360721442886	4.00000000010557\\
1.81963927855711	4.00000000010254\\
1.83567134268537	4.00000000009958\\
1.85170340681363	4.00000000009667\\
1.86773547094188	4.00000000009383\\
1.88376753507014	4.00000000009104\\
1.8997995991984	4.00000000008832\\
1.91583166332665	4.00000000008565\\
1.93186372745491	4.00000000008304\\
1.94789579158317	4.00000000008049\\
1.96392785571142	4.000000000078\\
1.97995991983968	4.00000000007557\\
1.99599198396794	4.00000000007319\\
2.01202404809619	4.00000000007087\\
2.02805611222445	4.00000000006861\\
2.04408817635271	4.0000000000664\\
2.06012024048096	4.00000000006425\\
2.07615230460922	4.00000000006215\\
2.09218436873747	4.0000000000601\\
2.10821643286573	4.0000000000581\\
2.12424849699399	4.00000000005616\\
2.14028056112224	4.00000000005426\\
2.1563126252505	4.00000000005242\\
2.17234468937876	4.00000000005063\\
2.18837675350701	4.00000000004888\\
2.20440881763527	4.00000000004718\\
2.22044088176353	4.00000000004553\\
2.23647294589178	4.00000000004393\\
2.25250501002004	4.00000000004237\\
2.2685370741483	4.00000000004086\\
2.28456913827655	4.00000000003939\\
2.30060120240481	4.00000000003796\\
2.31663326653307	4.00000000003657\\
2.33266533066132	4.00000000003523\\
2.34869739478958	4.00000000003392\\
2.36472945891784	4.00000000003266\\
2.38076152304609	4.00000000003143\\
2.39679358717435	4.00000000003025\\
2.41282565130261	4.00000000002909\\
2.42885771543086	4.00000000002798\\
2.44488977955912	4.0000000000269\\
2.46092184368737	4.00000000002586\\
2.47695390781563	4.00000000002485\\
2.49298597194389	4.00000000002387\\
2.50901803607214	4.00000000002292\\
2.5250501002004	4.00000000002201\\
2.54108216432866	4.00000000002113\\
2.55711422845691	4.00000000002027\\
2.57314629258517	4.00000000001945\\
2.58917835671343	4.00000000001865\\
2.60521042084168	4.00000000001789\\
2.62124248496994	4.00000000001714\\
2.6372745490982	4.00000000001643\\
2.65330661322645	4.00000000001574\\
2.66933867735471	4.00000000001507\\
2.68537074148297	4.00000000001443\\
2.70140280561122	4.00000000001382\\
2.71743486973948	4.00000000001322\\
2.73346693386774	4.00000000001265\\
2.74949899799599	4.0000000000121\\
2.76553106212425	4.00000000001157\\
2.7815631262525	4.00000000001105\\
2.79759519038076	4.00000000001056\\
2.81362725450902	4.00000000001009\\
2.82965931863727	4.00000000000964\\
2.84569138276553	4.0000000000092\\
2.86172344689379	4.00000000000878\\
2.87775551102204	4.00000000000838\\
2.8937875751503	4.00000000000799\\
2.90981963927856	4.00000000000762\\
2.92585170340681	4.00000000000726\\
2.94188376753507	4.00000000000692\\
2.95791583166333	4.00000000000659\\
2.97394789579158	4.00000000000628\\
2.98997995991984	4.00000000000598\\
3.0060120240481	4.00000000000569\\
3.02204408817635	4.00000000000541\\
3.03807615230461	4.00000000000515\\
3.05410821643287	4.00000000000489\\
3.07014028056112	4.00000000000465\\
3.08617234468938	4.00000000000442\\
3.10220440881764	4.00000000000419\\
3.11823647294589	4.00000000000398\\
3.13426853707415	4.00000000000378\\
3.1503006012024	4.00000000000358\\
3.16633266533066	4.0000000000034\\
3.18236472945892	4.00000000000322\\
3.19839679358717	4.00000000000305\\
3.21442885771543	4.00000000000289\\
3.23046092184369	4.00000000000273\\
3.24649298597194	4.00000000000259\\
3.2625250501002	4.00000000000245\\
3.27855711422846	4.00000000000231\\
3.29458917835671	4.00000000000218\\
3.31062124248497	4.00000000000206\\
3.32665330661323	4.00000000000195\\
3.34268537074148	4.00000000000183\\
3.35871743486974	4.00000000000173\\
3.374749498998	4.00000000000163\\
3.39078156312625	4.00000000000153\\
3.40681362725451	4.00000000000144\\
3.42284569138277	4.00000000000136\\
3.43887775551102	4.00000000000127\\
3.45490981963928	4.0000000000012\\
3.47094188376754	4.00000000000112\\
3.48697394789579	4.00000000000105\\
3.50300601202405	4.00000000000098\\
3.51903807615231	4.00000000000092\\
3.53507014028056	4.00000000000086\\
3.55110220440882	4.0000000000008\\
3.56713426853707	4.00000000000075\\
3.58316633266533	4.0000000000007\\
3.59919839679359	4.00000000000065\\
3.61523046092184	4.0000000000006\\
3.6312625250501	4.00000000000056\\
3.64729458917836	4.00000000000052\\
3.66332665330661	4.00000000000048\\
3.67935871743487	4.00000000000044\\
3.69539078156313	4.0000000000004\\
3.71142284569138	4.00000000000037\\
3.72745490981964	4.00000000000034\\
3.7434869739479	4.00000000000031\\
3.75951903807615	4.00000000000028\\
3.77555110220441	4.00000000000025\\
3.79158316633267	4.00000000000023\\
3.80761523046092	4.0000000000002\\
3.82364729458918	4.00000000000018\\
3.83967935871743	4.00000000000016\\
3.85571142284569	4.00000000000014\\
3.87174348697395	4.00000000000012\\
3.8877755511022	4.0000000000001\\
3.90380761523046	4.00000000000008\\
3.91983967935872	4.00000000000007\\
3.93587174348697	4.00000000000005\\
3.95190380761523	4.00000000000004\\
3.96793587174349	4.00000000000002\\
3.98396793587174	4.00000000000001\\
4	4\\
4.00000000000001	3.98396793587174\\
4.00000000000002	3.96793587174349\\
4.00000000000004	3.95190380761523\\
4.00000000000005	3.93587174348697\\
4.00000000000007	3.91983967935872\\
4.00000000000008	3.90380761523046\\
4.0000000000001	3.8877755511022\\
4.00000000000012	3.87174348697395\\
4.00000000000014	3.85571142284569\\
4.00000000000016	3.83967935871743\\
4.00000000000018	3.82364729458918\\
4.0000000000002	3.80761523046092\\
4.00000000000023	3.79158316633267\\
4.00000000000025	3.77555110220441\\
4.00000000000028	3.75951903807615\\
4.00000000000031	3.7434869739479\\
4.00000000000034	3.72745490981964\\
4.00000000000037	3.71142284569138\\
4.0000000000004	3.69539078156313\\
4.00000000000044	3.67935871743487\\
4.00000000000048	3.66332665330661\\
4.00000000000052	3.64729458917836\\
4.00000000000056	3.6312625250501\\
4.0000000000006	3.61523046092184\\
4.00000000000065	3.59919839679359\\
4.0000000000007	3.58316633266533\\
4.00000000000075	3.56713426853707\\
4.0000000000008	3.55110220440882\\
4.00000000000086	3.53507014028056\\
4.00000000000092	3.51903807615231\\
4.00000000000098	3.50300601202405\\
4.00000000000105	3.48697394789579\\
4.00000000000112	3.47094188376754\\
4.0000000000012	3.45490981963928\\
4.00000000000127	3.43887775551102\\
4.00000000000136	3.42284569138277\\
4.00000000000144	3.40681362725451\\
4.00000000000153	3.39078156312625\\
4.00000000000163	3.374749498998\\
4.00000000000173	3.35871743486974\\
4.00000000000183	3.34268537074148\\
4.00000000000195	3.32665330661323\\
4.00000000000206	3.31062124248497\\
4.00000000000218	3.29458917835671\\
4.00000000000231	3.27855711422846\\
4.00000000000245	3.2625250501002\\
4.00000000000259	3.24649298597194\\
4.00000000000273	3.23046092184369\\
4.00000000000289	3.21442885771543\\
4.00000000000305	3.19839679358717\\
4.00000000000322	3.18236472945892\\
4.0000000000034	3.16633266533066\\
4.00000000000358	3.1503006012024\\
4.00000000000378	3.13426853707415\\
4.00000000000398	3.11823647294589\\
4.00000000000419	3.10220440881764\\
4.00000000000442	3.08617234468938\\
4.00000000000465	3.07014028056112\\
4.00000000000489	3.05410821643287\\
4.00000000000515	3.03807615230461\\
4.00000000000541	3.02204408817635\\
4.00000000000569	3.0060120240481\\
4.00000000000598	2.98997995991984\\
4.00000000000628	2.97394789579158\\
4.00000000000659	2.95791583166333\\
4.00000000000692	2.94188376753507\\
4.00000000000726	2.92585170340681\\
4.00000000000762	2.90981963927856\\
4.00000000000799	2.8937875751503\\
4.00000000000838	2.87775551102204\\
4.00000000000878	2.86172344689379\\
4.0000000000092	2.84569138276553\\
4.00000000000964	2.82965931863727\\
4.00000000001009	2.81362725450902\\
4.00000000001056	2.79759519038076\\
4.00000000001105	2.7815631262525\\
4.00000000001157	2.76553106212425\\
4.0000000000121	2.74949899799599\\
4.00000000001265	2.73346693386774\\
4.00000000001322	2.71743486973948\\
4.00000000001382	2.70140280561122\\
4.00000000001443	2.68537074148297\\
4.00000000001507	2.66933867735471\\
4.00000000001574	2.65330661322645\\
4.00000000001643	2.6372745490982\\
4.00000000001714	2.62124248496994\\
4.00000000001789	2.60521042084168\\
4.00000000001865	2.58917835671343\\
4.00000000001945	2.57314629258517\\
4.00000000002027	2.55711422845691\\
4.00000000002113	2.54108216432866\\
4.00000000002201	2.5250501002004\\
4.00000000002292	2.50901803607214\\
4.00000000002387	2.49298597194389\\
4.00000000002485	2.47695390781563\\
4.00000000002586	2.46092184368737\\
4.0000000000269	2.44488977955912\\
4.00000000002798	2.42885771543086\\
4.00000000002909	2.41282565130261\\
4.00000000003025	2.39679358717435\\
4.00000000003143	2.38076152304609\\
4.00000000003266	2.36472945891784\\
4.00000000003392	2.34869739478958\\
4.00000000003523	2.33266533066132\\
4.00000000003657	2.31663326653307\\
4.00000000003796	2.30060120240481\\
4.00000000003939	2.28456913827655\\
4.00000000004086	2.2685370741483\\
4.00000000004237	2.25250501002004\\
4.00000000004393	2.23647294589178\\
4.00000000004553	2.22044088176353\\
4.00000000004718	2.20440881763527\\
4.00000000004888	2.18837675350701\\
4.00000000005063	2.17234468937876\\
4.00000000005242	2.1563126252505\\
4.00000000005426	2.14028056112224\\
4.00000000005616	2.12424849699399\\
4.0000000000581	2.10821643286573\\
4.0000000000601	2.09218436873747\\
4.00000000006215	2.07615230460922\\
4.00000000006425	2.06012024048096\\
4.0000000000664	2.04408817635271\\
4.00000000006861	2.02805611222445\\
4.00000000007087	2.01202404809619\\
4.00000000007319	1.99599198396794\\
4.00000000007557	1.97995991983968\\
4.000000000078	1.96392785571142\\
4.00000000008049	1.94789579158317\\
4.00000000008304	1.93186372745491\\
4.00000000008565	1.91583166332665\\
4.00000000008832	1.8997995991984\\
4.00000000009104	1.88376753507014\\
4.00000000009383	1.86773547094188\\
4.00000000009667	1.85170340681363\\
4.00000000009958	1.83567134268537\\
4.00000000010254	1.81963927855711\\
4.00000000010557	1.80360721442886\\
4.00000000010866	1.7875751503006\\
4.00000000011181	1.77154308617235\\
4.00000000011502	1.75551102204409\\
4.00000000011829	1.73947895791583\\
4.00000000012163	1.72344689378758\\
4.00000000012502	1.70741482965932\\
4.00000000012848	1.69138276553106\\
4.000000000132	1.67535070140281\\
4.00000000013558	1.65931863727455\\
4.00000000013922	1.64328657314629\\
4.00000000014293	1.62725450901804\\
4.00000000014669	1.61122244488978\\
4.00000000015052	1.59519038076152\\
4.0000000001544	1.57915831663327\\
4.00000000015834	1.56312625250501\\
4.00000000016234	1.54709418837675\\
4.0000000001664	1.5310621242485\\
4.00000000017052	1.51503006012024\\
4.0000000001747	1.49899799599198\\
4.00000000017893	1.48296593186373\\
4.00000000018321	1.46693386773547\\
4.00000000018755	1.45090180360721\\
4.00000000019195	1.43486973947896\\
4.00000000019639	1.4188376753507\\
4.00000000020089	1.40280561122244\\
4.00000000020544	1.38677354709419\\
4.00000000021003	1.37074148296593\\
4.00000000021467	1.35470941883768\\
4.00000000021936	1.33867735470942\\
4.0000000002241	1.32264529058116\\
4.00000000022887	1.30661322645291\\
4.00000000023369	1.29058116232465\\
4.00000000023855	1.27454909819639\\
4.00000000024345	1.25851703406814\\
4.00000000024838	1.24248496993988\\
4.00000000025335	1.22645290581162\\
4.00000000025835	1.21042084168337\\
4.00000000026338	1.19438877755511\\
4.00000000026844	1.17835671342685\\
4.00000000027353	1.1623246492986\\
4.00000000027865	1.14629258517034\\
4.00000000028378	1.13026052104208\\
4.00000000028894	1.11422845691383\\
4.00000000029411	1.09819639278557\\
4.0000000002993	1.08216432865731\\
4.0000000003045	1.06613226452906\\
4.00000000030971	1.0501002004008\\
4.00000000031493	1.03406813627254\\
4.00000000032016	1.01803607214429\\
4.00000000032539	1.00200400801603\\
4.00000000033062	0.985971943887775\\
4.00000000033585	0.969939879759519\\
4.00000000034107	0.953907815631262\\
4.00000000034628	0.937875751503006\\
4.00000000035149	0.921843687374749\\
4.00000000035668	0.905811623246493\\
4.00000000036185	0.889779559118236\\
4.000000000367	0.87374749498998\\
4.00000000037214	0.857715430861724\\
4.00000000037724	0.841683366733467\\
4.00000000038232	0.825651302605211\\
4.00000000038737	0.809619238476954\\
4.00000000039238	0.793587174348698\\
4.00000000039736	0.777555110220441\\
4.00000000040229	0.761523046092185\\
4.00000000040718	0.745490981963928\\
4.00000000041203	0.729458917835672\\
4.00000000041682	0.713426853707415\\
4.00000000042156	0.697394789579159\\
4.00000000042625	0.681362725450902\\
4.00000000043088	0.665330661322646\\
4.00000000043545	0.649298597194389\\
4.00000000043995	0.633266533066132\\
4.00000000044438	0.617234468937876\\
4.00000000044875	0.601202404809619\\
4.00000000045303	0.585170340681363\\
4.00000000045725	0.569138276553106\\
4.00000000046138	0.55310621242485\\
4.00000000046543	0.537074148296593\\
4.0000000004694	0.521042084168337\\
4.00000000047328	0.50501002004008\\
4.00000000047706	0.488977955911824\\
4.00000000048076	0.472945891783567\\
4.00000000048436	0.456913827655311\\
4.00000000048786	0.440881763527054\\
4.00000000049126	0.424849699398798\\
4.00000000049455	0.408817635270541\\
4.00000000049774	0.392785571142285\\
4.00000000050082	0.376753507014028\\
4.00000000050379	0.360721442885771\\
4.00000000050665	0.344689378757515\\
4.00000000050939	0.328657314629258\\
4.00000000051202	0.312625250501002\\
4.00000000051453	0.296593186372745\\
4.00000000051691	0.280561122244489\\
4.00000000051918	0.264529058116232\\
4.00000000052132	0.248496993987976\\
4.00000000052333	0.232464929859719\\
4.00000000052522	0.216432865731463\\
4.00000000052698	0.200400801603206\\
4.00000000052861	0.18436873747495\\
4.0000000005301	0.168336673346693\\
4.00000000053147	0.152304609218437\\
4.0000000005327	0.13627254509018\\
4.0000000005338	0.120240480961924\\
4.00000000053476	0.104208416833667\\
4.00000000053558	0.0881763527054105\\
4.00000000053627	0.0721442885771539\\
4.00000000053682	0.0561122244488974\\
4.00000000053724	0.0400801603206409\\
4.00000000053751	0.0240480961923843\\
4.00000000053765	0.00801603206412782\\
4.00000000053765	-0.00801603206412826\\
4.00000000053751	-0.0240480961923848\\
4.00000000053724	-0.0400801603206413\\
4.00000000053682	-0.0561122244488979\\
4.00000000053627	-0.0721442885771544\\
4.00000000053558	-0.0881763527054109\\
4.00000000053476	-0.104208416833667\\
4.0000000005338	-0.120240480961924\\
4.0000000005327	-0.13627254509018\\
4.00000000053147	-0.152304609218437\\
4.0000000005301	-0.168336673346694\\
4.00000000052861	-0.18436873747495\\
4.00000000052698	-0.200400801603207\\
4.00000000052522	-0.216432865731463\\
4.00000000052333	-0.232464929859719\\
4.00000000052132	-0.248496993987976\\
4.00000000051918	-0.264529058116232\\
4.00000000051691	-0.280561122244489\\
4.00000000051453	-0.296593186372745\\
4.00000000051202	-0.312625250501002\\
4.00000000050939	-0.328657314629258\\
4.00000000050665	-0.344689378757515\\
4.00000000050379	-0.360721442885771\\
4.00000000050082	-0.376753507014028\\
4.00000000049774	-0.392785571142285\\
4.00000000049455	-0.408817635270541\\
4.00000000049126	-0.424849699398798\\
4.00000000048786	-0.440881763527054\\
4.00000000048436	-0.456913827655311\\
4.00000000048076	-0.472945891783567\\
4.00000000047706	-0.488977955911824\\
4.00000000047328	-0.50501002004008\\
4.0000000004694	-0.521042084168337\\
4.00000000046543	-0.537074148296593\\
4.00000000046138	-0.55310621242485\\
4.00000000045725	-0.569138276553106\\
4.00000000045303	-0.585170340681363\\
4.00000000044875	-0.601202404809619\\
4.00000000044438	-0.617234468937876\\
4.00000000043995	-0.633266533066132\\
4.00000000043545	-0.649298597194389\\
4.00000000043088	-0.665330661322646\\
4.00000000042625	-0.681362725450902\\
4.00000000042156	-0.697394789579158\\
4.00000000041682	-0.713426853707415\\
4.00000000041203	-0.729458917835671\\
4.00000000040718	-0.745490981963928\\
4.00000000040229	-0.761523046092184\\
4.00000000039736	-0.777555110220441\\
4.00000000039238	-0.793587174348697\\
4.00000000038737	-0.809619238476954\\
4.00000000038232	-0.82565130260521\\
4.00000000037724	-0.841683366733467\\
4.00000000037214	-0.857715430861723\\
4.000000000367	-0.87374749498998\\
4.00000000036185	-0.889779559118236\\
4.00000000035668	-0.905811623246493\\
4.00000000035149	-0.92184368737475\\
4.00000000034628	-0.937875751503006\\
4.00000000034107	-0.953907815631263\\
4.00000000033585	-0.969939879759519\\
4.00000000033062	-0.985971943887776\\
4.00000000032539	-1.00200400801603\\
4.00000000032016	-1.01803607214429\\
4.00000000031493	-1.03406813627255\\
4.00000000030971	-1.0501002004008\\
4.0000000003045	-1.06613226452906\\
4.0000000002993	-1.08216432865731\\
4.00000000029411	-1.09819639278557\\
4.00000000028894	-1.11422845691383\\
4.00000000028378	-1.13026052104208\\
4.00000000027865	-1.14629258517034\\
4.00000000027353	-1.1623246492986\\
4.00000000026844	-1.17835671342685\\
4.00000000026338	-1.19438877755511\\
4.00000000025835	-1.21042084168337\\
4.00000000025335	-1.22645290581162\\
4.00000000024838	-1.24248496993988\\
4.00000000024345	-1.25851703406814\\
4.00000000023855	-1.27454909819639\\
4.00000000023369	-1.29058116232465\\
4.00000000022887	-1.30661322645291\\
4.0000000002241	-1.32264529058116\\
4.00000000021936	-1.33867735470942\\
4.00000000021467	-1.35470941883768\\
4.00000000021003	-1.37074148296593\\
4.00000000020544	-1.38677354709419\\
4.00000000020089	-1.40280561122244\\
4.00000000019639	-1.4188376753507\\
4.00000000019195	-1.43486973947896\\
4.00000000018755	-1.45090180360721\\
4.00000000018321	-1.46693386773547\\
4.00000000017893	-1.48296593186373\\
4.0000000001747	-1.49899799599198\\
4.00000000017052	-1.51503006012024\\
4.0000000001664	-1.5310621242485\\
4.00000000016234	-1.54709418837675\\
4.00000000015834	-1.56312625250501\\
4.0000000001544	-1.57915831663327\\
4.00000000015052	-1.59519038076152\\
4.00000000014669	-1.61122244488978\\
4.00000000014293	-1.62725450901804\\
4.00000000013922	-1.64328657314629\\
4.00000000013558	-1.65931863727455\\
4.000000000132	-1.67535070140281\\
4.00000000012848	-1.69138276553106\\
4.00000000012502	-1.70741482965932\\
4.00000000012163	-1.72344689378758\\
4.00000000011829	-1.73947895791583\\
4.00000000011502	-1.75551102204409\\
4.00000000011181	-1.77154308617234\\
4.00000000010866	-1.7875751503006\\
4.00000000010557	-1.80360721442886\\
4.00000000010254	-1.81963927855711\\
4.00000000009958	-1.83567134268537\\
4.00000000009667	-1.85170340681363\\
4.00000000009383	-1.86773547094188\\
4.00000000009104	-1.88376753507014\\
4.00000000008832	-1.8997995991984\\
4.00000000008565	-1.91583166332665\\
4.00000000008304	-1.93186372745491\\
4.00000000008049	-1.94789579158317\\
4.000000000078	-1.96392785571142\\
4.00000000007557	-1.97995991983968\\
4.00000000007319	-1.99599198396794\\
4.00000000007087	-2.01202404809619\\
4.00000000006861	-2.02805611222445\\
4.0000000000664	-2.04408817635271\\
4.00000000006425	-2.06012024048096\\
4.00000000006215	-2.07615230460922\\
4.0000000000601	-2.09218436873747\\
4.0000000000581	-2.10821643286573\\
4.00000000005616	-2.12424849699399\\
4.00000000005426	-2.14028056112224\\
4.00000000005242	-2.1563126252505\\
4.00000000005063	-2.17234468937876\\
4.00000000004888	-2.18837675350701\\
4.00000000004718	-2.20440881763527\\
4.00000000004553	-2.22044088176353\\
4.00000000004393	-2.23647294589178\\
4.00000000004237	-2.25250501002004\\
4.00000000004086	-2.2685370741483\\
4.00000000003939	-2.28456913827655\\
4.00000000003796	-2.30060120240481\\
4.00000000003657	-2.31663326653307\\
4.00000000003523	-2.33266533066132\\
4.00000000003392	-2.34869739478958\\
4.00000000003266	-2.36472945891784\\
4.00000000003143	-2.38076152304609\\
4.00000000003025	-2.39679358717435\\
4.00000000002909	-2.41282565130261\\
4.00000000002798	-2.42885771543086\\
4.0000000000269	-2.44488977955912\\
4.00000000002586	-2.46092184368737\\
4.00000000002485	-2.47695390781563\\
4.00000000002387	-2.49298597194389\\
4.00000000002292	-2.50901803607214\\
4.00000000002201	-2.5250501002004\\
4.00000000002113	-2.54108216432866\\
4.00000000002027	-2.55711422845691\\
4.00000000001945	-2.57314629258517\\
4.00000000001865	-2.58917835671343\\
4.00000000001789	-2.60521042084168\\
4.00000000001714	-2.62124248496994\\
4.00000000001643	-2.6372745490982\\
4.00000000001574	-2.65330661322645\\
4.00000000001507	-2.66933867735471\\
4.00000000001443	-2.68537074148297\\
4.00000000001382	-2.70140280561122\\
4.00000000001322	-2.71743486973948\\
4.00000000001265	-2.73346693386774\\
4.0000000000121	-2.74949899799599\\
4.00000000001157	-2.76553106212425\\
4.00000000001105	-2.7815631262525\\
4.00000000001056	-2.79759519038076\\
4.00000000001009	-2.81362725450902\\
4.00000000000964	-2.82965931863727\\
4.0000000000092	-2.84569138276553\\
4.00000000000878	-2.86172344689379\\
4.00000000000838	-2.87775551102204\\
4.00000000000799	-2.8937875751503\\
4.00000000000762	-2.90981963927856\\
4.00000000000726	-2.92585170340681\\
4.00000000000692	-2.94188376753507\\
4.00000000000659	-2.95791583166333\\
4.00000000000628	-2.97394789579158\\
4.00000000000598	-2.98997995991984\\
4.00000000000569	-3.0060120240481\\
4.00000000000541	-3.02204408817635\\
4.00000000000515	-3.03807615230461\\
4.00000000000489	-3.05410821643287\\
4.00000000000465	-3.07014028056112\\
4.00000000000442	-3.08617234468938\\
4.00000000000419	-3.10220440881764\\
4.00000000000398	-3.11823647294589\\
4.00000000000378	-3.13426853707415\\
4.00000000000358	-3.1503006012024\\
4.0000000000034	-3.16633266533066\\
4.00000000000322	-3.18236472945892\\
4.00000000000305	-3.19839679358717\\
4.00000000000289	-3.21442885771543\\
4.00000000000273	-3.23046092184369\\
4.00000000000259	-3.24649298597194\\
4.00000000000245	-3.2625250501002\\
4.00000000000231	-3.27855711422846\\
4.00000000000218	-3.29458917835671\\
4.00000000000206	-3.31062124248497\\
4.00000000000195	-3.32665330661323\\
4.00000000000183	-3.34268537074148\\
4.00000000000173	-3.35871743486974\\
4.00000000000163	-3.374749498998\\
4.00000000000153	-3.39078156312625\\
4.00000000000144	-3.40681362725451\\
4.00000000000136	-3.42284569138277\\
4.00000000000127	-3.43887775551102\\
4.0000000000012	-3.45490981963928\\
4.00000000000112	-3.47094188376753\\
4.00000000000105	-3.48697394789579\\
4.00000000000098	-3.50300601202405\\
4.00000000000092	-3.5190380761523\\
4.00000000000086	-3.53507014028056\\
4.0000000000008	-3.55110220440882\\
4.00000000000075	-3.56713426853707\\
4.0000000000007	-3.58316633266533\\
4.00000000000065	-3.59919839679359\\
4.0000000000006	-3.61523046092184\\
4.00000000000056	-3.6312625250501\\
4.00000000000052	-3.64729458917836\\
4.00000000000048	-3.66332665330661\\
4.00000000000044	-3.67935871743487\\
4.0000000000004	-3.69539078156313\\
4.00000000000037	-3.71142284569138\\
4.00000000000034	-3.72745490981964\\
4.00000000000031	-3.7434869739479\\
4.00000000000028	-3.75951903807615\\
4.00000000000025	-3.77555110220441\\
4.00000000000023	-3.79158316633267\\
4.0000000000002	-3.80761523046092\\
4.00000000000018	-3.82364729458918\\
4.00000000000016	-3.83967935871743\\
4.00000000000014	-3.85571142284569\\
4.00000000000012	-3.87174348697395\\
4.0000000000001	-3.8877755511022\\
4.00000000000008	-3.90380761523046\\
4.00000000000007	-3.91983967935872\\
4.00000000000005	-3.93587174348697\\
4.00000000000004	-3.95190380761523\\
4.00000000000002	-3.96793587174349\\
4.00000000000001	-3.98396793587174\\
4	-4\\
3.98396793587174	-4.00000000000001\\
3.96793587174349	-4.00000000000002\\
3.95190380761523	-4.00000000000004\\
3.93587174348697	-4.00000000000005\\
3.91983967935872	-4.00000000000007\\
3.90380761523046	-4.00000000000008\\
3.8877755511022	-4.0000000000001\\
3.87174348697395	-4.00000000000012\\
3.85571142284569	-4.00000000000014\\
3.83967935871743	-4.00000000000016\\
3.82364729458918	-4.00000000000018\\
3.80761523046092	-4.0000000000002\\
3.79158316633267	-4.00000000000023\\
3.77555110220441	-4.00000000000025\\
3.75951903807615	-4.00000000000028\\
3.7434869739479	-4.00000000000031\\
3.72745490981964	-4.00000000000034\\
3.71142284569138	-4.00000000000037\\
3.69539078156313	-4.0000000000004\\
3.67935871743487	-4.00000000000044\\
3.66332665330661	-4.00000000000048\\
3.64729458917836	-4.00000000000051\\
3.6312625250501	-4.00000000000056\\
3.61523046092184	-4.0000000000006\\
3.59919839679359	-4.00000000000065\\
3.58316633266533	-4.0000000000007\\
3.56713426853707	-4.00000000000075\\
3.55110220440882	-4.0000000000008\\
3.53507014028056	-4.00000000000086\\
3.51903807615231	-4.00000000000092\\
3.50300601202405	-4.00000000000098\\
3.48697394789579	-4.00000000000105\\
3.47094188376754	-4.00000000000112\\
3.45490981963928	-4.0000000000012\\
3.43887775551102	-4.00000000000127\\
3.42284569138277	-4.00000000000136\\
3.40681362725451	-4.00000000000144\\
3.39078156312625	-4.00000000000153\\
3.374749498998	-4.00000000000163\\
3.35871743486974	-4.00000000000173\\
3.34268537074148	-4.00000000000183\\
3.32665330661323	-4.00000000000195\\
3.31062124248497	-4.00000000000206\\
3.29458917835671	-4.00000000000218\\
3.27855711422846	-4.00000000000231\\
3.2625250501002	-4.00000000000245\\
3.24649298597194	-4.00000000000259\\
3.23046092184369	-4.00000000000273\\
3.21442885771543	-4.00000000000289\\
3.19839679358717	-4.00000000000305\\
3.18236472945892	-4.00000000000322\\
3.16633266533066	-4.0000000000034\\
3.1503006012024	-4.00000000000358\\
3.13426853707415	-4.00000000000378\\
3.11823647294589	-4.00000000000398\\
3.10220440881764	-4.00000000000419\\
3.08617234468938	-4.00000000000442\\
3.07014028056112	-4.00000000000465\\
3.05410821643287	-4.00000000000489\\
3.03807615230461	-4.00000000000515\\
3.02204408817635	-4.00000000000541\\
3.0060120240481	-4.00000000000569\\
2.98997995991984	-4.00000000000598\\
2.97394789579158	-4.00000000000628\\
2.95791583166333	-4.00000000000659\\
2.94188376753507	-4.00000000000692\\
2.92585170340681	-4.00000000000726\\
2.90981963927856	-4.00000000000762\\
2.8937875751503	-4.00000000000799\\
2.87775551102204	-4.00000000000838\\
2.86172344689379	-4.00000000000878\\
2.84569138276553	-4.0000000000092\\
2.82965931863727	-4.00000000000964\\
2.81362725450902	-4.00000000001009\\
2.79759519038076	-4.00000000001056\\
2.7815631262525	-4.00000000001105\\
2.76553106212425	-4.00000000001156\\
2.74949899799599	-4.0000000000121\\
2.73346693386774	-4.00000000001265\\
2.71743486973948	-4.00000000001322\\
2.70140280561122	-4.00000000001382\\
2.68537074148297	-4.00000000001443\\
2.66933867735471	-4.00000000001507\\
2.65330661322645	-4.00000000001574\\
2.6372745490982	-4.00000000001643\\
2.62124248496994	-4.00000000001714\\
2.60521042084168	-4.00000000001789\\
2.58917835671343	-4.00000000001865\\
2.57314629258517	-4.00000000001945\\
2.55711422845691	-4.00000000002027\\
2.54108216432866	-4.00000000002113\\
2.5250501002004	-4.00000000002201\\
2.50901803607214	-4.00000000002292\\
2.49298597194389	-4.00000000002387\\
2.47695390781563	-4.00000000002485\\
2.46092184368737	-4.00000000002586\\
2.44488977955912	-4.0000000000269\\
2.42885771543086	-4.00000000002798\\
2.41282565130261	-4.00000000002909\\
2.39679358717435	-4.00000000003025\\
2.38076152304609	-4.00000000003143\\
2.36472945891784	-4.00000000003266\\
2.34869739478958	-4.00000000003392\\
2.33266533066132	-4.00000000003523\\
2.31663326653307	-4.00000000003657\\
2.30060120240481	-4.00000000003796\\
2.28456913827655	-4.00000000003938\\
2.2685370741483	-4.00000000004086\\
2.25250501002004	-4.00000000004237\\
2.23647294589178	-4.00000000004393\\
2.22044088176353	-4.00000000004553\\
2.20440881763527	-4.00000000004718\\
2.18837675350701	-4.00000000004888\\
2.17234468937876	-4.00000000005063\\
2.1563126252505	-4.00000000005242\\
2.14028056112224	-4.00000000005426\\
2.12424849699399	-4.00000000005616\\
2.10821643286573	-4.0000000000581\\
2.09218436873747	-4.0000000000601\\
2.07615230460922	-4.00000000006215\\
2.06012024048096	-4.00000000006425\\
2.04408817635271	-4.0000000000664\\
2.02805611222445	-4.00000000006861\\
2.01202404809619	-4.00000000007087\\
1.99599198396794	-4.00000000007319\\
1.97995991983968	-4.00000000007557\\
1.96392785571142	-4.000000000078\\
1.94789579158317	-4.00000000008049\\
1.93186372745491	-4.00000000008304\\
1.91583166332665	-4.00000000008565\\
1.8997995991984	-4.00000000008832\\
1.88376753507014	-4.00000000009104\\
1.86773547094188	-4.00000000009383\\
1.85170340681363	-4.00000000009667\\
1.83567134268537	-4.00000000009958\\
1.81963927855711	-4.00000000010254\\
1.80360721442886	-4.00000000010557\\
1.7875751503006	-4.00000000010866\\
1.77154308617235	-4.00000000011181\\
1.75551102204409	-4.00000000011502\\
1.73947895791583	-4.00000000011829\\
1.72344689378758	-4.00000000012163\\
1.70741482965932	-4.00000000012502\\
1.69138276553106	-4.00000000012848\\
1.67535070140281	-4.000000000132\\
1.65931863727455	-4.00000000013558\\
1.64328657314629	-4.00000000013922\\
1.62725450901804	-4.00000000014293\\
1.61122244488978	-4.00000000014669\\
1.59519038076152	-4.00000000015052\\
1.57915831663327	-4.0000000001544\\
1.56312625250501	-4.00000000015834\\
1.54709418837675	-4.00000000016234\\
1.5310621242485	-4.0000000001664\\
1.51503006012024	-4.00000000017052\\
1.49899799599198	-4.0000000001747\\
1.48296593186373	-4.00000000017893\\
1.46693386773547	-4.00000000018321\\
1.45090180360721	-4.00000000018755\\
1.43486973947896	-4.00000000019195\\
1.4188376753507	-4.00000000019639\\
1.40280561122244	-4.00000000020089\\
1.38677354709419	-4.00000000020544\\
1.37074148296593	-4.00000000021003\\
1.35470941883768	-4.00000000021467\\
1.33867735470942	-4.00000000021936\\
1.32264529058116	-4.0000000002241\\
1.30661322645291	-4.00000000022887\\
1.29058116232465	-4.00000000023369\\
1.27454909819639	-4.00000000023855\\
1.25851703406814	-4.00000000024345\\
1.24248496993988	-4.00000000024838\\
1.22645290581162	-4.00000000025335\\
1.21042084168337	-4.00000000025835\\
1.19438877755511	-4.00000000026338\\
1.17835671342685	-4.00000000026844\\
1.1623246492986	-4.00000000027353\\
1.14629258517034	-4.00000000027865\\
1.13026052104208	-4.00000000028378\\
1.11422845691383	-4.00000000028894\\
1.09819639278557	-4.00000000029411\\
1.08216432865731	-4.0000000002993\\
1.06613226452906	-4.0000000003045\\
1.0501002004008	-4.00000000030971\\
1.03406813627254	-4.00000000031493\\
1.01803607214429	-4.00000000032016\\
1.00200400801603	-4.00000000032539\\
0.985971943887775	-4.00000000033062\\
0.969939879759519	-4.00000000033585\\
0.953907815631262	-4.00000000034107\\
0.937875751503006	-4.00000000034628\\
0.921843687374749	-4.00000000035149\\
0.905811623246493	-4.00000000035668\\
0.889779559118236	-4.00000000036185\\
0.87374749498998	-4.000000000367\\
0.857715430861724	-4.00000000037214\\
0.841683366733467	-4.00000000037724\\
0.825651302605211	-4.00000000038232\\
0.809619238476954	-4.00000000038737\\
0.793587174348698	-4.00000000039238\\
0.777555110220441	-4.00000000039736\\
0.761523046092185	-4.00000000040229\\
0.745490981963928	-4.00000000040718\\
0.729458917835672	-4.00000000041203\\
0.713426853707415	-4.00000000041682\\
0.697394789579159	-4.00000000042156\\
0.681362725450902	-4.00000000042625\\
0.665330661322646	-4.00000000043088\\
0.649298597194389	-4.00000000043545\\
0.633266533066132	-4.00000000043995\\
0.617234468937876	-4.00000000044438\\
0.601202404809619	-4.00000000044875\\
0.585170340681363	-4.00000000045303\\
0.569138276553106	-4.00000000045725\\
0.55310621242485	-4.00000000046138\\
0.537074148296593	-4.00000000046543\\
0.521042084168337	-4.0000000004694\\
0.50501002004008	-4.00000000047328\\
0.488977955911824	-4.00000000047706\\
0.472945891783567	-4.00000000048076\\
0.456913827655311	-4.00000000048436\\
0.440881763527054	-4.00000000048786\\
0.424849699398798	-4.00000000049126\\
0.408817635270541	-4.00000000049455\\
0.392785571142285	-4.00000000049774\\
0.376753507014028	-4.00000000050082\\
0.360721442885771	-4.00000000050379\\
0.344689378757515	-4.00000000050665\\
0.328657314629258	-4.00000000050939\\
0.312625250501002	-4.00000000051202\\
0.296593186372745	-4.00000000051453\\
0.280561122244489	-4.00000000051691\\
0.264529058116232	-4.00000000051918\\
0.248496993987976	-4.00000000052132\\
0.232464929859719	-4.00000000052333\\
0.216432865731463	-4.00000000052522\\
0.200400801603206	-4.00000000052698\\
0.18436873747495	-4.00000000052861\\
0.168336673346693	-4.0000000005301\\
0.152304609218437	-4.00000000053147\\
0.13627254509018	-4.0000000005327\\
0.120240480961924	-4.0000000005338\\
0.104208416833667	-4.00000000053476\\
0.0881763527054105	-4.00000000053558\\
0.0721442885771539	-4.00000000053627\\
0.0561122244488974	-4.00000000053682\\
0.0400801603206409	-4.00000000053724\\
0.0240480961923843	-4.00000000053751\\
0.00801603206412782	-4.00000000053765\\
-0.00801603206412826	-4.00000000053765\\
-0.0240480961923848	-4.00000000053751\\
-0.0400801603206413	-4.00000000053724\\
-0.0561122244488979	-4.00000000053682\\
-0.0721442885771544	-4.00000000053627\\
-0.0881763527054109	-4.00000000053558\\
-0.104208416833667	-4.00000000053476\\
-0.120240480961924	-4.0000000005338\\
-0.13627254509018	-4.0000000005327\\
-0.152304609218437	-4.00000000053147\\
-0.168336673346694	-4.0000000005301\\
-0.18436873747495	-4.00000000052861\\
-0.200400801603207	-4.00000000052698\\
-0.216432865731463	-4.00000000052522\\
-0.232464929859719	-4.00000000052333\\
-0.248496993987976	-4.00000000052132\\
-0.264529058116232	-4.00000000051918\\
-0.280561122244489	-4.00000000051691\\
-0.296593186372745	-4.00000000051453\\
-0.312625250501002	-4.00000000051202\\
-0.328657314629258	-4.00000000050939\\
-0.344689378757515	-4.00000000050665\\
-0.360721442885771	-4.00000000050379\\
-0.376753507014028	-4.00000000050082\\
-0.392785571142285	-4.00000000049774\\
-0.408817635270541	-4.00000000049455\\
-0.424849699398798	-4.00000000049126\\
-0.440881763527054	-4.00000000048786\\
-0.456913827655311	-4.00000000048436\\
-0.472945891783567	-4.00000000048076\\
-0.488977955911824	-4.00000000047706\\
-0.50501002004008	-4.00000000047328\\
-0.521042084168337	-4.0000000004694\\
-0.537074148296593	-4.00000000046543\\
-0.55310621242485	-4.00000000046138\\
-0.569138276553106	-4.00000000045725\\
-0.585170340681363	-4.00000000045303\\
-0.601202404809619	-4.00000000044875\\
-0.617234468937876	-4.00000000044438\\
-0.633266533066132	-4.00000000043995\\
-0.649298597194389	-4.00000000043545\\
-0.665330661322646	-4.00000000043088\\
-0.681362725450902	-4.00000000042625\\
-0.697394789579158	-4.00000000042156\\
-0.713426853707415	-4.00000000041682\\
-0.729458917835671	-4.00000000041203\\
-0.745490981963928	-4.00000000040718\\
-0.761523046092184	-4.00000000040229\\
-0.777555110220441	-4.00000000039736\\
-0.793587174348697	-4.00000000039238\\
-0.809619238476954	-4.00000000038737\\
-0.82565130260521	-4.00000000038232\\
-0.841683366733467	-4.00000000037724\\
-0.857715430861723	-4.00000000037214\\
-0.87374749498998	-4.000000000367\\
-0.889779559118236	-4.00000000036185\\
-0.905811623246493	-4.00000000035668\\
-0.92184368737475	-4.00000000035149\\
-0.937875751503006	-4.00000000034628\\
-0.953907815631263	-4.00000000034107\\
-0.969939879759519	-4.00000000033585\\
-0.985971943887776	-4.00000000033062\\
-1.00200400801603	-4.00000000032539\\
-1.01803607214429	-4.00000000032016\\
-1.03406813627255	-4.00000000031493\\
-1.0501002004008	-4.00000000030971\\
-1.06613226452906	-4.0000000003045\\
-1.08216432865731	-4.0000000002993\\
-1.09819639278557	-4.00000000029411\\
-1.11422845691383	-4.00000000028894\\
-1.13026052104208	-4.00000000028378\\
-1.14629258517034	-4.00000000027865\\
-1.1623246492986	-4.00000000027353\\
-1.17835671342685	-4.00000000026844\\
-1.19438877755511	-4.00000000026338\\
-1.21042084168337	-4.00000000025835\\
-1.22645290581162	-4.00000000025335\\
-1.24248496993988	-4.00000000024838\\
-1.25851703406814	-4.00000000024345\\
-1.27454909819639	-4.00000000023855\\
-1.29058116232465	-4.00000000023369\\
-1.30661322645291	-4.00000000022887\\
-1.32264529058116	-4.0000000002241\\
-1.33867735470942	-4.00000000021936\\
-1.35470941883768	-4.00000000021467\\
-1.37074148296593	-4.00000000021003\\
-1.38677354709419	-4.00000000020544\\
-1.40280561122244	-4.00000000020089\\
-1.4188376753507	-4.00000000019639\\
-1.43486973947896	-4.00000000019195\\
-1.45090180360721	-4.00000000018755\\
-1.46693386773547	-4.00000000018321\\
-1.48296593186373	-4.00000000017893\\
-1.49899799599198	-4.0000000001747\\
-1.51503006012024	-4.00000000017052\\
-1.5310621242485	-4.0000000001664\\
-1.54709418837675	-4.00000000016234\\
-1.56312625250501	-4.00000000015834\\
-1.57915831663327	-4.0000000001544\\
-1.59519038076152	-4.00000000015052\\
-1.61122244488978	-4.00000000014669\\
-1.62725450901804	-4.00000000014293\\
-1.64328657314629	-4.00000000013922\\
-1.65931863727455	-4.00000000013558\\
-1.67535070140281	-4.000000000132\\
-1.69138276553106	-4.00000000012848\\
-1.70741482965932	-4.00000000012502\\
-1.72344689378758	-4.00000000012163\\
-1.73947895791583	-4.00000000011829\\
-1.75551102204409	-4.00000000011502\\
-1.77154308617234	-4.00000000011181\\
-1.7875751503006	-4.00000000010866\\
-1.80360721442886	-4.00000000010557\\
-1.81963927855711	-4.00000000010254\\
-1.83567134268537	-4.00000000009958\\
-1.85170340681363	-4.00000000009667\\
-1.86773547094188	-4.00000000009383\\
-1.88376753507014	-4.00000000009104\\
-1.8997995991984	-4.00000000008832\\
-1.91583166332665	-4.00000000008565\\
-1.93186372745491	-4.00000000008304\\
-1.94789579158317	-4.00000000008049\\
-1.96392785571142	-4.000000000078\\
-1.97995991983968	-4.00000000007557\\
-1.99599198396794	-4.00000000007319\\
-2.01202404809619	-4.00000000007087\\
-2.02805611222445	-4.00000000006861\\
-2.04408817635271	-4.0000000000664\\
-2.06012024048096	-4.00000000006425\\
-2.07615230460922	-4.00000000006215\\
-2.09218436873747	-4.0000000000601\\
-2.10821643286573	-4.0000000000581\\
-2.12424849699399	-4.00000000005616\\
-2.14028056112224	-4.00000000005426\\
-2.1563126252505	-4.00000000005242\\
-2.17234468937876	-4.00000000005063\\
-2.18837675350701	-4.00000000004888\\
-2.20440881763527	-4.00000000004718\\
-2.22044088176353	-4.00000000004553\\
-2.23647294589178	-4.00000000004393\\
-2.25250501002004	-4.00000000004237\\
-2.2685370741483	-4.00000000004086\\
-2.28456913827655	-4.00000000003938\\
-2.30060120240481	-4.00000000003796\\
-2.31663326653307	-4.00000000003657\\
-2.33266533066132	-4.00000000003523\\
-2.34869739478958	-4.00000000003392\\
-2.36472945891784	-4.00000000003266\\
-2.38076152304609	-4.00000000003143\\
-2.39679358717435	-4.00000000003025\\
-2.41282565130261	-4.00000000002909\\
-2.42885771543086	-4.00000000002798\\
-2.44488977955912	-4.0000000000269\\
-2.46092184368737	-4.00000000002586\\
-2.47695390781563	-4.00000000002485\\
-2.49298597194389	-4.00000000002387\\
-2.50901803607214	-4.00000000002292\\
-2.5250501002004	-4.00000000002201\\
-2.54108216432866	-4.00000000002113\\
-2.55711422845691	-4.00000000002027\\
-2.57314629258517	-4.00000000001945\\
-2.58917835671343	-4.00000000001865\\
-2.60521042084168	-4.00000000001789\\
-2.62124248496994	-4.00000000001714\\
-2.6372745490982	-4.00000000001643\\
-2.65330661322645	-4.00000000001574\\
-2.66933867735471	-4.00000000001507\\
-2.68537074148297	-4.00000000001443\\
-2.70140280561122	-4.00000000001382\\
-2.71743486973948	-4.00000000001322\\
-2.73346693386774	-4.00000000001265\\
-2.74949899799599	-4.0000000000121\\
-2.76553106212425	-4.00000000001156\\
-2.7815631262525	-4.00000000001105\\
-2.79759519038076	-4.00000000001056\\
-2.81362725450902	-4.00000000001009\\
-2.82965931863727	-4.00000000000964\\
-2.84569138276553	-4.0000000000092\\
-2.86172344689379	-4.00000000000878\\
-2.87775551102204	-4.00000000000838\\
-2.8937875751503	-4.00000000000799\\
-2.90981963927856	-4.00000000000762\\
-2.92585170340681	-4.00000000000726\\
-2.94188376753507	-4.00000000000692\\
-2.95791583166333	-4.00000000000659\\
-2.97394789579158	-4.00000000000628\\
-2.98997995991984	-4.00000000000598\\
-3.0060120240481	-4.00000000000569\\
-3.02204408817635	-4.00000000000541\\
-3.03807615230461	-4.00000000000515\\
-3.05410821643287	-4.00000000000489\\
-3.07014028056112	-4.00000000000465\\
-3.08617234468938	-4.00000000000442\\
-3.10220440881764	-4.00000000000419\\
-3.11823647294589	-4.00000000000398\\
-3.13426853707415	-4.00000000000378\\
-3.1503006012024	-4.00000000000358\\
-3.16633266533066	-4.0000000000034\\
-3.18236472945892	-4.00000000000322\\
-3.19839679358717	-4.00000000000305\\
-3.21442885771543	-4.00000000000289\\
-3.23046092184369	-4.00000000000273\\
-3.24649298597194	-4.00000000000259\\
-3.2625250501002	-4.00000000000245\\
-3.27855711422846	-4.00000000000231\\
-3.29458917835671	-4.00000000000218\\
-3.31062124248497	-4.00000000000206\\
-3.32665330661323	-4.00000000000195\\
-3.34268537074148	-4.00000000000183\\
-3.35871743486974	-4.00000000000173\\
-3.374749498998	-4.00000000000163\\
-3.39078156312625	-4.00000000000153\\
-3.40681362725451	-4.00000000000144\\
-3.42284569138277	-4.00000000000136\\
-3.43887775551102	-4.00000000000127\\
-3.45490981963928	-4.0000000000012\\
-3.47094188376753	-4.00000000000112\\
-3.48697394789579	-4.00000000000105\\
-3.50300601202405	-4.00000000000098\\
-3.5190380761523	-4.00000000000092\\
-3.53507014028056	-4.00000000000086\\
-3.55110220440882	-4.0000000000008\\
-3.56713426853707	-4.00000000000075\\
-3.58316633266533	-4.0000000000007\\
-3.59919839679359	-4.00000000000065\\
-3.61523046092184	-4.0000000000006\\
-3.6312625250501	-4.00000000000056\\
-3.64729458917836	-4.00000000000051\\
-3.66332665330661	-4.00000000000048\\
-3.67935871743487	-4.00000000000044\\
-3.69539078156313	-4.0000000000004\\
-3.71142284569138	-4.00000000000037\\
-3.72745490981964	-4.00000000000034\\
-3.7434869739479	-4.00000000000031\\
-3.75951903807615	-4.00000000000028\\
-3.77555110220441	-4.00000000000025\\
-3.79158316633267	-4.00000000000023\\
-3.80761523046092	-4.0000000000002\\
-3.82364729458918	-4.00000000000018\\
-3.83967935871743	-4.00000000000016\\
-3.85571142284569	-4.00000000000014\\
-3.87174348697395	-4.00000000000012\\
-3.8877755511022	-4.0000000000001\\
-3.90380761523046	-4.00000000000008\\
-3.91983967935872	-4.00000000000007\\
-3.93587174348697	-4.00000000000005\\
-3.95190380761523	-4.00000000000004\\
-3.96793587174349	-4.00000000000002\\
-3.98396793587174	-4.00000000000001\\
-4	-4\\
-4.00000000000001	-3.98396793587174\\
-4.00000000000002	-3.96793587174349\\
-4.00000000000004	-3.95190380761523\\
-4.00000000000005	-3.93587174348697\\
-4.00000000000007	-3.91983967935872\\
-4.00000000000008	-3.90380761523046\\
-4.0000000000001	-3.8877755511022\\
-4.00000000000012	-3.87174348697395\\
-4.00000000000014	-3.85571142284569\\
-4.00000000000016	-3.83967935871743\\
-4.00000000000018	-3.82364729458918\\
-4.0000000000002	-3.80761523046092\\
-4.00000000000023	-3.79158316633267\\
-4.00000000000025	-3.77555110220441\\
-4.00000000000028	-3.75951903807615\\
-4.00000000000031	-3.7434869739479\\
-4.00000000000034	-3.72745490981964\\
-4.00000000000037	-3.71142284569138\\
-4.0000000000004	-3.69539078156313\\
-4.00000000000044	-3.67935871743487\\
-4.00000000000048	-3.66332665330661\\
-4.00000000000051	-3.64729458917836\\
-4.00000000000056	-3.6312625250501\\
-4.0000000000006	-3.61523046092184\\
-4.00000000000065	-3.59919839679359\\
-4.0000000000007	-3.58316633266533\\
-4.00000000000075	-3.56713426853707\\
-4.0000000000008	-3.55110220440882\\
-4.00000000000086	-3.53507014028056\\
-4.00000000000092	-3.5190380761523\\
-4.00000000000098	-3.50300601202405\\
-4.00000000000105	-3.48697394789579\\
-4.00000000000112	-3.47094188376753\\
-4.0000000000012	-3.45490981963928\\
-4.00000000000127	-3.43887775551102\\
-4.00000000000136	-3.42284569138277\\
-4.00000000000144	-3.40681362725451\\
-4.00000000000153	-3.39078156312625\\
-4.00000000000163	-3.374749498998\\
-4.00000000000173	-3.35871743486974\\
-4.00000000000183	-3.34268537074148\\
-4.00000000000195	-3.32665330661323\\
-4.00000000000206	-3.31062124248497\\
-4.00000000000218	-3.29458917835671\\
-4.00000000000231	-3.27855711422846\\
-4.00000000000245	-3.2625250501002\\
-4.00000000000259	-3.24649298597194\\
-4.00000000000273	-3.23046092184369\\
-4.00000000000289	-3.21442885771543\\
-4.00000000000305	-3.19839679358717\\
-4.00000000000322	-3.18236472945892\\
-4.0000000000034	-3.16633266533066\\
-4.00000000000358	-3.1503006012024\\
-4.00000000000378	-3.13426853707415\\
-4.00000000000398	-3.11823647294589\\
-4.00000000000419	-3.10220440881764\\
-4.00000000000442	-3.08617234468938\\
-4.00000000000465	-3.07014028056112\\
-4.00000000000489	-3.05410821643287\\
-4.00000000000515	-3.03807615230461\\
-4.00000000000541	-3.02204408817635\\
-4.00000000000569	-3.0060120240481\\
-4.00000000000598	-2.98997995991984\\
-4.00000000000628	-2.97394789579158\\
-4.00000000000659	-2.95791583166333\\
-4.00000000000692	-2.94188376753507\\
-4.00000000000726	-2.92585170340681\\
-4.00000000000762	-2.90981963927856\\
-4.00000000000799	-2.8937875751503\\
-4.00000000000838	-2.87775551102204\\
-4.00000000000878	-2.86172344689379\\
-4.0000000000092	-2.84569138276553\\
-4.00000000000964	-2.82965931863727\\
-4.00000000001009	-2.81362725450902\\
-4.00000000001056	-2.79759519038076\\
-4.00000000001105	-2.7815631262525\\
-4.00000000001157	-2.76553106212425\\
-4.0000000000121	-2.74949899799599\\
-4.00000000001265	-2.73346693386774\\
-4.00000000001322	-2.71743486973948\\
-4.00000000001382	-2.70140280561122\\
-4.00000000001443	-2.68537074148297\\
-4.00000000001507	-2.66933867735471\\
-4.00000000001574	-2.65330661322645\\
-4.00000000001643	-2.6372745490982\\
-4.00000000001714	-2.62124248496994\\
-4.00000000001789	-2.60521042084168\\
-4.00000000001865	-2.58917835671343\\
-4.00000000001945	-2.57314629258517\\
-4.00000000002027	-2.55711422845691\\
-4.00000000002113	-2.54108216432866\\
-4.00000000002201	-2.5250501002004\\
-4.00000000002292	-2.50901803607214\\
-4.00000000002387	-2.49298597194389\\
-4.00000000002485	-2.47695390781563\\
-4.00000000002586	-2.46092184368737\\
-4.0000000000269	-2.44488977955912\\
-4.00000000002798	-2.42885771543086\\
-4.00000000002909	-2.41282565130261\\
-4.00000000003025	-2.39679358717435\\
-4.00000000003143	-2.38076152304609\\
-4.00000000003266	-2.36472945891784\\
-4.00000000003392	-2.34869739478958\\
-4.00000000003523	-2.33266533066132\\
-4.00000000003657	-2.31663326653307\\
-4.00000000003796	-2.30060120240481\\
-4.00000000003938	-2.28456913827655\\
-4.00000000004086	-2.2685370741483\\
-4.00000000004237	-2.25250501002004\\
-4.00000000004393	-2.23647294589178\\
-4.00000000004553	-2.22044088176353\\
-4.00000000004718	-2.20440881763527\\
-4.00000000004888	-2.18837675350701\\
-4.00000000005063	-2.17234468937876\\
-4.00000000005242	-2.1563126252505\\
-4.00000000005426	-2.14028056112224\\
-4.00000000005616	-2.12424849699399\\
-4.0000000000581	-2.10821643286573\\
-4.0000000000601	-2.09218436873747\\
-4.00000000006215	-2.07615230460922\\
-4.00000000006425	-2.06012024048096\\
-4.0000000000664	-2.04408817635271\\
-4.00000000006861	-2.02805611222445\\
-4.00000000007087	-2.01202404809619\\
-4.00000000007319	-1.99599198396794\\
-4.00000000007557	-1.97995991983968\\
-4.000000000078	-1.96392785571142\\
-4.00000000008049	-1.94789579158317\\
-4.00000000008304	-1.93186372745491\\
-4.00000000008565	-1.91583166332665\\
-4.00000000008832	-1.8997995991984\\
-4.00000000009104	-1.88376753507014\\
-4.00000000009383	-1.86773547094188\\
-4.00000000009667	-1.85170340681363\\
-4.00000000009958	-1.83567134268537\\
-4.00000000010254	-1.81963927855711\\
-4.00000000010557	-1.80360721442886\\
-4.00000000010866	-1.7875751503006\\
-4.00000000011181	-1.77154308617234\\
-4.00000000011502	-1.75551102204409\\
-4.00000000011829	-1.73947895791583\\
-4.00000000012163	-1.72344689378758\\
-4.00000000012502	-1.70741482965932\\
-4.00000000012848	-1.69138276553106\\
-4.000000000132	-1.67535070140281\\
-4.00000000013558	-1.65931863727455\\
-4.00000000013922	-1.64328657314629\\
-4.00000000014293	-1.62725450901804\\
-4.00000000014669	-1.61122244488978\\
-4.00000000015052	-1.59519038076152\\
-4.0000000001544	-1.57915831663327\\
-4.00000000015834	-1.56312625250501\\
-4.00000000016234	-1.54709418837675\\
-4.0000000001664	-1.5310621242485\\
-4.00000000017052	-1.51503006012024\\
-4.0000000001747	-1.49899799599198\\
-4.00000000017893	-1.48296593186373\\
-4.00000000018321	-1.46693386773547\\
-4.00000000018755	-1.45090180360721\\
-4.00000000019195	-1.43486973947896\\
-4.00000000019639	-1.4188376753507\\
-4.00000000020089	-1.40280561122244\\
-4.00000000020544	-1.38677354709419\\
-4.00000000021003	-1.37074148296593\\
-4.00000000021467	-1.35470941883768\\
-4.00000000021936	-1.33867735470942\\
-4.0000000002241	-1.32264529058116\\
-4.00000000022887	-1.30661322645291\\
-4.00000000023369	-1.29058116232465\\
-4.00000000023855	-1.27454909819639\\
-4.00000000024345	-1.25851703406814\\
-4.00000000024838	-1.24248496993988\\
-4.00000000025335	-1.22645290581162\\
-4.00000000025835	-1.21042084168337\\
-4.00000000026338	-1.19438877755511\\
-4.00000000026844	-1.17835671342685\\
-4.00000000027353	-1.1623246492986\\
-4.00000000027865	-1.14629258517034\\
-4.00000000028378	-1.13026052104208\\
-4.00000000028894	-1.11422845691383\\
-4.00000000029411	-1.09819639278557\\
-4.0000000002993	-1.08216432865731\\
-4.0000000003045	-1.06613226452906\\
-4.00000000030971	-1.0501002004008\\
-4.00000000031493	-1.03406813627255\\
-4.00000000032016	-1.01803607214429\\
-4.00000000032539	-1.00200400801603\\
-4.00000000033062	-0.985971943887776\\
-4.00000000033585	-0.969939879759519\\
-4.00000000034107	-0.953907815631263\\
-4.00000000034628	-0.937875751503006\\
-4.00000000035149	-0.92184368737475\\
-4.00000000035668	-0.905811623246493\\
-4.00000000036185	-0.889779559118236\\
-4.000000000367	-0.87374749498998\\
-4.00000000037214	-0.857715430861723\\
-4.00000000037724	-0.841683366733467\\
-4.00000000038232	-0.82565130260521\\
-4.00000000038737	-0.809619238476954\\
-4.00000000039238	-0.793587174348697\\
-4.00000000039736	-0.777555110220441\\
-4.00000000040229	-0.761523046092184\\
-4.00000000040718	-0.745490981963928\\
-4.00000000041203	-0.729458917835671\\
-4.00000000041682	-0.713426853707415\\
-4.00000000042156	-0.697394789579158\\
-4.00000000042625	-0.681362725450902\\
-4.00000000043088	-0.665330661322646\\
-4.00000000043545	-0.649298597194389\\
-4.00000000043995	-0.633266533066132\\
-4.00000000044438	-0.617234468937876\\
-4.00000000044875	-0.601202404809619\\
-4.00000000045303	-0.585170340681363\\
-4.00000000045725	-0.569138276553106\\
-4.00000000046138	-0.55310621242485\\
-4.00000000046543	-0.537074148296593\\
-4.0000000004694	-0.521042084168337\\
-4.00000000047328	-0.50501002004008\\
-4.00000000047706	-0.488977955911824\\
-4.00000000048076	-0.472945891783567\\
-4.00000000048436	-0.456913827655311\\
-4.00000000048786	-0.440881763527054\\
-4.00000000049126	-0.424849699398798\\
-4.00000000049455	-0.408817635270541\\
-4.00000000049774	-0.392785571142285\\
-4.00000000050082	-0.376753507014028\\
-4.00000000050379	-0.360721442885771\\
-4.00000000050665	-0.344689378757515\\
-4.00000000050939	-0.328657314629258\\
-4.00000000051202	-0.312625250501002\\
-4.00000000051453	-0.296593186372745\\
-4.00000000051691	-0.280561122244489\\
-4.00000000051918	-0.264529058116232\\
-4.00000000052132	-0.248496993987976\\
-4.00000000052333	-0.232464929859719\\
-4.00000000052522	-0.216432865731463\\
-4.00000000052698	-0.200400801603207\\
-4.00000000052861	-0.18436873747495\\
-4.0000000005301	-0.168336673346694\\
-4.00000000053147	-0.152304609218437\\
-4.0000000005327	-0.13627254509018\\
-4.0000000005338	-0.120240480961924\\
-4.00000000053476	-0.104208416833667\\
-4.00000000053558	-0.0881763527054109\\
-4.00000000053627	-0.0721442885771544\\
-4.00000000053682	-0.0561122244488979\\
-4.00000000053724	-0.0400801603206413\\
-4.00000000053751	-0.0240480961923848\\
-4.00000000053765	-0.00801603206412826\\
-4.00000000053765	0.00801603206412782\\
-4.00000000053751	0.0240480961923843\\
-4.00000000053724	0.0400801603206409\\
-4.00000000053682	0.0561122244488974\\
-4.00000000053627	0.0721442885771539\\
-4.00000000053558	0.0881763527054105\\
-4.00000000053476	0.104208416833667\\
-4.0000000005338	0.120240480961924\\
-4.0000000005327	0.13627254509018\\
-4.00000000053147	0.152304609218437\\
-4.0000000005301	0.168336673346693\\
-4.00000000052861	0.18436873747495\\
-4.00000000052698	0.200400801603206\\
-4.00000000052522	0.216432865731463\\
-4.00000000052333	0.232464929859719\\
-4.00000000052132	0.248496993987976\\
-4.00000000051918	0.264529058116232\\
-4.00000000051691	0.280561122244489\\
-4.00000000051453	0.296593186372745\\
-4.00000000051202	0.312625250501002\\
-4.00000000050939	0.328657314629258\\
-4.00000000050665	0.344689378757515\\
-4.00000000050379	0.360721442885771\\
-4.00000000050082	0.376753507014028\\
-4.00000000049774	0.392785571142285\\
-4.00000000049455	0.408817635270541\\
-4.00000000049126	0.424849699398798\\
-4.00000000048786	0.440881763527054\\
-4.00000000048436	0.456913827655311\\
-4.00000000048076	0.472945891783567\\
-4.00000000047706	0.488977955911824\\
-4.00000000047328	0.50501002004008\\
-4.0000000004694	0.521042084168337\\
-4.00000000046543	0.537074148296593\\
-4.00000000046138	0.55310621242485\\
-4.00000000045725	0.569138276553106\\
-4.00000000045303	0.585170340681363\\
-4.00000000044875	0.601202404809619\\
-4.00000000044438	0.617234468937876\\
-4.00000000043995	0.633266533066132\\
-4.00000000043545	0.649298597194389\\
-4.00000000043088	0.665330661322646\\
-4.00000000042625	0.681362725450902\\
-4.00000000042156	0.697394789579159\\
-4.00000000041682	0.713426853707415\\
-4.00000000041203	0.729458917835672\\
-4.00000000040718	0.745490981963928\\
-4.00000000040229	0.761523046092185\\
-4.00000000039736	0.777555110220441\\
-4.00000000039238	0.793587174348698\\
-4.00000000038737	0.809619238476954\\
-4.00000000038232	0.825651302605211\\
-4.00000000037724	0.841683366733467\\
-4.00000000037214	0.857715430861724\\
-4.000000000367	0.87374749498998\\
-4.00000000036185	0.889779559118236\\
-4.00000000035668	0.905811623246493\\
-4.00000000035149	0.921843687374749\\
-4.00000000034628	0.937875751503006\\
-4.00000000034107	0.953907815631262\\
-4.00000000033585	0.969939879759519\\
-4.00000000033062	0.985971943887775\\
-4.00000000032539	1.00200400801603\\
-4.00000000032016	1.01803607214429\\
-4.00000000031493	1.03406813627254\\
-4.00000000030971	1.0501002004008\\
-4.0000000003045	1.06613226452906\\
-4.0000000002993	1.08216432865731\\
-4.00000000029411	1.09819639278557\\
-4.00000000028894	1.11422845691383\\
-4.00000000028378	1.13026052104208\\
-4.00000000027865	1.14629258517034\\
-4.00000000027353	1.1623246492986\\
-4.00000000026844	1.17835671342685\\
-4.00000000026338	1.19438877755511\\
-4.00000000025835	1.21042084168337\\
-4.00000000025335	1.22645290581162\\
-4.00000000024838	1.24248496993988\\
-4.00000000024345	1.25851703406814\\
-4.00000000023855	1.27454909819639\\
-4.00000000023369	1.29058116232465\\
-4.00000000022887	1.30661322645291\\
-4.0000000002241	1.32264529058116\\
-4.00000000021936	1.33867735470942\\
-4.00000000021467	1.35470941883768\\
-4.00000000021003	1.37074148296593\\
-4.00000000020544	1.38677354709419\\
-4.00000000020089	1.40280561122244\\
-4.00000000019639	1.4188376753507\\
-4.00000000019195	1.43486973947896\\
-4.00000000018755	1.45090180360721\\
-4.00000000018321	1.46693386773547\\
-4.00000000017893	1.48296593186373\\
-4.0000000001747	1.49899799599198\\
-4.00000000017052	1.51503006012024\\
-4.0000000001664	1.5310621242485\\
-4.00000000016234	1.54709418837675\\
-4.00000000015834	1.56312625250501\\
-4.0000000001544	1.57915831663327\\
-4.00000000015052	1.59519038076152\\
-4.00000000014669	1.61122244488978\\
-4.00000000014293	1.62725450901804\\
-4.00000000013922	1.64328657314629\\
-4.00000000013558	1.65931863727455\\
-4.000000000132	1.67535070140281\\
-4.00000000012848	1.69138276553106\\
-4.00000000012502	1.70741482965932\\
-4.00000000012163	1.72344689378758\\
-4.00000000011829	1.73947895791583\\
-4.00000000011502	1.75551102204409\\
-4.00000000011181	1.77154308617235\\
-4.00000000010866	1.7875751503006\\
-4.00000000010557	1.80360721442886\\
-4.00000000010254	1.81963927855711\\
-4.00000000009958	1.83567134268537\\
-4.00000000009667	1.85170340681363\\
-4.00000000009383	1.86773547094188\\
-4.00000000009104	1.88376753507014\\
-4.00000000008832	1.8997995991984\\
-4.00000000008565	1.91583166332665\\
-4.00000000008304	1.93186372745491\\
-4.00000000008049	1.94789579158317\\
-4.000000000078	1.96392785571142\\
-4.00000000007557	1.97995991983968\\
-4.00000000007319	1.99599198396794\\
-4.00000000007087	2.01202404809619\\
-4.00000000006861	2.02805611222445\\
-4.0000000000664	2.04408817635271\\
-4.00000000006425	2.06012024048096\\
-4.00000000006215	2.07615230460922\\
-4.0000000000601	2.09218436873747\\
-4.0000000000581	2.10821643286573\\
-4.00000000005616	2.12424849699399\\
-4.00000000005426	2.14028056112224\\
-4.00000000005242	2.1563126252505\\
-4.00000000005063	2.17234468937876\\
-4.00000000004888	2.18837675350701\\
-4.00000000004718	2.20440881763527\\
-4.00000000004553	2.22044088176353\\
-4.00000000004393	2.23647294589178\\
-4.00000000004237	2.25250501002004\\
-4.00000000004086	2.2685370741483\\
-4.00000000003938	2.28456913827655\\
-4.00000000003796	2.30060120240481\\
-4.00000000003657	2.31663326653307\\
-4.00000000003523	2.33266533066132\\
-4.00000000003392	2.34869739478958\\
-4.00000000003266	2.36472945891784\\
-4.00000000003143	2.38076152304609\\
-4.00000000003025	2.39679358717435\\
-4.00000000002909	2.41282565130261\\
-4.00000000002798	2.42885771543086\\
-4.0000000000269	2.44488977955912\\
-4.00000000002586	2.46092184368737\\
-4.00000000002485	2.47695390781563\\
-4.00000000002387	2.49298597194389\\
-4.00000000002292	2.50901803607214\\
-4.00000000002201	2.5250501002004\\
-4.00000000002113	2.54108216432866\\
-4.00000000002027	2.55711422845691\\
-4.00000000001945	2.57314629258517\\
-4.00000000001865	2.58917835671343\\
-4.00000000001789	2.60521042084168\\
-4.00000000001714	2.62124248496994\\
-4.00000000001643	2.6372745490982\\
-4.00000000001574	2.65330661322645\\
-4.00000000001507	2.66933867735471\\
-4.00000000001443	2.68537074148297\\
-4.00000000001382	2.70140280561122\\
-4.00000000001322	2.71743486973948\\
-4.00000000001265	2.73346693386774\\
-4.0000000000121	2.74949899799599\\
-4.00000000001157	2.76553106212425\\
-4.00000000001105	2.7815631262525\\
-4.00000000001056	2.79759519038076\\
-4.00000000001009	2.81362725450902\\
-4.00000000000964	2.82965931863727\\
-4.0000000000092	2.84569138276553\\
-4.00000000000878	2.86172344689379\\
-4.00000000000838	2.87775551102204\\
-4.00000000000799	2.8937875751503\\
-4.00000000000762	2.90981963927856\\
-4.00000000000726	2.92585170340681\\
-4.00000000000692	2.94188376753507\\
-4.00000000000659	2.95791583166333\\
-4.00000000000628	2.97394789579158\\
-4.00000000000598	2.98997995991984\\
-4.00000000000569	3.0060120240481\\
-4.00000000000541	3.02204408817635\\
-4.00000000000515	3.03807615230461\\
-4.00000000000489	3.05410821643287\\
-4.00000000000465	3.07014028056112\\
-4.00000000000442	3.08617234468938\\
-4.00000000000419	3.10220440881764\\
-4.00000000000398	3.11823647294589\\
-4.00000000000378	3.13426853707415\\
-4.00000000000358	3.1503006012024\\
-4.0000000000034	3.16633266533066\\
-4.00000000000322	3.18236472945892\\
-4.00000000000305	3.19839679358717\\
-4.00000000000289	3.21442885771543\\
-4.00000000000273	3.23046092184369\\
-4.00000000000259	3.24649298597194\\
-4.00000000000245	3.2625250501002\\
-4.00000000000231	3.27855711422846\\
-4.00000000000218	3.29458917835671\\
-4.00000000000206	3.31062124248497\\
-4.00000000000195	3.32665330661323\\
-4.00000000000183	3.34268537074148\\
-4.00000000000173	3.35871743486974\\
-4.00000000000163	3.374749498998\\
-4.00000000000153	3.39078156312625\\
-4.00000000000144	3.40681362725451\\
-4.00000000000136	3.42284569138277\\
-4.00000000000127	3.43887775551102\\
-4.0000000000012	3.45490981963928\\
-4.00000000000112	3.47094188376754\\
-4.00000000000105	3.48697394789579\\
-4.00000000000098	3.50300601202405\\
-4.00000000000092	3.51903807615231\\
-4.00000000000086	3.53507014028056\\
-4.0000000000008	3.55110220440882\\
-4.00000000000075	3.56713426853707\\
-4.0000000000007	3.58316633266533\\
-4.00000000000065	3.59919839679359\\
-4.0000000000006	3.61523046092184\\
-4.00000000000056	3.6312625250501\\
-4.00000000000051	3.64729458917836\\
-4.00000000000048	3.66332665330661\\
-4.00000000000044	3.67935871743487\\
-4.0000000000004	3.69539078156313\\
-4.00000000000037	3.71142284569138\\
-4.00000000000034	3.72745490981964\\
-4.00000000000031	3.7434869739479\\
-4.00000000000028	3.75951903807615\\
-4.00000000000025	3.77555110220441\\
-4.00000000000023	3.79158316633267\\
-4.0000000000002	3.80761523046092\\
-4.00000000000018	3.82364729458918\\
-4.00000000000016	3.83967935871743\\
-4.00000000000014	3.85571142284569\\
-4.00000000000012	3.87174348697395\\
-4.0000000000001	3.8877755511022\\
-4.00000000000008	3.90380761523046\\
-4.00000000000007	3.91983967935872\\
-4.00000000000005	3.93587174348697\\
-4.00000000000004	3.95190380761523\\
-4.00000000000002	3.96793587174349\\
-4.00000000000001	3.98396793587174\\
-4	4\\
}--cycle;


\addplot[area legend,solid,fill=mycolor2,draw=black,forget plot]
table[row sep=crcr] {%
x	y\\
-2.02805611222445	0.187721458164462\\
-2.02686616963263	0.200400801603206\\
-2.02523200293539	0.216432865731463\\
-2.02346601385099	0.232464929859719\\
-2.02156681761085	0.248496993987976\\
-2.01953292135492	0.264529058116232\\
-2.01736272217602	0.280561122244489\\
-2.01505450501464	0.296593186372745\\
-2.01260644040032	0.312625250501002\\
-2.01202404809619	0.316244833411446\\
-2.01006475210128	0.328657314629258\\
-2.00739663109156	0.344689378757515\\
-2.00458596724657	0.360721442885771\\
-2.00163050314481	0.376753507014028\\
-1.99852785627817	0.392785571142285\\
-1.99599198396794	0.405303545146575\\
-1.99529248331274	0.408817635270541\\
-1.99196843670764	0.424849699398798\\
-1.98849294979072	0.440881763527054\\
-1.98486317491207	0.456913827655311\\
-1.98107612481341	0.472945891783567\\
-1.97995991983968	0.477504321815857\\
-1.97719482809212	0.488977955911824\\
-1.9731797541918	0.50501002004008\\
-1.96900146876237	0.521042084168337\\
-1.96465646686352	0.537074148296593\\
-1.96392785571142	0.53968013010859\\
-1.9602283776053	0.55310621242485\\
-1.95564689307314	0.569138276553106\\
-1.95089131332239	0.585170340681363\\
-1.94789579158317	0.594940522759552\\
-1.94600162965943	0.601202404809619\\
-1.94100177610258	0.617234468937876\\
-1.93581931601292	0.633266533066132\\
-1.93186372745491	0.645108701946643\\
-1.93048142973643	0.649298597194389\\
-1.92504468441812	0.665330661322646\\
-1.91941561015907	0.681362725450902\\
-1.91583166332665	0.691266295395884\\
-1.91363876352957	0.697394789579159\\
-1.90774305930578	0.713426853707415\\
-1.9016439984945	0.729458917835672\\
-1.8997995991984	0.734185902005566\\
-1.89543326643157	0.745490981963928\\
-1.8890527520654	0.761523046092185\\
-1.88376753507014	0.774399924454319\\
-1.88248456766811	0.777555110220441\\
-1.87581265086804	0.793587174348698\\
-1.86891739916824	0.809619238476954\\
-1.86773547094188	0.812307901692387\\
-1.86191795049415	0.825651302605211\\
-1.85471196656349	0.841683366733467\\
-1.85170340681363	0.848215233038847\\
-1.84735992669908	0.857715430861724\\
-1.83983011800848	0.87374749498998\\
-1.83567134268537	0.882376565930459\\
-1.83212679912755	0.889779559118236\\
-1.82425870728494	0.905811623246493\\
-1.81963927855711	0.914983296901567\\
-1.81620363298823	0.921843687374749\\
-1.80798135507683	0.937875751503006\\
-1.80360721442886	0.946195618651337\\
-1.79957221763483	0.953907815631262\\
-1.79097831714864	0.969939879759519\\
-1.7875751503006	0.976147648477744\\
-1.78221092077393	0.985971943887775\\
-1.77322632735206	1.00200400801603\\
-1.77154308617234	1.00495154830762\\
-1.76409451676456	1.01803607214429\\
-1.75551102204409	1.03270232062295\\
-1.75471371668781	1.03406813627254\\
-1.74519398719865	1.0501002004008\\
-1.73947895791583	1.0594868384357\\
-1.73543980642559	1.06613226452906\\
-1.72547629165081	1.08216432865731\\
-1.72344689378758	1.08537102582778\\
-1.71533709498293	1.09819639278557\\
-1.70741482965932	1.11042202092204\\
-1.70494885309161	1.11422845691383\\
-1.69436592406958	1.13026052104208\\
-1.69138276553106	1.13469433852366\\
-1.68357446350079	1.14629258517034\\
-1.67535070140281	1.1582350096519\\
-1.67253107570689	1.1623246492986\\
-1.66126930815948	1.17835671342685\\
-1.65931863727455	1.18108768779127\\
-1.64979819740775	1.19438877755511\\
-1.64328657314629	1.20329940930239\\
-1.63806843238578	1.21042084168337\\
-1.62725450901804	1.22488603589304\\
-1.62607927445993	1.22645290581162\\
-1.61387110213118	1.24248496993988\\
-1.61122244488978	1.24590455775322\\
-1.6014125940218	1.25851703406814\\
-1.59519038076152	1.26637210710088\\
-1.58868212050701	1.27454909819639\\
-1.57915831663327	1.28630617282375\\
-1.57567662339068	1.29058116232465\\
-1.56312625250501	1.30573255097363\\
-1.56239235944496	1.30661322645291\\
-1.54885004938148	1.32264529058116\\
-1.54709418837675	1.3246936464482\\
-1.53502436633223	1.33867735470942\\
-1.5310621242485	1.34319868030208\\
-1.52089864098993	1.35470941883768\\
-1.51503006012024	1.36126002965037\\
-1.5064663542454	1.37074148296593\\
-1.49899799599198	1.37889621631559\\
-1.4917201577588	1.38677354709419\\
-1.48296593186373	1.39612432753368\\
-1.47665183552235	1.40280561122244\\
-1.46693386773547	1.41296010365205\\
-1.46125226145058	1.4188376753507\\
-1.45090180360721	1.42941801845526\\
-1.4455113527083	1.43486973947896\\
-1.43486973947896	1.4455113527083\\
-1.42941801845526	1.45090180360721\\
-1.4188376753507	1.46125226145058\\
-1.41296010365205	1.46693386773547\\
-1.40280561122244	1.47665183552235\\
-1.39612432753368	1.48296593186373\\
-1.38677354709419	1.4917201577588\\
-1.37889621631559	1.49899799599198\\
-1.37074148296593	1.5064663542454\\
-1.36126002965037	1.51503006012024\\
-1.35470941883768	1.52089864098993\\
-1.34319868030208	1.5310621242485\\
-1.33867735470942	1.53502436633223\\
-1.3246936464482	1.54709418837675\\
-1.32264529058116	1.54885004938148\\
-1.30661322645291	1.56239235944496\\
-1.30573255097363	1.56312625250501\\
-1.29058116232465	1.57567662339068\\
-1.28630617282375	1.57915831663327\\
-1.27454909819639	1.58868212050701\\
-1.26637210710088	1.59519038076152\\
-1.25851703406814	1.6014125940218\\
-1.24590455775322	1.61122244488978\\
-1.24248496993988	1.61387110213118\\
-1.22645290581162	1.62607927445993\\
-1.22488603589304	1.62725450901804\\
-1.21042084168337	1.63806843238578\\
-1.20329940930239	1.64328657314629\\
-1.19438877755511	1.64979819740775\\
-1.18108768779127	1.65931863727455\\
-1.17835671342685	1.66126930815948\\
-1.1623246492986	1.67253107570689\\
-1.1582350096519	1.67535070140281\\
-1.14629258517034	1.6835744635008\\
-1.13469433852365	1.69138276553106\\
-1.13026052104208	1.69436592406958\\
-1.11422845691383	1.70494885309161\\
-1.11042202092204	1.70741482965932\\
-1.09819639278557	1.71533709498293\\
-1.08537102582778	1.72344689378758\\
-1.08216432865731	1.72547629165081\\
-1.06613226452906	1.73543980642559\\
-1.0594868384357	1.73947895791583\\
-1.0501002004008	1.74519398719865\\
-1.03406813627255	1.75471371668781\\
-1.03270232062295	1.75551102204409\\
-1.01803607214429	1.76409451676456\\
-1.00495154830762	1.77154308617235\\
-1.00200400801603	1.77322632735206\\
-0.985971943887776	1.78221092077393\\
-0.976147648477745	1.7875751503006\\
-0.969939879759519	1.79097831714863\\
-0.953907815631263	1.79957221763483\\
-0.946195618651339	1.80360721442886\\
-0.937875751503006	1.80798135507683\\
-0.92184368737475	1.81620363298823\\
-0.914983296901567	1.81963927855711\\
-0.905811623246493	1.82425870728494\\
-0.889779559118236	1.83212679912755\\
-0.882376565930461	1.83567134268537\\
-0.87374749498998	1.83983011800848\\
-0.857715430861723	1.84735992669908\\
-0.848215233038848	1.85170340681363\\
-0.841683366733467	1.85471196656349\\
-0.82565130260521	1.86191795049415\\
-0.812307901692389	1.86773547094188\\
-0.809619238476954	1.86891739916824\\
-0.793587174348697	1.87581265086804\\
-0.777555110220441	1.88248456766812\\
-0.774399924454321	1.88376753507014\\
-0.761523046092184	1.8890527520654\\
-0.745490981963928	1.89543326643157\\
-0.734185902005567	1.8997995991984\\
-0.729458917835671	1.9016439984945\\
-0.713426853707415	1.90774305930578\\
-0.697394789579158	1.91363876352957\\
-0.691266295395886	1.91583166332665\\
-0.681362725450902	1.91941561015907\\
-0.665330661322646	1.92504468441812\\
-0.649298597194389	1.93048142973643\\
-0.645108701946644	1.93186372745491\\
-0.633266533066132	1.93581931601292\\
-0.617234468937876	1.94100177610258\\
-0.601202404809619	1.94600162965943\\
-0.594940522759554	1.94789579158317\\
-0.585170340681363	1.95089131332239\\
-0.569138276553106	1.95564689307314\\
-0.55310621242485	1.9602283776053\\
-0.539680130108592	1.96392785571142\\
-0.537074148296593	1.96465646686352\\
-0.521042084168337	1.96900146876237\\
-0.50501002004008	1.9731797541918\\
-0.488977955911824	1.97719482809212\\
-0.477504321815859	1.97995991983968\\
-0.472945891783567	1.98107612481341\\
-0.456913827655311	1.98486317491207\\
-0.440881763527054	1.98849294979072\\
-0.424849699398798	1.99196843670764\\
-0.408817635270541	1.99529248331274\\
-0.405303545146577	1.99599198396794\\
-0.392785571142285	1.99852785627817\\
-0.376753507014028	2.00163050314481\\
-0.360721442885771	2.00458596724657\\
-0.344689378757515	2.00739663109156\\
-0.328657314629258	2.01006475210128\\
-0.316244833411446	2.01202404809619\\
-0.312625250501002	2.01260644040032\\
-0.296593186372745	2.01505450501464\\
-0.280561122244489	2.01736272217602\\
-0.264529058116232	2.01953292135492\\
-0.248496993987976	2.02156681761085\\
-0.232464929859719	2.02346601385099\\
-0.216432865731463	2.02523200293539\\
-0.200400801603207	2.02686616963263\\
-0.187721458164461	2.02805611222445\\
-0.18436873747495	2.02837760724697\\
-0.168336673346694	2.02978609722387\\
-0.152304609218437	2.03106309159634\\
-0.13627254509018	2.03220958359571\\
-0.120240480961924	2.03322646351064\\
-0.104208416833667	2.03411451983944\\
-0.0881763527054109	2.03487444030915\\
-0.0721442885771544	2.03550681276325\\
-0.0561122244488979	2.03601212591963\\
-0.0400801603206413	2.03639077000011\\
-0.0240480961923848	2.03664303723276\\
-0.00801603206412826	2.03676912222778\\
0.00801603206412782	2.03676912222778\\
0.0240480961923843	2.03664303723276\\
0.0400801603206409	2.03639077000011\\
0.0561122244488974	2.03601212591963\\
0.0721442885771539	2.03550681276325\\
0.0881763527054105	2.03487444030915\\
0.104208416833667	2.03411451983944\\
0.120240480961924	2.03322646351064\\
0.13627254509018	2.03220958359571\\
0.152304609218437	2.03106309159634\\
0.168336673346693	2.02978609722387\\
0.18436873747495	2.02837760724697\\
0.187721458164462	2.02805611222445\\
0.200400801603206	2.02686616963263\\
0.216432865731463	2.02523200293539\\
0.232464929859719	2.02346601385099\\
0.248496993987976	2.02156681761085\\
0.264529058116232	2.01953292135492\\
0.280561122244489	2.01736272217602\\
0.296593186372745	2.01505450501464\\
0.312625250501002	2.01260644040032\\
0.316244833411446	2.01202404809619\\
0.328657314629258	2.01006475210128\\
0.344689378757515	2.00739663109156\\
0.360721442885771	2.00458596724657\\
0.376753507014028	2.00163050314481\\
0.392785571142285	1.99852785627817\\
0.405303545146577	1.99599198396794\\
0.408817635270541	1.99529248331274\\
0.424849699398798	1.99196843670764\\
0.440881763527054	1.98849294979072\\
0.456913827655311	1.98486317491207\\
0.472945891783567	1.98107612481341\\
0.477504321815859	1.97995991983968\\
0.488977955911824	1.97719482809212\\
0.50501002004008	1.9731797541918\\
0.521042084168337	1.96900146876237\\
0.537074148296593	1.96465646686352\\
0.539680130108592	1.96392785571142\\
0.55310621242485	1.9602283776053\\
0.569138276553106	1.95564689307314\\
0.585170340681363	1.95089131332239\\
0.594940522759554	1.94789579158317\\
0.601202404809619	1.94600162965943\\
0.617234468937876	1.94100177610258\\
0.633266533066132	1.93581931601292\\
0.645108701946645	1.93186372745491\\
0.649298597194389	1.93048142973643\\
0.665330661322646	1.92504468441812\\
0.681362725450902	1.91941561015907\\
0.691266295395886	1.91583166332665\\
0.697394789579159	1.91363876352957\\
0.713426853707415	1.90774305930578\\
0.729458917835672	1.9016439984945\\
0.734185902005567	1.8997995991984\\
0.745490981963928	1.89543326643157\\
0.761523046092185	1.8890527520654\\
0.77439992445432	1.88376753507014\\
0.777555110220441	1.88248456766811\\
0.793587174348698	1.87581265086804\\
0.809619238476954	1.86891739916824\\
0.812307901692388	1.86773547094188\\
0.825651302605211	1.86191795049415\\
0.841683366733467	1.85471196656349\\
0.848215233038848	1.85170340681363\\
0.857715430861724	1.84735992669908\\
0.87374749498998	1.83983011800848\\
0.88237656593046	1.83567134268537\\
0.889779559118236	1.83212679912755\\
0.905811623246493	1.82425870728494\\
0.914983296901568	1.81963927855711\\
0.921843687374749	1.81620363298823\\
0.937875751503006	1.80798135507683\\
0.946195618651338	1.80360721442886\\
0.953907815631262	1.79957221763483\\
0.969939879759519	1.79097831714864\\
0.976147648477745	1.7875751503006\\
0.985971943887775	1.78221092077393\\
1.00200400801603	1.77322632735206\\
1.00495154830761	1.77154308617235\\
1.01803607214429	1.76409451676456\\
1.03270232062295	1.75551102204409\\
1.03406813627254	1.75471371668781\\
1.0501002004008	1.74519398719865\\
1.0594868384357	1.73947895791583\\
1.06613226452906	1.73543980642559\\
1.08216432865731	1.72547629165081\\
1.08537102582777	1.72344689378758\\
1.09819639278557	1.71533709498293\\
1.11042202092204	1.70741482965932\\
1.11422845691383	1.70494885309161\\
1.13026052104208	1.69436592406958\\
1.13469433852365	1.69138276553106\\
1.14629258517034	1.6835744635008\\
1.1582350096519	1.67535070140281\\
1.1623246492986	1.67253107570689\\
1.17835671342685	1.66126930815948\\
1.18108768779127	1.65931863727455\\
1.19438877755511	1.64979819740775\\
1.20329940930239	1.64328657314629\\
1.21042084168337	1.63806843238578\\
1.22488603589304	1.62725450901804\\
1.22645290581162	1.62607927445993\\
1.24248496993988	1.61387110213118\\
1.24590455775322	1.61122244488978\\
1.25851703406814	1.6014125940218\\
1.26637210710088	1.59519038076152\\
1.27454909819639	1.58868212050701\\
1.28630617282375	1.57915831663327\\
1.29058116232465	1.57567662339068\\
1.30573255097363	1.56312625250501\\
1.30661322645291	1.56239235944496\\
1.32264529058116	1.54885004938148\\
1.3246936464482	1.54709418837675\\
1.33867735470942	1.53502436633223\\
1.34319868030208	1.5310621242485\\
1.35470941883768	1.52089864098993\\
1.36126002965037	1.51503006012024\\
1.37074148296593	1.5064663542454\\
1.37889621631559	1.49899799599198\\
1.38677354709419	1.4917201577588\\
1.39612432753368	1.48296593186373\\
1.40280561122244	1.47665183552235\\
1.41296010365205	1.46693386773547\\
1.4188376753507	1.46125226145058\\
1.42941801845526	1.45090180360721\\
1.43486973947896	1.4455113527083\\
1.4455113527083	1.43486973947896\\
1.45090180360721	1.42941801845526\\
1.46125226145058	1.4188376753507\\
1.46693386773547	1.41296010365205\\
1.47665183552235	1.40280561122244\\
1.48296593186373	1.39612432753368\\
1.4917201577588	1.38677354709419\\
1.49899799599198	1.37889621631559\\
1.5064663542454	1.37074148296593\\
1.51503006012024	1.36126002965037\\
1.52089864098993	1.35470941883768\\
1.5310621242485	1.34319868030208\\
1.53502436633223	1.33867735470942\\
1.54709418837675	1.3246936464482\\
1.54885004938148	1.32264529058116\\
1.56239235944496	1.30661322645291\\
1.56312625250501	1.30573255097363\\
1.57567662339068	1.29058116232465\\
1.57915831663327	1.28630617282375\\
1.58868212050701	1.27454909819639\\
1.59519038076152	1.26637210710088\\
1.6014125940218	1.25851703406814\\
1.61122244488978	1.24590455775322\\
1.61387110213118	1.24248496993988\\
1.62607927445993	1.22645290581162\\
1.62725450901804	1.22488603589304\\
1.63806843238578	1.21042084168337\\
1.64328657314629	1.20329940930239\\
1.64979819740775	1.19438877755511\\
1.65931863727455	1.18108768779127\\
1.66126930815948	1.17835671342685\\
1.67253107570689	1.1623246492986\\
1.67535070140281	1.1582350096519\\
1.6835744635008	1.14629258517034\\
1.69138276553106	1.13469433852365\\
1.69436592406958	1.13026052104208\\
1.70494885309161	1.11422845691383\\
1.70741482965932	1.11042202092204\\
1.71533709498293	1.09819639278557\\
1.72344689378758	1.08537102582777\\
1.72547629165081	1.08216432865731\\
1.73543980642559	1.06613226452906\\
1.73947895791583	1.0594868384357\\
1.74519398719865	1.0501002004008\\
1.75471371668781	1.03406813627254\\
1.75551102204409	1.03270232062295\\
1.76409451676456	1.01803607214429\\
1.77154308617235	1.00495154830761\\
1.77322632735206	1.00200400801603\\
1.78221092077393	0.985971943887775\\
1.7875751503006	0.976147648477745\\
1.79097831714864	0.969939879759519\\
1.79957221763483	0.953907815631262\\
1.80360721442886	0.946195618651338\\
1.80798135507683	0.937875751503006\\
1.81620363298823	0.921843687374749\\
1.81963927855711	0.914983296901568\\
1.82425870728494	0.905811623246493\\
1.83212679912755	0.889779559118236\\
1.83567134268537	0.88237656593046\\
1.83983011800848	0.87374749498998\\
1.84735992669908	0.857715430861724\\
1.85170340681363	0.848215233038848\\
1.85471196656349	0.841683366733467\\
1.86191795049415	0.825651302605211\\
1.86773547094188	0.812307901692388\\
1.86891739916824	0.809619238476954\\
1.87581265086804	0.793587174348698\\
1.88248456766811	0.777555110220441\\
1.88376753507014	0.77439992445432\\
1.8890527520654	0.761523046092185\\
1.89543326643157	0.745490981963928\\
1.8997995991984	0.734185902005567\\
1.9016439984945	0.729458917835672\\
1.90774305930578	0.713426853707415\\
1.91363876352957	0.697394789579159\\
1.91583166332665	0.691266295395886\\
1.91941561015907	0.681362725450902\\
1.92504468441812	0.665330661322646\\
1.93048142973643	0.649298597194389\\
1.93186372745491	0.645108701946645\\
1.93581931601292	0.633266533066132\\
1.94100177610258	0.617234468937876\\
1.94600162965943	0.601202404809619\\
1.94789579158317	0.594940522759554\\
1.95089131332239	0.585170340681363\\
1.95564689307314	0.569138276553106\\
1.9602283776053	0.55310621242485\\
1.96392785571142	0.539680130108592\\
1.96465646686352	0.537074148296593\\
1.96900146876237	0.521042084168337\\
1.9731797541918	0.50501002004008\\
1.97719482809212	0.488977955911824\\
1.97995991983968	0.477504321815859\\
1.98107612481341	0.472945891783567\\
1.98486317491207	0.456913827655311\\
1.98849294979072	0.440881763527054\\
1.99196843670764	0.424849699398798\\
1.99529248331274	0.408817635270541\\
1.99599198396794	0.405303545146577\\
1.99852785627817	0.392785571142285\\
2.00163050314481	0.376753507014028\\
2.00458596724657	0.360721442885771\\
2.00739663109156	0.344689378757515\\
2.01006475210128	0.328657314629258\\
2.01202404809619	0.316244833411446\\
2.01260644040032	0.312625250501002\\
2.01505450501464	0.296593186372745\\
2.01736272217602	0.280561122244489\\
2.01953292135492	0.264529058116232\\
2.02156681761085	0.248496993987976\\
2.02346601385099	0.232464929859719\\
2.02523200293539	0.216432865731463\\
2.02686616963263	0.200400801603206\\
2.02805611222445	0.187721458164462\\
2.02837760724697	0.18436873747495\\
2.02978609722387	0.168336673346693\\
2.03106309159634	0.152304609218437\\
2.03220958359571	0.13627254509018\\
2.03322646351064	0.120240480961924\\
2.03411451983944	0.104208416833667\\
2.03487444030915	0.0881763527054105\\
2.03550681276325	0.0721442885771539\\
2.03601212591963	0.0561122244488974\\
2.03639077000011	0.0400801603206409\\
2.03664303723276	0.0240480961923843\\
2.03676912222778	0.00801603206412782\\
2.03676912222778	-0.00801603206412826\\
2.03664303723276	-0.0240480961923848\\
2.03639077000011	-0.0400801603206413\\
2.03601212591963	-0.0561122244488979\\
2.03550681276325	-0.0721442885771544\\
2.03487444030915	-0.0881763527054109\\
2.03411451983944	-0.104208416833667\\
2.03322646351064	-0.120240480961924\\
2.03220958359571	-0.13627254509018\\
2.03106309159634	-0.152304609218437\\
2.02978609722387	-0.168336673346694\\
2.02837760724697	-0.18436873747495\\
2.02805611222445	-0.187721458164461\\
2.02686616963263	-0.200400801603207\\
2.02523200293539	-0.216432865731463\\
2.02346601385099	-0.232464929859719\\
2.02156681761085	-0.248496993987976\\
2.01953292135492	-0.264529058116232\\
2.01736272217602	-0.280561122244489\\
2.01505450501464	-0.296593186372745\\
2.01260644040032	-0.312625250501002\\
2.01202404809619	-0.316244833411446\\
2.01006475210128	-0.328657314629258\\
2.00739663109156	-0.344689378757515\\
2.00458596724657	-0.360721442885771\\
2.00163050314481	-0.376753507014028\\
1.99852785627817	-0.392785571142285\\
1.99599198396794	-0.405303545146577\\
1.99529248331274	-0.408817635270541\\
1.99196843670764	-0.424849699398798\\
1.98849294979072	-0.440881763527054\\
1.98486317491207	-0.456913827655311\\
1.98107612481341	-0.472945891783567\\
1.97995991983968	-0.477504321815859\\
1.97719482809212	-0.488977955911824\\
1.9731797541918	-0.50501002004008\\
1.96900146876237	-0.521042084168337\\
1.96465646686352	-0.537074148296593\\
1.96392785571142	-0.539680130108592\\
1.9602283776053	-0.55310621242485\\
1.95564689307314	-0.569138276553106\\
1.95089131332239	-0.585170340681363\\
1.94789579158317	-0.594940522759554\\
1.94600162965943	-0.601202404809619\\
1.94100177610258	-0.617234468937876\\
1.93581931601292	-0.633266533066132\\
1.93186372745491	-0.645108701946644\\
1.93048142973643	-0.649298597194389\\
1.92504468441812	-0.665330661322646\\
1.91941561015907	-0.681362725450902\\
1.91583166332665	-0.691266295395886\\
1.91363876352957	-0.697394789579158\\
1.90774305930578	-0.713426853707415\\
1.9016439984945	-0.729458917835671\\
1.8997995991984	-0.734185902005567\\
1.89543326643157	-0.745490981963928\\
1.8890527520654	-0.761523046092184\\
1.88376753507014	-0.774399924454321\\
1.88248456766812	-0.777555110220441\\
1.87581265086804	-0.793587174348697\\
1.86891739916824	-0.809619238476954\\
1.86773547094188	-0.812307901692389\\
1.86191795049415	-0.82565130260521\\
1.85471196656349	-0.841683366733467\\
1.85170340681363	-0.848215233038848\\
1.84735992669908	-0.857715430861723\\
1.83983011800848	-0.87374749498998\\
1.83567134268537	-0.882376565930461\\
1.83212679912755	-0.889779559118236\\
1.82425870728494	-0.905811623246493\\
1.81963927855711	-0.914983296901567\\
1.81620363298823	-0.92184368737475\\
1.80798135507683	-0.937875751503006\\
1.80360721442886	-0.946195618651339\\
1.79957221763483	-0.953907815631263\\
1.79097831714863	-0.969939879759519\\
1.7875751503006	-0.976147648477745\\
1.78221092077393	-0.985971943887776\\
1.77322632735206	-1.00200400801603\\
1.77154308617235	-1.00495154830762\\
1.76409451676456	-1.01803607214429\\
1.75551102204409	-1.03270232062295\\
1.75471371668781	-1.03406813627255\\
1.74519398719865	-1.0501002004008\\
1.73947895791583	-1.0594868384357\\
1.73543980642559	-1.06613226452906\\
1.72547629165081	-1.08216432865731\\
1.72344689378758	-1.08537102582778\\
1.71533709498293	-1.09819639278557\\
1.70741482965932	-1.11042202092204\\
1.70494885309161	-1.11422845691383\\
1.69436592406958	-1.13026052104208\\
1.69138276553106	-1.13469433852365\\
1.6835744635008	-1.14629258517034\\
1.67535070140281	-1.1582350096519\\
1.67253107570689	-1.1623246492986\\
1.66126930815948	-1.17835671342685\\
1.65931863727455	-1.18108768779127\\
1.64979819740775	-1.19438877755511\\
1.64328657314629	-1.20329940930239\\
1.63806843238578	-1.21042084168337\\
1.62725450901804	-1.22488603589304\\
1.62607927445993	-1.22645290581162\\
1.61387110213118	-1.24248496993988\\
1.61122244488978	-1.24590455775322\\
1.6014125940218	-1.25851703406814\\
1.59519038076152	-1.26637210710088\\
1.58868212050701	-1.27454909819639\\
1.57915831663327	-1.28630617282375\\
1.57567662339068	-1.29058116232465\\
1.56312625250501	-1.30573255097363\\
1.56239235944496	-1.30661322645291\\
1.54885004938148	-1.32264529058116\\
1.54709418837675	-1.3246936464482\\
1.53502436633223	-1.33867735470942\\
1.5310621242485	-1.34319868030208\\
1.52089864098993	-1.35470941883768\\
1.51503006012024	-1.36126002965037\\
1.5064663542454	-1.37074148296593\\
1.49899799599198	-1.37889621631559\\
1.4917201577588	-1.38677354709419\\
1.48296593186373	-1.39612432753368\\
1.47665183552235	-1.40280561122244\\
1.46693386773547	-1.41296010365205\\
1.46125226145058	-1.4188376753507\\
1.45090180360721	-1.42941801845526\\
1.4455113527083	-1.43486973947896\\
1.43486973947896	-1.4455113527083\\
1.42941801845526	-1.45090180360721\\
1.4188376753507	-1.46125226145058\\
1.41296010365205	-1.46693386773547\\
1.40280561122244	-1.47665183552235\\
1.39612432753368	-1.48296593186373\\
1.38677354709419	-1.4917201577588\\
1.37889621631559	-1.49899799599198\\
1.37074148296593	-1.5064663542454\\
1.36126002965037	-1.51503006012024\\
1.35470941883768	-1.52089864098993\\
1.34319868030208	-1.5310621242485\\
1.33867735470942	-1.53502436633223\\
1.3246936464482	-1.54709418837675\\
1.32264529058116	-1.54885004938148\\
1.30661322645291	-1.56239235944496\\
1.30573255097363	-1.56312625250501\\
1.29058116232465	-1.57567662339068\\
1.28630617282375	-1.57915831663327\\
1.27454909819639	-1.58868212050701\\
1.26637210710088	-1.59519038076152\\
1.25851703406814	-1.6014125940218\\
1.24590455775322	-1.61122244488978\\
1.24248496993988	-1.61387110213118\\
1.22645290581162	-1.62607927445993\\
1.22488603589304	-1.62725450901804\\
1.21042084168337	-1.63806843238578\\
1.20329940930239	-1.64328657314629\\
1.19438877755511	-1.64979819740775\\
1.18108768779127	-1.65931863727455\\
1.17835671342685	-1.66126930815948\\
1.1623246492986	-1.67253107570689\\
1.1582350096519	-1.67535070140281\\
1.14629258517034	-1.68357446350079\\
1.13469433852366	-1.69138276553106\\
1.13026052104208	-1.69436592406958\\
1.11422845691383	-1.70494885309161\\
1.11042202092204	-1.70741482965932\\
1.09819639278557	-1.71533709498293\\
1.08537102582778	-1.72344689378758\\
1.08216432865731	-1.72547629165081\\
1.06613226452906	-1.73543980642559\\
1.0594868384357	-1.73947895791583\\
1.0501002004008	-1.74519398719865\\
1.03406813627254	-1.75471371668781\\
1.03270232062295	-1.75551102204409\\
1.01803607214429	-1.76409451676456\\
1.00495154830762	-1.77154308617234\\
1.00200400801603	-1.77322632735206\\
0.985971943887775	-1.78221092077393\\
0.976147648477744	-1.7875751503006\\
0.969939879759519	-1.79097831714864\\
0.953907815631262	-1.79957221763483\\
0.946195618651337	-1.80360721442886\\
0.937875751503006	-1.80798135507683\\
0.921843687374749	-1.81620363298823\\
0.914983296901567	-1.81963927855711\\
0.905811623246493	-1.82425870728494\\
0.889779559118236	-1.83212679912755\\
0.882376565930459	-1.83567134268537\\
0.87374749498998	-1.83983011800848\\
0.857715430861724	-1.84735992669908\\
0.848215233038847	-1.85170340681363\\
0.841683366733467	-1.85471196656349\\
0.825651302605211	-1.86191795049415\\
0.812307901692387	-1.86773547094188\\
0.809619238476954	-1.86891739916824\\
0.793587174348698	-1.87581265086804\\
0.777555110220441	-1.88248456766811\\
0.774399924454319	-1.88376753507014\\
0.761523046092185	-1.8890527520654\\
0.745490981963928	-1.89543326643157\\
0.734185902005566	-1.8997995991984\\
0.729458917835672	-1.9016439984945\\
0.713426853707415	-1.90774305930578\\
0.697394789579159	-1.91363876352957\\
0.691266295395884	-1.91583166332665\\
0.681362725450902	-1.91941561015907\\
0.665330661322646	-1.92504468441812\\
0.649298597194389	-1.93048142973643\\
0.645108701946643	-1.93186372745491\\
0.633266533066132	-1.93581931601292\\
0.617234468937876	-1.94100177610258\\
0.601202404809619	-1.94600162965943\\
0.594940522759552	-1.94789579158317\\
0.585170340681363	-1.95089131332239\\
0.569138276553106	-1.95564689307314\\
0.55310621242485	-1.9602283776053\\
0.53968013010859	-1.96392785571142\\
0.537074148296593	-1.96465646686352\\
0.521042084168337	-1.96900146876237\\
0.50501002004008	-1.9731797541918\\
0.488977955911824	-1.97719482809212\\
0.477504321815857	-1.97995991983968\\
0.472945891783567	-1.98107612481341\\
0.456913827655311	-1.98486317491207\\
0.440881763527054	-1.98849294979072\\
0.424849699398798	-1.99196843670764\\
0.408817635270541	-1.99529248331274\\
0.405303545146575	-1.99599198396794\\
0.392785571142285	-1.99852785627817\\
0.376753507014028	-2.00163050314481\\
0.360721442885771	-2.00458596724657\\
0.344689378757515	-2.00739663109156\\
0.328657314629258	-2.01006475210128\\
0.316244833411446	-2.01202404809619\\
0.312625250501002	-2.01260644040032\\
0.296593186372745	-2.01505450501464\\
0.280561122244489	-2.01736272217602\\
0.264529058116232	-2.01953292135492\\
0.248496993987976	-2.02156681761085\\
0.232464929859719	-2.02346601385099\\
0.216432865731463	-2.02523200293539\\
0.200400801603206	-2.02686616963263\\
0.187721458164462	-2.02805611222445\\
0.18436873747495	-2.02837760724697\\
0.168336673346693	-2.02978609722387\\
0.152304609218437	-2.03106309159634\\
0.13627254509018	-2.03220958359571\\
0.120240480961924	-2.03322646351064\\
0.104208416833667	-2.03411451983944\\
0.0881763527054105	-2.03487444030915\\
0.0721442885771539	-2.03550681276325\\
0.0561122244488974	-2.03601212591963\\
0.0400801603206409	-2.03639077000011\\
0.0240480961923843	-2.03664303723276\\
0.00801603206412782	-2.03676912222778\\
-0.00801603206412826	-2.03676912222778\\
-0.0240480961923848	-2.03664303723276\\
-0.0400801603206413	-2.03639077000011\\
-0.0561122244488979	-2.03601212591963\\
-0.0721442885771544	-2.03550681276325\\
-0.0881763527054109	-2.03487444030915\\
-0.104208416833667	-2.03411451983944\\
-0.120240480961924	-2.03322646351064\\
-0.13627254509018	-2.03220958359571\\
-0.152304609218437	-2.03106309159634\\
-0.168336673346694	-2.02978609722387\\
-0.18436873747495	-2.02837760724697\\
-0.187721458164461	-2.02805611222445\\
-0.200400801603207	-2.02686616963263\\
-0.216432865731463	-2.02523200293539\\
-0.232464929859719	-2.02346601385099\\
-0.248496993987976	-2.02156681761085\\
-0.264529058116232	-2.01953292135492\\
-0.280561122244489	-2.01736272217602\\
-0.296593186372745	-2.01505450501464\\
-0.312625250501002	-2.01260644040032\\
-0.316244833411446	-2.01202404809619\\
-0.328657314629258	-2.01006475210128\\
-0.344689378757515	-2.00739663109156\\
-0.360721442885771	-2.00458596724657\\
-0.376753507014028	-2.00163050314481\\
-0.392785571142285	-1.99852785627817\\
-0.405303545146575	-1.99599198396794\\
-0.408817635270541	-1.99529248331274\\
-0.424849699398798	-1.99196843670764\\
-0.440881763527054	-1.98849294979072\\
-0.456913827655311	-1.98486317491207\\
-0.472945891783567	-1.98107612481341\\
-0.477504321815856	-1.97995991983968\\
-0.488977955911824	-1.97719482809212\\
-0.50501002004008	-1.9731797541918\\
-0.521042084168337	-1.96900146876237\\
-0.537074148296593	-1.96465646686352\\
-0.53968013010859	-1.96392785571142\\
-0.55310621242485	-1.9602283776053\\
-0.569138276553106	-1.95564689307314\\
-0.585170340681363	-1.95089131332239\\
-0.594940522759553	-1.94789579158317\\
-0.601202404809619	-1.94600162965943\\
-0.617234468937876	-1.94100177610258\\
-0.633266533066132	-1.93581931601292\\
-0.645108701946643	-1.93186372745491\\
-0.649298597194389	-1.93048142973643\\
-0.665330661322646	-1.92504468441812\\
-0.681362725450902	-1.91941561015907\\
-0.691266295395885	-1.91583166332665\\
-0.697394789579158	-1.91363876352957\\
-0.713426853707415	-1.90774305930578\\
-0.729458917835671	-1.9016439984945\\
-0.734185902005566	-1.8997995991984\\
-0.745490981963928	-1.89543326643157\\
-0.761523046092184	-1.8890527520654\\
-0.77439992445432	-1.88376753507014\\
-0.777555110220441	-1.88248456766811\\
-0.793587174348697	-1.87581265086804\\
-0.809619238476954	-1.86891739916824\\
-0.812307901692387	-1.86773547094188\\
-0.82565130260521	-1.86191795049415\\
-0.841683366733467	-1.85471196656349\\
-0.848215233038847	-1.85170340681363\\
-0.857715430861723	-1.84735992669908\\
-0.87374749498998	-1.83983011800848\\
-0.882376565930459	-1.83567134268537\\
-0.889779559118236	-1.83212679912755\\
-0.905811623246493	-1.82425870728494\\
-0.914983296901566	-1.81963927855711\\
-0.92184368737475	-1.81620363298823\\
-0.937875751503006	-1.80798135507683\\
-0.946195618651338	-1.80360721442886\\
-0.953907815631263	-1.79957221763483\\
-0.969939879759519	-1.79097831714863\\
-0.976147648477744	-1.7875751503006\\
-0.985971943887776	-1.78221092077393\\
-1.00200400801603	-1.77322632735206\\
-1.00495154830762	-1.77154308617234\\
-1.01803607214429	-1.76409451676456\\
-1.03270232062295	-1.75551102204409\\
-1.03406813627255	-1.75471371668781\\
-1.0501002004008	-1.74519398719865\\
-1.0594868384357	-1.73947895791583\\
-1.06613226452906	-1.73543980642559\\
-1.08216432865731	-1.72547629165081\\
-1.08537102582778	-1.72344689378758\\
-1.09819639278557	-1.71533709498293\\
-1.11042202092204	-1.70741482965932\\
-1.11422845691383	-1.70494885309161\\
-1.13026052104208	-1.69436592406958\\
-1.13469433852366	-1.69138276553106\\
-1.14629258517034	-1.68357446350079\\
-1.1582350096519	-1.67535070140281\\
-1.1623246492986	-1.67253107570689\\
-1.17835671342685	-1.66126930815948\\
-1.18108768779127	-1.65931863727455\\
-1.19438877755511	-1.64979819740775\\
-1.20329940930239	-1.64328657314629\\
-1.21042084168337	-1.63806843238578\\
-1.22488603589304	-1.62725450901804\\
-1.22645290581162	-1.62607927445993\\
-1.24248496993988	-1.61387110213118\\
-1.24590455775322	-1.61122244488978\\
-1.25851703406814	-1.6014125940218\\
-1.26637210710088	-1.59519038076152\\
-1.27454909819639	-1.58868212050701\\
-1.28630617282375	-1.57915831663327\\
-1.29058116232465	-1.57567662339068\\
-1.30573255097363	-1.56312625250501\\
-1.30661322645291	-1.56239235944496\\
-1.32264529058116	-1.54885004938148\\
-1.3246936464482	-1.54709418837675\\
-1.33867735470942	-1.53502436633223\\
-1.34319868030208	-1.5310621242485\\
-1.35470941883768	-1.52089864098993\\
-1.36126002965037	-1.51503006012024\\
-1.37074148296593	-1.5064663542454\\
-1.37889621631559	-1.49899799599198\\
-1.38677354709419	-1.4917201577588\\
-1.39612432753368	-1.48296593186373\\
-1.40280561122244	-1.47665183552235\\
-1.41296010365205	-1.46693386773547\\
-1.4188376753507	-1.46125226145058\\
-1.42941801845526	-1.45090180360721\\
-1.43486973947896	-1.4455113527083\\
-1.4455113527083	-1.43486973947896\\
-1.45090180360721	-1.42941801845526\\
-1.46125226145058	-1.4188376753507\\
-1.46693386773547	-1.41296010365205\\
-1.47665183552235	-1.40280561122244\\
-1.48296593186373	-1.39612432753368\\
-1.4917201577588	-1.38677354709419\\
-1.49899799599198	-1.37889621631559\\
-1.5064663542454	-1.37074148296593\\
-1.51503006012024	-1.36126002965037\\
-1.52089864098993	-1.35470941883768\\
-1.5310621242485	-1.34319868030208\\
-1.53502436633223	-1.33867735470942\\
-1.54709418837675	-1.3246936464482\\
-1.54885004938148	-1.32264529058116\\
-1.56239235944496	-1.30661322645291\\
-1.56312625250501	-1.30573255097363\\
-1.57567662339068	-1.29058116232465\\
-1.57915831663327	-1.28630617282375\\
-1.58868212050701	-1.27454909819639\\
-1.59519038076152	-1.26637210710088\\
-1.6014125940218	-1.25851703406814\\
-1.61122244488978	-1.24590455775322\\
-1.61387110213118	-1.24248496993988\\
-1.62607927445993	-1.22645290581162\\
-1.62725450901804	-1.22488603589304\\
-1.63806843238578	-1.21042084168337\\
-1.64328657314629	-1.20329940930239\\
-1.64979819740775	-1.19438877755511\\
-1.65931863727455	-1.18108768779127\\
-1.66126930815948	-1.17835671342685\\
-1.67253107570689	-1.1623246492986\\
-1.67535070140281	-1.1582350096519\\
-1.68357446350079	-1.14629258517034\\
-1.69138276553106	-1.13469433852366\\
-1.69436592406958	-1.13026052104208\\
-1.70494885309161	-1.11422845691383\\
-1.70741482965932	-1.11042202092204\\
-1.71533709498293	-1.09819639278557\\
-1.72344689378758	-1.08537102582778\\
-1.72547629165081	-1.08216432865731\\
-1.73543980642559	-1.06613226452906\\
-1.73947895791583	-1.0594868384357\\
-1.74519398719865	-1.0501002004008\\
-1.75471371668781	-1.03406813627255\\
-1.75551102204409	-1.03270232062295\\
-1.76409451676456	-1.01803607214429\\
-1.77154308617234	-1.00495154830762\\
-1.77322632735206	-1.00200400801603\\
-1.78221092077393	-0.985971943887776\\
-1.7875751503006	-0.976147648477744\\
-1.79097831714864	-0.969939879759519\\
-1.79957221763483	-0.953907815631263\\
-1.80360721442886	-0.946195618651338\\
-1.80798135507683	-0.937875751503006\\
-1.81620363298823	-0.92184368737475\\
-1.81963927855711	-0.914983296901566\\
-1.82425870728494	-0.905811623246493\\
-1.83212679912755	-0.889779559118236\\
-1.83567134268537	-0.882376565930459\\
-1.83983011800848	-0.87374749498998\\
-1.84735992669908	-0.857715430861723\\
-1.85170340681363	-0.848215233038847\\
-1.85471196656349	-0.841683366733467\\
-1.86191795049415	-0.82565130260521\\
-1.86773547094188	-0.812307901692387\\
-1.86891739916824	-0.809619238476954\\
-1.87581265086804	-0.793587174348697\\
-1.88248456766811	-0.777555110220441\\
-1.88376753507014	-0.77439992445432\\
-1.8890527520654	-0.761523046092184\\
-1.89543326643157	-0.745490981963928\\
-1.8997995991984	-0.734185902005566\\
-1.9016439984945	-0.729458917835671\\
-1.90774305930578	-0.713426853707415\\
-1.91363876352957	-0.697394789579158\\
-1.91583166332665	-0.691266295395885\\
-1.91941561015907	-0.681362725450902\\
-1.92504468441812	-0.665330661322646\\
-1.93048142973643	-0.649298597194389\\
-1.93186372745491	-0.645108701946643\\
-1.93581931601292	-0.633266533066132\\
-1.94100177610258	-0.617234468937876\\
-1.94600162965943	-0.601202404809619\\
-1.94789579158317	-0.594940522759553\\
-1.95089131332239	-0.585170340681363\\
-1.95564689307314	-0.569138276553106\\
-1.9602283776053	-0.55310621242485\\
-1.96392785571142	-0.53968013010859\\
-1.96465646686352	-0.537074148296593\\
-1.96900146876237	-0.521042084168337\\
-1.9731797541918	-0.50501002004008\\
-1.97719482809212	-0.488977955911824\\
-1.97995991983968	-0.477504321815856\\
-1.98107612481341	-0.472945891783567\\
-1.98486317491207	-0.456913827655311\\
-1.98849294979072	-0.440881763527054\\
-1.99196843670764	-0.424849699398798\\
-1.99529248331274	-0.408817635270541\\
-1.99599198396794	-0.405303545146575\\
-1.99852785627817	-0.392785571142285\\
-2.00163050314481	-0.376753507014028\\
-2.00458596724657	-0.360721442885771\\
-2.00739663109156	-0.344689378757515\\
-2.01006475210128	-0.328657314629258\\
-2.01202404809619	-0.316244833411446\\
-2.01260644040032	-0.312625250501002\\
-2.01505450501464	-0.296593186372745\\
-2.01736272217602	-0.280561122244489\\
-2.01953292135492	-0.264529058116232\\
-2.02156681761085	-0.248496993987976\\
-2.02346601385099	-0.232464929859719\\
-2.02523200293539	-0.216432865731463\\
-2.02686616963263	-0.200400801603207\\
-2.02805611222445	-0.187721458164461\\
-2.02837760724697	-0.18436873747495\\
-2.02978609722387	-0.168336673346694\\
-2.03106309159634	-0.152304609218437\\
-2.03220958359571	-0.13627254509018\\
-2.03322646351064	-0.120240480961924\\
-2.03411451983944	-0.104208416833667\\
-2.03487444030915	-0.0881763527054109\\
-2.03550681276325	-0.0721442885771544\\
-2.03601212591963	-0.0561122244488979\\
-2.03639077000011	-0.0400801603206413\\
-2.03664303723276	-0.0240480961923848\\
-2.03676912222778	-0.00801603206412826\\
-2.03676912222778	0.00801603206412782\\
-2.03664303723276	0.0240480961923843\\
-2.03639077000011	0.0400801603206409\\
-2.03601212591963	0.0561122244488974\\
-2.03550681276325	0.0721442885771539\\
-2.03487444030915	0.0881763527054105\\
-2.03411451983944	0.104208416833667\\
-2.03322646351064	0.120240480961924\\
-2.03220958359571	0.13627254509018\\
-2.03106309159634	0.152304609218437\\
-2.02978609722387	0.168336673346693\\
-2.02837760724697	0.18436873747495\\
-2.02805611222445	0.187721458164462\\
}--cycle;


\addplot[area legend,solid,fill=mycolor3,draw=black,forget plot]
table[row sep=crcr] {%
x	y\\
-1.65931863727455	0.0927743990960814\\
-1.65866084856646	0.104208416833667\\
-1.65758349716071	0.120240480961924\\
-1.65634986256987	0.13627254509018\\
-1.654958988222	0.152304609218437\\
-1.65340979405752	0.168336673346693\\
-1.65170107513128	0.18436873747495\\
-1.64983150005048	0.200400801603206\\
-1.6477996092456	0.216432865731463\\
-1.64560381307121	0.232464929859719\\
-1.64328657314629	0.248198158796911\\
-1.64324311662568	0.248496993987976\\
-1.64075581475657	0.264529058116232\\
-1.63810182468295	0.280561122244489\\
-1.63527904901627	0.296593186372745\\
-1.63228525045199	0.312625250501002\\
-1.62911804881478	0.328657314629258\\
-1.62725450901804	0.337615305768523\\
-1.62579874569908	0.344689378757515\\
-1.62233372506126	0.360721442885771\\
-1.61869019290241	0.376753507014028\\
-1.6148652120375	0.392785571142285\\
-1.61122244488978	0.407359695034668\\
-1.6108614653649	0.408817635270541\\
-1.60673026725506	0.424849699398798\\
-1.60241085580038	0.440881763527054\\
-1.59789969187904	0.456913827655311\\
-1.59519038076152	0.466171656065012\\
-1.59322382843788	0.472945891783567\\
-1.58839341217624	0.488977955911824\\
-1.58336269797091	0.50501002004008\\
-1.57915831663327	0.517906118968101\\
-1.57814300060829	0.521042084168337\\
-1.57278083378248	0.537074148296593\\
-1.56720840109552	0.55310621242485\\
-1.56312625250501	0.564444322148378\\
-1.56144603693606	0.569138276553106\\
-1.55552696719553	0.585170340681363\\
-1.54938613689153	0.601202404809619\\
-1.54709418837675	0.607008839778904\\
-1.5430766875631	0.617234468937876\\
-1.53657085965327	0.633266533066132\\
-1.5310621242485	0.646392966685537\\
-1.52984734267091	0.649298597194389\\
-1.52296336149711	0.665330661322646\\
-1.51583585432942	0.681362725450902\\
-1.51503006012024	0.683131146227723\\
-1.50854795198306	0.697394789579159\\
-1.50101806572959	0.713426853707415\\
-1.49899799599198	0.717620144892956\\
-1.4933048892418	0.729458917835672\\
-1.48535523526307	0.745490981963928\\
-1.48296593186373	0.75019096948737\\
-1.47721020094599	0.761523046092185\\
-1.46882150145949	0.777555110220441\\
-1.46693386773547	0.781079766536612\\
-1.46023547149044	0.793587174348698\\
-1.45138643762594	0.809619238476954\\
-1.45090180360721	0.810479557237613\\
-1.4423475929317	0.825651302605211\\
-1.43486973947896	0.838530882065264\\
-1.43303660210122	0.841683366733467\\
-1.42350847681662	0.857715430861724\\
-1.4188376753507	0.865379160805918\\
-1.41372501786664	0.87374749498998\\
-1.40367472356486	0.889779559118236\\
-1.40280561122244	0.891139719322691\\
-1.39339929109565	0.905811623246493\\
-1.38677354709419	0.915885806153943\\
-1.38283945322034	0.921843687374749\\
-1.37200700027747	0.937875751503006\\
-1.37074148296593	0.939714034061706\\
-1.36092229447382	0.953907815631262\\
-1.35470941883768	0.962681244334189\\
-1.34954069520418	0.969939879759519\\
-1.33867735470942	0.984855219059202\\
-1.33785895352791	0.985971943887775\\
-1.32590242505713	1.00200400801603\\
-1.32264529058116	1.00628417259526\\
-1.31363847993417	1.01803607214429\\
-1.30661322645291	1.02701644544191\\
-1.30105371722761	1.03406813627254\\
-1.29058116232465	1.04709182767735\\
-1.28814158144397	1.0501002004008\\
-1.27489724077813	1.06613226452906\\
-1.27454909819639	1.06654726543905\\
-1.26132472085874	1.08216432865731\\
-1.25851703406814	1.08542160669045\\
-1.24739438911309	1.09819639278557\\
-1.24248496993988	1.10373977588461\\
-1.23309571036244	1.11422845691383\\
-1.22645290581162	1.12152899213824\\
-1.21841692631164	1.13026052104208\\
-1.21042084168337	1.13881422943531\\
-1.20334498127854	1.14629258517034\\
-1.19438877755511	1.15561837920824\\
-1.18786544014215	1.1623246492986\\
-1.17835671342685	1.17196239778372\\
-1.17196239778372	1.17835671342685\\
-1.1623246492986	1.18786544014215\\
-1.15561837920824	1.19438877755511\\
-1.14629258517034	1.20334498127854\\
-1.13881422943531	1.21042084168337\\
-1.13026052104208	1.21841692631164\\
-1.12152899213824	1.22645290581162\\
-1.11422845691383	1.23309571036244\\
-1.10373977588461	1.24248496993988\\
-1.09819639278557	1.24739438911309\\
-1.08542160669045	1.25851703406814\\
-1.08216432865731	1.26132472085874\\
-1.06654726543905	1.27454909819639\\
-1.06613226452906	1.27489724077813\\
-1.0501002004008	1.28814158144397\\
-1.04709182767735	1.29058116232465\\
-1.03406813627255	1.30105371722761\\
-1.02701644544191	1.30661322645291\\
-1.01803607214429	1.31363847993417\\
-1.00628417259526	1.32264529058116\\
-1.00200400801603	1.32590242505713\\
-0.985971943887776	1.33785895352791\\
-0.984855219059203	1.33867735470942\\
-0.969939879759519	1.34954069520418\\
-0.962681244334189	1.35470941883768\\
-0.953907815631263	1.36092229447382\\
-0.939714034061706	1.37074148296593\\
-0.937875751503006	1.37200700027747\\
-0.92184368737475	1.38283945322034\\
-0.915885806153944	1.38677354709419\\
-0.905811623246493	1.39339929109565\\
-0.891139719322691	1.40280561122244\\
-0.889779559118236	1.40367472356486\\
-0.87374749498998	1.41372501786664\\
-0.865379160805918	1.4188376753507\\
-0.857715430861723	1.42350847681662\\
-0.841683366733467	1.43303660210122\\
-0.838530882065264	1.43486973947896\\
-0.82565130260521	1.4423475929317\\
-0.810479557237613	1.45090180360721\\
-0.809619238476954	1.45138643762594\\
-0.793587174348697	1.46023547149044\\
-0.781079766536612	1.46693386773547\\
-0.777555110220441	1.46882150145949\\
-0.761523046092184	1.47721020094599\\
-0.75019096948737	1.48296593186373\\
-0.745490981963928	1.48535523526307\\
-0.729458917835671	1.4933048892418\\
-0.717620144892956	1.49899799599198\\
-0.713426853707415	1.50101806572959\\
-0.697394789579158	1.50854795198306\\
-0.683131146227722	1.51503006012024\\
-0.681362725450902	1.51583585432942\\
-0.665330661322646	1.52296336149711\\
-0.649298597194389	1.52984734267091\\
-0.646392966685537	1.5310621242485\\
-0.633266533066132	1.53657085965327\\
-0.617234468937876	1.5430766875631\\
-0.607008839778903	1.54709418837675\\
-0.601202404809619	1.54938613689153\\
-0.585170340681363	1.55552696719553\\
-0.569138276553106	1.56144603693606\\
-0.564444322148376	1.56312625250501\\
-0.55310621242485	1.56720840109552\\
-0.537074148296593	1.57278083378248\\
-0.521042084168337	1.57814300060829\\
-0.517906118968099	1.57915831663327\\
-0.50501002004008	1.58336269797091\\
-0.488977955911824	1.58839341217624\\
-0.472945891783567	1.59322382843788\\
-0.466171656065011	1.59519038076152\\
-0.456913827655311	1.59789969187904\\
-0.440881763527054	1.60241085580038\\
-0.424849699398798	1.60673026725506\\
-0.408817635270541	1.6108614653649\\
-0.407359695034667	1.61122244488978\\
-0.392785571142285	1.6148652120375\\
-0.376753507014028	1.61869019290241\\
-0.360721442885771	1.62233372506126\\
-0.344689378757515	1.62579874569908\\
-0.337615305768519	1.62725450901804\\
-0.328657314629258	1.62911804881478\\
-0.312625250501002	1.63228525045199\\
-0.296593186372745	1.63527904901627\\
-0.280561122244489	1.63810182468295\\
-0.264529058116232	1.64075581475657\\
-0.248496993987976	1.64324311662568\\
-0.248198158796908	1.64328657314629\\
-0.232464929859719	1.64560381307121\\
-0.216432865731463	1.6477996092456\\
-0.200400801603207	1.64983150005048\\
-0.18436873747495	1.65170107513128\\
-0.168336673346694	1.65340979405752\\
-0.152304609218437	1.654958988222\\
-0.13627254509018	1.65634986256987\\
-0.120240480961924	1.65758349716071\\
-0.104208416833667	1.65866084856646\\
-0.0927743990960769	1.65931863727455\\
-0.0881763527054109	1.65958726325867\\
-0.0721442885771544	1.66036753644748\\
-0.0561122244488979	1.66099103341563\\
-0.0400801603206413	1.66145823565094\\
-0.0240480961923848	1.66176950372843\\
-0.00801603206412826	1.66192507777422\\
0.00801603206412782	1.66192507777422\\
0.0240480961923843	1.66176950372843\\
0.0400801603206409	1.66145823565094\\
0.0561122244488974	1.66099103341563\\
0.0721442885771539	1.66036753644748\\
0.0881763527054105	1.65958726325867\\
0.0927743990960769	1.65931863727455\\
0.104208416833667	1.65866084856646\\
0.120240480961924	1.65758349716071\\
0.13627254509018	1.65634986256987\\
0.152304609218437	1.654958988222\\
0.168336673346693	1.65340979405752\\
0.18436873747495	1.65170107513128\\
0.200400801603206	1.64983150005048\\
0.216432865731463	1.6477996092456\\
0.232464929859719	1.64560381307121\\
0.248198158796908	1.64328657314629\\
0.248496993987976	1.64324311662568\\
0.264529058116232	1.64075581475657\\
0.280561122244489	1.63810182468295\\
0.296593186372745	1.63527904901627\\
0.312625250501002	1.63228525045199\\
0.328657314629258	1.62911804881478\\
0.33761530576852	1.62725450901804\\
0.344689378757515	1.62579874569908\\
0.360721442885771	1.62233372506126\\
0.376753507014028	1.61869019290241\\
0.392785571142285	1.6148652120375\\
0.407359695034667	1.61122244488978\\
0.408817635270541	1.6108614653649\\
0.424849699398798	1.60673026725506\\
0.440881763527054	1.60241085580038\\
0.456913827655311	1.59789969187904\\
0.466171656065011	1.59519038076152\\
0.472945891783567	1.59322382843788\\
0.488977955911824	1.58839341217624\\
0.50501002004008	1.58336269797091\\
0.517906118968099	1.57915831663327\\
0.521042084168337	1.57814300060829\\
0.537074148296593	1.57278083378248\\
0.55310621242485	1.56720840109552\\
0.564444322148376	1.56312625250501\\
0.569138276553106	1.56144603693606\\
0.585170340681363	1.55552696719553\\
0.601202404809619	1.54938613689153\\
0.607008839778904	1.54709418837675\\
0.617234468937876	1.5430766875631\\
0.633266533066132	1.53657085965327\\
0.646392966685537	1.5310621242485\\
0.649298597194389	1.52984734267091\\
0.665330661322646	1.52296336149711\\
0.681362725450902	1.51583585432942\\
0.683131146227723	1.51503006012024\\
0.697394789579159	1.50854795198306\\
0.713426853707415	1.50101806572959\\
0.717620144892956	1.49899799599198\\
0.729458917835672	1.4933048892418\\
0.745490981963928	1.48535523526307\\
0.75019096948737	1.48296593186373\\
0.761523046092185	1.47721020094599\\
0.777555110220441	1.46882150145949\\
0.781079766536612	1.46693386773547\\
0.793587174348698	1.46023547149044\\
0.809619238476954	1.45138643762594\\
0.810479557237613	1.45090180360721\\
0.825651302605211	1.4423475929317\\
0.838530882065264	1.43486973947896\\
0.841683366733467	1.43303660210122\\
0.857715430861724	1.42350847681662\\
0.865379160805918	1.4188376753507\\
0.87374749498998	1.41372501786664\\
0.889779559118236	1.40367472356486\\
0.891139719322691	1.40280561122244\\
0.905811623246493	1.39339929109565\\
0.915885806153943	1.38677354709419\\
0.921843687374749	1.38283945322034\\
0.937875751503006	1.37200700027747\\
0.939714034061706	1.37074148296593\\
0.953907815631262	1.36092229447382\\
0.962681244334189	1.35470941883768\\
0.969939879759519	1.34954069520418\\
0.984855219059202	1.33867735470942\\
0.985971943887775	1.33785895352791\\
1.00200400801603	1.32590242505713\\
1.00628417259526	1.32264529058116\\
1.01803607214429	1.31363847993417\\
1.02701644544191	1.30661322645291\\
1.03406813627254	1.30105371722761\\
1.04709182767735	1.29058116232465\\
1.0501002004008	1.28814158144397\\
1.06613226452906	1.27489724077813\\
1.06654726543905	1.27454909819639\\
1.08216432865731	1.26132472085874\\
1.08542160669045	1.25851703406814\\
1.09819639278557	1.24739438911309\\
1.10373977588461	1.24248496993988\\
1.11422845691383	1.23309571036244\\
1.12152899213824	1.22645290581162\\
1.13026052104208	1.21841692631164\\
1.13881422943531	1.21042084168337\\
1.14629258517034	1.20334498127854\\
1.15561837920824	1.19438877755511\\
1.1623246492986	1.18786544014215\\
1.17196239778372	1.17835671342685\\
1.17835671342685	1.17196239778372\\
1.18786544014215	1.1623246492986\\
1.19438877755511	1.15561837920824\\
1.20334498127854	1.14629258517034\\
1.21042084168337	1.13881422943531\\
1.21841692631164	1.13026052104208\\
1.22645290581162	1.12152899213824\\
1.23309571036244	1.11422845691383\\
1.24248496993988	1.10373977588461\\
1.24739438911309	1.09819639278557\\
1.25851703406814	1.08542160669045\\
1.26132472085874	1.08216432865731\\
1.27454909819639	1.06654726543905\\
1.27489724077813	1.06613226452906\\
1.28814158144397	1.0501002004008\\
1.29058116232465	1.04709182767735\\
1.30105371722761	1.03406813627254\\
1.30661322645291	1.02701644544191\\
1.31363847993417	1.01803607214429\\
1.32264529058116	1.00628417259526\\
1.32590242505713	1.00200400801603\\
1.33785895352791	0.985971943887775\\
1.33867735470942	0.984855219059202\\
1.34954069520418	0.969939879759519\\
1.35470941883768	0.962681244334189\\
1.36092229447382	0.953907815631262\\
1.37074148296593	0.939714034061706\\
1.37200700027747	0.937875751503006\\
1.38283945322034	0.921843687374749\\
1.38677354709419	0.915885806153943\\
1.39339929109565	0.905811623246493\\
1.40280561122244	0.891139719322691\\
1.40367472356486	0.889779559118236\\
1.41372501786664	0.87374749498998\\
1.4188376753507	0.865379160805918\\
1.42350847681662	0.857715430861724\\
1.43303660210122	0.841683366733467\\
1.43486973947896	0.838530882065264\\
1.4423475929317	0.825651302605211\\
1.45090180360721	0.810479557237613\\
1.45138643762594	0.809619238476954\\
1.46023547149044	0.793587174348698\\
1.46693386773547	0.781079766536612\\
1.46882150145949	0.777555110220441\\
1.47721020094599	0.761523046092185\\
1.48296593186373	0.75019096948737\\
1.48535523526307	0.745490981963928\\
1.4933048892418	0.729458917835672\\
1.49899799599198	0.717620144892956\\
1.50101806572959	0.713426853707415\\
1.50854795198306	0.697394789579159\\
1.51503006012024	0.683131146227723\\
1.51583585432942	0.681362725450902\\
1.52296336149711	0.665330661322646\\
1.52984734267091	0.649298597194389\\
1.5310621242485	0.646392966685537\\
1.53657085965327	0.633266533066132\\
1.5430766875631	0.617234468937876\\
1.54709418837675	0.607008839778904\\
1.54938613689153	0.601202404809619\\
1.55552696719553	0.585170340681363\\
1.56144603693606	0.569138276553106\\
1.56312625250501	0.564444322148376\\
1.56720840109552	0.55310621242485\\
1.57278083378248	0.537074148296593\\
1.57814300060829	0.521042084168337\\
1.57915831663327	0.517906118968099\\
1.58336269797091	0.50501002004008\\
1.58839341217624	0.488977955911824\\
1.59322382843788	0.472945891783567\\
1.59519038076152	0.466171656065011\\
1.59789969187904	0.456913827655311\\
1.60241085580038	0.440881763527054\\
1.60673026725506	0.424849699398798\\
1.6108614653649	0.408817635270541\\
1.61122244488978	0.407359695034667\\
1.6148652120375	0.392785571142285\\
1.61869019290241	0.376753507014028\\
1.62233372506126	0.360721442885771\\
1.62579874569908	0.344689378757515\\
1.62725450901804	0.33761530576852\\
1.62911804881478	0.328657314629258\\
1.63228525045199	0.312625250501002\\
1.63527904901627	0.296593186372745\\
1.63810182468295	0.280561122244489\\
1.64075581475657	0.264529058116232\\
1.64324311662568	0.248496993987976\\
1.64328657314629	0.248198158796908\\
1.64560381307121	0.232464929859719\\
1.6477996092456	0.216432865731463\\
1.64983150005048	0.200400801603206\\
1.65170107513128	0.18436873747495\\
1.65340979405752	0.168336673346693\\
1.654958988222	0.152304609218437\\
1.65634986256987	0.13627254509018\\
1.65758349716071	0.120240480961924\\
1.65866084856646	0.104208416833667\\
1.65931863727455	0.0927743990960769\\
1.65958726325867	0.0881763527054105\\
1.66036753644748	0.0721442885771539\\
1.66099103341563	0.0561122244488974\\
1.66145823565094	0.0400801603206409\\
1.66176950372843	0.0240480961923843\\
1.66192507777422	0.00801603206412782\\
1.66192507777422	-0.00801603206412826\\
1.66176950372843	-0.0240480961923848\\
1.66145823565094	-0.0400801603206413\\
1.66099103341563	-0.0561122244488979\\
1.66036753644748	-0.0721442885771544\\
1.65958726325867	-0.0881763527054109\\
1.65931863727455	-0.0927743990960769\\
1.65866084856646	-0.104208416833667\\
1.65758349716071	-0.120240480961924\\
1.65634986256987	-0.13627254509018\\
1.654958988222	-0.152304609218437\\
1.65340979405752	-0.168336673346694\\
1.65170107513128	-0.18436873747495\\
1.64983150005048	-0.200400801603207\\
1.6477996092456	-0.216432865731463\\
1.64560381307121	-0.232464929859719\\
1.64328657314629	-0.248198158796908\\
1.64324311662568	-0.248496993987976\\
1.64075581475657	-0.264529058116232\\
1.63810182468295	-0.280561122244489\\
1.63527904901627	-0.296593186372745\\
1.63228525045199	-0.312625250501002\\
1.62911804881478	-0.328657314629258\\
1.62725450901804	-0.337615305768519\\
1.62579874569908	-0.344689378757515\\
1.62233372506126	-0.360721442885771\\
1.61869019290241	-0.376753507014028\\
1.6148652120375	-0.392785571142285\\
1.61122244488978	-0.407359695034667\\
1.6108614653649	-0.408817635270541\\
1.60673026725506	-0.424849699398798\\
1.60241085580038	-0.440881763527054\\
1.59789969187904	-0.456913827655311\\
1.59519038076152	-0.466171656065011\\
1.59322382843788	-0.472945891783567\\
1.58839341217624	-0.488977955911824\\
1.58336269797091	-0.50501002004008\\
1.57915831663327	-0.517906118968099\\
1.57814300060829	-0.521042084168337\\
1.57278083378248	-0.537074148296593\\
1.56720840109552	-0.55310621242485\\
1.56312625250501	-0.564444322148376\\
1.56144603693606	-0.569138276553106\\
1.55552696719553	-0.585170340681363\\
1.54938613689153	-0.601202404809619\\
1.54709418837675	-0.607008839778903\\
1.5430766875631	-0.617234468937876\\
1.53657085965327	-0.633266533066132\\
1.5310621242485	-0.646392966685537\\
1.52984734267091	-0.649298597194389\\
1.52296336149711	-0.665330661322646\\
1.51583585432942	-0.681362725450902\\
1.51503006012024	-0.683131146227722\\
1.50854795198306	-0.697394789579158\\
1.50101806572959	-0.713426853707415\\
1.49899799599198	-0.717620144892956\\
1.4933048892418	-0.729458917835671\\
1.48535523526307	-0.745490981963928\\
1.48296593186373	-0.75019096948737\\
1.47721020094599	-0.761523046092184\\
1.46882150145949	-0.777555110220441\\
1.46693386773547	-0.781079766536612\\
1.46023547149044	-0.793587174348697\\
1.45138643762594	-0.809619238476954\\
1.45090180360721	-0.810479557237613\\
1.4423475929317	-0.82565130260521\\
1.43486973947896	-0.838530882065264\\
1.43303660210122	-0.841683366733467\\
1.42350847681662	-0.857715430861723\\
1.4188376753507	-0.865379160805918\\
1.41372501786664	-0.87374749498998\\
1.40367472356486	-0.889779559118236\\
1.40280561122244	-0.891139719322691\\
1.39339929109565	-0.905811623246493\\
1.38677354709419	-0.915885806153944\\
1.38283945322034	-0.92184368737475\\
1.37200700027747	-0.937875751503006\\
1.37074148296593	-0.939714034061706\\
1.36092229447382	-0.953907815631263\\
1.35470941883768	-0.962681244334189\\
1.34954069520418	-0.969939879759519\\
1.33867735470942	-0.984855219059203\\
1.33785895352791	-0.985971943887776\\
1.32590242505713	-1.00200400801603\\
1.32264529058116	-1.00628417259526\\
1.31363847993417	-1.01803607214429\\
1.30661322645291	-1.02701644544191\\
1.30105371722761	-1.03406813627255\\
1.29058116232465	-1.04709182767735\\
1.28814158144397	-1.0501002004008\\
1.27489724077813	-1.06613226452906\\
1.27454909819639	-1.06654726543905\\
1.26132472085874	-1.08216432865731\\
1.25851703406814	-1.08542160669045\\
1.24739438911309	-1.09819639278557\\
1.24248496993988	-1.10373977588461\\
1.23309571036244	-1.11422845691383\\
1.22645290581162	-1.12152899213824\\
1.21841692631164	-1.13026052104208\\
1.21042084168337	-1.13881422943531\\
1.20334498127854	-1.14629258517034\\
1.19438877755511	-1.15561837920824\\
1.18786544014215	-1.1623246492986\\
1.17835671342685	-1.17196239778372\\
1.17196239778372	-1.17835671342685\\
1.1623246492986	-1.18786544014215\\
1.15561837920824	-1.19438877755511\\
1.14629258517034	-1.20334498127854\\
1.13881422943531	-1.21042084168337\\
1.13026052104208	-1.21841692631164\\
1.12152899213824	-1.22645290581162\\
1.11422845691383	-1.23309571036244\\
1.10373977588461	-1.24248496993988\\
1.09819639278557	-1.24739438911309\\
1.08542160669045	-1.25851703406814\\
1.08216432865731	-1.26132472085874\\
1.06654726543905	-1.27454909819639\\
1.06613226452906	-1.27489724077813\\
1.0501002004008	-1.28814158144397\\
1.04709182767735	-1.29058116232465\\
1.03406813627254	-1.30105371722761\\
1.02701644544191	-1.30661322645291\\
1.01803607214429	-1.31363847993417\\
1.00628417259526	-1.32264529058116\\
1.00200400801603	-1.32590242505713\\
0.985971943887775	-1.33785895352791\\
0.984855219059202	-1.33867735470942\\
0.969939879759519	-1.34954069520418\\
0.962681244334189	-1.35470941883768\\
0.953907815631262	-1.36092229447382\\
0.939714034061706	-1.37074148296593\\
0.937875751503006	-1.37200700027747\\
0.921843687374749	-1.38283945322034\\
0.915885806153943	-1.38677354709419\\
0.905811623246493	-1.39339929109565\\
0.891139719322691	-1.40280561122244\\
0.889779559118236	-1.40367472356486\\
0.87374749498998	-1.41372501786664\\
0.865379160805918	-1.4188376753507\\
0.857715430861724	-1.42350847681662\\
0.841683366733467	-1.43303660210122\\
0.838530882065264	-1.43486973947896\\
0.825651302605211	-1.4423475929317\\
0.810479557237613	-1.45090180360721\\
0.809619238476954	-1.45138643762594\\
0.793587174348698	-1.46023547149044\\
0.781079766536612	-1.46693386773547\\
0.777555110220441	-1.46882150145949\\
0.761523046092185	-1.47721020094599\\
0.75019096948737	-1.48296593186373\\
0.745490981963928	-1.48535523526307\\
0.729458917835672	-1.4933048892418\\
0.717620144892956	-1.49899799599198\\
0.713426853707415	-1.50101806572959\\
0.697394789579159	-1.50854795198306\\
0.683131146227723	-1.51503006012024\\
0.681362725450902	-1.51583585432942\\
0.665330661322646	-1.52296336149711\\
0.649298597194389	-1.52984734267091\\
0.646392966685537	-1.5310621242485\\
0.633266533066132	-1.53657085965327\\
0.617234468937876	-1.5430766875631\\
0.607008839778904	-1.54709418837675\\
0.601202404809619	-1.54938613689153\\
0.585170340681363	-1.55552696719553\\
0.569138276553106	-1.56144603693606\\
0.564444322148378	-1.56312625250501\\
0.55310621242485	-1.56720840109552\\
0.537074148296593	-1.57278083378248\\
0.521042084168337	-1.57814300060829\\
0.517906118968101	-1.57915831663327\\
0.50501002004008	-1.58336269797091\\
0.488977955911824	-1.58839341217624\\
0.472945891783567	-1.59322382843788\\
0.466171656065012	-1.59519038076152\\
0.456913827655311	-1.59789969187904\\
0.440881763527054	-1.60241085580038\\
0.424849699398798	-1.60673026725506\\
0.408817635270541	-1.6108614653649\\
0.407359695034668	-1.61122244488978\\
0.392785571142285	-1.6148652120375\\
0.376753507014028	-1.61869019290241\\
0.360721442885771	-1.62233372506126\\
0.344689378757515	-1.62579874569908\\
0.337615305768523	-1.62725450901804\\
0.328657314629258	-1.62911804881478\\
0.312625250501002	-1.63228525045199\\
0.296593186372745	-1.63527904901627\\
0.280561122244489	-1.63810182468295\\
0.264529058116232	-1.64075581475657\\
0.248496993987976	-1.64324311662568\\
0.248198158796911	-1.64328657314629\\
0.232464929859719	-1.64560381307121\\
0.216432865731463	-1.6477996092456\\
0.200400801603206	-1.64983150005048\\
0.18436873747495	-1.65170107513128\\
0.168336673346693	-1.65340979405752\\
0.152304609218437	-1.654958988222\\
0.13627254509018	-1.65634986256987\\
0.120240480961924	-1.65758349716071\\
0.104208416833667	-1.65866084856646\\
0.0927743990960814	-1.65931863727455\\
0.0881763527054105	-1.65958726325867\\
0.0721442885771539	-1.66036753644748\\
0.0561122244488974	-1.66099103341563\\
0.0400801603206409	-1.66145823565094\\
0.0240480961923843	-1.66176950372843\\
0.00801603206412782	-1.66192507777422\\
-0.00801603206412826	-1.66192507777422\\
-0.0240480961923848	-1.66176950372843\\
-0.0400801603206413	-1.66145823565094\\
-0.0561122244488979	-1.66099103341563\\
-0.0721442885771544	-1.66036753644748\\
-0.0881763527054109	-1.65958726325867\\
-0.0927743990960823	-1.65931863727455\\
-0.104208416833667	-1.65866084856646\\
-0.120240480961924	-1.65758349716071\\
-0.13627254509018	-1.65634986256987\\
-0.152304609218437	-1.654958988222\\
-0.168336673346694	-1.65340979405752\\
-0.18436873747495	-1.65170107513128\\
-0.200400801603207	-1.64983150005048\\
-0.216432865731463	-1.6477996092456\\
-0.232464929859719	-1.64560381307121\\
-0.248198158796911	-1.64328657314629\\
-0.248496993987976	-1.64324311662568\\
-0.264529058116232	-1.64075581475657\\
-0.280561122244489	-1.63810182468295\\
-0.296593186372745	-1.63527904901627\\
-0.312625250501002	-1.63228525045199\\
-0.328657314629258	-1.62911804881478\\
-0.337615305768522	-1.62725450901804\\
-0.344689378757515	-1.62579874569908\\
-0.360721442885771	-1.62233372506126\\
-0.376753507014028	-1.61869019290241\\
-0.392785571142285	-1.6148652120375\\
-0.407359695034668	-1.61122244488978\\
-0.408817635270541	-1.6108614653649\\
-0.424849699398798	-1.60673026725506\\
-0.440881763527054	-1.60241085580038\\
-0.456913827655311	-1.59789969187904\\
-0.466171656065013	-1.59519038076152\\
-0.472945891783567	-1.59322382843788\\
-0.488977955911824	-1.58839341217624\\
-0.50501002004008	-1.58336269797091\\
-0.517906118968101	-1.57915831663327\\
-0.521042084168337	-1.57814300060829\\
-0.537074148296593	-1.57278083378248\\
-0.55310621242485	-1.56720840109552\\
-0.564444322148377	-1.56312625250501\\
-0.569138276553106	-1.56144603693606\\
-0.585170340681363	-1.55552696719553\\
-0.601202404809619	-1.54938613689153\\
-0.607008839778903	-1.54709418837675\\
-0.617234468937876	-1.5430766875631\\
-0.633266533066132	-1.53657085965327\\
-0.646392966685537	-1.5310621242485\\
-0.649298597194389	-1.52984734267091\\
-0.665330661322646	-1.52296336149711\\
-0.681362725450902	-1.51583585432942\\
-0.683131146227722	-1.51503006012024\\
-0.697394789579158	-1.50854795198306\\
-0.713426853707415	-1.50101806572959\\
-0.717620144892956	-1.49899799599198\\
-0.729458917835671	-1.4933048892418\\
-0.745490981963928	-1.48535523526307\\
-0.75019096948737	-1.48296593186373\\
-0.761523046092184	-1.47721020094599\\
-0.777555110220441	-1.46882150145949\\
-0.781079766536612	-1.46693386773547\\
-0.793587174348697	-1.46023547149044\\
-0.809619238476954	-1.45138643762594\\
-0.810479557237613	-1.45090180360721\\
-0.82565130260521	-1.4423475929317\\
-0.838530882065264	-1.43486973947896\\
-0.841683366733467	-1.43303660210122\\
-0.857715430861723	-1.42350847681662\\
-0.865379160805918	-1.4188376753507\\
-0.87374749498998	-1.41372501786664\\
-0.889779559118236	-1.40367472356486\\
-0.891139719322691	-1.40280561122244\\
-0.905811623246493	-1.39339929109565\\
-0.915885806153944	-1.38677354709419\\
-0.92184368737475	-1.38283945322034\\
-0.937875751503006	-1.37200700027747\\
-0.939714034061706	-1.37074148296593\\
-0.953907815631263	-1.36092229447382\\
-0.962681244334189	-1.35470941883768\\
-0.969939879759519	-1.34954069520418\\
-0.984855219059203	-1.33867735470942\\
-0.985971943887776	-1.33785895352791\\
-1.00200400801603	-1.32590242505713\\
-1.00628417259526	-1.32264529058116\\
-1.01803607214429	-1.31363847993417\\
-1.02701644544191	-1.30661322645291\\
-1.03406813627255	-1.30105371722761\\
-1.04709182767735	-1.29058116232465\\
-1.0501002004008	-1.28814158144397\\
-1.06613226452906	-1.27489724077813\\
-1.06654726543905	-1.27454909819639\\
-1.08216432865731	-1.26132472085874\\
-1.08542160669045	-1.25851703406814\\
-1.09819639278557	-1.24739438911309\\
-1.10373977588461	-1.24248496993988\\
-1.11422845691383	-1.23309571036244\\
-1.12152899213824	-1.22645290581162\\
-1.13026052104208	-1.21841692631164\\
-1.13881422943531	-1.21042084168337\\
-1.14629258517034	-1.20334498127854\\
-1.15561837920824	-1.19438877755511\\
-1.1623246492986	-1.18786544014215\\
-1.17196239778372	-1.17835671342685\\
-1.17835671342685	-1.17196239778372\\
-1.18786544014215	-1.1623246492986\\
-1.19438877755511	-1.15561837920824\\
-1.20334498127854	-1.14629258517034\\
-1.21042084168337	-1.13881422943531\\
-1.21841692631164	-1.13026052104208\\
-1.22645290581162	-1.12152899213824\\
-1.23309571036244	-1.11422845691383\\
-1.24248496993988	-1.10373977588461\\
-1.24739438911309	-1.09819639278557\\
-1.25851703406814	-1.08542160669045\\
-1.26132472085874	-1.08216432865731\\
-1.27454909819639	-1.06654726543905\\
-1.27489724077813	-1.06613226452906\\
-1.28814158144397	-1.0501002004008\\
-1.29058116232465	-1.04709182767735\\
-1.30105371722761	-1.03406813627255\\
-1.30661322645291	-1.02701644544191\\
-1.31363847993417	-1.01803607214429\\
-1.32264529058116	-1.00628417259526\\
-1.32590242505713	-1.00200400801603\\
-1.33785895352791	-0.985971943887776\\
-1.33867735470942	-0.984855219059203\\
-1.34954069520418	-0.969939879759519\\
-1.35470941883768	-0.962681244334189\\
-1.36092229447382	-0.953907815631263\\
-1.37074148296593	-0.939714034061706\\
-1.37200700027747	-0.937875751503006\\
-1.38283945322034	-0.92184368737475\\
-1.38677354709419	-0.915885806153944\\
-1.39339929109565	-0.905811623246493\\
-1.40280561122244	-0.891139719322691\\
-1.40367472356486	-0.889779559118236\\
-1.41372501786664	-0.87374749498998\\
-1.4188376753507	-0.865379160805918\\
-1.42350847681662	-0.857715430861723\\
-1.43303660210122	-0.841683366733467\\
-1.43486973947896	-0.838530882065264\\
-1.4423475929317	-0.82565130260521\\
-1.45090180360721	-0.810479557237613\\
-1.45138643762594	-0.809619238476954\\
-1.46023547149044	-0.793587174348697\\
-1.46693386773547	-0.781079766536612\\
-1.46882150145949	-0.777555110220441\\
-1.47721020094599	-0.761523046092184\\
-1.48296593186373	-0.75019096948737\\
-1.48535523526307	-0.745490981963928\\
-1.4933048892418	-0.729458917835671\\
-1.49899799599198	-0.717620144892956\\
-1.50101806572959	-0.713426853707415\\
-1.50854795198306	-0.697394789579158\\
-1.51503006012024	-0.683131146227722\\
-1.51583585432942	-0.681362725450902\\
-1.52296336149711	-0.665330661322646\\
-1.52984734267091	-0.649298597194389\\
-1.5310621242485	-0.646392966685537\\
-1.53657085965327	-0.633266533066132\\
-1.5430766875631	-0.617234468937876\\
-1.54709418837675	-0.607008839778903\\
-1.54938613689153	-0.601202404809619\\
-1.55552696719553	-0.585170340681363\\
-1.56144603693606	-0.569138276553106\\
-1.56312625250501	-0.564444322148377\\
-1.56720840109552	-0.55310621242485\\
-1.57278083378248	-0.537074148296593\\
-1.57814300060829	-0.521042084168337\\
-1.57915831663327	-0.517906118968101\\
-1.58336269797091	-0.50501002004008\\
-1.58839341217624	-0.488977955911824\\
-1.59322382843788	-0.472945891783567\\
-1.59519038076152	-0.466171656065013\\
-1.59789969187904	-0.456913827655311\\
-1.60241085580038	-0.440881763527054\\
-1.60673026725506	-0.424849699398798\\
-1.6108614653649	-0.408817635270541\\
-1.61122244488978	-0.407359695034668\\
-1.6148652120375	-0.392785571142285\\
-1.61869019290241	-0.376753507014028\\
-1.62233372506126	-0.360721442885771\\
-1.62579874569908	-0.344689378757515\\
-1.62725450901804	-0.337615305768522\\
-1.62911804881478	-0.328657314629258\\
-1.63228525045199	-0.312625250501002\\
-1.63527904901627	-0.296593186372745\\
-1.63810182468295	-0.280561122244489\\
-1.64075581475657	-0.264529058116232\\
-1.64324311662568	-0.248496993987976\\
-1.64328657314629	-0.248198158796911\\
-1.64560381307121	-0.232464929859719\\
-1.6477996092456	-0.216432865731463\\
-1.64983150005048	-0.200400801603207\\
-1.65170107513128	-0.18436873747495\\
-1.65340979405752	-0.168336673346694\\
-1.654958988222	-0.152304609218437\\
-1.65634986256987	-0.13627254509018\\
-1.65758349716071	-0.120240480961924\\
-1.65866084856646	-0.104208416833667\\
-1.65931863727455	-0.0927743990960823\\
-1.65958726325867	-0.0881763527054109\\
-1.66036753644748	-0.0721442885771544\\
-1.66099103341563	-0.0561122244488979\\
-1.66145823565094	-0.0400801603206413\\
-1.66176950372843	-0.0240480961923848\\
-1.66192507777422	-0.00801603206412826\\
-1.66192507777422	0.00801603206412782\\
-1.66176950372843	0.0240480961923843\\
-1.66145823565094	0.0400801603206409\\
-1.66099103341563	0.0561122244488974\\
-1.66036753644748	0.0721442885771539\\
-1.65958726325867	0.0881763527054105\\
-1.65931863727455	0.0927743990960814\\
}--cycle;


\addplot[area legend,solid,fill=mycolor4,draw=black,forget plot]
table[row sep=crcr] {%
x	y\\
-1.38677354709419	0.167056987109561\\
-1.38662619559488	0.168336673346693\\
-1.38459487613202	0.18436873747495\\
-1.38237233139946	0.200400801603206\\
-1.37995682626576	0.216432865731463\\
-1.37734647096564	0.232464929859719\\
-1.37453921863592	0.248496993987976\\
-1.37153286264059	0.264529058116232\\
-1.37074148296593	0.268497242466576\\
-1.36834974586935	0.280561122244489\\
-1.36497280054311	0.296593186372745\\
-1.36139125696221	0.312625250501002\\
-1.35760226767331	0.328657314629258\\
-1.35470941883768	0.340270770314924\\
-1.35361369422421	0.344689378757515\\
-1.34944200523125	0.360721442885771\\
-1.34505539845706	0.376753507014028\\
-1.34045033769688	0.392785571142285\\
-1.33867735470942	0.398697900007648\\
-1.33565192935957	0.408817635270541\\
-1.33064629029176	0.424849699398798\\
-1.32541259920442	0.440881763527054\\
-1.32264529058116	0.449027216786655\\
-1.3199709691539	0.456913827655311\\
-1.31431966325302	0.472945891783567\\
-1.30842898782783	0.488977955911824\\
-1.30661322645291	0.493749636586757\\
-1.30233136585276	0.50501002004008\\
-1.29600220259937	0.521042084168337\\
-1.29058116232465	0.534266644543034\\
-1.28943005782504	0.537074148296593\\
-1.28264658561912	0.55310621242485\\
-1.27560140166228	0.569138276553106\\
-1.27454909819639	0.571463575681558\\
-1.26833748336955	0.585170340681363\\
-1.26080996048874	0.601202404809619\\
-1.25851703406814	0.605944289907037\\
-1.25304496252397	0.617234468937876\\
-1.24501397850635	0.633266533066132\\
-1.24248496993988	0.638174135675986\\
-1.23673351106921	0.649298597194389\\
-1.22817552985137	0.665330661322646\\
-1.22645290581162	0.66847466809011\\
-1.21936168955823	0.681362725450902\\
-1.21042084168337	0.697098448994779\\
-1.21025163878264	0.697394789579159\\
-1.20088172800499	0.713426853707415\\
-1.19438877755511	0.724206903335001\\
-1.1912068959483	0.729458917835672\\
-1.18123906320708	0.745490981963928\\
-1.17835671342685	0.750013470828635\\
-1.17097122138537	0.761523046092185\\
-1.1623246492986	0.774630969271663\\
-1.16038124081018	0.777555110220441\\
-1.1494781855221	0.793587174348698\\
-1.14629258517034	0.798161521404323\\
-1.13824557324775	0.809619238476954\\
-1.13026052104208	0.8207072225884\\
-1.12666721310274	0.825651302605211\\
-1.11473692098132	0.841683366733467\\
-1.11422845691383	0.842353461338047\\
-1.10245298445587	0.857715430861724\\
-1.09819639278557	0.863143962040195\\
-1.08979111603593	0.87374749498998\\
-1.08216432865731	0.883161278542966\\
-1.07673980561052	0.889779559118236\\
-1.06613226452906	0.902452843362276\\
-1.06328606036738	0.905811623246493\\
-1.0501002004008	0.921061858665545\\
-1.0494153031161	0.921843687374749\\
-1.03511237960294	0.937875751503006\\
-1.03406813627255	0.939025304685701\\
-1.02035715778336	0.953907815631262\\
-1.01803607214429	0.956382371310518\\
-1.00512914955283	0.969939879759519\\
-1.00200400801603	0.973166657680673\\
-0.989406881954105	0.985971943887775\\
-0.985971943887776	0.989406881954104\\
-0.973166657680673	1.00200400801603\\
-0.969939879759519	1.00512914955283\\
-0.956382371310518	1.01803607214429\\
-0.953907815631263	1.02035715778336\\
-0.939025304685701	1.03406813627254\\
-0.937875751503006	1.03511237960294\\
-0.92184368737475	1.0494153031161\\
-0.921061858665545	1.0501002004008\\
-0.905811623246493	1.06328606036738\\
-0.902452843362276	1.06613226452906\\
-0.889779559118236	1.07673980561052\\
-0.883161278542967	1.08216432865731\\
-0.87374749498998	1.08979111603593\\
-0.863143962040196	1.09819639278557\\
-0.857715430861723	1.10245298445587\\
-0.842353461338047	1.11422845691383\\
-0.841683366733467	1.11473692098132\\
-0.82565130260521	1.12666721310274\\
-0.8207072225884	1.13026052104208\\
-0.809619238476954	1.13824557324775\\
-0.798161521404323	1.14629258517034\\
-0.793587174348697	1.1494781855221\\
-0.777555110220441	1.16038124081018\\
-0.774630969271663	1.1623246492986\\
-0.761523046092184	1.17097122138537\\
-0.750013470828635	1.17835671342685\\
-0.745490981963928	1.18123906320708\\
-0.729458917835671	1.1912068959483\\
-0.724206903335001	1.19438877755511\\
-0.713426853707415	1.20088172800499\\
-0.697394789579158	1.21025163878264\\
-0.697098448994779	1.21042084168337\\
-0.681362725450902	1.21936168955823\\
-0.66847466809011	1.22645290581162\\
-0.665330661322646	1.22817552985137\\
-0.649298597194389	1.23673351106921\\
-0.638174135675986	1.24248496993988\\
-0.633266533066132	1.24501397850635\\
-0.617234468937876	1.25304496252397\\
-0.605944289907037	1.25851703406814\\
-0.601202404809619	1.26080996048874\\
-0.585170340681363	1.26833748336955\\
-0.571463575681558	1.27454909819639\\
-0.569138276553106	1.27560140166228\\
-0.55310621242485	1.28264658561912\\
-0.537074148296593	1.28943005782504\\
-0.534266644543035	1.29058116232465\\
-0.521042084168337	1.29600220259937\\
-0.50501002004008	1.30233136585276\\
-0.493749636586757	1.30661322645291\\
-0.488977955911824	1.30842898782783\\
-0.472945891783567	1.31431966325302\\
-0.456913827655311	1.3199709691539\\
-0.449027216786655	1.32264529058116\\
-0.440881763527054	1.32541259920442\\
-0.424849699398798	1.33064629029176\\
-0.408817635270541	1.33565192935957\\
-0.398697900007648	1.33867735470942\\
-0.392785571142285	1.34045033769688\\
-0.376753507014028	1.34505539845706\\
-0.360721442885771	1.34944200523125\\
-0.344689378757515	1.35361369422421\\
-0.340270770314924	1.35470941883768\\
-0.328657314629258	1.35760226767331\\
-0.312625250501002	1.36139125696221\\
-0.296593186372745	1.36497280054311\\
-0.280561122244489	1.36834974586935\\
-0.268497242466576	1.37074148296593\\
-0.264529058116232	1.37153286264059\\
-0.248496993987976	1.37453921863592\\
-0.232464929859719	1.37734647096564\\
-0.216432865731463	1.37995682626576\\
-0.200400801603207	1.38237233139946\\
-0.18436873747495	1.38459487613202\\
-0.168336673346694	1.38662619559488\\
-0.167056987109562	1.38677354709419\\
-0.152304609218437	1.38848605043233\\
-0.13627254509018	1.3901572568314\\
-0.120240480961924	1.39163953165047\\
-0.104208416833667	1.39293402425944\\
-0.0881763527054109	1.39404173713703\\
-0.0721442885771544	1.39496352716476\\
-0.0561122244488979	1.39570010673177\\
-0.0400801603206413	1.3962520446525\\
-0.0240480961923848	1.39661976689889\\
-0.00801603206412826	1.3968035571483\\
0.00801603206412782	1.3968035571483\\
0.0240480961923843	1.39661976689889\\
0.0400801603206409	1.3962520446525\\
0.0561122244488974	1.39570010673177\\
0.0721442885771539	1.39496352716476\\
0.0881763527054105	1.39404173713704\\
0.104208416833667	1.39293402425944\\
0.120240480961924	1.39163953165047\\
0.13627254509018	1.3901572568314\\
0.152304609218437	1.38848605043233\\
0.167056987109561	1.38677354709419\\
0.168336673346693	1.38662619559488\\
0.18436873747495	1.38459487613202\\
0.200400801603206	1.38237233139946\\
0.216432865731463	1.37995682626576\\
0.232464929859719	1.37734647096564\\
0.248496993987976	1.37453921863592\\
0.264529058116232	1.37153286264059\\
0.268497242466576	1.37074148296593\\
0.280561122244489	1.36834974586935\\
0.296593186372745	1.36497280054311\\
0.312625250501002	1.36139125696221\\
0.328657314629258	1.35760226767331\\
0.340270770314924	1.35470941883768\\
0.344689378757515	1.35361369422421\\
0.360721442885771	1.34944200523125\\
0.376753507014028	1.34505539845706\\
0.392785571142285	1.34045033769688\\
0.398697900007648	1.33867735470942\\
0.408817635270541	1.33565192935957\\
0.424849699398798	1.33064629029176\\
0.440881763527054	1.32541259920442\\
0.449027216786655	1.32264529058116\\
0.456913827655311	1.3199709691539\\
0.472945891783567	1.31431966325302\\
0.488977955911824	1.30842898782783\\
0.493749636586757	1.30661322645291\\
0.50501002004008	1.30233136585276\\
0.521042084168337	1.29600220259937\\
0.534266644543034	1.29058116232465\\
0.537074148296593	1.28943005782504\\
0.55310621242485	1.28264658561912\\
0.569138276553106	1.27560140166228\\
0.571463575681558	1.27454909819639\\
0.585170340681363	1.26833748336955\\
0.601202404809619	1.26080996048874\\
0.605944289907037	1.25851703406814\\
0.617234468937876	1.25304496252397\\
0.633266533066132	1.24501397850635\\
0.638174135675986	1.24248496993988\\
0.649298597194389	1.23673351106921\\
0.665330661322646	1.22817552985137\\
0.66847466809011	1.22645290581162\\
0.681362725450902	1.21936168955823\\
0.697098448994779	1.21042084168337\\
0.697394789579159	1.21025163878264\\
0.713426853707415	1.20088172800499\\
0.724206903335001	1.19438877755511\\
0.729458917835672	1.1912068959483\\
0.745490981963928	1.18123906320708\\
0.750013470828635	1.17835671342685\\
0.761523046092185	1.17097122138537\\
0.774630969271663	1.1623246492986\\
0.777555110220441	1.16038124081018\\
0.793587174348698	1.1494781855221\\
0.798161521404323	1.14629258517034\\
0.809619238476954	1.13824557324775\\
0.8207072225884	1.13026052104208\\
0.825651302605211	1.12666721310274\\
0.841683366733467	1.11473692098132\\
0.842353461338047	1.11422845691383\\
0.857715430861724	1.10245298445587\\
0.863143962040196	1.09819639278557\\
0.87374749498998	1.08979111603593\\
0.883161278542967	1.08216432865731\\
0.889779559118236	1.07673980561052\\
0.902452843362276	1.06613226452906\\
0.905811623246493	1.06328606036738\\
0.921061858665545	1.0501002004008\\
0.921843687374749	1.0494153031161\\
0.937875751503006	1.03511237960294\\
0.939025304685701	1.03406813627254\\
0.953907815631262	1.02035715778336\\
0.956382371310518	1.01803607214429\\
0.969939879759519	1.00512914955283\\
0.973166657680673	1.00200400801603\\
0.985971943887775	0.989406881954104\\
0.989406881954104	0.985971943887775\\
1.00200400801603	0.973166657680673\\
1.00512914955283	0.969939879759519\\
1.01803607214429	0.956382371310518\\
1.02035715778336	0.953907815631262\\
1.03406813627254	0.939025304685701\\
1.03511237960294	0.937875751503006\\
1.0494153031161	0.921843687374749\\
1.0501002004008	0.921061858665545\\
1.06328606036738	0.905811623246493\\
1.06613226452906	0.902452843362276\\
1.07673980561052	0.889779559118236\\
1.08216432865731	0.883161278542967\\
1.08979111603593	0.87374749498998\\
1.09819639278557	0.863143962040196\\
1.10245298445587	0.857715430861724\\
1.11422845691383	0.842353461338047\\
1.11473692098132	0.841683366733467\\
1.12666721310274	0.825651302605211\\
1.13026052104208	0.8207072225884\\
1.13824557324775	0.809619238476954\\
1.14629258517034	0.798161521404323\\
1.1494781855221	0.793587174348698\\
1.16038124081018	0.777555110220441\\
1.1623246492986	0.774630969271663\\
1.17097122138537	0.761523046092185\\
1.17835671342685	0.750013470828635\\
1.18123906320708	0.745490981963928\\
1.1912068959483	0.729458917835672\\
1.19438877755511	0.724206903335001\\
1.20088172800499	0.713426853707415\\
1.21025163878264	0.697394789579159\\
1.21042084168337	0.697098448994779\\
1.21936168955823	0.681362725450902\\
1.22645290581162	0.66847466809011\\
1.22817552985137	0.665330661322646\\
1.23673351106921	0.649298597194389\\
1.24248496993988	0.638174135675986\\
1.24501397850635	0.633266533066132\\
1.25304496252397	0.617234468937876\\
1.25851703406814	0.605944289907037\\
1.26080996048874	0.601202404809619\\
1.26833748336955	0.585170340681363\\
1.27454909819639	0.571463575681558\\
1.27560140166228	0.569138276553106\\
1.28264658561912	0.55310621242485\\
1.28943005782504	0.537074148296593\\
1.29058116232465	0.534266644543034\\
1.29600220259937	0.521042084168337\\
1.30233136585276	0.50501002004008\\
1.30661322645291	0.493749636586757\\
1.30842898782783	0.488977955911824\\
1.31431966325302	0.472945891783567\\
1.3199709691539	0.456913827655311\\
1.32264529058116	0.449027216786655\\
1.32541259920442	0.440881763527054\\
1.33064629029176	0.424849699398798\\
1.33565192935957	0.408817635270541\\
1.33867735470942	0.398697900007648\\
1.34045033769688	0.392785571142285\\
1.34505539845706	0.376753507014028\\
1.34944200523125	0.360721442885771\\
1.35361369422421	0.344689378757515\\
1.35470941883768	0.340270770314924\\
1.35760226767331	0.328657314629258\\
1.36139125696221	0.312625250501002\\
1.36497280054311	0.296593186372745\\
1.36834974586935	0.280561122244489\\
1.37074148296593	0.268497242466576\\
1.37153286264059	0.264529058116232\\
1.37453921863592	0.248496993987976\\
1.37734647096564	0.232464929859719\\
1.37995682626576	0.216432865731463\\
1.38237233139946	0.200400801603206\\
1.38459487613202	0.18436873747495\\
1.38662619559488	0.168336673346693\\
1.38677354709419	0.167056987109561\\
1.38848605043233	0.152304609218437\\
1.3901572568314	0.13627254509018\\
1.39163953165047	0.120240480961924\\
1.39293402425944	0.104208416833667\\
1.39404173713704	0.0881763527054105\\
1.39496352716476	0.0721442885771539\\
1.39570010673177	0.0561122244488974\\
1.3962520446525	0.0400801603206409\\
1.39661976689889	0.0240480961923843\\
1.3968035571483	0.00801603206412782\\
1.3968035571483	-0.00801603206412826\\
1.39661976689889	-0.0240480961923848\\
1.3962520446525	-0.0400801603206413\\
1.39570010673177	-0.0561122244488979\\
1.39496352716476	-0.0721442885771544\\
1.39404173713703	-0.0881763527054109\\
1.39293402425944	-0.104208416833667\\
1.39163953165047	-0.120240480961924\\
1.3901572568314	-0.13627254509018\\
1.38848605043233	-0.152304609218437\\
1.38677354709419	-0.167056987109562\\
1.38662619559488	-0.168336673346694\\
1.38459487613202	-0.18436873747495\\
1.38237233139946	-0.200400801603207\\
1.37995682626576	-0.216432865731463\\
1.37734647096564	-0.232464929859719\\
1.37453921863592	-0.248496993987976\\
1.37153286264059	-0.264529058116232\\
1.37074148296593	-0.268497242466576\\
1.36834974586935	-0.280561122244489\\
1.36497280054311	-0.296593186372745\\
1.36139125696221	-0.312625250501002\\
1.35760226767331	-0.328657314629258\\
1.35470941883768	-0.340270770314924\\
1.35361369422421	-0.344689378757515\\
1.34944200523125	-0.360721442885771\\
1.34505539845706	-0.376753507014028\\
1.34045033769688	-0.392785571142285\\
1.33867735470942	-0.398697900007648\\
1.33565192935957	-0.408817635270541\\
1.33064629029176	-0.424849699398798\\
1.32541259920442	-0.440881763527054\\
1.32264529058116	-0.449027216786655\\
1.3199709691539	-0.456913827655311\\
1.31431966325302	-0.472945891783567\\
1.30842898782783	-0.488977955911824\\
1.30661322645291	-0.493749636586757\\
1.30233136585276	-0.50501002004008\\
1.29600220259937	-0.521042084168337\\
1.29058116232465	-0.534266644543035\\
1.28943005782504	-0.537074148296593\\
1.28264658561912	-0.55310621242485\\
1.27560140166228	-0.569138276553106\\
1.27454909819639	-0.571463575681558\\
1.26833748336955	-0.585170340681363\\
1.26080996048874	-0.601202404809619\\
1.25851703406814	-0.605944289907037\\
1.25304496252397	-0.617234468937876\\
1.24501397850635	-0.633266533066132\\
1.24248496993988	-0.638174135675986\\
1.23673351106921	-0.649298597194389\\
1.22817552985137	-0.665330661322646\\
1.22645290581162	-0.66847466809011\\
1.21936168955823	-0.681362725450902\\
1.21042084168337	-0.697098448994779\\
1.21025163878264	-0.697394789579158\\
1.20088172800499	-0.713426853707415\\
1.19438877755511	-0.724206903335001\\
1.1912068959483	-0.729458917835671\\
1.18123906320708	-0.745490981963928\\
1.17835671342685	-0.750013470828635\\
1.17097122138537	-0.761523046092184\\
1.1623246492986	-0.774630969271663\\
1.16038124081018	-0.777555110220441\\
1.1494781855221	-0.793587174348697\\
1.14629258517034	-0.798161521404323\\
1.13824557324775	-0.809619238476954\\
1.13026052104208	-0.8207072225884\\
1.12666721310274	-0.82565130260521\\
1.11473692098132	-0.841683366733467\\
1.11422845691383	-0.842353461338047\\
1.10245298445587	-0.857715430861723\\
1.09819639278557	-0.863143962040196\\
1.08979111603593	-0.87374749498998\\
1.08216432865731	-0.883161278542967\\
1.07673980561052	-0.889779559118236\\
1.06613226452906	-0.902452843362276\\
1.06328606036738	-0.905811623246493\\
1.0501002004008	-0.921061858665545\\
1.0494153031161	-0.92184368737475\\
1.03511237960294	-0.937875751503006\\
1.03406813627254	-0.939025304685701\\
1.02035715778336	-0.953907815631263\\
1.01803607214429	-0.956382371310518\\
1.00512914955283	-0.969939879759519\\
1.00200400801603	-0.973166657680673\\
0.989406881954104	-0.985971943887776\\
0.985971943887775	-0.989406881954105\\
0.973166657680673	-1.00200400801603\\
0.969939879759519	-1.00512914955283\\
0.956382371310518	-1.01803607214429\\
0.953907815631262	-1.02035715778336\\
0.939025304685701	-1.03406813627255\\
0.937875751503006	-1.03511237960294\\
0.921843687374749	-1.0494153031161\\
0.921061858665545	-1.0501002004008\\
0.905811623246493	-1.06328606036738\\
0.902452843362276	-1.06613226452906\\
0.889779559118236	-1.07673980561052\\
0.883161278542966	-1.08216432865731\\
0.87374749498998	-1.08979111603593\\
0.863143962040195	-1.09819639278557\\
0.857715430861724	-1.10245298445587\\
0.842353461338047	-1.11422845691383\\
0.841683366733467	-1.11473692098132\\
0.825651302605211	-1.12666721310274\\
0.8207072225884	-1.13026052104208\\
0.809619238476954	-1.13824557324775\\
0.798161521404323	-1.14629258517034\\
0.793587174348698	-1.1494781855221\\
0.777555110220441	-1.16038124081018\\
0.774630969271663	-1.1623246492986\\
0.761523046092185	-1.17097122138537\\
0.750013470828635	-1.17835671342685\\
0.745490981963928	-1.18123906320708\\
0.729458917835672	-1.1912068959483\\
0.724206903335001	-1.19438877755511\\
0.713426853707415	-1.20088172800499\\
0.697394789579159	-1.21025163878264\\
0.697098448994779	-1.21042084168337\\
0.681362725450902	-1.21936168955823\\
0.66847466809011	-1.22645290581162\\
0.665330661322646	-1.22817552985137\\
0.649298597194389	-1.23673351106921\\
0.638174135675986	-1.24248496993988\\
0.633266533066132	-1.24501397850635\\
0.617234468937876	-1.25304496252397\\
0.605944289907037	-1.25851703406814\\
0.601202404809619	-1.26080996048874\\
0.585170340681363	-1.26833748336955\\
0.571463575681558	-1.27454909819639\\
0.569138276553106	-1.27560140166228\\
0.55310621242485	-1.28264658561912\\
0.537074148296593	-1.28943005782504\\
0.534266644543034	-1.29058116232465\\
0.521042084168337	-1.29600220259937\\
0.50501002004008	-1.30233136585276\\
0.493749636586757	-1.30661322645291\\
0.488977955911824	-1.30842898782783\\
0.472945891783567	-1.31431966325302\\
0.456913827655311	-1.3199709691539\\
0.449027216786655	-1.32264529058116\\
0.440881763527054	-1.32541259920442\\
0.424849699398798	-1.33064629029176\\
0.408817635270541	-1.33565192935957\\
0.398697900007648	-1.33867735470942\\
0.392785571142285	-1.34045033769688\\
0.376753507014028	-1.34505539845706\\
0.360721442885771	-1.34944200523125\\
0.344689378757515	-1.35361369422421\\
0.340270770314924	-1.35470941883768\\
0.328657314629258	-1.35760226767331\\
0.312625250501002	-1.36139125696221\\
0.296593186372745	-1.36497280054311\\
0.280561122244489	-1.36834974586935\\
0.268497242466576	-1.37074148296593\\
0.264529058116232	-1.37153286264059\\
0.248496993987976	-1.37453921863592\\
0.232464929859719	-1.37734647096564\\
0.216432865731463	-1.37995682626576\\
0.200400801603206	-1.38237233139946\\
0.18436873747495	-1.38459487613202\\
0.168336673346693	-1.38662619559488\\
0.167056987109561	-1.38677354709419\\
0.152304609218437	-1.38848605043233\\
0.13627254509018	-1.3901572568314\\
0.120240480961924	-1.39163953165047\\
0.104208416833667	-1.39293402425944\\
0.0881763527054105	-1.39404173713704\\
0.0721442885771539	-1.39496352716476\\
0.0561122244488974	-1.39570010673177\\
0.0400801603206409	-1.3962520446525\\
0.0240480961923843	-1.39661976689889\\
0.00801603206412782	-1.3968035571483\\
-0.00801603206412826	-1.3968035571483\\
-0.0240480961923848	-1.39661976689889\\
-0.0400801603206413	-1.3962520446525\\
-0.0561122244488979	-1.39570010673177\\
-0.0721442885771544	-1.39496352716476\\
-0.0881763527054109	-1.39404173713703\\
-0.104208416833667	-1.39293402425944\\
-0.120240480961924	-1.39163953165047\\
-0.13627254509018	-1.3901572568314\\
-0.152304609218437	-1.38848605043233\\
-0.167056987109562	-1.38677354709419\\
-0.168336673346694	-1.38662619559488\\
-0.18436873747495	-1.38459487613202\\
-0.200400801603207	-1.38237233139946\\
-0.216432865731463	-1.37995682626576\\
-0.232464929859719	-1.37734647096564\\
-0.248496993987976	-1.37453921863592\\
-0.264529058116232	-1.37153286264059\\
-0.268497242466576	-1.37074148296593\\
-0.280561122244489	-1.36834974586935\\
-0.296593186372745	-1.36497280054311\\
-0.312625250501002	-1.36139125696221\\
-0.328657314629258	-1.35760226767331\\
-0.340270770314924	-1.35470941883768\\
-0.344689378757515	-1.35361369422421\\
-0.360721442885771	-1.34944200523125\\
-0.376753507014028	-1.34505539845706\\
-0.392785571142285	-1.34045033769688\\
-0.398697900007648	-1.33867735470942\\
-0.408817635270541	-1.33565192935957\\
-0.424849699398798	-1.33064629029176\\
-0.440881763527054	-1.32541259920442\\
-0.449027216786655	-1.32264529058116\\
-0.456913827655311	-1.3199709691539\\
-0.472945891783567	-1.31431966325302\\
-0.488977955911824	-1.30842898782783\\
-0.493749636586757	-1.30661322645291\\
-0.50501002004008	-1.30233136585276\\
-0.521042084168337	-1.29600220259937\\
-0.534266644543035	-1.29058116232465\\
-0.537074148296593	-1.28943005782504\\
-0.55310621242485	-1.28264658561912\\
-0.569138276553106	-1.27560140166228\\
-0.571463575681558	-1.27454909819639\\
-0.585170340681363	-1.26833748336955\\
-0.601202404809619	-1.26080996048874\\
-0.605944289907037	-1.25851703406814\\
-0.617234468937876	-1.25304496252397\\
-0.633266533066132	-1.24501397850635\\
-0.638174135675986	-1.24248496993988\\
-0.649298597194389	-1.23673351106921\\
-0.665330661322646	-1.22817552985137\\
-0.66847466809011	-1.22645290581162\\
-0.681362725450902	-1.21936168955823\\
-0.697098448994779	-1.21042084168337\\
-0.697394789579158	-1.21025163878264\\
-0.713426853707415	-1.20088172800499\\
-0.724206903335001	-1.19438877755511\\
-0.729458917835671	-1.1912068959483\\
-0.745490981963928	-1.18123906320708\\
-0.750013470828635	-1.17835671342685\\
-0.761523046092184	-1.17097122138537\\
-0.774630969271663	-1.1623246492986\\
-0.777555110220441	-1.16038124081018\\
-0.793587174348697	-1.1494781855221\\
-0.798161521404323	-1.14629258517034\\
-0.809619238476954	-1.13824557324775\\
-0.8207072225884	-1.13026052104208\\
-0.82565130260521	-1.12666721310274\\
-0.841683366733467	-1.11473692098132\\
-0.842353461338047	-1.11422845691383\\
-0.857715430861723	-1.10245298445587\\
-0.863143962040195	-1.09819639278557\\
-0.87374749498998	-1.08979111603593\\
-0.883161278542966	-1.08216432865731\\
-0.889779559118236	-1.07673980561052\\
-0.902452843362275	-1.06613226452906\\
-0.905811623246493	-1.06328606036738\\
-0.921061858665545	-1.0501002004008\\
-0.92184368737475	-1.0494153031161\\
-0.937875751503006	-1.03511237960294\\
-0.939025304685701	-1.03406813627255\\
-0.953907815631263	-1.02035715778336\\
-0.956382371310518	-1.01803607214429\\
-0.969939879759519	-1.00512914955283\\
-0.973166657680673	-1.00200400801603\\
-0.985971943887776	-0.989406881954104\\
-0.989406881954104	-0.985971943887776\\
-1.00200400801603	-0.973166657680673\\
-1.00512914955283	-0.969939879759519\\
-1.01803607214429	-0.956382371310518\\
-1.02035715778336	-0.953907815631263\\
-1.03406813627255	-0.939025304685701\\
-1.03511237960294	-0.937875751503006\\
-1.0494153031161	-0.92184368737475\\
-1.0501002004008	-0.921061858665545\\
-1.06328606036738	-0.905811623246493\\
-1.06613226452906	-0.902452843362275\\
-1.07673980561052	-0.889779559118236\\
-1.08216432865731	-0.883161278542966\\
-1.08979111603593	-0.87374749498998\\
-1.09819639278557	-0.863143962040195\\
-1.10245298445587	-0.857715430861723\\
-1.11422845691383	-0.842353461338047\\
-1.11473692098132	-0.841683366733467\\
-1.12666721310274	-0.82565130260521\\
-1.13026052104208	-0.8207072225884\\
-1.13824557324775	-0.809619238476954\\
-1.14629258517034	-0.798161521404323\\
-1.1494781855221	-0.793587174348697\\
-1.16038124081018	-0.777555110220441\\
-1.1623246492986	-0.774630969271663\\
-1.17097122138537	-0.761523046092184\\
-1.17835671342685	-0.750013470828635\\
-1.18123906320708	-0.745490981963928\\
-1.1912068959483	-0.729458917835671\\
-1.19438877755511	-0.724206903335001\\
-1.20088172800499	-0.713426853707415\\
-1.21025163878264	-0.697394789579158\\
-1.21042084168337	-0.697098448994779\\
-1.21936168955823	-0.681362725450902\\
-1.22645290581162	-0.66847466809011\\
-1.22817552985137	-0.665330661322646\\
-1.23673351106921	-0.649298597194389\\
-1.24248496993988	-0.638174135675986\\
-1.24501397850635	-0.633266533066132\\
-1.25304496252397	-0.617234468937876\\
-1.25851703406814	-0.605944289907037\\
-1.26080996048874	-0.601202404809619\\
-1.26833748336955	-0.585170340681363\\
-1.27454909819639	-0.571463575681558\\
-1.27560140166228	-0.569138276553106\\
-1.28264658561912	-0.55310621242485\\
-1.28943005782504	-0.537074148296593\\
-1.29058116232465	-0.534266644543035\\
-1.29600220259937	-0.521042084168337\\
-1.30233136585276	-0.50501002004008\\
-1.30661322645291	-0.493749636586757\\
-1.30842898782783	-0.488977955911824\\
-1.31431966325302	-0.472945891783567\\
-1.3199709691539	-0.456913827655311\\
-1.32264529058116	-0.449027216786655\\
-1.32541259920442	-0.440881763527054\\
-1.33064629029176	-0.424849699398798\\
-1.33565192935957	-0.408817635270541\\
-1.33867735470942	-0.398697900007648\\
-1.34045033769688	-0.392785571142285\\
-1.34505539845706	-0.376753507014028\\
-1.34944200523125	-0.360721442885771\\
-1.35361369422421	-0.344689378757515\\
-1.35470941883768	-0.340270770314924\\
-1.35760226767331	-0.328657314629258\\
-1.36139125696221	-0.312625250501002\\
-1.36497280054311	-0.296593186372745\\
-1.36834974586935	-0.280561122244489\\
-1.37074148296593	-0.268497242466576\\
-1.37153286264059	-0.264529058116232\\
-1.37453921863592	-0.248496993987976\\
-1.37734647096564	-0.232464929859719\\
-1.37995682626576	-0.216432865731463\\
-1.38237233139946	-0.200400801603207\\
-1.38459487613202	-0.18436873747495\\
-1.38662619559488	-0.168336673346694\\
-1.38677354709419	-0.167056987109562\\
-1.38848605043233	-0.152304609218437\\
-1.3901572568314	-0.13627254509018\\
-1.39163953165047	-0.120240480961924\\
-1.39293402425944	-0.104208416833667\\
-1.39404173713703	-0.0881763527054109\\
-1.39496352716476	-0.0721442885771544\\
-1.39570010673177	-0.0561122244488979\\
-1.3962520446525	-0.0400801603206413\\
-1.39661976689889	-0.0240480961923848\\
-1.3968035571483	-0.00801603206412826\\
-1.3968035571483	0.00801603206412782\\
-1.39661976689889	0.0240480961923843\\
-1.3962520446525	0.0400801603206409\\
-1.39570010673177	0.0561122244488974\\
-1.39496352716476	0.0721442885771539\\
-1.39404173713704	0.0881763527054105\\
-1.39293402425944	0.104208416833667\\
-1.39163953165047	0.120240480961924\\
-1.3901572568314	0.13627254509018\\
-1.38848605043233	0.152304609218437\\
-1.38677354709419	0.167056987109561\\
}--cycle;


\addplot[area legend,solid,fill=mycolor5,draw=black,forget plot]
table[row sep=crcr] {%
x	y\\
-1.1623246492986	0.157011554848384\\
-1.16076422109209	0.168336673346693\\
-1.15832961992506	0.18436873747495\\
-1.15566582917301	0.200400801603206\\
-1.15277076922552	0.216432865731463\\
-1.14964217513866	0.232464929859719\\
-1.14629258517034	0.248425833941866\\
-1.14627765936197	0.248496993987976\\
-1.14269023212082	0.264529058116232\\
-1.13886239103483	0.280561122244489\\
-1.13479111104744	0.296593186372745\\
-1.13047316530174	0.312625250501002\\
-1.13026052104208	0.313375215708044\\
-1.12592222737734	0.328657314629258\\
-1.12111937905092	0.344689378757515\\
-1.1160599413062	0.360721442885771\\
-1.11422845691383	0.366262025735924\\
-1.11075194989165	0.376753507014028\\
-1.10518628768033	0.392785571142285\\
-1.09935209868595	0.408817635270541\\
-1.09819639278557	0.411867833113997\\
-1.09325951117303	0.424849699398798\\
-1.08689297962479	0.440881763527054\\
-1.08216432865731	0.452306869103173\\
-1.08024867757908	0.456913827655311\\
-1.07332896840687	0.472945891783567\\
-1.06613226452906	0.488942444476404\\
-1.0661161976266	0.488977955911824\\
-1.05861972184843	0.50501002004008\\
-1.05081851661693	0.521042084168337\\
-1.0501002004008	0.522472649684677\\
-1.04271761875319	0.537074148296593\\
-1.03430023072212	0.55310621242485\\
-1.03406813627255	0.553535589913566\\
-1.02556722554118	0.569138276553106\\
-1.01803607214429	0.582479005420596\\
-1.01650350463642	0.585170340681363\\
-1.00710467936514	0.601202404809619\\
-1.00200400801603	0.609631714207155\\
-0.997358635855956	0.617234468937876\\
-0.987256979427527	0.633266533066132\\
-0.985971943887776	0.635252726948219\\
-0.976786414186834	0.649298597194389\\
-0.969939879759519	0.659458813615669\\
-0.965937257062677	0.665330661322646\\
-0.954697356078673	0.681362725450902\\
-0.953907815631263	0.682462010838719\\
-0.943047008045976	0.697394789579159\\
-0.937875751503006	0.704315649747757\\
-0.930976255515486	0.713426853707415\\
-0.92184368737475	0.725178113738267\\
-0.918469324409645	0.729458917835672\\
-0.905811623246493	0.745121352655245\\
-0.905508389970308	0.745490981963928\\
-0.892064810423639	0.761523046092185\\
-0.889779559118236	0.764186439056493\\
-0.878122869363835	0.777555110220441\\
-0.87374749498998	0.782462467726666\\
-0.863659352503656	0.793587174348698\\
-0.857715430861723	0.800003317979591\\
-0.848647312269042	0.809619238476954\\
-0.841683366733467	0.816854424181742\\
-0.833056770200732	0.825651302605211\\
-0.82565130260521	0.833056770200732\\
-0.816854424181742	0.841683366733467\\
-0.809619238476954	0.848647312269042\\
-0.800003317979591	0.857715430861724\\
-0.793587174348697	0.863659352503656\\
-0.782462467726666	0.87374749498998\\
-0.777555110220441	0.878122869363835\\
-0.764186439056493	0.889779559118236\\
-0.761523046092184	0.892064810423638\\
-0.745490981963928	0.905508389970308\\
-0.745121352655245	0.905811623246493\\
-0.729458917835671	0.918469324409646\\
-0.725178113738267	0.921843687374749\\
-0.713426853707415	0.930976255515487\\
-0.704315649747757	0.937875751503006\\
-0.697394789579158	0.943047008045977\\
-0.68246201083872	0.953907815631262\\
-0.681362725450902	0.954697356078673\\
-0.665330661322646	0.965937257062677\\
-0.65945881361567	0.969939879759519\\
-0.649298597194389	0.976786414186834\\
-0.63525272694822	0.985971943887775\\
-0.633266533066132	0.987256979427528\\
-0.617234468937876	0.997358635855957\\
-0.609631714207156	1.00200400801603\\
-0.601202404809619	1.00710467936514\\
-0.585170340681363	1.01650350463642\\
-0.582479005420597	1.01803607214429\\
-0.569138276553106	1.02556722554118\\
-0.553535589913567	1.03406813627254\\
-0.55310621242485	1.03430023072212\\
-0.537074148296593	1.04271761875319\\
-0.522472649684677	1.0501002004008\\
-0.521042084168337	1.05081851661693\\
-0.50501002004008	1.05861972184843\\
-0.488977955911824	1.0661161976266\\
-0.488942444476405	1.06613226452906\\
-0.472945891783567	1.07332896840687\\
-0.456913827655311	1.08024867757908\\
-0.452306869103174	1.08216432865731\\
-0.440881763527054	1.08689297962479\\
-0.424849699398798	1.09325951117303\\
-0.411867833113998	1.09819639278557\\
-0.408817635270541	1.09935209868595\\
-0.392785571142285	1.10518628768033\\
-0.376753507014028	1.11075194989165\\
-0.366262025735923	1.11422845691383\\
-0.360721442885771	1.1160599413062\\
-0.344689378757515	1.12111937905092\\
-0.328657314629258	1.12592222737734\\
-0.313375215708044	1.13026052104208\\
-0.312625250501002	1.13047316530174\\
-0.296593186372745	1.13479111104744\\
-0.280561122244489	1.13886239103483\\
-0.264529058116232	1.14269023212082\\
-0.248496993987976	1.14627765936197\\
-0.248425833941866	1.14629258517034\\
-0.232464929859719	1.14964217513866\\
-0.216432865731463	1.15277076922552\\
-0.200400801603207	1.15566582917301\\
-0.18436873747495	1.15832961992506\\
-0.168336673346694	1.16076422109209\\
-0.157011554848383	1.1623246492986\\
-0.152304609218437	1.16297466862924\\
-0.13627254509018	1.16496601772424\\
-0.120240480961924	1.16673224278246\\
-0.104208416833667	1.16827471335151\\
-0.0881763527054109	1.16959462394876\\
-0.0721442885771544	1.17069299560322\\
-0.0561122244488979	1.17157067717204\\
-0.0400801603206413	1.17222834643394\\
-0.0240480961923848	1.17266651096165\\
-0.00801603206412826	1.17288550877487\\
0.00801603206412782	1.17288550877487\\
0.0240480961923843	1.17266651096165\\
0.0400801603206409	1.17222834643394\\
0.0561122244488974	1.17157067717204\\
0.0721442885771539	1.17069299560322\\
0.0881763527054105	1.16959462394876\\
0.104208416833667	1.16827471335151\\
0.120240480961924	1.16673224278246\\
0.13627254509018	1.16496601772424\\
0.152304609218437	1.16297466862924\\
0.157011554848384	1.1623246492986\\
0.168336673346693	1.16076422109209\\
0.18436873747495	1.15832961992506\\
0.200400801603206	1.15566582917301\\
0.216432865731463	1.15277076922552\\
0.232464929859719	1.14964217513866\\
0.248425833941866	1.14629258517034\\
0.248496993987976	1.14627765936197\\
0.264529058116232	1.14269023212082\\
0.280561122244489	1.13886239103483\\
0.296593186372745	1.13479111104744\\
0.312625250501002	1.13047316530174\\
0.313375215708044	1.13026052104208\\
0.328657314629258	1.12592222737734\\
0.344689378757515	1.12111937905092\\
0.360721442885771	1.1160599413062\\
0.366262025735924	1.11422845691383\\
0.376753507014028	1.11075194989165\\
0.392785571142285	1.10518628768033\\
0.408817635270541	1.09935209868595\\
0.411867833113999	1.09819639278557\\
0.424849699398798	1.09325951117303\\
0.440881763527054	1.08689297962479\\
0.452306869103174	1.08216432865731\\
0.456913827655311	1.08024867757908\\
0.472945891783567	1.07332896840687\\
0.488942444476405	1.06613226452906\\
0.488977955911824	1.0661161976266\\
0.50501002004008	1.05861972184843\\
0.521042084168337	1.05081851661693\\
0.522472649684677	1.0501002004008\\
0.537074148296593	1.04271761875319\\
0.55310621242485	1.03430023072212\\
0.553535589913567	1.03406813627254\\
0.569138276553106	1.02556722554118\\
0.582479005420597	1.01803607214429\\
0.585170340681363	1.01650350463642\\
0.601202404809619	1.00710467936514\\
0.609631714207156	1.00200400801603\\
0.617234468937876	0.997358635855957\\
0.633266533066132	0.987256979427528\\
0.63525272694822	0.985971943887775\\
0.649298597194389	0.976786414186834\\
0.65945881361567	0.969939879759519\\
0.665330661322646	0.965937257062677\\
0.681362725450902	0.954697356078672\\
0.682462010838719	0.953907815631262\\
0.697394789579159	0.943047008045977\\
0.704315649747757	0.937875751503006\\
0.713426853707415	0.930976255515486\\
0.725178113738268	0.921843687374749\\
0.729458917835672	0.918469324409646\\
0.745121352655245	0.905811623246493\\
0.745490981963928	0.905508389970307\\
0.761523046092185	0.892064810423638\\
0.764186439056493	0.889779559118236\\
0.777555110220441	0.878122869363834\\
0.782462467726666	0.87374749498998\\
0.793587174348698	0.863659352503656\\
0.800003317979591	0.857715430861724\\
0.809619238476954	0.848647312269042\\
0.816854424181742	0.841683366733467\\
0.825651302605211	0.833056770200732\\
0.833056770200732	0.825651302605211\\
0.841683366733467	0.816854424181742\\
0.848647312269042	0.809619238476954\\
0.857715430861724	0.800003317979591\\
0.863659352503656	0.793587174348698\\
0.87374749498998	0.782462467726666\\
0.878122869363834	0.777555110220441\\
0.889779559118236	0.764186439056493\\
0.892064810423638	0.761523046092185\\
0.905508389970307	0.745490981963928\\
0.905811623246493	0.745121352655245\\
0.918469324409646	0.729458917835672\\
0.921843687374749	0.725178113738268\\
0.930976255515486	0.713426853707415\\
0.937875751503006	0.704315649747757\\
0.943047008045977	0.697394789579159\\
0.953907815631262	0.682462010838719\\
0.954697356078672	0.681362725450902\\
0.965937257062677	0.665330661322646\\
0.969939879759519	0.65945881361567\\
0.976786414186834	0.649298597194389\\
0.985971943887775	0.63525272694822\\
0.987256979427528	0.633266533066132\\
0.997358635855957	0.617234468937876\\
1.00200400801603	0.609631714207156\\
1.00710467936514	0.601202404809619\\
1.01650350463642	0.585170340681363\\
1.01803607214429	0.582479005420597\\
1.02556722554118	0.569138276553106\\
1.03406813627254	0.553535589913567\\
1.03430023072212	0.55310621242485\\
1.04271761875319	0.537074148296593\\
1.0501002004008	0.522472649684677\\
1.05081851661693	0.521042084168337\\
1.05861972184843	0.50501002004008\\
1.0661161976266	0.488977955911824\\
1.06613226452906	0.488942444476405\\
1.07332896840687	0.472945891783567\\
1.08024867757908	0.456913827655311\\
1.08216432865731	0.452306869103174\\
1.08689297962479	0.440881763527054\\
1.09325951117303	0.424849699398798\\
1.09819639278557	0.411867833113999\\
1.09935209868595	0.408817635270541\\
1.10518628768033	0.392785571142285\\
1.11075194989165	0.376753507014028\\
1.11422845691383	0.366262025735924\\
1.1160599413062	0.360721442885771\\
1.12111937905092	0.344689378757515\\
1.12592222737734	0.328657314629258\\
1.13026052104208	0.313375215708044\\
1.13047316530174	0.312625250501002\\
1.13479111104744	0.296593186372745\\
1.13886239103483	0.280561122244489\\
1.14269023212082	0.264529058116232\\
1.14627765936197	0.248496993987976\\
1.14629258517034	0.248425833941866\\
1.14964217513866	0.232464929859719\\
1.15277076922552	0.216432865731463\\
1.15566582917301	0.200400801603206\\
1.15832961992506	0.18436873747495\\
1.16076422109209	0.168336673346693\\
1.1623246492986	0.157011554848384\\
1.16297466862924	0.152304609218437\\
1.16496601772424	0.13627254509018\\
1.16673224278246	0.120240480961924\\
1.16827471335151	0.104208416833667\\
1.16959462394876	0.0881763527054105\\
1.17069299560322	0.0721442885771539\\
1.17157067717204	0.0561122244488974\\
1.17222834643394	0.0400801603206409\\
1.17266651096165	0.0240480961923843\\
1.17288550877487	0.00801603206412782\\
1.17288550877487	-0.00801603206412826\\
1.17266651096165	-0.0240480961923848\\
1.17222834643394	-0.0400801603206413\\
1.17157067717204	-0.0561122244488979\\
1.17069299560322	-0.0721442885771544\\
1.16959462394876	-0.0881763527054109\\
1.16827471335151	-0.104208416833667\\
1.16673224278246	-0.120240480961924\\
1.16496601772424	-0.13627254509018\\
1.16297466862924	-0.152304609218437\\
1.1623246492986	-0.157011554848383\\
1.16076422109209	-0.168336673346694\\
1.15832961992506	-0.18436873747495\\
1.15566582917301	-0.200400801603207\\
1.15277076922552	-0.216432865731463\\
1.14964217513866	-0.232464929859719\\
1.14629258517034	-0.248425833941866\\
1.14627765936197	-0.248496993987976\\
1.14269023212082	-0.264529058116232\\
1.13886239103483	-0.280561122244489\\
1.13479111104744	-0.296593186372745\\
1.13047316530174	-0.312625250501002\\
1.13026052104208	-0.313375215708044\\
1.12592222737734	-0.328657314629258\\
1.12111937905092	-0.344689378757515\\
1.1160599413062	-0.360721442885771\\
1.11422845691383	-0.366262025735923\\
1.11075194989165	-0.376753507014028\\
1.10518628768033	-0.392785571142285\\
1.09935209868595	-0.408817635270541\\
1.09819639278557	-0.411867833113998\\
1.09325951117303	-0.424849699398798\\
1.08689297962479	-0.440881763527054\\
1.08216432865731	-0.452306869103174\\
1.08024867757908	-0.456913827655311\\
1.07332896840687	-0.472945891783567\\
1.06613226452906	-0.488942444476405\\
1.0661161976266	-0.488977955911824\\
1.05861972184843	-0.50501002004008\\
1.05081851661693	-0.521042084168337\\
1.0501002004008	-0.522472649684677\\
1.04271761875319	-0.537074148296593\\
1.03430023072212	-0.55310621242485\\
1.03406813627254	-0.553535589913567\\
1.02556722554118	-0.569138276553106\\
1.01803607214429	-0.582479005420597\\
1.01650350463642	-0.585170340681363\\
1.00710467936514	-0.601202404809619\\
1.00200400801603	-0.609631714207156\\
0.997358635855957	-0.617234468937876\\
0.987256979427528	-0.633266533066132\\
0.985971943887775	-0.63525272694822\\
0.976786414186834	-0.649298597194389\\
0.969939879759519	-0.65945881361567\\
0.965937257062677	-0.665330661322646\\
0.954697356078673	-0.681362725450902\\
0.953907815631262	-0.68246201083872\\
0.943047008045977	-0.697394789579158\\
0.937875751503006	-0.704315649747757\\
0.930976255515487	-0.713426853707415\\
0.921843687374749	-0.725178113738267\\
0.918469324409646	-0.729458917835671\\
0.905811623246493	-0.745121352655245\\
0.905508389970308	-0.745490981963928\\
0.892064810423638	-0.761523046092184\\
0.889779559118236	-0.764186439056493\\
0.878122869363835	-0.777555110220441\\
0.87374749498998	-0.782462467726666\\
0.863659352503656	-0.793587174348697\\
0.857715430861724	-0.800003317979591\\
0.848647312269042	-0.809619238476954\\
0.841683366733467	-0.816854424181742\\
0.833056770200732	-0.82565130260521\\
0.825651302605211	-0.833056770200732\\
0.816854424181742	-0.841683366733467\\
0.809619238476954	-0.848647312269042\\
0.800003317979591	-0.857715430861723\\
0.793587174348698	-0.863659352503656\\
0.782462467726666	-0.87374749498998\\
0.777555110220441	-0.878122869363835\\
0.764186439056493	-0.889779559118236\\
0.761523046092185	-0.892064810423639\\
0.745490981963928	-0.905508389970308\\
0.745121352655245	-0.905811623246493\\
0.729458917835672	-0.918469324409645\\
0.725178113738267	-0.92184368737475\\
0.713426853707415	-0.930976255515486\\
0.704315649747757	-0.937875751503006\\
0.697394789579159	-0.943047008045976\\
0.682462010838719	-0.953907815631263\\
0.681362725450902	-0.954697356078673\\
0.665330661322646	-0.965937257062677\\
0.659458813615669	-0.969939879759519\\
0.649298597194389	-0.976786414186834\\
0.635252726948219	-0.985971943887776\\
0.633266533066132	-0.987256979427527\\
0.617234468937876	-0.997358635855956\\
0.609631714207155	-1.00200400801603\\
0.601202404809619	-1.00710467936514\\
0.585170340681363	-1.01650350463642\\
0.582479005420596	-1.01803607214429\\
0.569138276553106	-1.02556722554118\\
0.553535589913566	-1.03406813627255\\
0.55310621242485	-1.03430023072212\\
0.537074148296593	-1.04271761875319\\
0.522472649684677	-1.0501002004008\\
0.521042084168337	-1.05081851661693\\
0.50501002004008	-1.05861972184843\\
0.488977955911824	-1.0661161976266\\
0.488942444476404	-1.06613226452906\\
0.472945891783567	-1.07332896840687\\
0.456913827655311	-1.08024867757908\\
0.452306869103173	-1.08216432865731\\
0.440881763527054	-1.08689297962479\\
0.424849699398798	-1.09325951117303\\
0.411867833113997	-1.09819639278557\\
0.408817635270541	-1.09935209868595\\
0.392785571142285	-1.10518628768033\\
0.376753507014028	-1.11075194989165\\
0.366262025735924	-1.11422845691383\\
0.360721442885771	-1.1160599413062\\
0.344689378757515	-1.12111937905092\\
0.328657314629258	-1.12592222737734\\
0.313375215708044	-1.13026052104208\\
0.312625250501002	-1.13047316530174\\
0.296593186372745	-1.13479111104744\\
0.280561122244489	-1.13886239103483\\
0.264529058116232	-1.14269023212082\\
0.248496993987976	-1.14627765936197\\
0.248425833941866	-1.14629258517034\\
0.232464929859719	-1.14964217513866\\
0.216432865731463	-1.15277076922552\\
0.200400801603206	-1.15566582917301\\
0.18436873747495	-1.15832961992506\\
0.168336673346693	-1.16076422109209\\
0.157011554848384	-1.1623246492986\\
0.152304609218437	-1.16297466862924\\
0.13627254509018	-1.16496601772424\\
0.120240480961924	-1.16673224278246\\
0.104208416833667	-1.16827471335151\\
0.0881763527054105	-1.16959462394876\\
0.0721442885771539	-1.17069299560322\\
0.0561122244488974	-1.17157067717204\\
0.0400801603206409	-1.17222834643394\\
0.0240480961923843	-1.17266651096165\\
0.00801603206412782	-1.17288550877487\\
-0.00801603206412826	-1.17288550877487\\
-0.0240480961923848	-1.17266651096165\\
-0.0400801603206413	-1.17222834643394\\
-0.0561122244488979	-1.17157067717204\\
-0.0721442885771544	-1.17069299560322\\
-0.0881763527054109	-1.16959462394876\\
-0.104208416833667	-1.16827471335151\\
-0.120240480961924	-1.16673224278246\\
-0.13627254509018	-1.16496601772424\\
-0.152304609218437	-1.16297466862924\\
-0.157011554848383	-1.1623246492986\\
-0.168336673346694	-1.16076422109209\\
-0.18436873747495	-1.15832961992506\\
-0.200400801603207	-1.15566582917301\\
-0.216432865731463	-1.15277076922552\\
-0.232464929859719	-1.14964217513866\\
-0.248425833941866	-1.14629258517034\\
-0.248496993987976	-1.14627765936197\\
-0.264529058116232	-1.14269023212082\\
-0.280561122244489	-1.13886239103483\\
-0.296593186372745	-1.13479111104744\\
-0.312625250501002	-1.13047316530174\\
-0.313375215708044	-1.13026052104208\\
-0.328657314629258	-1.12592222737734\\
-0.344689378757515	-1.12111937905092\\
-0.360721442885771	-1.1160599413062\\
-0.366262025735923	-1.11422845691383\\
-0.376753507014028	-1.11075194989165\\
-0.392785571142285	-1.10518628768033\\
-0.408817635270541	-1.09935209868595\\
-0.411867833113997	-1.09819639278557\\
-0.424849699398798	-1.09325951117303\\
-0.440881763527054	-1.08689297962479\\
-0.452306869103173	-1.08216432865731\\
-0.456913827655311	-1.08024867757908\\
-0.472945891783567	-1.07332896840687\\
-0.488942444476404	-1.06613226452906\\
-0.488977955911824	-1.0661161976266\\
-0.50501002004008	-1.05861972184843\\
-0.521042084168337	-1.05081851661693\\
-0.522472649684677	-1.0501002004008\\
-0.537074148296593	-1.04271761875319\\
-0.55310621242485	-1.03430023072212\\
-0.553535589913566	-1.03406813627255\\
-0.569138276553106	-1.02556722554118\\
-0.582479005420596	-1.01803607214429\\
-0.585170340681363	-1.01650350463642\\
-0.601202404809619	-1.00710467936514\\
-0.609631714207155	-1.00200400801603\\
-0.617234468937876	-0.997358635855956\\
-0.633266533066132	-0.987256979427527\\
-0.635252726948219	-0.985971943887776\\
-0.649298597194389	-0.976786414186834\\
-0.659458813615668	-0.969939879759519\\
-0.665330661322646	-0.965937257062677\\
-0.681362725450902	-0.954697356078673\\
-0.682462010838719	-0.953907815631263\\
-0.697394789579158	-0.943047008045977\\
-0.704315649747757	-0.937875751503006\\
-0.713426853707415	-0.930976255515486\\
-0.725178113738267	-0.92184368737475\\
-0.729458917835671	-0.918469324409646\\
-0.745121352655244	-0.905811623246493\\
-0.745490981963928	-0.905508389970308\\
-0.761523046092184	-0.892064810423639\\
-0.764186439056493	-0.889779559118236\\
-0.777555110220441	-0.878122869363835\\
-0.782462467726666	-0.87374749498998\\
-0.793587174348697	-0.863659352503656\\
-0.800003317979592	-0.857715430861723\\
-0.809619238476954	-0.848647312269043\\
-0.816854424181743	-0.841683366733467\\
-0.82565130260521	-0.833056770200732\\
-0.833056770200732	-0.82565130260521\\
-0.841683366733467	-0.816854424181743\\
-0.848647312269043	-0.809619238476954\\
-0.857715430861723	-0.800003317979592\\
-0.863659352503656	-0.793587174348697\\
-0.87374749498998	-0.782462467726666\\
-0.878122869363835	-0.777555110220441\\
-0.889779559118236	-0.764186439056493\\
-0.892064810423639	-0.761523046092184\\
-0.905508389970308	-0.745490981963928\\
-0.905811623246493	-0.745121352655244\\
-0.918469324409646	-0.729458917835671\\
-0.92184368737475	-0.725178113738267\\
-0.930976255515486	-0.713426853707415\\
-0.937875751503006	-0.704315649747757\\
-0.943047008045977	-0.697394789579158\\
-0.953907815631263	-0.682462010838719\\
-0.954697356078673	-0.681362725450902\\
-0.965937257062677	-0.665330661322646\\
-0.969939879759519	-0.659458813615668\\
-0.976786414186834	-0.649298597194389\\
-0.985971943887776	-0.635252726948219\\
-0.987256979427527	-0.633266533066132\\
-0.997358635855956	-0.617234468937876\\
-1.00200400801603	-0.609631714207155\\
-1.00710467936514	-0.601202404809619\\
-1.01650350463642	-0.585170340681363\\
-1.01803607214429	-0.582479005420596\\
-1.02556722554118	-0.569138276553106\\
-1.03406813627255	-0.553535589913566\\
-1.03430023072212	-0.55310621242485\\
-1.04271761875319	-0.537074148296593\\
-1.0501002004008	-0.522472649684677\\
-1.05081851661693	-0.521042084168337\\
-1.05861972184843	-0.50501002004008\\
-1.0661161976266	-0.488977955911824\\
-1.06613226452906	-0.488942444476404\\
-1.07332896840687	-0.472945891783567\\
-1.08024867757908	-0.456913827655311\\
-1.08216432865731	-0.452306869103173\\
-1.08689297962479	-0.440881763527054\\
-1.09325951117303	-0.424849699398798\\
-1.09819639278557	-0.411867833113997\\
-1.09935209868595	-0.408817635270541\\
-1.10518628768033	-0.392785571142285\\
-1.11075194989165	-0.376753507014028\\
-1.11422845691383	-0.366262025735923\\
-1.1160599413062	-0.360721442885771\\
-1.12111937905092	-0.344689378757515\\
-1.12592222737734	-0.328657314629258\\
-1.13026052104208	-0.313375215708044\\
-1.13047316530174	-0.312625250501002\\
-1.13479111104744	-0.296593186372745\\
-1.13886239103483	-0.280561122244489\\
-1.14269023212082	-0.264529058116232\\
-1.14627765936197	-0.248496993987976\\
-1.14629258517034	-0.248425833941866\\
-1.14964217513866	-0.232464929859719\\
-1.15277076922552	-0.216432865731463\\
-1.15566582917301	-0.200400801603207\\
-1.15832961992506	-0.18436873747495\\
-1.16076422109209	-0.168336673346694\\
-1.1623246492986	-0.157011554848383\\
-1.16297466862924	-0.152304609218437\\
-1.16496601772424	-0.13627254509018\\
-1.16673224278246	-0.120240480961924\\
-1.16827471335151	-0.104208416833667\\
-1.16959462394876	-0.0881763527054109\\
-1.17069299560322	-0.0721442885771544\\
-1.17157067717204	-0.0561122244488979\\
-1.17222834643394	-0.0400801603206413\\
-1.17266651096165	-0.0240480961923848\\
-1.17288550877487	-0.00801603206412826\\
-1.17288550877487	0.00801603206412782\\
-1.17266651096165	0.0240480961923843\\
-1.17222834643394	0.0400801603206409\\
-1.17157067717204	0.0561122244488974\\
-1.17069299560322	0.0721442885771539\\
-1.16959462394876	0.0881763527054105\\
-1.16827471335151	0.104208416833667\\
-1.16673224278246	0.120240480961924\\
-1.16496601772424	0.13627254509018\\
-1.16297466862924	0.152304609218437\\
-1.1623246492986	0.157011554848384\\
}--cycle;


\addplot[area legend,solid,fill=mycolor6,draw=black,forget plot]
table[row sep=crcr] {%
x	y\\
-0.953907815631263	0.139414057102507\\
-0.951954073165497	0.152304609218437\\
-0.94924882869708	0.168336673346693\\
-0.946265017804119	0.18436873747495\\
-0.943000315572727	0.200400801603206\\
-0.939452173263398	0.216432865731463\\
-0.937875751503006	0.223038108003678\\
-0.935613161857691	0.232464929859719\\
-0.931481083144931	0.248496993987976\\
-0.92705593778845	0.264529058116232\\
-0.922334238345253	0.280561122244489\\
-0.92184368737475	0.282133695020578\\
-0.917300397232555	0.296593186372745\\
-0.911960210630024	0.312625250501002\\
-0.906310716388034	0.328657314629258\\
-0.905811623246493	0.330005753495267\\
-0.90033001628045	0.344689378757515\\
-0.894028127260021	0.360721442885771\\
-0.889779559118236	0.371022079725329\\
-0.887392633186274	0.376753507014028\\
-0.880409916040137	0.392785571142285\\
-0.87374749498998	0.40738659275849\\
-0.873087444015272	0.408817635270541\\
-0.865391650239212	0.424849699398798\\
-0.857715430861723	0.440149051670939\\
-0.857343409827083	0.440881763527054\\
-0.848898168490759	0.456913827655311\\
-0.841683366733467	0.470058964797171\\
-0.840078109872531	0.472945891783567\\
-0.830842575729169	0.488977955911824\\
-0.82565130260521	0.497662470418559\\
-0.821196629596634	0.50501002004008\\
-0.811125081109855	0.521042084168337\\
-0.809619238476954	0.523363927988226\\
-0.800590275629333	0.537074148296593\\
-0.793587174348697	0.547351710115214\\
-0.789601957786188	0.55310621242485\\
-0.778135092720253	0.569138276553106\\
-0.777555110220441	0.569926381809957\\
-0.766139790648048	0.585170340681363\\
-0.761523046092184	0.591151185536746\\
-0.753620564694188	0.601202404809619\\
-0.745490981963928	0.611245813055966\\
-0.74054892418732	0.617234468937876\\
-0.729458917835671	0.630303248777647\\
-0.72689272119104	0.633266533066132\\
-0.713426853707415	0.648405920500061\\
-0.712615747250241	0.649298597194389\\
-0.697673954629186	0.665330661322646\\
-0.697394789579158	0.665623044924786\\
-0.682023127325728	0.681362725450902\\
-0.681362725450902	0.682023127325728\\
-0.665623044924786	0.697394789579159\\
-0.665330661322646	0.697673954629185\\
-0.649298597194389	0.71261574725024\\
-0.64840592050006	0.713426853707415\\
-0.633266533066132	0.726892721191039\\
-0.630303248777646	0.729458917835672\\
-0.617234468937876	0.74054892418732\\
-0.611245813055966	0.745490981963928\\
-0.601202404809619	0.753620564694188\\
-0.591151185536745	0.761523046092185\\
-0.585170340681363	0.766139790648048\\
-0.569926381809957	0.777555110220441\\
-0.569138276553106	0.778135092720253\\
-0.55310621242485	0.789601957786188\\
-0.547351710115214	0.793587174348698\\
-0.537074148296593	0.800590275629333\\
-0.523363927988226	0.809619238476954\\
-0.521042084168337	0.811125081109856\\
-0.50501002004008	0.821196629596634\\
-0.497662470418559	0.825651302605211\\
-0.488977955911824	0.830842575729169\\
-0.472945891783567	0.84007810987253\\
-0.47005896479717	0.841683366733467\\
-0.456913827655311	0.848898168490759\\
-0.440881763527054	0.857343409827082\\
-0.440149051670938	0.857715430861724\\
-0.424849699398798	0.865391650239212\\
-0.408817635270541	0.873087444015272\\
-0.407386592758489	0.87374749498998\\
-0.392785571142285	0.880409916040137\\
-0.376753507014028	0.887392633186274\\
-0.37102207972533	0.889779559118236\\
-0.360721442885771	0.894028127260021\\
-0.344689378757515	0.90033001628045\\
-0.330005753495267	0.905811623246493\\
-0.328657314629258	0.906310716388034\\
-0.312625250501002	0.911960210630025\\
-0.296593186372745	0.917300397232555\\
-0.28213369502058	0.921843687374749\\
-0.280561122244489	0.922334238345253\\
-0.264529058116232	0.92705593778845\\
-0.248496993987976	0.931481083144931\\
-0.232464929859719	0.935613161857691\\
-0.22303810800368	0.937875751503006\\
-0.216432865731463	0.939452173263398\\
-0.200400801603207	0.943000315572727\\
-0.18436873747495	0.946265017804119\\
-0.168336673346694	0.94924882869708\\
-0.152304609218437	0.951954073165498\\
-0.139414057102511	0.953907815631262\\
-0.13627254509018	0.954382136492342\\
-0.120240480961924	0.956533082636206\\
-0.104208416833667	0.95841153584186\\
-0.0881763527054109	0.960018950817768\\
-0.0721442885771544	0.961356570994602\\
-0.0561122244488979	0.962425430128496\\
-0.0400801603206413	0.963226353632688\\
-0.0240480961923848	0.963759959639964\\
-0.00801603206412826	0.96402665979783\\
0.00801603206412782	0.96402665979783\\
0.0240480961923843	0.963759959639964\\
0.0400801603206409	0.963226353632688\\
0.0561122244488974	0.962425430128496\\
0.0721442885771539	0.961356570994602\\
0.0881763527054105	0.960018950817768\\
0.104208416833667	0.958411535841861\\
0.120240480961924	0.956533082636206\\
0.13627254509018	0.954382136492342\\
0.139414057102511	0.953907815631262\\
0.152304609218437	0.951954073165498\\
0.168336673346693	0.94924882869708\\
0.18436873747495	0.946265017804119\\
0.200400801603206	0.943000315572727\\
0.216432865731463	0.939452173263398\\
0.22303810800368	0.937875751503006\\
0.232464929859719	0.935613161857691\\
0.248496993987976	0.931481083144931\\
0.264529058116232	0.92705593778845\\
0.280561122244489	0.922334238345253\\
0.28213369502058	0.921843687374749\\
0.296593186372745	0.917300397232555\\
0.312625250501002	0.911960210630025\\
0.328657314629258	0.906310716388034\\
0.330005753495267	0.905811623246493\\
0.344689378757515	0.90033001628045\\
0.360721442885771	0.894028127260021\\
0.37102207972533	0.889779559118236\\
0.376753507014028	0.887392633186274\\
0.392785571142285	0.880409916040137\\
0.407386592758489	0.87374749498998\\
0.408817635270541	0.873087444015272\\
0.424849699398798	0.865391650239212\\
0.440149051670938	0.857715430861724\\
0.440881763527054	0.857343409827082\\
0.456913827655311	0.848898168490759\\
0.47005896479717	0.841683366733467\\
0.472945891783567	0.84007810987253\\
0.488977955911824	0.830842575729169\\
0.497662470418559	0.825651302605211\\
0.50501002004008	0.821196629596634\\
0.521042084168337	0.811125081109856\\
0.523363927988226	0.809619238476954\\
0.537074148296593	0.800590275629333\\
0.547351710115213	0.793587174348698\\
0.55310621242485	0.789601957786188\\
0.569138276553106	0.778135092720253\\
0.569926381809957	0.777555110220441\\
0.585170340681363	0.766139790648048\\
0.591151185536745	0.761523046092185\\
0.601202404809619	0.753620564694188\\
0.611245813055966	0.745490981963928\\
0.617234468937876	0.74054892418732\\
0.630303248777646	0.729458917835672\\
0.633266533066132	0.726892721191039\\
0.64840592050006	0.713426853707415\\
0.649298597194389	0.71261574725024\\
0.665330661322646	0.697673954629185\\
0.665623044924786	0.697394789579159\\
0.681362725450902	0.682023127325728\\
0.682023127325728	0.681362725450902\\
0.697394789579159	0.665623044924786\\
0.697673954629185	0.665330661322646\\
0.71261574725024	0.649298597194389\\
0.713426853707415	0.64840592050006\\
0.726892721191039	0.633266533066132\\
0.729458917835672	0.630303248777646\\
0.74054892418732	0.617234468937876\\
0.745490981963928	0.611245813055966\\
0.753620564694188	0.601202404809619\\
0.761523046092185	0.591151185536745\\
0.766139790648048	0.585170340681363\\
0.777555110220441	0.569926381809957\\
0.778135092720253	0.569138276553106\\
0.789601957786188	0.55310621242485\\
0.793587174348698	0.547351710115213\\
0.800590275629333	0.537074148296593\\
0.809619238476954	0.523363927988226\\
0.811125081109856	0.521042084168337\\
0.821196629596634	0.50501002004008\\
0.825651302605211	0.497662470418559\\
0.830842575729169	0.488977955911824\\
0.84007810987253	0.472945891783567\\
0.841683366733467	0.47005896479717\\
0.848898168490759	0.456913827655311\\
0.857343409827082	0.440881763527054\\
0.857715430861724	0.440149051670938\\
0.865391650239212	0.424849699398798\\
0.873087444015272	0.408817635270541\\
0.87374749498998	0.407386592758489\\
0.880409916040137	0.392785571142285\\
0.887392633186274	0.376753507014028\\
0.889779559118236	0.37102207972533\\
0.894028127260021	0.360721442885771\\
0.90033001628045	0.344689378757515\\
0.905811623246493	0.330005753495267\\
0.906310716388034	0.328657314629258\\
0.911960210630025	0.312625250501002\\
0.917300397232555	0.296593186372745\\
0.921843687374749	0.28213369502058\\
0.922334238345253	0.280561122244489\\
0.92705593778845	0.264529058116232\\
0.931481083144931	0.248496993987976\\
0.935613161857691	0.232464929859719\\
0.937875751503006	0.22303810800368\\
0.939452173263398	0.216432865731463\\
0.943000315572727	0.200400801603206\\
0.946265017804119	0.18436873747495\\
0.94924882869708	0.168336673346693\\
0.951954073165498	0.152304609218437\\
0.953907815631262	0.139414057102511\\
0.954382136492342	0.13627254509018\\
0.956533082636206	0.120240480961924\\
0.958411535841861	0.104208416833667\\
0.960018950817768	0.0881763527054105\\
0.961356570994602	0.0721442885771539\\
0.962425430128496	0.0561122244488974\\
0.963226353632688	0.0400801603206409\\
0.963759959639964	0.0240480961923843\\
0.96402665979783	0.00801603206412782\\
0.96402665979783	-0.00801603206412826\\
0.963759959639964	-0.0240480961923848\\
0.963226353632688	-0.0400801603206413\\
0.962425430128496	-0.0561122244488979\\
0.961356570994602	-0.0721442885771544\\
0.960018950817768	-0.0881763527054109\\
0.95841153584186	-0.104208416833667\\
0.956533082636206	-0.120240480961924\\
0.954382136492342	-0.13627254509018\\
0.953907815631262	-0.139414057102511\\
0.951954073165498	-0.152304609218437\\
0.94924882869708	-0.168336673346694\\
0.946265017804119	-0.18436873747495\\
0.943000315572727	-0.200400801603207\\
0.939452173263398	-0.216432865731463\\
0.937875751503006	-0.22303810800368\\
0.935613161857691	-0.232464929859719\\
0.931481083144931	-0.248496993987976\\
0.92705593778845	-0.264529058116232\\
0.922334238345253	-0.280561122244489\\
0.921843687374749	-0.28213369502058\\
0.917300397232555	-0.296593186372745\\
0.911960210630025	-0.312625250501002\\
0.906310716388034	-0.328657314629258\\
0.905811623246493	-0.330005753495267\\
0.90033001628045	-0.344689378757515\\
0.894028127260021	-0.360721442885771\\
0.889779559118236	-0.37102207972533\\
0.887392633186274	-0.376753507014028\\
0.880409916040137	-0.392785571142285\\
0.87374749498998	-0.407386592758489\\
0.873087444015272	-0.408817635270541\\
0.865391650239212	-0.424849699398798\\
0.857715430861724	-0.440149051670938\\
0.857343409827082	-0.440881763527054\\
0.848898168490759	-0.456913827655311\\
0.841683366733467	-0.47005896479717\\
0.84007810987253	-0.472945891783567\\
0.830842575729169	-0.488977955911824\\
0.825651302605211	-0.497662470418559\\
0.821196629596634	-0.50501002004008\\
0.811125081109856	-0.521042084168337\\
0.809619238476954	-0.523363927988226\\
0.800590275629333	-0.537074148296593\\
0.793587174348698	-0.547351710115214\\
0.789601957786188	-0.55310621242485\\
0.778135092720253	-0.569138276553106\\
0.777555110220441	-0.569926381809957\\
0.766139790648048	-0.585170340681363\\
0.761523046092185	-0.591151185536745\\
0.753620564694188	-0.601202404809619\\
0.745490981963928	-0.611245813055966\\
0.74054892418732	-0.617234468937876\\
0.729458917835672	-0.630303248777646\\
0.726892721191039	-0.633266533066132\\
0.713426853707415	-0.64840592050006\\
0.71261574725024	-0.649298597194389\\
0.697673954629185	-0.665330661322646\\
0.697394789579159	-0.665623044924786\\
0.682023127325728	-0.681362725450902\\
0.681362725450902	-0.682023127325728\\
0.665623044924786	-0.697394789579158\\
0.665330661322646	-0.697673954629186\\
0.649298597194389	-0.712615747250241\\
0.648405920500061	-0.713426853707415\\
0.633266533066132	-0.72689272119104\\
0.630303248777647	-0.729458917835671\\
0.617234468937876	-0.74054892418732\\
0.611245813055966	-0.745490981963928\\
0.601202404809619	-0.753620564694188\\
0.591151185536746	-0.761523046092184\\
0.585170340681363	-0.766139790648048\\
0.569926381809957	-0.777555110220441\\
0.569138276553106	-0.778135092720253\\
0.55310621242485	-0.789601957786188\\
0.547351710115214	-0.793587174348697\\
0.537074148296593	-0.800590275629333\\
0.523363927988226	-0.809619238476954\\
0.521042084168337	-0.811125081109855\\
0.50501002004008	-0.821196629596634\\
0.497662470418559	-0.82565130260521\\
0.488977955911824	-0.830842575729169\\
0.472945891783567	-0.840078109872531\\
0.470058964797171	-0.841683366733467\\
0.456913827655311	-0.848898168490759\\
0.440881763527054	-0.857343409827083\\
0.440149051670939	-0.857715430861723\\
0.424849699398798	-0.865391650239212\\
0.408817635270541	-0.873087444015272\\
0.40738659275849	-0.87374749498998\\
0.392785571142285	-0.880409916040137\\
0.376753507014028	-0.887392633186274\\
0.371022079725329	-0.889779559118236\\
0.360721442885771	-0.894028127260021\\
0.344689378757515	-0.90033001628045\\
0.330005753495267	-0.905811623246493\\
0.328657314629258	-0.906310716388034\\
0.312625250501002	-0.911960210630024\\
0.296593186372745	-0.917300397232555\\
0.282133695020578	-0.92184368737475\\
0.280561122244489	-0.922334238345253\\
0.264529058116232	-0.92705593778845\\
0.248496993987976	-0.931481083144931\\
0.232464929859719	-0.935613161857691\\
0.223038108003678	-0.937875751503006\\
0.216432865731463	-0.939452173263398\\
0.200400801603206	-0.943000315572727\\
0.18436873747495	-0.946265017804119\\
0.168336673346693	-0.94924882869708\\
0.152304609218437	-0.951954073165497\\
0.139414057102507	-0.953907815631263\\
0.13627254509018	-0.954382136492342\\
0.120240480961924	-0.956533082636207\\
0.104208416833667	-0.958411535841861\\
0.0881763527054105	-0.960018950817768\\
0.0721442885771539	-0.961356570994601\\
0.0561122244488974	-0.962425430128497\\
0.0400801603206409	-0.963226353632688\\
0.0240480961923843	-0.963759959639963\\
0.00801603206412782	-0.96402665979783\\
-0.00801603206412826	-0.96402665979783\\
-0.0240480961923848	-0.963759959639963\\
-0.0400801603206413	-0.963226353632688\\
-0.0561122244488979	-0.962425430128497\\
-0.0721442885771544	-0.961356570994601\\
-0.0881763527054109	-0.960018950817768\\
-0.104208416833667	-0.958411535841861\\
-0.120240480961924	-0.956533082636206\\
-0.13627254509018	-0.954382136492342\\
-0.139414057102508	-0.953907815631263\\
-0.152304609218437	-0.951954073165497\\
-0.168336673346694	-0.94924882869708\\
-0.18436873747495	-0.946265017804119\\
-0.200400801603207	-0.943000315572727\\
-0.216432865731463	-0.939452173263398\\
-0.223038108003678	-0.937875751503006\\
-0.232464929859719	-0.935613161857691\\
-0.248496993987976	-0.931481083144931\\
-0.264529058116232	-0.92705593778845\\
-0.280561122244489	-0.922334238345253\\
-0.282133695020577	-0.92184368737475\\
-0.296593186372745	-0.917300397232555\\
-0.312625250501002	-0.911960210630024\\
-0.328657314629258	-0.906310716388034\\
-0.330005753495267	-0.905811623246493\\
-0.344689378757515	-0.90033001628045\\
-0.360721442885771	-0.894028127260021\\
-0.371022079725329	-0.889779559118236\\
-0.376753507014028	-0.887392633186274\\
-0.392785571142285	-0.880409916040137\\
-0.40738659275849	-0.87374749498998\\
-0.408817635270541	-0.873087444015272\\
-0.424849699398798	-0.865391650239212\\
-0.440149051670939	-0.857715430861723\\
-0.440881763527054	-0.857343409827083\\
-0.456913827655311	-0.848898168490759\\
-0.470058964797171	-0.841683366733467\\
-0.472945891783567	-0.840078109872531\\
-0.488977955911824	-0.830842575729169\\
-0.497662470418559	-0.82565130260521\\
-0.50501002004008	-0.821196629596634\\
-0.521042084168337	-0.811125081109855\\
-0.523363927988226	-0.809619238476954\\
-0.537074148296593	-0.800590275629333\\
-0.547351710115215	-0.793587174348697\\
-0.55310621242485	-0.789601957786188\\
-0.569138276553106	-0.778135092720253\\
-0.569926381809957	-0.777555110220441\\
-0.585170340681363	-0.766139790648048\\
-0.591151185536746	-0.761523046092184\\
-0.601202404809619	-0.753620564694188\\
-0.611245813055966	-0.745490981963928\\
-0.617234468937876	-0.74054892418732\\
-0.630303248777647	-0.729458917835671\\
-0.633266533066132	-0.72689272119104\\
-0.648405920500061	-0.713426853707415\\
-0.649298597194389	-0.712615747250241\\
-0.665330661322646	-0.697673954629186\\
-0.665623044924786	-0.697394789579158\\
-0.681362725450902	-0.682023127325728\\
-0.682023127325728	-0.681362725450902\\
-0.697394789579158	-0.665623044924786\\
-0.697673954629186	-0.665330661322646\\
-0.712615747250241	-0.649298597194389\\
-0.713426853707415	-0.648405920500061\\
-0.72689272119104	-0.633266533066132\\
-0.729458917835671	-0.630303248777647\\
-0.74054892418732	-0.617234468937876\\
-0.745490981963928	-0.611245813055966\\
-0.753620564694188	-0.601202404809619\\
-0.761523046092184	-0.591151185536746\\
-0.766139790648048	-0.585170340681363\\
-0.777555110220441	-0.569926381809957\\
-0.778135092720253	-0.569138276553106\\
-0.789601957786188	-0.55310621242485\\
-0.793587174348697	-0.547351710115215\\
-0.800590275629333	-0.537074148296593\\
-0.809619238476954	-0.523363927988226\\
-0.811125081109855	-0.521042084168337\\
-0.821196629596634	-0.50501002004008\\
-0.82565130260521	-0.497662470418559\\
-0.830842575729169	-0.488977955911824\\
-0.840078109872531	-0.472945891783567\\
-0.841683366733467	-0.470058964797171\\
-0.848898168490759	-0.456913827655311\\
-0.857343409827083	-0.440881763527054\\
-0.857715430861723	-0.440149051670939\\
-0.865391650239212	-0.424849699398798\\
-0.873087444015272	-0.408817635270541\\
-0.87374749498998	-0.40738659275849\\
-0.880409916040137	-0.392785571142285\\
-0.887392633186274	-0.376753507014028\\
-0.889779559118236	-0.371022079725329\\
-0.894028127260021	-0.360721442885771\\
-0.90033001628045	-0.344689378757515\\
-0.905811623246493	-0.330005753495267\\
-0.906310716388034	-0.328657314629258\\
-0.911960210630024	-0.312625250501002\\
-0.917300397232555	-0.296593186372745\\
-0.92184368737475	-0.282133695020577\\
-0.922334238345253	-0.280561122244489\\
-0.92705593778845	-0.264529058116232\\
-0.931481083144931	-0.248496993987976\\
-0.935613161857691	-0.232464929859719\\
-0.937875751503006	-0.223038108003678\\
-0.939452173263398	-0.216432865731463\\
-0.943000315572727	-0.200400801603207\\
-0.946265017804119	-0.18436873747495\\
-0.94924882869708	-0.168336673346694\\
-0.951954073165497	-0.152304609218437\\
-0.953907815631263	-0.139414057102508\\
-0.954382136492342	-0.13627254509018\\
-0.956533082636206	-0.120240480961924\\
-0.958411535841861	-0.104208416833667\\
-0.960018950817768	-0.0881763527054109\\
-0.961356570994601	-0.0721442885771544\\
-0.962425430128497	-0.0561122244488979\\
-0.963226353632688	-0.0400801603206413\\
-0.963759959639963	-0.0240480961923848\\
-0.96402665979783	-0.00801603206412826\\
-0.96402665979783	0.00801603206412782\\
-0.963759959639963	0.0240480961923843\\
-0.963226353632688	0.0400801603206409\\
-0.962425430128497	0.0561122244488974\\
-0.961356570994601	0.0721442885771539\\
-0.960018950817768	0.0881763527054105\\
-0.958411535841861	0.104208416833667\\
-0.956533082636207	0.120240480961924\\
-0.954382136492342	0.13627254509018\\
-0.953907815631263	0.139414057102507\\
}--cycle;


\addplot[area legend,solid,fill=mycolor7,draw=black,forget plot]
table[row sep=crcr] {%
x	y\\
-0.745490981963928	0.0946302843170873\\
-0.744248073158869	0.104208416833667\\
-0.7418183838588	0.120240480961924\\
-0.739036237987377	0.13627254509018\\
-0.735899478242375	0.152304609218437\\
-0.732405668826911	0.168336673346693\\
-0.729458917835671	0.18060421452615\\
-0.728542717995451	0.18436873747495\\
-0.724282785292518	0.200400801603206\\
-0.719653007019685	0.216432865731463\\
-0.71464976107984	0.232464929859719\\
-0.713426853707415	0.236119552901773\\
-0.7092229879799	0.248496993987976\\
-0.703396789908915	0.264529058116232\\
-0.697394789579158	0.280009907862129\\
-0.697177595485539	0.280561122244489\\
-0.69048704685053	0.296593186372745\\
-0.68339113939529	0.312625250501002\\
-0.681362725450902	0.316973486297228\\
-0.675814726603702	0.328657314629258\\
-0.667790301361433	0.344689378757515\\
-0.665330661322646	0.349373018034459\\
-0.659256133332064	0.360721442885771\\
-0.650247328500351	0.376753507014028\\
-0.649298597194389	0.378370724212566\\
-0.640667479122466	0.392785571142285\\
-0.633266533066132	0.404604311616511\\
-0.630570546188811	0.408817635270541\\
-0.619882292787882	0.424849699398798\\
-0.617234468937876	0.428668448511858\\
-0.608568863082977	0.440881763527054\\
-0.601202404809619	0.450866278215741\\
-0.596631178517965	0.456913827655311\\
-0.585170340681363	0.471521271424407\\
-0.584023763193007	0.472945891783567\\
-0.570665146802879	0.488977955911824\\
-0.569138276553106	0.490749958194447\\
-0.556514436245467	0.50501002004008\\
-0.55310621242485	0.508736329985527\\
-0.541527306818175	0.521042084168337\\
-0.537074148296593	0.525629598138159\\
-0.52562959813816	0.537074148296593\\
-0.521042084168337	0.541527306818175\\
-0.508736329985527	0.55310621242485\\
-0.50501002004008	0.556514436245467\\
-0.490749958194447	0.569138276553106\\
-0.488977955911824	0.570665146802878\\
-0.472945891783567	0.584023763193007\\
-0.471521271424407	0.585170340681363\\
-0.456913827655311	0.596631178517966\\
-0.450866278215741	0.601202404809619\\
-0.440881763527054	0.608568863082978\\
-0.428668448511858	0.617234468937876\\
-0.424849699398798	0.619882292787882\\
-0.408817635270541	0.630570546188811\\
-0.404604311616511	0.633266533066132\\
-0.392785571142285	0.640667479122466\\
-0.378370724212566	0.649298597194389\\
-0.376753507014028	0.650247328500351\\
-0.360721442885771	0.659256133332064\\
-0.349373018034459	0.665330661322646\\
-0.344689378757515	0.667790301361433\\
-0.328657314629258	0.675814726603702\\
-0.316973486297227	0.681362725450902\\
-0.312625250501002	0.68339113939529\\
-0.296593186372745	0.69048704685053\\
-0.280561122244489	0.697177595485538\\
-0.280009907862128	0.697394789579159\\
-0.264529058116232	0.703396789908914\\
-0.248496993987976	0.7092229879799\\
-0.236119552901772	0.713426853707415\\
-0.232464929859719	0.71464976107984\\
-0.216432865731463	0.719653007019685\\
-0.200400801603207	0.724282785292518\\
-0.18436873747495	0.728542717995451\\
-0.180604214526149	0.729458917835672\\
-0.168336673346694	0.732405668826911\\
-0.152304609218437	0.735899478242375\\
-0.13627254509018	0.739036237987377\\
-0.120240480961924	0.7418183838588\\
-0.104208416833667	0.744248073158869\\
-0.0946302843170828	0.745490981963928\\
-0.0881763527054109	0.746319235837106\\
-0.0721442885771544	0.748032931182374\\
-0.0561122244488979	0.749402302663327\\
-0.0400801603206413	0.75042840776014\\
-0.0240480961923848	0.751112038395138\\
-0.00801603206412826	0.751453721951527\\
0.00801603206412782	0.751453721951527\\
0.0240480961923843	0.751112038395138\\
0.0400801603206409	0.75042840776014\\
0.0561122244488974	0.749402302663327\\
0.0721442885771539	0.748032931182374\\
0.0881763527054105	0.746319235837106\\
0.0946302843170819	0.745490981963928\\
0.104208416833667	0.744248073158869\\
0.120240480961924	0.7418183838588\\
0.13627254509018	0.739036237987377\\
0.152304609218437	0.735899478242375\\
0.168336673346693	0.732405668826912\\
0.180604214526148	0.729458917835672\\
0.18436873747495	0.728542717995451\\
0.200400801603206	0.724282785292518\\
0.216432865731463	0.719653007019685\\
0.232464929859719	0.71464976107984\\
0.236119552901772	0.713426853707415\\
0.248496993987976	0.7092229879799\\
0.264529058116232	0.703396789908914\\
0.280009907862128	0.697394789579159\\
0.280561122244489	0.697177595485538\\
0.296593186372745	0.69048704685053\\
0.312625250501002	0.68339113939529\\
0.316973486297227	0.681362725450902\\
0.328657314629258	0.675814726603702\\
0.344689378757515	0.667790301361433\\
0.349373018034459	0.665330661322646\\
0.360721442885771	0.659256133332064\\
0.376753507014028	0.650247328500351\\
0.378370724212566	0.649298597194389\\
0.392785571142285	0.640667479122466\\
0.404604311616511	0.633266533066132\\
0.408817635270541	0.630570546188811\\
0.424849699398798	0.619882292787882\\
0.428668448511858	0.617234468937876\\
0.440881763527054	0.608568863082978\\
0.450866278215741	0.601202404809619\\
0.456913827655311	0.596631178517966\\
0.471521271424407	0.585170340681363\\
0.472945891783567	0.584023763193007\\
0.488977955911824	0.570665146802878\\
0.490749958194447	0.569138276553106\\
0.50501002004008	0.556514436245467\\
0.508736329985527	0.55310621242485\\
0.521042084168337	0.541527306818175\\
0.525629598138159	0.537074148296593\\
0.537074148296593	0.525629598138159\\
0.541527306818175	0.521042084168337\\
0.55310621242485	0.508736329985527\\
0.556514436245467	0.50501002004008\\
0.569138276553106	0.490749958194447\\
0.570665146802878	0.488977955911824\\
0.584023763193007	0.472945891783567\\
0.585170340681363	0.471521271424407\\
0.596631178517966	0.456913827655311\\
0.601202404809619	0.450866278215741\\
0.608568863082978	0.440881763527054\\
0.617234468937876	0.428668448511858\\
0.619882292787882	0.424849699398798\\
0.630570546188811	0.408817635270541\\
0.633266533066132	0.404604311616511\\
0.640667479122466	0.392785571142285\\
0.649298597194389	0.378370724212566\\
0.650247328500351	0.376753507014028\\
0.659256133332064	0.360721442885771\\
0.665330661322646	0.349373018034459\\
0.667790301361433	0.344689378757515\\
0.675814726603702	0.328657314629258\\
0.681362725450902	0.316973486297227\\
0.68339113939529	0.312625250501002\\
0.69048704685053	0.296593186372745\\
0.697177595485538	0.280561122244489\\
0.697394789579159	0.280009907862128\\
0.703396789908914	0.264529058116232\\
0.7092229879799	0.248496993987976\\
0.713426853707415	0.236119552901772\\
0.71464976107984	0.232464929859719\\
0.719653007019685	0.216432865731463\\
0.724282785292518	0.200400801603206\\
0.728542717995451	0.18436873747495\\
0.729458917835672	0.180604214526148\\
0.732405668826912	0.168336673346693\\
0.735899478242375	0.152304609218437\\
0.739036237987377	0.13627254509018\\
0.7418183838588	0.120240480961924\\
0.744248073158869	0.104208416833667\\
0.745490981963928	0.0946302843170819\\
0.746319235837106	0.0881763527054105\\
0.748032931182374	0.0721442885771539\\
0.749402302663327	0.0561122244488974\\
0.75042840776014	0.0400801603206409\\
0.751112038395138	0.0240480961923843\\
0.751453721951527	0.00801603206412782\\
0.751453721951527	-0.00801603206412826\\
0.751112038395138	-0.0240480961923848\\
0.75042840776014	-0.0400801603206413\\
0.749402302663327	-0.0561122244488979\\
0.748032931182374	-0.0721442885771544\\
0.746319235837106	-0.0881763527054109\\
0.745490981963928	-0.0946302843170828\\
0.744248073158869	-0.104208416833667\\
0.7418183838588	-0.120240480961924\\
0.739036237987377	-0.13627254509018\\
0.735899478242375	-0.152304609218437\\
0.732405668826911	-0.168336673346694\\
0.729458917835672	-0.180604214526149\\
0.728542717995451	-0.18436873747495\\
0.724282785292518	-0.200400801603207\\
0.719653007019685	-0.216432865731463\\
0.71464976107984	-0.232464929859719\\
0.713426853707415	-0.236119552901772\\
0.7092229879799	-0.248496993987976\\
0.703396789908914	-0.264529058116232\\
0.697394789579159	-0.280009907862128\\
0.697177595485538	-0.280561122244489\\
0.69048704685053	-0.296593186372745\\
0.68339113939529	-0.312625250501002\\
0.681362725450902	-0.316973486297227\\
0.675814726603702	-0.328657314629258\\
0.667790301361433	-0.344689378757515\\
0.665330661322646	-0.349373018034459\\
0.659256133332064	-0.360721442885771\\
0.650247328500351	-0.376753507014028\\
0.649298597194389	-0.378370724212566\\
0.640667479122466	-0.392785571142285\\
0.633266533066132	-0.404604311616511\\
0.630570546188811	-0.408817635270541\\
0.619882292787882	-0.424849699398798\\
0.617234468937876	-0.428668448511858\\
0.608568863082978	-0.440881763527054\\
0.601202404809619	-0.450866278215741\\
0.596631178517966	-0.456913827655311\\
0.585170340681363	-0.471521271424407\\
0.584023763193007	-0.472945891783567\\
0.570665146802878	-0.488977955911824\\
0.569138276553106	-0.490749958194447\\
0.556514436245467	-0.50501002004008\\
0.55310621242485	-0.508736329985527\\
0.541527306818175	-0.521042084168337\\
0.537074148296593	-0.52562959813816\\
0.525629598138159	-0.537074148296593\\
0.521042084168337	-0.541527306818175\\
0.508736329985527	-0.55310621242485\\
0.50501002004008	-0.556514436245467\\
0.490749958194447	-0.569138276553106\\
0.488977955911824	-0.570665146802879\\
0.472945891783567	-0.584023763193007\\
0.471521271424407	-0.585170340681363\\
0.456913827655311	-0.596631178517965\\
0.450866278215741	-0.601202404809619\\
0.440881763527054	-0.608568863082977\\
0.428668448511858	-0.617234468937876\\
0.424849699398798	-0.619882292787882\\
0.408817635270541	-0.630570546188811\\
0.404604311616511	-0.633266533066132\\
0.392785571142285	-0.640667479122466\\
0.378370724212566	-0.649298597194389\\
0.376753507014028	-0.650247328500351\\
0.360721442885771	-0.659256133332064\\
0.349373018034459	-0.665330661322646\\
0.344689378757515	-0.667790301361433\\
0.328657314629258	-0.675814726603702\\
0.316973486297228	-0.681362725450902\\
0.312625250501002	-0.68339113939529\\
0.296593186372745	-0.69048704685053\\
0.280561122244489	-0.697177595485539\\
0.280009907862129	-0.697394789579158\\
0.264529058116232	-0.703396789908915\\
0.248496993987976	-0.7092229879799\\
0.236119552901773	-0.713426853707415\\
0.232464929859719	-0.71464976107984\\
0.216432865731463	-0.719653007019685\\
0.200400801603206	-0.724282785292518\\
0.18436873747495	-0.728542717995451\\
0.18060421452615	-0.729458917835671\\
0.168336673346693	-0.732405668826911\\
0.152304609218437	-0.735899478242375\\
0.13627254509018	-0.739036237987377\\
0.120240480961924	-0.7418183838588\\
0.104208416833667	-0.744248073158869\\
0.0946302843170873	-0.745490981963928\\
0.0881763527054105	-0.746319235837107\\
0.0721442885771539	-0.748032931182375\\
0.0561122244488974	-0.749402302663327\\
0.0400801603206409	-0.750428407760139\\
0.0240480961923843	-0.751112038395138\\
0.00801603206412782	-0.751453721951527\\
-0.00801603206412826	-0.751453721951527\\
-0.0240480961923848	-0.751112038395138\\
-0.0400801603206413	-0.750428407760139\\
-0.0561122244488979	-0.749402302663327\\
-0.0721442885771544	-0.748032931182375\\
-0.0881763527054109	-0.746319235837107\\
-0.0946302843170873	-0.745490981963928\\
-0.104208416833667	-0.744248073158869\\
-0.120240480961924	-0.7418183838588\\
-0.13627254509018	-0.739036237987377\\
-0.152304609218437	-0.735899478242375\\
-0.168336673346694	-0.732405668826911\\
-0.180604214526151	-0.729458917835671\\
-0.18436873747495	-0.728542717995451\\
-0.200400801603207	-0.724282785292518\\
-0.216432865731463	-0.719653007019685\\
-0.232464929859719	-0.71464976107984\\
-0.236119552901774	-0.713426853707415\\
-0.248496993987976	-0.7092229879799\\
-0.264529058116232	-0.703396789908915\\
-0.280009907862129	-0.697394789579158\\
-0.280561122244489	-0.697177595485539\\
-0.296593186372745	-0.69048704685053\\
-0.312625250501002	-0.68339113939529\\
-0.316973486297228	-0.681362725450902\\
-0.328657314629258	-0.675814726603702\\
-0.344689378757515	-0.667790301361433\\
-0.349373018034459	-0.665330661322646\\
-0.360721442885771	-0.659256133332064\\
-0.376753507014028	-0.650247328500351\\
-0.378370724212566	-0.649298597194389\\
-0.392785571142285	-0.640667479122466\\
-0.404604311616511	-0.633266533066132\\
-0.408817635270541	-0.630570546188811\\
-0.424849699398798	-0.619882292787882\\
-0.428668448511858	-0.617234468937876\\
-0.440881763527054	-0.608568863082977\\
-0.450866278215741	-0.601202404809619\\
-0.456913827655311	-0.596631178517965\\
-0.471521271424407	-0.585170340681363\\
-0.472945891783567	-0.584023763193007\\
-0.488977955911824	-0.570665146802879\\
-0.490749958194447	-0.569138276553106\\
-0.50501002004008	-0.556514436245467\\
-0.508736329985527	-0.55310621242485\\
-0.521042084168337	-0.541527306818175\\
-0.52562959813816	-0.537074148296593\\
-0.537074148296593	-0.52562959813816\\
-0.541527306818175	-0.521042084168337\\
-0.55310621242485	-0.508736329985527\\
-0.556514436245467	-0.50501002004008\\
-0.569138276553106	-0.490749958194447\\
-0.570665146802879	-0.488977955911824\\
-0.584023763193007	-0.472945891783567\\
-0.585170340681363	-0.471521271424407\\
-0.596631178517965	-0.456913827655311\\
-0.601202404809619	-0.450866278215741\\
-0.608568863082977	-0.440881763527054\\
-0.617234468937876	-0.428668448511858\\
-0.619882292787882	-0.424849699398798\\
-0.630570546188811	-0.408817635270541\\
-0.633266533066132	-0.404604311616511\\
-0.640667479122466	-0.392785571142285\\
-0.649298597194389	-0.378370724212566\\
-0.650247328500351	-0.376753507014028\\
-0.659256133332064	-0.360721442885771\\
-0.665330661322646	-0.349373018034459\\
-0.667790301361433	-0.344689378757515\\
-0.675814726603702	-0.328657314629258\\
-0.681362725450902	-0.316973486297228\\
-0.68339113939529	-0.312625250501002\\
-0.69048704685053	-0.296593186372745\\
-0.697177595485539	-0.280561122244489\\
-0.697394789579158	-0.280009907862129\\
-0.703396789908915	-0.264529058116232\\
-0.7092229879799	-0.248496993987976\\
-0.713426853707415	-0.236119552901774\\
-0.71464976107984	-0.232464929859719\\
-0.719653007019685	-0.216432865731463\\
-0.724282785292518	-0.200400801603207\\
-0.728542717995451	-0.18436873747495\\
-0.729458917835671	-0.180604214526151\\
-0.732405668826911	-0.168336673346694\\
-0.735899478242375	-0.152304609218437\\
-0.739036237987377	-0.13627254509018\\
-0.7418183838588	-0.120240480961924\\
-0.744248073158869	-0.104208416833667\\
-0.745490981963928	-0.0946302843170873\\
-0.746319235837107	-0.0881763527054109\\
-0.748032931182375	-0.0721442885771544\\
-0.749402302663327	-0.0561122244488979\\
-0.750428407760139	-0.0400801603206413\\
-0.751112038395138	-0.0240480961923848\\
-0.751453721951527	-0.00801603206412826\\
-0.751453721951527	0.00801603206412782\\
-0.751112038395138	0.0240480961923843\\
-0.750428407760139	0.0400801603206409\\
-0.749402302663327	0.0561122244488974\\
-0.748032931182375	0.0721442885771539\\
-0.746319235837107	0.0881763527054105\\
-0.745490981963928	0.0946302843170873\\
}--cycle;


\addplot[area legend,solid,fill=mycolor8,draw=black,forget plot]
table[row sep=crcr] {%
x	y\\
-0.50501002004008	0.0374304203479412\\
-0.504839705452896	0.0400801603206409\\
-0.503293660866937	0.0561122244488974\\
-0.501230412793354	0.0721442885771539\\
-0.49864836791436	0.0881763527054105\\
-0.495545530742166	0.104208416833667\\
-0.491919501048469	0.120240480961924\\
-0.488977955911824	0.131605354392381\\
-0.487736888422236	0.13627254509018\\
-0.482937368408193	0.152304609218437\\
-0.477591530799338	0.168336673346693\\
-0.472945891783567	0.180975725896018\\
-0.471661863650401	0.18436873747495\\
-0.465038429892986	0.200400801603206\\
-0.45783995239579	0.216432865731463\\
-0.456913827655311	0.21834757583315\\
-0.449867850526322	0.232464929859719\\
-0.441266377197455	0.248496993987976\\
-0.440881763527054	0.249169031698515\\
-0.431792395762406	0.264529058116232\\
-0.424849699398798	0.275541545237294\\
-0.421571465645375	0.280561122244489\\
-0.410468558458255	0.296593186372745\\
-0.408817635270541	0.298850314963264\\
-0.398356761163983	0.312625250501002\\
-0.392785571142285	0.319584768942361\\
-0.385229005548519	0.328657314629258\\
-0.376753507014028	0.338339868530592\\
-0.37095761940702	0.344689378757515\\
-0.360721442885771	0.355389462161335\\
-0.355389462161336	0.360721442885771\\
-0.344689378757515	0.37095761940702\\
-0.338339868530592	0.376753507014028\\
-0.328657314629258	0.385229005548519\\
-0.319584768942361	0.392785571142285\\
-0.312625250501002	0.398356761163984\\
-0.298850314963264	0.408817635270541\\
-0.296593186372745	0.410468558458255\\
-0.280561122244489	0.421571465645375\\
-0.275541545237294	0.424849699398798\\
-0.264529058116232	0.431792395762406\\
-0.249169031698516	0.440881763527054\\
-0.248496993987976	0.441266377197454\\
-0.232464929859719	0.449867850526322\\
-0.21834757583315	0.456913827655311\\
-0.216432865731463	0.45783995239579\\
-0.200400801603207	0.465038429892986\\
-0.18436873747495	0.4716618636504\\
-0.180975725896018	0.472945891783567\\
-0.168336673346694	0.477591530799338\\
-0.152304609218437	0.482937368408193\\
-0.13627254509018	0.487736888422235\\
-0.13160535439238	0.488977955911824\\
-0.120240480961924	0.491919501048468\\
-0.104208416833667	0.495545530742166\\
-0.0881763527054109	0.49864836791436\\
-0.0721442885771544	0.501230412793354\\
-0.0561122244488979	0.503293660866937\\
-0.0400801603206413	0.504839705452896\\
-0.0374304203479417	0.50501002004008\\
-0.0240480961923848	0.505849643810035\\
-0.00801603206412826	0.506352428552228\\
0.00801603206412782	0.506352428552228\\
0.0240480961923843	0.505849643810035\\
0.0374304203479412	0.50501002004008\\
0.0400801603206409	0.504839705452896\\
0.0561122244488974	0.503293660866937\\
0.0721442885771539	0.501230412793354\\
0.0881763527054105	0.49864836791436\\
0.104208416833667	0.495545530742166\\
0.120240480961924	0.491919501048468\\
0.131605354392381	0.488977955911824\\
0.13627254509018	0.487736888422236\\
0.152304609218437	0.482937368408193\\
0.168336673346693	0.477591530799339\\
0.180975725896018	0.472945891783567\\
0.18436873747495	0.4716618636504\\
0.200400801603206	0.465038429892986\\
0.216432865731463	0.45783995239579\\
0.21834757583315	0.456913827655311\\
0.232464929859719	0.449867850526322\\
0.248496993987976	0.441266377197454\\
0.249169031698515	0.440881763527054\\
0.264529058116232	0.431792395762406\\
0.275541545237294	0.424849699398798\\
0.280561122244489	0.421571465645375\\
0.296593186372745	0.410468558458255\\
0.298850314963264	0.408817635270541\\
0.312625250501002	0.398356761163984\\
0.319584768942361	0.392785571142285\\
0.328657314629258	0.385229005548519\\
0.338339868530592	0.376753507014028\\
0.344689378757515	0.37095761940702\\
0.355389462161335	0.360721442885771\\
0.360721442885771	0.355389462161335\\
0.37095761940702	0.344689378757515\\
0.376753507014028	0.338339868530592\\
0.385229005548519	0.328657314629258\\
0.392785571142285	0.319584768942361\\
0.398356761163984	0.312625250501002\\
0.408817635270541	0.298850314963264\\
0.410468558458255	0.296593186372745\\
0.421571465645375	0.280561122244489\\
0.424849699398798	0.275541545237294\\
0.431792395762406	0.264529058116232\\
0.440881763527054	0.249169031698515\\
0.441266377197454	0.248496993987976\\
0.449867850526322	0.232464929859719\\
0.456913827655311	0.21834757583315\\
0.45783995239579	0.216432865731463\\
0.465038429892986	0.200400801603206\\
0.4716618636504	0.18436873747495\\
0.472945891783567	0.180975725896018\\
0.477591530799339	0.168336673346693\\
0.482937368408193	0.152304609218437\\
0.487736888422236	0.13627254509018\\
0.488977955911824	0.131605354392381\\
0.491919501048468	0.120240480961924\\
0.495545530742166	0.104208416833667\\
0.49864836791436	0.0881763527054105\\
0.501230412793354	0.0721442885771539\\
0.503293660866937	0.0561122244488974\\
0.504839705452896	0.0400801603206409\\
0.50501002004008	0.0374304203479412\\
0.505849643810035	0.0240480961923843\\
0.506352428552228	0.00801603206412782\\
0.506352428552228	-0.00801603206412826\\
0.505849643810035	-0.0240480961923848\\
0.50501002004008	-0.0374304203479417\\
0.504839705452896	-0.0400801603206413\\
0.503293660866937	-0.0561122244488979\\
0.501230412793354	-0.0721442885771544\\
0.49864836791436	-0.0881763527054109\\
0.495545530742166	-0.104208416833667\\
0.491919501048468	-0.120240480961924\\
0.488977955911824	-0.13160535439238\\
0.487736888422235	-0.13627254509018\\
0.482937368408193	-0.152304609218437\\
0.477591530799338	-0.168336673346694\\
0.472945891783567	-0.180975725896018\\
0.4716618636504	-0.18436873747495\\
0.465038429892986	-0.200400801603207\\
0.45783995239579	-0.216432865731463\\
0.456913827655311	-0.21834757583315\\
0.449867850526322	-0.232464929859719\\
0.441266377197454	-0.248496993987976\\
0.440881763527054	-0.249169031698516\\
0.431792395762406	-0.264529058116232\\
0.424849699398798	-0.275541545237294\\
0.421571465645375	-0.280561122244489\\
0.410468558458255	-0.296593186372745\\
0.408817635270541	-0.298850314963264\\
0.398356761163984	-0.312625250501002\\
0.392785571142285	-0.319584768942361\\
0.385229005548519	-0.328657314629258\\
0.376753507014028	-0.338339868530592\\
0.37095761940702	-0.344689378757515\\
0.360721442885771	-0.355389462161336\\
0.355389462161335	-0.360721442885771\\
0.344689378757515	-0.37095761940702\\
0.338339868530592	-0.376753507014028\\
0.328657314629258	-0.385229005548519\\
0.319584768942361	-0.392785571142285\\
0.312625250501002	-0.398356761163983\\
0.298850314963264	-0.408817635270541\\
0.296593186372745	-0.410468558458255\\
0.280561122244489	-0.421571465645375\\
0.275541545237294	-0.424849699398798\\
0.264529058116232	-0.431792395762406\\
0.249169031698515	-0.440881763527054\\
0.248496993987976	-0.441266377197455\\
0.232464929859719	-0.449867850526322\\
0.21834757583315	-0.456913827655311\\
0.216432865731463	-0.45783995239579\\
0.200400801603206	-0.465038429892986\\
0.18436873747495	-0.471661863650401\\
0.180975725896018	-0.472945891783567\\
0.168336673346693	-0.477591530799338\\
0.152304609218437	-0.482937368408193\\
0.13627254509018	-0.487736888422236\\
0.131605354392381	-0.488977955911824\\
0.120240480961924	-0.491919501048469\\
0.104208416833667	-0.495545530742166\\
0.0881763527054105	-0.49864836791436\\
0.0721442885771539	-0.501230412793354\\
0.0561122244488974	-0.503293660866937\\
0.0400801603206409	-0.504839705452896\\
0.0374304203479412	-0.50501002004008\\
0.0240480961923843	-0.505849643810035\\
0.00801603206412782	-0.506352428552228\\
-0.00801603206412826	-0.506352428552228\\
-0.0240480961923848	-0.505849643810035\\
-0.0374304203479417	-0.50501002004008\\
-0.0400801603206413	-0.504839705452896\\
-0.0561122244488979	-0.503293660866937\\
-0.0721442885771544	-0.501230412793354\\
-0.0881763527054109	-0.49864836791436\\
-0.104208416833667	-0.495545530742166\\
-0.120240480961924	-0.491919501048468\\
-0.13160535439238	-0.488977955911824\\
-0.13627254509018	-0.487736888422235\\
-0.152304609218437	-0.482937368408192\\
-0.168336673346694	-0.477591530799338\\
-0.180975725896018	-0.472945891783567\\
-0.18436873747495	-0.4716618636504\\
-0.200400801603207	-0.465038429892985\\
-0.216432865731463	-0.45783995239579\\
-0.21834757583315	-0.456913827655311\\
-0.232464929859719	-0.449867850526322\\
-0.248496993987976	-0.441266377197455\\
-0.249169031698516	-0.440881763527054\\
-0.264529058116232	-0.431792395762406\\
-0.275541545237294	-0.424849699398798\\
-0.280561122244489	-0.421571465645375\\
-0.296593186372745	-0.410468558458255\\
-0.298850314963264	-0.408817635270541\\
-0.312625250501002	-0.398356761163983\\
-0.319584768942361	-0.392785571142285\\
-0.328657314629258	-0.385229005548519\\
-0.338339868530592	-0.376753507014028\\
-0.344689378757515	-0.37095761940702\\
-0.355389462161336	-0.360721442885771\\
-0.360721442885771	-0.355389462161336\\
-0.37095761940702	-0.344689378757515\\
-0.376753507014028	-0.338339868530592\\
-0.385229005548519	-0.328657314629258\\
-0.392785571142285	-0.319584768942361\\
-0.398356761163983	-0.312625250501002\\
-0.408817635270541	-0.298850314963264\\
-0.410468558458255	-0.296593186372745\\
-0.421571465645375	-0.280561122244489\\
-0.424849699398798	-0.275541545237294\\
-0.431792395762406	-0.264529058116232\\
-0.440881763527054	-0.249169031698516\\
-0.441266377197455	-0.248496993987976\\
-0.449867850526322	-0.232464929859719\\
-0.456913827655311	-0.21834757583315\\
-0.45783995239579	-0.216432865731463\\
-0.465038429892985	-0.200400801603207\\
-0.4716618636504	-0.18436873747495\\
-0.472945891783567	-0.180975725896018\\
-0.477591530799338	-0.168336673346694\\
-0.482937368408192	-0.152304609218437\\
-0.487736888422235	-0.13627254509018\\
-0.488977955911824	-0.13160535439238\\
-0.491919501048468	-0.120240480961924\\
-0.495545530742166	-0.104208416833667\\
-0.49864836791436	-0.0881763527054109\\
-0.501230412793354	-0.0721442885771544\\
-0.503293660866937	-0.0561122244488979\\
-0.504839705452896	-0.0400801603206413\\
-0.50501002004008	-0.0374304203479417\\
-0.505849643810035	-0.0240480961923848\\
-0.506352428552228	-0.00801603206412826\\
-0.506352428552228	0.00801603206412782\\
-0.505849643810035	0.0240480961923843\\
-0.50501002004008	0.0374304203479412\\
}--cycle;

\end{axis}
\end{tikzpicture}%
    \caption{$\mat{\Sigma} = \begin{bmatrix} 1 & 0 \\ 0 & 1 \end{bmatrix}$}
    \label{2d_example_1}
  \end{subfigure}
  \begin{subfigure}{0.33\textwidth}
    % This file was created by matlab2tikz.
% Minimal pgfplots version: 1.3
%
\tikzsetnextfilename{2d_gaussian_pdf_2}
\definecolor{mycolor1}{rgb}{0.01430,0.01430,0.01430}%
\definecolor{mycolor2}{rgb}{0.17056,0.07842,0.22082}%
\definecolor{mycolor3}{rgb}{0.15929,0.15355,0.47279}%
\definecolor{mycolor4}{rgb}{0.06860,0.30440,0.50680}%
\definecolor{mycolor5}{rgb}{0.00010,0.44723,0.37602}%
\definecolor{mycolor6}{rgb}{0.00000,0.58628,0.21866}%
\definecolor{mycolor7}{rgb}{0.21670,0.69660,0.00000}%
\definecolor{mycolor8}{rgb}{0.61067,0.77320,0.00000}%
\definecolor{mycolor9}{rgb}{0.90816,0.81535,0.56731}%
\definecolor{mycolor10}{rgb}{0.96920,0.92730,0.89610}%
%
\begin{tikzpicture}

\begin{axis}[%
width=0.95092\smallsquarefigurewidth,
height=\smallsquarefigureheight,
at={(0\smallsquarefigurewidth,0\smallsquarefigureheight)},
scale only axis,
xmin=-4,
xmax=4,
xlabel={$x_1$},
ymin=-4,
ymax=4,
ylabel={$x_2$},
axis x line*=bottom,
axis y line*=left
]

\addplot[area legend,solid,fill=mycolor1,draw=black,forget plot]
table[row sep=crcr] {%
x	y\\
-4	4\\
-3.98396793587174	4\\
-3.96793587174349	4\\
-3.95190380761523	4\\
-3.93587174348697	4\\
-3.91983967935872	4\\
-3.90380761523046	4\\
-3.8877755511022	4\\
-3.87174348697395	4\\
-3.85571142284569	4\\
-3.83967935871743	4\\
-3.82364729458918	4\\
-3.80761523046092	4\\
-3.79158316633267	4\\
-3.77555110220441	4\\
-3.75951903807615	4\\
-3.7434869739479	4\\
-3.72745490981964	4\\
-3.71142284569138	4\\
-3.69539078156313	4\\
-3.67935871743487	4\\
-3.66332665330661	4\\
-3.64729458917836	4\\
-3.6312625250501	4\\
-3.61523046092184	4\\
-3.59919839679359	4\\
-3.58316633266533	4\\
-3.56713426853707	4\\
-3.55110220440882	4\\
-3.53507014028056	4\\
-3.5190380761523	4\\
-3.50300601202405	4\\
-3.48697394789579	4\\
-3.47094188376753	4\\
-3.45490981963928	4\\
-3.43887775551102	4\\
-3.42284569138277	4\\
-3.40681362725451	4\\
-3.39078156312625	4\\
-3.374749498998	4\\
-3.35871743486974	4\\
-3.34268537074148	4\\
-3.32665330661323	4\\
-3.31062124248497	4\\
-3.29458917835671	4\\
-3.27855711422846	4\\
-3.2625250501002	4\\
-3.24649298597194	4\\
-3.23046092184369	4\\
-3.21442885771543	4\\
-3.19839679358717	4\\
-3.18236472945892	4\\
-3.16633266533066	4\\
-3.1503006012024	4\\
-3.13426853707415	4\\
-3.11823647294589	4\\
-3.10220440881764	4\\
-3.08617234468938	4\\
-3.07014028056112	4\\
-3.05410821643287	4\\
-3.03807615230461	4\\
-3.02204408817635	4\\
-3.0060120240481	4\\
-2.98997995991984	4\\
-2.97394789579158	4\\
-2.95791583166333	4\\
-2.94188376753507	4\\
-2.92585170340681	4\\
-2.90981963927856	4\\
-2.8937875751503	4\\
-2.87775551102204	4\\
-2.86172344689379	4\\
-2.84569138276553	4\\
-2.82965931863727	4\\
-2.81362725450902	4\\
-2.79759519038076	4\\
-2.7815631262525	4\\
-2.76553106212425	4\\
-2.74949899799599	4\\
-2.73346693386774	4\\
-2.71743486973948	4\\
-2.70140280561122	4\\
-2.68537074148297	4\\
-2.66933867735471	4\\
-2.65330661322645	4\\
-2.6372745490982	4\\
-2.62124248496994	4\\
-2.60521042084168	4\\
-2.58917835671343	4\\
-2.57314629258517	4\\
-2.55711422845691	4\\
-2.54108216432866	4\\
-2.5250501002004	4\\
-2.50901803607214	4\\
-2.49298597194389	4\\
-2.47695390781563	4\\
-2.46092184368737	4\\
-2.44488977955912	4\\
-2.42885771543086	4\\
-2.41282565130261	4\\
-2.39679358717435	4\\
-2.38076152304609	4\\
-2.36472945891784	4\\
-2.34869739478958	4\\
-2.33266533066132	4\\
-2.31663326653307	4\\
-2.30060120240481	4\\
-2.28456913827655	4\\
-2.2685370741483	4\\
-2.25250501002004	4\\
-2.23647294589178	4\\
-2.22044088176353	4\\
-2.20440881763527	4\\
-2.18837675350701	4\\
-2.17234468937876	4\\
-2.1563126252505	4.00000000000001\\
-2.14028056112224	4.00000000000001\\
-2.12424849699399	4.00000000000001\\
-2.10821643286573	4.00000000000001\\
-2.09218436873747	4.00000000000001\\
-2.07615230460922	4.00000000000001\\
-2.06012024048096	4.00000000000001\\
-2.04408817635271	4.00000000000001\\
-2.02805611222445	4.00000000000001\\
-2.01202404809619	4.00000000000001\\
-1.99599198396794	4.00000000000001\\
-1.97995991983968	4.00000000000001\\
-1.96392785571142	4.00000000000002\\
-1.94789579158317	4.00000000000002\\
-1.93186372745491	4.00000000000002\\
-1.91583166332665	4.00000000000002\\
-1.8997995991984	4.00000000000002\\
-1.88376753507014	4.00000000000002\\
-1.86773547094188	4.00000000000003\\
-1.85170340681363	4.00000000000003\\
-1.83567134268537	4.00000000000003\\
-1.81963927855711	4.00000000000003\\
-1.80360721442886	4.00000000000003\\
-1.7875751503006	4.00000000000004\\
-1.77154308617234	4.00000000000004\\
-1.75551102204409	4.00000000000004\\
-1.73947895791583	4.00000000000005\\
-1.72344689378758	4.00000000000005\\
-1.70741482965932	4.00000000000006\\
-1.69138276553106	4.00000000000006\\
-1.67535070140281	4.00000000000007\\
-1.65931863727455	4.00000000000007\\
-1.64328657314629	4.00000000000008\\
-1.62725450901804	4.00000000000008\\
-1.61122244488978	4.00000000000009\\
-1.59519038076152	4.0000000000001\\
-1.57915831663327	4.0000000000001\\
-1.56312625250501	4.00000000000011\\
-1.54709418837675	4.00000000000012\\
-1.5310621242485	4.00000000000013\\
-1.51503006012024	4.00000000000014\\
-1.49899799599198	4.00000000000015\\
-1.48296593186373	4.00000000000017\\
-1.46693386773547	4.00000000000018\\
-1.45090180360721	4.00000000000019\\
-1.43486973947896	4.00000000000021\\
-1.4188376753507	4.00000000000022\\
-1.40280561122244	4.00000000000024\\
-1.38677354709419	4.00000000000026\\
-1.37074148296593	4.00000000000028\\
-1.35470941883768	4.0000000000003\\
-1.33867735470942	4.00000000000032\\
-1.32264529058116	4.00000000000034\\
-1.30661322645291	4.00000000000037\\
-1.29058116232465	4.00000000000039\\
-1.27454909819639	4.00000000000042\\
-1.25851703406814	4.00000000000045\\
-1.24248496993988	4.00000000000049\\
-1.22645290581162	4.00000000000052\\
-1.21042084168337	4.00000000000056\\
-1.19438877755511	4.0000000000006\\
-1.17835671342685	4.00000000000064\\
-1.1623246492986	4.00000000000068\\
-1.14629258517034	4.00000000000073\\
-1.13026052104208	4.00000000000078\\
-1.11422845691383	4.00000000000084\\
-1.09819639278557	4.00000000000089\\
-1.08216432865731	4.00000000000095\\
-1.06613226452906	4.00000000000102\\
-1.0501002004008	4.00000000000109\\
-1.03406813627255	4.00000000000116\\
-1.01803607214429	4.00000000000124\\
-1.00200400801603	4.00000000000132\\
-0.985971943887776	4.00000000000141\\
-0.969939879759519	4.0000000000015\\
-0.953907815631263	4.0000000000016\\
-0.937875751503006	4.00000000000171\\
-0.92184368737475	4.00000000000182\\
-0.905811623246493	4.00000000000193\\
-0.889779559118236	4.00000000000206\\
-0.87374749498998	4.00000000000218\\
-0.857715430861723	4.00000000000232\\
-0.841683366733467	4.00000000000247\\
-0.82565130260521	4.00000000000262\\
-0.809619238476954	4.00000000000279\\
-0.793587174348697	4.00000000000296\\
-0.777555110220441	4.00000000000314\\
-0.761523046092184	4.00000000000333\\
-0.745490981963928	4.00000000000353\\
-0.729458917835671	4.00000000000375\\
-0.713426853707415	4.00000000000397\\
-0.697394789579158	4.00000000000421\\
-0.681362725450902	4.00000000000446\\
-0.665330661322646	4.00000000000472\\
-0.649298597194389	4.000000000005\\
-0.633266533066132	4.00000000000528\\
-0.617234468937876	4.00000000000559\\
-0.601202404809619	4.00000000000591\\
-0.585170340681363	4.00000000000625\\
-0.569138276553106	4.0000000000066\\
-0.55310621242485	4.00000000000697\\
-0.537074148296593	4.00000000000736\\
-0.521042084168337	4.00000000000777\\
-0.50501002004008	4.0000000000082\\
-0.488977955911824	4.00000000000865\\
-0.472945891783567	4.00000000000912\\
-0.456913827655311	4.00000000000961\\
-0.440881763527054	4.00000000001013\\
-0.424849699398798	4.00000000001067\\
-0.408817635270541	4.00000000001124\\
-0.392785571142285	4.00000000001183\\
-0.376753507014028	4.00000000001245\\
-0.360721442885771	4.00000000001309\\
-0.344689378757515	4.00000000001377\\
-0.328657314629258	4.00000000001448\\
-0.312625250501002	4.00000000001521\\
-0.296593186372745	4.00000000001598\\
-0.280561122244489	4.00000000001678\\
-0.264529058116232	4.00000000001762\\
-0.248496993987976	4.00000000001849\\
-0.232464929859719	4.00000000001939\\
-0.216432865731463	4.00000000002034\\
-0.200400801603207	4.00000000002132\\
-0.18436873747495	4.00000000002234\\
-0.168336673346694	4.00000000002341\\
-0.152304609218437	4.00000000002452\\
-0.13627254509018	4.00000000002566\\
-0.120240480961924	4.00000000002686\\
-0.104208416833667	4.0000000000281\\
-0.0881763527054109	4.00000000002939\\
-0.0721442885771544	4.00000000003072\\
-0.0561122244488979	4.00000000003211\\
-0.0400801603206413	4.00000000003355\\
-0.0240480961923848	4.00000000003504\\
-0.00801603206412826	4.00000000003658\\
0.00801603206412782	4.00000000003818\\
0.0240480961923843	4.00000000003983\\
0.0400801603206409	4.00000000004154\\
0.0561122244488974	4.00000000004331\\
0.0721442885771539	4.00000000004514\\
0.0881763527054105	4.00000000004703\\
0.104208416833667	4.00000000004899\\
0.120240480961924	4.000000000051\\
0.13627254509018	4.00000000005309\\
0.152304609218437	4.00000000005523\\
0.168336673346693	4.00000000005745\\
0.18436873747495	4.00000000005973\\
0.200400801603206	4.00000000006209\\
0.216432865731463	4.00000000006451\\
0.232464929859719	4.00000000006701\\
0.248496993987976	4.00000000006957\\
0.264529058116232	4.00000000007222\\
0.280561122244489	4.00000000007493\\
0.296593186372745	4.00000000007772\\
0.312625250501002	4.00000000008059\\
0.328657314629258	4.00000000008354\\
0.344689378757515	4.00000000008656\\
0.360721442885771	4.00000000008966\\
0.376753507014028	4.00000000009285\\
0.392785571142285	4.00000000009611\\
0.408817635270541	4.00000000009945\\
0.424849699398798	4.00000000010287\\
0.440881763527054	4.00000000010638\\
0.456913827655311	4.00000000010996\\
0.472945891783567	4.00000000011363\\
0.488977955911824	4.00000000011738\\
0.50501002004008	4.00000000012121\\
0.521042084168337	4.00000000012513\\
0.537074148296593	4.00000000012913\\
0.55310621242485	4.00000000013321\\
0.569138276553106	4.00000000013737\\
0.585170340681363	4.00000000014161\\
0.601202404809619	4.00000000014593\\
0.617234468937876	4.00000000015033\\
0.633266533066132	4.00000000015482\\
0.649298597194389	4.00000000015938\\
0.665330661322646	4.00000000016402\\
0.681362725450902	4.00000000016874\\
0.697394789579159	4.00000000017353\\
0.713426853707415	4.0000000001784\\
0.729458917835672	4.00000000018335\\
0.745490981963928	4.00000000018836\\
0.761523046092185	4.00000000019345\\
0.777555110220441	4.0000000001986\\
0.793587174348698	4.00000000020383\\
0.809619238476954	4.00000000020912\\
0.825651302605211	4.00000000021447\\
0.841683366733467	4.00000000021988\\
0.857715430861724	4.00000000022536\\
0.87374749498998	4.00000000023089\\
0.889779559118236	4.00000000023647\\
0.905811623246493	4.00000000024211\\
0.921843687374749	4.0000000002478\\
0.937875751503006	4.00000000025353\\
0.953907815631262	4.00000000025931\\
0.969939879759519	4.00000000026513\\
0.985971943887775	4.00000000027098\\
1.00200400801603	4.00000000027687\\
1.01803607214429	4.0000000002828\\
1.03406813627254	4.00000000028874\\
1.0501002004008	4.00000000029472\\
1.06613226452906	4.00000000030071\\
1.08216432865731	4.00000000030672\\
1.09819639278557	4.00000000031275\\
1.11422845691383	4.00000000031878\\
1.13026052104208	4.00000000032482\\
1.14629258517034	4.00000000033086\\
1.1623246492986	4.00000000033689\\
1.17835671342685	4.00000000034292\\
1.19438877755511	4.00000000034894\\
1.21042084168337	4.00000000035494\\
1.22645290581162	4.00000000036091\\
1.24248496993988	4.00000000036687\\
1.25851703406814	4.00000000037279\\
1.27454909819639	4.00000000037869\\
1.29058116232465	4.00000000038454\\
1.30661322645291	4.00000000039035\\
1.32264529058116	4.00000000039611\\
1.33867735470942	4.00000000040182\\
1.35470941883768	4.00000000040747\\
1.37074148296593	4.00000000041306\\
1.38677354709419	4.00000000041858\\
1.40280561122244	4.00000000042403\\
1.4188376753507	4.0000000004294\\
1.43486973947896	4.00000000043469\\
1.45090180360721	4.0000000004399\\
1.46693386773547	4.00000000044502\\
1.48296593186373	4.00000000045004\\
1.49899799599198	4.00000000045497\\
1.51503006012024	4.00000000045979\\
1.5310621242485	4.0000000004645\\
1.54709418837675	4.0000000004691\\
1.56312625250501	4.00000000047358\\
1.57915831663327	4.00000000047794\\
1.59519038076152	4.00000000048218\\
1.61122244488978	4.00000000048628\\
1.62725450901804	4.00000000049026\\
1.64328657314629	4.0000000004941\\
1.65931863727455	4.00000000049779\\
1.67535070140281	4.00000000050134\\
1.69138276553106	4.00000000050475\\
1.70741482965932	4.000000000508\\
1.72344689378758	4.0000000005111\\
1.73947895791583	4.00000000051404\\
1.75551102204409	4.00000000051683\\
1.77154308617235	4.00000000051945\\
1.7875751503006	4.0000000005219\\
1.80360721442886	4.00000000052419\\
1.81963927855711	4.0000000005263\\
1.83567134268537	4.00000000052824\\
1.85170340681363	4.00000000053001\\
1.86773547094188	4.0000000005316\\
1.88376753507014	4.00000000053302\\
1.8997995991984	4.00000000053425\\
1.91583166332665	4.00000000053531\\
1.93186372745491	4.00000000053618\\
1.94789579158317	4.00000000053687\\
1.96392785571142	4.00000000053737\\
1.97995991983968	4.00000000053769\\
1.99599198396794	4.00000000053783\\
2.01202404809619	4.00000000053779\\
2.02805611222445	4.00000000053756\\
2.04408817635271	4.00000000053714\\
2.06012024048096	4.00000000053654\\
2.07615230460922	4.00000000053576\\
2.09218436873747	4.0000000005348\\
2.10821643286573	4.00000000053366\\
2.12424849699399	4.00000000053233\\
2.14028056112224	4.00000000053083\\
2.1563126252505	4.00000000052915\\
2.17234468937876	4.00000000052729\\
2.18837675350701	4.00000000052526\\
2.20440881763527	4.00000000052306\\
2.22044088176353	4.00000000052069\\
2.23647294589178	4.00000000051816\\
2.25250501002004	4.00000000051546\\
2.2685370741483	4.00000000051259\\
2.28456913827655	4.00000000050957\\
2.30060120240481	4.00000000050639\\
2.31663326653307	4.00000000050307\\
2.33266533066132	4.00000000049959\\
2.34869739478958	4.00000000049596\\
2.36472945891784	4.00000000049219\\
2.38076152304609	4.00000000048829\\
2.39679358717435	4.00000000048425\\
2.41282565130261	4.00000000048007\\
2.42885771543086	4.00000000047578\\
2.44488977955912	4.00000000047135\\
2.46092184368737	4.00000000046681\\
2.47695390781563	4.00000000046216\\
2.49298597194389	4.00000000045739\\
2.50901803607214	4.00000000045252\\
2.5250501002004	4.00000000044754\\
2.54108216432866	4.00000000044247\\
2.55711422845691	4.00000000043731\\
2.57314629258517	4.00000000043206\\
2.58917835671343	4.00000000042672\\
2.60521042084168	4.00000000042131\\
2.62124248496994	4.00000000041582\\
2.6372745490982	4.00000000041027\\
2.65330661322645	4.00000000040465\\
2.66933867735471	4.00000000039897\\
2.68537074148297	4.00000000039323\\
2.70140280561122	4.00000000038745\\
2.71743486973948	4.00000000038162\\
2.73346693386774	4.00000000037574\\
2.74949899799599	4.00000000036984\\
2.76553106212425	4.0000000003639\\
2.7815631262525	4.00000000035793\\
2.79759519038076	4.00000000035194\\
2.81362725450902	4.00000000034593\\
2.82965931863727	4.00000000033991\\
2.84569138276553	4.00000000033387\\
2.86172344689379	4.00000000032784\\
2.87775551102204	4.0000000003218\\
2.8937875751503	4.00000000031576\\
2.90981963927856	4.00000000030973\\
2.92585170340681	4.00000000030372\\
2.94188376753507	4.00000000029771\\
2.95791583166333	4.00000000029173\\
2.97394789579158	4.00000000028577\\
2.98997995991984	4.00000000027983\\
3.0060120240481	4.00000000027392\\
3.02204408817635	4.00000000026805\\
3.03807615230461	4.00000000026221\\
3.05410821643287	4.00000000025641\\
3.07014028056112	4.00000000025066\\
3.08617234468938	4.00000000024495\\
3.10220440881764	4.00000000023929\\
3.11823647294589	4.00000000023367\\
3.13426853707415	4.00000000022811\\
3.1503006012024	4.00000000022261\\
3.16633266533066	4.00000000021717\\
3.18236472945892	4.00000000021178\\
3.19839679358717	4.00000000020646\\
3.21442885771543	4.00000000020121\\
3.23046092184369	4.00000000019602\\
3.24649298597194	4.0000000001909\\
3.2625250501002	4.00000000018585\\
3.27855711422846	4.00000000018087\\
3.29458917835671	4.00000000017596\\
3.31062124248497	4.00000000017113\\
3.32665330661323	4.00000000016637\\
3.34268537074148	4.00000000016169\\
3.35871743486974	4.00000000015709\\
3.374749498998	4.00000000015257\\
3.39078156312625	4.00000000014812\\
3.40681362725451	4.00000000014376\\
3.42284569138277	4.00000000013948\\
3.43887775551102	4.00000000013527\\
3.45490981963928	4.00000000013116\\
3.47094188376754	4.00000000012712\\
3.48697394789579	4.00000000012316\\
3.50300601202405	4.00000000011929\\
3.51903807615231	4.0000000001155\\
3.53507014028056	4.00000000011179\\
3.55110220440882	4.00000000010816\\
3.56713426853707	4.00000000010461\\
3.58316633266533	4.00000000010115\\
3.59919839679359	4.00000000009777\\
3.61523046092184	4.00000000009447\\
3.6312625250501	4.00000000009124\\
3.64729458917836	4.0000000000881\\
3.66332665330661	4.00000000008504\\
3.67935871743487	4.00000000008206\\
3.69539078156313	4.00000000007915\\
3.71142284569138	4.00000000007632\\
3.72745490981964	4.00000000007356\\
3.7434869739479	4.00000000007088\\
3.75951903807615	4.00000000006828\\
3.77555110220441	4.00000000006575\\
3.79158316633267	4.00000000006329\\
3.80761523046092	4.0000000000609\\
3.82364729458918	4.00000000005858\\
3.83967935871743	4.00000000005633\\
3.85571142284569	4.00000000005415\\
3.87174348697395	4.00000000005204\\
3.8877755511022	4.00000000004999\\
3.90380761523046	4.000000000048\\
3.91983967935872	4.00000000004608\\
3.93587174348697	4.00000000004422\\
3.95190380761523	4.00000000004242\\
3.96793587174349	4.00000000004068\\
3.98396793587174	4.000000000039\\
4	4.00000000003737\\
4.00000000003737	4\\
4.000000000039	3.98396793587174\\
4.00000000004068	3.96793587174349\\
4.00000000004242	3.95190380761523\\
4.00000000004422	3.93587174348697\\
4.00000000004608	3.91983967935872\\
4.000000000048	3.90380761523046\\
4.00000000004999	3.8877755511022\\
4.00000000005204	3.87174348697395\\
4.00000000005415	3.85571142284569\\
4.00000000005633	3.83967935871743\\
4.00000000005858	3.82364729458918\\
4.0000000000609	3.80761523046092\\
4.00000000006329	3.79158316633267\\
4.00000000006575	3.77555110220441\\
4.00000000006828	3.75951903807615\\
4.00000000007088	3.7434869739479\\
4.00000000007356	3.72745490981964\\
4.00000000007632	3.71142284569138\\
4.00000000007915	3.69539078156313\\
4.00000000008206	3.67935871743487\\
4.00000000008504	3.66332665330661\\
4.0000000000881	3.64729458917836\\
4.00000000009124	3.6312625250501\\
4.00000000009447	3.61523046092184\\
4.00000000009777	3.59919839679359\\
4.00000000010115	3.58316633266533\\
4.00000000010461	3.56713426853707\\
4.00000000010816	3.55110220440882\\
4.00000000011179	3.53507014028056\\
4.0000000001155	3.51903807615231\\
4.00000000011929	3.50300601202405\\
4.00000000012316	3.48697394789579\\
4.00000000012712	3.47094188376754\\
4.00000000013116	3.45490981963928\\
4.00000000013527	3.43887775551102\\
4.00000000013948	3.42284569138277\\
4.00000000014376	3.40681362725451\\
4.00000000014812	3.39078156312625\\
4.00000000015257	3.374749498998\\
4.00000000015709	3.35871743486974\\
4.00000000016169	3.34268537074148\\
4.00000000016637	3.32665330661323\\
4.00000000017113	3.31062124248497\\
4.00000000017596	3.29458917835671\\
4.00000000018087	3.27855711422846\\
4.00000000018585	3.2625250501002\\
4.0000000001909	3.24649298597194\\
4.00000000019602	3.23046092184369\\
4.00000000020121	3.21442885771543\\
4.00000000020646	3.19839679358717\\
4.00000000021178	3.18236472945892\\
4.00000000021717	3.16633266533066\\
4.00000000022261	3.1503006012024\\
4.00000000022811	3.13426853707415\\
4.00000000023367	3.11823647294589\\
4.00000000023929	3.10220440881764\\
4.00000000024495	3.08617234468938\\
4.00000000025066	3.07014028056112\\
4.00000000025641	3.05410821643287\\
4.00000000026221	3.03807615230461\\
4.00000000026805	3.02204408817635\\
4.00000000027392	3.0060120240481\\
4.00000000027983	2.98997995991984\\
4.00000000028577	2.97394789579158\\
4.00000000029173	2.95791583166333\\
4.00000000029771	2.94188376753507\\
4.00000000030372	2.92585170340681\\
4.00000000030973	2.90981963927856\\
4.00000000031576	2.8937875751503\\
4.0000000003218	2.87775551102204\\
4.00000000032784	2.86172344689379\\
4.00000000033387	2.84569138276553\\
4.00000000033991	2.82965931863727\\
4.00000000034593	2.81362725450902\\
4.00000000035194	2.79759519038076\\
4.00000000035793	2.7815631262525\\
4.0000000003639	2.76553106212425\\
4.00000000036984	2.74949899799599\\
4.00000000037574	2.73346693386774\\
4.00000000038162	2.71743486973948\\
4.00000000038745	2.70140280561122\\
4.00000000039323	2.68537074148297\\
4.00000000039897	2.66933867735471\\
4.00000000040465	2.65330661322645\\
4.00000000041027	2.6372745490982\\
4.00000000041582	2.62124248496994\\
4.00000000042131	2.60521042084168\\
4.00000000042672	2.58917835671343\\
4.00000000043206	2.57314629258517\\
4.00000000043731	2.55711422845691\\
4.00000000044247	2.54108216432866\\
4.00000000044754	2.5250501002004\\
4.00000000045252	2.50901803607214\\
4.00000000045739	2.49298597194389\\
4.00000000046216	2.47695390781563\\
4.00000000046681	2.46092184368737\\
4.00000000047135	2.44488977955912\\
4.00000000047578	2.42885771543086\\
4.00000000048007	2.41282565130261\\
4.00000000048425	2.39679358717435\\
4.00000000048829	2.38076152304609\\
4.00000000049219	2.36472945891784\\
4.00000000049596	2.34869739478958\\
4.00000000049959	2.33266533066132\\
4.00000000050307	2.31663326653307\\
4.00000000050639	2.30060120240481\\
4.00000000050957	2.28456913827655\\
4.00000000051259	2.2685370741483\\
4.00000000051546	2.25250501002004\\
4.00000000051816	2.23647294589178\\
4.00000000052069	2.22044088176353\\
4.00000000052306	2.20440881763527\\
4.00000000052526	2.18837675350701\\
4.00000000052729	2.17234468937876\\
4.00000000052915	2.1563126252505\\
4.00000000053083	2.14028056112224\\
4.00000000053233	2.12424849699399\\
4.00000000053366	2.10821643286573\\
4.0000000005348	2.09218436873747\\
4.00000000053576	2.07615230460922\\
4.00000000053654	2.06012024048096\\
4.00000000053714	2.04408817635271\\
4.00000000053756	2.02805611222445\\
4.00000000053779	2.01202404809619\\
4.00000000053783	1.99599198396794\\
4.00000000053769	1.97995991983968\\
4.00000000053737	1.96392785571142\\
4.00000000053687	1.94789579158317\\
4.00000000053618	1.93186372745491\\
4.00000000053531	1.91583166332665\\
4.00000000053425	1.8997995991984\\
4.00000000053302	1.88376753507014\\
4.0000000005316	1.86773547094188\\
4.00000000053001	1.85170340681363\\
4.00000000052824	1.83567134268537\\
4.0000000005263	1.81963927855711\\
4.00000000052419	1.80360721442886\\
4.0000000005219	1.7875751503006\\
4.00000000051945	1.77154308617235\\
4.00000000051683	1.75551102204409\\
4.00000000051404	1.73947895791583\\
4.0000000005111	1.72344689378758\\
4.000000000508	1.70741482965932\\
4.00000000050475	1.69138276553106\\
4.00000000050134	1.67535070140281\\
4.00000000049779	1.65931863727455\\
4.0000000004941	1.64328657314629\\
4.00000000049026	1.62725450901804\\
4.00000000048628	1.61122244488978\\
4.00000000048218	1.59519038076152\\
4.00000000047794	1.57915831663327\\
4.00000000047358	1.56312625250501\\
4.0000000004691	1.54709418837675\\
4.0000000004645	1.5310621242485\\
4.00000000045979	1.51503006012024\\
4.00000000045497	1.49899799599198\\
4.00000000045004	1.48296593186373\\
4.00000000044502	1.46693386773547\\
4.0000000004399	1.45090180360721\\
4.00000000043469	1.43486973947896\\
4.0000000004294	1.4188376753507\\
4.00000000042403	1.40280561122244\\
4.00000000041858	1.38677354709419\\
4.00000000041306	1.37074148296593\\
4.00000000040747	1.35470941883768\\
4.00000000040182	1.33867735470942\\
4.00000000039611	1.32264529058116\\
4.00000000039035	1.30661322645291\\
4.00000000038454	1.29058116232465\\
4.00000000037869	1.27454909819639\\
4.00000000037279	1.25851703406814\\
4.00000000036687	1.24248496993988\\
4.00000000036091	1.22645290581162\\
4.00000000035494	1.21042084168337\\
4.00000000034894	1.19438877755511\\
4.00000000034292	1.17835671342685\\
4.00000000033689	1.1623246492986\\
4.00000000033086	1.14629258517034\\
4.00000000032482	1.13026052104208\\
4.00000000031878	1.11422845691383\\
4.00000000031275	1.09819639278557\\
4.00000000030672	1.08216432865731\\
4.00000000030071	1.06613226452906\\
4.00000000029472	1.0501002004008\\
4.00000000028874	1.03406813627254\\
4.0000000002828	1.01803607214429\\
4.00000000027687	1.00200400801603\\
4.00000000027098	0.985971943887775\\
4.00000000026513	0.969939879759519\\
4.00000000025931	0.953907815631262\\
4.00000000025353	0.937875751503006\\
4.0000000002478	0.921843687374749\\
4.00000000024211	0.905811623246493\\
4.00000000023647	0.889779559118236\\
4.00000000023089	0.87374749498998\\
4.00000000022536	0.857715430861724\\
4.00000000021988	0.841683366733467\\
4.00000000021447	0.825651302605211\\
4.00000000020912	0.809619238476954\\
4.00000000020383	0.793587174348698\\
4.0000000001986	0.777555110220441\\
4.00000000019345	0.761523046092185\\
4.00000000018836	0.745490981963928\\
4.00000000018335	0.729458917835672\\
4.0000000001784	0.713426853707415\\
4.00000000017353	0.697394789579159\\
4.00000000016874	0.681362725450902\\
4.00000000016402	0.665330661322646\\
4.00000000015938	0.649298597194389\\
4.00000000015482	0.633266533066132\\
4.00000000015033	0.617234468937876\\
4.00000000014593	0.601202404809619\\
4.00000000014161	0.585170340681363\\
4.00000000013737	0.569138276553106\\
4.00000000013321	0.55310621242485\\
4.00000000012913	0.537074148296593\\
4.00000000012513	0.521042084168337\\
4.00000000012121	0.50501002004008\\
4.00000000011738	0.488977955911824\\
4.00000000011363	0.472945891783567\\
4.00000000010996	0.456913827655311\\
4.00000000010638	0.440881763527054\\
4.00000000010287	0.424849699398798\\
4.00000000009945	0.408817635270541\\
4.00000000009611	0.392785571142285\\
4.00000000009285	0.376753507014028\\
4.00000000008966	0.360721442885771\\
4.00000000008656	0.344689378757515\\
4.00000000008354	0.328657314629258\\
4.00000000008059	0.312625250501002\\
4.00000000007772	0.296593186372745\\
4.00000000007493	0.280561122244489\\
4.00000000007222	0.264529058116232\\
4.00000000006957	0.248496993987976\\
4.00000000006701	0.232464929859719\\
4.00000000006451	0.216432865731463\\
4.00000000006209	0.200400801603206\\
4.00000000005973	0.18436873747495\\
4.00000000005745	0.168336673346693\\
4.00000000005523	0.152304609218437\\
4.00000000005309	0.13627254509018\\
4.000000000051	0.120240480961924\\
4.00000000004899	0.104208416833667\\
4.00000000004703	0.0881763527054105\\
4.00000000004514	0.0721442885771539\\
4.00000000004331	0.0561122244488974\\
4.00000000004154	0.0400801603206409\\
4.00000000003983	0.0240480961923843\\
4.00000000003818	0.00801603206412782\\
4.00000000003658	-0.00801603206412826\\
4.00000000003504	-0.0240480961923848\\
4.00000000003355	-0.0400801603206413\\
4.00000000003211	-0.0561122244488979\\
4.00000000003072	-0.0721442885771544\\
4.00000000002939	-0.0881763527054109\\
4.0000000000281	-0.104208416833667\\
4.00000000002686	-0.120240480961924\\
4.00000000002566	-0.13627254509018\\
4.00000000002452	-0.152304609218437\\
4.00000000002341	-0.168336673346694\\
4.00000000002234	-0.18436873747495\\
4.00000000002132	-0.200400801603207\\
4.00000000002034	-0.216432865731463\\
4.00000000001939	-0.232464929859719\\
4.00000000001849	-0.248496993987976\\
4.00000000001762	-0.264529058116232\\
4.00000000001678	-0.280561122244489\\
4.00000000001598	-0.296593186372745\\
4.00000000001521	-0.312625250501002\\
4.00000000001448	-0.328657314629258\\
4.00000000001377	-0.344689378757515\\
4.00000000001309	-0.360721442885771\\
4.00000000001245	-0.376753507014028\\
4.00000000001183	-0.392785571142285\\
4.00000000001124	-0.408817635270541\\
4.00000000001067	-0.424849699398798\\
4.00000000001013	-0.440881763527054\\
4.00000000000961	-0.456913827655311\\
4.00000000000912	-0.472945891783567\\
4.00000000000865	-0.488977955911824\\
4.0000000000082	-0.50501002004008\\
4.00000000000777	-0.521042084168337\\
4.00000000000736	-0.537074148296593\\
4.00000000000697	-0.55310621242485\\
4.0000000000066	-0.569138276553106\\
4.00000000000625	-0.585170340681363\\
4.00000000000591	-0.601202404809619\\
4.00000000000559	-0.617234468937876\\
4.00000000000528	-0.633266533066132\\
4.000000000005	-0.649298597194389\\
4.00000000000472	-0.665330661322646\\
4.00000000000446	-0.681362725450902\\
4.00000000000421	-0.697394789579158\\
4.00000000000397	-0.713426853707415\\
4.00000000000375	-0.729458917835671\\
4.00000000000353	-0.745490981963928\\
4.00000000000333	-0.761523046092184\\
4.00000000000314	-0.777555110220441\\
4.00000000000296	-0.793587174348697\\
4.00000000000279	-0.809619238476954\\
4.00000000000262	-0.82565130260521\\
4.00000000000247	-0.841683366733467\\
4.00000000000232	-0.857715430861723\\
4.00000000000218	-0.87374749498998\\
4.00000000000206	-0.889779559118236\\
4.00000000000193	-0.905811623246493\\
4.00000000000182	-0.92184368737475\\
4.00000000000171	-0.937875751503006\\
4.0000000000016	-0.953907815631263\\
4.0000000000015	-0.969939879759519\\
4.00000000000141	-0.985971943887776\\
4.00000000000132	-1.00200400801603\\
4.00000000000124	-1.01803607214429\\
4.00000000000116	-1.03406813627255\\
4.00000000000109	-1.0501002004008\\
4.00000000000102	-1.06613226452906\\
4.00000000000095	-1.08216432865731\\
4.00000000000089	-1.09819639278557\\
4.00000000000084	-1.11422845691383\\
4.00000000000078	-1.13026052104208\\
4.00000000000073	-1.14629258517034\\
4.00000000000068	-1.1623246492986\\
4.00000000000064	-1.17835671342685\\
4.0000000000006	-1.19438877755511\\
4.00000000000056	-1.21042084168337\\
4.00000000000052	-1.22645290581162\\
4.00000000000049	-1.24248496993988\\
4.00000000000045	-1.25851703406814\\
4.00000000000042	-1.27454909819639\\
4.00000000000039	-1.29058116232465\\
4.00000000000037	-1.30661322645291\\
4.00000000000034	-1.32264529058116\\
4.00000000000032	-1.33867735470942\\
4.0000000000003	-1.35470941883768\\
4.00000000000028	-1.37074148296593\\
4.00000000000026	-1.38677354709419\\
4.00000000000024	-1.40280561122244\\
4.00000000000022	-1.4188376753507\\
4.00000000000021	-1.43486973947896\\
4.00000000000019	-1.45090180360721\\
4.00000000000018	-1.46693386773547\\
4.00000000000017	-1.48296593186373\\
4.00000000000015	-1.49899799599198\\
4.00000000000014	-1.51503006012024\\
4.00000000000013	-1.5310621242485\\
4.00000000000012	-1.54709418837675\\
4.00000000000011	-1.56312625250501\\
4.0000000000001	-1.57915831663327\\
4.0000000000001	-1.59519038076152\\
4.00000000000009	-1.61122244488978\\
4.00000000000008	-1.62725450901804\\
4.00000000000008	-1.64328657314629\\
4.00000000000007	-1.65931863727455\\
4.00000000000007	-1.67535070140281\\
4.00000000000006	-1.69138276553106\\
4.00000000000006	-1.70741482965932\\
4.00000000000005	-1.72344689378758\\
4.00000000000005	-1.73947895791583\\
4.00000000000004	-1.75551102204409\\
4.00000000000004	-1.77154308617234\\
4.00000000000004	-1.7875751503006\\
4.00000000000003	-1.80360721442886\\
4.00000000000003	-1.81963927855711\\
4.00000000000003	-1.83567134268537\\
4.00000000000003	-1.85170340681363\\
4.00000000000003	-1.86773547094188\\
4.00000000000002	-1.88376753507014\\
4.00000000000002	-1.8997995991984\\
4.00000000000002	-1.91583166332665\\
4.00000000000002	-1.93186372745491\\
4.00000000000002	-1.94789579158317\\
4.00000000000002	-1.96392785571142\\
4.00000000000001	-1.97995991983968\\
4.00000000000001	-1.99599198396794\\
4.00000000000001	-2.01202404809619\\
4.00000000000001	-2.02805611222445\\
4.00000000000001	-2.04408817635271\\
4.00000000000001	-2.06012024048096\\
4.00000000000001	-2.07615230460922\\
4.00000000000001	-2.09218436873747\\
4.00000000000001	-2.10821643286573\\
4.00000000000001	-2.12424849699399\\
4.00000000000001	-2.14028056112224\\
4.00000000000001	-2.1563126252505\\
4	-2.17234468937876\\
4	-2.18837675350701\\
4	-2.20440881763527\\
4	-2.22044088176353\\
4	-2.23647294589178\\
4	-2.25250501002004\\
4	-2.2685370741483\\
4	-2.28456913827655\\
4	-2.30060120240481\\
4	-2.31663326653307\\
4	-2.33266533066132\\
4	-2.34869739478958\\
4	-2.36472945891784\\
4	-2.38076152304609\\
4	-2.39679358717435\\
4	-2.41282565130261\\
4	-2.42885771543086\\
4	-2.44488977955912\\
4	-2.46092184368737\\
4	-2.47695390781563\\
4	-2.49298597194389\\
4	-2.50901803607214\\
4	-2.5250501002004\\
4	-2.54108216432866\\
4	-2.55711422845691\\
4	-2.57314629258517\\
4	-2.58917835671343\\
4	-2.60521042084168\\
4	-2.62124248496994\\
4	-2.6372745490982\\
4	-2.65330661322645\\
4	-2.66933867735471\\
4	-2.68537074148297\\
4	-2.70140280561122\\
4	-2.71743486973948\\
4	-2.73346693386774\\
4	-2.74949899799599\\
4	-2.76553106212425\\
4	-2.7815631262525\\
4	-2.79759519038076\\
4	-2.81362725450902\\
4	-2.82965931863727\\
4	-2.84569138276553\\
4	-2.86172344689379\\
4	-2.87775551102204\\
4	-2.8937875751503\\
4	-2.90981963927856\\
4	-2.92585170340681\\
4	-2.94188376753507\\
4	-2.95791583166333\\
4	-2.97394789579158\\
4	-2.98997995991984\\
4	-3.0060120240481\\
4	-3.02204408817635\\
4	-3.03807615230461\\
4	-3.05410821643287\\
4	-3.07014028056112\\
4	-3.08617234468938\\
4	-3.10220440881764\\
4	-3.11823647294589\\
4	-3.13426853707415\\
4	-3.1503006012024\\
4	-3.16633266533066\\
4	-3.18236472945892\\
4	-3.19839679358717\\
4	-3.21442885771543\\
4	-3.23046092184369\\
4	-3.24649298597194\\
4	-3.2625250501002\\
4	-3.27855711422846\\
4	-3.29458917835671\\
4	-3.31062124248497\\
4	-3.32665330661323\\
4	-3.34268537074148\\
4	-3.35871743486974\\
4	-3.374749498998\\
4	-3.39078156312625\\
4	-3.40681362725451\\
4	-3.42284569138277\\
4	-3.43887775551102\\
4	-3.45490981963928\\
4	-3.47094188376753\\
4	-3.48697394789579\\
4	-3.50300601202405\\
4	-3.5190380761523\\
4	-3.53507014028056\\
4	-3.55110220440882\\
4	-3.56713426853707\\
4	-3.58316633266533\\
4	-3.59919839679359\\
4	-3.61523046092184\\
4	-3.6312625250501\\
4	-3.64729458917836\\
4	-3.66332665330661\\
4	-3.67935871743487\\
4	-3.69539078156313\\
4	-3.71142284569138\\
4	-3.72745490981964\\
4	-3.7434869739479\\
4	-3.75951903807615\\
4	-3.77555110220441\\
4	-3.79158316633267\\
4	-3.80761523046092\\
4	-3.82364729458918\\
4	-3.83967935871743\\
4	-3.85571142284569\\
4	-3.87174348697395\\
4	-3.8877755511022\\
4	-3.90380761523046\\
4	-3.91983967935872\\
4	-3.93587174348697\\
4	-3.95190380761523\\
4	-3.96793587174349\\
4	-3.98396793587174\\
4	-4\\
3.98396793587174	-4\\
3.96793587174349	-4\\
3.95190380761523	-4\\
3.93587174348697	-4\\
3.91983967935872	-4\\
3.90380761523046	-4\\
3.8877755511022	-4\\
3.87174348697395	-4\\
3.85571142284569	-4\\
3.83967935871743	-4\\
3.82364729458918	-4\\
3.80761523046092	-4\\
3.79158316633267	-4\\
3.77555110220441	-4\\
3.75951903807615	-4\\
3.7434869739479	-4\\
3.72745490981964	-4\\
3.71142284569138	-4\\
3.69539078156313	-4\\
3.67935871743487	-4\\
3.66332665330661	-4\\
3.64729458917836	-4\\
3.6312625250501	-4\\
3.61523046092184	-4\\
3.59919839679359	-4\\
3.58316633266533	-4\\
3.56713426853707	-4\\
3.55110220440882	-4\\
3.53507014028056	-4\\
3.51903807615231	-4\\
3.50300601202405	-4\\
3.48697394789579	-4\\
3.47094188376754	-4\\
3.45490981963928	-4\\
3.43887775551102	-4\\
3.42284569138277	-4\\
3.40681362725451	-4\\
3.39078156312625	-4\\
3.374749498998	-4\\
3.35871743486974	-4\\
3.34268537074148	-4\\
3.32665330661323	-4\\
3.31062124248497	-4\\
3.29458917835671	-4\\
3.27855711422846	-4\\
3.2625250501002	-4\\
3.24649298597194	-4\\
3.23046092184369	-4\\
3.21442885771543	-4\\
3.19839679358717	-4\\
3.18236472945892	-4\\
3.16633266533066	-4\\
3.1503006012024	-4\\
3.13426853707415	-4\\
3.11823647294589	-4\\
3.10220440881764	-4\\
3.08617234468938	-4\\
3.07014028056112	-4\\
3.05410821643287	-4\\
3.03807615230461	-4\\
3.02204408817635	-4\\
3.0060120240481	-4\\
2.98997995991984	-4\\
2.97394789579158	-4\\
2.95791583166333	-4\\
2.94188376753507	-4\\
2.92585170340681	-4\\
2.90981963927856	-4\\
2.8937875751503	-4\\
2.87775551102204	-4\\
2.86172344689379	-4\\
2.84569138276553	-4\\
2.82965931863727	-4\\
2.81362725450902	-4\\
2.79759519038076	-4\\
2.7815631262525	-4\\
2.76553106212425	-4\\
2.74949899799599	-4\\
2.73346693386774	-4\\
2.71743486973948	-4\\
2.70140280561122	-4\\
2.68537074148297	-4\\
2.66933867735471	-4\\
2.65330661322645	-4\\
2.6372745490982	-4\\
2.62124248496994	-4\\
2.60521042084168	-4\\
2.58917835671343	-4\\
2.57314629258517	-4\\
2.55711422845691	-4\\
2.54108216432866	-4\\
2.5250501002004	-4\\
2.50901803607214	-4\\
2.49298597194389	-4\\
2.47695390781563	-4\\
2.46092184368737	-4\\
2.44488977955912	-4\\
2.42885771543086	-4\\
2.41282565130261	-4\\
2.39679358717435	-4\\
2.38076152304609	-4\\
2.36472945891784	-4\\
2.34869739478958	-4\\
2.33266533066132	-4\\
2.31663326653307	-4\\
2.30060120240481	-4\\
2.28456913827655	-4\\
2.2685370741483	-4\\
2.25250501002004	-4\\
2.23647294589178	-4\\
2.22044088176353	-4\\
2.20440881763527	-4\\
2.18837675350701	-4\\
2.17234468937876	-4.00000000000001\\
2.1563126252505	-4.00000000000001\\
2.14028056112224	-4.00000000000001\\
2.12424849699399	-4.00000000000001\\
2.10821643286573	-4.00000000000001\\
2.09218436873747	-4.00000000000001\\
2.07615230460922	-4.00000000000001\\
2.06012024048096	-4.00000000000001\\
2.04408817635271	-4.00000000000001\\
2.02805611222445	-4.00000000000001\\
2.01202404809619	-4.00000000000001\\
1.99599198396794	-4.00000000000001\\
1.97995991983968	-4.00000000000001\\
1.96392785571142	-4.00000000000002\\
1.94789579158317	-4.00000000000002\\
1.93186372745491	-4.00000000000002\\
1.91583166332665	-4.00000000000002\\
1.8997995991984	-4.00000000000002\\
1.88376753507014	-4.00000000000002\\
1.86773547094188	-4.00000000000002\\
1.85170340681363	-4.00000000000003\\
1.83567134268537	-4.00000000000003\\
1.81963927855711	-4.00000000000003\\
1.80360721442886	-4.00000000000003\\
1.7875751503006	-4.00000000000004\\
1.77154308617235	-4.00000000000004\\
1.75551102204409	-4.00000000000004\\
1.73947895791583	-4.00000000000005\\
1.72344689378758	-4.00000000000005\\
1.70741482965932	-4.00000000000006\\
1.69138276553106	-4.00000000000006\\
1.67535070140281	-4.00000000000007\\
1.65931863727455	-4.00000000000007\\
1.64328657314629	-4.00000000000008\\
1.62725450901804	-4.00000000000008\\
1.61122244488978	-4.00000000000009\\
1.59519038076152	-4.0000000000001\\
1.57915831663327	-4.0000000000001\\
1.56312625250501	-4.00000000000011\\
1.54709418837675	-4.00000000000012\\
1.5310621242485	-4.00000000000013\\
1.51503006012024	-4.00000000000014\\
1.49899799599198	-4.00000000000015\\
1.48296593186373	-4.00000000000017\\
1.46693386773547	-4.00000000000018\\
1.45090180360721	-4.00000000000019\\
1.43486973947896	-4.00000000000021\\
1.4188376753507	-4.00000000000022\\
1.40280561122244	-4.00000000000024\\
1.38677354709419	-4.00000000000026\\
1.37074148296593	-4.00000000000028\\
1.35470941883768	-4.0000000000003\\
1.33867735470942	-4.00000000000032\\
1.32264529058116	-4.00000000000034\\
1.30661322645291	-4.00000000000037\\
1.29058116232465	-4.00000000000039\\
1.27454909819639	-4.00000000000042\\
1.25851703406814	-4.00000000000045\\
1.24248496993988	-4.00000000000049\\
1.22645290581162	-4.00000000000052\\
1.21042084168337	-4.00000000000056\\
1.19438877755511	-4.0000000000006\\
1.17835671342685	-4.00000000000064\\
1.1623246492986	-4.00000000000068\\
1.14629258517034	-4.00000000000073\\
1.13026052104208	-4.00000000000078\\
1.11422845691383	-4.00000000000084\\
1.09819639278557	-4.00000000000089\\
1.08216432865731	-4.00000000000096\\
1.06613226452906	-4.00000000000102\\
1.0501002004008	-4.00000000000109\\
1.03406813627254	-4.00000000000116\\
1.01803607214429	-4.00000000000124\\
1.00200400801603	-4.00000000000132\\
0.985971943887775	-4.00000000000141\\
0.969939879759519	-4.0000000000015\\
0.953907815631262	-4.0000000000016\\
0.937875751503006	-4.00000000000171\\
0.921843687374749	-4.00000000000182\\
0.905811623246493	-4.00000000000193\\
0.889779559118236	-4.00000000000206\\
0.87374749498998	-4.00000000000219\\
0.857715430861724	-4.00000000000232\\
0.841683366733467	-4.00000000000247\\
0.825651302605211	-4.00000000000262\\
0.809619238476954	-4.00000000000279\\
0.793587174348698	-4.00000000000296\\
0.777555110220441	-4.00000000000314\\
0.761523046092185	-4.00000000000333\\
0.745490981963928	-4.00000000000353\\
0.729458917835672	-4.00000000000375\\
0.713426853707415	-4.00000000000397\\
0.697394789579159	-4.00000000000421\\
0.681362725450902	-4.00000000000446\\
0.665330661322646	-4.00000000000472\\
0.649298597194389	-4.00000000000499\\
0.633266533066132	-4.00000000000528\\
0.617234468937876	-4.00000000000559\\
0.601202404809619	-4.00000000000591\\
0.585170340681363	-4.00000000000625\\
0.569138276553106	-4.0000000000066\\
0.55310621242485	-4.00000000000697\\
0.537074148296593	-4.00000000000736\\
0.521042084168337	-4.00000000000777\\
0.50501002004008	-4.0000000000082\\
0.488977955911824	-4.00000000000865\\
0.472945891783567	-4.00000000000912\\
0.456913827655311	-4.00000000000961\\
0.440881763527054	-4.00000000001013\\
0.424849699398798	-4.00000000001067\\
0.408817635270541	-4.00000000001124\\
0.392785571142285	-4.00000000001183\\
0.376753507014028	-4.00000000001245\\
0.360721442885771	-4.0000000000131\\
0.344689378757515	-4.00000000001377\\
0.328657314629258	-4.00000000001448\\
0.312625250501002	-4.00000000001521\\
0.296593186372745	-4.00000000001598\\
0.280561122244489	-4.00000000001678\\
0.264529058116232	-4.00000000001762\\
0.248496993987976	-4.00000000001849\\
0.232464929859719	-4.00000000001939\\
0.216432865731463	-4.00000000002034\\
0.200400801603206	-4.00000000002132\\
0.18436873747495	-4.00000000002234\\
0.168336673346693	-4.00000000002341\\
0.152304609218437	-4.00000000002451\\
0.13627254509018	-4.00000000002566\\
0.120240480961924	-4.00000000002686\\
0.104208416833667	-4.0000000000281\\
0.0881763527054105	-4.00000000002939\\
0.0721442885771539	-4.00000000003072\\
0.0561122244488974	-4.00000000003211\\
0.0400801603206409	-4.00000000003355\\
0.0240480961923843	-4.00000000003504\\
0.00801603206412782	-4.00000000003658\\
-0.00801603206412826	-4.00000000003818\\
-0.0240480961923848	-4.00000000003983\\
-0.0400801603206413	-4.00000000004154\\
-0.0561122244488979	-4.00000000004331\\
-0.0721442885771544	-4.00000000004514\\
-0.0881763527054109	-4.00000000004703\\
-0.104208416833667	-4.00000000004899\\
-0.120240480961924	-4.000000000051\\
-0.13627254509018	-4.00000000005309\\
-0.152304609218437	-4.00000000005523\\
-0.168336673346694	-4.00000000005745\\
-0.18436873747495	-4.00000000005973\\
-0.200400801603207	-4.00000000006209\\
-0.216432865731463	-4.00000000006451\\
-0.232464929859719	-4.00000000006701\\
-0.248496993987976	-4.00000000006957\\
-0.264529058116232	-4.00000000007222\\
-0.280561122244489	-4.00000000007493\\
-0.296593186372745	-4.00000000007772\\
-0.312625250501002	-4.00000000008059\\
-0.328657314629258	-4.00000000008354\\
-0.344689378757515	-4.00000000008656\\
-0.360721442885771	-4.00000000008966\\
-0.376753507014028	-4.00000000009285\\
-0.392785571142285	-4.00000000009611\\
-0.408817635270541	-4.00000000009945\\
-0.424849699398798	-4.00000000010287\\
-0.440881763527054	-4.00000000010638\\
-0.456913827655311	-4.00000000010996\\
-0.472945891783567	-4.00000000011363\\
-0.488977955911824	-4.00000000011738\\
-0.50501002004008	-4.00000000012121\\
-0.521042084168337	-4.00000000012513\\
-0.537074148296593	-4.00000000012913\\
-0.55310621242485	-4.00000000013321\\
-0.569138276553106	-4.00000000013737\\
-0.585170340681363	-4.00000000014161\\
-0.601202404809619	-4.00000000014593\\
-0.617234468937876	-4.00000000015033\\
-0.633266533066132	-4.00000000015482\\
-0.649298597194389	-4.00000000015938\\
-0.665330661322646	-4.00000000016402\\
-0.681362725450902	-4.00000000016874\\
-0.697394789579158	-4.00000000017353\\
-0.713426853707415	-4.0000000001784\\
-0.729458917835671	-4.00000000018335\\
-0.745490981963928	-4.00000000018836\\
-0.761523046092184	-4.00000000019345\\
-0.777555110220441	-4.0000000001986\\
-0.793587174348697	-4.00000000020383\\
-0.809619238476954	-4.00000000020912\\
-0.82565130260521	-4.00000000021447\\
-0.841683366733467	-4.00000000021988\\
-0.857715430861723	-4.00000000022536\\
-0.87374749498998	-4.00000000023089\\
-0.889779559118236	-4.00000000023647\\
-0.905811623246493	-4.00000000024211\\
-0.92184368737475	-4.0000000002478\\
-0.937875751503006	-4.00000000025353\\
-0.953907815631263	-4.00000000025931\\
-0.969939879759519	-4.00000000026513\\
-0.985971943887776	-4.00000000027098\\
-1.00200400801603	-4.00000000027687\\
-1.01803607214429	-4.0000000002828\\
-1.03406813627255	-4.00000000028875\\
-1.0501002004008	-4.00000000029472\\
-1.06613226452906	-4.00000000030071\\
-1.08216432865731	-4.00000000030672\\
-1.09819639278557	-4.00000000031275\\
-1.11422845691383	-4.00000000031878\\
-1.13026052104208	-4.00000000032482\\
-1.14629258517034	-4.00000000033086\\
-1.1623246492986	-4.00000000033689\\
-1.17835671342685	-4.00000000034292\\
-1.19438877755511	-4.00000000034894\\
-1.21042084168337	-4.00000000035494\\
-1.22645290581162	-4.00000000036091\\
-1.24248496993988	-4.00000000036687\\
-1.25851703406814	-4.00000000037279\\
-1.27454909819639	-4.00000000037869\\
-1.29058116232465	-4.00000000038454\\
-1.30661322645291	-4.00000000039035\\
-1.32264529058116	-4.00000000039611\\
-1.33867735470942	-4.00000000040182\\
-1.35470941883768	-4.00000000040747\\
-1.37074148296593	-4.00000000041306\\
-1.38677354709419	-4.00000000041858\\
-1.40280561122244	-4.00000000042403\\
-1.4188376753507	-4.0000000004294\\
-1.43486973947896	-4.00000000043469\\
-1.45090180360721	-4.0000000004399\\
-1.46693386773547	-4.00000000044502\\
-1.48296593186373	-4.00000000045004\\
-1.49899799599198	-4.00000000045497\\
-1.51503006012024	-4.00000000045979\\
-1.5310621242485	-4.0000000004645\\
-1.54709418837675	-4.0000000004691\\
-1.56312625250501	-4.00000000047358\\
-1.57915831663327	-4.00000000047794\\
-1.59519038076152	-4.00000000048218\\
-1.61122244488978	-4.00000000048628\\
-1.62725450901804	-4.00000000049026\\
-1.64328657314629	-4.0000000004941\\
-1.65931863727455	-4.00000000049779\\
-1.67535070140281	-4.00000000050134\\
-1.69138276553106	-4.00000000050475\\
-1.70741482965932	-4.000000000508\\
-1.72344689378758	-4.0000000005111\\
-1.73947895791583	-4.00000000051405\\
-1.75551102204409	-4.00000000051683\\
-1.77154308617234	-4.00000000051945\\
-1.7875751503006	-4.0000000005219\\
-1.80360721442886	-4.00000000052419\\
-1.81963927855711	-4.0000000005263\\
-1.83567134268537	-4.00000000052824\\
-1.85170340681363	-4.00000000053001\\
-1.86773547094188	-4.0000000005316\\
-1.88376753507014	-4.00000000053302\\
-1.8997995991984	-4.00000000053425\\
-1.91583166332665	-4.00000000053531\\
-1.93186372745491	-4.00000000053618\\
-1.94789579158317	-4.00000000053687\\
-1.96392785571142	-4.00000000053737\\
-1.97995991983968	-4.0000000005377\\
-1.99599198396794	-4.00000000053783\\
-2.01202404809619	-4.00000000053779\\
-2.02805611222445	-4.00000000053756\\
-2.04408817635271	-4.00000000053714\\
-2.06012024048096	-4.00000000053654\\
-2.07615230460922	-4.00000000053576\\
-2.09218436873747	-4.0000000005348\\
-2.10821643286573	-4.00000000053366\\
-2.12424849699399	-4.00000000053233\\
-2.14028056112224	-4.00000000053083\\
-2.1563126252505	-4.00000000052915\\
-2.17234468937876	-4.00000000052729\\
-2.18837675350701	-4.00000000052526\\
-2.20440881763527	-4.00000000052306\\
-2.22044088176353	-4.00000000052069\\
-2.23647294589178	-4.00000000051816\\
-2.25250501002004	-4.00000000051546\\
-2.2685370741483	-4.00000000051259\\
-2.28456913827655	-4.00000000050957\\
-2.30060120240481	-4.0000000005064\\
-2.31663326653307	-4.00000000050307\\
-2.33266533066132	-4.00000000049959\\
-2.34869739478958	-4.00000000049596\\
-2.36472945891784	-4.00000000049219\\
-2.38076152304609	-4.00000000048829\\
-2.39679358717435	-4.00000000048425\\
-2.41282565130261	-4.00000000048007\\
-2.42885771543086	-4.00000000047578\\
-2.44488977955912	-4.00000000047135\\
-2.46092184368737	-4.00000000046681\\
-2.47695390781563	-4.00000000046216\\
-2.49298597194389	-4.00000000045739\\
-2.50901803607214	-4.00000000045252\\
-2.5250501002004	-4.00000000044754\\
-2.54108216432866	-4.00000000044247\\
-2.55711422845691	-4.00000000043731\\
-2.57314629258517	-4.00000000043206\\
-2.58917835671343	-4.00000000042672\\
-2.60521042084168	-4.00000000042131\\
-2.62124248496994	-4.00000000041582\\
-2.6372745490982	-4.00000000041027\\
-2.65330661322645	-4.00000000040465\\
-2.66933867735471	-4.00000000039897\\
-2.68537074148297	-4.00000000039323\\
-2.70140280561122	-4.00000000038745\\
-2.71743486973948	-4.00000000038162\\
-2.73346693386774	-4.00000000037574\\
-2.74949899799599	-4.00000000036984\\
-2.76553106212425	-4.0000000003639\\
-2.7815631262525	-4.00000000035793\\
-2.79759519038076	-4.00000000035194\\
-2.81362725450902	-4.00000000034593\\
-2.82965931863727	-4.00000000033991\\
-2.84569138276553	-4.00000000033387\\
-2.86172344689379	-4.00000000032784\\
-2.87775551102204	-4.0000000003218\\
-2.8937875751503	-4.00000000031576\\
-2.90981963927856	-4.00000000030973\\
-2.92585170340681	-4.00000000030372\\
-2.94188376753507	-4.00000000029771\\
-2.95791583166333	-4.00000000029173\\
-2.97394789579158	-4.00000000028577\\
-2.98997995991984	-4.00000000027983\\
-3.0060120240481	-4.00000000027392\\
-3.02204408817635	-4.00000000026805\\
-3.03807615230461	-4.00000000026221\\
-3.05410821643287	-4.00000000025641\\
-3.07014028056112	-4.00000000025066\\
-3.08617234468938	-4.00000000024495\\
-3.10220440881764	-4.00000000023929\\
-3.11823647294589	-4.00000000023367\\
-3.13426853707415	-4.00000000022811\\
-3.1503006012024	-4.00000000022261\\
-3.16633266533066	-4.00000000021717\\
-3.18236472945892	-4.00000000021178\\
-3.19839679358717	-4.00000000020646\\
-3.21442885771543	-4.00000000020121\\
-3.23046092184369	-4.00000000019602\\
-3.24649298597194	-4.0000000001909\\
-3.2625250501002	-4.00000000018585\\
-3.27855711422846	-4.00000000018087\\
-3.29458917835671	-4.00000000017596\\
-3.31062124248497	-4.00000000017113\\
-3.32665330661323	-4.00000000016637\\
-3.34268537074148	-4.00000000016169\\
-3.35871743486974	-4.00000000015709\\
-3.374749498998	-4.00000000015257\\
-3.39078156312625	-4.00000000014812\\
-3.40681362725451	-4.00000000014376\\
-3.42284569138277	-4.00000000013948\\
-3.43887775551102	-4.00000000013528\\
-3.45490981963928	-4.00000000013116\\
-3.47094188376753	-4.00000000012712\\
-3.48697394789579	-4.00000000012316\\
-3.50300601202405	-4.00000000011929\\
-3.5190380761523	-4.0000000001155\\
-3.53507014028056	-4.00000000011179\\
-3.55110220440882	-4.00000000010816\\
-3.56713426853707	-4.00000000010461\\
-3.58316633266533	-4.00000000010115\\
-3.59919839679359	-4.00000000009777\\
-3.61523046092184	-4.00000000009447\\
-3.6312625250501	-4.00000000009125\\
-3.64729458917836	-4.0000000000881\\
-3.66332665330661	-4.00000000008504\\
-3.67935871743487	-4.00000000008206\\
-3.69539078156313	-4.00000000007915\\
-3.71142284569138	-4.00000000007632\\
-3.72745490981964	-4.00000000007356\\
-3.7434869739479	-4.00000000007089\\
-3.75951903807615	-4.00000000006828\\
-3.77555110220441	-4.00000000006575\\
-3.79158316633267	-4.00000000006329\\
-3.80761523046092	-4.0000000000609\\
-3.82364729458918	-4.00000000005858\\
-3.83967935871743	-4.00000000005633\\
-3.85571142284569	-4.00000000005415\\
-3.87174348697395	-4.00000000005204\\
-3.8877755511022	-4.00000000004999\\
-3.90380761523046	-4.000000000048\\
-3.91983967935872	-4.00000000004608\\
-3.93587174348697	-4.00000000004422\\
-3.95190380761523	-4.00000000004242\\
-3.96793587174349	-4.00000000004068\\
-3.98396793587174	-4.000000000039\\
-4	-4.00000000003737\\
-4.00000000003737	-4\\
-4.000000000039	-3.98396793587174\\
-4.00000000004068	-3.96793587174349\\
-4.00000000004242	-3.95190380761523\\
-4.00000000004422	-3.93587174348697\\
-4.00000000004608	-3.91983967935872\\
-4.000000000048	-3.90380761523046\\
-4.00000000004999	-3.8877755511022\\
-4.00000000005204	-3.87174348697395\\
-4.00000000005415	-3.85571142284569\\
-4.00000000005633	-3.83967935871743\\
-4.00000000005858	-3.82364729458918\\
-4.0000000000609	-3.80761523046092\\
-4.00000000006329	-3.79158316633267\\
-4.00000000006575	-3.77555110220441\\
-4.00000000006828	-3.75951903807615\\
-4.00000000007089	-3.7434869739479\\
-4.00000000007356	-3.72745490981964\\
-4.00000000007632	-3.71142284569138\\
-4.00000000007915	-3.69539078156313\\
-4.00000000008206	-3.67935871743487\\
-4.00000000008504	-3.66332665330661\\
-4.0000000000881	-3.64729458917836\\
-4.00000000009125	-3.6312625250501\\
-4.00000000009447	-3.61523046092184\\
-4.00000000009777	-3.59919839679359\\
-4.00000000010115	-3.58316633266533\\
-4.00000000010461	-3.56713426853707\\
-4.00000000010816	-3.55110220440882\\
-4.00000000011179	-3.53507014028056\\
-4.0000000001155	-3.5190380761523\\
-4.00000000011929	-3.50300601202405\\
-4.00000000012316	-3.48697394789579\\
-4.00000000012712	-3.47094188376753\\
-4.00000000013116	-3.45490981963928\\
-4.00000000013528	-3.43887775551102\\
-4.00000000013948	-3.42284569138277\\
-4.00000000014376	-3.40681362725451\\
-4.00000000014812	-3.39078156312625\\
-4.00000000015257	-3.374749498998\\
-4.00000000015709	-3.35871743486974\\
-4.00000000016169	-3.34268537074148\\
-4.00000000016637	-3.32665330661323\\
-4.00000000017113	-3.31062124248497\\
-4.00000000017596	-3.29458917835671\\
-4.00000000018087	-3.27855711422846\\
-4.00000000018585	-3.2625250501002\\
-4.0000000001909	-3.24649298597194\\
-4.00000000019602	-3.23046092184369\\
-4.00000000020121	-3.21442885771543\\
-4.00000000020646	-3.19839679358717\\
-4.00000000021178	-3.18236472945892\\
-4.00000000021717	-3.16633266533066\\
-4.00000000022261	-3.1503006012024\\
-4.00000000022811	-3.13426853707415\\
-4.00000000023367	-3.11823647294589\\
-4.00000000023929	-3.10220440881764\\
-4.00000000024495	-3.08617234468938\\
-4.00000000025066	-3.07014028056112\\
-4.00000000025641	-3.05410821643287\\
-4.00000000026221	-3.03807615230461\\
-4.00000000026805	-3.02204408817635\\
-4.00000000027392	-3.0060120240481\\
-4.00000000027983	-2.98997995991984\\
-4.00000000028577	-2.97394789579158\\
-4.00000000029173	-2.95791583166333\\
-4.00000000029771	-2.94188376753507\\
-4.00000000030372	-2.92585170340681\\
-4.00000000030973	-2.90981963927856\\
-4.00000000031576	-2.8937875751503\\
-4.0000000003218	-2.87775551102204\\
-4.00000000032784	-2.86172344689379\\
-4.00000000033387	-2.84569138276553\\
-4.00000000033991	-2.82965931863727\\
-4.00000000034593	-2.81362725450902\\
-4.00000000035194	-2.79759519038076\\
-4.00000000035793	-2.7815631262525\\
-4.0000000003639	-2.76553106212425\\
-4.00000000036984	-2.74949899799599\\
-4.00000000037574	-2.73346693386774\\
-4.00000000038162	-2.71743486973948\\
-4.00000000038745	-2.70140280561122\\
-4.00000000039323	-2.68537074148297\\
-4.00000000039897	-2.66933867735471\\
-4.00000000040465	-2.65330661322645\\
-4.00000000041027	-2.6372745490982\\
-4.00000000041582	-2.62124248496994\\
-4.00000000042131	-2.60521042084168\\
-4.00000000042672	-2.58917835671343\\
-4.00000000043206	-2.57314629258517\\
-4.00000000043731	-2.55711422845691\\
-4.00000000044247	-2.54108216432866\\
-4.00000000044754	-2.5250501002004\\
-4.00000000045252	-2.50901803607214\\
-4.00000000045739	-2.49298597194389\\
-4.00000000046216	-2.47695390781563\\
-4.00000000046681	-2.46092184368737\\
-4.00000000047135	-2.44488977955912\\
-4.00000000047578	-2.42885771543086\\
-4.00000000048007	-2.41282565130261\\
-4.00000000048425	-2.39679358717435\\
-4.00000000048829	-2.38076152304609\\
-4.00000000049219	-2.36472945891784\\
-4.00000000049596	-2.34869739478958\\
-4.00000000049959	-2.33266533066132\\
-4.00000000050307	-2.31663326653307\\
-4.0000000005064	-2.30060120240481\\
-4.00000000050957	-2.28456913827655\\
-4.00000000051259	-2.2685370741483\\
-4.00000000051546	-2.25250501002004\\
-4.00000000051816	-2.23647294589178\\
-4.00000000052069	-2.22044088176353\\
-4.00000000052306	-2.20440881763527\\
-4.00000000052526	-2.18837675350701\\
-4.00000000052729	-2.17234468937876\\
-4.00000000052915	-2.1563126252505\\
-4.00000000053083	-2.14028056112224\\
-4.00000000053233	-2.12424849699399\\
-4.00000000053366	-2.10821643286573\\
-4.0000000005348	-2.09218436873747\\
-4.00000000053576	-2.07615230460922\\
-4.00000000053654	-2.06012024048096\\
-4.00000000053714	-2.04408817635271\\
-4.00000000053756	-2.02805611222445\\
-4.00000000053779	-2.01202404809619\\
-4.00000000053783	-1.99599198396794\\
-4.0000000005377	-1.97995991983968\\
-4.00000000053737	-1.96392785571142\\
-4.00000000053687	-1.94789579158317\\
-4.00000000053618	-1.93186372745491\\
-4.00000000053531	-1.91583166332665\\
-4.00000000053425	-1.8997995991984\\
-4.00000000053302	-1.88376753507014\\
-4.0000000005316	-1.86773547094188\\
-4.00000000053001	-1.85170340681363\\
-4.00000000052824	-1.83567134268537\\
-4.0000000005263	-1.81963927855711\\
-4.00000000052419	-1.80360721442886\\
-4.0000000005219	-1.7875751503006\\
-4.00000000051945	-1.77154308617234\\
-4.00000000051683	-1.75551102204409\\
-4.00000000051405	-1.73947895791583\\
-4.0000000005111	-1.72344689378758\\
-4.000000000508	-1.70741482965932\\
-4.00000000050475	-1.69138276553106\\
-4.00000000050134	-1.67535070140281\\
-4.00000000049779	-1.65931863727455\\
-4.0000000004941	-1.64328657314629\\
-4.00000000049026	-1.62725450901804\\
-4.00000000048628	-1.61122244488978\\
-4.00000000048218	-1.59519038076152\\
-4.00000000047794	-1.57915831663327\\
-4.00000000047358	-1.56312625250501\\
-4.0000000004691	-1.54709418837675\\
-4.0000000004645	-1.5310621242485\\
-4.00000000045979	-1.51503006012024\\
-4.00000000045497	-1.49899799599198\\
-4.00000000045004	-1.48296593186373\\
-4.00000000044502	-1.46693386773547\\
-4.0000000004399	-1.45090180360721\\
-4.00000000043469	-1.43486973947896\\
-4.0000000004294	-1.4188376753507\\
-4.00000000042403	-1.40280561122244\\
-4.00000000041858	-1.38677354709419\\
-4.00000000041306	-1.37074148296593\\
-4.00000000040747	-1.35470941883768\\
-4.00000000040182	-1.33867735470942\\
-4.00000000039611	-1.32264529058116\\
-4.00000000039035	-1.30661322645291\\
-4.00000000038454	-1.29058116232465\\
-4.00000000037869	-1.27454909819639\\
-4.00000000037279	-1.25851703406814\\
-4.00000000036687	-1.24248496993988\\
-4.00000000036091	-1.22645290581162\\
-4.00000000035494	-1.21042084168337\\
-4.00000000034894	-1.19438877755511\\
-4.00000000034292	-1.17835671342685\\
-4.00000000033689	-1.1623246492986\\
-4.00000000033086	-1.14629258517034\\
-4.00000000032482	-1.13026052104208\\
-4.00000000031878	-1.11422845691383\\
-4.00000000031275	-1.09819639278557\\
-4.00000000030672	-1.08216432865731\\
-4.00000000030071	-1.06613226452906\\
-4.00000000029472	-1.0501002004008\\
-4.00000000028875	-1.03406813627255\\
-4.0000000002828	-1.01803607214429\\
-4.00000000027687	-1.00200400801603\\
-4.00000000027098	-0.985971943887776\\
-4.00000000026513	-0.969939879759519\\
-4.00000000025931	-0.953907815631263\\
-4.00000000025353	-0.937875751503006\\
-4.0000000002478	-0.92184368737475\\
-4.00000000024211	-0.905811623246493\\
-4.00000000023647	-0.889779559118236\\
-4.00000000023089	-0.87374749498998\\
-4.00000000022536	-0.857715430861723\\
-4.00000000021988	-0.841683366733467\\
-4.00000000021447	-0.82565130260521\\
-4.00000000020912	-0.809619238476954\\
-4.00000000020383	-0.793587174348697\\
-4.0000000001986	-0.777555110220441\\
-4.00000000019345	-0.761523046092184\\
-4.00000000018836	-0.745490981963928\\
-4.00000000018335	-0.729458917835671\\
-4.0000000001784	-0.713426853707415\\
-4.00000000017353	-0.697394789579158\\
-4.00000000016874	-0.681362725450902\\
-4.00000000016402	-0.665330661322646\\
-4.00000000015938	-0.649298597194389\\
-4.00000000015482	-0.633266533066132\\
-4.00000000015033	-0.617234468937876\\
-4.00000000014593	-0.601202404809619\\
-4.00000000014161	-0.585170340681363\\
-4.00000000013737	-0.569138276553106\\
-4.00000000013321	-0.55310621242485\\
-4.00000000012913	-0.537074148296593\\
-4.00000000012513	-0.521042084168337\\
-4.00000000012121	-0.50501002004008\\
-4.00000000011738	-0.488977955911824\\
-4.00000000011363	-0.472945891783567\\
-4.00000000010996	-0.456913827655311\\
-4.00000000010638	-0.440881763527054\\
-4.00000000010287	-0.424849699398798\\
-4.00000000009945	-0.408817635270541\\
-4.00000000009611	-0.392785571142285\\
-4.00000000009285	-0.376753507014028\\
-4.00000000008966	-0.360721442885771\\
-4.00000000008656	-0.344689378757515\\
-4.00000000008354	-0.328657314629258\\
-4.00000000008059	-0.312625250501002\\
-4.00000000007772	-0.296593186372745\\
-4.00000000007493	-0.280561122244489\\
-4.00000000007222	-0.264529058116232\\
-4.00000000006957	-0.248496993987976\\
-4.00000000006701	-0.232464929859719\\
-4.00000000006451	-0.216432865731463\\
-4.00000000006209	-0.200400801603207\\
-4.00000000005973	-0.18436873747495\\
-4.00000000005745	-0.168336673346694\\
-4.00000000005523	-0.152304609218437\\
-4.00000000005309	-0.13627254509018\\
-4.000000000051	-0.120240480961924\\
-4.00000000004899	-0.104208416833667\\
-4.00000000004703	-0.0881763527054109\\
-4.00000000004514	-0.0721442885771544\\
-4.00000000004331	-0.0561122244488979\\
-4.00000000004154	-0.0400801603206413\\
-4.00000000003983	-0.0240480961923848\\
-4.00000000003818	-0.00801603206412826\\
-4.00000000003658	0.00801603206412782\\
-4.00000000003504	0.0240480961923843\\
-4.00000000003355	0.0400801603206409\\
-4.00000000003211	0.0561122244488974\\
-4.00000000003072	0.0721442885771539\\
-4.00000000002939	0.0881763527054105\\
-4.0000000000281	0.104208416833667\\
-4.00000000002686	0.120240480961924\\
-4.00000000002566	0.13627254509018\\
-4.00000000002451	0.152304609218437\\
-4.00000000002341	0.168336673346693\\
-4.00000000002234	0.18436873747495\\
-4.00000000002132	0.200400801603206\\
-4.00000000002034	0.216432865731463\\
-4.00000000001939	0.232464929859719\\
-4.00000000001849	0.248496993987976\\
-4.00000000001762	0.264529058116232\\
-4.00000000001678	0.280561122244489\\
-4.00000000001598	0.296593186372745\\
-4.00000000001521	0.312625250501002\\
-4.00000000001448	0.328657314629258\\
-4.00000000001377	0.344689378757515\\
-4.0000000000131	0.360721442885771\\
-4.00000000001245	0.376753507014028\\
-4.00000000001183	0.392785571142285\\
-4.00000000001124	0.408817635270541\\
-4.00000000001067	0.424849699398798\\
-4.00000000001013	0.440881763527054\\
-4.00000000000961	0.456913827655311\\
-4.00000000000912	0.472945891783567\\
-4.00000000000865	0.488977955911824\\
-4.0000000000082	0.50501002004008\\
-4.00000000000777	0.521042084168337\\
-4.00000000000736	0.537074148296593\\
-4.00000000000697	0.55310621242485\\
-4.0000000000066	0.569138276553106\\
-4.00000000000625	0.585170340681363\\
-4.00000000000591	0.601202404809619\\
-4.00000000000559	0.617234468937876\\
-4.00000000000528	0.633266533066132\\
-4.00000000000499	0.649298597194389\\
-4.00000000000472	0.665330661322646\\
-4.00000000000446	0.681362725450902\\
-4.00000000000421	0.697394789579159\\
-4.00000000000397	0.713426853707415\\
-4.00000000000375	0.729458917835672\\
-4.00000000000353	0.745490981963928\\
-4.00000000000333	0.761523046092185\\
-4.00000000000314	0.777555110220441\\
-4.00000000000296	0.793587174348698\\
-4.00000000000279	0.809619238476954\\
-4.00000000000262	0.825651302605211\\
-4.00000000000247	0.841683366733467\\
-4.00000000000232	0.857715430861724\\
-4.00000000000219	0.87374749498998\\
-4.00000000000206	0.889779559118236\\
-4.00000000000193	0.905811623246493\\
-4.00000000000182	0.921843687374749\\
-4.00000000000171	0.937875751503006\\
-4.0000000000016	0.953907815631262\\
-4.0000000000015	0.969939879759519\\
-4.00000000000141	0.985971943887775\\
-4.00000000000132	1.00200400801603\\
-4.00000000000124	1.01803607214429\\
-4.00000000000116	1.03406813627254\\
-4.00000000000109	1.0501002004008\\
-4.00000000000102	1.06613226452906\\
-4.00000000000096	1.08216432865731\\
-4.00000000000089	1.09819639278557\\
-4.00000000000084	1.11422845691383\\
-4.00000000000078	1.13026052104208\\
-4.00000000000073	1.14629258517034\\
-4.00000000000068	1.1623246492986\\
-4.00000000000064	1.17835671342685\\
-4.0000000000006	1.19438877755511\\
-4.00000000000056	1.21042084168337\\
-4.00000000000052	1.22645290581162\\
-4.00000000000049	1.24248496993988\\
-4.00000000000045	1.25851703406814\\
-4.00000000000042	1.27454909819639\\
-4.00000000000039	1.29058116232465\\
-4.00000000000037	1.30661322645291\\
-4.00000000000034	1.32264529058116\\
-4.00000000000032	1.33867735470942\\
-4.0000000000003	1.35470941883768\\
-4.00000000000028	1.37074148296593\\
-4.00000000000026	1.38677354709419\\
-4.00000000000024	1.40280561122244\\
-4.00000000000022	1.4188376753507\\
-4.00000000000021	1.43486973947896\\
-4.00000000000019	1.45090180360721\\
-4.00000000000018	1.46693386773547\\
-4.00000000000017	1.48296593186373\\
-4.00000000000015	1.49899799599198\\
-4.00000000000014	1.51503006012024\\
-4.00000000000013	1.5310621242485\\
-4.00000000000012	1.54709418837675\\
-4.00000000000011	1.56312625250501\\
-4.0000000000001	1.57915831663327\\
-4.0000000000001	1.59519038076152\\
-4.00000000000009	1.61122244488978\\
-4.00000000000008	1.62725450901804\\
-4.00000000000008	1.64328657314629\\
-4.00000000000007	1.65931863727455\\
-4.00000000000007	1.67535070140281\\
-4.00000000000006	1.69138276553106\\
-4.00000000000006	1.70741482965932\\
-4.00000000000005	1.72344689378758\\
-4.00000000000005	1.73947895791583\\
-4.00000000000004	1.75551102204409\\
-4.00000000000004	1.77154308617235\\
-4.00000000000004	1.7875751503006\\
-4.00000000000003	1.80360721442886\\
-4.00000000000003	1.81963927855711\\
-4.00000000000003	1.83567134268537\\
-4.00000000000003	1.85170340681363\\
-4.00000000000002	1.86773547094188\\
-4.00000000000002	1.88376753507014\\
-4.00000000000002	1.8997995991984\\
-4.00000000000002	1.91583166332665\\
-4.00000000000002	1.93186372745491\\
-4.00000000000002	1.94789579158317\\
-4.00000000000002	1.96392785571142\\
-4.00000000000001	1.97995991983968\\
-4.00000000000001	1.99599198396794\\
-4.00000000000001	2.01202404809619\\
-4.00000000000001	2.02805611222445\\
-4.00000000000001	2.04408817635271\\
-4.00000000000001	2.06012024048096\\
-4.00000000000001	2.07615230460922\\
-4.00000000000001	2.09218436873747\\
-4.00000000000001	2.10821643286573\\
-4.00000000000001	2.12424849699399\\
-4.00000000000001	2.14028056112224\\
-4.00000000000001	2.1563126252505\\
-4.00000000000001	2.17234468937876\\
-4	2.18837675350701\\
-4	2.20440881763527\\
-4	2.22044088176353\\
-4	2.23647294589178\\
-4	2.25250501002004\\
-4	2.2685370741483\\
-4	2.28456913827655\\
-4	2.30060120240481\\
-4	2.31663326653307\\
-4	2.33266533066132\\
-4	2.34869739478958\\
-4	2.36472945891784\\
-4	2.38076152304609\\
-4	2.39679358717435\\
-4	2.41282565130261\\
-4	2.42885771543086\\
-4	2.44488977955912\\
-4	2.46092184368737\\
-4	2.47695390781563\\
-4	2.49298597194389\\
-4	2.50901803607214\\
-4	2.5250501002004\\
-4	2.54108216432866\\
-4	2.55711422845691\\
-4	2.57314629258517\\
-4	2.58917835671343\\
-4	2.60521042084168\\
-4	2.62124248496994\\
-4	2.6372745490982\\
-4	2.65330661322645\\
-4	2.66933867735471\\
-4	2.68537074148297\\
-4	2.70140280561122\\
-4	2.71743486973948\\
-4	2.73346693386774\\
-4	2.74949899799599\\
-4	2.76553106212425\\
-4	2.7815631262525\\
-4	2.79759519038076\\
-4	2.81362725450902\\
-4	2.82965931863727\\
-4	2.84569138276553\\
-4	2.86172344689379\\
-4	2.87775551102204\\
-4	2.8937875751503\\
-4	2.90981963927856\\
-4	2.92585170340681\\
-4	2.94188376753507\\
-4	2.95791583166333\\
-4	2.97394789579158\\
-4	2.98997995991984\\
-4	3.0060120240481\\
-4	3.02204408817635\\
-4	3.03807615230461\\
-4	3.05410821643287\\
-4	3.07014028056112\\
-4	3.08617234468938\\
-4	3.10220440881764\\
-4	3.11823647294589\\
-4	3.13426853707415\\
-4	3.1503006012024\\
-4	3.16633266533066\\
-4	3.18236472945892\\
-4	3.19839679358717\\
-4	3.21442885771543\\
-4	3.23046092184369\\
-4	3.24649298597194\\
-4	3.2625250501002\\
-4	3.27855711422846\\
-4	3.29458917835671\\
-4	3.31062124248497\\
-4	3.32665330661323\\
-4	3.34268537074148\\
-4	3.35871743486974\\
-4	3.374749498998\\
-4	3.39078156312625\\
-4	3.40681362725451\\
-4	3.42284569138277\\
-4	3.43887775551102\\
-4	3.45490981963928\\
-4	3.47094188376754\\
-4	3.48697394789579\\
-4	3.50300601202405\\
-4	3.51903807615231\\
-4	3.53507014028056\\
-4	3.55110220440882\\
-4	3.56713426853707\\
-4	3.58316633266533\\
-4	3.59919839679359\\
-4	3.61523046092184\\
-4	3.6312625250501\\
-4	3.64729458917836\\
-4	3.66332665330661\\
-4	3.67935871743487\\
-4	3.69539078156313\\
-4	3.71142284569138\\
-4	3.72745490981964\\
-4	3.7434869739479\\
-4	3.75951903807615\\
-4	3.77555110220441\\
-4	3.79158316633267\\
-4	3.80761523046092\\
-4	3.82364729458918\\
-4	3.83967935871743\\
-4	3.85571142284569\\
-4	3.87174348697395\\
-4	3.8877755511022\\
-4	3.90380761523046\\
-4	3.91983967935872\\
-4	3.93587174348697\\
-4	3.95190380761523\\
-4	3.96793587174349\\
-4	3.98396793587174\\
-4	4\\
}--cycle;


\addplot[area legend,solid,fill=mycolor2,draw=black,forget plot]
table[row sep=crcr] {%
x	y\\
-2.09218436873747	-0.836330035182676\\
-2.09084818907255	-0.82565130260521\\
-2.08870087255636	-0.809619238476954\\
-2.08641228950813	-0.793587174348697\\
-2.08398206346184	-0.777555110220441\\
-2.08140964485727	-0.761523046092184\\
-2.07869431061906	-0.745490981963928\\
-2.07615230460922	-0.731234316369284\\
-2.07584367793316	-0.729458917835671\\
-2.07292099143853	-0.713426853707415\\
-2.06985740859541	-0.697394789579158\\
-2.06665162814523	-0.681362725450902\\
-2.06330217413261	-0.665330661322646\\
-2.06012024048096	-0.650729265294393\\
-2.05981598892093	-0.649298597194389\\
-2.05627493125352	-0.633266533066132\\
-2.0525903722525	-0.617234468937876\\
-2.04876026202671	-0.601202404809619\\
-2.04478236859634	-0.585170340681363\\
-2.04408817635271	-0.582462999001817\\
-2.04075075529471	-0.569138276553106\\
-2.03659172603068	-0.55310621242485\\
-2.03228328811775	-0.537074148296593\\
-2.02805611222445	-0.521876581220279\\
-2.02782922718771	-0.521042084168337\\
-2.02334495314417	-0.50501002004008\\
-2.01870865987881	-0.488977955911824\\
-2.01391711244512	-0.472945891783567\\
-2.01202404809619	-0.46678777122706\\
-2.00905466022014	-0.456913827655311\\
-2.00409056711776	-0.440881763527054\\
-1.99896720468222	-0.424849699398798\\
-1.99599198396794	-0.415795209752654\\
-1.99374754557004	-0.408817635270541\\
-1.98845312818685	-0.392785571142285\\
-1.98299442095986	-0.376753507014028\\
-1.97995991983968	-0.368072784278691\\
-1.97744261981634	-0.360721442885771\\
-1.97181279512412	-0.344689378757515\\
-1.96601263028277	-0.328657314629258\\
-1.96392785571142	-0.323028016018515\\
-1.96015144129628	-0.312625250501002\\
-1.95417855750996	-0.296593186372745\\
-1.94802821710527	-0.280561122244489\\
-1.94789579158317	-0.280222369316579\\
-1.94187826005622	-0.264529058116232\\
-1.93555205510367	-0.248496993987976\\
-1.93186372745491	-0.239372439405238\\
-1.9291235561351	-0.232464929859719\\
-1.92262007290555	-0.216432865731463\\
-1.91592760119292	-0.200400801603207\\
-1.91583166332665	-0.20017481680906\\
-1.90924200950305	-0.18436873747495\\
-1.90236659048268	-0.168336673346694\\
-1.8997995991984	-0.162472699644748\\
-1.89542616460125	-0.152304609218437\\
-1.88836484201108	-0.13627254509018\\
-1.88376753507014	-0.126078357635205\\
-1.88117944188881	-0.120240480961924\\
-1.87392866962192	-0.104208416833667\\
-1.86773547094188	-0.0908662865788124\\
-1.86650738489166	-0.0881763527054109\\
-1.85906302184108	-0.0721442885771544\\
-1.85170340681363	-0.0567278387126689\\
-1.8514142056704	-0.0561122244488979\\
-1.84377150745912	-0.0400801603206413\\
-1.83590990103079	-0.0240480961923848\\
-1.83567134268537	-0.023568326147905\\
-1.82805641414634	-0.00801603206412826\\
-1.8199850788953	0.00801603206412782\\
-1.81963927855711	0.00869358889537093\\
-1.81191872023512	0.0240480961923843\\
-1.80363120607876	0.0400801603206409\\
-1.80360721442886	0.04012599089673\\
-1.79535809975437	0.0561122244488974\\
-1.7875751503006	0.0707933414835986\\
-1.78686893509702	0.0721442885771539\\
-1.77837292075237	0.0881763527054105\\
-1.77154308617234	0.100747303667208\\
-1.76968786917051	0.104208416833667\\
-1.76096023689598	0.120240480961924\\
-1.75551102204409	0.130033023613417\\
-1.75208373990267	0.13627254509018\\
-1.74311577228557	0.152304609218437\\
-1.73947895791583	0.158689220974882\\
-1.73405167264888	0.168336673346693\\
-1.72483389937621	0.18436873747495\\
-1.72344689378758	0.18674875500433\\
-1.71558544783079	0.200400801603206\\
-1.70741482965932	0.214256473023492\\
-1.70614621853222	0.216432865731463\\
-1.69667746869006	0.232464929859719\\
-1.69138276553106	0.241247104283951\\
-1.68706003540055	0.248496993987976\\
-1.67731872176217	0.264529058116232\\
-1.67535070140281	0.267723468721069\\
-1.6675247902261	0.280561122244489\\
-1.65931863727455	0.293729391380358\\
-1.65755210885658	0.296593186372745\\
-1.64752942021547	0.312625250501002\\
-1.64328657314629	0.319292043707351\\
-1.6373838586197	0.328657314629258\\
-1.62725450901804	0.344395257151599\\
-1.62706696249171	0.344689378757515\\
-1.61674306778591	0.360721442885771\\
-1.61122244488978	0.369125324313253\\
-1.60625536673355	0.376753507014028\\
-1.59561495069234	0.392785571142285\\
-1.59519038076152	0.393419074108392\\
-1.58495566667189	0.408817635270541\\
-1.57915831663327	0.41737359708186\\
-1.57413271058397	0.424849699398798\\
-1.56315077044347	0.440881763527054\\
-1.56312625250501	0.440917249974618\\
-1.55215580674429	0.456913827655311\\
-1.54709418837675	0.464161096508277\\
-1.54100190570865	0.472945891783567\\
-1.5310621242485	0.487025080597279\\
-1.52969261560049	0.488977955911824\\
-1.51832452342484	0.50501002004008\\
-1.51503006012024	0.509589305568402\\
-1.50684141353711	0.521042084168337\\
-1.49899799599198	0.531828963983769\\
-1.49520622302631	0.537074148296593\\
-1.48343506295179	0.55310621242485\\
-1.48296593186373	0.553739649023431\\
-1.47162187478752	0.569138276553106\\
-1.46693386773547	0.575402002446096\\
-1.45965892841028	0.585170340681363\\
-1.45090180360721	0.596748747598595\\
-1.4475487490333	0.601202404809619\\
-1.4353057453109	0.617234468937876\\
-1.43486973947896	0.617800762670221\\
-1.42301051043611	0.633266533066132\\
-1.4188376753507	0.638630392780364\\
-1.41056897016755	0.649298597194389\\
-1.40280561122244	0.659174997971831\\
-1.39798306621906	0.665330661322646\\
-1.38677354709419	0.679444486071188\\
-1.38525456128759	0.681362725450902\\
-1.3724307481579	0.697394789579159\\
-1.37074148296593	0.699485488778341\\
-1.35950466155993	0.713426853707415\\
-1.35470941883768	0.719301238459291\\
-1.34643508799382	0.729458917835672\\
-1.33867735470942	0.738865882965851\\
-1.33322321551226	0.745490981963928\\
-1.32264529058116	0.758187441716244\\
-1.31987005815617	0.761523046092185\\
-1.30661322645291	0.777273522548638\\
-1.30637645725026	0.777555110220441\\
-1.29280128824555	0.793587174348698\\
-1.29058116232465	0.796181734635507\\
-1.27908848480037	0.809619238476954\\
-1.27454909819639	0.814870935176106\\
-1.26523204637232	0.825651302605211\\
-1.25851703406814	0.833342057411939\\
-1.25123222431763	0.841683366733467\\
-1.24248496993988	0.851601198605366\\
-1.23708909456899	0.857715430861724\\
-1.22645290581162	0.869654123233241\\
-1.22280255716274	0.87374749498998\\
-1.21042084168337	0.887506275980729\\
-1.20837233543093	0.889779559118236\\
-1.19438877755511	0.905162793967734\\
-1.19379797485529	0.905811623246493\\
-1.17909731511517	0.921843687374749\\
-1.17835671342685	0.92264572608842\\
-1.16426203204616	0.937875751503006\\
-1.1623246492986	0.939953035042293\\
-1.14927595227682	0.953907815631262\\
-1.14629258517034	0.957074890637135\\
-1.13413795978372	0.969939879759519\\
-1.13026052104208	0.974015270485195\\
-1.11884674649862	0.985971943887775\\
-1.11422845691383	0.990777893947399\\
-1.10340080916829	1.00200400801603\\
-1.09819639278557	1.00736623032159\\
-1.08779844582577	1.01803607214429\\
-1.08216432865731	1.02378350648121\\
-1.0720377518629	1.03406813627254\\
-1.06613226452906	1.04003271398696\\
-1.0561166156924	1.0501002004008\\
-1.0501002004008	1.0561166156924\\
-1.04003271398696	1.06613226452906\\
-1.03406813627255	1.07203775186289\\
-1.02378350648121	1.08216432865731\\
-1.01803607214429	1.08779844582577\\
-1.00736623032159	1.09819639278557\\
-1.00200400801603	1.10340080916829\\
-0.990777893947399	1.11422845691383\\
-0.985971943887776	1.11884674649862\\
-0.974015270485194	1.13026052104208\\
-0.969939879759519	1.13413795978371\\
-0.957074890637135	1.14629258517034\\
-0.953907815631263	1.14927595227682\\
-0.939953035042293	1.1623246492986\\
-0.937875751503006	1.16426203204616\\
-0.92264572608842	1.17835671342685\\
-0.92184368737475	1.17909731511517\\
-0.905811623246493	1.19379797485529\\
-0.905162793967734	1.19438877755511\\
-0.889779559118236	1.20837233543093\\
-0.887506275980729	1.21042084168337\\
-0.87374749498998	1.22280255716274\\
-0.869654123233241	1.22645290581162\\
-0.857715430861723	1.23708909456899\\
-0.851601198605366	1.24248496993988\\
-0.841683366733467	1.25123222431763\\
-0.833342057411939	1.25851703406814\\
-0.82565130260521	1.26523204637232\\
-0.814870935176106	1.27454909819639\\
-0.809619238476954	1.27908848480037\\
-0.796181734635507	1.29058116232465\\
-0.793587174348697	1.29280128824555\\
-0.777555110220441	1.30637645725026\\
-0.777273522548639	1.30661322645291\\
-0.761523046092184	1.31987005815617\\
-0.758187441716244	1.32264529058116\\
-0.745490981963928	1.33322321551226\\
-0.73886588296585	1.33867735470942\\
-0.729458917835671	1.34643508799382\\
-0.719301238459291	1.35470941883768\\
-0.713426853707415	1.35950466155993\\
-0.699485488778341	1.37074148296593\\
-0.697394789579158	1.3724307481579\\
-0.681362725450902	1.38525456128759\\
-0.679444486071188	1.38677354709419\\
-0.665330661322646	1.39798306621906\\
-0.659174997971831	1.40280561122244\\
-0.649298597194389	1.41056897016755\\
-0.638630392780365	1.4188376753507\\
-0.633266533066132	1.42301051043611\\
-0.617800762670221	1.43486973947896\\
-0.617234468937876	1.4353057453109\\
-0.601202404809619	1.4475487490333\\
-0.596748747598595	1.45090180360721\\
-0.585170340681363	1.45965892841028\\
-0.575402002446096	1.46693386773547\\
-0.569138276553106	1.47162187478752\\
-0.553739649023432	1.48296593186373\\
-0.55310621242485	1.48343506295179\\
-0.537074148296593	1.49520622302631\\
-0.531828963983769	1.49899799599198\\
-0.521042084168337	1.50684141353711\\
-0.509589305568402	1.51503006012024\\
-0.50501002004008	1.51832452342484\\
-0.488977955911824	1.52969261560049\\
-0.487025080597279	1.5310621242485\\
-0.472945891783567	1.54100190570865\\
-0.464161096508276	1.54709418837675\\
-0.456913827655311	1.55215580674429\\
-0.440917249974618	1.56312625250501\\
-0.440881763527054	1.56315077044347\\
-0.424849699398798	1.57413271058397\\
-0.41737359708186	1.57915831663327\\
-0.408817635270541	1.58495566667189\\
-0.393419074108392	1.59519038076152\\
-0.392785571142285	1.59561495069234\\
-0.376753507014028	1.60625536673355\\
-0.369125324313252	1.61122244488978\\
-0.360721442885771	1.61674306778591\\
-0.344689378757515	1.62706696249171\\
-0.344395257151599	1.62725450901804\\
-0.328657314629258	1.6373838586197\\
-0.31929204370735	1.64328657314629\\
-0.312625250501002	1.64752942021547\\
-0.296593186372745	1.65755210885658\\
-0.293729391380358	1.65931863727455\\
-0.280561122244489	1.6675247902261\\
-0.267723468721068	1.67535070140281\\
-0.264529058116232	1.67731872176217\\
-0.248496993987976	1.68706003540055\\
-0.24124710428395	1.69138276553106\\
-0.232464929859719	1.69667746869006\\
-0.216432865731463	1.70614621853222\\
-0.214256473023491	1.70741482965932\\
-0.200400801603207	1.71558544783079\\
-0.186748755004329	1.72344689378758\\
-0.18436873747495	1.72483389937621\\
-0.168336673346694	1.73405167264888\\
-0.158689220974881	1.73947895791583\\
-0.152304609218437	1.74311577228557\\
-0.13627254509018	1.75208373990267\\
-0.130033023613416	1.75551102204409\\
-0.120240480961924	1.76096023689598\\
-0.104208416833667	1.76968786917051\\
-0.100747303667207	1.77154308617235\\
-0.0881763527054109	1.77837292075237\\
-0.0721442885771544	1.78686893509702\\
-0.0707933414835995	1.7875751503006\\
-0.0561122244488979	1.79535809975437\\
-0.0401259908967309	1.80360721442886\\
-0.0400801603206413	1.80363120607876\\
-0.0240480961923848	1.81191872023512\\
-0.00869358889537183	1.81963927855711\\
-0.00801603206412826	1.8199850788953\\
0.00801603206412782	1.82805641414634\\
0.0235683261479037	1.83567134268537\\
0.0240480961923843	1.83590990103079\\
0.0400801603206409	1.84377150745912\\
0.0561122244488974	1.8514142056704\\
0.056727838712668	1.85170340681363\\
0.0721442885771539	1.85906302184108\\
0.0881763527054105	1.86650738489166\\
0.0908662865788111	1.86773547094188\\
0.104208416833667	1.87392866962192\\
0.120240480961924	1.88117944188881\\
0.126078357635204	1.88376753507014\\
0.13627254509018	1.88836484201108\\
0.152304609218437	1.89542616460125\\
0.162472699644747	1.8997995991984\\
0.168336673346693	1.90236659048268\\
0.18436873747495	1.90924200950305\\
0.20017481680906	1.91583166332665\\
0.200400801603206	1.91592760119292\\
0.216432865731463	1.92262007290555\\
0.232464929859719	1.9291235561351\\
0.239372439405237	1.93186372745491\\
0.248496993987976	1.93555205510367\\
0.264529058116232	1.94187826005622\\
0.280222369316579	1.94789579158317\\
0.280561122244489	1.94802821710527\\
0.296593186372745	1.95417855750995\\
0.312625250501002	1.96015144129628\\
0.323028016018513	1.96392785571142\\
0.328657314629258	1.96601263028277\\
0.344689378757515	1.97181279512412\\
0.360721442885771	1.97744261981634\\
0.368072784278691	1.97995991983968\\
0.376753507014028	1.98299442095986\\
0.392785571142285	1.98845312818685\\
0.408817635270541	1.99374754557004\\
0.415795209752653	1.99599198396794\\
0.424849699398798	1.99896720468222\\
0.440881763527054	2.00409056711776\\
0.456913827655311	2.00905466022014\\
0.466787771227061	2.01202404809619\\
0.472945891783567	2.01391711244512\\
0.488977955911824	2.01870865987881\\
0.50501002004008	2.02334495314417\\
0.521042084168337	2.02782922718771\\
0.521876581220279	2.02805611222445\\
0.537074148296593	2.03228328811775\\
0.55310621242485	2.03659172603068\\
0.569138276553106	2.04075075529471\\
0.582462999001817	2.04408817635271\\
0.585170340681363	2.04478236859634\\
0.601202404809619	2.04876026202671\\
0.617234468937876	2.0525903722525\\
0.633266533066132	2.05627493125352\\
0.649298597194389	2.05981598892093\\
0.650729265294393	2.06012024048096\\
0.665330661322646	2.06330217413261\\
0.681362725450902	2.06665162814523\\
0.697394789579159	2.06985740859541\\
0.713426853707415	2.07292099143853\\
0.729458917835672	2.07584367793316\\
0.731234316369284	2.07615230460922\\
0.745490981963928	2.07869431061906\\
0.761523046092185	2.08140964485727\\
0.777555110220441	2.08398206346184\\
0.793587174348698	2.08641228950813\\
0.809619238476954	2.08870087255636\\
0.825651302605211	2.09084818907255\\
0.836330035182677	2.09218436873747\\
0.841683366733467	2.09287218461091\\
0.857715430861724	2.09478622602467\\
0.87374749498998	2.09655458123605\\
0.889779559118236	2.09817701635387\\
0.905811623246493	2.09965311765325\\
0.921843687374749	2.10098229017225\\
0.937875751503006	2.102163755959\\
0.953907815631262	2.10319655196402\\
0.969939879759519	2.10407952757117\\
0.985971943887775	2.10481134175974\\
1.00200400801603	2.10539045988925\\
1.01803607214429	2.10581515009742\\
1.03406813627254	2.10608347930069\\
1.0501002004008	2.10619330878542\\
1.06613226452906	2.10614228937705\\
1.08216432865731	2.10592785617284\\
1.09819639278557	2.10554722282287\\
1.11422845691383	2.10499737534251\\
1.13026052104208	2.10427506543814\\
1.14629258517034	2.10337680332655\\
1.1623246492986	2.10229885002674\\
1.17835671342685	2.10103720910131\\
1.19438877755511	2.09958761782298\\
1.21042084168337	2.09794553773978\\
1.22645290581162	2.09610614461074\\
1.24248496993988	2.09406431768172\\
1.25590346580642	2.09218436873747\\
1.25851703406814	2.09182206257695\\
1.27454909819639	2.08940764771662\\
1.29058116232465	2.08677673471063\\
1.30661322645291	2.08392286405226\\
1.32264529058116	2.08083921494485\\
1.33867735470942	2.07751858965312\\
1.34486757761647	2.07615230460922\\
1.35470941883768	2.07399357000261\\
1.37074148296593	2.0702441537894\\
1.38677354709419	2.06623659422859\\
1.40280561122244	2.06196197232951\\
1.40934737375819	2.06012024048096\\
1.4188376753507	2.0574565725202\\
1.43486973947896	2.05269891199702\\
1.45090180360721	2.04764705295235\\
1.46160706456837	2.04408817635271\\
1.46693386773547	2.04231815372394\\
1.48296593186373	2.03673135760558\\
1.49899799599198	2.03081783808899\\
1.5061447449282	2.02805611222445\\
1.51503006012024	2.02461511587724\\
1.5310621242485	2.01810170530396\\
1.54525566659975	2.01202404809619\\
1.54709418837675	2.01123348827771\\
1.56312625250501	2.00406935430044\\
1.57915831663327	1.99651012678033\\
1.58021923467439	1.99599198396794\\
1.59519038076152	1.98862963410871\\
1.61122244488978	1.9803269409903\\
1.61190801891683	1.97995991983968\\
1.62725450901804	1.97167005797801\\
1.64090782627769	1.96392785571142\\
1.64328657314629	1.96256481200851\\
1.65931863727455	1.95305382050197\\
1.66765816091769	1.94789579158317\\
1.67535070140281	1.94307765066655\\
1.69138276553106	1.93261614787968\\
1.69250092360229	1.93186372745491\\
1.70741482965932	1.92167867067516\\
1.71564742351291	1.91583166332665\\
1.72344689378758	1.91020121995158\\
1.73732988436364	1.8997995991984\\
1.73947895791583	1.89816042213038\\
1.75551102204409	1.88553794022295\\
1.75769148765487	1.88376753507014\\
1.77154308617235	1.87229200108879\\
1.77686884116719	1.86773547094188\\
1.7875751503006	1.85837397478382\\
1.79497588917214	1.85170340681363\\
1.80360721442886	1.84373936746472\\
1.81210503783506	1.83567134268537\\
1.81963927855711	1.82833639725782\\
1.82833639725782	1.81963927855711\\
1.83567134268537	1.81210503783506\\
1.84373936746472	1.80360721442886\\
1.85170340681363	1.79497588917214\\
1.85837397478382	1.7875751503006\\
1.86773547094188	1.77686884116719\\
1.87229200108879	1.77154308617235\\
1.88376753507014	1.75769148765487\\
1.88553794022295	1.75551102204409\\
1.89816042213038	1.73947895791583\\
1.8997995991984	1.73732988436364\\
1.91020121995158	1.72344689378758\\
1.91583166332665	1.71564742351291\\
1.92167867067516	1.70741482965932\\
1.93186372745491	1.69250092360229\\
1.93261614787968	1.69138276553106\\
1.94307765066655	1.67535070140281\\
1.94789579158317	1.66765816091769\\
1.95305382050197	1.65931863727455\\
1.96256481200851	1.64328657314629\\
1.96392785571142	1.64090782627769\\
1.97167005797801	1.62725450901804\\
1.97995991983968	1.61190801891683\\
1.9803269409903	1.61122244488978\\
1.98862963410871	1.59519038076152\\
1.99599198396794	1.58021923467439\\
1.99651012678033	1.57915831663327\\
2.00406935430044	1.56312625250501\\
2.01123348827771	1.54709418837675\\
2.01202404809619	1.54525566659975\\
2.01810170530396	1.5310621242485\\
2.02461511587724	1.51503006012024\\
2.02805611222445	1.5061447449282\\
2.03081783808899	1.49899799599198\\
2.03673135760558	1.48296593186373\\
2.04231815372394	1.46693386773547\\
2.04408817635271	1.46160706456837\\
2.04764705295235	1.45090180360721\\
2.05269891199702	1.43486973947896\\
2.0574565725202	1.4188376753507\\
2.06012024048096	1.40934737375819\\
2.06196197232951	1.40280561122244\\
2.06623659422859	1.38677354709419\\
2.0702441537894	1.37074148296593\\
2.07399357000261	1.35470941883768\\
2.07615230460922	1.34486757761647\\
2.07751858965312	1.33867735470942\\
2.08083921494485	1.32264529058116\\
2.08392286405226	1.30661322645291\\
2.08677673471063	1.29058116232465\\
2.08940764771662	1.27454909819639\\
2.09182206257695	1.25851703406814\\
2.09218436873747	1.25590346580642\\
2.09406431768172	1.24248496993988\\
2.09610614461074	1.22645290581162\\
2.09794553773978	1.21042084168337\\
2.09958761782298	1.19438877755511\\
2.10103720910131	1.17835671342685\\
2.10229885002674	1.1623246492986\\
2.10337680332655	1.14629258517034\\
2.10427506543814	1.13026052104208\\
2.10499737534251	1.11422845691383\\
2.10554722282287	1.09819639278557\\
2.10592785617284	1.08216432865731\\
2.10614228937705	1.06613226452906\\
2.10619330878542	1.0501002004008\\
2.10608347930069	1.03406813627254\\
2.10581515009742	1.01803607214429\\
2.10539045988925	1.00200400801603\\
2.10481134175974	0.985971943887775\\
2.10407952757117	0.969939879759519\\
2.10319655196402	0.953907815631262\\
2.102163755959	0.937875751503006\\
2.10098229017225	0.921843687374749\\
2.09965311765325	0.905811623246493\\
2.09817701635387	0.889779559118236\\
2.09655458123605	0.87374749498998\\
2.09478622602467	0.857715430861724\\
2.09287218461091	0.841683366733467\\
2.09218436873747	0.836330035182676\\
2.09084818907255	0.825651302605211\\
2.08870087255636	0.809619238476954\\
2.08641228950813	0.793587174348698\\
2.08398206346184	0.777555110220441\\
2.08140964485727	0.761523046092185\\
2.07869431061906	0.745490981963928\\
2.07615230460922	0.731234316369284\\
2.07584367793316	0.729458917835672\\
2.07292099143853	0.713426853707415\\
2.06985740859541	0.697394789579159\\
2.06665162814523	0.681362725450902\\
2.06330217413261	0.665330661322646\\
2.06012024048096	0.650729265294393\\
2.05981598892093	0.649298597194389\\
2.05627493125352	0.633266533066132\\
2.0525903722525	0.617234468937876\\
2.04876026202671	0.601202404809619\\
2.04478236859634	0.585170340681363\\
2.04408817635271	0.582462999001817\\
2.04075075529471	0.569138276553106\\
2.03659172603068	0.55310621242485\\
2.03228328811776	0.537074148296593\\
2.02805611222445	0.521876581220279\\
2.02782922718771	0.521042084168337\\
2.02334495314417	0.50501002004008\\
2.01870865987881	0.488977955911824\\
2.01391711244512	0.472945891783567\\
2.01202404809619	0.46678777122706\\
2.00905466022014	0.456913827655311\\
2.00409056711776	0.440881763527054\\
1.99896720468222	0.424849699398798\\
1.99599198396794	0.415795209752652\\
1.99374754557004	0.408817635270541\\
1.98845312818685	0.392785571142285\\
1.98299442095986	0.376753507014028\\
1.97995991983968	0.368072784278691\\
1.97744261981634	0.360721442885771\\
1.97181279512412	0.344689378757515\\
1.96601263028277	0.328657314629258\\
1.96392785571142	0.323028016018513\\
1.96015144129628	0.312625250501002\\
1.95417855750996	0.296593186372745\\
1.94802821710527	0.280561122244489\\
1.94789579158317	0.280222369316578\\
1.94187826005622	0.264529058116232\\
1.93555205510367	0.248496993987976\\
1.93186372745491	0.239372439405237\\
1.9291235561351	0.232464929859719\\
1.92262007290555	0.216432865731463\\
1.91592760119292	0.200400801603206\\
1.91583166332665	0.20017481680906\\
1.90924200950305	0.18436873747495\\
1.90236659048268	0.168336673346693\\
1.8997995991984	0.162472699644747\\
1.89542616460125	0.152304609218437\\
1.88836484201108	0.13627254509018\\
1.88376753507014	0.126078357635204\\
1.88117944188881	0.120240480961924\\
1.87392866962192	0.104208416833667\\
1.86773547094188	0.0908662865788111\\
1.86650738489166	0.0881763527054105\\
1.85906302184108	0.0721442885771539\\
1.85170340681363	0.0567278387126675\\
1.8514142056704	0.0561122244488974\\
1.84377150745912	0.0400801603206409\\
1.83590990103079	0.0240480961923843\\
1.83567134268537	0.0235683261479041\\
1.82805641414634	0.00801603206412782\\
1.8199850788953	-0.00801603206412826\\
1.81963927855711	-0.00869358889537183\\
1.81191872023512	-0.0240480961923848\\
1.80363120607876	-0.0400801603206413\\
1.80360721442886	-0.0401259908967309\\
1.79535809975437	-0.0561122244488979\\
1.7875751503006	-0.0707933414835995\\
1.78686893509702	-0.0721442885771544\\
1.77837292075237	-0.0881763527054109\\
1.77154308617235	-0.100747303667208\\
1.76968786917051	-0.104208416833667\\
1.76096023689598	-0.120240480961924\\
1.75551102204409	-0.130033023613416\\
1.75208373990267	-0.13627254509018\\
1.74311577228557	-0.152304609218437\\
1.73947895791583	-0.158689220974882\\
1.73405167264888	-0.168336673346694\\
1.72483389937621	-0.18436873747495\\
1.72344689378758	-0.186748755004329\\
1.71558544783079	-0.200400801603207\\
1.70741482965932	-0.214256473023491\\
1.70614621853222	-0.216432865731463\\
1.69667746869006	-0.232464929859719\\
1.69138276553106	-0.24124710428395\\
1.68706003540055	-0.248496993987976\\
1.67731872176217	-0.264529058116232\\
1.67535070140281	-0.267723468721068\\
1.6675247902261	-0.280561122244489\\
1.65931863727455	-0.293729391380358\\
1.65755210885658	-0.296593186372745\\
1.64752942021547	-0.312625250501002\\
1.64328657314629	-0.31929204370735\\
1.6373838586197	-0.328657314629258\\
1.62725450901804	-0.344395257151599\\
1.62706696249171	-0.344689378757515\\
1.61674306778591	-0.360721442885771\\
1.61122244488978	-0.369125324313252\\
1.60625536673355	-0.376753507014028\\
1.59561495069234	-0.392785571142285\\
1.59519038076152	-0.393419074108392\\
1.58495566667189	-0.408817635270541\\
1.57915831663327	-0.41737359708186\\
1.57413271058397	-0.424849699398798\\
1.56315077044347	-0.440881763527054\\
1.56312625250501	-0.440917249974618\\
1.55215580674429	-0.456913827655311\\
1.54709418837675	-0.464161096508276\\
1.54100190570865	-0.472945891783567\\
1.5310621242485	-0.487025080597279\\
1.52969261560049	-0.488977955911824\\
1.51832452342484	-0.50501002004008\\
1.51503006012024	-0.509589305568402\\
1.50684141353711	-0.521042084168337\\
1.49899799599198	-0.531828963983769\\
1.49520622302631	-0.537074148296593\\
1.48343506295179	-0.55310621242485\\
1.48296593186373	-0.553739649023432\\
1.47162187478752	-0.569138276553106\\
1.46693386773547	-0.575402002446096\\
1.45965892841028	-0.585170340681363\\
1.45090180360721	-0.596748747598595\\
1.4475487490333	-0.601202404809619\\
1.4353057453109	-0.617234468937876\\
1.43486973947896	-0.617800762670221\\
1.42301051043611	-0.633266533066132\\
1.4188376753507	-0.638630392780365\\
1.41056897016755	-0.649298597194389\\
1.40280561122244	-0.659174997971831\\
1.39798306621906	-0.665330661322646\\
1.38677354709419	-0.679444486071188\\
1.38525456128759	-0.681362725450902\\
1.3724307481579	-0.697394789579158\\
1.37074148296593	-0.699485488778341\\
1.35950466155993	-0.713426853707415\\
1.35470941883768	-0.719301238459291\\
1.34643508799382	-0.729458917835671\\
1.33867735470942	-0.73886588296585\\
1.33322321551226	-0.745490981963928\\
1.32264529058116	-0.758187441716244\\
1.31987005815617	-0.761523046092184\\
1.30661322645291	-0.777273522548639\\
1.30637645725026	-0.777555110220441\\
1.29280128824555	-0.793587174348697\\
1.29058116232465	-0.796181734635507\\
1.27908848480037	-0.809619238476954\\
1.27454909819639	-0.814870935176106\\
1.26523204637232	-0.82565130260521\\
1.25851703406814	-0.833342057411939\\
1.25123222431763	-0.841683366733467\\
1.24248496993988	-0.851601198605366\\
1.23708909456899	-0.857715430861723\\
1.22645290581162	-0.869654123233241\\
1.22280255716274	-0.87374749498998\\
1.21042084168337	-0.887506275980729\\
1.20837233543093	-0.889779559118236\\
1.19438877755511	-0.905162793967734\\
1.19379797485529	-0.905811623246493\\
1.17909731511516	-0.92184368737475\\
1.17835671342685	-0.92264572608842\\
1.16426203204616	-0.937875751503006\\
1.1623246492986	-0.939953035042293\\
1.14927595227682	-0.953907815631263\\
1.14629258517034	-0.957074890637135\\
1.13413795978371	-0.969939879759519\\
1.13026052104208	-0.974015270485194\\
1.11884674649862	-0.985971943887776\\
1.11422845691383	-0.990777893947399\\
1.10340080916829	-1.00200400801603\\
1.09819639278557	-1.00736623032159\\
1.08779844582577	-1.01803607214429\\
1.08216432865731	-1.02378350648121\\
1.07203775186289	-1.03406813627255\\
1.06613226452906	-1.04003271398696\\
1.0561166156924	-1.0501002004008\\
1.0501002004008	-1.0561166156924\\
1.04003271398696	-1.06613226452906\\
1.03406813627254	-1.0720377518629\\
1.02378350648121	-1.08216432865731\\
1.01803607214429	-1.08779844582577\\
1.00736623032159	-1.09819639278557\\
1.00200400801603	-1.10340080916829\\
0.990777893947399	-1.11422845691383\\
0.985971943887775	-1.11884674649862\\
0.974015270485195	-1.13026052104208\\
0.969939879759519	-1.13413795978372\\
0.957074890637135	-1.14629258517034\\
0.953907815631262	-1.14927595227682\\
0.939953035042293	-1.1623246492986\\
0.937875751503006	-1.16426203204616\\
0.92264572608842	-1.17835671342685\\
0.921843687374749	-1.17909731511517\\
0.905811623246493	-1.19379797485529\\
0.905162793967733	-1.19438877755511\\
0.889779559118236	-1.20837233543093\\
0.887506275980729	-1.21042084168337\\
0.87374749498998	-1.22280255716274\\
0.869654123233241	-1.22645290581162\\
0.857715430861724	-1.23708909456899\\
0.851601198605366	-1.24248496993988\\
0.841683366733467	-1.25123222431763\\
0.833342057411939	-1.25851703406814\\
0.825651302605211	-1.26523204637232\\
0.814870935176105	-1.27454909819639\\
0.809619238476954	-1.27908848480037\\
0.796181734635507	-1.29058116232465\\
0.793587174348698	-1.29280128824555\\
0.777555110220441	-1.30637645725026\\
0.777273522548639	-1.30661322645291\\
0.761523046092185	-1.31987005815617\\
0.758187441716243	-1.32264529058116\\
0.745490981963928	-1.33322321551226\\
0.738865882965851	-1.33867735470942\\
0.729458917835672	-1.34643508799382\\
0.719301238459291	-1.35470941883768\\
0.713426853707415	-1.35950466155993\\
0.699485488778341	-1.37074148296593\\
0.697394789579159	-1.3724307481579\\
0.681362725450902	-1.38525456128759\\
0.679444486071188	-1.38677354709419\\
0.665330661322646	-1.39798306621906\\
0.659174997971831	-1.40280561122244\\
0.649298597194389	-1.41056897016755\\
0.638630392780364	-1.4188376753507\\
0.633266533066132	-1.42301051043611\\
0.617800762670221	-1.43486973947896\\
0.617234468937876	-1.4353057453109\\
0.601202404809619	-1.4475487490333\\
0.596748747598595	-1.45090180360721\\
0.585170340681363	-1.45965892841028\\
0.575402002446096	-1.46693386773547\\
0.569138276553106	-1.47162187478752\\
0.553739649023432	-1.48296593186373\\
0.55310621242485	-1.48343506295179\\
0.537074148296593	-1.49520622302631\\
0.531828963983769	-1.49899799599198\\
0.521042084168337	-1.50684141353711\\
0.509589305568402	-1.51503006012024\\
0.50501002004008	-1.51832452342484\\
0.488977955911824	-1.52969261560049\\
0.487025080597279	-1.5310621242485\\
0.472945891783567	-1.54100190570865\\
0.464161096508276	-1.54709418837675\\
0.456913827655311	-1.55215580674429\\
0.440917249974618	-1.56312625250501\\
0.440881763527054	-1.56315077044347\\
0.424849699398798	-1.57413271058397\\
0.41737359708186	-1.57915831663327\\
0.408817635270541	-1.58495566667189\\
0.393419074108392	-1.59519038076152\\
0.392785571142285	-1.59561495069234\\
0.376753507014028	-1.60625536673355\\
0.369125324313253	-1.61122244488978\\
0.360721442885771	-1.61674306778591\\
0.344689378757515	-1.62706696249171\\
0.344395257151599	-1.62725450901804\\
0.328657314629258	-1.6373838586197\\
0.319292043707351	-1.64328657314629\\
0.312625250501002	-1.64752942021547\\
0.296593186372745	-1.65755210885658\\
0.293729391380358	-1.65931863727455\\
0.280561122244489	-1.6675247902261\\
0.267723468721069	-1.67535070140281\\
0.264529058116232	-1.67731872176217\\
0.248496993987976	-1.68706003540055\\
0.24124710428395	-1.69138276553106\\
0.232464929859719	-1.69667746869006\\
0.216432865731463	-1.70614621853222\\
0.214256473023491	-1.70741482965932\\
0.200400801603206	-1.71558544783079\\
0.186748755004329	-1.72344689378758\\
0.18436873747495	-1.72483389937621\\
0.168336673346693	-1.73405167264888\\
0.158689220974882	-1.73947895791583\\
0.152304609218437	-1.74311577228557\\
0.13627254509018	-1.75208373990267\\
0.130033023613416	-1.75551102204409\\
0.120240480961924	-1.76096023689598\\
0.104208416833667	-1.76968786917051\\
0.100747303667208	-1.77154308617234\\
0.0881763527054105	-1.77837292075237\\
0.0721442885771539	-1.78686893509702\\
0.0707933414835986	-1.7875751503006\\
0.0561122244488974	-1.79535809975437\\
0.04012599089673	-1.80360721442886\\
0.0400801603206409	-1.80363120607876\\
0.0240480961923843	-1.81191872023512\\
0.00869358889537093	-1.81963927855711\\
0.00801603206412782	-1.8199850788953\\
-0.00801603206412826	-1.82805641414634\\
-0.023568326147905	-1.83567134268537\\
-0.0240480961923848	-1.83590990103079\\
-0.0400801603206413	-1.84377150745912\\
-0.0561122244488979	-1.8514142056704\\
-0.0567278387126693	-1.85170340681363\\
-0.0721442885771544	-1.85906302184108\\
-0.0881763527054109	-1.86650738489166\\
-0.0908662865788124	-1.86773547094188\\
-0.104208416833667	-1.87392866962192\\
-0.120240480961924	-1.88117944188881\\
-0.126078357635206	-1.88376753507014\\
-0.13627254509018	-1.88836484201108\\
-0.152304609218437	-1.89542616460125\\
-0.162472699644748	-1.8997995991984\\
-0.168336673346694	-1.90236659048268\\
-0.18436873747495	-1.90924200950305\\
-0.20017481680906	-1.91583166332665\\
-0.200400801603207	-1.91592760119292\\
-0.216432865731463	-1.92262007290555\\
-0.232464929859719	-1.9291235561351\\
-0.239372439405238	-1.93186372745491\\
-0.248496993987976	-1.93555205510367\\
-0.264529058116232	-1.94187826005622\\
-0.280222369316579	-1.94789579158317\\
-0.280561122244489	-1.94802821710527\\
-0.296593186372745	-1.95417855750996\\
-0.312625250501002	-1.96015144129628\\
-0.323028016018515	-1.96392785571142\\
-0.328657314629258	-1.96601263028277\\
-0.344689378757515	-1.97181279512412\\
-0.360721442885771	-1.97744261981634\\
-0.368072784278691	-1.97995991983968\\
-0.376753507014028	-1.98299442095986\\
-0.392785571142285	-1.98845312818685\\
-0.408817635270541	-1.99374754557004\\
-0.415795209752654	-1.99599198396794\\
-0.424849699398798	-1.99896720468222\\
-0.440881763527054	-2.00409056711776\\
-0.456913827655311	-2.00905466022014\\
-0.466787771227061	-2.01202404809619\\
-0.472945891783567	-2.01391711244512\\
-0.488977955911824	-2.01870865987881\\
-0.50501002004008	-2.02334495314417\\
-0.521042084168337	-2.02782922718771\\
-0.521876581220279	-2.02805611222445\\
-0.537074148296593	-2.03228328811775\\
-0.55310621242485	-2.03659172603068\\
-0.569138276553106	-2.04075075529471\\
-0.582462999001817	-2.04408817635271\\
-0.585170340681363	-2.04478236859634\\
-0.601202404809619	-2.04876026202671\\
-0.617234468937876	-2.0525903722525\\
-0.633266533066132	-2.05627493125352\\
-0.649298597194389	-2.05981598892093\\
-0.650729265294393	-2.06012024048096\\
-0.665330661322646	-2.06330217413261\\
-0.681362725450902	-2.06665162814523\\
-0.697394789579158	-2.06985740859541\\
-0.713426853707415	-2.07292099143853\\
-0.729458917835671	-2.07584367793316\\
-0.731234316369284	-2.07615230460922\\
-0.745490981963928	-2.07869431061906\\
-0.761523046092184	-2.08140964485727\\
-0.777555110220441	-2.08398206346184\\
-0.793587174348697	-2.08641228950813\\
-0.809619238476954	-2.08870087255636\\
-0.82565130260521	-2.09084818907255\\
-0.836330035182677	-2.09218436873747\\
-0.841683366733467	-2.09287218461091\\
-0.857715430861723	-2.09478622602467\\
-0.87374749498998	-2.09655458123605\\
-0.889779559118236	-2.09817701635387\\
-0.905811623246493	-2.09965311765325\\
-0.92184368737475	-2.10098229017225\\
-0.937875751503006	-2.102163755959\\
-0.953907815631263	-2.10319655196402\\
-0.969939879759519	-2.10407952757117\\
-0.985971943887776	-2.10481134175974\\
-1.00200400801603	-2.10539045988925\\
-1.01803607214429	-2.10581515009742\\
-1.03406813627255	-2.10608347930069\\
-1.0501002004008	-2.10619330878542\\
-1.06613226452906	-2.10614228937705\\
-1.08216432865731	-2.10592785617284\\
-1.09819639278557	-2.10554722282287\\
-1.11422845691383	-2.10499737534251\\
-1.13026052104208	-2.10427506543814\\
-1.14629258517034	-2.10337680332655\\
-1.1623246492986	-2.10229885002674\\
-1.17835671342685	-2.10103720910131\\
-1.19438877755511	-2.09958761782298\\
-1.21042084168337	-2.09794553773978\\
-1.22645290581162	-2.09610614461074\\
-1.24248496993988	-2.09406431768172\\
-1.25590346580642	-2.09218436873747\\
-1.25851703406814	-2.09182206257695\\
-1.27454909819639	-2.08940764771662\\
-1.29058116232465	-2.08677673471063\\
-1.30661322645291	-2.08392286405226\\
-1.32264529058116	-2.08083921494485\\
-1.33867735470942	-2.07751858965312\\
-1.34486757761647	-2.07615230460922\\
-1.35470941883768	-2.07399357000261\\
-1.37074148296593	-2.0702441537894\\
-1.38677354709419	-2.06623659422859\\
-1.40280561122244	-2.06196197232951\\
-1.40934737375819	-2.06012024048096\\
-1.4188376753507	-2.0574565725202\\
-1.43486973947896	-2.05269891199702\\
-1.45090180360721	-2.04764705295235\\
-1.46160706456837	-2.04408817635271\\
-1.46693386773547	-2.04231815372394\\
-1.48296593186373	-2.03673135760558\\
-1.49899799599198	-2.03081783808899\\
-1.5061447449282	-2.02805611222445\\
-1.51503006012024	-2.02461511587724\\
-1.5310621242485	-2.01810170530396\\
-1.54525566659975	-2.01202404809619\\
-1.54709418837675	-2.01123348827771\\
-1.56312625250501	-2.00406935430044\\
-1.57915831663327	-1.99651012678033\\
-1.58021923467439	-1.99599198396794\\
-1.59519038076152	-1.98862963410871\\
-1.61122244488978	-1.9803269409903\\
-1.61190801891683	-1.97995991983968\\
-1.62725450901804	-1.97167005797801\\
-1.64090782627769	-1.96392785571142\\
-1.64328657314629	-1.96256481200851\\
-1.65931863727455	-1.95305382050197\\
-1.66765816091769	-1.94789579158317\\
-1.67535070140281	-1.94307765066655\\
-1.69138276553106	-1.93261614787968\\
-1.69250092360229	-1.93186372745491\\
-1.70741482965932	-1.92167867067516\\
-1.71564742351291	-1.91583166332665\\
-1.72344689378758	-1.91020121995158\\
-1.73732988436364	-1.8997995991984\\
-1.73947895791583	-1.89816042213038\\
-1.75551102204409	-1.88553794022294\\
-1.75769148765487	-1.88376753507014\\
-1.77154308617234	-1.87229200108879\\
-1.77686884116719	-1.86773547094188\\
-1.7875751503006	-1.85837397478382\\
-1.79497588917214	-1.85170340681363\\
-1.80360721442886	-1.84373936746472\\
-1.81210503783506	-1.83567134268537\\
-1.81963927855711	-1.82833639725782\\
-1.82833639725782	-1.81963927855711\\
-1.83567134268537	-1.81210503783506\\
-1.84373936746472	-1.80360721442886\\
-1.85170340681363	-1.79497588917214\\
-1.85837397478382	-1.7875751503006\\
-1.86773547094188	-1.77686884116719\\
-1.87229200108879	-1.77154308617234\\
-1.88376753507014	-1.75769148765487\\
-1.88553794022294	-1.75551102204409\\
-1.89816042213038	-1.73947895791583\\
-1.8997995991984	-1.73732988436364\\
-1.91020121995158	-1.72344689378758\\
-1.91583166332665	-1.71564742351291\\
-1.92167867067516	-1.70741482965932\\
-1.93186372745491	-1.69250092360229\\
-1.93261614787968	-1.69138276553106\\
-1.94307765066655	-1.67535070140281\\
-1.94789579158317	-1.66765816091769\\
-1.95305382050197	-1.65931863727455\\
-1.96256481200851	-1.64328657314629\\
-1.96392785571142	-1.64090782627769\\
-1.97167005797801	-1.62725450901804\\
-1.97995991983968	-1.61190801891683\\
-1.9803269409903	-1.61122244488978\\
-1.98862963410871	-1.59519038076152\\
-1.99599198396794	-1.58021923467439\\
-1.99651012678033	-1.57915831663327\\
-2.00406935430044	-1.56312625250501\\
-2.01123348827771	-1.54709418837675\\
-2.01202404809619	-1.54525566659975\\
-2.01810170530396	-1.5310621242485\\
-2.02461511587724	-1.51503006012024\\
-2.02805611222445	-1.5061447449282\\
-2.03081783808899	-1.49899799599198\\
-2.03673135760558	-1.48296593186373\\
-2.04231815372394	-1.46693386773547\\
-2.04408817635271	-1.46160706456837\\
-2.04764705295235	-1.45090180360721\\
-2.05269891199702	-1.43486973947896\\
-2.0574565725202	-1.4188376753507\\
-2.06012024048096	-1.40934737375819\\
-2.06196197232951	-1.40280561122244\\
-2.06623659422859	-1.38677354709419\\
-2.0702441537894	-1.37074148296593\\
-2.07399357000261	-1.35470941883768\\
-2.07615230460922	-1.34486757761647\\
-2.07751858965312	-1.33867735470942\\
-2.08083921494485	-1.32264529058116\\
-2.08392286405226	-1.30661322645291\\
-2.08677673471063	-1.29058116232465\\
-2.08940764771662	-1.27454909819639\\
-2.09182206257695	-1.25851703406814\\
-2.09218436873747	-1.25590346580642\\
-2.09406431768172	-1.24248496993988\\
-2.09610614461074	-1.22645290581162\\
-2.09794553773978	-1.21042084168337\\
-2.09958761782298	-1.19438877755511\\
-2.10103720910131	-1.17835671342685\\
-2.10229885002674	-1.1623246492986\\
-2.10337680332655	-1.14629258517034\\
-2.10427506543814	-1.13026052104208\\
-2.10499737534251	-1.11422845691383\\
-2.10554722282287	-1.09819639278557\\
-2.10592785617284	-1.08216432865731\\
-2.10614228937705	-1.06613226452906\\
-2.10619330878542	-1.0501002004008\\
-2.10608347930069	-1.03406813627255\\
-2.10581515009742	-1.01803607214429\\
-2.10539045988925	-1.00200400801603\\
-2.10481134175974	-0.985971943887776\\
-2.10407952757117	-0.969939879759519\\
-2.10319655196402	-0.953907815631263\\
-2.102163755959	-0.937875751503006\\
-2.10098229017225	-0.92184368737475\\
-2.09965311765325	-0.905811623246493\\
-2.09817701635387	-0.889779559118236\\
-2.09655458123605	-0.87374749498998\\
-2.09478622602467	-0.857715430861723\\
-2.09287218461091	-0.841683366733467\\
-2.09218436873747	-0.836330035182676\\
}--cycle;


\addplot[area legend,solid,fill=mycolor3,draw=black,forget plot]
table[row sep=crcr] {%
x	y\\
-1.73947895791583	-0.736103830127027\\
-1.73884435303621	-0.729458917835671\\
-1.73713928085242	-0.713426853707415\\
-1.73526431832439	-0.697394789579158\\
-1.73322049782681	-0.681362725450902\\
-1.73100862614715	-0.665330661322646\\
-1.72862928729111	-0.649298597194389\\
-1.72608284482946	-0.633266533066132\\
-1.72344689378758	-0.617692306947827\\
-1.72337089469617	-0.617234468937876\\
-1.72054510585176	-0.601202404809619\\
-1.71755653473623	-0.585170340681363\\
-1.71440479617826	-0.569138276553106\\
-1.71108929377282	-0.55310621242485\\
-1.70760921987228	-0.537074148296593\\
-1.70741482965932	-0.53621808903568\\
-1.70403137936056	-0.521042084168337\\
-1.70029546943752	-0.50501002004008\\
-1.69639640372907	-0.488977955911824\\
-1.69233266126292	-0.472945891783567\\
-1.69138276553106	-0.469338124237107\\
-1.68816846448149	-0.456913827655311\\
-1.68386013331379	-0.440881763527054\\
-1.67938676932915	-0.424849699398798\\
-1.67535070140281	-0.410899598565508\\
-1.67475843626556	-0.408817635270541\\
-1.67004713451038	-0.392785571142285\\
-1.66516920494166	-0.376753507014028\\
-1.66012192256018	-0.360721442885771\\
-1.65931863727455	-0.358243878823841\\
-1.65499412798524	-0.344689378757515\\
-1.64971340211983	-0.328657314629258\\
-1.64426000301504	-0.312625250501002\\
-1.64328657314629	-0.309839766781271\\
-1.63872829016932	-0.296593186372745\\
-1.63304332259554	-0.280561122244489\\
-1.62725450901804	-0.264728499008128\\
-1.62718266880621	-0.264529058116232\\
-1.62126759432409	-0.248496993987976\\
-1.61517375675617	-0.232464929859719\\
-1.61122244488978	-0.222345839245399\\
-1.60894625940275	-0.216432865731463\\
-1.60262103759575	-0.200400801603207\\
-1.5961105114565	-0.18436873747495\\
-1.59519038076152	-0.18215184455341\\
-1.58953387438088	-0.168336673346694\\
-1.58278875798391	-0.152304609218437\\
-1.57915831663327	-0.143881511891677\\
-1.57592145783621	-0.13627254509018\\
-1.56894006534054	-0.120240480961924\\
-1.56312625250501	-0.107232969746725\\
-1.56179131254381	-0.104208416833667\\
-1.55457121294478	-0.0881763527054109\\
-1.54715054509611	-0.0721442885771544\\
-1.54709418837675	-0.0720246507521712\\
-1.53968735190831	-0.0561122244488979\\
-1.53202038856421	-0.0400801603206413\\
-1.5310621242485	-0.0381120853590133\\
-1.52429203163416	-0.0240480961923848\\
-1.51637435979378	-0.00801603206412826\\
-1.51503006012024	-0.00534208362977375\\
-1.50838721810992	0.00801603206412782\\
-1.50021365028534	0.0240480961923843\\
-1.49899799599198	0.0263923548855087\\
-1.49197330402804	0.0400801603206409\\
-1.4835378630422	0.0561122244488974\\
-1.48296593186373	0.0571824839285042\\
-1.47504911080628	0.0721442885771539\\
-1.46693386773547	0.0871039964118488\\
-1.46635764453439	0.0881763527054105\\
-1.45761188252278	0.104208416833667\\
-1.45090180360721	0.116225809613082\\
-1.44868021399324	0.120240480961924\\
-1.43965727172208	0.13627254509018\\
-1.43486973947896	0.144609615201841\\
-1.43048820471916	0.152304609218437\\
-1.42117931698976	0.168336673346693\\
-1.4188376753507	0.172304746209675\\
-1.41177491003306	0.18436873747495\\
-1.40280561122244	0.199354164605838\\
-1.40218389871713	0.200400801603206\\
-1.3925319777012	0.216432865731463\\
-1.38677354709419	0.225804070271314\\
-1.38270942820289	0.232464929859719\\
-1.37274936655861	0.248496993987976\\
-1.37074148296593	0.251681988483172\\
-1.36269539866052	0.264529058116232\\
-1.35470941883768	0.277025015543469\\
-1.35246348991562	0.280561122244489\\
-1.34212926903676	0.296593186372745\\
-1.33867735470942	0.301862507325588\\
-1.33166661958351	0.312625250501002\\
-1.32264529058116	0.32621290583393\\
-1.32103094497034	0.328657314629258\\
-1.31030143811245	0.344689378757515\\
-1.30661322645291	0.350113050912796\\
-1.29943376454435	0.360721442885771\\
-1.29058116232465	0.373569228672947\\
-1.28839674703226	0.376753507014028\\
-1.2772481293273	0.392785571142285\\
-1.27454909819639	0.396612912558848\\
-1.26597588984069	0.408817635270541\\
-1.25851703406814	0.419258454076814\\
-1.25453676251989	0.424849699398798\\
-1.24294231115383	0.440881763527054\\
-1.24248496993988	0.441507718566507\\
-1.23126272035249	0.456913827655311\\
-1.22645290581162	0.463413492946463\\
-1.21941746732097	0.472945891783567\\
-1.21042084168337	0.484949536722487\\
-1.20740842663627	0.488977955911824\\
-1.19525426983635	0.50501002004008\\
-1.19438877755511	0.506140069367307\\
-1.18299552609021	0.521042084168337\\
-1.17835671342685	0.527023324535393\\
-1.17057247205745	0.537074148296593\\
-1.1623246492986	0.547576412709568\\
-1.15798627375855	0.55310621242485\\
-1.14629258517034	0.567811139716855\\
-1.14523788703217	0.569138276553106\\
-1.13236802527859	0.585170340681363\\
-1.13026052104208	0.587766190578265\\
-1.11935333090457	0.601202404809619\\
-1.11422845691383	0.607437360561867\\
-1.10617360742911	0.617234468937876\\
-1.09819639278557	0.626820685959868\\
-1.09282908504719	0.633266533066132\\
-1.08216432865731	0.645925521633466\\
-1.07931977971233	0.649298597194389\\
-1.06613226452906	0.664760715493397\\
-1.06564549199492	0.665330661322646\\
-1.0518371336525	0.681362725450902\\
-1.0501002004008	0.683359431420935\\
-1.03786772734357	0.697394789579159\\
-1.03406813627255	0.701709868598308\\
-1.0237274586441	0.713426853707415\\
-1.01803607214429	0.719812356934714\\
-1.00941534488687	0.729458917835672\\
-1.00200400801603	0.737673823769273\\
-0.994930173386556	0.745490981963928\\
-0.985971943887776	0.755300793077052\\
-0.980270497807276	0.761523046092185\\
-0.969939879759519	0.772699401575495\\
-0.965434634038255	0.777555110220441\\
-0.953907815631263	0.789875413833204\\
-0.950420655563976	0.793587174348698\\
-0.937875751503006	0.806834236431902\\
-0.935226388313504	0.809619238476954\\
-0.92184368737475	0.823580931228913\\
-0.919849404972134	0.825651302605211\\
-0.905811623246493	0.840120227764304\\
-0.904287018736818	0.841683366733467\\
-0.889779559118236	0.856456534853726\\
-0.888536276495058	0.857715430861724\\
-0.87374749498998	0.87259395140516\\
-0.87259395140516	0.87374749498998\\
-0.857715430861723	0.888536276495059\\
-0.856456534853726	0.889779559118236\\
-0.841683366733467	0.904287018736818\\
-0.840120227764304	0.905811623246493\\
-0.82565130260521	0.919849404972135\\
-0.823580931228913	0.921843687374749\\
-0.809619238476954	0.935226388313504\\
-0.806834236431903	0.937875751503006\\
-0.793587174348697	0.950420655563976\\
-0.789875413833205	0.953907815631262\\
-0.777555110220441	0.965434634038255\\
-0.772699401575495	0.969939879759519\\
-0.761523046092184	0.980270497807276\\
-0.755300793077052	0.985971943887775\\
-0.745490981963928	0.994930173386556\\
-0.737673823769273	1.00200400801603\\
-0.729458917835671	1.00941534488687\\
-0.719812356934714	1.01803607214429\\
-0.713426853707415	1.0237274586441\\
-0.701709868598309	1.03406813627254\\
-0.697394789579158	1.03786772734357\\
-0.683359431420935	1.0501002004008\\
-0.681362725450902	1.0518371336525\\
-0.665330661322646	1.06564549199492\\
-0.664760715493398	1.06613226452906\\
-0.649298597194389	1.07931977971233\\
-0.645925521633467	1.08216432865731\\
-0.633266533066132	1.09282908504719\\
-0.626820685959869	1.09819639278557\\
-0.617234468937876	1.10617360742911\\
-0.607437360561867	1.11422845691383\\
-0.601202404809619	1.11935333090457\\
-0.587766190578266	1.13026052104208\\
-0.585170340681363	1.13236802527859\\
-0.569138276553106	1.14523788703217\\
-0.567811139716855	1.14629258517034\\
-0.55310621242485	1.15798627375855\\
-0.547576412709568	1.1623246492986\\
-0.537074148296593	1.17057247205745\\
-0.527023324535393	1.17835671342685\\
-0.521042084168337	1.18299552609021\\
-0.506140069367307	1.19438877755511\\
-0.50501002004008	1.19525426983635\\
-0.488977955911824	1.20740842663627\\
-0.484949536722487	1.21042084168337\\
-0.472945891783567	1.21941746732097\\
-0.463413492946464	1.22645290581162\\
-0.456913827655311	1.23126272035249\\
-0.441507718566508	1.24248496993988\\
-0.440881763527054	1.24294231115383\\
-0.424849699398798	1.25453676251989\\
-0.419258454076814	1.25851703406814\\
-0.408817635270541	1.26597588984069\\
-0.396612912558848	1.27454909819639\\
-0.392785571142285	1.2772481293273\\
-0.376753507014028	1.28839674703226\\
-0.373569228672946	1.29058116232465\\
-0.360721442885771	1.29943376454435\\
-0.350113050912796	1.30661322645291\\
-0.344689378757515	1.31030143811245\\
-0.328657314629258	1.32103094497034\\
-0.326212905833931	1.32264529058116\\
-0.312625250501002	1.33166661958351\\
-0.301862507325587	1.33867735470942\\
-0.296593186372745	1.34212926903676\\
-0.280561122244489	1.35246348991562\\
-0.277025015543469	1.35470941883768\\
-0.264529058116232	1.36269539866052\\
-0.251681988483172	1.37074148296593\\
-0.248496993987976	1.37274936655861\\
-0.232464929859719	1.38270942820289\\
-0.225804070271314	1.38677354709419\\
-0.216432865731463	1.3925319777012\\
-0.200400801603207	1.40218389871713\\
-0.199354164605839	1.40280561122244\\
-0.18436873747495	1.41177491003306\\
-0.172304746209675	1.4188376753507\\
-0.168336673346694	1.42117931698976\\
-0.152304609218437	1.43048820471916\\
-0.144609615201841	1.43486973947896\\
-0.13627254509018	1.43965727172208\\
-0.120240480961924	1.44868021399324\\
-0.116225809613082	1.45090180360721\\
-0.104208416833667	1.45761188252278\\
-0.0881763527054109	1.46635764453439\\
-0.0871039964118488	1.46693386773547\\
-0.0721442885771544	1.47504911080628\\
-0.0571824839285046	1.48296593186373\\
-0.0561122244488979	1.4835378630422\\
-0.0400801603206413	1.49197330402804\\
-0.0263923548855083	1.49899799599198\\
-0.0240480961923848	1.50021365028534\\
-0.00801603206412826	1.50838721810992\\
0.00534208362977422	1.51503006012024\\
0.00801603206412782	1.51637435979378\\
0.0240480961923843	1.52429203163416\\
0.0381120853590138	1.5310621242485\\
0.0400801603206409	1.53202038856421\\
0.0561122244488974	1.53968735190831\\
0.0720246507521716	1.54709418837675\\
0.0721442885771539	1.54715054509611\\
0.0881763527054105	1.55457121294478\\
0.104208416833667	1.56179131254381\\
0.107232969746725	1.56312625250501\\
0.120240480961924	1.56894006534054\\
0.13627254509018	1.57592145783621\\
0.143881511891679	1.57915831663327\\
0.152304609218437	1.58278875798391\\
0.168336673346693	1.58953387438088\\
0.182151844553411	1.59519038076152\\
0.18436873747495	1.5961105114565\\
0.200400801603206	1.60262103759574\\
0.216432865731463	1.60894625940275\\
0.2223458392454	1.61122244488978\\
0.232464929859719	1.61517375675617\\
0.248496993987976	1.62126759432409\\
0.264529058116232	1.62718266880621\\
0.264728499008129	1.62725450901804\\
0.280561122244489	1.63304332259554\\
0.296593186372745	1.63872829016932\\
0.309839766781273	1.64328657314629\\
0.312625250501002	1.64426000301504\\
0.328657314629258	1.64971340211983\\
0.344689378757515	1.65499412798524\\
0.358243878823842	1.65931863727455\\
0.360721442885771	1.66012192256018\\
0.376753507014028	1.66516920494166\\
0.392785571142285	1.67004713451038\\
0.408817635270541	1.67475843626556\\
0.41089959856551	1.67535070140281\\
0.424849699398798	1.67938676932915\\
0.440881763527054	1.68386013331379\\
0.456913827655311	1.68816846448149\\
0.469338124237108	1.69138276553106\\
0.472945891783567	1.69233266126292\\
0.488977955911824	1.69639640372907\\
0.50501002004008	1.70029546943752\\
0.521042084168337	1.70403137936056\\
0.536218089035682	1.70741482965932\\
0.537074148296593	1.70760921987228\\
0.55310621242485	1.71108929377282\\
0.569138276553106	1.71440479617826\\
0.585170340681363	1.71755653473623\\
0.601202404809619	1.72054510585176\\
0.617234468937876	1.72337089469617\\
0.61769230694783	1.72344689378758\\
0.633266533066132	1.72608284482946\\
0.649298597194389	1.72862928729111\\
0.665330661322646	1.73100862614715\\
0.681362725450902	1.73322049782681\\
0.697394789579159	1.73526431832439\\
0.713426853707415	1.73713928085242\\
0.729458917835672	1.73884435303621\\
0.736103830127034	1.73947895791583\\
0.745490981963928	1.74039413575393\\
0.761523046092185	1.74177886775506\\
0.777555110220441	1.74298556426633\\
0.793587174348698	1.74401234525025\\
0.809619238476954	1.74485708238278\\
0.825651302605211	1.7455173934902\\
0.841683366733467	1.74599063643181\\
0.857715430861724	1.74627390240855\\
0.87374749498998	1.74636400867599\\
0.889779559118236	1.74625749063801\\
0.905811623246493	1.7459505932957\\
0.921843687374749	1.74543926202365\\
0.937875751503006	1.74471913264387\\
0.953907815631262	1.74378552076484\\
0.969939879759519	1.74263341035096\\
0.985971943887775	1.74125744148496\\
1.00200400801603	1.73965189728283\\
1.00352268049825	1.73947895791583\\
1.01803607214429	1.73783268519641\\
1.03406813627254	1.73577585804698\\
1.0501002004008	1.73347211234391\\
1.06613226452906	1.73091404589739\\
1.08216432865731	1.72809381556575\\
1.09819639278557	1.72500311756322\\
1.10564751865407	1.72344689378758\\
1.11422845691383	1.72165334000978\\
1.13026052104208	1.71803409669879\\
1.14629258517034	1.71411771416669\\
1.1623246492986	1.70989367604385\\
1.17112750336767	1.70741482965932\\
1.17835671342685	1.7053705271386\\
1.19438877755511	1.70054175342807\\
1.21042084168337	1.69537053404818\\
1.22203719000819	1.69138276553106\\
1.22645290581162	1.68985615820079\\
1.24248496993988	1.68400551663308\\
1.25851703406814	1.67777005794228\\
1.26442417389282	1.67535070140281\\
1.27454909819639	1.67116233612137\\
1.29058116232465	1.66415314949634\\
1.30105597861111	1.65931863727455\\
1.30661322645291	1.65672216252411\\
1.32264529058116	1.64886225851567\\
1.33343388691115	1.64328657314629\\
1.33867735470942	1.64053691421055\\
1.35470941883768	1.63173558771747\\
1.36249278329385	1.62725450901804\\
1.37074148296593	1.62242437242321\\
1.38677354709419	1.61257486773745\\
1.38889116990037	1.61122244488978\\
1.40280561122244	1.60216208173138\\
1.4130266243971	1.59519038076152\\
1.4188376753507	1.59114209782024\\
1.43486973947896	1.57947972258033\\
1.435296637493	1.57915831663327\\
1.45090180360721	1.56712796955356\\
1.45588907348538	1.56312625250501\\
1.46693386773547	1.55403510022378\\
1.47505423605264	1.54709418837675\\
1.48296593186373	1.54014383542533\\
1.492943121336	1.5310621242485\\
1.49899799599198	1.52538664453138\\
1.50968413770199	1.51503006012024\\
1.51503006012024	1.50968413770199\\
1.52538664453138	1.49899799599198\\
1.5310621242485	1.492943121336\\
1.54014383542533	1.48296593186373\\
1.54709418837675	1.47505423605264\\
1.55403510022378	1.46693386773547\\
1.56312625250501	1.45588907348538\\
1.56712796955356	1.45090180360721\\
1.57915831663327	1.435296637493\\
1.57947972258033	1.43486973947896\\
1.59114209782024	1.4188376753507\\
1.59519038076152	1.4130266243971\\
1.60216208173138	1.40280561122244\\
1.61122244488978	1.38889116990037\\
1.61257486773745	1.38677354709419\\
1.62242437242321	1.37074148296593\\
1.62725450901804	1.36249278329385\\
1.63173558771747	1.35470941883768\\
1.64053691421055	1.33867735470942\\
1.64328657314629	1.33343388691115\\
1.64886225851567	1.32264529058116\\
1.65672216252411	1.30661322645291\\
1.65931863727455	1.30105597861111\\
1.66415314949634	1.29058116232465\\
1.67116233612137	1.27454909819639\\
1.67535070140281	1.26442417389282\\
1.67777005794228	1.25851703406814\\
1.68400551663308	1.24248496993988\\
1.68985615820079	1.22645290581162\\
1.69138276553106	1.22203719000819\\
1.69537053404818	1.21042084168337\\
1.70054175342807	1.19438877755511\\
1.7053705271386	1.17835671342685\\
1.70741482965932	1.17112750336767\\
1.70989367604385	1.1623246492986\\
1.71411771416669	1.14629258517034\\
1.71803409669879	1.13026052104208\\
1.72165334000978	1.11422845691383\\
1.72344689378758	1.10564751865407\\
1.72500311756322	1.09819639278557\\
1.72809381556575	1.08216432865731\\
1.73091404589739	1.06613226452906\\
1.73347211234391	1.0501002004008\\
1.73577585804698	1.03406813627254\\
1.73783268519641	1.01803607214429\\
1.73947895791583	1.00352268049825\\
1.73965189728283	1.00200400801603\\
1.74125744148496	0.985971943887775\\
1.74263341035096	0.969939879759519\\
1.74378552076484	0.953907815631262\\
1.74471913264387	0.937875751503006\\
1.74543926202365	0.921843687374749\\
1.7459505932957	0.905811623246493\\
1.74625749063801	0.889779559118236\\
1.74636400867599	0.87374749498998\\
1.74627390240855	0.857715430861724\\
1.74599063643181	0.841683366733467\\
1.7455173934902	0.825651302605211\\
1.74485708238278	0.809619238476954\\
1.74401234525025	0.793587174348698\\
1.74298556426633	0.777555110220441\\
1.74177886775506	0.761523046092185\\
1.74039413575393	0.745490981963928\\
1.73947895791583	0.736103830127031\\
1.73884435303621	0.729458917835672\\
1.73713928085242	0.713426853707415\\
1.73526431832439	0.697394789579159\\
1.73322049782681	0.681362725450902\\
1.73100862614715	0.665330661322646\\
1.72862928729111	0.649298597194389\\
1.72608284482946	0.633266533066132\\
1.72344689378758	0.61769230694783\\
1.72337089469617	0.617234468937876\\
1.72054510585176	0.601202404809619\\
1.71755653473623	0.585170340681363\\
1.71440479617826	0.569138276553106\\
1.71108929377282	0.55310621242485\\
1.70760921987228	0.537074148296593\\
1.70741482965932	0.536218089035681\\
1.70403137936056	0.521042084168337\\
1.70029546943752	0.50501002004008\\
1.69639640372907	0.488977955911824\\
1.69233266126292	0.472945891783567\\
1.69138276553106	0.469338124237108\\
1.68816846448149	0.456913827655311\\
1.68386013331379	0.440881763527054\\
1.67938676932915	0.424849699398798\\
1.67535070140281	0.410899598565509\\
1.67475843626556	0.408817635270541\\
1.67004713451038	0.392785571142285\\
1.66516920494166	0.376753507014028\\
1.66012192256018	0.360721442885771\\
1.65931863727455	0.358243878823842\\
1.65499412798524	0.344689378757515\\
1.64971340211983	0.328657314629258\\
1.64426000301504	0.312625250501002\\
1.64328657314629	0.309839766781273\\
1.63872829016932	0.296593186372745\\
1.63304332259554	0.280561122244489\\
1.62725450901804	0.264728499008129\\
1.62718266880621	0.264529058116232\\
1.6212675943241	0.248496993987976\\
1.61517375675617	0.232464929859719\\
1.61122244488978	0.2223458392454\\
1.60894625940275	0.216432865731463\\
1.60262103759574	0.200400801603206\\
1.5961105114565	0.18436873747495\\
1.59519038076152	0.182151844553411\\
1.58953387438088	0.168336673346693\\
1.58278875798391	0.152304609218437\\
1.57915831663327	0.143881511891678\\
1.57592145783621	0.13627254509018\\
1.56894006534054	0.120240480961924\\
1.56312625250501	0.107232969746725\\
1.56179131254381	0.104208416833667\\
1.55457121294478	0.0881763527054105\\
1.54715054509611	0.0721442885771539\\
1.54709418837675	0.0720246507521716\\
1.53968735190831	0.0561122244488974\\
1.53202038856421	0.0400801603206409\\
1.5310621242485	0.0381120853590138\\
1.52429203163416	0.0240480961923843\\
1.51637435979378	0.00801603206412782\\
1.51503006012024	0.00534208362977422\\
1.50838721810992	-0.00801603206412826\\
1.50021365028534	-0.0240480961923848\\
1.49899799599198	-0.0263923548855087\\
1.49197330402804	-0.0400801603206413\\
1.4835378630422	-0.0561122244488979\\
1.48296593186373	-0.0571824839285046\\
1.47504911080628	-0.0721442885771544\\
1.46693386773547	-0.0871039964118484\\
1.46635764453439	-0.0881763527054109\\
1.45761188252278	-0.104208416833667\\
1.45090180360721	-0.116225809613082\\
1.44868021399324	-0.120240480961924\\
1.43965727172208	-0.13627254509018\\
1.43486973947896	-0.144609615201841\\
1.43048820471916	-0.152304609218437\\
1.42117931698976	-0.168336673346694\\
1.4188376753507	-0.172304746209675\\
1.41177491003306	-0.18436873747495\\
1.40280561122244	-0.199354164605839\\
1.40218389871713	-0.200400801603207\\
1.3925319777012	-0.216432865731463\\
1.38677354709419	-0.225804070271314\\
1.38270942820289	-0.232464929859719\\
1.37274936655861	-0.248496993987976\\
1.37074148296593	-0.251681988483172\\
1.36269539866052	-0.264529058116232\\
1.35470941883768	-0.277025015543469\\
1.35246348991562	-0.280561122244489\\
1.34212926903676	-0.296593186372745\\
1.33867735470942	-0.301862507325588\\
1.33166661958351	-0.312625250501002\\
1.32264529058116	-0.326212905833931\\
1.32103094497034	-0.328657314629258\\
1.31030143811245	-0.344689378757515\\
1.30661322645291	-0.350113050912796\\
1.29943376454435	-0.360721442885771\\
1.29058116232465	-0.373569228672946\\
1.28839674703226	-0.376753507014028\\
1.2772481293273	-0.392785571142285\\
1.27454909819639	-0.396612912558848\\
1.26597588984069	-0.408817635270541\\
1.25851703406814	-0.419258454076814\\
1.25453676251989	-0.424849699398798\\
1.24294231115383	-0.440881763527054\\
1.24248496993988	-0.441507718566508\\
1.23126272035249	-0.456913827655311\\
1.22645290581162	-0.463413492946464\\
1.21941746732097	-0.472945891783567\\
1.21042084168337	-0.484949536722487\\
1.20740842663627	-0.488977955911824\\
1.19525426983635	-0.50501002004008\\
1.19438877755511	-0.506140069367307\\
1.18299552609021	-0.521042084168337\\
1.17835671342685	-0.527023324535393\\
1.17057247205745	-0.537074148296593\\
1.1623246492986	-0.547576412709568\\
1.15798627375855	-0.55310621242485\\
1.14629258517034	-0.567811139716855\\
1.14523788703217	-0.569138276553106\\
1.13236802527859	-0.585170340681363\\
1.13026052104208	-0.587766190578266\\
1.11935333090457	-0.601202404809619\\
1.11422845691383	-0.607437360561867\\
1.10617360742911	-0.617234468937876\\
1.09819639278557	-0.626820685959869\\
1.09282908504719	-0.633266533066132\\
1.08216432865731	-0.645925521633467\\
1.07931977971233	-0.649298597194389\\
1.06613226452906	-0.664760715493398\\
1.06564549199492	-0.665330661322646\\
1.0518371336525	-0.681362725450902\\
1.0501002004008	-0.683359431420935\\
1.03786772734357	-0.697394789579158\\
1.03406813627254	-0.701709868598309\\
1.0237274586441	-0.713426853707415\\
1.01803607214429	-0.719812356934714\\
1.00941534488687	-0.729458917835671\\
1.00200400801603	-0.737673823769273\\
0.994930173386556	-0.745490981963928\\
0.985971943887775	-0.755300793077053\\
0.980270497807276	-0.761523046092184\\
0.969939879759519	-0.772699401575495\\
0.965434634038255	-0.777555110220441\\
0.953907815631262	-0.789875413833205\\
0.950420655563976	-0.793587174348697\\
0.937875751503006	-0.806834236431903\\
0.935226388313504	-0.809619238476954\\
0.921843687374749	-0.823580931228913\\
0.919849404972135	-0.82565130260521\\
0.905811623246493	-0.840120227764305\\
0.904287018736818	-0.841683366733467\\
0.889779559118236	-0.856456534853726\\
0.888536276495059	-0.857715430861723\\
0.87374749498998	-0.87259395140516\\
0.87259395140516	-0.87374749498998\\
0.857715430861724	-0.888536276495058\\
0.856456534853726	-0.889779559118236\\
0.841683366733467	-0.904287018736818\\
0.840120227764304	-0.905811623246493\\
0.825651302605211	-0.919849404972134\\
0.823580931228913	-0.92184368737475\\
0.809619238476954	-0.935226388313504\\
0.806834236431902	-0.937875751503006\\
0.793587174348698	-0.950420655563976\\
0.789875413833205	-0.953907815631263\\
0.777555110220441	-0.965434634038255\\
0.772699401575495	-0.969939879759519\\
0.761523046092185	-0.980270497807276\\
0.755300793077052	-0.985971943887776\\
0.745490981963928	-0.994930173386556\\
0.737673823769273	-1.00200400801603\\
0.729458917835672	-1.00941534488687\\
0.719812356934714	-1.01803607214429\\
0.713426853707415	-1.0237274586441\\
0.701709868598308	-1.03406813627255\\
0.697394789579159	-1.03786772734357\\
0.683359431420935	-1.0501002004008\\
0.681362725450902	-1.0518371336525\\
0.665330661322646	-1.06564549199492\\
0.664760715493397	-1.06613226452906\\
0.649298597194389	-1.07931977971233\\
0.645925521633466	-1.08216432865731\\
0.633266533066132	-1.09282908504719\\
0.626820685959868	-1.09819639278557\\
0.617234468937876	-1.10617360742911\\
0.607437360561867	-1.11422845691383\\
0.601202404809619	-1.11935333090457\\
0.587766190578265	-1.13026052104208\\
0.585170340681363	-1.13236802527859\\
0.569138276553106	-1.14523788703217\\
0.567811139716855	-1.14629258517034\\
0.55310621242485	-1.15798627375855\\
0.547576412709567	-1.1623246492986\\
0.537074148296593	-1.17057247205745\\
0.527023324535393	-1.17835671342685\\
0.521042084168337	-1.18299552609021\\
0.506140069367307	-1.19438877755511\\
0.50501002004008	-1.19525426983635\\
0.488977955911824	-1.20740842663627\\
0.484949536722487	-1.21042084168337\\
0.472945891783567	-1.21941746732097\\
0.463413492946463	-1.22645290581162\\
0.456913827655311	-1.23126272035249\\
0.441507718566507	-1.24248496993988\\
0.440881763527054	-1.24294231115383\\
0.424849699398798	-1.25453676251989\\
0.419258454076814	-1.25851703406814\\
0.408817635270541	-1.26597588984069\\
0.396612912558848	-1.27454909819639\\
0.392785571142285	-1.2772481293273\\
0.376753507014028	-1.28839674703226\\
0.373569228672946	-1.29058116232465\\
0.360721442885771	-1.29943376454435\\
0.350113050912796	-1.30661322645291\\
0.344689378757515	-1.31030143811245\\
0.328657314629258	-1.32103094497034\\
0.32621290583393	-1.32264529058116\\
0.312625250501002	-1.33166661958351\\
0.301862507325587	-1.33867735470942\\
0.296593186372745	-1.34212926903676\\
0.280561122244489	-1.35246348991562\\
0.277025015543468	-1.35470941883768\\
0.264529058116232	-1.36269539866052\\
0.251681988483172	-1.37074148296593\\
0.248496993987976	-1.37274936655861\\
0.232464929859719	-1.38270942820289\\
0.225804070271314	-1.38677354709419\\
0.216432865731463	-1.3925319777012\\
0.200400801603206	-1.40218389871713\\
0.199354164605838	-1.40280561122244\\
0.18436873747495	-1.41177491003306\\
0.172304746209675	-1.4188376753507\\
0.168336673346693	-1.42117931698976\\
0.152304609218437	-1.43048820471916\\
0.144609615201841	-1.43486973947896\\
0.13627254509018	-1.43965727172208\\
0.120240480961924	-1.44868021399324\\
0.116225809613082	-1.45090180360721\\
0.104208416833667	-1.45761188252278\\
0.0881763527054105	-1.46635764453439\\
0.0871039964118479	-1.46693386773547\\
0.0721442885771539	-1.47504911080628\\
0.0571824839285042	-1.48296593186373\\
0.0561122244488974	-1.4835378630422\\
0.0400801603206409	-1.49197330402804\\
0.0263923548855087	-1.49899799599198\\
0.0240480961923843	-1.50021365028534\\
0.00801603206412782	-1.50838721810992\\
-0.00534208362977421	-1.51503006012024\\
-0.00801603206412826	-1.51637435979378\\
-0.0240480961923848	-1.52429203163416\\
-0.0381120853590133	-1.5310621242485\\
-0.0400801603206413	-1.53202038856421\\
-0.0561122244488979	-1.53968735190831\\
-0.0720246507521716	-1.54709418837675\\
-0.0721442885771544	-1.54715054509611\\
-0.0881763527054109	-1.55457121294478\\
-0.104208416833667	-1.56179131254381\\
-0.107232969746724	-1.56312625250501\\
-0.120240480961924	-1.56894006534054\\
-0.13627254509018	-1.57592145783621\\
-0.143881511891677	-1.57915831663327\\
-0.152304609218437	-1.58278875798391\\
-0.168336673346694	-1.58953387438088\\
-0.18215184455341	-1.59519038076152\\
-0.18436873747495	-1.5961105114565\\
-0.200400801603207	-1.60262103759575\\
-0.216432865731463	-1.60894625940275\\
-0.222345839245399	-1.61122244488978\\
-0.232464929859719	-1.61517375675617\\
-0.248496993987976	-1.62126759432409\\
-0.264529058116232	-1.62718266880621\\
-0.264728499008128	-1.62725450901804\\
-0.280561122244489	-1.63304332259554\\
-0.296593186372745	-1.63872829016932\\
-0.309839766781271	-1.64328657314629\\
-0.312625250501002	-1.64426000301504\\
-0.328657314629258	-1.64971340211983\\
-0.344689378757515	-1.65499412798524\\
-0.358243878823842	-1.65931863727455\\
-0.360721442885771	-1.66012192256018\\
-0.376753507014028	-1.66516920494166\\
-0.392785571142285	-1.67004713451038\\
-0.408817635270541	-1.67475843626556\\
-0.410899598565508	-1.67535070140281\\
-0.424849699398798	-1.67938676932915\\
-0.440881763527054	-1.68386013331379\\
-0.456913827655311	-1.68816846448149\\
-0.469338124237107	-1.69138276553106\\
-0.472945891783567	-1.69233266126292\\
-0.488977955911824	-1.69639640372907\\
-0.50501002004008	-1.70029546943752\\
-0.521042084168337	-1.70403137936056\\
-0.53621808903568	-1.70741482965932\\
-0.537074148296593	-1.70760921987228\\
-0.55310621242485	-1.71108929377282\\
-0.569138276553106	-1.71440479617826\\
-0.585170340681363	-1.71755653473623\\
-0.601202404809619	-1.72054510585176\\
-0.617234468937876	-1.72337089469617\\
-0.617692306947828	-1.72344689378758\\
-0.633266533066132	-1.72608284482946\\
-0.649298597194389	-1.72862928729111\\
-0.665330661322646	-1.73100862614715\\
-0.681362725450902	-1.73322049782681\\
-0.697394789579158	-1.73526431832439\\
-0.713426853707415	-1.73713928085242\\
-0.729458917835671	-1.73884435303621\\
-0.736103830127028	-1.73947895791583\\
-0.745490981963928	-1.74039413575393\\
-0.761523046092184	-1.74177886775506\\
-0.777555110220441	-1.74298556426633\\
-0.793587174348697	-1.74401234525024\\
-0.809619238476954	-1.74485708238278\\
-0.82565130260521	-1.7455173934902\\
-0.841683366733467	-1.74599063643181\\
-0.857715430861723	-1.74627390240855\\
-0.87374749498998	-1.74636400867599\\
-0.889779559118236	-1.74625749063801\\
-0.905811623246493	-1.7459505932957\\
-0.92184368737475	-1.74543926202365\\
-0.937875751503006	-1.74471913264387\\
-0.953907815631263	-1.74378552076484\\
-0.969939879759519	-1.74263341035096\\
-0.985971943887776	-1.74125744148496\\
-1.00200400801603	-1.73965189728283\\
-1.00352268049825	-1.73947895791583\\
-1.01803607214429	-1.7378326851964\\
-1.03406813627255	-1.73577585804698\\
-1.0501002004008	-1.73347211234391\\
-1.06613226452906	-1.73091404589739\\
-1.08216432865731	-1.72809381556575\\
-1.09819639278557	-1.72500311756322\\
-1.10564751865407	-1.72344689378758\\
-1.11422845691383	-1.72165334000978\\
-1.13026052104208	-1.71803409669879\\
-1.14629258517034	-1.71411771416669\\
-1.1623246492986	-1.70989367604385\\
-1.17112750336767	-1.70741482965932\\
-1.17835671342685	-1.70537052713861\\
-1.19438877755511	-1.70054175342807\\
-1.21042084168337	-1.69537053404818\\
-1.22203719000819	-1.69138276553106\\
-1.22645290581162	-1.68985615820079\\
-1.24248496993988	-1.68400551663308\\
-1.25851703406814	-1.67777005794228\\
-1.26442417389282	-1.67535070140281\\
-1.27454909819639	-1.67116233612137\\
-1.29058116232465	-1.66415314949634\\
-1.30105597861111	-1.65931863727455\\
-1.30661322645291	-1.65672216252411\\
-1.32264529058116	-1.64886225851567\\
-1.33343388691115	-1.64328657314629\\
-1.33867735470942	-1.64053691421055\\
-1.35470941883768	-1.63173558771747\\
-1.36249278329385	-1.62725450901804\\
-1.37074148296593	-1.62242437242321\\
-1.38677354709419	-1.61257486773745\\
-1.38889116990037	-1.61122244488978\\
-1.40280561122244	-1.60216208173138\\
-1.4130266243971	-1.59519038076152\\
-1.4188376753507	-1.59114209782024\\
-1.43486973947896	-1.57947972258033\\
-1.435296637493	-1.57915831663327\\
-1.45090180360721	-1.56712796955356\\
-1.45588907348538	-1.56312625250501\\
-1.46693386773547	-1.55403510022378\\
-1.47505423605264	-1.54709418837675\\
-1.48296593186373	-1.54014383542533\\
-1.492943121336	-1.5310621242485\\
-1.49899799599198	-1.52538664453138\\
-1.50968413770199	-1.51503006012024\\
-1.51503006012024	-1.50968413770199\\
-1.52538664453138	-1.49899799599198\\
-1.5310621242485	-1.492943121336\\
-1.54014383542533	-1.48296593186373\\
-1.54709418837675	-1.47505423605264\\
-1.55403510022378	-1.46693386773547\\
-1.56312625250501	-1.45588907348538\\
-1.56712796955356	-1.45090180360721\\
-1.57915831663327	-1.435296637493\\
-1.57947972258033	-1.43486973947896\\
-1.59114209782024	-1.4188376753507\\
-1.59519038076152	-1.4130266243971\\
-1.60216208173138	-1.40280561122244\\
-1.61122244488978	-1.38889116990037\\
-1.61257486773745	-1.38677354709419\\
-1.62242437242321	-1.37074148296593\\
-1.62725450901804	-1.36249278329385\\
-1.63173558771747	-1.35470941883768\\
-1.64053691421055	-1.33867735470942\\
-1.64328657314629	-1.33343388691115\\
-1.64886225851567	-1.32264529058116\\
-1.65672216252411	-1.30661322645291\\
-1.65931863727455	-1.30105597861111\\
-1.66415314949634	-1.29058116232465\\
-1.67116233612137	-1.27454909819639\\
-1.67535070140281	-1.26442417389282\\
-1.67777005794228	-1.25851703406814\\
-1.68400551663308	-1.24248496993988\\
-1.68985615820079	-1.22645290581162\\
-1.69138276553106	-1.22203719000819\\
-1.69537053404818	-1.21042084168337\\
-1.70054175342807	-1.19438877755511\\
-1.70537052713861	-1.17835671342685\\
-1.70741482965932	-1.17112750336767\\
-1.70989367604385	-1.1623246492986\\
-1.71411771416669	-1.14629258517034\\
-1.71803409669879	-1.13026052104208\\
-1.72165334000978	-1.11422845691383\\
-1.72344689378758	-1.10564751865407\\
-1.72500311756322	-1.09819639278557\\
-1.72809381556575	-1.08216432865731\\
-1.73091404589739	-1.06613226452906\\
-1.73347211234391	-1.0501002004008\\
-1.73577585804698	-1.03406813627255\\
-1.73783268519641	-1.01803607214429\\
-1.73947895791583	-1.00352268049825\\
-1.73965189728283	-1.00200400801603\\
-1.74125744148496	-0.985971943887776\\
-1.74263341035096	-0.969939879759519\\
-1.74378552076484	-0.953907815631263\\
-1.74471913264387	-0.937875751503006\\
-1.74543926202365	-0.92184368737475\\
-1.7459505932957	-0.905811623246493\\
-1.74625749063801	-0.889779559118236\\
-1.74636400867599	-0.87374749498998\\
-1.74627390240855	-0.857715430861723\\
-1.74599063643181	-0.841683366733467\\
-1.7455173934902	-0.82565130260521\\
-1.74485708238278	-0.809619238476954\\
-1.74401234525024	-0.793587174348697\\
-1.74298556426633	-0.777555110220441\\
-1.74177886775506	-0.761523046092184\\
-1.74039413575393	-0.745490981963928\\
-1.73947895791583	-0.736103830127027\\
}--cycle;


\addplot[area legend,solid,fill=mycolor4,draw=black,forget plot]
table[row sep=crcr] {%
x	y\\
-1.48296593186373	-0.569234343815055\\
-1.48295281504123	-0.569138276553106\\
-1.4805661763775	-0.55310621242485\\
-1.47799052016043	-0.537074148296593\\
-1.47522695329842	-0.521042084168337\\
-1.47227631523678	-0.50501002004008\\
-1.46913918171142	-0.488977955911824\\
-1.46693386773547	-0.478342874648239\\
-1.46583060818897	-0.472945891783567\\
-1.46236860897112	-0.456913827655311\\
-1.45872459997283	-0.440881763527054\\
-1.4548982625651	-0.424849699398798\\
-1.45090180360721	-0.408868709077131\\
-1.45088920268315	-0.408817635270541\\
-1.44675469643562	-0.392785571142285\\
-1.44244011560983	-0.376753507014028\\
-1.43794428572241	-0.360721442885771\\
-1.43486973947896	-0.350180124738909\\
-1.43328854112771	-0.344689378757515\\
-1.42849629502376	-0.328657314629258\\
-1.42352292752558	-0.312625250501002\\
-1.4188376753507	-0.298055089992475\\
-1.41837322806493	-0.296593186372745\\
-1.41310987459442	-0.280561122244489\\
-1.40766404416331	-0.264529058116232\\
-1.40280561122244	-0.250690013324401\\
-1.40204447569671	-0.248496993987976\\
-1.39631268255537	-0.232464929859719\\
-1.39039559253707	-0.216432865731463\\
-1.38677354709419	-0.206905683974875\\
-1.38432708256506	-0.200400801603207\\
-1.37812570799675	-0.18436873747495\\
-1.37173475319888	-0.168336673346694\\
-1.37074148296593	-0.165907613521522\\
-1.36523524615038	-0.152304609218437\\
-1.35855937216241	-0.13627254509018\\
-1.35470941883768	-0.127267448086717\\
-1.35173399332406	-0.120240480961924\\
-1.34477275047677	-0.104208416833667\\
-1.33867735470942	-0.0905458980445204\\
-1.33762980774476	-0.0881763527054109\\
-1.3303818507885	-0.0721442885771544\\
-1.32293302577473	-0.0561122244488979\\
-1.32264529058116	-0.0555054938388772\\
-1.31539175071737	-0.0400801603206413\\
-1.30765040111404	-0.0240480961923848\\
-1.30661322645291	-0.0219430554655078\\
-1.29980566904814	-0.00801603206412826\\
-1.2917681093382	0.00801603206412782\\
-1.29058116232465	0.0103380441932435\\
-1.28362498230843	0.0240480961923843\\
-1.27528661264128	0.0400801603206409\\
-1.27454909819639	0.0414726849145835\\
-1.26684923185225	0.0561122244488974\\
-1.25851703406814	0.0715695813918983\\
-1.25820926529107	0.0721442885771539\\
-1.24947612150027	0.0881763527054105\\
-1.24248496993988	0.100714368128787\\
-1.24054843329175	0.104208416833667\\
-1.2315015057197	0.120240480961924\\
-1.22645290581162	0.129003296617097\\
-1.22228788771598	0.13627254509018\\
-1.2129193682588	0.152304609218437\\
-1.21042084168337	0.156506128689257\\
-1.20342071850968	0.168336673346693\\
-1.19438877755511	0.183276866398329\\
-1.19373174742705	0.18436873747495\\
-1.18393810910758	0.200400801603206\\
-1.17835671342685	0.209357201517582\\
-1.17396545655276	0.216432865731463\\
-1.16382929072242	0.232464929859719\\
-1.1623246492986	0.234809343276196\\
-1.15357152140237	0.248496993987976\\
-1.14629258517034	0.259660803112709\\
-1.14312863030488	0.264529058116232\\
-1.13253623903881	0.280561122244489\\
-1.13026052104208	0.283953974625914\\
-1.12180582169426	0.296593186372745\\
-1.11422845691383	0.307716868567398\\
-1.11089266623192	0.312625250501002\\
-1.09982201900944	0.328657314629258\\
-1.09819639278557	0.330979562311906\\
-1.08861656130972	0.344689378757515\\
-1.08216432865731	0.353767807594801\\
-1.07722927765783	0.360721442885771\\
-1.06613226452906	0.376100759386693\\
-1.06566176267025	0.376753507014028\\
-1.05396831223254	0.392785571142285\\
-1.0501002004008	0.398008471694325\\
-1.04209910255601	0.408817635270541\\
-1.03406813627255	0.419501460522291\\
-1.03004848067485	0.424849699398798\\
-1.01803607214429	0.440594941548498\\
-1.01781719660689	0.440881763527054\\
-1.00545120195497	0.456913827655311\\
-1.00200400801603	0.461320868074722\\
-0.992904040742536	0.472945891783567\\
-0.985971943887776	0.481679842319414\\
-0.980172864193027	0.488977955911824\\
-0.969939879759519	0.501684404901084\\
-0.967257549946234	0.50501002004008\\
-0.95416092736609	0.521042084168337\\
-0.953907815631263	0.521348942385413\\
-0.940910424542357	0.537074148296593\\
-0.937875751503006	0.540700813740435\\
-0.927470125414012	0.55310621242485\\
-0.92184368737475	0.559734593334918\\
-0.91383898556294	0.569138276553106\\
-0.905811623246493	0.578460782177181\\
-0.900015688263405	0.585170340681363\\
-0.889779559118236	0.596889279502508\\
-0.885998639694972	0.601202404809619\\
-0.87374749498998	0.615029410609041\\
-0.871785963527117	0.617234468937876\\
-0.857715430861723	0.632889952947605\\
-0.857375494855174	0.633266533066132\\
-0.84277693688987	0.649298597194389\\
-0.841683366733467	0.650488515867726\\
-0.827976283664954	0.665330661322646\\
-0.82565130260521	0.667825098240989\\
-0.812966673195804	0.681362725450902\\
-0.809619238476954	0.684903928573191\\
-0.797744839820967	0.697394789579159\\
-0.793587174348697	0.701731659444613\\
-0.782307190244573	0.713426853707415\\
-0.777555110220441	0.718314509342967\\
-0.766649793471145	0.729458917835672\\
-0.761523046092184	0.73465827955036\\
-0.750768369901502	0.745490981963928\\
-0.745490981963928	0.750768369901502\\
-0.73465827955036	0.761523046092185\\
-0.729458917835671	0.766649793471146\\
-0.718314509342966	0.777555110220441\\
-0.713426853707415	0.782307190244573\\
-0.701731659444612	0.793587174348698\\
-0.697394789579158	0.797744839820967\\
-0.68490392857319	0.809619238476954\\
-0.681362725450902	0.812966673195805\\
-0.667825098240989	0.825651302605211\\
-0.665330661322646	0.827976283664954\\
-0.650488515867726	0.841683366733467\\
-0.649298597194389	0.842776936889869\\
-0.633266533066132	0.857375494855175\\
-0.632889952947604	0.857715430861724\\
-0.617234468937876	0.871785963527117\\
-0.61502941060904	0.87374749498998\\
-0.601202404809619	0.885998639694972\\
-0.596889279502509	0.889779559118236\\
-0.585170340681363	0.900015688263405\\
-0.578460782177182	0.905811623246493\\
-0.569138276553106	0.91383898556294\\
-0.559734593334919	0.921843687374749\\
-0.55310621242485	0.927470125414012\\
-0.540700813740435	0.937875751503006\\
-0.537074148296593	0.940910424542356\\
-0.521348942385413	0.953907815631262\\
-0.521042084168337	0.954160927366089\\
-0.50501002004008	0.967257549946234\\
-0.501684404901084	0.969939879759519\\
-0.488977955911824	0.980172864193027\\
-0.481679842319414	0.985971943887775\\
-0.472945891783567	0.992904040742536\\
-0.461320868074722	1.00200400801603\\
-0.456913827655311	1.00545120195497\\
-0.440881763527054	1.01781719660689\\
-0.440594941548499	1.01803607214429\\
-0.424849699398798	1.03004848067485\\
-0.419501460522291	1.03406813627254\\
-0.408817635270541	1.04209910255601\\
-0.398008471694326	1.0501002004008\\
-0.392785571142285	1.05396831223254\\
-0.376753507014028	1.06566176267025\\
-0.376100759386693	1.06613226452906\\
-0.360721442885771	1.07722927765783\\
-0.353767807594802	1.08216432865731\\
-0.344689378757515	1.08861656130973\\
-0.330979562311907	1.09819639278557\\
-0.328657314629258	1.09982201900944\\
-0.312625250501002	1.11089266623192\\
-0.307716868567398	1.11422845691383\\
-0.296593186372745	1.12180582169426\\
-0.283953974625914	1.13026052104208\\
-0.280561122244489	1.13253623903881\\
-0.264529058116232	1.14312863030488\\
-0.259660803112709	1.14629258517034\\
-0.248496993987976	1.15357152140237\\
-0.234809343276196	1.1623246492986\\
-0.232464929859719	1.16382929072242\\
-0.216432865731463	1.17396545655276\\
-0.209357201517583	1.17835671342685\\
-0.200400801603207	1.18393810910758\\
-0.18436873747495	1.19373174742705\\
-0.183276866398329	1.19438877755511\\
-0.168336673346694	1.20342071850968\\
-0.156506128689257	1.21042084168337\\
-0.152304609218437	1.2129193682588\\
-0.13627254509018	1.22228788771598\\
-0.129003296617097	1.22645290581162\\
-0.120240480961924	1.2315015057197\\
-0.104208416833667	1.24054843329175\\
-0.100714368128787	1.24248496993988\\
-0.0881763527054109	1.24947612150027\\
-0.0721442885771544	1.25820926529107\\
-0.0715695813918983	1.25851703406814\\
-0.0561122244488979	1.26684923185225\\
-0.041472684914584	1.27454909819639\\
-0.0400801603206413	1.27528661264128\\
-0.0240480961923848	1.28362498230843\\
-0.0103380441932435	1.29058116232465\\
-0.00801603206412826	1.2917681093382\\
0.00801603206412782	1.29980566904814\\
0.0219430554655078	1.30661322645291\\
0.0240480961923843	1.30765040111404\\
0.0400801603206409	1.31539175071737\\
0.0555054938388772	1.32264529058116\\
0.0561122244488974	1.32293302577472\\
0.0721442885771539	1.3303818507885\\
0.0881763527054105	1.33762980774475\\
0.0905458980445213	1.33867735470942\\
0.104208416833667	1.34477275047677\\
0.120240480961924	1.35173399332406\\
0.127267448086717	1.35470941883768\\
0.13627254509018	1.35855937216241\\
0.152304609218437	1.36523524615038\\
0.165907613521522	1.37074148296593\\
0.168336673346693	1.37173475319888\\
0.18436873747495	1.37812570799675\\
0.200400801603206	1.38432708256506\\
0.206905683974875	1.38677354709419\\
0.216432865731463	1.39039559253707\\
0.232464929859719	1.39631268255537\\
0.248496993987976	1.40204447569671\\
0.250690013324401	1.40280561122244\\
0.264529058116232	1.40766404416331\\
0.280561122244489	1.41310987459442\\
0.296593186372745	1.41837322806493\\
0.298055089992475	1.4188376753507\\
0.312625250501002	1.42352292752558\\
0.328657314629258	1.42849629502376\\
0.344689378757515	1.43328854112771\\
0.350180124738909	1.43486973947896\\
0.360721442885771	1.43794428572241\\
0.376753507014028	1.44244011560983\\
0.392785571142285	1.44675469643562\\
0.408817635270541	1.45088920268315\\
0.408868709077129	1.45090180360721\\
0.424849699398798	1.4548982625651\\
0.440881763527054	1.45872459997283\\
0.456913827655311	1.46236860897112\\
0.472945891783567	1.46583060818897\\
0.478342874648239	1.46693386773547\\
0.488977955911824	1.46913918171142\\
0.50501002004008	1.47227631523678\\
0.521042084168337	1.47522695329842\\
0.537074148296593	1.47799052016043\\
0.55310621242485	1.4805661763775\\
0.569138276553106	1.48295281504123\\
0.569234343815057	1.48296593186373\\
0.585170340681363	1.48517536135449\\
0.601202404809619	1.48720258701048\\
0.617234468937876	1.4890325162843\\
0.633266533066132	1.49066304635659\\
0.649298597194389	1.49209177925457\\
0.665330661322646	1.49331601461501\\
0.681362725450902	1.49433274173063\\
0.697394789579159	1.49513863085095\\
0.713426853707415	1.49573002370592\\
0.729458917835672	1.49610292321817\\
0.745490981963928	1.4962529823663\\
0.761523046092185	1.49617549215899\\
0.777555110220441	1.49586536867595\\
0.793587174348698	1.49531713912846\\
0.809619238476954	1.49452492688824\\
0.825651302605211	1.49348243542961\\
0.841683366733467	1.49218293112526\\
0.857715430861724	1.49061922483166\\
0.87374749498998	1.48878365219481\\
0.889779559118236	1.4866680526022\\
0.905811623246493	1.48426374670064\\
0.91355685098122	1.48296593186373\\
0.921843687374749	1.48157013714079\\
0.937875751503006	1.47857737254268\\
0.953907815631262	1.47526648644312\\
0.969939879759519	1.47162622253922\\
0.985971943887775	1.46764465392901\\
0.988630527721551	1.46693386773547\\
1.00200400801603	1.46332379681357\\
1.01803607214429	1.45863741265707\\
1.03406813627254	1.4535679301691\\
1.04194631406362	1.45090180360721\\
1.0501002004008	1.4481069472665\\
1.06613226452906	1.44223614224617\\
1.08216432865731	1.43593050221557\\
1.08471625209384	1.43486973947896\\
1.09819639278557	1.42917617946644\\
1.11422845691383	1.421945674948\\
1.12074455217999	1.4188376753507\\
1.13026052104208	1.41421392881124\\
1.14629258517034	1.40595435077796\\
1.15209251422735	1.40280561122244\\
1.1623246492986	1.39713185710627\\
1.17835671342685	1.38771867472358\\
1.1798960501539	1.38677354709419\\
1.19438877755511	1.37766019652171\\
1.20481783866288	1.37074148296593\\
1.21042084168337	1.36692705011549\\
1.22645290581162	1.35546566023473\\
1.22746821035219	1.35470941883768\\
1.24248496993988	1.34319798849057\\
1.24811929248096	1.33867735470942\\
1.25851703406814	1.33007317305076\\
1.2671166818126	1.32264529058116\\
1.27454909819639	1.31600962282144\\
1.28465850444922	1.30661322645291\\
1.29058116232465	1.30091004993293\\
1.30091004993293	1.29058116232465\\
1.30661322645291	1.28465850444922\\
1.31600962282144	1.27454909819639\\
1.32264529058116	1.2671166818126\\
1.33007317305076	1.25851703406814\\
1.33867735470942	1.24811929248096\\
1.34319798849057	1.24248496993988\\
1.35470941883768	1.22746821035219\\
1.35546566023473	1.22645290581162\\
1.36692705011549	1.21042084168337\\
1.37074148296593	1.20481783866288\\
1.37766019652171	1.19438877755511\\
1.38677354709419	1.1798960501539\\
1.38771867472358	1.17835671342685\\
1.39713185710627	1.1623246492986\\
1.40280561122244	1.15209251422735\\
1.40595435077796	1.14629258517034\\
1.41421392881124	1.13026052104208\\
1.4188376753507	1.12074455217999\\
1.421945674948	1.11422845691383\\
1.42917617946644	1.09819639278557\\
1.43486973947896	1.08471625209384\\
1.43593050221557	1.08216432865731\\
1.44223614224617	1.06613226452906\\
1.4481069472665	1.0501002004008\\
1.45090180360721	1.04194631406362\\
1.4535679301691	1.03406813627254\\
1.45863741265707	1.01803607214429\\
1.46332379681357	1.00200400801603\\
1.46693386773547	0.988630527721552\\
1.46764465392901	0.985971943887775\\
1.47162622253922	0.969939879759519\\
1.47526648644312	0.953907815631262\\
1.47857737254268	0.937875751503006\\
1.48157013714079	0.921843687374749\\
1.48296593186373	0.91355685098122\\
1.48426374670064	0.905811623246493\\
1.4866680526022	0.889779559118236\\
1.48878365219481	0.87374749498998\\
1.49061922483166	0.857715430861724\\
1.49218293112526	0.841683366733467\\
1.49348243542961	0.825651302605211\\
1.49452492688824	0.809619238476954\\
1.49531713912846	0.793587174348698\\
1.49586536867595	0.777555110220441\\
1.49617549215899	0.761523046092185\\
1.4962529823663	0.745490981963928\\
1.49610292321817	0.729458917835672\\
1.49573002370592	0.713426853707415\\
1.49513863085095	0.697394789579159\\
1.49433274173063	0.681362725450902\\
1.49331601461501	0.665330661322646\\
1.49209177925457	0.649298597194389\\
1.49066304635659	0.633266533066132\\
1.4890325162843	0.617234468937876\\
1.48720258701048	0.601202404809619\\
1.48517536135449	0.585170340681363\\
1.48296593186373	0.569234343815055\\
1.48295281504123	0.569138276553106\\
1.4805661763775	0.55310621242485\\
1.47799052016043	0.537074148296593\\
1.47522695329842	0.521042084168337\\
1.47227631523678	0.50501002004008\\
1.46913918171142	0.488977955911824\\
1.46693386773547	0.478342874648239\\
1.46583060818897	0.472945891783567\\
1.46236860897112	0.456913827655311\\
1.45872459997283	0.440881763527054\\
1.4548982625651	0.424849699398798\\
1.45090180360721	0.40886870907713\\
1.45088920268315	0.408817635270541\\
1.44675469643562	0.392785571142285\\
1.44244011560983	0.376753507014028\\
1.43794428572241	0.360721442885771\\
1.43486973947896	0.350180124738909\\
1.43328854112771	0.344689378757515\\
1.42849629502376	0.328657314629258\\
1.42352292752558	0.312625250501002\\
1.4188376753507	0.298055089992475\\
1.41837322806493	0.296593186372745\\
1.41310987459442	0.280561122244489\\
1.40766404416331	0.264529058116232\\
1.40280561122244	0.250690013324401\\
1.40204447569671	0.248496993987976\\
1.39631268255537	0.232464929859719\\
1.39039559253707	0.216432865731463\\
1.38677354709419	0.206905683974875\\
1.38432708256506	0.200400801603206\\
1.37812570799675	0.18436873747495\\
1.37173475319888	0.168336673346693\\
1.37074148296593	0.165907613521522\\
1.36523524615038	0.152304609218437\\
1.35855937216241	0.13627254509018\\
1.35470941883768	0.127267448086717\\
1.35173399332406	0.120240480961924\\
1.34477275047677	0.104208416833667\\
1.33867735470942	0.0905458980445204\\
1.33762980774476	0.0881763527054105\\
1.3303818507885	0.0721442885771539\\
1.32293302577472	0.0561122244488974\\
1.32264529058116	0.0555054938388772\\
1.31539175071737	0.0400801603206409\\
1.30765040111404	0.0240480961923843\\
1.30661322645291	0.0219430554655074\\
1.29980566904814	0.00801603206412782\\
1.2917681093382	-0.00801603206412826\\
1.29058116232465	-0.0103380441932435\\
1.28362498230843	-0.0240480961923848\\
1.27528661264128	-0.0400801603206413\\
1.27454909819639	-0.0414726849145835\\
1.26684923185225	-0.0561122244488979\\
1.25851703406814	-0.0715695813918983\\
1.25820926529107	-0.0721442885771544\\
1.24947612150027	-0.0881763527054109\\
1.24248496993988	-0.100714368128787\\
1.24054843329175	-0.104208416833667\\
1.2315015057197	-0.120240480961924\\
1.22645290581162	-0.129003296617097\\
1.22228788771598	-0.13627254509018\\
1.2129193682588	-0.152304609218437\\
1.21042084168337	-0.156506128689257\\
1.20342071850968	-0.168336673346694\\
1.19438877755511	-0.183276866398329\\
1.19373174742705	-0.18436873747495\\
1.18393810910758	-0.200400801603207\\
1.17835671342685	-0.209357201517583\\
1.17396545655276	-0.216432865731463\\
1.16382929072242	-0.232464929859719\\
1.1623246492986	-0.234809343276196\\
1.15357152140237	-0.248496993987976\\
1.14629258517034	-0.259660803112709\\
1.14312863030488	-0.264529058116232\\
1.13253623903881	-0.280561122244489\\
1.13026052104208	-0.283953974625914\\
1.12180582169426	-0.296593186372745\\
1.11422845691383	-0.307716868567398\\
1.11089266623192	-0.312625250501002\\
1.09982201900944	-0.328657314629258\\
1.09819639278557	-0.330979562311907\\
1.08861656130973	-0.344689378757515\\
1.08216432865731	-0.353767807594802\\
1.07722927765783	-0.360721442885771\\
1.06613226452906	-0.376100759386693\\
1.06566176267025	-0.376753507014028\\
1.05396831223254	-0.392785571142285\\
1.0501002004008	-0.398008471694326\\
1.04209910255601	-0.408817635270541\\
1.03406813627254	-0.419501460522291\\
1.03004848067485	-0.424849699398798\\
1.01803607214429	-0.440594941548499\\
1.01781719660689	-0.440881763527054\\
1.00545120195497	-0.456913827655311\\
1.00200400801603	-0.461320868074722\\
0.992904040742536	-0.472945891783567\\
0.985971943887775	-0.481679842319414\\
0.980172864193027	-0.488977955911824\\
0.969939879759519	-0.501684404901084\\
0.967257549946234	-0.50501002004008\\
0.954160927366089	-0.521042084168337\\
0.953907815631262	-0.521348942385413\\
0.940910424542356	-0.537074148296593\\
0.937875751503006	-0.540700813740435\\
0.927470125414012	-0.55310621242485\\
0.921843687374749	-0.559734593334919\\
0.91383898556294	-0.569138276553106\\
0.905811623246493	-0.578460782177182\\
0.900015688263405	-0.585170340681363\\
0.889779559118236	-0.596889279502509\\
0.885998639694972	-0.601202404809619\\
0.87374749498998	-0.61502941060904\\
0.871785963527117	-0.617234468937876\\
0.857715430861724	-0.632889952947604\\
0.857375494855175	-0.633266533066132\\
0.842776936889869	-0.649298597194389\\
0.841683366733467	-0.650488515867726\\
0.827976283664954	-0.665330661322646\\
0.825651302605211	-0.667825098240989\\
0.812966673195805	-0.681362725450902\\
0.809619238476954	-0.68490392857319\\
0.797744839820967	-0.697394789579158\\
0.793587174348698	-0.701731659444612\\
0.782307190244573	-0.713426853707415\\
0.777555110220441	-0.718314509342966\\
0.766649793471146	-0.729458917835671\\
0.761523046092185	-0.73465827955036\\
0.750768369901502	-0.745490981963928\\
0.745490981963928	-0.750768369901502\\
0.73465827955036	-0.761523046092184\\
0.729458917835672	-0.766649793471145\\
0.718314509342967	-0.777555110220441\\
0.713426853707415	-0.782307190244573\\
0.701731659444613	-0.793587174348697\\
0.697394789579159	-0.797744839820967\\
0.684903928573191	-0.809619238476954\\
0.681362725450902	-0.812966673195804\\
0.667825098240989	-0.82565130260521\\
0.665330661322646	-0.827976283664954\\
0.650488515867726	-0.841683366733467\\
0.649298597194389	-0.84277693688987\\
0.633266533066132	-0.857375494855174\\
0.632889952947605	-0.857715430861723\\
0.617234468937876	-0.871785963527117\\
0.615029410609041	-0.87374749498998\\
0.601202404809619	-0.885998639694972\\
0.596889279502508	-0.889779559118236\\
0.585170340681363	-0.900015688263405\\
0.578460782177181	-0.905811623246493\\
0.569138276553106	-0.91383898556294\\
0.559734593334918	-0.92184368737475\\
0.55310621242485	-0.927470125414012\\
0.540700813740435	-0.937875751503006\\
0.537074148296593	-0.940910424542356\\
0.521348942385413	-0.953907815631263\\
0.521042084168337	-0.95416092736609\\
0.50501002004008	-0.967257549946234\\
0.501684404901084	-0.969939879759519\\
0.488977955911824	-0.980172864193027\\
0.481679842319414	-0.985971943887776\\
0.472945891783567	-0.992904040742536\\
0.461320868074722	-1.00200400801603\\
0.456913827655311	-1.00545120195497\\
0.440881763527054	-1.01781719660689\\
0.440594941548498	-1.01803607214429\\
0.424849699398798	-1.03004848067485\\
0.419501460522291	-1.03406813627255\\
0.408817635270541	-1.04209910255601\\
0.398008471694325	-1.0501002004008\\
0.392785571142285	-1.05396831223254\\
0.376753507014028	-1.06566176267025\\
0.376100759386693	-1.06613226452906\\
0.360721442885771	-1.07722927765783\\
0.353767807594801	-1.08216432865731\\
0.344689378757515	-1.08861656130972\\
0.330979562311906	-1.09819639278557\\
0.328657314629258	-1.09982201900945\\
0.312625250501002	-1.11089266623192\\
0.307716868567398	-1.11422845691383\\
0.296593186372745	-1.12180582169426\\
0.283953974625914	-1.13026052104208\\
0.280561122244489	-1.13253623903881\\
0.264529058116232	-1.14312863030488\\
0.259660803112709	-1.14629258517034\\
0.248496993987976	-1.15357152140237\\
0.234809343276196	-1.1623246492986\\
0.232464929859719	-1.16382929072242\\
0.216432865731463	-1.17396545655276\\
0.209357201517583	-1.17835671342685\\
0.200400801603206	-1.18393810910758\\
0.18436873747495	-1.19373174742705\\
0.183276866398329	-1.19438877755511\\
0.168336673346693	-1.20342071850968\\
0.156506128689257	-1.21042084168337\\
0.152304609218437	-1.2129193682588\\
0.13627254509018	-1.22228788771598\\
0.129003296617097	-1.22645290581162\\
0.120240480961924	-1.2315015057197\\
0.104208416833667	-1.24054843329175\\
0.100714368128787	-1.24248496993988\\
0.0881763527054105	-1.24947612150027\\
0.0721442885771539	-1.25820926529107\\
0.0715695813918983	-1.25851703406814\\
0.0561122244488974	-1.26684923185225\\
0.0414726849145835	-1.27454909819639\\
0.0400801603206409	-1.27528661264128\\
0.0240480961923843	-1.28362498230843\\
0.0103380441932435	-1.29058116232465\\
0.00801603206412782	-1.2917681093382\\
-0.00801603206412826	-1.29980566904814\\
-0.0219430554655074	-1.30661322645291\\
-0.0240480961923848	-1.30765040111404\\
-0.0400801603206413	-1.31539175071737\\
-0.0555054938388768	-1.32264529058116\\
-0.0561122244488979	-1.32293302577473\\
-0.0721442885771544	-1.3303818507885\\
-0.0881763527054109	-1.33762980774476\\
-0.0905458980445204	-1.33867735470942\\
-0.104208416833667	-1.34477275047677\\
-0.120240480961924	-1.35173399332406\\
-0.127267448086717	-1.35470941883768\\
-0.13627254509018	-1.35855937216241\\
-0.152304609218437	-1.36523524615038\\
-0.165907613521522	-1.37074148296593\\
-0.168336673346694	-1.37173475319888\\
-0.18436873747495	-1.37812570799675\\
-0.200400801603207	-1.38432708256506\\
-0.206905683974875	-1.38677354709419\\
-0.216432865731463	-1.39039559253707\\
-0.232464929859719	-1.39631268255537\\
-0.248496993987976	-1.40204447569671\\
-0.250690013324401	-1.40280561122244\\
-0.264529058116232	-1.40766404416331\\
-0.280561122244489	-1.41310987459442\\
-0.296593186372745	-1.41837322806493\\
-0.298055089992474	-1.4188376753507\\
-0.312625250501002	-1.42352292752558\\
-0.328657314629258	-1.42849629502376\\
-0.344689378757515	-1.43328854112771\\
-0.350180124738909	-1.43486973947896\\
-0.360721442885771	-1.43794428572241\\
-0.376753507014028	-1.44244011560983\\
-0.392785571142285	-1.44675469643562\\
-0.408817635270541	-1.45088920268315\\
-0.408868709077129	-1.45090180360721\\
-0.424849699398798	-1.4548982625651\\
-0.440881763527054	-1.45872459997283\\
-0.456913827655311	-1.46236860897112\\
-0.472945891783567	-1.46583060818897\\
-0.478342874648239	-1.46693386773547\\
-0.488977955911824	-1.46913918171142\\
-0.50501002004008	-1.47227631523678\\
-0.521042084168337	-1.47522695329842\\
-0.537074148296593	-1.47799052016043\\
-0.55310621242485	-1.4805661763775\\
-0.569138276553106	-1.48295281504123\\
-0.569234343815057	-1.48296593186373\\
-0.585170340681363	-1.48517536135449\\
-0.601202404809619	-1.48720258701048\\
-0.617234468937876	-1.4890325162843\\
-0.633266533066132	-1.49066304635659\\
-0.649298597194389	-1.49209177925457\\
-0.665330661322646	-1.49331601461501\\
-0.681362725450902	-1.49433274173063\\
-0.697394789579158	-1.49513863085095\\
-0.713426853707415	-1.49573002370592\\
-0.729458917835671	-1.49610292321817\\
-0.745490981963928	-1.4962529823663\\
-0.761523046092184	-1.49617549215899\\
-0.777555110220441	-1.49586536867595\\
-0.793587174348697	-1.49531713912846\\
-0.809619238476954	-1.49452492688824\\
-0.82565130260521	-1.49348243542961\\
-0.841683366733467	-1.49218293112526\\
-0.857715430861723	-1.49061922483166\\
-0.87374749498998	-1.48878365219481\\
-0.889779559118236	-1.4866680526022\\
-0.905811623246493	-1.48426374670064\\
-0.91355685098122	-1.48296593186373\\
-0.92184368737475	-1.48157013714079\\
-0.937875751503006	-1.47857737254268\\
-0.953907815631263	-1.47526648644312\\
-0.969939879759519	-1.47162622253922\\
-0.985971943887776	-1.46764465392901\\
-0.988630527721551	-1.46693386773547\\
-1.00200400801603	-1.46332379681357\\
-1.01803607214429	-1.45863741265707\\
-1.03406813627255	-1.4535679301691\\
-1.04194631406362	-1.45090180360721\\
-1.0501002004008	-1.4481069472665\\
-1.06613226452906	-1.44223614224617\\
-1.08216432865731	-1.43593050221557\\
-1.08471625209384	-1.43486973947896\\
-1.09819639278557	-1.42917617946644\\
-1.11422845691383	-1.421945674948\\
-1.12074455217999	-1.4188376753507\\
-1.13026052104208	-1.41421392881124\\
-1.14629258517034	-1.40595435077796\\
-1.15209251422735	-1.40280561122244\\
-1.1623246492986	-1.39713185710627\\
-1.17835671342685	-1.38771867472358\\
-1.1798960501539	-1.38677354709419\\
-1.19438877755511	-1.37766019652171\\
-1.20481783866288	-1.37074148296593\\
-1.21042084168337	-1.36692705011549\\
-1.22645290581162	-1.35546566023473\\
-1.22746821035219	-1.35470941883768\\
-1.24248496993988	-1.34319798849057\\
-1.24811929248096	-1.33867735470942\\
-1.25851703406814	-1.33007317305076\\
-1.2671166818126	-1.32264529058116\\
-1.27454909819639	-1.31600962282144\\
-1.28465850444922	-1.30661322645291\\
-1.29058116232465	-1.30091004993293\\
-1.30091004993293	-1.29058116232465\\
-1.30661322645291	-1.28465850444922\\
-1.31600962282144	-1.27454909819639\\
-1.32264529058116	-1.2671166818126\\
-1.33007317305076	-1.25851703406814\\
-1.33867735470942	-1.24811929248096\\
-1.34319798849057	-1.24248496993988\\
-1.35470941883768	-1.22746821035219\\
-1.35546566023473	-1.22645290581162\\
-1.36692705011549	-1.21042084168337\\
-1.37074148296593	-1.20481783866288\\
-1.37766019652171	-1.19438877755511\\
-1.38677354709419	-1.1798960501539\\
-1.38771867472358	-1.17835671342685\\
-1.39713185710627	-1.1623246492986\\
-1.40280561122244	-1.15209251422735\\
-1.40595435077796	-1.14629258517034\\
-1.41421392881124	-1.13026052104208\\
-1.4188376753507	-1.12074455217999\\
-1.421945674948	-1.11422845691383\\
-1.42917617946644	-1.09819639278557\\
-1.43486973947896	-1.08471625209384\\
-1.43593050221557	-1.08216432865731\\
-1.44223614224617	-1.06613226452906\\
-1.4481069472665	-1.0501002004008\\
-1.45090180360721	-1.04194631406362\\
-1.4535679301691	-1.03406813627255\\
-1.45863741265707	-1.01803607214429\\
-1.46332379681357	-1.00200400801603\\
-1.46693386773547	-0.988630527721551\\
-1.46764465392901	-0.985971943887776\\
-1.47162622253922	-0.969939879759519\\
-1.47526648644312	-0.953907815631263\\
-1.47857737254268	-0.937875751503006\\
-1.48157013714079	-0.92184368737475\\
-1.48296593186373	-0.91355685098122\\
-1.48426374670064	-0.905811623246493\\
-1.4866680526022	-0.889779559118236\\
-1.48878365219481	-0.87374749498998\\
-1.49061922483166	-0.857715430861723\\
-1.49218293112526	-0.841683366733467\\
-1.49348243542961	-0.82565130260521\\
-1.49452492688824	-0.809619238476954\\
-1.49531713912846	-0.793587174348697\\
-1.49586536867595	-0.777555110220441\\
-1.49617549215899	-0.761523046092184\\
-1.4962529823663	-0.745490981963928\\
-1.49610292321817	-0.729458917835671\\
-1.49573002370592	-0.713426853707415\\
-1.49513863085095	-0.697394789579158\\
-1.49433274173063	-0.681362725450902\\
-1.49331601461501	-0.665330661322646\\
-1.49209177925457	-0.649298597194389\\
-1.49066304635659	-0.633266533066132\\
-1.4890325162843	-0.617234468937876\\
-1.48720258701048	-0.601202404809619\\
-1.48517536135449	-0.585170340681363\\
-1.48296593186373	-0.569234343815055\\
}--cycle;


\addplot[area legend,solid,fill=mycolor5,draw=black,forget plot]
table[row sep=crcr] {%
x	y\\
-1.27454909819639	-0.466618769091882\\
-1.27295272498976	-0.456913827655311\\
-1.27009810536008	-0.440881763527054\\
-1.26703228117326	-0.424849699398798\\
-1.26375687753508	-0.408817635270541\\
-1.26027319515493	-0.392785571142285\\
-1.25851703406814	-0.385166144757688\\
-1.25659742828046	-0.376753507014028\\
-1.25273210222657	-0.360721442885771\\
-1.24866441748673	-0.344689378757515\\
-1.24439458242074	-0.328657314629258\\
-1.24248496993988	-0.321814798203553\\
-1.23994408480234	-0.312625250501002\\
-1.23531074967436	-0.296593186372745\\
-1.23047849560665	-0.280561122244489\\
-1.22645290581162	-0.267735318731354\\
-1.22545525336684	-0.264529058116232\\
-1.22026963766297	-0.248496993987976\\
-1.21488649857533	-0.232464929859719\\
-1.21042084168337	-0.219637060641907\\
-1.20931432076376	-0.216432865731463\\
-1.20358502963993	-0.200400801603207\\
-1.19765782987516	-0.18436873747495\\
-1.19438877755511	-0.17580780366481\\
-1.19155691862157	-0.168336673346694\\
-1.18528794843221	-0.152304609218437\\
-1.17881893596678	-0.13627254509018\\
-1.17835671342685	-0.135158981634855\\
-1.17220606465721	-0.120240480961924\\
-1.16539690016691	-0.104208416833667\\
-1.1623246492986	-0.0971721154486229\\
-1.1584213389276	-0.0881763527054109\\
-1.15127247152226	-0.0721442885771544\\
-1.14629258517034	-0.0612750213456795\\
-1.14394073144626	-0.0561122244488979\\
-1.13645155209077	-0.0400801603206413\\
-1.13026052104208	-0.0271736951113896\\
-1.12876903818089	-0.0240480961923848\\
-1.12093788328182	-0.00801603206412826\\
-1.11422845691383	0.0053708048779547\\
-1.11290891558182	0.00801603206412782\\
-1.10473306512577	0.0240480961923843\\
-1.09819639278557	0.0365537814851361\\
-1.0963608942567	0.0400801603206409\\
-1.08783656450764	0.0561122244488974\\
-1.08216432865731	0.0665355788520805\\
-1.07912338170044	0.0721442885771539\\
-1.07024571242419	0.0881763527054105\\
-1.06613226452906	0.0954482527274049\\
-1.0611926541003	0.104208416833667\\
-1.0519556902447	0.120240480961924\\
-1.0501002004008	0.123400748854277\\
-1.04256283715695	0.13627254509018\\
-1.03406813627255	0.150465303703465\\
-1.03296975725443	0.152304609218437\\
-1.0232258757832	0.168336673346693\\
-1.01803607214429	0.176705582299162\\
-1.01329223212558	0.18436873747495\\
-1.00317149246144	0.200400801603206\\
-1.00200400801603	0.202220096964106\\
-0.992894599814432	0.216432865731463\\
-0.985971943887776	0.22702472474986\\
-0.982419303769028	0.232464929859719\\
-0.971763208128649	0.248496993987976\\
-0.969939879759519	0.251196951761693\\
-0.960939406949382	0.264529058116232\\
-0.953907815631263	0.274756804532005\\
-0.949917055490699	0.280561122244489\\
-0.93870416733807	0.296593186372745\\
-0.937875751503006	0.297760991405\\
-0.927325195503836	0.312625250501002\\
-0.92184368737475	0.320218462475229\\
-0.91574572928674	0.328657314629258\\
-0.905811623246493	0.34218052476774\\
-0.90396608206399	0.344689378757515\\
-0.892003757694275	0.360721442885771\\
-0.889779559118236	0.363660486686324\\
-0.879852286421367	0.376753507014028\\
-0.87374749498998	0.384683606859431\\
-0.867496080266672	0.392785571142285\\
-0.857715430861723	0.405275725637624\\
-0.854934394033778	0.408817635270541\\
-0.842169717944105	0.424849699398798\\
-0.841683366733467	0.425453442460483\\
-0.829215012001877	0.440881763527054\\
-0.82565130260521	0.445232167779849\\
-0.816047487147658	0.456913827655311\\
-0.809619238476954	0.464630990225297\\
-0.802665264709008	0.472945891783567\\
-0.793587174348697	0.483663762658664\\
-0.789066127873535	0.488977955911824\\
-0.777555110220441	0.502343514150096\\
-0.775247513321119	0.50501002004008\\
-0.761523046092184	0.520682492232649\\
-0.76120650196431	0.521042084168337\\
-0.746948330694268	0.537074148296593\\
-0.745490981963928	0.538695159984643\\
-0.73246043573346	0.55310621242485\\
-0.729458917835671	0.556390503194091\\
-0.717736985050553	0.569138276553106\\
-0.713426853707415	0.573777812948876\\
-0.702773705324161	0.585170340681363\\
-0.697394789579158	0.5908662972224\\
-0.687565912335408	0.601202404809619\\
-0.681362725450902	0.607664570261767\\
-0.672108496173315	0.617234468937876\\
-0.665330661322646	0.624180679215762\\
-0.656395905235999	0.633266533066132\\
-0.649298597194389	0.640422128961411\\
-0.640422128961412	0.649298597194389\\
-0.633266533066132	0.656395905235999\\
-0.624180679215762	0.665330661322646\\
-0.617234468937876	0.672108496173314\\
-0.607664570261767	0.681362725450902\\
-0.601202404809619	0.687565912335408\\
-0.5908662972224	0.697394789579159\\
-0.585170340681363	0.70277370532416\\
-0.573777812948876	0.713426853707415\\
-0.569138276553106	0.717736985050553\\
-0.55639050319409	0.729458917835672\\
-0.55310621242485	0.73246043573346\\
-0.538695159984642	0.745490981963928\\
-0.537074148296593	0.746948330694268\\
-0.521042084168337	0.76120650196431\\
-0.520682492232649	0.761523046092185\\
-0.50501002004008	0.775247513321119\\
-0.502343514150095	0.777555110220441\\
-0.488977955911824	0.789066127873536\\
-0.483663762658664	0.793587174348698\\
-0.472945891783567	0.802665264709008\\
-0.464630990225296	0.809619238476954\\
-0.456913827655311	0.816047487147659\\
-0.445232167779848	0.825651302605211\\
-0.440881763527054	0.829215012001877\\
-0.425453442460482	0.841683366733467\\
-0.424849699398798	0.842169717944105\\
-0.408817635270541	0.854934394033778\\
-0.405275725637624	0.857715430861724\\
-0.392785571142285	0.867496080266671\\
-0.38468360685943	0.87374749498998\\
-0.376753507014028	0.879852286421368\\
-0.363660486686325	0.889779559118236\\
-0.360721442885771	0.892003757694275\\
-0.344689378757515	0.903966082063991\\
-0.342180524767741	0.905811623246493\\
-0.328657314629258	0.91574572928674\\
-0.320218462475229	0.921843687374749\\
-0.312625250501002	0.927325195503836\\
-0.297760991405	0.937875751503006\\
-0.296593186372745	0.93870416733807\\
-0.280561122244489	0.949917055490699\\
-0.274756804532005	0.953907815631262\\
-0.264529058116232	0.960939406949383\\
-0.251196951761693	0.969939879759519\\
-0.248496993987976	0.971763208128649\\
-0.232464929859719	0.982419303769028\\
-0.227024724749861	0.985971943887775\\
-0.216432865731463	0.992894599814431\\
-0.202220096964107	1.00200400801603\\
-0.200400801603207	1.00317149246144\\
-0.18436873747495	1.01329223212558\\
-0.176705582299162	1.01803607214429\\
-0.168336673346694	1.0232258757832\\
-0.152304609218437	1.03296975725443\\
-0.150465303703465	1.03406813627254\\
-0.13627254509018	1.04256283715695\\
-0.123400748854278	1.0501002004008\\
-0.120240480961924	1.0519556902447\\
-0.104208416833667	1.0611926541003\\
-0.0954482527274053	1.06613226452906\\
-0.0881763527054109	1.07024571242419\\
-0.0721442885771544	1.07912338170044\\
-0.0665355788520814	1.08216432865731\\
-0.0561122244488979	1.08783656450764\\
-0.0400801603206413	1.0963608942567\\
-0.036553781485137	1.09819639278557\\
-0.0240480961923848	1.10473306512577\\
-0.00801603206412826	1.11290891558182\\
-0.00537080487795514	1.11422845691383\\
0.00801603206412782	1.12093788328182\\
0.0240480961923843	1.12876903818089\\
0.0271736951113897	1.13026052104208\\
0.0400801603206409	1.13645155209077\\
0.0561122244488974	1.14394073144626\\
0.0612750213456795	1.14629258517034\\
0.0721442885771539	1.15127247152226\\
0.0881763527054105	1.1584213389276\\
0.0971721154486234	1.1623246492986\\
0.104208416833667	1.16539690016691\\
0.120240480961924	1.17220606465721\\
0.135158981634855	1.17835671342685\\
0.13627254509018	1.17881893596678\\
0.152304609218437	1.18528794843221\\
0.168336673346693	1.19155691862157\\
0.17580780366481	1.19438877755511\\
0.18436873747495	1.19765782987516\\
0.200400801603206	1.20358502963993\\
0.216432865731463	1.20931432076376\\
0.219637060641906	1.21042084168337\\
0.232464929859719	1.21488649857533\\
0.248496993987976	1.22026963766297\\
0.264529058116232	1.22545525336684\\
0.267735318731354	1.22645290581162\\
0.280561122244489	1.23047849560665\\
0.296593186372745	1.23531074967436\\
0.312625250501002	1.23994408480234\\
0.321814798203554	1.24248496993988\\
0.328657314629258	1.24439458242074\\
0.344689378757515	1.24866441748673\\
0.360721442885771	1.25273210222657\\
0.376753507014028	1.25659742828046\\
0.385166144757689	1.25851703406814\\
0.392785571142285	1.26027319515493\\
0.408817635270541	1.26375687753508\\
0.424849699398798	1.26703228117326\\
0.440881763527054	1.27009810536008\\
0.456913827655311	1.27295272498976\\
0.466618769091882	1.27454909819639\\
0.472945891783567	1.27560125659468\\
0.488977955911824	1.27804265880056\\
0.50501002004008	1.28026317884145\\
0.521042084168337	1.28225997929246\\
0.537074148296593	1.2840298585041\\
0.55310621242485	1.28556923975273\\
0.569138276553106	1.28687415938753\\
0.585170340681363	1.28794025392489\\
0.601202404809619	1.28876274603689\\
0.617234468937876	1.2893364293756\\
0.633266533066132	1.28965565217012\\
0.649298597194389	1.28971429952802\\
0.665330661322646	1.28950577436658\\
0.681362725450902	1.28902297689371\\
0.697394789579159	1.28825828255116\\
0.713426853707415	1.28720351832573\\
0.729458917835672	1.28584993732643\\
0.745490981963928	1.2841881915169\\
0.761523046092185	1.2822083024837\\
0.777555110220441	1.27989963011091\\
0.793587174348698	1.27725083902098\\
0.808038563734407	1.27454909819639\\
0.809619238476954	1.27424990972415\\
0.825651302605211	1.27088318491219\\
0.841683366733467	1.26713525853387\\
0.857715430861724	1.26299118366574\\
0.87346321594382	1.25851703406814\\
0.87374749498998	1.25843492014508\\
0.889779559118236	1.25343834555487\\
0.905811623246493	1.24799009278663\\
0.920738111686163	1.24248496993988\\
0.921843687374749	1.24206894699635\\
0.937875751503006	1.23563013002986\\
0.953907815631262	1.22866964242502\\
0.958702530137318	1.22645290581162\\
0.969939879759519	1.22113252637798\\
0.985971943887775	1.21300192598344\\
0.990770928698353	1.21042084168337\\
1.00200400801603	1.20421531621666\\
1.01803607214429	1.19475759815478\\
1.0186307093375	1.19438877755511\\
1.03406813627254	1.18452317118343\\
1.04316038206823	1.17835671342685\\
1.0501002004008	1.17349593394672\\
1.06518717699268	1.1623246492986\\
1.06613226452906	1.16160018625485\\
1.08216432865731	1.14871094060872\\
1.08502966923079	1.14629258517034\\
1.09819639278557	1.13474703702453\\
1.10307920456994	1.13026052104208\\
1.11422845691383	1.11958845373758\\
1.11958845373758	1.11422845691383\\
1.13026052104208	1.10307920456994\\
1.13474703702453	1.09819639278557\\
1.14629258517034	1.08502966923079\\
1.14871094060872	1.08216432865731\\
1.16160018625485	1.06613226452906\\
1.1623246492986	1.06518717699268\\
1.17349593394672	1.0501002004008\\
1.17835671342685	1.04316038206823\\
1.18452317118343	1.03406813627254\\
1.19438877755511	1.0186307093375\\
1.19475759815478	1.01803607214429\\
1.20421531621666	1.00200400801603\\
1.21042084168337	0.990770928698354\\
1.21300192598344	0.985971943887775\\
1.22113252637798	0.969939879759519\\
1.22645290581162	0.958702530137318\\
1.22866964242502	0.953907815631262\\
1.23563013002986	0.937875751503006\\
1.24206894699635	0.921843687374749\\
1.24248496993988	0.920738111686164\\
1.24799009278663	0.905811623246493\\
1.25343834555487	0.889779559118236\\
1.25843492014508	0.87374749498998\\
1.25851703406814	0.87346321594382\\
1.26299118366574	0.857715430861724\\
1.26713525853387	0.841683366733467\\
1.27088318491219	0.825651302605211\\
1.27424990972415	0.809619238476954\\
1.27454909819639	0.808038563734407\\
1.27725083902098	0.793587174348698\\
1.27989963011091	0.777555110220441\\
1.2822083024837	0.761523046092185\\
1.2841881915169	0.745490981963928\\
1.28584993732643	0.729458917835672\\
1.28720351832573	0.713426853707415\\
1.28825828255116	0.697394789579159\\
1.28902297689371	0.681362725450902\\
1.28950577436658	0.665330661322646\\
1.28971429952802	0.649298597194389\\
1.28965565217012	0.633266533066132\\
1.2893364293756	0.617234468937876\\
1.28876274603689	0.601202404809619\\
1.28794025392489	0.585170340681363\\
1.28687415938753	0.569138276553106\\
1.28556923975273	0.55310621242485\\
1.2840298585041	0.537074148296593\\
1.28225997929246	0.521042084168337\\
1.28026317884145	0.50501002004008\\
1.27804265880056	0.488977955911824\\
1.27560125659468	0.472945891783567\\
1.27454909819639	0.466618769091882\\
1.27295272498976	0.456913827655311\\
1.27009810536008	0.440881763527054\\
1.26703228117326	0.424849699398798\\
1.26375687753508	0.408817635270541\\
1.26027319515493	0.392785571142285\\
1.25851703406814	0.385166144757688\\
1.25659742828046	0.376753507014028\\
1.25273210222657	0.360721442885771\\
1.24866441748673	0.344689378757515\\
1.24439458242074	0.328657314629258\\
1.24248496993988	0.321814798203553\\
1.23994408480234	0.312625250501002\\
1.23531074967436	0.296593186372745\\
1.23047849560665	0.280561122244489\\
1.22645290581162	0.267735318731354\\
1.22545525336684	0.264529058116232\\
1.22026963766297	0.248496993987976\\
1.21488649857533	0.232464929859719\\
1.21042084168337	0.219637060641906\\
1.20931432076376	0.216432865731463\\
1.20358502963993	0.200400801603206\\
1.19765782987516	0.18436873747495\\
1.19438877755511	0.17580780366481\\
1.19155691862157	0.168336673346693\\
1.18528794843221	0.152304609218437\\
1.17881893596678	0.13627254509018\\
1.17835671342685	0.135158981634855\\
1.17220606465721	0.120240480961924\\
1.16539690016691	0.104208416833667\\
1.1623246492986	0.0971721154486234\\
1.1584213389276	0.0881763527054105\\
1.15127247152226	0.0721442885771539\\
1.14629258517034	0.0612750213456795\\
1.14394073144626	0.0561122244488974\\
1.13645155209077	0.0400801603206409\\
1.13026052104208	0.0271736951113897\\
1.12876903818089	0.0240480961923843\\
1.12093788328182	0.00801603206412782\\
1.11422845691383	-0.00537080487795514\\
1.11290891558182	-0.00801603206412826\\
1.10473306512577	-0.0240480961923848\\
1.09819639278557	-0.036553781485137\\
1.0963608942567	-0.0400801603206413\\
1.08783656450764	-0.0561122244488979\\
1.08216432865731	-0.0665355788520818\\
1.07912338170044	-0.0721442885771544\\
1.07024571242419	-0.0881763527054109\\
1.06613226452906	-0.0954482527274053\\
1.0611926541003	-0.104208416833667\\
1.0519556902447	-0.120240480961924\\
1.0501002004008	-0.123400748854278\\
1.04256283715695	-0.13627254509018\\
1.03406813627254	-0.150465303703465\\
1.03296975725443	-0.152304609218437\\
1.0232258757832	-0.168336673346694\\
1.01803607214429	-0.176705582299162\\
1.01329223212558	-0.18436873747495\\
1.00317149246144	-0.200400801603207\\
1.00200400801603	-0.202220096964107\\
0.992894599814432	-0.216432865731463\\
0.985971943887775	-0.227024724749861\\
0.982419303769028	-0.232464929859719\\
0.971763208128649	-0.248496993987976\\
0.969939879759519	-0.251196951761693\\
0.960939406949383	-0.264529058116232\\
0.953907815631262	-0.274756804532005\\
0.949917055490699	-0.280561122244489\\
0.93870416733807	-0.296593186372745\\
0.937875751503006	-0.297760991405\\
0.927325195503836	-0.312625250501002\\
0.921843687374749	-0.320218462475229\\
0.91574572928674	-0.328657314629258\\
0.905811623246493	-0.342180524767741\\
0.903966082063991	-0.344689378757515\\
0.892003757694274	-0.360721442885771\\
0.889779559118236	-0.363660486686324\\
0.879852286421368	-0.376753507014028\\
0.87374749498998	-0.38468360685943\\
0.867496080266671	-0.392785571142285\\
0.857715430861724	-0.405275725637624\\
0.854934394033778	-0.408817635270541\\
0.842169717944105	-0.424849699398798\\
0.841683366733467	-0.425453442460482\\
0.829215012001877	-0.440881763527054\\
0.825651302605211	-0.445232167779848\\
0.816047487147659	-0.456913827655311\\
0.809619238476954	-0.464630990225296\\
0.802665264709009	-0.472945891783567\\
0.793587174348698	-0.483663762658664\\
0.789066127873536	-0.488977955911824\\
0.777555110220441	-0.502343514150095\\
0.775247513321119	-0.50501002004008\\
0.761523046092185	-0.520682492232649\\
0.76120650196431	-0.521042084168337\\
0.746948330694268	-0.537074148296593\\
0.745490981963928	-0.538695159984642\\
0.73246043573346	-0.55310621242485\\
0.729458917835672	-0.55639050319409\\
0.717736985050553	-0.569138276553106\\
0.713426853707415	-0.573777812948876\\
0.70277370532416	-0.585170340681363\\
0.697394789579159	-0.590866297222399\\
0.687565912335408	-0.601202404809619\\
0.681362725450902	-0.607664570261767\\
0.672108496173314	-0.617234468937876\\
0.665330661322646	-0.624180679215762\\
0.656395905235999	-0.633266533066132\\
0.649298597194389	-0.640422128961412\\
0.640422128961411	-0.649298597194389\\
0.633266533066132	-0.656395905235999\\
0.624180679215762	-0.665330661322646\\
0.617234468937876	-0.672108496173315\\
0.607664570261768	-0.681362725450902\\
0.601202404809619	-0.687565912335408\\
0.5908662972224	-0.697394789579158\\
0.585170340681363	-0.702773705324161\\
0.573777812948876	-0.713426853707415\\
0.569138276553106	-0.717736985050553\\
0.556390503194091	-0.729458917835671\\
0.55310621242485	-0.73246043573346\\
0.538695159984642	-0.745490981963928\\
0.537074148296593	-0.746948330694268\\
0.521042084168337	-0.76120650196431\\
0.520682492232649	-0.761523046092184\\
0.50501002004008	-0.775247513321118\\
0.502343514150095	-0.777555110220441\\
0.488977955911824	-0.789066127873535\\
0.483663762658664	-0.793587174348697\\
0.472945891783567	-0.802665264709008\\
0.464630990225297	-0.809619238476954\\
0.456913827655311	-0.816047487147658\\
0.445232167779849	-0.82565130260521\\
0.440881763527054	-0.829215012001877\\
0.425453442460483	-0.841683366733467\\
0.424849699398798	-0.842169717944105\\
0.408817635270541	-0.854934394033778\\
0.405275725637624	-0.857715430861723\\
0.392785571142285	-0.867496080266672\\
0.384683606859431	-0.87374749498998\\
0.376753507014028	-0.879852286421367\\
0.363660486686324	-0.889779559118236\\
0.360721442885771	-0.892003757694275\\
0.344689378757515	-0.90396608206399\\
0.342180524767739	-0.905811623246493\\
0.328657314629258	-0.91574572928674\\
0.320218462475229	-0.92184368737475\\
0.312625250501002	-0.927325195503836\\
0.297760991405	-0.937875751503006\\
0.296593186372745	-0.93870416733807\\
0.280561122244489	-0.949917055490699\\
0.274756804532005	-0.953907815631263\\
0.264529058116232	-0.960939406949382\\
0.251196951761693	-0.969939879759519\\
0.248496993987976	-0.971763208128649\\
0.232464929859719	-0.982419303769028\\
0.22702472474986	-0.985971943887776\\
0.216432865731463	-0.992894599814432\\
0.202220096964106	-1.00200400801603\\
0.200400801603206	-1.00317149246144\\
0.18436873747495	-1.01329223212558\\
0.176705582299162	-1.01803607214429\\
0.168336673346693	-1.0232258757832\\
0.152304609218437	-1.03296975725443\\
0.150465303703464	-1.03406813627255\\
0.13627254509018	-1.04256283715695\\
0.123400748854277	-1.0501002004008\\
0.120240480961924	-1.0519556902447\\
0.104208416833667	-1.0611926541003\\
0.0954482527274049	-1.06613226452906\\
0.0881763527054105	-1.07024571242419\\
0.0721442885771539	-1.07912338170044\\
0.0665355788520805	-1.08216432865731\\
0.0561122244488974	-1.08783656450764\\
0.0400801603206409	-1.0963608942567\\
0.0365537814851361	-1.09819639278557\\
0.0240480961923843	-1.10473306512577\\
0.00801603206412782	-1.11290891558182\\
0.0053708048779547	-1.11422845691383\\
-0.00801603206412826	-1.12093788328182\\
-0.0240480961923848	-1.12876903818089\\
-0.0271736951113896	-1.13026052104208\\
-0.0400801603206413	-1.13645155209077\\
-0.0561122244488979	-1.14394073144626\\
-0.0612750213456795	-1.14629258517034\\
-0.0721442885771544	-1.15127247152226\\
-0.0881763527054109	-1.1584213389276\\
-0.0971721154486234	-1.1623246492986\\
-0.104208416833667	-1.16539690016691\\
-0.120240480961924	-1.17220606465721\\
-0.135158981634855	-1.17835671342685\\
-0.13627254509018	-1.17881893596678\\
-0.152304609218437	-1.18528794843221\\
-0.168336673346694	-1.19155691862157\\
-0.17580780366481	-1.19438877755511\\
-0.18436873747495	-1.19765782987516\\
-0.200400801603207	-1.20358502963993\\
-0.216432865731463	-1.20931432076376\\
-0.219637060641907	-1.21042084168337\\
-0.232464929859719	-1.21488649857533\\
-0.248496993987976	-1.22026963766297\\
-0.264529058116232	-1.22545525336684\\
-0.267735318731354	-1.22645290581162\\
-0.280561122244489	-1.23047849560665\\
-0.296593186372745	-1.23531074967436\\
-0.312625250501002	-1.23994408480234\\
-0.321814798203553	-1.24248496993988\\
-0.328657314629258	-1.24439458242074\\
-0.344689378757515	-1.24866441748673\\
-0.360721442885771	-1.25273210222657\\
-0.376753507014028	-1.25659742828046\\
-0.385166144757689	-1.25851703406814\\
-0.392785571142285	-1.26027319515493\\
-0.408817635270541	-1.26375687753508\\
-0.424849699398798	-1.26703228117326\\
-0.440881763527054	-1.27009810536008\\
-0.456913827655311	-1.27295272498976\\
-0.466618769091882	-1.27454909819639\\
-0.472945891783567	-1.27560125659468\\
-0.488977955911824	-1.27804265880056\\
-0.50501002004008	-1.28026317884145\\
-0.521042084168337	-1.28225997929246\\
-0.537074148296593	-1.2840298585041\\
-0.55310621242485	-1.28556923975273\\
-0.569138276553106	-1.28687415938753\\
-0.585170340681363	-1.28794025392489\\
-0.601202404809619	-1.28876274603689\\
-0.617234468937876	-1.2893364293756\\
-0.633266533066132	-1.28965565217012\\
-0.649298597194389	-1.28971429952802\\
-0.665330661322646	-1.28950577436657\\
-0.681362725450902	-1.28902297689371\\
-0.697394789579158	-1.28825828255116\\
-0.713426853707415	-1.28720351832573\\
-0.729458917835671	-1.28584993732643\\
-0.745490981963928	-1.2841881915169\\
-0.761523046092184	-1.2822083024837\\
-0.777555110220441	-1.27989963011091\\
-0.793587174348697	-1.27725083902098\\
-0.808038563734407	-1.27454909819639\\
-0.809619238476954	-1.27424990972415\\
-0.82565130260521	-1.27088318491219\\
-0.841683366733467	-1.26713525853387\\
-0.857715430861723	-1.26299118366574\\
-0.873463215943819	-1.25851703406814\\
-0.87374749498998	-1.25843492014508\\
-0.889779559118236	-1.25343834555487\\
-0.905811623246493	-1.24799009278663\\
-0.920738111686163	-1.24248496993988\\
-0.92184368737475	-1.24206894699635\\
-0.937875751503006	-1.23563013002986\\
-0.953907815631263	-1.22866964242502\\
-0.958702530137318	-1.22645290581162\\
-0.969939879759519	-1.22113252637798\\
-0.985971943887776	-1.21300192598344\\
-0.990770928698353	-1.21042084168337\\
-1.00200400801603	-1.20421531621666\\
-1.01803607214429	-1.19475759815478\\
-1.0186307093375	-1.19438877755511\\
-1.03406813627255	-1.18452317118343\\
-1.04316038206823	-1.17835671342685\\
-1.0501002004008	-1.17349593394672\\
-1.06518717699268	-1.1623246492986\\
-1.06613226452906	-1.16160018625485\\
-1.08216432865731	-1.14871094060872\\
-1.08502966923079	-1.14629258517034\\
-1.09819639278557	-1.13474703702453\\
-1.10307920456994	-1.13026052104208\\
-1.11422845691383	-1.11958845373758\\
-1.11958845373758	-1.11422845691383\\
-1.13026052104208	-1.10307920456994\\
-1.13474703702453	-1.09819639278557\\
-1.14629258517034	-1.08502966923079\\
-1.14871094060872	-1.08216432865731\\
-1.16160018625485	-1.06613226452906\\
-1.1623246492986	-1.06518717699268\\
-1.17349593394672	-1.0501002004008\\
-1.17835671342685	-1.04316038206823\\
-1.18452317118343	-1.03406813627255\\
-1.19438877755511	-1.0186307093375\\
-1.19475759815478	-1.01803607214429\\
-1.20421531621666	-1.00200400801603\\
-1.21042084168337	-0.990770928698353\\
-1.21300192598344	-0.985971943887776\\
-1.22113252637798	-0.969939879759519\\
-1.22645290581162	-0.958702530137318\\
-1.22866964242502	-0.953907815631263\\
-1.23563013002986	-0.937875751503006\\
-1.24206894699635	-0.92184368737475\\
-1.24248496993988	-0.920738111686163\\
-1.24799009278663	-0.905811623246493\\
-1.25343834555487	-0.889779559118236\\
-1.25843492014508	-0.87374749498998\\
-1.25851703406814	-0.873463215943821\\
-1.26299118366574	-0.857715430861723\\
-1.26713525853387	-0.841683366733467\\
-1.27088318491219	-0.82565130260521\\
-1.27424990972415	-0.809619238476954\\
-1.27454909819639	-0.808038563734407\\
-1.27725083902098	-0.793587174348697\\
-1.27989963011091	-0.777555110220441\\
-1.2822083024837	-0.761523046092184\\
-1.2841881915169	-0.745490981963928\\
-1.28584993732643	-0.729458917835671\\
-1.28720351832573	-0.713426853707415\\
-1.28825828255116	-0.697394789579158\\
-1.28902297689371	-0.681362725450902\\
-1.28950577436658	-0.665330661322646\\
-1.28971429952802	-0.649298597194389\\
-1.28965565217012	-0.633266533066132\\
-1.2893364293756	-0.617234468937876\\
-1.28876274603689	-0.601202404809619\\
-1.28794025392489	-0.585170340681363\\
-1.28687415938753	-0.569138276553106\\
-1.28556923975273	-0.55310621242485\\
-1.2840298585041	-0.537074148296593\\
-1.28225997929246	-0.521042084168337\\
-1.28026317884145	-0.50501002004008\\
-1.27804265880056	-0.488977955911824\\
-1.27560125659468	-0.472945891783567\\
-1.27454909819639	-0.466618769091882\\
}--cycle;


\addplot[area legend,solid,fill=mycolor6,draw=black,forget plot]
table[row sep=crcr] {%
x	y\\
-1.09819639278557	-0.458105987806395\\
-1.09808091455203	-0.456913827655311\\
-1.09624988655646	-0.440881763527054\\
-1.09415730302145	-0.424849699398798\\
-1.09180817711469	-0.408817635270541\\
-1.08920703640823	-0.392785571142285\\
-1.08635794039673	-0.376753507014028\\
-1.08326449655531	-0.360721442885771\\
-1.08216432865731	-0.355448600916093\\
-1.07993395190822	-0.344689378757515\\
-1.07636930323629	-0.328657314629258\\
-1.07257143447101	-0.312625250501002\\
-1.0685424709325	-0.296593186372745\\
-1.06613226452906	-0.287532100349295\\
-1.06428870957631	-0.280561122244489\\
-1.05981546901216	-0.264529058116232\\
-1.05511871649359	-0.248496993987976\\
-1.05019922703141	-0.232464929859719\\
-1.0501002004008	-0.232156761170561\\
-1.04507307688802	-0.216432865731463\\
-1.03972891611292	-0.200400801603207\\
-1.03416633561998	-0.18436873747495\\
-1.03406813627255	-0.184096707551949\\
-1.0284042932585	-0.168336673346694\\
-1.0224272466568	-0.152304609218437\\
-1.01803607214429	-0.140939390644108\\
-1.01624004989504	-0.13627254509018\\
-1.00985351655426	-0.120240480961924\\
-1.00325248472249	-0.104208416833667\\
-1.00200400801603	-0.101271471803078\\
-0.996455883518593	-0.0881763527054109\\
-0.989450110635424	-0.0721442885771544\\
-0.985971943887776	-0.0644189979121709\\
-0.982242506612476	-0.0561122244488979\\
-0.974833661435717	-0.0400801603206413\\
-0.969939879759519	-0.029785173267226\\
-0.967219007884824	-0.0240480961923848\\
-0.959407510172255	-0.00801603206412826\\
-0.953907815631263	0.00297308262392777\\
-0.951388501346152	0.00801603206412782\\
-0.943173526998847	0.0240480961923843\\
-0.937875751503006	0.0341282246478093\\
-0.934751608998573	0.0400801603206409\\
-0.926131088797936	0.0561122244488974\\
-0.92184368737475	0.0638982675385622\\
-0.917306464036044	0.0721442885771539\\
-0.908277075974615	0.0881763527054105\\
-0.905811623246493	0.092458695957912\\
-0.899048701237884	0.104208416833667\\
-0.889779559118236	0.119946606167284\\
-0.889606450575453	0.120240480961924\\
-0.879971434486363	0.13627254509018\\
-0.87374749498998	0.146405562385132\\
-0.870121701542776	0.152304609218437\\
-0.86006522124617	0.168336673346693\\
-0.857715430861723	0.17201123034611\\
-0.849803674740785	0.18436873747495\\
-0.841683366733467	0.196797492050924\\
-0.839325362166738	0.200400801603206\\
-0.828639019075151	0.216432865731463\\
-0.82565130260521	0.220834360339976\\
-0.817739751860099	0.232464929859719\\
-0.809619238476954	0.244180887536617\\
-0.806620075253506	0.248496993987976\\
-0.795283684773293	0.264529058116232\\
-0.793587174348697	0.266889382045004\\
-0.783730413183033	0.280561122244489\\
-0.777555110220441	0.288979596256871\\
-0.771950811339603	0.296593186372745\\
-0.761523046092184	0.310523299816696\\
-0.759943442990866	0.312625250501002\\
-0.747710534149224	0.328657314629258\\
-0.745490981963928	0.331522567525352\\
-0.735246379197619	0.344689378757515\\
-0.729458917835671	0.352013752816698\\
-0.72254514330734	0.360721442885771\\
-0.713426853707415	0.372035050466633\\
-0.709603979216751	0.376753507014028\\
-0.697394789579158	0.391605974154438\\
-0.696419612532127	0.392785571142285\\
-0.682989683805159	0.408817635270541\\
-0.681362725450902	0.410734488282712\\
-0.669307930425406	0.424849699398798\\
-0.665330661322646	0.42944577747185\\
-0.655368801508418	0.440881763527054\\
-0.649298597194389	0.447762070693669\\
-0.64116740465962	0.456913827655311\\
-0.633266533066132	0.465697821782685\\
-0.626698348370895	0.472945891783567\\
-0.617234468937876	0.483266553454052\\
-0.611955721937785	0.488977955911824\\
-0.601202404809619	0.500480907385029\\
-0.596933073680442	0.50501002004008\\
-0.585170340681363	0.517352690728566\\
-0.581623387361593	0.521042084168337\\
-0.569138276553106	0.533892919311483\\
-0.566019056685135	0.537074148296593\\
-0.55310621242485	0.550111857748621\\
-0.550111857748621	0.55310621242485\\
-0.537074148296593	0.566019056685135\\
-0.533892919311483	0.569138276553106\\
-0.521042084168337	0.581623387361592\\
-0.517352690728566	0.585170340681363\\
-0.50501002004008	0.596933073680443\\
-0.500480907385029	0.601202404809619\\
-0.488977955911824	0.611955721937785\\
-0.483266553454052	0.617234468937876\\
-0.472945891783567	0.626698348370895\\
-0.465697821782685	0.633266533066132\\
-0.456913827655311	0.641167404659621\\
-0.447762070693669	0.649298597194389\\
-0.440881763527054	0.655368801508418\\
-0.42944577747185	0.665330661322646\\
-0.424849699398798	0.669307930425406\\
-0.410734488282711	0.681362725450902\\
-0.408817635270541	0.682989683805159\\
-0.392785571142285	0.696419612532128\\
-0.391605974154438	0.697394789579159\\
-0.376753507014028	0.709603979216751\\
-0.372035050466633	0.713426853707415\\
-0.360721442885771	0.72254514330734\\
-0.352013752816697	0.729458917835672\\
-0.344689378757515	0.735246379197618\\
-0.331522567525351	0.745490981963928\\
-0.328657314629258	0.747710534149224\\
-0.312625250501002	0.759943442990867\\
-0.310523299816695	0.761523046092185\\
-0.296593186372745	0.771950811339603\\
-0.28897959625687	0.777555110220441\\
-0.280561122244489	0.783730413183033\\
-0.266889382045003	0.793587174348698\\
-0.264529058116232	0.795283684773293\\
-0.248496993987976	0.806620075253505\\
-0.244180887536616	0.809619238476954\\
-0.232464929859719	0.817739751860098\\
-0.220834360339975	0.825651302605211\\
-0.216432865731463	0.82863901907515\\
-0.200400801603207	0.839325362166738\\
-0.196797492050924	0.841683366733467\\
-0.18436873747495	0.849803674740784\\
-0.172011230346109	0.857715430861724\\
-0.168336673346694	0.86006522124617\\
-0.152304609218437	0.870121701542775\\
-0.146405562385132	0.87374749498998\\
-0.13627254509018	0.879971434486362\\
-0.120240480961924	0.889606450575453\\
-0.119946606167284	0.889779559118236\\
-0.104208416833667	0.899048701237884\\
-0.0924586959579133	0.905811623246493\\
-0.0881763527054109	0.908277075974615\\
-0.0721442885771544	0.917306464036044\\
-0.0638982675385626	0.921843687374749\\
-0.0561122244488979	0.926131088797935\\
-0.0400801603206413	0.934751608998573\\
-0.0341282246478097	0.937875751503006\\
-0.0240480961923848	0.943173526998847\\
-0.00801603206412826	0.951388501346151\\
-0.00297308262392867	0.953907815631262\\
0.00801603206412782	0.959407510172254\\
0.0240480961923843	0.967219007884824\\
0.0297851732672251	0.969939879759519\\
0.0400801603206409	0.974833661435716\\
0.0561122244488974	0.982242506612475\\
0.06441899791217	0.985971943887775\\
0.0721442885771539	0.989450110635423\\
0.0881763527054105	0.996455883518593\\
0.101271471803077	1.00200400801603\\
0.104208416833667	1.00325248472249\\
0.120240480961924	1.00985351655426\\
0.13627254509018	1.01624004989504\\
0.140939390644107	1.01803607214429\\
0.152304609218437	1.0224272466568\\
0.168336673346693	1.0284042932585\\
0.184096707551948	1.03406813627254\\
0.18436873747495	1.03416633561997\\
0.200400801603206	1.03972891611292\\
0.216432865731463	1.04507307688802\\
0.23215676117056	1.0501002004008\\
0.232464929859719	1.05019922703141\\
0.248496993987976	1.05511871649359\\
0.264529058116232	1.05981546901216\\
0.280561122244489	1.06428870957631\\
0.287532100349293	1.06613226452906\\
0.296593186372745	1.0685424709325\\
0.312625250501002	1.07257143447101\\
0.328657314629258	1.07636930323629\\
0.344689378757515	1.07993395190822\\
0.355448600916092	1.08216432865731\\
0.360721442885771	1.08326449655531\\
0.376753507014028	1.08635794039673\\
0.392785571142285	1.08920703640823\\
0.408817635270541	1.09180817711469\\
0.424849699398798	1.09415730302145\\
0.440881763527054	1.09624988655646\\
0.456913827655311	1.09808091455203\\
0.458105987806388	1.09819639278557\\
0.472945891783567	1.09964460060043\\
0.488977955911824	1.10093425171172\\
0.50501002004008	1.10194353560832\\
0.521042084168337	1.10266558088415\\
0.537074148296593	1.10309293407056\\
0.55310621242485	1.10321753288994\\
0.569138276553106	1.10303067738788\\
0.585170340681363	1.10252299879944\\
0.601202404809619	1.10168442599249\\
0.617234468937876	1.10050414931681\\
0.633266533066132	1.09897058167272\\
0.639848463230357	1.09819639278557\\
0.649298597194389	1.09706659128809\\
0.665330661322646	1.09477746165347\\
0.681362725450902	1.09209141523596\\
0.697394789579159	1.08899271578527\\
0.713426853707415	1.08546456154148\\
0.726768823634913	1.08216432865731\\
0.729458917835672	1.0814842688884\\
0.745490981963928	1.07700876866326\\
0.761523046092185	1.07203864851023\\
0.777555110220441	1.06655124060643\\
0.77868709056565	1.06613226452906\\
0.793587174348698	1.06046798608601\\
0.809619238476954	1.05380220460014\\
0.817853603043534	1.0501002004008\\
0.825651302605211	1.0464875629191\\
0.841683366733467	1.03847870732543\\
0.849894653121231	1.03406813627254\\
0.857715430861724	1.02972391583877\\
0.87374749498998	1.02016751833616\\
0.877108783973042	1.01803607214429\\
0.889779559118236	1.00969594782289\\
0.900715087092005	1.00200400801603\\
0.905811623246493	0.998272317112233\\
0.921606112540625	0.985971943887775\\
0.921843687374749	0.98577876806274\\
0.937875751503006	0.972024688558594\\
0.940176924381384	0.969939879759519\\
0.953907815631262	0.956893218578027\\
0.956893218578027	0.953907815631262\\
0.969939879759519	0.940176924381384\\
0.972024688558594	0.937875751503006\\
0.98577876806274	0.921843687374749\\
0.985971943887775	0.921606112540625\\
0.998272317112233	0.905811623246493\\
1.00200400801603	0.900715087092005\\
1.00969594782289	0.889779559118236\\
1.01803607214429	0.877108783973042\\
1.02016751833616	0.87374749498998\\
1.02972391583877	0.857715430861724\\
1.03406813627254	0.849894653121231\\
1.03847870732543	0.841683366733467\\
1.0464875629191	0.825651302605211\\
1.0501002004008	0.817853603043535\\
1.05380220460014	0.809619238476954\\
1.06046798608601	0.793587174348698\\
1.06613226452906	0.778687090565648\\
1.06655124060643	0.777555110220441\\
1.07203864851023	0.761523046092185\\
1.07700876866326	0.745490981963928\\
1.0814842688884	0.729458917835672\\
1.08216432865731	0.726768823634913\\
1.08546456154148	0.713426853707415\\
1.08899271578527	0.697394789579159\\
1.09209141523596	0.681362725450902\\
1.09477746165347	0.665330661322646\\
1.09706659128809	0.649298597194389\\
1.09819639278557	0.639848463230357\\
1.09897058167272	0.633266533066132\\
1.10050414931681	0.617234468937876\\
1.10168442599249	0.601202404809619\\
1.10252299879944	0.585170340681363\\
1.10303067738788	0.569138276553106\\
1.10321753288994	0.55310621242485\\
1.10309293407056	0.537074148296593\\
1.10266558088415	0.521042084168337\\
1.10194353560832	0.50501002004008\\
1.10093425171172	0.488977955911824\\
1.09964460060043	0.472945891783567\\
1.09819639278557	0.458105987806388\\
1.09808091455203	0.456913827655311\\
1.09624988655646	0.440881763527054\\
1.09415730302145	0.424849699398798\\
1.09180817711469	0.408817635270541\\
1.08920703640823	0.392785571142285\\
1.08635794039673	0.376753507014028\\
1.08326449655531	0.360721442885771\\
1.08216432865731	0.35544860091609\\
1.07993395190822	0.344689378757515\\
1.07636930323629	0.328657314629258\\
1.07257143447101	0.312625250501002\\
1.0685424709325	0.296593186372745\\
1.06613226452906	0.287532100349293\\
1.06428870957631	0.280561122244489\\
1.05981546901216	0.264529058116232\\
1.05511871649359	0.248496993987976\\
1.05019922703141	0.232464929859719\\
1.0501002004008	0.23215676117056\\
1.04507307688802	0.216432865731463\\
1.03972891611292	0.200400801603206\\
1.03416633561997	0.18436873747495\\
1.03406813627254	0.184096707551948\\
1.0284042932585	0.168336673346693\\
1.0224272466568	0.152304609218437\\
1.01803607214429	0.140939390644107\\
1.01624004989504	0.13627254509018\\
1.00985351655426	0.120240480961924\\
1.00325248472249	0.104208416833667\\
1.00200400801603	0.101271471803078\\
0.996455883518593	0.0881763527054105\\
0.989450110635423	0.0721442885771539\\
0.985971943887775	0.06441899791217\\
0.982242506612475	0.0561122244488974\\
0.974833661435716	0.0400801603206409\\
0.969939879759519	0.0297851732672255\\
0.967219007884824	0.0240480961923843\\
0.959407510172254	0.00801603206412782\\
0.953907815631262	-0.00297308262392821\\
0.951388501346151	-0.00801603206412826\\
0.943173526998847	-0.0240480961923848\\
0.937875751503006	-0.0341282246478097\\
0.934751608998573	-0.0400801603206413\\
0.926131088797935	-0.0561122244488979\\
0.921843687374749	-0.0638982675385626\\
0.917306464036044	-0.0721442885771544\\
0.908277075974615	-0.0881763527054109\\
0.905811623246493	-0.0924586959579133\\
0.899048701237884	-0.104208416833667\\
0.889779559118236	-0.119946606167284\\
0.889606450575453	-0.120240480961924\\
0.879971434486362	-0.13627254509018\\
0.87374749498998	-0.146405562385132\\
0.870121701542775	-0.152304609218437\\
0.86006522124617	-0.168336673346694\\
0.857715430861724	-0.172011230346109\\
0.849803674740784	-0.18436873747495\\
0.841683366733467	-0.196797492050924\\
0.839325362166738	-0.200400801603207\\
0.82863901907515	-0.216432865731463\\
0.825651302605211	-0.220834360339975\\
0.817739751860098	-0.232464929859719\\
0.809619238476954	-0.244180887536617\\
0.806620075253506	-0.248496993987976\\
0.795283684773293	-0.264529058116232\\
0.793587174348698	-0.266889382045003\\
0.783730413183033	-0.280561122244489\\
0.777555110220441	-0.288979596256871\\
0.771950811339603	-0.296593186372745\\
0.761523046092185	-0.310523299816695\\
0.759943442990867	-0.312625250501002\\
0.747710534149224	-0.328657314629258\\
0.745490981963928	-0.331522567525351\\
0.735246379197618	-0.344689378757515\\
0.729458917835672	-0.352013752816697\\
0.72254514330734	-0.360721442885771\\
0.713426853707415	-0.372035050466633\\
0.709603979216751	-0.376753507014028\\
0.697394789579159	-0.391605974154438\\
0.696419612532128	-0.392785571142285\\
0.682989683805159	-0.408817635270541\\
0.681362725450902	-0.410734488282711\\
0.669307930425406	-0.424849699398798\\
0.665330661322646	-0.42944577747185\\
0.655368801508418	-0.440881763527054\\
0.649298597194389	-0.447762070693669\\
0.64116740465962	-0.456913827655311\\
0.633266533066132	-0.465697821782685\\
0.626698348370895	-0.472945891783567\\
0.617234468937876	-0.483266553454052\\
0.611955721937785	-0.488977955911824\\
0.601202404809619	-0.500480907385029\\
0.596933073680443	-0.50501002004008\\
0.585170340681363	-0.517352690728566\\
0.581623387361592	-0.521042084168337\\
0.569138276553106	-0.533892919311483\\
0.566019056685135	-0.537074148296593\\
0.55310621242485	-0.550111857748621\\
0.550111857748621	-0.55310621242485\\
0.537074148296593	-0.566019056685135\\
0.533892919311483	-0.569138276553106\\
0.521042084168337	-0.581623387361593\\
0.517352690728566	-0.585170340681363\\
0.50501002004008	-0.596933073680442\\
0.500480907385029	-0.601202404809619\\
0.488977955911824	-0.611955721937785\\
0.483266553454052	-0.617234468937876\\
0.472945891783567	-0.626698348370895\\
0.465697821782685	-0.633266533066132\\
0.456913827655311	-0.64116740465962\\
0.447762070693669	-0.649298597194389\\
0.440881763527054	-0.655368801508418\\
0.42944577747185	-0.665330661322646\\
0.424849699398798	-0.669307930425406\\
0.410734488282712	-0.681362725450902\\
0.408817635270541	-0.682989683805159\\
0.392785571142285	-0.696419612532127\\
0.391605974154438	-0.697394789579158\\
0.376753507014028	-0.709603979216751\\
0.372035050466633	-0.713426853707415\\
0.360721442885771	-0.72254514330734\\
0.352013752816698	-0.729458917835671\\
0.344689378757515	-0.735246379197618\\
0.331522567525352	-0.745490981963928\\
0.328657314629258	-0.747710534149224\\
0.312625250501002	-0.759943442990866\\
0.310523299816696	-0.761523046092184\\
0.296593186372745	-0.771950811339603\\
0.288979596256871	-0.777555110220441\\
0.280561122244489	-0.783730413183033\\
0.266889382045004	-0.793587174348697\\
0.264529058116232	-0.795283684773293\\
0.248496993987976	-0.806620075253506\\
0.244180887536617	-0.809619238476954\\
0.232464929859719	-0.817739751860099\\
0.220834360339976	-0.82565130260521\\
0.216432865731463	-0.828639019075151\\
0.200400801603206	-0.839325362166738\\
0.196797492050924	-0.841683366733467\\
0.18436873747495	-0.849803674740785\\
0.17201123034611	-0.857715430861723\\
0.168336673346693	-0.86006522124617\\
0.152304609218437	-0.870121701542776\\
0.146405562385132	-0.87374749498998\\
0.13627254509018	-0.879971434486362\\
0.120240480961924	-0.889606450575453\\
0.119946606167284	-0.889779559118236\\
0.104208416833667	-0.899048701237884\\
0.092458695957912	-0.905811623246493\\
0.0881763527054105	-0.908277075974615\\
0.0721442885771539	-0.917306464036044\\
0.0638982675385617	-0.92184368737475\\
0.0561122244488974	-0.926131088797936\\
0.0400801603206409	-0.934751608998573\\
0.0341282246478093	-0.937875751503006\\
0.0240480961923843	-0.943173526998847\\
0.00801603206412782	-0.951388501346152\\
0.00297308262392777	-0.953907815631263\\
-0.00801603206412826	-0.959407510172255\\
-0.0240480961923848	-0.967219007884824\\
-0.0297851732672264	-0.969939879759519\\
-0.0400801603206413	-0.974833661435717\\
-0.0561122244488979	-0.982242506612476\\
-0.0644189979121714	-0.985971943887776\\
-0.0721442885771544	-0.989450110635423\\
-0.0881763527054109	-0.996455883518593\\
-0.101271471803078	-1.00200400801603\\
-0.104208416833667	-1.00325248472249\\
-0.120240480961924	-1.00985351655426\\
-0.13627254509018	-1.01624004989504\\
-0.140939390644108	-1.01803607214429\\
-0.152304609218437	-1.0224272466568\\
-0.168336673346694	-1.0284042932585\\
-0.18409670755195	-1.03406813627255\\
-0.18436873747495	-1.03416633561997\\
-0.200400801603207	-1.03972891611292\\
-0.216432865731463	-1.04507307688802\\
-0.232156761170561	-1.0501002004008\\
-0.232464929859719	-1.05019922703141\\
-0.248496993987976	-1.05511871649359\\
-0.264529058116232	-1.05981546901216\\
-0.280561122244489	-1.06428870957631\\
-0.287532100349295	-1.06613226452906\\
-0.296593186372745	-1.0685424709325\\
-0.312625250501002	-1.07257143447101\\
-0.328657314629258	-1.07636930323629\\
-0.344689378757515	-1.07993395190822\\
-0.355448600916093	-1.08216432865731\\
-0.360721442885771	-1.08326449655531\\
-0.376753507014028	-1.08635794039673\\
-0.392785571142285	-1.08920703640823\\
-0.408817635270541	-1.09180817711469\\
-0.424849699398798	-1.09415730302145\\
-0.440881763527054	-1.09624988655646\\
-0.456913827655311	-1.09808091455203\\
-0.458105987806391	-1.09819639278557\\
-0.472945891783567	-1.09964460060043\\
-0.488977955911824	-1.10093425171172\\
-0.50501002004008	-1.10194353560832\\
-0.521042084168337	-1.10266558088415\\
-0.537074148296593	-1.10309293407056\\
-0.55310621242485	-1.10321753288994\\
-0.569138276553106	-1.10303067738788\\
-0.585170340681363	-1.10252299879944\\
-0.601202404809619	-1.10168442599249\\
-0.617234468937876	-1.10050414931681\\
-0.633266533066132	-1.09897058167272\\
-0.639848463230354	-1.09819639278557\\
-0.649298597194389	-1.09706659128809\\
-0.665330661322646	-1.09477746165347\\
-0.681362725450902	-1.09209141523596\\
-0.697394789579158	-1.08899271578527\\
-0.713426853707415	-1.08546456154148\\
-0.726768823634912	-1.08216432865731\\
-0.729458917835671	-1.0814842688884\\
-0.745490981963928	-1.07700876866326\\
-0.761523046092184	-1.07203864851023\\
-0.777555110220441	-1.06655124060643\\
-0.778687090565648	-1.06613226452906\\
-0.793587174348697	-1.06046798608601\\
-0.809619238476954	-1.05380220460014\\
-0.817853603043533	-1.0501002004008\\
-0.82565130260521	-1.0464875629191\\
-0.841683366733467	-1.03847870732543\\
-0.849894653121231	-1.03406813627255\\
-0.857715430861723	-1.02972391583877\\
-0.87374749498998	-1.02016751833616\\
-0.877108783973041	-1.01803607214429\\
-0.889779559118236	-1.00969594782289\\
-0.900715087092005	-1.00200400801603\\
-0.905811623246493	-0.998272317112232\\
-0.921606112540625	-0.985971943887776\\
-0.92184368737475	-0.985778768062739\\
-0.937875751503006	-0.972024688558594\\
-0.940176924381383	-0.969939879759519\\
-0.953907815631263	-0.956893218578027\\
-0.956893218578027	-0.953907815631263\\
-0.969939879759519	-0.940176924381383\\
-0.972024688558594	-0.937875751503006\\
-0.985778768062739	-0.92184368737475\\
-0.985971943887776	-0.921606112540625\\
-0.998272317112232	-0.905811623246493\\
-1.00200400801603	-0.900715087092005\\
-1.00969594782289	-0.889779559118236\\
-1.01803607214429	-0.877108783973041\\
-1.02016751833616	-0.87374749498998\\
-1.02972391583877	-0.857715430861723\\
-1.03406813627255	-0.84989465312123\\
-1.03847870732543	-0.841683366733467\\
-1.0464875629191	-0.82565130260521\\
-1.0501002004008	-0.817853603043533\\
-1.05380220460014	-0.809619238476954\\
-1.06046798608601	-0.793587174348697\\
-1.06613226452906	-0.778687090565648\\
-1.06655124060643	-0.777555110220441\\
-1.07203864851023	-0.761523046092184\\
-1.07700876866326	-0.745490981963928\\
-1.0814842688884	-0.729458917835671\\
-1.08216432865731	-0.726768823634912\\
-1.08546456154148	-0.713426853707415\\
-1.08899271578527	-0.697394789579158\\
-1.09209141523596	-0.681362725450902\\
-1.09477746165347	-0.665330661322646\\
-1.09706659128809	-0.649298597194389\\
-1.09819639278557	-0.639848463230354\\
-1.09897058167272	-0.633266533066132\\
-1.10050414931681	-0.617234468937876\\
-1.10168442599249	-0.601202404809619\\
-1.10252299879944	-0.585170340681363\\
-1.10303067738788	-0.569138276553106\\
-1.10321753288994	-0.55310621242485\\
-1.10309293407056	-0.537074148296593\\
-1.10266558088415	-0.521042084168337\\
-1.10194353560832	-0.50501002004008\\
-1.10093425171172	-0.488977955911824\\
-1.09964460060043	-0.472945891783567\\
-1.09819639278557	-0.458105987806395\\
}--cycle;


\addplot[area legend,solid,fill=mycolor7,draw=black,forget plot]
table[row sep=crcr] {%
x	y\\
-0.92184368737475	-0.416986049814432\\
-0.921339647680416	-0.408817635270541\\
-0.920010630598887	-0.392785571142285\\
-0.918356819288345	-0.376753507014028\\
-0.916387246039448	-0.360721442885771\\
-0.914110191542783	-0.344689378757515\\
-0.911533221746024	-0.328657314629258\\
-0.908663221752169	-0.312625250501002\\
-0.905811623246493	-0.298153363030755\\
-0.905504905577801	-0.296593186372745\\
-0.902051270200767	-0.280561122244489\\
-0.898322994652482	-0.264529058116232\\
-0.894324816055936	-0.248496993987976\\
-0.89006090829412	-0.232464929859719\\
-0.889779559118236	-0.231473372997378\\
-0.885518712181567	-0.216432865731463\\
-0.880718908002189	-0.200400801603207\\
-0.875665524260327	-0.18436873747495\\
-0.87374749498998	-0.178588350230955\\
-0.870349281857059	-0.168336673346694\\
-0.864779600693688	-0.152304609218437\\
-0.858965409904354	-0.13627254509018\\
-0.857715430861723	-0.132973470523187\\
-0.852893163390885	-0.120240480961924\\
-0.846577830613032	-0.104208416833667\\
-0.841683366733467	-0.0922442551460976\\
-0.84001908461802	-0.0881763527054109\\
-0.833209472667109	-0.0721442885771544\\
-0.826165572875657	-0.0561122244488979\\
-0.82565130260521	-0.0549816377184614\\
-0.818868577341389	-0.0400801603206413\\
-0.811338827125735	-0.0240480961923848\\
-0.809619238476954	-0.0205006307918331\\
-0.803560368912398	-0.00801603206412826\\
-0.7955476408492	0.00801603206412782\\
-0.793587174348697	0.0118258041904915\\
-0.787287081565235	0.0240480961923843\\
-0.778792757389597	0.0400801603206409\\
-0.777555110220441	0.0423534700487836\\
-0.770047970890559	0.0561122244488974\\
-0.761523046092184	0.0713382692943995\\
-0.761070566807339	0.0721442885771539\\
-0.751839310054558	0.0881763527054105\\
-0.745490981963928	0.0989361594819078\\
-0.742370479277433	0.104208416833667\\
-0.73265437061431	0.120240480961924\\
-0.729458917835671	0.125393594333595\\
-0.722688023489471	0.13627254509018\\
-0.713426853707415	0.150823303610486\\
-0.712480148433525	0.152304609218437\\
-0.702011880343739	0.168336673346693\\
-0.697394789579158	0.175260374883034\\
-0.691292630115563	0.18436873747495\\
-0.681362725450902	0.198889094256091\\
-0.680323659285709	0.200400801603206\\
-0.669086727678451	0.216432865731463\\
-0.665330661322646	0.22168957204435\\
-0.65758814407391	0.232464929859719\\
-0.649298597194389	0.243786427352779\\
-0.645828621121683	0.248496993987976\\
-0.633801937239361	0.264529058116232\\
-0.633266533066132	0.265230623268067\\
-0.621489939254334	0.280561122244489\\
-0.617234468937876	0.286006339135174\\
-0.608902547903239	0.296593186372745\\
-0.601202404809619	0.306215475620127\\
-0.596034592302318	0.312625250501002\\
-0.585170340681363	0.325884501213203\\
-0.582880297999845	0.328657314629258\\
-0.569431561804693	0.344689378757515\\
-0.569138276553106	0.345033899262017\\
-0.555670514206936	0.360721442885771\\
-0.55310621242485	0.363664980485939\\
-0.541602043627914	0.376753507014028\\
-0.537074148296593	0.381832629645389\\
-0.52721807104739	0.392785571142285\\
-0.521042084168337	0.399555811117788\\
-0.512509779684498	0.408817635270541\\
-0.50501002004008	0.416852178755406\\
-0.497467576372659	0.424849699398798\\
-0.488977955911824	0.433738156290938\\
-0.482081049626793	0.440881763527054\\
-0.472945891783567	0.450229011429202\\
-0.466338924141053	0.456913827655311\\
-0.456913827655311	0.466338924141053\\
-0.450229011429202	0.472945891783567\\
-0.440881763527054	0.482081049626793\\
-0.433738156290937	0.488977955911824\\
-0.424849699398798	0.497467576372659\\
-0.416852178755406	0.50501002004008\\
-0.408817635270541	0.512509779684498\\
-0.399555811117788	0.521042084168337\\
-0.392785571142285	0.52721807104739\\
-0.381832629645389	0.537074148296593\\
-0.376753507014028	0.541602043627914\\
-0.363664980485939	0.55310621242485\\
-0.360721442885771	0.555670514206936\\
-0.345033899262018	0.569138276553106\\
-0.344689378757515	0.569431561804693\\
-0.328657314629258	0.582880297999845\\
-0.325884501213203	0.585170340681363\\
-0.312625250501002	0.596034592302318\\
-0.306215475620127	0.601202404809619\\
-0.296593186372745	0.608902547903239\\
-0.286006339135175	0.617234468937876\\
-0.280561122244489	0.621489939254334\\
-0.265230623268067	0.633266533066132\\
-0.264529058116232	0.633801937239361\\
-0.248496993987976	0.645828621121683\\
-0.243786427352779	0.649298597194389\\
-0.232464929859719	0.65758814407391\\
-0.221689572044349	0.665330661322646\\
-0.216432865731463	0.669086727678451\\
-0.200400801603207	0.68032365928571\\
-0.198889094256091	0.681362725450902\\
-0.18436873747495	0.691292630115562\\
-0.175260374883033	0.697394789579159\\
-0.168336673346694	0.702011880343739\\
-0.152304609218437	0.712480148433525\\
-0.150823303610486	0.713426853707415\\
-0.13627254509018	0.72268802348947\\
-0.125393594333595	0.729458917835672\\
-0.120240480961924	0.73265437061431\\
-0.104208416833667	0.742370479277433\\
-0.0989361594819068	0.745490981963928\\
-0.0881763527054109	0.751839310054558\\
-0.0721442885771544	0.761070566807339\\
-0.0713382692943986	0.761523046092185\\
-0.0561122244488979	0.770047970890558\\
-0.0423534700487827	0.777555110220441\\
-0.0400801603206413	0.778792757389597\\
-0.0240480961923848	0.787287081565235\\
-0.0118258041904905	0.793587174348698\\
-0.00801603206412826	0.7955476408492\\
0.00801603206412782	0.803560368912398\\
0.0205006307918341	0.809619238476954\\
0.0240480961923843	0.811338827125735\\
0.0400801603206409	0.818868577341389\\
0.0549816377184628	0.825651302605211\\
0.0561122244488974	0.826165572875656\\
0.0721442885771539	0.833209472667109\\
0.0881763527054105	0.840019084618019\\
0.0922442551460981	0.841683366733467\\
0.104208416833667	0.846577830613032\\
0.120240480961924	0.852893163390885\\
0.132973470523188	0.857715430861724\\
0.13627254509018	0.858965409904354\\
0.152304609218437	0.864779600693688\\
0.168336673346693	0.870349281857059\\
0.178588350230956	0.87374749498998\\
0.18436873747495	0.875665524260326\\
0.200400801603206	0.880718908002189\\
0.216432865731463	0.885518712181567\\
0.231473372997376	0.889779559118236\\
0.232464929859719	0.89006090829412\\
0.248496993987976	0.894324816055937\\
0.264529058116232	0.898322994652482\\
0.280561122244489	0.902051270200767\\
0.296593186372745	0.9055049055778\\
0.298153363030755	0.905811623246493\\
0.312625250501002	0.908663221752169\\
0.328657314629258	0.911533221746024\\
0.344689378757515	0.914110191542783\\
0.360721442885771	0.916387246039447\\
0.376753507014028	0.918356819288345\\
0.392785571142285	0.920010630598886\\
0.408817635270541	0.921339647680415\\
0.416986049814428	0.921843687374749\\
0.424849699398798	0.92233041397223\\
0.440881763527054	0.922974241062274\\
0.456913827655311	0.923263635405709\\
0.472945891783567	0.923186687929635\\
0.488977955911824	0.922730544294115\\
0.50501002004008	0.921881348958793\\
0.505496363908077	0.921843687374749\\
0.521042084168337	0.920611795888819\\
0.537074148296593	0.91891216463631\\
0.55310621242485	0.916764743403446\\
0.569138276553106	0.914150108842143\\
0.585170340681363	0.911047400191296\\
0.601202404809619	0.907434220486349\\
0.607537010412693	0.905811623246493\\
0.617234468937876	0.903250719373583\\
0.633266533066132	0.898472010618524\\
0.649298597194389	0.893091063063893\\
0.658199076733441	0.889779559118236\\
0.665330661322646	0.887030548203673\\
0.681362725450902	0.880229441753713\\
0.695209896826667	0.87374749498998\\
0.697394789579159	0.872683471048268\\
0.713426853707415	0.86422428015813\\
0.724724341547229	0.857715430861724\\
0.729458917835672	0.85486490905248\\
0.745490981963928	0.844442551451622\\
0.749440834284703	0.841683366733467\\
0.761523046092185	0.832820235383348\\
0.770615834543674	0.825651302605211\\
0.777555110220441	0.819883836842907\\
0.789110356454346	0.809619238476954\\
0.793587174348698	0.805409453842751\\
0.805409453842752	0.793587174348698\\
0.809619238476954	0.789110356454347\\
0.819883836842907	0.777555110220441\\
0.825651302605211	0.770615834543674\\
0.832820235383348	0.761523046092185\\
0.841683366733467	0.749440834284703\\
0.844442551451622	0.745490981963928\\
0.85486490905248	0.729458917835672\\
0.857715430861724	0.724724341547229\\
0.86422428015813	0.713426853707415\\
0.872683471048268	0.697394789579159\\
0.87374749498998	0.695209896826667\\
0.880229441753713	0.681362725450902\\
0.887030548203673	0.665330661322646\\
0.889779559118236	0.658199076733441\\
0.893091063063893	0.649298597194389\\
0.898472010618524	0.633266533066132\\
0.903250719373583	0.617234468937876\\
0.905811623246493	0.607537010412693\\
0.907434220486349	0.601202404809619\\
0.911047400191296	0.585170340681363\\
0.914150108842143	0.569138276553106\\
0.916764743403446	0.55310621242485\\
0.91891216463631	0.537074148296593\\
0.92061179588882	0.521042084168337\\
0.921843687374749	0.505496363908077\\
0.921881348958793	0.50501002004008\\
0.922730544294115	0.488977955911824\\
0.923186687929635	0.472945891783567\\
0.923263635405709	0.456913827655311\\
0.922974241062275	0.440881763527054\\
0.92233041397223	0.424849699398798\\
0.921843687374749	0.416986049814428\\
0.921339647680415	0.408817635270541\\
0.920010630598886	0.392785571142285\\
0.918356819288345	0.376753507014028\\
0.916387246039448	0.360721442885771\\
0.914110191542783	0.344689378757515\\
0.911533221746024	0.328657314629258\\
0.908663221752169	0.312625250501002\\
0.905811623246493	0.298153363030755\\
0.9055049055778	0.296593186372745\\
0.902051270200767	0.280561122244489\\
0.898322994652482	0.264529058116232\\
0.894324816055937	0.248496993987976\\
0.89006090829412	0.232464929859719\\
0.889779559118236	0.231473372997376\\
0.885518712181567	0.216432865731463\\
0.880718908002189	0.200400801603206\\
0.875665524260326	0.18436873747495\\
0.87374749498998	0.178588350230956\\
0.870349281857059	0.168336673346693\\
0.864779600693688	0.152304609218437\\
0.858965409904354	0.13627254509018\\
0.857715430861724	0.132973470523188\\
0.852893163390885	0.120240480961924\\
0.846577830613033	0.104208416833667\\
0.841683366733467	0.0922442551460981\\
0.840019084618019	0.0881763527054105\\
0.833209472667109	0.0721442885771539\\
0.826165572875656	0.0561122244488974\\
0.825651302605211	0.0549816377184628\\
0.818868577341389	0.0400801603206409\\
0.811338827125735	0.0240480961923843\\
0.809619238476954	0.0205006307918341\\
0.803560368912398	0.00801603206412782\\
0.7955476408492	-0.00801603206412826\\
0.793587174348698	-0.011825804190491\\
0.787287081565235	-0.0240480961923848\\
0.778792757389597	-0.0400801603206413\\
0.777555110220441	-0.0423534700487832\\
0.770047970890558	-0.0561122244488979\\
0.761523046092185	-0.0713382692943986\\
0.761070566807339	-0.0721442885771544\\
0.751839310054558	-0.0881763527054109\\
0.745490981963928	-0.0989361594819068\\
0.742370479277433	-0.104208416833667\\
0.73265437061431	-0.120240480961924\\
0.729458917835672	-0.125393594333595\\
0.72268802348947	-0.13627254509018\\
0.713426853707415	-0.150823303610486\\
0.712480148433525	-0.152304609218437\\
0.702011880343739	-0.168336673346694\\
0.697394789579159	-0.175260374883033\\
0.691292630115562	-0.18436873747495\\
0.681362725450902	-0.198889094256091\\
0.68032365928571	-0.200400801603207\\
0.669086727678451	-0.216432865731463\\
0.665330661322646	-0.221689572044349\\
0.65758814407391	-0.232464929859719\\
0.649298597194389	-0.243786427352779\\
0.645828621121683	-0.248496993987976\\
0.633801937239361	-0.264529058116232\\
0.633266533066132	-0.265230623268067\\
0.621489939254334	-0.280561122244489\\
0.617234468937876	-0.286006339135175\\
0.608902547903239	-0.296593186372745\\
0.601202404809619	-0.306215475620127\\
0.596034592302318	-0.312625250501002\\
0.585170340681363	-0.325884501213203\\
0.582880297999845	-0.328657314629258\\
0.569431561804693	-0.344689378757515\\
0.569138276553106	-0.345033899262018\\
0.555670514206936	-0.360721442885771\\
0.55310621242485	-0.363664980485939\\
0.541602043627914	-0.376753507014028\\
0.537074148296593	-0.381832629645389\\
0.52721807104739	-0.392785571142285\\
0.521042084168337	-0.399555811117788\\
0.512509779684498	-0.408817635270541\\
0.50501002004008	-0.416852178755406\\
0.497467576372659	-0.424849699398798\\
0.488977955911824	-0.433738156290937\\
0.482081049626793	-0.440881763527054\\
0.472945891783567	-0.450229011429202\\
0.466338924141053	-0.456913827655311\\
0.456913827655311	-0.466338924141053\\
0.450229011429202	-0.472945891783567\\
0.440881763527054	-0.482081049626793\\
0.433738156290938	-0.488977955911824\\
0.424849699398798	-0.497467576372659\\
0.416852178755406	-0.50501002004008\\
0.408817635270541	-0.512509779684498\\
0.399555811117788	-0.521042084168337\\
0.392785571142285	-0.52721807104739\\
0.381832629645389	-0.537074148296593\\
0.376753507014028	-0.541602043627914\\
0.363664980485939	-0.55310621242485\\
0.360721442885771	-0.555670514206936\\
0.345033899262017	-0.569138276553106\\
0.344689378757515	-0.569431561804693\\
0.328657314629258	-0.582880297999845\\
0.325884501213203	-0.585170340681363\\
0.312625250501002	-0.596034592302318\\
0.306215475620127	-0.601202404809619\\
0.296593186372745	-0.608902547903239\\
0.286006339135174	-0.617234468937876\\
0.280561122244489	-0.621489939254334\\
0.265230623268067	-0.633266533066132\\
0.264529058116232	-0.633801937239361\\
0.248496993987976	-0.645828621121683\\
0.243786427352779	-0.649298597194389\\
0.232464929859719	-0.65758814407391\\
0.22168957204435	-0.665330661322646\\
0.216432865731463	-0.669086727678451\\
0.200400801603206	-0.680323659285709\\
0.198889094256091	-0.681362725450902\\
0.18436873747495	-0.691292630115563\\
0.175260374883034	-0.697394789579158\\
0.168336673346693	-0.702011880343739\\
0.152304609218437	-0.712480148433525\\
0.150823303610486	-0.713426853707415\\
0.13627254509018	-0.722688023489471\\
0.125393594333595	-0.729458917835671\\
0.120240480961924	-0.73265437061431\\
0.104208416833667	-0.742370479277433\\
0.0989361594819078	-0.745490981963928\\
0.0881763527054105	-0.751839310054558\\
0.0721442885771539	-0.761070566807338\\
0.071338269294399	-0.761523046092184\\
0.0561122244488974	-0.770047970890559\\
0.0423534700487836	-0.777555110220441\\
0.0400801603206409	-0.778792757389597\\
0.0240480961923843	-0.787287081565235\\
0.0118258041904915	-0.793587174348697\\
0.00801603206412782	-0.7955476408492\\
-0.00801603206412826	-0.803560368912398\\
-0.0205006307918331	-0.809619238476954\\
-0.0240480961923848	-0.811338827125735\\
-0.0400801603206413	-0.818868577341389\\
-0.0549816377184614	-0.82565130260521\\
-0.0561122244488979	-0.826165572875657\\
-0.0721442885771544	-0.833209472667109\\
-0.0881763527054109	-0.84001908461802\\
-0.0922442551460976	-0.841683366733467\\
-0.104208416833667	-0.846577830613032\\
-0.120240480961924	-0.852893163390885\\
-0.132973470523187	-0.857715430861723\\
-0.13627254509018	-0.858965409904354\\
-0.152304609218437	-0.864779600693688\\
-0.168336673346694	-0.870349281857059\\
-0.178588350230955	-0.87374749498998\\
-0.18436873747495	-0.875665524260327\\
-0.200400801603207	-0.880718908002189\\
-0.216432865731463	-0.885518712181567\\
-0.231473372997378	-0.889779559118236\\
-0.232464929859719	-0.89006090829412\\
-0.248496993987976	-0.894324816055936\\
-0.264529058116232	-0.898322994652482\\
-0.280561122244489	-0.902051270200767\\
-0.296593186372745	-0.905504905577801\\
-0.298153363030755	-0.905811623246493\\
-0.312625250501002	-0.908663221752169\\
-0.328657314629258	-0.911533221746024\\
-0.344689378757515	-0.914110191542783\\
-0.360721442885771	-0.916387246039448\\
-0.376753507014028	-0.918356819288345\\
-0.392785571142285	-0.920010630598886\\
-0.408817635270541	-0.921339647680416\\
-0.416986049814432	-0.92184368737475\\
-0.424849699398798	-0.92233041397223\\
-0.440881763527054	-0.922974241062274\\
-0.456913827655311	-0.923263635405709\\
-0.472945891783567	-0.923186687929636\\
-0.488977955911824	-0.922730544294115\\
-0.50501002004008	-0.921881348958793\\
-0.505496363908071	-0.92184368737475\\
-0.521042084168337	-0.92061179588882\\
-0.537074148296593	-0.91891216463631\\
-0.55310621242485	-0.916764743403446\\
-0.569138276553106	-0.914150108842143\\
-0.585170340681363	-0.911047400191295\\
-0.601202404809619	-0.907434220486349\\
-0.607537010412692	-0.905811623246493\\
-0.617234468937876	-0.903250719373583\\
-0.633266533066132	-0.898472010618524\\
-0.649298597194389	-0.893091063063893\\
-0.658199076733439	-0.889779559118236\\
-0.665330661322646	-0.887030548203673\\
-0.681362725450902	-0.880229441753714\\
-0.69520989682667	-0.87374749498998\\
-0.697394789579158	-0.872683471048268\\
-0.713426853707415	-0.864224280158129\\
-0.72472434154723	-0.857715430861723\\
-0.729458917835671	-0.85486490905248\\
-0.745490981963928	-0.844442551451622\\
-0.749440834284703	-0.841683366733467\\
-0.761523046092184	-0.832820235383349\\
-0.770615834543675	-0.82565130260521\\
-0.777555110220441	-0.819883836842908\\
-0.789110356454347	-0.809619238476954\\
-0.793587174348697	-0.805409453842752\\
-0.805409453842752	-0.793587174348697\\
-0.809619238476954	-0.789110356454347\\
-0.819883836842908	-0.777555110220441\\
-0.82565130260521	-0.770615834543674\\
-0.832820235383348	-0.761523046092184\\
-0.841683366733467	-0.749440834284703\\
-0.844442551451622	-0.745490981963928\\
-0.85486490905248	-0.729458917835671\\
-0.857715430861723	-0.72472434154723\\
-0.86422428015813	-0.713426853707415\\
-0.872683471048268	-0.697394789579158\\
-0.87374749498998	-0.69520989682667\\
-0.880229441753714	-0.681362725450902\\
-0.887030548203673	-0.665330661322646\\
-0.889779559118236	-0.658199076733439\\
-0.893091063063893	-0.649298597194389\\
-0.898472010618524	-0.633266533066132\\
-0.903250719373583	-0.617234468937876\\
-0.905811623246493	-0.607537010412692\\
-0.907434220486349	-0.601202404809619\\
-0.911047400191296	-0.585170340681363\\
-0.914150108842143	-0.569138276553106\\
-0.916764743403446	-0.55310621242485\\
-0.91891216463631	-0.537074148296593\\
-0.92061179588882	-0.521042084168337\\
-0.92184368737475	-0.505496363908071\\
-0.921881348958793	-0.50501002004008\\
-0.922730544294115	-0.488977955911824\\
-0.923186687929636	-0.472945891783567\\
-0.923263635405709	-0.456913827655311\\
-0.922974241062274	-0.440881763527054\\
-0.92233041397223	-0.424849699398798\\
-0.92184368737475	-0.416986049814432\\
}--cycle;


\addplot[area legend,solid,fill=mycolor8,draw=black,forget plot]
table[row sep=crcr] {%
x	y\\
-0.729458917835671	-0.269990729016995\\
-0.728608903396484	-0.264529058116232\\
-0.725728466131442	-0.248496993987976\\
-0.722493950193577	-0.232464929859719\\
-0.718915994936335	-0.216432865731463\\
-0.715004246748003	-0.200400801603207\\
-0.713426853707415	-0.194466496838941\\
-0.710733887635478	-0.18436873747495\\
-0.706126038379098	-0.168336673346694\\
-0.701208832942026	-0.152304609218437\\
-0.697394789579158	-0.140616453694156\\
-0.695971866346518	-0.13627254509018\\
-0.690391674265861	-0.120240480961924\\
-0.68452178183024	-0.104208416833667\\
-0.681362725450902	-0.0960085969224055\\
-0.678332132446254	-0.0881763527054109\\
-0.67182619198489	-0.0721442885771544\\
-0.665330661322646	-0.0567900554454374\\
-0.66504251073098	-0.0561122244488979\\
-0.657912432516187	-0.0400801603206413\\
-0.65051714751943	-0.0240480961923848\\
-0.649298597194389	-0.0215093544722066\\
-0.642787047187999	-0.00801603206412826\\
-0.634785530765667	0.00801603206412782\\
-0.633266533066132	0.0109527098428172\\
-0.62645283447006	0.0240480961923843\\
-0.617851892794969	0.0400801603206409\\
-0.617234468937876	0.0411936396557999\\
-0.608908861938725	0.0561122244488974\\
-0.601202404809619	0.0695382561562252\\
-0.599696174091843	0.0721442885771539\\
-0.590150461479603	0.0881763527054105\\
-0.585170340681363	0.0963180305242836\\
-0.58030792515261	0.104208416833667\\
-0.570169205257474	0.120240480961924\\
-0.569138276553106	0.12182810857171\\
-0.559683926397824	0.13627254509018\\
-0.55310621242485	0.146085950191334\\
-0.548902602517359	0.152304609218437\\
-0.537809939200304	0.168336673346693\\
-0.537074148296593	0.1693753417743\\
-0.526357199831222	0.18436873747495\\
-0.521042084168337	0.191646302447054\\
-0.514587433616802	0.200400801603206\\
-0.50501002004008	0.213122240134114\\
-0.502492575352915	0.216432865731463\\
-0.490049038473626	0.232464929859719\\
-0.488977955911824	0.233817422274425\\
-0.477229992186369	0.248496993987976\\
-0.472945891783567	0.253750044496973\\
-0.46405770415209	0.264529058116232\\
-0.456913827655311	0.273035825651279\\
-0.450520805470499	0.280561122244489\\
-0.440881763527054	0.291708587488053\\
-0.436606832638368	0.296593186372745\\
-0.424849699398798	0.309799531610842\\
-0.422302155581194	0.312625250501002\\
-0.408817635270541	0.327337444040233\\
-0.407591898736459	0.328657314629258\\
-0.392785571142285	0.344348880453285\\
-0.392459854497961	0.344689378757515\\
-0.376886131836701	0.360721442885771\\
-0.376753507014028	0.360855964476592\\
-0.360855964476592	0.376753507014028\\
-0.360721442885771	0.376886131836701\\
-0.344689378757515	0.392459854497961\\
-0.344348880453285	0.392785571142285\\
-0.328657314629258	0.407591898736458\\
-0.327337444040234	0.408817635270541\\
-0.312625250501002	0.422302155581194\\
-0.309799531610842	0.424849699398798\\
-0.296593186372745	0.436606832638368\\
-0.291708587488053	0.440881763527054\\
-0.280561122244489	0.450520805470499\\
-0.273035825651279	0.456913827655311\\
-0.264529058116232	0.46405770415209\\
-0.253750044496972	0.472945891783567\\
-0.248496993987976	0.477229992186369\\
-0.233817422274425	0.488977955911824\\
-0.232464929859719	0.490049038473626\\
-0.216432865731463	0.502492575352914\\
-0.213122240134114	0.50501002004008\\
-0.200400801603207	0.514587433616802\\
-0.191646302447055	0.521042084168337\\
-0.18436873747495	0.526357199831221\\
-0.1693753417743	0.537074148296593\\
-0.168336673346694	0.537809939200304\\
-0.152304609218437	0.548902602517359\\
-0.146085950191334	0.55310621242485\\
-0.13627254509018	0.559683926397823\\
-0.121828108571709	0.569138276553106\\
-0.120240480961924	0.570169205257473\\
-0.104208416833667	0.58030792515261\\
-0.0963180305242831	0.585170340681363\\
-0.0881763527054109	0.590150461479602\\
-0.0721442885771544	0.599696174091843\\
-0.0695382561562248	0.601202404809619\\
-0.0561122244488979	0.608908861938725\\
-0.0411936396557998	0.617234468937876\\
-0.0400801603206413	0.617851892794968\\
-0.0240480961923848	0.62645283447006\\
-0.0109527098428168	0.633266533066132\\
-0.00801603206412826	0.634785530765666\\
0.00801603206412782	0.642787047187999\\
0.0215093544722066	0.649298597194389\\
0.0240480961923843	0.65051714751943\\
0.0400801603206409	0.657912432516187\\
0.0561122244488974	0.66504251073098\\
0.0567900554454374	0.665330661322646\\
0.0721442885771539	0.671826191984889\\
0.0881763527054105	0.678332132446255\\
0.0960085969224064	0.681362725450902\\
0.104208416833667	0.68452178183024\\
0.120240480961924	0.690391674265861\\
0.13627254509018	0.695971866346518\\
0.140616453694158	0.697394789579159\\
0.152304609218437	0.701208832942025\\
0.168336673346693	0.706126038379098\\
0.18436873747495	0.710733887635478\\
0.194466496838943	0.713426853707415\\
0.200400801603206	0.715004246748002\\
0.216432865731463	0.718915994936334\\
0.232464929859719	0.722493950193577\\
0.248496993987976	0.725728466131442\\
0.264529058116232	0.728608903396483\\
0.269990729016998	0.729458917835672\\
0.280561122244489	0.73109940926959\\
0.296593186372745	0.733202433816114\\
0.312625250501002	0.734916610868685\\
0.328657314629258	0.736227477754899\\
0.344689378757515	0.737119278633694\\
0.360721442885771	0.737574873909199\\
0.376753507014028	0.737575640878828\\
0.392785571142285	0.737101364710597\\
0.408817635270541	0.736130118733727\\
0.424849699398798	0.734638132900445\\
0.440881763527054	0.732599649133125\\
0.456913827655311	0.729986762106782\\
0.459583598287602	0.729458917835672\\
0.472945891783567	0.726709455491159\\
0.488977955911824	0.722764126967297\\
0.50501002004008	0.718124216345629\\
0.519055152176373	0.713426853707415\\
0.521042084168337	0.712730894650601\\
0.537074148296593	0.706408747703523\\
0.55310621242485	0.699224080847994\\
0.556790786467564	0.697394789579159\\
0.569138276553106	0.690928737524885\\
0.585170340681363	0.681551276387957\\
0.585466178956859	0.681362725450902\\
0.601202404809619	0.670716407499915\\
0.608454003374408	0.665330661322646\\
0.617234468937876	0.658372217382182\\
0.627754465375622	0.649298597194389\\
0.633266533066132	0.64419614391258\\
0.64419614391258	0.633266533066132\\
0.649298597194389	0.627754465375622\\
0.658372217382182	0.617234468937876\\
0.665330661322646	0.608454003374408\\
0.670716407499915	0.601202404809619\\
0.681362725450902	0.585466178956859\\
0.681551276387957	0.585170340681363\\
0.690928737524885	0.569138276553106\\
0.697394789579159	0.556790786467564\\
0.699224080847994	0.55310621242485\\
0.706408747703523	0.537074148296593\\
0.712730894650601	0.521042084168337\\
0.713426853707415	0.519055152176373\\
0.718124216345629	0.50501002004008\\
0.722764126967297	0.488977955911824\\
0.726709455491159	0.472945891783567\\
0.729458917835672	0.459583598287602\\
0.729986762106782	0.456913827655311\\
0.732599649133125	0.440881763527054\\
0.734638132900445	0.424849699398798\\
0.736130118733727	0.408817635270541\\
0.737101364710597	0.392785571142285\\
0.737575640878828	0.376753507014028\\
0.737574873909199	0.360721442885771\\
0.737119278633695	0.344689378757515\\
0.736227477754899	0.328657314629258\\
0.734916610868685	0.312625250501002\\
0.733202433816115	0.296593186372745\\
0.73109940926959	0.280561122244489\\
0.729458917835672	0.269990729016997\\
0.728608903396484	0.264529058116232\\
0.725728466131442	0.248496993987976\\
0.722493950193577	0.232464929859719\\
0.718915994936334	0.216432865731463\\
0.715004246748002	0.200400801603206\\
0.713426853707415	0.194466496838943\\
0.710733887635478	0.18436873747495\\
0.706126038379098	0.168336673346693\\
0.701208832942025	0.152304609218437\\
0.697394789579159	0.140616453694158\\
0.695971866346518	0.13627254509018\\
0.690391674265861	0.120240480961924\\
0.68452178183024	0.104208416833667\\
0.681362725450902	0.0960085969224064\\
0.678332132446255	0.0881763527054105\\
0.671826191984889	0.0721442885771539\\
0.665330661322646	0.0567900554454374\\
0.66504251073098	0.0561122244488974\\
0.657912432516187	0.0400801603206409\\
0.65051714751943	0.0240480961923843\\
0.649298597194389	0.0215093544722066\\
0.642787047187999	0.00801603206412782\\
0.634785530765666	-0.00801603206412826\\
0.633266533066132	-0.0109527098428168\\
0.62645283447006	-0.0240480961923848\\
0.617851892794968	-0.0400801603206413\\
0.617234468937876	-0.0411936396558003\\
0.608908861938725	-0.0561122244488979\\
0.601202404809619	-0.0695382561562252\\
0.599696174091843	-0.0721442885771544\\
0.590150461479603	-0.0881763527054109\\
0.585170340681363	-0.0963180305242836\\
0.58030792515261	-0.104208416833667\\
0.570169205257473	-0.120240480961924\\
0.569138276553106	-0.121828108571709\\
0.559683926397823	-0.13627254509018\\
0.55310621242485	-0.146085950191334\\
0.548902602517359	-0.152304609218437\\
0.537809939200304	-0.168336673346694\\
0.537074148296593	-0.1693753417743\\
0.526357199831221	-0.18436873747495\\
0.521042084168337	-0.191646302447054\\
0.514587433616802	-0.200400801603207\\
0.50501002004008	-0.213122240134114\\
0.502492575352914	-0.216432865731463\\
0.490049038473626	-0.232464929859719\\
0.488977955911824	-0.233817422274425\\
0.477229992186369	-0.248496993987976\\
0.472945891783567	-0.253750044496972\\
0.46405770415209	-0.264529058116232\\
0.456913827655311	-0.273035825651279\\
0.450520805470499	-0.280561122244489\\
0.440881763527054	-0.291708587488053\\
0.436606832638368	-0.296593186372745\\
0.424849699398798	-0.309799531610842\\
0.422302155581194	-0.312625250501002\\
0.408817635270541	-0.327337444040234\\
0.407591898736458	-0.328657314629258\\
0.392785571142285	-0.344348880453285\\
0.392459854497961	-0.344689378757515\\
0.376886131836701	-0.360721442885771\\
0.376753507014028	-0.360855964476592\\
0.360855964476592	-0.376753507014028\\
0.360721442885771	-0.376886131836701\\
0.344689378757515	-0.392459854497961\\
0.344348880453285	-0.392785571142285\\
0.328657314629258	-0.407591898736459\\
0.327337444040233	-0.408817635270541\\
0.312625250501002	-0.422302155581194\\
0.309799531610842	-0.424849699398798\\
0.296593186372745	-0.436606832638368\\
0.291708587488053	-0.440881763527054\\
0.280561122244489	-0.450520805470499\\
0.273035825651279	-0.456913827655311\\
0.264529058116232	-0.46405770415209\\
0.253750044496973	-0.472945891783567\\
0.248496993987976	-0.477229992186369\\
0.233817422274425	-0.488977955911824\\
0.232464929859719	-0.490049038473626\\
0.216432865731463	-0.502492575352915\\
0.213122240134114	-0.50501002004008\\
0.200400801603206	-0.514587433616802\\
0.191646302447054	-0.521042084168337\\
0.18436873747495	-0.526357199831222\\
0.1693753417743	-0.537074148296593\\
0.168336673346693	-0.537809939200304\\
0.152304609218437	-0.548902602517359\\
0.146085950191334	-0.55310621242485\\
0.13627254509018	-0.559683926397824\\
0.12182810857171	-0.569138276553106\\
0.120240480961924	-0.570169205257473\\
0.104208416833667	-0.58030792515261\\
0.0963180305242836	-0.585170340681363\\
0.0881763527054105	-0.590150461479603\\
0.0721442885771539	-0.599696174091843\\
0.0695382561562248	-0.601202404809619\\
0.0561122244488974	-0.608908861938725\\
0.0411936396557999	-0.617234468937876\\
0.0400801603206409	-0.617851892794969\\
0.0240480961923843	-0.62645283447006\\
0.0109527098428172	-0.633266533066132\\
0.00801603206412782	-0.634785530765667\\
-0.00801603206412826	-0.642787047187999\\
-0.0215093544722066	-0.649298597194389\\
-0.0240480961923848	-0.65051714751943\\
-0.0400801603206413	-0.657912432516187\\
-0.0561122244488979	-0.66504251073098\\
-0.0567900554454374	-0.665330661322646\\
-0.0721442885771544	-0.671826191984889\\
-0.0881763527054109	-0.678332132446254\\
-0.0960085969224055	-0.681362725450902\\
-0.104208416833667	-0.68452178183024\\
-0.120240480961924	-0.690391674265861\\
-0.13627254509018	-0.695971866346518\\
-0.140616453694156	-0.697394789579158\\
-0.152304609218437	-0.701208832942026\\
-0.168336673346694	-0.706126038379098\\
-0.18436873747495	-0.710733887635478\\
-0.194466496838941	-0.713426853707415\\
-0.200400801603207	-0.715004246748003\\
-0.216432865731463	-0.718915994936335\\
-0.232464929859719	-0.722493950193577\\
-0.248496993987976	-0.725728466131442\\
-0.264529058116232	-0.728608903396484\\
-0.269990729016995	-0.729458917835671\\
-0.280561122244489	-0.731099409269591\\
-0.296593186372745	-0.733202433816114\\
-0.312625250501002	-0.734916610868685\\
-0.328657314629258	-0.736227477754899\\
-0.344689378757515	-0.737119278633694\\
-0.360721442885771	-0.7375748739092\\
-0.376753507014028	-0.737575640878828\\
-0.392785571142285	-0.737101364710597\\
-0.408817635270541	-0.736130118733727\\
-0.424849699398798	-0.734638132900445\\
-0.440881763527054	-0.732599649133125\\
-0.456913827655311	-0.729986762106782\\
-0.459583598287604	-0.729458917835671\\
-0.472945891783567	-0.726709455491159\\
-0.488977955911824	-0.722764126967297\\
-0.50501002004008	-0.718124216345629\\
-0.519055152176374	-0.713426853707415\\
-0.521042084168337	-0.712730894650601\\
-0.537074148296593	-0.706408747703524\\
-0.55310621242485	-0.699224080847994\\
-0.556790786467565	-0.697394789579158\\
-0.569138276553106	-0.690928737524885\\
-0.585170340681363	-0.681551276387957\\
-0.585466178956859	-0.681362725450902\\
-0.601202404809619	-0.670716407499915\\
-0.608454003374408	-0.665330661322646\\
-0.617234468937876	-0.658372217382182\\
-0.627754465375623	-0.649298597194389\\
-0.633266533066132	-0.64419614391258\\
-0.64419614391258	-0.633266533066132\\
-0.649298597194389	-0.627754465375623\\
-0.658372217382182	-0.617234468937876\\
-0.665330661322646	-0.608454003374408\\
-0.670716407499915	-0.601202404809619\\
-0.681362725450902	-0.585466178956859\\
-0.681551276387957	-0.585170340681363\\
-0.690928737524885	-0.569138276553106\\
-0.697394789579158	-0.556790786467565\\
-0.699224080847994	-0.55310621242485\\
-0.706408747703524	-0.537074148296593\\
-0.712730894650601	-0.521042084168337\\
-0.713426853707415	-0.519055152176374\\
-0.718124216345629	-0.50501002004008\\
-0.722764126967297	-0.488977955911824\\
-0.726709455491159	-0.472945891783567\\
-0.729458917835671	-0.459583598287604\\
-0.729986762106782	-0.456913827655311\\
-0.732599649133125	-0.440881763527054\\
-0.734638132900445	-0.424849699398798\\
-0.736130118733727	-0.408817635270541\\
-0.737101364710597	-0.392785571142285\\
-0.737575640878828	-0.376753507014028\\
-0.7375748739092	-0.360721442885771\\
-0.737119278633694	-0.344689378757515\\
-0.736227477754899	-0.328657314629258\\
-0.734916610868685	-0.312625250501002\\
-0.733202433816114	-0.296593186372745\\
-0.731099409269591	-0.280561122244489\\
-0.729458917835671	-0.269990729016995\\
}--cycle;


\addplot[area legend,solid,fill=mycolor9,draw=black,forget plot]
table[row sep=crcr] {%
x	y\\
-0.521042084168337	-0.196068449747432\\
-0.519154978563505	-0.18436873747495\\
-0.516052177101956	-0.168336673346694\\
-0.512469794680332	-0.152304609218437\\
-0.508430178244887	-0.13627254509018\\
-0.50501002004008	-0.124068026587659\\
-0.503925086860612	-0.120240480961924\\
-0.498902820319506	-0.104208416833667\\
-0.493475691338864	-0.0881763527054109\\
-0.488977955911824	-0.0758182881556076\\
-0.487624960667906	-0.0721442885771544\\
-0.481278029335211	-0.0561122244488979\\
-0.474569018648952	-0.0400801603206413\\
-0.472945891783567	-0.0364301770819189\\
-0.467372359051814	-0.0240480961923848\\
-0.459796091888512	-0.00801603206412826\\
-0.456913827655311	-0.00221535008127976\\
-0.45176524586637	0.00801603206412782\\
-0.443344423177162	0.0240480961923843\\
-0.440881763527054	0.0285334374862488\\
-0.434458001167779	0.0400801603206409\\
-0.425212717624745	0.0561122244488974\\
-0.424849699398798	0.0567160653254005\\
-0.415446689103312	0.0721442885771539\\
-0.408817635270541	0.0826758142063546\\
-0.405305304228254	0.0881763527054105\\
-0.394722106324946	0.104208416833667\\
-0.392785571142285	0.107045971687435\\
-0.383645962578917	0.120240480961924\\
-0.376753507014028	0.129911568945012\\
-0.37214918717905	0.13627254509018\\
-0.360721442885771	0.15163221141154\\
-0.360212998835158	0.152304609218437\\
-0.347729519488568	0.168336673346693\\
-0.344689378757515	0.172142002363666\\
-0.334756540167215	0.18436873747495\\
-0.328657314629258	0.191695870164209\\
-0.321283911441855	0.200400801603206\\
-0.312625250501002	0.21038558264467\\
-0.30728525480633	0.216432865731463\\
-0.296593186372745	0.228269554372747\\
-0.292731443345579	0.232464929859719\\
-0.280561122244489	0.245400582233377\\
-0.277590327834089	0.248496993987976\\
-0.264529058116232	0.261826427776919\\
-0.261826427776919	0.264529058116232\\
-0.248496993987976	0.277590327834089\\
-0.245400582233377	0.280561122244489\\
-0.232464929859719	0.292731443345579\\
-0.228269554372747	0.296593186372745\\
-0.216432865731463	0.307285254806329\\
-0.21038558264467	0.312625250501002\\
-0.200400801603207	0.321283911441855\\
-0.191695870164209	0.328657314629258\\
-0.18436873747495	0.334756540167214\\
-0.172142002363666	0.344689378757515\\
-0.168336673346694	0.347729519488568\\
-0.152304609218437	0.360212998835158\\
-0.15163221141154	0.360721442885771\\
-0.13627254509018	0.372149187179049\\
-0.129911568945011	0.376753507014028\\
-0.120240480961924	0.383645962578916\\
-0.107045971687435	0.392785571142285\\
-0.104208416833667	0.394722106324945\\
-0.0881763527054109	0.405305304228253\\
-0.0826758142063546	0.408817635270541\\
-0.0721442885771544	0.415446689103312\\
-0.0567160653254	0.424849699398798\\
-0.0561122244488979	0.425212717624744\\
-0.0400801603206413	0.434458001167779\\
-0.0285334374862483	0.440881763527054\\
-0.0240480961923848	0.443344423177162\\
-0.00801603206412826	0.45176524586637\\
0.00221535008127977	0.456913827655311\\
0.00801603206412782	0.459796091888513\\
0.0240480961923843	0.467372359051813\\
0.0364301770819184	0.472945891783567\\
0.0400801603206409	0.474569018648952\\
0.0561122244488974	0.481278029335211\\
0.0721442885771539	0.487624960667906\\
0.0758182881556076	0.488977955911824\\
0.0881763527054105	0.493475691338864\\
0.104208416833667	0.498902820319506\\
0.120240480961924	0.503925086860611\\
0.124068026587661	0.50501002004008\\
0.13627254509018	0.508430178244887\\
0.152304609218437	0.512469794680332\\
0.168336673346693	0.516052177101955\\
0.18436873747495	0.519154978563505\\
0.196068449747432	0.521042084168337\\
0.200400801603206	0.521733058260525\\
0.216432865731463	0.52373812947242\\
0.232464929859719	0.525194634256982\\
0.248496993987976	0.526071621835439\\
0.264529058116232	0.526334970956469\\
0.280561122244489	0.525947064794894\\
0.296593186372745	0.524866423591917\\
0.312625250501002	0.523047288676869\\
0.325016079117948	0.521042084168337\\
0.328657314629258	0.520416219701775\\
0.344689378757515	0.516826882149682\\
0.360721442885771	0.512285217733694\\
0.376753507014028	0.506715782229231\\
0.380931784566237	0.50501002004008\\
0.392785571142285	0.499803736704689\\
0.408817635270541	0.491530346397246\\
0.413145202830435	0.488977955911824\\
0.424849699398798	0.481473839907374\\
0.436501382747325	0.472945891783567\\
0.440881763527054	0.469429034995746\\
0.45475870196107	0.456913827655311\\
0.456913827655311	0.45475870196107\\
0.469429034995746	0.440881763527054\\
0.472945891783567	0.436501382747325\\
0.481473839907374	0.424849699398798\\
0.488977955911824	0.413145202830435\\
0.491530346397246	0.408817635270541\\
0.499803736704689	0.392785571142285\\
0.50501002004008	0.380931784566237\\
0.506715782229231	0.376753507014028\\
0.512285217733694	0.360721442885771\\
0.516826882149682	0.344689378757515\\
0.520416219701775	0.328657314629258\\
0.521042084168337	0.325016079117948\\
0.523047288676869	0.312625250501002\\
0.524866423591917	0.296593186372745\\
0.525947064794894	0.280561122244489\\
0.526334970956469	0.264529058116232\\
0.526071621835439	0.248496993987976\\
0.525194634256982	0.232464929859719\\
0.52373812947242	0.216432865731463\\
0.521733058260525	0.200400801603206\\
0.521042084168337	0.196068449747432\\
0.519154978563505	0.18436873747495\\
0.516052177101955	0.168336673346693\\
0.512469794680332	0.152304609218437\\
0.508430178244887	0.13627254509018\\
0.50501002004008	0.124068026587659\\
0.503925086860612	0.120240480961924\\
0.498902820319506	0.104208416833667\\
0.493475691338864	0.0881763527054105\\
0.488977955911824	0.0758182881556076\\
0.487624960667906	0.0721442885771539\\
0.481278029335211	0.0561122244488974\\
0.474569018648952	0.0400801603206409\\
0.472945891783567	0.0364301770819184\\
0.467372359051813	0.0240480961923843\\
0.459796091888513	0.00801603206412782\\
0.456913827655311	0.00221535008127977\\
0.45176524586637	-0.00801603206412826\\
0.443344423177162	-0.0240480961923848\\
0.440881763527054	-0.0285334374862483\\
0.434458001167779	-0.0400801603206413\\
0.425212717624745	-0.0561122244488979\\
0.424849699398798	-0.0567160653254009\\
0.415446689103312	-0.0721442885771544\\
0.408817635270541	-0.0826758142063546\\
0.405305304228253	-0.0881763527054109\\
0.394722106324945	-0.104208416833667\\
0.392785571142285	-0.107045971687435\\
0.383645962578916	-0.120240480961924\\
0.376753507014028	-0.129911568945011\\
0.372149187179049	-0.13627254509018\\
0.360721442885771	-0.15163221141154\\
0.360212998835158	-0.152304609218437\\
0.347729519488568	-0.168336673346694\\
0.344689378757515	-0.172142002363666\\
0.334756540167214	-0.18436873747495\\
0.328657314629258	-0.191695870164209\\
0.321283911441855	-0.200400801603207\\
0.312625250501002	-0.21038558264467\\
0.307285254806329	-0.216432865731463\\
0.296593186372745	-0.228269554372747\\
0.292731443345579	-0.232464929859719\\
0.280561122244489	-0.245400582233378\\
0.277590327834089	-0.248496993987976\\
0.264529058116232	-0.261826427776919\\
0.261826427776919	-0.264529058116232\\
0.248496993987976	-0.277590327834089\\
0.245400582233377	-0.280561122244489\\
0.232464929859719	-0.292731443345579\\
0.228269554372747	-0.296593186372745\\
0.216432865731463	-0.30728525480633\\
0.21038558264467	-0.312625250501002\\
0.200400801603206	-0.321283911441855\\
0.191695870164209	-0.328657314629258\\
0.18436873747495	-0.334756540167215\\
0.172142002363666	-0.344689378757515\\
0.168336673346693	-0.347729519488568\\
0.152304609218437	-0.360212998835158\\
0.15163221141154	-0.360721442885771\\
0.13627254509018	-0.37214918717905\\
0.129911568945011	-0.376753507014028\\
0.120240480961924	-0.383645962578916\\
0.107045971687435	-0.392785571142285\\
0.104208416833667	-0.394722106324946\\
0.0881763527054105	-0.405305304228254\\
0.0826758142063546	-0.408817635270541\\
0.0721442885771539	-0.415446689103312\\
0.0567160653254005	-0.424849699398798\\
0.0561122244488974	-0.425212717624745\\
0.0400801603206409	-0.434458001167779\\
0.0285334374862488	-0.440881763527054\\
0.0240480961923843	-0.443344423177162\\
0.00801603206412782	-0.45176524586637\\
-0.00221535008127976	-0.456913827655311\\
-0.00801603206412826	-0.459796091888512\\
-0.0240480961923848	-0.467372359051814\\
-0.0364301770819189	-0.472945891783567\\
-0.0400801603206413	-0.474569018648952\\
-0.0561122244488979	-0.481278029335211\\
-0.0721442885771544	-0.487624960667906\\
-0.0758182881556076	-0.488977955911824\\
-0.0881763527054109	-0.493475691338864\\
-0.104208416833667	-0.498902820319506\\
-0.120240480961924	-0.503925086860612\\
-0.12406802658766	-0.50501002004008\\
-0.13627254509018	-0.508430178244887\\
-0.152304609218437	-0.512469794680332\\
-0.168336673346694	-0.516052177101956\\
-0.18436873747495	-0.519154978563505\\
-0.196068449747432	-0.521042084168337\\
-0.200400801603207	-0.521733058260525\\
-0.216432865731463	-0.523738129472421\\
-0.232464929859719	-0.525194634256982\\
-0.248496993987976	-0.526071621835439\\
-0.264529058116232	-0.526334970956469\\
-0.280561122244489	-0.525947064794894\\
-0.296593186372745	-0.524866423591916\\
-0.312625250501002	-0.523047288676869\\
-0.325016079117948	-0.521042084168337\\
-0.328657314629258	-0.520416219701775\\
-0.344689378757515	-0.516826882149682\\
-0.360721442885771	-0.512285217733694\\
-0.376753507014028	-0.506715782229231\\
-0.380931784566238	-0.50501002004008\\
-0.392785571142285	-0.499803736704689\\
-0.408817635270541	-0.491530346397246\\
-0.413145202830435	-0.488977955911824\\
-0.424849699398798	-0.481473839907375\\
-0.436501382747325	-0.472945891783567\\
-0.440881763527054	-0.469429034995746\\
-0.454758701961071	-0.456913827655311\\
-0.456913827655311	-0.454758701961071\\
-0.469429034995746	-0.440881763527054\\
-0.472945891783567	-0.436501382747325\\
-0.481473839907375	-0.424849699398798\\
-0.488977955911824	-0.413145202830435\\
-0.491530346397246	-0.408817635270541\\
-0.499803736704689	-0.392785571142285\\
-0.50501002004008	-0.380931784566238\\
-0.506715782229231	-0.376753507014028\\
-0.512285217733694	-0.360721442885771\\
-0.516826882149681	-0.344689378757515\\
-0.520416219701775	-0.328657314629258\\
-0.521042084168337	-0.325016079117948\\
-0.523047288676869	-0.312625250501002\\
-0.524866423591916	-0.296593186372745\\
-0.525947064794894	-0.280561122244489\\
-0.526334970956469	-0.264529058116232\\
-0.526071621835439	-0.248496993987976\\
-0.525194634256982	-0.232464929859719\\
-0.523738129472421	-0.216432865731463\\
-0.521733058260525	-0.200400801603207\\
-0.521042084168337	-0.196068449747432\\
}--cycle;


\addplot[area legend,solid,fill=mycolor10,draw=black,forget plot]
table[row sep=crcr] {%
x	y\\
-0.200400801603207	-0.0688708081643159\\
-0.197603974805902	-0.0561122244488979\\
-0.192890641105737	-0.0400801603206413\\
-0.187165288096577	-0.0240480961923848\\
-0.18436873747495	-0.0174320563990579\\
-0.180212445791678	-0.00801603206412826\\
-0.172203233598904	0.00801603206412782\\
-0.168336673346694	0.0149698831851242\\
-0.163062712724746	0.0240480961923843\\
-0.1529603153378	0.0400801603206409\\
-0.152304609218437	0.0410314214231889\\
-0.141437992804138	0.0561122244488974\\
-0.13627254509018	0.0628400782283337\\
-0.128786357751926	0.0721442885771539\\
-0.120240480961924	0.0821526014749899\\
-0.114835477094186	0.0881763527054105\\
-0.104208416833667	0.0993768895215626\\
-0.0993768895215625	0.104208416833667\\
-0.0881763527054109	0.114835477094185\\
-0.0821526014749908	0.120240480961924\\
-0.0721442885771544	0.128786357751925\\
-0.0628400782283342	0.13627254509018\\
-0.0561122244488979	0.141437992804137\\
-0.0410314214231894	0.152304609218437\\
-0.0400801603206413	0.152960315337799\\
-0.0240480961923848	0.163062712724745\\
-0.0149698831851247	0.168336673346693\\
-0.00801603206412826	0.172203233598903\\
0.00801603206412782	0.180212445791678\\
0.0174320563990574	0.18436873747495\\
0.0240480961923843	0.187165288096577\\
0.0400801603206409	0.192890641105736\\
0.0561122244488974	0.197603974805901\\
0.0688708081643154	0.200400801603206\\
0.0721442885771539	0.201086946930052\\
0.0881763527054105	0.203078193878165\\
0.104208416833667	0.203626762013655\\
0.120240480961924	0.202499595261507\\
0.131319861405914	0.200400801603206\\
0.13627254509018	0.199283603178087\\
0.152304609218437	0.193128887322223\\
0.167240716832899	0.18436873747495\\
0.168336673346693	0.183558890899121\\
0.183558890899121	0.168336673346693\\
0.18436873747495	0.167240716832899\\
0.193128887322223	0.152304609218437\\
0.199283603178087	0.13627254509018\\
0.200400801603206	0.131319861405914\\
0.202499595261507	0.120240480961924\\
0.203626762013655	0.104208416833667\\
0.203078193878165	0.0881763527054105\\
0.201086946930052	0.0721442885771539\\
0.200400801603206	0.0688708081643154\\
0.197603974805901	0.0561122244488974\\
0.192890641105736	0.0400801603206409\\
0.187165288096577	0.0240480961923843\\
0.18436873747495	0.0174320563990574\\
0.180212445791678	0.00801603206412782\\
0.172203233598903	-0.00801603206412826\\
0.168336673346693	-0.0149698831851247\\
0.163062712724745	-0.0240480961923848\\
0.152960315337799	-0.0400801603206413\\
0.152304609218437	-0.0410314214231894\\
0.141437992804137	-0.0561122244488979\\
0.13627254509018	-0.0628400782283342\\
0.128786357751925	-0.0721442885771544\\
0.120240480961924	-0.0821526014749908\\
0.114835477094185	-0.0881763527054109\\
0.104208416833667	-0.0993768895215625\\
0.0993768895215626	-0.104208416833667\\
0.0881763527054105	-0.114835477094186\\
0.0821526014749899	-0.120240480961924\\
0.0721442885771539	-0.128786357751926\\
0.0628400782283337	-0.13627254509018\\
0.0561122244488974	-0.141437992804138\\
0.0410314214231889	-0.152304609218437\\
0.0400801603206409	-0.1529603153378\\
0.0240480961923843	-0.163062712724746\\
0.0149698831851242	-0.168336673346694\\
0.00801603206412782	-0.172203233598904\\
-0.00801603206412826	-0.180212445791678\\
-0.0174320563990579	-0.18436873747495\\
-0.0240480961923848	-0.187165288096577\\
-0.0400801603206413	-0.192890641105737\\
-0.0561122244488979	-0.197603974805902\\
-0.0688708081643159	-0.200400801603207\\
-0.0721442885771544	-0.201086946930052\\
-0.0881763527054109	-0.203078193878165\\
-0.104208416833667	-0.203626762013654\\
-0.120240480961924	-0.202499595261507\\
-0.131319861405914	-0.200400801603207\\
-0.13627254509018	-0.199283603178088\\
-0.152304609218437	-0.193128887322223\\
-0.167240716832896	-0.18436873747495\\
-0.168336673346694	-0.183558890899118\\
-0.183558890899118	-0.168336673346694\\
-0.18436873747495	-0.167240716832896\\
-0.193128887322223	-0.152304609218437\\
-0.199283603178088	-0.13627254509018\\
-0.200400801603207	-0.131319861405914\\
-0.202499595261507	-0.120240480961924\\
-0.203626762013654	-0.104208416833667\\
-0.203078193878165	-0.0881763527054109\\
-0.201086946930052	-0.0721442885771544\\
-0.200400801603207	-0.0688708081643159\\
}--cycle;

\end{axis}
\end{tikzpicture}%
    \caption{$\mat{\Sigma} = \begin{bmatrix} 1 & \nicefrac{1}{2} \\ \nicefrac{1}{2} & 1 \end{bmatrix}$}
  \end{subfigure}
  \begin{subfigure}{0.33\textwidth}
    % This file was created by matlab2tikz.
% Minimal pgfplots version: 1.3
%
\tikzsetnextfilename{2d_gaussian_pdf_3}
\definecolor{mycolor1}{rgb}{0.01430,0.01430,0.01430}%
\definecolor{mycolor2}{rgb}{0.15932,0.06827,0.17506}%
\definecolor{mycolor3}{rgb}{0.17345,0.11709,0.41691}%
\definecolor{mycolor4}{rgb}{0.10466,0.22842,0.49922}%
\definecolor{mycolor5}{rgb}{0.03136,0.34573,0.47968}%
\definecolor{mycolor6}{rgb}{0.00003,0.46181,0.36160}%
\definecolor{mycolor7}{rgb}{0.00000,0.57116,0.23204}%
\definecolor{mycolor8}{rgb}{0.09251,0.67012,0.06175}%
\definecolor{mycolor9}{rgb}{0.37724,0.75416,0.00000}%
\definecolor{mycolor10}{rgb}{0.74811,0.78629,0.07418}%
\definecolor{mycolor11}{rgb}{0.94890,0.82638,0.64748}%
\definecolor{mycolor12}{rgb}{0.96920,0.92730,0.89610}%
%
\begin{tikzpicture}

\begin{axis}[%
width=0.95092\smallsquarefigurewidth,
height=\smallsquarefigureheight,
at={(0\smallsquarefigurewidth,0\smallsquarefigureheight)},
scale only axis,
xmin=-4,
xmax=4,
xlabel={$x_1$},
ymin=-4,
ymax=4,
ylabel={$x_2$},
axis x line*=bottom,
axis y line*=left
]

\addplot[area legend,solid,fill=mycolor1,draw=black,forget plot]
table[row sep=crcr] {%
x	y\\
-4	4.00000000053783\\
-3.98396793587174	4.00000000057334\\
-3.96793587174349	4.00000000061096\\
-3.95190380761523	4.0000000006508\\
-3.93587174348697	4.00000000069296\\
-3.91983967935872	4.00000000073758\\
-3.90380761523046	4.00000000078476\\
-3.8877755511022	4.00000000083464\\
-3.87174348697395	4.00000000088735\\
-3.85571142284569	4.00000000094302\\
-3.83967935871743	4.0000000010018\\
-3.82364729458918	4.00000000106383\\
-3.80761523046092	4.00000000112927\\
-3.79158316633267	4.00000000119827\\
-3.77555110220441	4.000000001271\\
-3.75951903807615	4.00000000134762\\
-3.7434869739479	4.00000000142832\\
-3.72745490981964	4.00000000151325\\
-3.71142284569138	4.00000000160263\\
-3.69539078156313	4.00000000169662\\
-3.67935871743487	4.00000000179544\\
-3.66332665330661	4.00000000189928\\
-3.64729458917836	4.00000000200835\\
-3.6312625250501	4.00000000212287\\
-3.61523046092184	4.00000000224305\\
-3.59919839679359	4.00000000236912\\
-3.58316633266533	4.00000000250131\\
-3.56713426853707	4.00000000263986\\
-3.55110220440882	4.00000000278501\\
-3.53507014028056	4.00000000293701\\
-3.5190380761523	4.00000000309611\\
-3.50300601202405	4.00000000326257\\
-3.48697394789579	4.00000000343666\\
-3.47094188376753	4.00000000361864\\
-3.45490981963928	4.00000000380879\\
-3.43887775551102	4.00000000400738\\
-3.42284569138277	4.00000000421471\\
-3.40681362725451	4.00000000443105\\
-3.39078156312625	4.0000000046567\\
-3.374749498998	4.00000000489195\\
-3.35871743486974	4.00000000513711\\
-3.34268537074148	4.00000000539247\\
-3.32665330661323	4.00000000565835\\
-3.31062124248497	4.00000000593505\\
-3.29458917835671	4.00000000622288\\
-3.27855711422846	4.00000000652215\\
-3.2625250501002	4.00000000683318\\
-3.24649298597194	4.00000000715629\\
-3.23046092184369	4.00000000749178\\
-3.21442885771543	4.00000000783997\\
-3.19839679358717	4.00000000820119\\
-3.18236472945892	4.00000000857575\\
-3.16633266533066	4.00000000896395\\
-3.1503006012024	4.00000000936611\\
-3.13426853707415	4.00000000978255\\
-3.11823647294589	4.00000001021356\\
-3.10220440881764	4.00000001065945\\
-3.08617234468938	4.00000001112052\\
-3.07014028056112	4.00000001159706\\
-3.05410821643287	4.00000001208936\\
-3.03807615230461	4.0000000125977\\
-3.02204408817635	4.00000001312236\\
-3.0060120240481	4.00000001366359\\
-2.98997995991984	4.00000001422167\\
-2.97394789579158	4.00000001479684\\
-2.95791583166333	4.00000001538933\\
-2.94188376753507	4.00000001599937\\
-2.92585170340681	4.00000001662719\\
-2.90981963927856	4.00000001727298\\
-2.8937875751503	4.00000001793694\\
-2.87775551102204	4.00000001861924\\
-2.86172344689379	4.00000001932004\\
-2.84569138276553	4.00000002003949\\
-2.82965931863727	4.00000002077772\\
-2.81362725450902	4.00000002153485\\
-2.79759519038076	4.00000002231096\\
-2.7815631262525	4.00000002310613\\
-2.76553106212425	4.00000002392041\\
-2.74949899799599	4.00000002475385\\
-2.73346693386774	4.00000002560645\\
-2.71743486973948	4.0000000264782\\
-2.70140280561122	4.00000002736908\\
-2.68537074148297	4.00000002827904\\
-2.66933867735471	4.00000002920798\\
-2.65330661322645	4.0000000301558\\
-2.6372745490982	4.00000003112239\\
-2.62124248496994	4.00000003210757\\
-2.60521042084168	4.00000003311117\\
-2.58917835671343	4.00000003413298\\
-2.57314629258517	4.00000003517276\\
-2.55711422845691	4.00000003623025\\
-2.54108216432866	4.00000003730514\\
-2.5250501002004	4.00000003839712\\
-2.50901803607214	4.00000003950582\\
-2.49298597194389	4.00000004063087\\
-2.47695390781563	4.00000004177185\\
-2.46092184368737	4.00000004292832\\
-2.44488977955912	4.0000000440998\\
-2.42885771543086	4.00000004528579\\
-2.41282565130261	4.00000004648575\\
-2.39679358717435	4.0000000476991\\
-2.38076152304609	4.00000004892527\\
-2.36472945891784	4.0000000501636\\
-2.34869739478958	4.00000005141346\\
-2.33266533066132	4.00000005267415\\
-2.31663326653307	4.00000005394494\\
-2.30060120240481	4.0000000552251\\
-2.28456913827655	4.00000005651385\\
-2.2685370741483	4.00000005781037\\
-2.25250501002004	4.00000005911385\\
-2.23647294589178	4.00000006042342\\
-2.22044088176353	4.00000006173819\\
-2.20440881763527	4.00000006305725\\
-2.18837675350701	4.00000006437967\\
-2.17234468937876	4.00000006570449\\
-2.1563126252505	4.00000006703072\\
-2.14028056112224	4.00000006835736\\
-2.12424849699399	4.00000006968339\\
-2.10821643286573	4.00000007100775\\
-2.09218436873747	4.0000000723294\\
-2.07615230460922	4.00000007364725\\
-2.06012024048096	4.0000000749602\\
-2.04408817635271	4.00000007626715\\
-2.02805611222445	4.00000007756697\\
-2.01202404809619	4.00000007885854\\
-1.99599198396794	4.00000008014071\\
-1.97995991983968	4.00000008141234\\
-1.96392785571142	4.00000008267226\\
-1.94789579158317	4.00000008391932\\
-1.93186372745491	4.00000008515236\\
-1.91583166332665	4.0000000863702\\
-1.8997995991984	4.0000000875717\\
-1.88376753507014	4.00000008875568\\
-1.86773547094188	4.000000089921\\
-1.85170340681363	4.0000000910665\\
-1.83567134268537	4.00000009219105\\
-1.81963927855711	4.0000000932935\\
-1.80360721442886	4.00000009437275\\
-1.7875751503006	4.00000009542768\\
-1.77154308617234	4.00000009645721\\
-1.75551102204409	4.00000009746027\\
-1.73947895791583	4.0000000984358\\
-1.72344689378758	4.00000009938277\\
-1.70741482965932	4.00000010030017\\
-1.69138276553106	4.00000010118702\\
-1.67535070140281	4.00000010204237\\
-1.65931863727455	4.00000010286528\\
-1.64328657314629	4.00000010365485\\
-1.62725450901804	4.00000010441023\\
-1.61122244488978	4.00000010513056\\
-1.59519038076152	4.00000010581507\\
-1.57915831663327	4.00000010646298\\
-1.56312625250501	4.00000010707356\\
-1.54709418837675	4.00000010764614\\
-1.5310621242485	4.00000010818007\\
-1.51503006012024	4.00000010867473\\
-1.49899799599198	4.00000010912958\\
-1.48296593186373	4.00000010954409\\
-1.46693386773547	4.00000010991779\\
-1.45090180360721	4.00000011025024\\
-1.43486973947896	4.00000011054108\\
-1.4188376753507	4.00000011078997\\
-1.40280561122244	4.00000011099661\\
-1.38677354709419	4.00000011116077\\
-1.37074148296593	4.00000011128227\\
-1.35470941883768	4.00000011136095\\
-1.33867735470942	4.00000011139673\\
-1.32264529058116	4.00000011138958\\
-1.30661322645291	4.00000011133949\\
-1.29058116232465	4.00000011124652\\
-1.27454909819639	4.00000011111078\\
-1.25851703406814	4.00000011093244\\
-1.24248496993988	4.00000011071169\\
-1.22645290581162	4.00000011044878\\
-1.21042084168337	4.00000011014403\\
-1.19438877755511	4.00000010979778\\
-1.17835671342685	4.00000010941043\\
-1.1623246492986	4.00000010898242\\
-1.14629258517034	4.00000010851424\\
-1.13026052104208	4.00000010800642\\
-1.11422845691383	4.00000010745955\\
-1.09819639278557	4.00000010687423\\
-1.08216432865731	4.00000010625112\\
-1.06613226452906	4.00000010559093\\
-1.0501002004008	4.00000010489439\\
-1.03406813627255	4.00000010416229\\
-1.01803607214429	4.00000010339542\\
-1.00200400801603	4.00000010259463\\
-0.985971943887776	4.00000010176081\\
-0.969939879759519	4.00000010089486\\
-0.953907815631263	4.00000009999772\\
-0.937875751503006	4.00000009907035\\
-0.92184368737475	4.00000009811374\\
-0.905811623246493	4.00000009712892\\
-0.889779559118236	4.00000009611693\\
-0.87374749498998	4.00000009507881\\
-0.857715430861723	4.00000009401565\\
-0.841683366733467	4.00000009292854\\
-0.82565130260521	4.0000000918186\\
-0.809619238476954	4.00000009068694\\
-0.793587174348697	4.00000008953471\\
-0.777555110220441	4.00000008836304\\
-0.761523046092184	4.00000008717309\\
-0.745490981963928	4.00000008596601\\
-0.729458917835671	4.00000008474298\\
-0.713426853707415	4.00000008350514\\
-0.697394789579158	4.00000008225366\\
-0.681362725450902	4.00000008098971\\
-0.665330661322646	4.00000007971444\\
-0.649298597194389	4.00000007842901\\
-0.633266533066132	4.00000007713456\\
-0.617234468937876	4.00000007583223\\
-0.601202404809619	4.00000007452316\\
-0.585170340681363	4.00000007320845\\
-0.569138276553106	4.00000007188922\\
-0.55310621242485	4.00000007056655\\
-0.537074148296593	4.00000006924151\\
-0.521042084168337	4.00000006791516\\
-0.50501002004008	4.00000006658855\\
-0.488977955911824	4.00000006526268\\
-0.472945891783567	4.00000006393855\\
-0.456913827655311	4.00000006261714\\
-0.440881763527054	4.00000006129941\\
-0.424849699398798	4.00000005998627\\
-0.408817635270541	4.00000005867864\\
-0.392785571142285	4.00000005737738\\
-0.376753507014028	4.00000005608336\\
-0.360721442885771	4.00000005479739\\
-0.344689378757515	4.00000005352027\\
-0.328657314629258	4.00000005225276\\
-0.312625250501002	4.00000005099561\\
-0.296593186372745	4.00000004974951\\
-0.280561122244489	4.00000004851516\\
-0.264529058116232	4.0000000472932\\
-0.248496993987976	4.00000004608424\\
-0.232464929859719	4.00000004488888\\
-0.216432865731463	4.00000004370767\\
-0.200400801603207	4.00000004254114\\
-0.18436873747495	4.00000004138978\\
-0.168336673346694	4.00000004025406\\
-0.152304609218437	4.00000003913442\\
-0.13627254509018	4.00000003803125\\
-0.120240480961924	4.00000003694493\\
-0.104208416833667	4.0000000358758\\
-0.0881763527054109	4.00000003482419\\
-0.0721442885771544	4.00000003379037\\
-0.0561122244488979	4.00000003277461\\
-0.0400801603206413	4.00000003177712\\
-0.0240480961923848	4.00000003079812\\
-0.00801603206412826	4.00000002983777\\
0.00801603206412782	4.00000002889622\\
0.0240480961923843	4.0000000279736\\
0.0400801603206409	4.00000002707\\
0.0561122244488974	4.00000002618549\\
0.0721442885771539	4.00000002532012\\
0.0881763527054105	4.00000002447391\\
0.104208416833667	4.00000002364686\\
0.120240480961924	4.00000002283895\\
0.13627254509018	4.00000002205014\\
0.152304609218437	4.00000002128037\\
0.168336673346693	4.00000002052955\\
0.18436873747495	4.00000001979759\\
0.200400801603206	4.00000001908437\\
0.216432865731463	4.00000001838976\\
0.232464929859719	4.00000001771359\\
0.248496993987976	4.00000001705571\\
0.264529058116232	4.00000001641593\\
0.280561122244489	4.00000001579406\\
0.296593186372745	4.00000001518989\\
0.312625250501002	4.0000000146032\\
0.328657314629258	4.00000001403376\\
0.344689378757515	4.00000001348132\\
0.360721442885771	4.00000001294564\\
0.376753507014028	4.00000001242646\\
0.392785571142285	4.0000000119235\\
0.408817635270541	4.00000001143648\\
0.424849699398798	4.00000001096513\\
0.440881763527054	4.00000001050915\\
0.456913827655311	4.00000001006825\\
0.472945891783567	4.00000000964213\\
0.488977955911824	4.00000000923049\\
0.50501002004008	4.00000000883301\\
0.521042084168337	4.00000000844939\\
0.537074148296593	4.00000000807932\\
0.55310621242485	4.00000000772248\\
0.569138276553106	4.00000000737855\\
0.585170340681363	4.00000000704722\\
0.601202404809619	4.00000000672818\\
0.617234468937876	4.0000000064211\\
0.633266533066132	4.00000000612568\\
0.649298597194389	4.00000000584159\\
0.665330661322646	4.00000000556854\\
0.681362725450902	4.0000000053062\\
0.697394789579159	4.00000000505427\\
0.713426853707415	4.00000000481245\\
0.729458917835672	4.00000000458043\\
0.745490981963928	4.00000000435791\\
0.761523046092185	4.00000000414461\\
0.777555110220441	4.00000000394023\\
0.793587174348698	4.00000000374448\\
0.809619238476954	4.00000000355709\\
0.825651302605211	4.00000000337777\\
0.841683366733467	4.00000000320625\\
0.857715430861724	4.00000000304227\\
0.87374749498998	4.00000000288557\\
0.889779559118236	4.00000000273588\\
0.905811623246493	4.00000000259296\\
0.921843687374749	4.00000000245655\\
0.937875751503006	4.00000000232643\\
0.953907815631262	4.00000000220234\\
0.969939879759519	4.00000000208408\\
0.985971943887775	4.0000000019714\\
1.00200400801603	4.00000000186409\\
1.01803607214429	4.00000000176195\\
1.03406813627254	4.00000000166477\\
1.0501002004008	4.00000000157233\\
1.06613226452906	4.00000000148446\\
1.08216432865731	4.00000000140096\\
1.09819639278557	4.00000000132164\\
1.11422845691383	4.00000000124633\\
1.13026052104208	4.00000000117487\\
1.14629258517034	4.00000000110707\\
1.1623246492986	4.00000000104278\\
1.17835671342685	4.00000000098185\\
1.19438877755511	4.00000000092412\\
1.21042084168337	4.00000000086946\\
1.22645290581162	4.00000000081771\\
1.24248496993988	4.00000000076874\\
1.25851703406814	4.00000000072243\\
1.27454909819639	4.00000000067864\\
1.29058116232465	4.00000000063727\\
1.30661322645291	4.00000000059818\\
1.32264529058116	4.00000000056128\\
1.33867735470942	4.00000000052645\\
1.35470941883768	4.00000000049359\\
1.37074148296593	4.0000000004626\\
1.38677354709419	4.00000000043339\\
1.40280561122244	4.00000000040587\\
1.4188376753507	4.00000000037995\\
1.43486973947896	4.00000000035555\\
1.45090180360721	4.00000000033259\\
1.46693386773547	4.00000000031099\\
1.48296593186373	4.00000000029068\\
1.49899799599198	4.00000000027159\\
1.51503006012024	4.00000000025366\\
1.5310621242485	4.00000000023682\\
1.54709418837675	4.00000000022102\\
1.56312625250501	4.00000000020618\\
1.57915831663327	4.00000000019228\\
1.59519038076152	4.00000000017923\\
1.61122244488978	4.00000000016701\\
1.62725450901804	4.00000000015557\\
1.64328657314629	4.00000000014485\\
1.65931863727455	4.00000000013482\\
1.67535070140281	4.00000000012543\\
1.69138276553106	4.00000000011665\\
1.70741482965932	4.00000000010845\\
1.72344689378758	4.00000000010078\\
1.73947895791583	4.00000000009362\\
1.75551102204409	4.00000000008693\\
1.77154308617235	4.0000000000807\\
1.7875751503006	4.00000000007488\\
1.80360721442886	4.00000000006945\\
1.81963927855711	4.00000000006439\\
1.83567134268537	4.00000000005968\\
1.85170340681363	4.00000000005529\\
1.86773547094188	4.0000000000512\\
1.88376753507014	4.0000000000474\\
1.8997995991984	4.00000000004386\\
1.91583166332665	4.00000000004057\\
1.93186372745491	4.00000000003751\\
1.94789579158317	4.00000000003468\\
1.96392785571142	4.00000000003204\\
1.97995991983968	4.00000000002959\\
1.99599198396794	4.00000000002732\\
2.01202404809619	4.00000000002521\\
2.02805611222445	4.00000000002326\\
2.04408817635271	4.00000000002145\\
2.06012024048096	4.00000000001977\\
2.07615230460922	4.00000000001822\\
2.09218436873747	4.00000000001678\\
2.10821643286573	4.00000000001545\\
2.12424849699399	4.00000000001422\\
2.14028056112224	4.00000000001308\\
2.1563126252505	4.00000000001203\\
2.17234468937876	4.00000000001106\\
2.18837675350701	4.00000000001017\\
2.20440881763527	4.00000000000934\\
2.22044088176353	4.00000000000858\\
2.23647294589178	4.00000000000787\\
2.25250501002004	4.00000000000722\\
2.2685370741483	4.00000000000662\\
2.28456913827655	4.00000000000607\\
2.30060120240481	4.00000000000557\\
2.31663326653307	4.0000000000051\\
2.33266533066132	4.00000000000467\\
2.34869739478958	4.00000000000427\\
2.36472945891784	4.00000000000391\\
2.38076152304609	4.00000000000358\\
2.39679358717435	4.00000000000327\\
2.41282565130261	4.00000000000299\\
2.42885771543086	4.00000000000273\\
2.44488977955912	4.0000000000025\\
2.46092184368737	4.00000000000228\\
2.47695390781563	4.00000000000208\\
2.49298597194389	4.0000000000019\\
2.50901803607214	4.00000000000173\\
2.5250501002004	4.00000000000158\\
2.54108216432866	4.00000000000144\\
2.55711422845691	4.00000000000131\\
2.57314629258517	4.00000000000119\\
2.58917835671343	4.00000000000108\\
2.60521042084168	4.00000000000099\\
2.62124248496994	4.0000000000009\\
2.6372745490982	4.00000000000082\\
2.65330661322645	4.00000000000074\\
2.66933867735471	4.00000000000067\\
2.68537074148297	4.00000000000061\\
2.70140280561122	4.00000000000056\\
2.71743486973948	4.0000000000005\\
2.73346693386774	4.00000000000046\\
2.74949899799599	4.00000000000041\\
2.76553106212425	4.00000000000038\\
2.7815631262525	4.00000000000034\\
2.79759519038076	4.00000000000031\\
2.81362725450902	4.00000000000028\\
2.82965931863727	4.00000000000025\\
2.84569138276553	4.00000000000023\\
2.86172344689379	4.00000000000021\\
2.87775551102204	4.00000000000019\\
2.8937875751503	4.00000000000017\\
2.90981963927856	4.00000000000015\\
2.92585170340681	4.00000000000014\\
2.94188376753507	4.00000000000012\\
2.95791583166333	4.00000000000011\\
2.97394789579158	4.0000000000001\\
2.98997995991984	4.00000000000009\\
3.0060120240481	4.00000000000008\\
3.02204408817635	4.00000000000007\\
3.03807615230461	4.00000000000007\\
3.05410821643287	4.00000000000006\\
3.07014028056112	4.00000000000005\\
3.08617234468938	4.00000000000005\\
3.10220440881764	4.00000000000004\\
3.11823647294589	4.00000000000004\\
3.13426853707415	4.00000000000003\\
3.1503006012024	4.00000000000003\\
3.16633266533066	4.00000000000003\\
3.18236472945892	4.00000000000003\\
3.19839679358717	4.00000000000002\\
3.21442885771543	4.00000000000002\\
3.23046092184369	4.00000000000002\\
3.24649298597194	4.00000000000002\\
3.2625250501002	4.00000000000001\\
3.27855711422846	4.00000000000001\\
3.29458917835671	4.00000000000001\\
3.31062124248497	4.00000000000001\\
3.32665330661323	4.00000000000001\\
3.34268537074148	4.00000000000001\\
3.35871743486974	4.00000000000001\\
3.374749498998	4.00000000000001\\
3.39078156312625	4.00000000000001\\
3.40681362725451	4.00000000000001\\
3.42284569138277	4\\
3.43887775551102	4\\
3.45490981963928	4\\
3.47094188376754	4\\
3.48697394789579	4\\
3.50300601202405	4\\
3.51903807615231	4\\
3.53507014028056	4\\
3.55110220440882	4\\
3.56713426853707	4\\
3.58316633266533	4\\
3.59919839679359	4\\
3.61523046092184	4\\
3.6312625250501	4\\
3.64729458917836	4\\
3.66332665330661	4\\
3.67935871743487	4\\
3.69539078156313	4\\
3.71142284569138	4\\
3.72745490981964	4\\
3.7434869739479	4\\
3.75951903807615	4\\
3.77555110220441	4\\
3.79158316633267	4\\
3.80761523046092	4\\
3.82364729458918	4\\
3.83967935871743	4\\
3.85571142284569	4\\
3.87174348697395	4\\
3.8877755511022	4\\
3.90380761523046	4\\
3.91983967935872	4\\
3.93587174348697	4\\
3.95190380761523	4\\
3.96793587174349	4\\
3.98396793587174	4\\
4	4\\
4	3.98396793587174\\
4	3.96793587174349\\
4	3.95190380761523\\
4	3.93587174348697\\
4	3.91983967935872\\
4	3.90380761523046\\
4	3.8877755511022\\
4	3.87174348697395\\
4	3.85571142284569\\
4	3.83967935871743\\
4	3.82364729458918\\
4	3.80761523046092\\
4	3.79158316633267\\
4	3.77555110220441\\
4	3.75951903807615\\
4	3.7434869739479\\
4	3.72745490981964\\
4	3.71142284569138\\
4	3.69539078156313\\
4	3.67935871743487\\
4	3.66332665330661\\
4	3.64729458917836\\
4	3.6312625250501\\
4	3.61523046092184\\
4	3.59919839679359\\
4	3.58316633266533\\
4	3.56713426853707\\
4	3.55110220440882\\
4	3.53507014028056\\
4	3.51903807615231\\
4	3.50300601202405\\
4	3.48697394789579\\
4	3.47094188376754\\
4	3.45490981963928\\
4	3.43887775551102\\
4	3.42284569138277\\
4	3.40681362725451\\
4	3.39078156312625\\
4	3.374749498998\\
4	3.35871743486974\\
4	3.34268537074148\\
4	3.32665330661323\\
4	3.31062124248497\\
4	3.29458917835671\\
4	3.27855711422846\\
4	3.2625250501002\\
4	3.24649298597194\\
4	3.23046092184369\\
4	3.21442885771543\\
4	3.19839679358717\\
4	3.18236472945892\\
4	3.16633266533066\\
4	3.1503006012024\\
4	3.13426853707415\\
4	3.11823647294589\\
4	3.10220440881764\\
4	3.08617234468938\\
4	3.07014028056112\\
4	3.05410821643287\\
4	3.03807615230461\\
4	3.02204408817635\\
4	3.0060120240481\\
4	2.98997995991984\\
4	2.97394789579158\\
4	2.95791583166333\\
4	2.94188376753507\\
4	2.92585170340681\\
4	2.90981963927856\\
4	2.8937875751503\\
4	2.87775551102204\\
4	2.86172344689379\\
4	2.84569138276553\\
4	2.82965931863727\\
4	2.81362725450902\\
4.00000000000001	2.79759519038076\\
4.00000000000001	2.7815631262525\\
4.00000000000001	2.76553106212425\\
4.00000000000001	2.74949899799599\\
4.00000000000001	2.73346693386774\\
4.00000000000001	2.71743486973948\\
4.00000000000001	2.70140280561122\\
4.00000000000001	2.68537074148297\\
4.00000000000001	2.66933867735471\\
4.00000000000001	2.65330661322645\\
4.00000000000001	2.6372745490982\\
4.00000000000001	2.62124248496994\\
4.00000000000001	2.60521042084168\\
4.00000000000001	2.58917835671343\\
4.00000000000001	2.57314629258517\\
4.00000000000001	2.55711422845691\\
4.00000000000001	2.54108216432866\\
4.00000000000001	2.5250501002004\\
4.00000000000001	2.50901803607214\\
4.00000000000001	2.49298597194389\\
4.00000000000001	2.47695390781563\\
4.00000000000002	2.46092184368737\\
4.00000000000002	2.44488977955912\\
4.00000000000002	2.42885771543086\\
4.00000000000002	2.41282565130261\\
4.00000000000002	2.39679358717435\\
4.00000000000002	2.38076152304609\\
4.00000000000002	2.36472945891784\\
4.00000000000002	2.34869739478958\\
4.00000000000002	2.33266533066132\\
4.00000000000002	2.31663326653307\\
4.00000000000003	2.30060120240481\\
4.00000000000003	2.28456913827655\\
4.00000000000003	2.2685370741483\\
4.00000000000003	2.25250501002004\\
4.00000000000003	2.23647294589178\\
4.00000000000003	2.22044088176353\\
4.00000000000004	2.20440881763527\\
4.00000000000004	2.18837675350701\\
4.00000000000004	2.17234468937876\\
4.00000000000004	2.1563126252505\\
4.00000000000004	2.14028056112224\\
4.00000000000005	2.12424849699399\\
4.00000000000005	2.10821643286573\\
4.00000000000005	2.09218436873747\\
4.00000000000005	2.07615230460922\\
4.00000000000006	2.06012024048096\\
4.00000000000006	2.04408817635271\\
4.00000000000006	2.02805611222445\\
4.00000000000006	2.01202404809619\\
4.00000000000007	1.99599198396794\\
4.00000000000007	1.97995991983968\\
4.00000000000007	1.96392785571142\\
4.00000000000008	1.94789579158317\\
4.00000000000008	1.93186372745491\\
4.00000000000009	1.91583166332665\\
4.00000000000009	1.8997995991984\\
4.00000000000009	1.88376753507014\\
4.0000000000001	1.86773547094188\\
4.0000000000001	1.85170340681363\\
4.00000000000011	1.83567134268537\\
4.00000000000011	1.81963927855711\\
4.00000000000012	1.80360721442886\\
4.00000000000012	1.7875751503006\\
4.00000000000013	1.77154308617235\\
4.00000000000014	1.75551102204409\\
4.00000000000014	1.73947895791583\\
4.00000000000015	1.72344689378758\\
4.00000000000016	1.70741482965932\\
4.00000000000016	1.69138276553106\\
4.00000000000017	1.67535070140281\\
4.00000000000018	1.65931863727455\\
4.00000000000019	1.64328657314629\\
4.0000000000002	1.62725450901804\\
4.00000000000021	1.61122244488978\\
4.00000000000021	1.59519038076152\\
4.00000000000022	1.57915831663327\\
4.00000000000023	1.56312625250501\\
4.00000000000025	1.54709418837675\\
4.00000000000026	1.5310621242485\\
4.00000000000027	1.51503006012024\\
4.00000000000028	1.49899799599198\\
4.00000000000029	1.48296593186373\\
4.00000000000031	1.46693386773547\\
4.00000000000032	1.45090180360721\\
4.00000000000033	1.43486973947896\\
4.00000000000035	1.4188376753507\\
4.00000000000036	1.40280561122244\\
4.00000000000038	1.38677354709419\\
4.0000000000004	1.37074148296593\\
4.00000000000041	1.35470941883768\\
4.00000000000043	1.33867735470942\\
4.00000000000045	1.32264529058116\\
4.00000000000047	1.30661322645291\\
4.00000000000049	1.29058116232465\\
4.00000000000051	1.27454909819639\\
4.00000000000053	1.25851703406814\\
4.00000000000056	1.24248496993988\\
4.00000000000058	1.22645290581162\\
4.00000000000061	1.21042084168337\\
4.00000000000063	1.19438877755511\\
4.00000000000066	1.17835671342685\\
4.00000000000069	1.1623246492986\\
4.00000000000072	1.14629258517034\\
4.00000000000075	1.13026052104208\\
4.00000000000078	1.11422845691383\\
4.00000000000081	1.09819639278557\\
4.00000000000084	1.08216432865731\\
4.00000000000088	1.06613226452906\\
4.00000000000091	1.0501002004008\\
4.00000000000095	1.03406813627254\\
4.00000000000099	1.01803607214429\\
4.00000000000103	1.00200400801603\\
4.00000000000108	0.985971943887775\\
4.00000000000112	0.969939879759519\\
4.00000000000116	0.953907815631262\\
4.00000000000121	0.937875751503006\\
4.00000000000126	0.921843687374749\\
4.00000000000131	0.905811623246493\\
4.00000000000136	0.889779559118236\\
4.00000000000142	0.87374749498998\\
4.00000000000147	0.857715430861724\\
4.00000000000153	0.841683366733467\\
4.00000000000159	0.825651302605211\\
4.00000000000166	0.809619238476954\\
4.00000000000172	0.793587174348698\\
4.00000000000179	0.777555110220441\\
4.00000000000186	0.761523046092185\\
4.00000000000193	0.745490981963928\\
4.000000000002	0.729458917835672\\
4.00000000000208	0.713426853707415\\
4.00000000000216	0.697394789579159\\
4.00000000000225	0.681362725450902\\
4.00000000000233	0.665330661322646\\
4.00000000000242	0.649298597194389\\
4.00000000000251	0.633266533066132\\
4.00000000000261	0.617234468937876\\
4.0000000000027	0.601202404809619\\
4.00000000000281	0.585170340681363\\
4.00000000000291	0.569138276553106\\
4.00000000000302	0.55310621242485\\
4.00000000000313	0.537074148296593\\
4.00000000000325	0.521042084168337\\
4.00000000000337	0.50501002004008\\
4.00000000000349	0.488977955911824\\
4.00000000000362	0.472945891783567\\
4.00000000000375	0.456913827655311\\
4.00000000000388	0.440881763527054\\
4.00000000000403	0.424849699398798\\
4.00000000000417	0.408817635270541\\
4.00000000000432	0.392785571142285\\
4.00000000000448	0.376753507014028\\
4.00000000000464	0.360721442885771\\
4.0000000000048	0.344689378757515\\
4.00000000000497	0.328657314629258\\
4.00000000000514	0.312625250501002\\
4.00000000000532	0.296593186372745\\
4.00000000000551	0.280561122244489\\
4.0000000000057	0.264529058116232\\
4.0000000000059	0.248496993987976\\
4.00000000000611	0.232464929859719\\
4.00000000000632	0.216432865731463\\
4.00000000000653	0.200400801603206\\
4.00000000000676	0.18436873747495\\
4.00000000000699	0.168336673346693\\
4.00000000000722	0.152304609218437\\
4.00000000000747	0.13627254509018\\
4.00000000000772	0.120240480961924\\
4.00000000000798	0.104208416833667\\
4.00000000000824	0.0881763527054105\\
4.00000000000852	0.0721442885771539\\
4.0000000000088	0.0561122244488974\\
4.00000000000909	0.0400801603206409\\
4.00000000000939	0.0240480961923843\\
4.00000000000969	0.00801603206412782\\
4.00000000001001	-0.00801603206412826\\
4.00000000001033	-0.0240480961923848\\
4.00000000001067	-0.0400801603206413\\
4.00000000001101	-0.0561122244488979\\
4.00000000001137	-0.0721442885771544\\
4.00000000001173	-0.0881763527054109\\
4.0000000000121	-0.104208416833667\\
4.00000000001248	-0.120240480961924\\
4.00000000001288	-0.13627254509018\\
4.00000000001328	-0.152304609218437\\
4.0000000000137	-0.168336673346694\\
4.00000000001412	-0.18436873747495\\
4.00000000001456	-0.200400801603207\\
4.00000000001501	-0.216432865731463\\
4.00000000001547	-0.232464929859719\\
4.00000000001594	-0.248496993987976\\
4.00000000001643	-0.264529058116232\\
4.00000000001693	-0.280561122244489\\
4.00000000001744	-0.296593186372745\\
4.00000000001796	-0.312625250501002\\
4.0000000000185	-0.328657314629258\\
4.00000000001905	-0.344689378757515\\
4.00000000001962	-0.360721442885771\\
4.0000000000202	-0.376753507014028\\
4.00000000002079	-0.392785571142285\\
4.0000000000214	-0.408817635270541\\
4.00000000002202	-0.424849699398798\\
4.00000000002266	-0.440881763527054\\
4.00000000002332	-0.456913827655311\\
4.00000000002399	-0.472945891783567\\
4.00000000002467	-0.488977955911824\\
4.00000000002538	-0.50501002004008\\
4.0000000000261	-0.521042084168337\\
4.00000000002683	-0.537074148296593\\
4.00000000002759	-0.55310621242485\\
4.00000000002836	-0.569138276553106\\
4.00000000002914	-0.585170340681363\\
4.00000000002995	-0.601202404809619\\
4.00000000003078	-0.617234468937876\\
4.00000000003162	-0.633266533066132\\
4.00000000003248	-0.649298597194389\\
4.00000000003337	-0.665330661322646\\
4.00000000003427	-0.681362725450902\\
4.00000000003519	-0.697394789579158\\
4.00000000003613	-0.713426853707415\\
4.00000000003709	-0.729458917835671\\
4.00000000003808	-0.745490981963928\\
4.00000000003908	-0.761523046092184\\
4.00000000004011	-0.777555110220441\\
4.00000000004115	-0.793587174348697\\
4.00000000004222	-0.809619238476954\\
4.00000000004331	-0.82565130260521\\
4.00000000004442	-0.841683366733467\\
4.00000000004556	-0.857715430861723\\
4.00000000004672	-0.87374749498998\\
4.0000000000479	-0.889779559118236\\
4.00000000004911	-0.905811623246493\\
4.00000000005034	-0.92184368737475\\
4.00000000005159	-0.937875751503006\\
4.00000000005287	-0.953907815631263\\
4.00000000005418	-0.969939879759519\\
4.0000000000555	-0.985971943887776\\
4.00000000005686	-1.00200400801603\\
4.00000000005824	-1.01803607214429\\
4.00000000005964	-1.03406813627255\\
4.00000000006107	-1.0501002004008\\
4.00000000006253	-1.06613226452906\\
4.00000000006401	-1.08216432865731\\
4.00000000006552	-1.09819639278557\\
4.00000000006706	-1.11422845691383\\
4.00000000006863	-1.13026052104208\\
4.00000000007022	-1.14629258517034\\
4.00000000007184	-1.1623246492986\\
4.00000000007349	-1.17835671342685\\
4.00000000007516	-1.19438877755511\\
4.00000000007687	-1.21042084168337\\
4.0000000000786	-1.22645290581162\\
4.00000000008036	-1.24248496993988\\
4.00000000008216	-1.25851703406814\\
4.00000000008398	-1.27454909819639\\
4.00000000008583	-1.29058116232465\\
4.0000000000877	-1.30661322645291\\
4.00000000008961	-1.32264529058116\\
4.00000000009155	-1.33867735470942\\
4.00000000009352	-1.35470941883768\\
4.00000000009552	-1.37074148296593\\
4.00000000009755	-1.38677354709419\\
4.0000000000996	-1.40280561122244\\
4.00000000010169	-1.4188376753507\\
4.00000000010381	-1.43486973947896\\
4.00000000010596	-1.45090180360721\\
4.00000000010814	-1.46693386773547\\
4.00000000011035	-1.48296593186373\\
4.0000000001126	-1.49899799599198\\
4.00000000011487	-1.51503006012024\\
4.00000000011717	-1.5310621242485\\
4.00000000011951	-1.54709418837675\\
4.00000000012187	-1.56312625250501\\
4.00000000012427	-1.57915831663327\\
4.00000000012669	-1.59519038076152\\
4.00000000012915	-1.61122244488978\\
4.00000000013164	-1.62725450901804\\
4.00000000013416	-1.64328657314629\\
4.00000000013671	-1.65931863727455\\
4.00000000013929	-1.67535070140281\\
4.0000000001419	-1.69138276553106\\
4.00000000014454	-1.70741482965932\\
4.00000000014721	-1.72344689378758\\
4.00000000014991	-1.73947895791583\\
4.00000000015265	-1.75551102204409\\
4.00000000015541	-1.77154308617234\\
4.0000000001582	-1.7875751503006\\
4.00000000016102	-1.80360721442886\\
4.00000000016387	-1.81963927855711\\
4.00000000016675	-1.83567134268537\\
4.00000000016965	-1.85170340681363\\
4.00000000017259	-1.86773547094188\\
4.00000000017555	-1.88376753507014\\
4.00000000017854	-1.8997995991984\\
4.00000000018156	-1.91583166332665\\
4.00000000018461	-1.93186372745491\\
4.00000000018768	-1.94789579158317\\
4.00000000019079	-1.96392785571142\\
4.00000000019391	-1.97995991983968\\
4.00000000019707	-1.99599198396794\\
4.00000000020024	-2.01202404809619\\
4.00000000020345	-2.02805611222445\\
4.00000000020668	-2.04408817635271\\
4.00000000020993	-2.06012024048096\\
4.0000000002132	-2.07615230460922\\
4.0000000002165	-2.09218436873747\\
4.00000000021983	-2.10821643286573\\
4.00000000022317	-2.12424849699399\\
4.00000000022654	-2.14028056112224\\
4.00000000022992	-2.1563126252505\\
4.00000000023333	-2.17234468937876\\
4.00000000023676	-2.18837675350701\\
4.00000000024021	-2.20440881763527\\
4.00000000024368	-2.22044088176353\\
4.00000000024716	-2.23647294589178\\
4.00000000025066	-2.25250501002004\\
4.00000000025418	-2.2685370741483\\
4.00000000025772	-2.28456913827655\\
4.00000000026127	-2.30060120240481\\
4.00000000026484	-2.31663326653307\\
4.00000000026842	-2.33266533066132\\
4.00000000027201	-2.34869739478958\\
4.00000000027562	-2.36472945891784\\
4.00000000027924	-2.38076152304609\\
4.00000000028287	-2.39679358717435\\
4.00000000028651	-2.41282565130261\\
4.00000000029016	-2.42885771543086\\
4.00000000029382	-2.44488977955912\\
4.00000000029748	-2.46092184368737\\
4.00000000030116	-2.47695390781563\\
4.00000000030484	-2.49298597194389\\
4.00000000030852	-2.50901803607214\\
4.00000000031221	-2.5250501002004\\
4.0000000003159	-2.54108216432866\\
4.0000000003196	-2.55711422845691\\
4.0000000003233	-2.57314629258517\\
4.00000000032699	-2.58917835671343\\
4.00000000033069	-2.60521042084168\\
4.00000000033439	-2.62124248496994\\
4.00000000033808	-2.6372745490982\\
4.00000000034177	-2.65330661322645\\
4.00000000034546	-2.66933867735471\\
4.00000000034914	-2.68537074148297\\
4.00000000035282	-2.70140280561122\\
4.00000000035649	-2.71743486973948\\
4.00000000036015	-2.73346693386774\\
4.0000000003638	-2.74949899799599\\
4.00000000036744	-2.76553106212425\\
4.00000000037107	-2.7815631262525\\
4.00000000037469	-2.79759519038076\\
4.0000000003783	-2.81362725450902\\
4.00000000038189	-2.82965931863727\\
4.00000000038546	-2.84569138276553\\
4.00000000038902	-2.86172344689379\\
4.00000000039256	-2.87775551102204\\
4.00000000039608	-2.8937875751503\\
4.00000000039958	-2.90981963927856\\
4.00000000040307	-2.92585170340681\\
4.00000000040652	-2.94188376753507\\
4.00000000040996	-2.95791583166333\\
4.00000000041337	-2.97394789579158\\
4.00000000041676	-2.98997995991984\\
4.00000000042012	-3.0060120240481\\
4.00000000042346	-3.02204408817635\\
4.00000000042676	-3.03807615230461\\
4.00000000043004	-3.05410821643287\\
4.00000000043328	-3.07014028056112\\
4.0000000004365	-3.08617234468938\\
4.00000000043968	-3.10220440881764\\
4.00000000044282	-3.11823647294589\\
4.00000000044594	-3.13426853707415\\
4.00000000044901	-3.1503006012024\\
4.00000000045205	-3.16633266533066\\
4.00000000045505	-3.18236472945892\\
4.00000000045802	-3.19839679358717\\
4.00000000046094	-3.21442885771543\\
4.00000000046382	-3.23046092184369\\
4.00000000046666	-3.24649298597194\\
4.00000000046946	-3.2625250501002\\
4.00000000047221	-3.27855711422846\\
4.00000000047492	-3.29458917835671\\
4.00000000047758	-3.31062124248497\\
4.0000000004802	-3.32665330661323\\
4.00000000048277	-3.34268537074148\\
4.00000000048529	-3.35871743486974\\
4.00000000048776	-3.374749498998\\
4.00000000049018	-3.39078156312625\\
4.00000000049254	-3.40681362725451\\
4.00000000049486	-3.42284569138277\\
4.00000000049712	-3.43887775551102\\
4.00000000049933	-3.45490981963928\\
4.00000000050148	-3.47094188376753\\
4.00000000050358	-3.48697394789579\\
4.00000000050563	-3.50300601202405\\
4.00000000050761	-3.5190380761523\\
4.00000000050954	-3.53507014028056\\
4.00000000051141	-3.55110220440882\\
4.00000000051322	-3.56713426853707\\
4.00000000051497	-3.58316633266533\\
4.00000000051666	-3.59919839679359\\
4.00000000051829	-3.61523046092184\\
4.00000000051986	-3.6312625250501\\
4.00000000052136	-3.64729458917836\\
4.00000000052281	-3.66332665330661\\
4.00000000052419	-3.67935871743487\\
4.0000000005255	-3.69539078156313\\
4.00000000052675	-3.71142284569138\\
4.00000000052794	-3.72745490981964\\
4.00000000052906	-3.7434869739479\\
4.00000000053011	-3.75951903807615\\
4.0000000005311	-3.77555110220441\\
4.00000000053202	-3.79158316633267\\
4.00000000053288	-3.80761523046092\\
4.00000000053367	-3.82364729458918\\
4.00000000053439	-3.83967935871743\\
4.00000000053504	-3.85571142284569\\
4.00000000053563	-3.87174348697395\\
4.00000000053614	-3.8877755511022\\
4.00000000053659	-3.90380761523046\\
4.00000000053697	-3.91983967935872\\
4.00000000053728	-3.93587174348697\\
4.00000000053752	-3.95190380761523\\
4.00000000053769	-3.96793587174349\\
4.0000000005378	-3.98396793587174\\
4.00000000053783	-4\\
4	-4.00000000053783\\
3.98396793587174	-4.00000000057334\\
3.96793587174349	-4.00000000061096\\
3.95190380761523	-4.0000000006508\\
3.93587174348697	-4.00000000069296\\
3.91983967935872	-4.00000000073758\\
3.90380761523046	-4.00000000078476\\
3.8877755511022	-4.00000000083464\\
3.87174348697395	-4.00000000088735\\
3.85571142284569	-4.00000000094302\\
3.83967935871743	-4.0000000010018\\
3.82364729458918	-4.00000000106383\\
3.80761523046092	-4.00000000112927\\
3.79158316633267	-4.00000000119827\\
3.77555110220441	-4.000000001271\\
3.75951903807615	-4.00000000134762\\
3.7434869739479	-4.00000000142832\\
3.72745490981964	-4.00000000151325\\
3.71142284569138	-4.00000000160263\\
3.69539078156313	-4.00000000169662\\
3.67935871743487	-4.00000000179544\\
3.66332665330661	-4.00000000189928\\
3.64729458917836	-4.00000000200835\\
3.6312625250501	-4.00000000212287\\
3.61523046092184	-4.00000000224305\\
3.59919839679359	-4.00000000236912\\
3.58316633266533	-4.00000000250131\\
3.56713426853707	-4.00000000263986\\
3.55110220440882	-4.00000000278501\\
3.53507014028056	-4.00000000293701\\
3.51903807615231	-4.00000000309611\\
3.50300601202405	-4.00000000326257\\
3.48697394789579	-4.00000000343666\\
3.47094188376754	-4.00000000361864\\
3.45490981963928	-4.00000000380879\\
3.43887775551102	-4.00000000400738\\
3.42284569138277	-4.00000000421471\\
3.40681362725451	-4.00000000443105\\
3.39078156312625	-4.0000000046567\\
3.374749498998	-4.00000000489195\\
3.35871743486974	-4.00000000513711\\
3.34268537074148	-4.00000000539247\\
3.32665330661323	-4.00000000565835\\
3.31062124248497	-4.00000000593505\\
3.29458917835671	-4.00000000622288\\
3.27855711422846	-4.00000000652215\\
3.2625250501002	-4.00000000683318\\
3.24649298597194	-4.00000000715629\\
3.23046092184369	-4.00000000749178\\
3.21442885771543	-4.00000000783997\\
3.19839679358717	-4.00000000820119\\
3.18236472945892	-4.00000000857575\\
3.16633266533066	-4.00000000896395\\
3.1503006012024	-4.00000000936611\\
3.13426853707415	-4.00000000978255\\
3.11823647294589	-4.00000001021356\\
3.10220440881764	-4.00000001065945\\
3.08617234468938	-4.00000001112052\\
3.07014028056112	-4.00000001159706\\
3.05410821643287	-4.00000001208936\\
3.03807615230461	-4.0000000125977\\
3.02204408817635	-4.00000001312236\\
3.0060120240481	-4.0000000136636\\
2.98997995991984	-4.00000001422167\\
2.97394789579158	-4.00000001479684\\
2.95791583166333	-4.00000001538933\\
2.94188376753507	-4.00000001599937\\
2.92585170340681	-4.00000001662719\\
2.90981963927856	-4.00000001727298\\
2.8937875751503	-4.00000001793694\\
2.87775551102204	-4.00000001861924\\
2.86172344689379	-4.00000001932004\\
2.84569138276553	-4.00000002003949\\
2.82965931863727	-4.00000002077772\\
2.81362725450902	-4.00000002153485\\
2.79759519038076	-4.00000002231096\\
2.7815631262525	-4.00000002310613\\
2.76553106212425	-4.00000002392041\\
2.74949899799599	-4.00000002475385\\
2.73346693386774	-4.00000002560645\\
2.71743486973948	-4.0000000264782\\
2.70140280561122	-4.00000002736909\\
2.68537074148297	-4.00000002827904\\
2.66933867735471	-4.00000002920798\\
2.65330661322645	-4.0000000301558\\
2.6372745490982	-4.00000003112239\\
2.62124248496994	-4.00000003210757\\
2.60521042084168	-4.00000003311117\\
2.58917835671343	-4.00000003413298\\
2.57314629258517	-4.00000003517276\\
2.55711422845691	-4.00000003623025\\
2.54108216432866	-4.00000003730514\\
2.5250501002004	-4.00000003839712\\
2.50901803607214	-4.00000003950582\\
2.49298597194389	-4.00000004063087\\
2.47695390781563	-4.00000004177185\\
2.46092184368737	-4.00000004292832\\
2.44488977955912	-4.0000000440998\\
2.42885771543086	-4.00000004528579\\
2.41282565130261	-4.00000004648575\\
2.39679358717435	-4.0000000476991\\
2.38076152304609	-4.00000004892527\\
2.36472945891784	-4.0000000501636\\
2.34869739478958	-4.00000005141346\\
2.33266533066132	-4.00000005267415\\
2.31663326653307	-4.00000005394494\\
2.30060120240481	-4.0000000552251\\
2.28456913827655	-4.00000005651385\\
2.2685370741483	-4.00000005781037\\
2.25250501002004	-4.00000005911385\\
2.23647294589178	-4.00000006042342\\
2.22044088176353	-4.00000006173819\\
2.20440881763527	-4.00000006305725\\
2.18837675350701	-4.00000006437967\\
2.17234468937876	-4.00000006570449\\
2.1563126252505	-4.00000006703072\\
2.14028056112224	-4.00000006835736\\
2.12424849699399	-4.00000006968339\\
2.10821643286573	-4.00000007100775\\
2.09218436873747	-4.0000000723294\\
2.07615230460922	-4.00000007364725\\
2.06012024048096	-4.0000000749602\\
2.04408817635271	-4.00000007626715\\
2.02805611222445	-4.00000007756697\\
2.01202404809619	-4.00000007885854\\
1.99599198396794	-4.00000008014071\\
1.97995991983968	-4.00000008141234\\
1.96392785571142	-4.00000008267226\\
1.94789579158317	-4.00000008391932\\
1.93186372745491	-4.00000008515236\\
1.91583166332665	-4.0000000863702\\
1.8997995991984	-4.0000000875717\\
1.88376753507014	-4.00000008875568\\
1.86773547094188	-4.000000089921\\
1.85170340681363	-4.0000000910665\\
1.83567134268537	-4.00000009219105\\
1.81963927855711	-4.0000000932935\\
1.80360721442886	-4.00000009437275\\
1.7875751503006	-4.00000009542768\\
1.77154308617235	-4.00000009645721\\
1.75551102204409	-4.00000009746027\\
1.73947895791583	-4.0000000984358\\
1.72344689378758	-4.00000009938277\\
1.70741482965932	-4.00000010030017\\
1.69138276553106	-4.00000010118702\\
1.67535070140281	-4.00000010204237\\
1.65931863727455	-4.00000010286528\\
1.64328657314629	-4.00000010365485\\
1.62725450901804	-4.00000010441023\\
1.61122244488978	-4.00000010513056\\
1.59519038076152	-4.00000010581507\\
1.57915831663327	-4.00000010646298\\
1.56312625250501	-4.00000010707356\\
1.54709418837675	-4.00000010764614\\
1.5310621242485	-4.00000010818007\\
1.51503006012024	-4.00000010867473\\
1.49899799599198	-4.00000010912958\\
1.48296593186373	-4.00000010954409\\
1.46693386773547	-4.00000010991779\\
1.45090180360721	-4.00000011025025\\
1.43486973947896	-4.00000011054108\\
1.4188376753507	-4.00000011078997\\
1.40280561122244	-4.00000011099661\\
1.38677354709419	-4.00000011116077\\
1.37074148296593	-4.00000011128227\\
1.35470941883768	-4.00000011136095\\
1.33867735470942	-4.00000011139673\\
1.32264529058116	-4.00000011138958\\
1.30661322645291	-4.00000011133948\\
1.29058116232465	-4.00000011124652\\
1.27454909819639	-4.00000011111078\\
1.25851703406814	-4.00000011093244\\
1.24248496993988	-4.00000011071169\\
1.22645290581162	-4.00000011044878\\
1.21042084168337	-4.00000011014403\\
1.19438877755511	-4.00000010979778\\
1.17835671342685	-4.00000010941043\\
1.1623246492986	-4.00000010898242\\
1.14629258517034	-4.00000010851424\\
1.13026052104208	-4.00000010800642\\
1.11422845691383	-4.00000010745955\\
1.09819639278557	-4.00000010687422\\
1.08216432865731	-4.00000010625112\\
1.06613226452906	-4.00000010559093\\
1.0501002004008	-4.00000010489439\\
1.03406813627254	-4.00000010416229\\
1.01803607214429	-4.00000010339542\\
1.00200400801603	-4.00000010259463\\
0.985971943887775	-4.00000010176081\\
0.969939879759519	-4.00000010089486\\
0.953907815631262	-4.00000009999772\\
0.937875751503006	-4.00000009907035\\
0.921843687374749	-4.00000009811374\\
0.905811623246493	-4.00000009712892\\
0.889779559118236	-4.00000009611692\\
0.87374749498998	-4.00000009507881\\
0.857715430861724	-4.00000009401565\\
0.841683366733467	-4.00000009292854\\
0.825651302605211	-4.0000000918186\\
0.809619238476954	-4.00000009068694\\
0.793587174348698	-4.00000008953471\\
0.777555110220441	-4.00000008836304\\
0.761523046092185	-4.00000008717309\\
0.745490981963928	-4.00000008596601\\
0.729458917835672	-4.00000008474298\\
0.713426853707415	-4.00000008350514\\
0.697394789579159	-4.00000008225366\\
0.681362725450902	-4.00000008098971\\
0.665330661322646	-4.00000007971444\\
0.649298597194389	-4.00000007842901\\
0.633266533066132	-4.00000007713456\\
0.617234468937876	-4.00000007583223\\
0.601202404809619	-4.00000007452316\\
0.585170340681363	-4.00000007320845\\
0.569138276553106	-4.00000007188922\\
0.55310621242485	-4.00000007056655\\
0.537074148296593	-4.00000006924151\\
0.521042084168337	-4.00000006791516\\
0.50501002004008	-4.00000006658855\\
0.488977955911824	-4.00000006526268\\
0.472945891783567	-4.00000006393855\\
0.456913827655311	-4.00000006261714\\
0.440881763527054	-4.00000006129941\\
0.424849699398798	-4.00000005998627\\
0.408817635270541	-4.00000005867864\\
0.392785571142285	-4.00000005737738\\
0.376753507014028	-4.00000005608336\\
0.360721442885771	-4.00000005479739\\
0.344689378757515	-4.00000005352027\\
0.328657314629258	-4.00000005225276\\
0.312625250501002	-4.00000005099561\\
0.296593186372745	-4.00000004974951\\
0.280561122244489	-4.00000004851516\\
0.264529058116232	-4.0000000472932\\
0.248496993987976	-4.00000004608424\\
0.232464929859719	-4.00000004488888\\
0.216432865731463	-4.00000004370767\\
0.200400801603206	-4.00000004254114\\
0.18436873747495	-4.00000004138978\\
0.168336673346693	-4.00000004025406\\
0.152304609218437	-4.00000003913442\\
0.13627254509018	-4.00000003803125\\
0.120240480961924	-4.00000003694493\\
0.104208416833667	-4.0000000358758\\
0.0881763527054105	-4.00000003482419\\
0.0721442885771539	-4.00000003379037\\
0.0561122244488974	-4.00000003277461\\
0.0400801603206409	-4.00000003177712\\
0.0240480961923843	-4.00000003079812\\
0.00801603206412782	-4.00000002983777\\
-0.00801603206412826	-4.00000002889623\\
-0.0240480961923848	-4.0000000279736\\
-0.0400801603206413	-4.00000002707\\
-0.0561122244488979	-4.00000002618549\\
-0.0721442885771544	-4.00000002532012\\
-0.0881763527054109	-4.00000002447391\\
-0.104208416833667	-4.00000002364686\\
-0.120240480961924	-4.00000002283895\\
-0.13627254509018	-4.00000002205014\\
-0.152304609218437	-4.00000002128037\\
-0.168336673346694	-4.00000002052955\\
-0.18436873747495	-4.00000001979759\\
-0.200400801603207	-4.00000001908437\\
-0.216432865731463	-4.00000001838976\\
-0.232464929859719	-4.00000001771359\\
-0.248496993987976	-4.00000001705571\\
-0.264529058116232	-4.00000001641593\\
-0.280561122244489	-4.00000001579406\\
-0.296593186372745	-4.00000001518989\\
-0.312625250501002	-4.0000000146032\\
-0.328657314629258	-4.00000001403376\\
-0.344689378757515	-4.00000001348132\\
-0.360721442885771	-4.00000001294564\\
-0.376753507014028	-4.00000001242646\\
-0.392785571142285	-4.00000001192349\\
-0.408817635270541	-4.00000001143648\\
-0.424849699398798	-4.00000001096513\\
-0.440881763527054	-4.00000001050915\\
-0.456913827655311	-4.00000001006825\\
-0.472945891783567	-4.00000000964213\\
-0.488977955911824	-4.00000000923049\\
-0.50501002004008	-4.00000000883301\\
-0.521042084168337	-4.00000000844939\\
-0.537074148296593	-4.00000000807932\\
-0.55310621242485	-4.00000000772248\\
-0.569138276553106	-4.00000000737855\\
-0.585170340681363	-4.00000000704722\\
-0.601202404809619	-4.00000000672818\\
-0.617234468937876	-4.0000000064211\\
-0.633266533066132	-4.00000000612568\\
-0.649298597194389	-4.0000000058416\\
-0.665330661322646	-4.00000000556854\\
-0.681362725450902	-4.0000000053062\\
-0.697394789579158	-4.00000000505427\\
-0.713426853707415	-4.00000000481245\\
-0.729458917835671	-4.00000000458043\\
-0.745490981963928	-4.00000000435791\\
-0.761523046092184	-4.00000000414461\\
-0.777555110220441	-4.00000000394023\\
-0.793587174348697	-4.00000000374448\\
-0.809619238476954	-4.00000000355709\\
-0.82565130260521	-4.00000000337777\\
-0.841683366733467	-4.00000000320625\\
-0.857715430861723	-4.00000000304227\\
-0.87374749498998	-4.00000000288557\\
-0.889779559118236	-4.00000000273588\\
-0.905811623246493	-4.00000000259296\\
-0.92184368737475	-4.00000000245655\\
-0.937875751503006	-4.00000000232643\\
-0.953907815631263	-4.00000000220234\\
-0.969939879759519	-4.00000000208408\\
-0.985971943887776	-4.0000000019714\\
-1.00200400801603	-4.0000000018641\\
-1.01803607214429	-4.00000000176195\\
-1.03406813627255	-4.00000000166477\\
-1.0501002004008	-4.00000000157233\\
-1.06613226452906	-4.00000000148446\\
-1.08216432865731	-4.00000000140096\\
-1.09819639278557	-4.00000000132164\\
-1.11422845691383	-4.00000000124633\\
-1.13026052104208	-4.00000000117487\\
-1.14629258517034	-4.00000000110707\\
-1.1623246492986	-4.00000000104278\\
-1.17835671342685	-4.00000000098185\\
-1.19438877755511	-4.00000000092412\\
-1.21042084168337	-4.00000000086946\\
-1.22645290581162	-4.00000000081771\\
-1.24248496993988	-4.00000000076874\\
-1.25851703406814	-4.00000000072243\\
-1.27454909819639	-4.00000000067864\\
-1.29058116232465	-4.00000000063727\\
-1.30661322645291	-4.00000000059818\\
-1.32264529058116	-4.00000000056128\\
-1.33867735470942	-4.00000000052645\\
-1.35470941883768	-4.00000000049359\\
-1.37074148296593	-4.0000000004626\\
-1.38677354709419	-4.00000000043339\\
-1.40280561122244	-4.00000000040587\\
-1.4188376753507	-4.00000000037995\\
-1.43486973947896	-4.00000000035555\\
-1.45090180360721	-4.00000000033259\\
-1.46693386773547	-4.00000000031099\\
-1.48296593186373	-4.00000000029068\\
-1.49899799599198	-4.00000000027159\\
-1.51503006012024	-4.00000000025366\\
-1.5310621242485	-4.00000000023682\\
-1.54709418837675	-4.00000000022102\\
-1.56312625250501	-4.00000000020618\\
-1.57915831663327	-4.00000000019227\\
-1.59519038076152	-4.00000000017923\\
-1.61122244488978	-4.00000000016701\\
-1.62725450901804	-4.00000000015557\\
-1.64328657314629	-4.00000000014485\\
-1.65931863727455	-4.00000000013482\\
-1.67535070140281	-4.00000000012543\\
-1.69138276553106	-4.00000000011665\\
-1.70741482965932	-4.00000000010845\\
-1.72344689378758	-4.00000000010078\\
-1.73947895791583	-4.00000000009362\\
-1.75551102204409	-4.00000000008693\\
-1.77154308617234	-4.0000000000807\\
-1.7875751503006	-4.00000000007488\\
-1.80360721442886	-4.00000000006945\\
-1.81963927855711	-4.00000000006439\\
-1.83567134268537	-4.00000000005968\\
-1.85170340681363	-4.00000000005529\\
-1.86773547094188	-4.0000000000512\\
-1.88376753507014	-4.0000000000474\\
-1.8997995991984	-4.00000000004386\\
-1.91583166332665	-4.00000000004057\\
-1.93186372745491	-4.00000000003751\\
-1.94789579158317	-4.00000000003468\\
-1.96392785571142	-4.00000000003204\\
-1.97995991983968	-4.00000000002959\\
-1.99599198396794	-4.00000000002732\\
-2.01202404809619	-4.00000000002521\\
-2.02805611222445	-4.00000000002326\\
-2.04408817635271	-4.00000000002145\\
-2.06012024048096	-4.00000000001977\\
-2.07615230460922	-4.00000000001822\\
-2.09218436873747	-4.00000000001678\\
-2.10821643286573	-4.00000000001545\\
-2.12424849699399	-4.00000000001422\\
-2.14028056112224	-4.00000000001308\\
-2.1563126252505	-4.00000000001203\\
-2.17234468937876	-4.00000000001106\\
-2.18837675350701	-4.00000000001017\\
-2.20440881763527	-4.00000000000934\\
-2.22044088176353	-4.00000000000858\\
-2.23647294589178	-4.00000000000787\\
-2.25250501002004	-4.00000000000722\\
-2.2685370741483	-4.00000000000662\\
-2.28456913827655	-4.00000000000607\\
-2.30060120240481	-4.00000000000557\\
-2.31663326653307	-4.0000000000051\\
-2.33266533066132	-4.00000000000467\\
-2.34869739478958	-4.00000000000428\\
-2.36472945891784	-4.00000000000391\\
-2.38076152304609	-4.00000000000358\\
-2.39679358717435	-4.00000000000327\\
-2.41282565130261	-4.00000000000299\\
-2.42885771543086	-4.00000000000273\\
-2.44488977955912	-4.0000000000025\\
-2.46092184368737	-4.00000000000228\\
-2.47695390781563	-4.00000000000208\\
-2.49298597194389	-4.0000000000019\\
-2.50901803607214	-4.00000000000173\\
-2.5250501002004	-4.00000000000158\\
-2.54108216432866	-4.00000000000144\\
-2.55711422845691	-4.00000000000131\\
-2.57314629258517	-4.00000000000119\\
-2.58917835671343	-4.00000000000108\\
-2.60521042084168	-4.00000000000099\\
-2.62124248496994	-4.0000000000009\\
-2.6372745490982	-4.00000000000082\\
-2.65330661322645	-4.00000000000074\\
-2.66933867735471	-4.00000000000067\\
-2.68537074148297	-4.00000000000061\\
-2.70140280561122	-4.00000000000056\\
-2.71743486973948	-4.0000000000005\\
-2.73346693386774	-4.00000000000046\\
-2.74949899799599	-4.00000000000041\\
-2.76553106212425	-4.00000000000038\\
-2.7815631262525	-4.00000000000034\\
-2.79759519038076	-4.00000000000031\\
-2.81362725450902	-4.00000000000028\\
-2.82965931863727	-4.00000000000025\\
-2.84569138276553	-4.00000000000023\\
-2.86172344689379	-4.00000000000021\\
-2.87775551102204	-4.00000000000019\\
-2.8937875751503	-4.00000000000017\\
-2.90981963927856	-4.00000000000015\\
-2.92585170340681	-4.00000000000014\\
-2.94188376753507	-4.00000000000012\\
-2.95791583166333	-4.00000000000011\\
-2.97394789579158	-4.0000000000001\\
-2.98997995991984	-4.00000000000009\\
-3.0060120240481	-4.00000000000008\\
-3.02204408817635	-4.00000000000007\\
-3.03807615230461	-4.00000000000007\\
-3.05410821643287	-4.00000000000006\\
-3.07014028056112	-4.00000000000005\\
-3.08617234468938	-4.00000000000005\\
-3.10220440881764	-4.00000000000004\\
-3.11823647294589	-4.00000000000004\\
-3.13426853707415	-4.00000000000003\\
-3.1503006012024	-4.00000000000003\\
-3.16633266533066	-4.00000000000003\\
-3.18236472945892	-4.00000000000003\\
-3.19839679358717	-4.00000000000002\\
-3.21442885771543	-4.00000000000002\\
-3.23046092184369	-4.00000000000002\\
-3.24649298597194	-4.00000000000002\\
-3.2625250501002	-4.00000000000001\\
-3.27855711422846	-4.00000000000001\\
-3.29458917835671	-4.00000000000001\\
-3.31062124248497	-4.00000000000001\\
-3.32665330661323	-4.00000000000001\\
-3.34268537074148	-4.00000000000001\\
-3.35871743486974	-4.00000000000001\\
-3.374749498998	-4.00000000000001\\
-3.39078156312625	-4.00000000000001\\
-3.40681362725451	-4.00000000000001\\
-3.42284569138277	-4\\
-3.43887775551102	-4\\
-3.45490981963928	-4\\
-3.47094188376753	-4\\
-3.48697394789579	-4\\
-3.50300601202405	-4\\
-3.5190380761523	-4\\
-3.53507014028056	-4\\
-3.55110220440882	-4\\
-3.56713426853707	-4\\
-3.58316633266533	-4\\
-3.59919839679359	-4\\
-3.61523046092184	-4\\
-3.6312625250501	-4\\
-3.64729458917836	-4\\
-3.66332665330661	-4\\
-3.67935871743487	-4\\
-3.69539078156313	-4\\
-3.71142284569138	-4\\
-3.72745490981964	-4\\
-3.7434869739479	-4\\
-3.75951903807615	-4\\
-3.77555110220441	-4\\
-3.79158316633267	-4\\
-3.80761523046092	-4\\
-3.82364729458918	-4\\
-3.83967935871743	-4\\
-3.85571142284569	-4\\
-3.87174348697395	-4\\
-3.8877755511022	-4\\
-3.90380761523046	-4\\
-3.91983967935872	-4\\
-3.93587174348697	-4\\
-3.95190380761523	-4\\
-3.96793587174349	-4\\
-3.98396793587174	-4\\
-4	-4\\
-4	-3.98396793587174\\
-4	-3.96793587174349\\
-4	-3.95190380761523\\
-4	-3.93587174348697\\
-4	-3.91983967935872\\
-4	-3.90380761523046\\
-4	-3.8877755511022\\
-4	-3.87174348697395\\
-4	-3.85571142284569\\
-4	-3.83967935871743\\
-4	-3.82364729458918\\
-4	-3.80761523046092\\
-4	-3.79158316633267\\
-4	-3.77555110220441\\
-4	-3.75951903807615\\
-4	-3.7434869739479\\
-4	-3.72745490981964\\
-4	-3.71142284569138\\
-4	-3.69539078156313\\
-4	-3.67935871743487\\
-4	-3.66332665330661\\
-4	-3.64729458917836\\
-4	-3.6312625250501\\
-4	-3.61523046092184\\
-4	-3.59919839679359\\
-4	-3.58316633266533\\
-4	-3.56713426853707\\
-4	-3.55110220440882\\
-4	-3.53507014028056\\
-4	-3.5190380761523\\
-4	-3.50300601202405\\
-4	-3.48697394789579\\
-4	-3.47094188376753\\
-4	-3.45490981963928\\
-4	-3.43887775551102\\
-4	-3.42284569138277\\
-4	-3.40681362725451\\
-4	-3.39078156312625\\
-4	-3.374749498998\\
-4	-3.35871743486974\\
-4	-3.34268537074148\\
-4	-3.32665330661323\\
-4	-3.31062124248497\\
-4	-3.29458917835671\\
-4	-3.27855711422846\\
-4	-3.2625250501002\\
-4	-3.24649298597194\\
-4	-3.23046092184369\\
-4	-3.21442885771543\\
-4	-3.19839679358717\\
-4	-3.18236472945892\\
-4	-3.16633266533066\\
-4	-3.1503006012024\\
-4	-3.13426853707415\\
-4	-3.11823647294589\\
-4	-3.10220440881764\\
-4	-3.08617234468938\\
-4	-3.07014028056112\\
-4	-3.05410821643287\\
-4	-3.03807615230461\\
-4	-3.02204408817635\\
-4	-3.0060120240481\\
-4	-2.98997995991984\\
-4	-2.97394789579158\\
-4	-2.95791583166333\\
-4	-2.94188376753507\\
-4	-2.92585170340681\\
-4	-2.90981963927856\\
-4	-2.8937875751503\\
-4	-2.87775551102204\\
-4	-2.86172344689379\\
-4	-2.84569138276553\\
-4	-2.82965931863727\\
-4	-2.81362725450902\\
-4.00000000000001	-2.79759519038076\\
-4.00000000000001	-2.7815631262525\\
-4.00000000000001	-2.76553106212425\\
-4.00000000000001	-2.74949899799599\\
-4.00000000000001	-2.73346693386774\\
-4.00000000000001	-2.71743486973948\\
-4.00000000000001	-2.70140280561122\\
-4.00000000000001	-2.68537074148297\\
-4.00000000000001	-2.66933867735471\\
-4.00000000000001	-2.65330661322645\\
-4.00000000000001	-2.6372745490982\\
-4.00000000000001	-2.62124248496994\\
-4.00000000000001	-2.60521042084168\\
-4.00000000000001	-2.58917835671343\\
-4.00000000000001	-2.57314629258517\\
-4.00000000000001	-2.55711422845691\\
-4.00000000000001	-2.54108216432866\\
-4.00000000000001	-2.5250501002004\\
-4.00000000000001	-2.50901803607214\\
-4.00000000000001	-2.49298597194389\\
-4.00000000000002	-2.47695390781563\\
-4.00000000000002	-2.46092184368737\\
-4.00000000000002	-2.44488977955912\\
-4.00000000000002	-2.42885771543086\\
-4.00000000000002	-2.41282565130261\\
-4.00000000000002	-2.39679358717435\\
-4.00000000000002	-2.38076152304609\\
-4.00000000000002	-2.36472945891784\\
-4.00000000000002	-2.34869739478958\\
-4.00000000000002	-2.33266533066132\\
-4.00000000000002	-2.31663326653307\\
-4.00000000000003	-2.30060120240481\\
-4.00000000000003	-2.28456913827655\\
-4.00000000000003	-2.2685370741483\\
-4.00000000000003	-2.25250501002004\\
-4.00000000000003	-2.23647294589178\\
-4.00000000000003	-2.22044088176353\\
-4.00000000000004	-2.20440881763527\\
-4.00000000000004	-2.18837675350701\\
-4.00000000000004	-2.17234468937876\\
-4.00000000000004	-2.1563126252505\\
-4.00000000000004	-2.14028056112224\\
-4.00000000000005	-2.12424849699399\\
-4.00000000000005	-2.10821643286573\\
-4.00000000000005	-2.09218436873747\\
-4.00000000000005	-2.07615230460922\\
-4.00000000000006	-2.06012024048096\\
-4.00000000000006	-2.04408817635271\\
-4.00000000000006	-2.02805611222445\\
-4.00000000000006	-2.01202404809619\\
-4.00000000000007	-1.99599198396794\\
-4.00000000000007	-1.97995991983968\\
-4.00000000000007	-1.96392785571142\\
-4.00000000000008	-1.94789579158317\\
-4.00000000000008	-1.93186372745491\\
-4.00000000000009	-1.91583166332665\\
-4.00000000000009	-1.8997995991984\\
-4.00000000000009	-1.88376753507014\\
-4.0000000000001	-1.86773547094188\\
-4.0000000000001	-1.85170340681363\\
-4.00000000000011	-1.83567134268537\\
-4.00000000000011	-1.81963927855711\\
-4.00000000000012	-1.80360721442886\\
-4.00000000000012	-1.7875751503006\\
-4.00000000000013	-1.77154308617234\\
-4.00000000000014	-1.75551102204409\\
-4.00000000000014	-1.73947895791583\\
-4.00000000000015	-1.72344689378758\\
-4.00000000000016	-1.70741482965932\\
-4.00000000000016	-1.69138276553106\\
-4.00000000000017	-1.67535070140281\\
-4.00000000000018	-1.65931863727455\\
-4.00000000000019	-1.64328657314629\\
-4.0000000000002	-1.62725450901804\\
-4.00000000000021	-1.61122244488978\\
-4.00000000000021	-1.59519038076152\\
-4.00000000000022	-1.57915831663327\\
-4.00000000000023	-1.56312625250501\\
-4.00000000000025	-1.54709418837675\\
-4.00000000000026	-1.5310621242485\\
-4.00000000000027	-1.51503006012024\\
-4.00000000000028	-1.49899799599198\\
-4.00000000000029	-1.48296593186373\\
-4.00000000000031	-1.46693386773547\\
-4.00000000000032	-1.45090180360721\\
-4.00000000000033	-1.43486973947896\\
-4.00000000000035	-1.4188376753507\\
-4.00000000000036	-1.40280561122244\\
-4.00000000000038	-1.38677354709419\\
-4.0000000000004	-1.37074148296593\\
-4.00000000000041	-1.35470941883768\\
-4.00000000000043	-1.33867735470942\\
-4.00000000000045	-1.32264529058116\\
-4.00000000000047	-1.30661322645291\\
-4.00000000000049	-1.29058116232465\\
-4.00000000000051	-1.27454909819639\\
-4.00000000000053	-1.25851703406814\\
-4.00000000000056	-1.24248496993988\\
-4.00000000000058	-1.22645290581162\\
-4.00000000000061	-1.21042084168337\\
-4.00000000000063	-1.19438877755511\\
-4.00000000000066	-1.17835671342685\\
-4.00000000000069	-1.1623246492986\\
-4.00000000000072	-1.14629258517034\\
-4.00000000000075	-1.13026052104208\\
-4.00000000000078	-1.11422845691383\\
-4.00000000000081	-1.09819639278557\\
-4.00000000000084	-1.08216432865731\\
-4.00000000000088	-1.06613226452906\\
-4.00000000000092	-1.0501002004008\\
-4.00000000000095	-1.03406813627255\\
-4.00000000000099	-1.01803607214429\\
-4.00000000000103	-1.00200400801603\\
-4.00000000000108	-0.985971943887776\\
-4.00000000000112	-0.969939879759519\\
-4.00000000000116	-0.953907815631263\\
-4.00000000000121	-0.937875751503006\\
-4.00000000000126	-0.92184368737475\\
-4.00000000000131	-0.905811623246493\\
-4.00000000000136	-0.889779559118236\\
-4.00000000000142	-0.87374749498998\\
-4.00000000000147	-0.857715430861723\\
-4.00000000000153	-0.841683366733467\\
-4.00000000000159	-0.82565130260521\\
-4.00000000000166	-0.809619238476954\\
-4.00000000000172	-0.793587174348697\\
-4.00000000000179	-0.777555110220441\\
-4.00000000000186	-0.761523046092184\\
-4.00000000000193	-0.745490981963928\\
-4.000000000002	-0.729458917835671\\
-4.00000000000208	-0.713426853707415\\
-4.00000000000216	-0.697394789579158\\
-4.00000000000225	-0.681362725450902\\
-4.00000000000233	-0.665330661322646\\
-4.00000000000242	-0.649298597194389\\
-4.00000000000251	-0.633266533066132\\
-4.00000000000261	-0.617234468937876\\
-4.0000000000027	-0.601202404809619\\
-4.00000000000281	-0.585170340681363\\
-4.00000000000291	-0.569138276553106\\
-4.00000000000302	-0.55310621242485\\
-4.00000000000313	-0.537074148296593\\
-4.00000000000325	-0.521042084168337\\
-4.00000000000337	-0.50501002004008\\
-4.00000000000349	-0.488977955911824\\
-4.00000000000362	-0.472945891783567\\
-4.00000000000375	-0.456913827655311\\
-4.00000000000388	-0.440881763527054\\
-4.00000000000403	-0.424849699398798\\
-4.00000000000417	-0.408817635270541\\
-4.00000000000432	-0.392785571142285\\
-4.00000000000448	-0.376753507014028\\
-4.00000000000463	-0.360721442885771\\
-4.0000000000048	-0.344689378757515\\
-4.00000000000497	-0.328657314629258\\
-4.00000000000514	-0.312625250501002\\
-4.00000000000532	-0.296593186372745\\
-4.00000000000551	-0.280561122244489\\
-4.0000000000057	-0.264529058116232\\
-4.0000000000059	-0.248496993987976\\
-4.00000000000611	-0.232464929859719\\
-4.00000000000631	-0.216432865731463\\
-4.00000000000653	-0.200400801603207\\
-4.00000000000676	-0.18436873747495\\
-4.00000000000699	-0.168336673346694\\
-4.00000000000722	-0.152304609218437\\
-4.00000000000747	-0.13627254509018\\
-4.00000000000772	-0.120240480961924\\
-4.00000000000798	-0.104208416833667\\
-4.00000000000824	-0.0881763527054109\\
-4.00000000000852	-0.0721442885771544\\
-4.0000000000088	-0.0561122244488979\\
-4.00000000000909	-0.0400801603206413\\
-4.00000000000939	-0.0240480961923848\\
-4.00000000000969	-0.00801603206412826\\
-4.00000000001001	0.00801603206412782\\
-4.00000000001033	0.0240480961923843\\
-4.00000000001067	0.0400801603206409\\
-4.00000000001101	0.0561122244488974\\
-4.00000000001137	0.0721442885771539\\
-4.00000000001173	0.0881763527054105\\
-4.0000000000121	0.104208416833667\\
-4.00000000001248	0.120240480961924\\
-4.00000000001288	0.13627254509018\\
-4.00000000001328	0.152304609218437\\
-4.0000000000137	0.168336673346693\\
-4.00000000001412	0.18436873747495\\
-4.00000000001456	0.200400801603206\\
-4.00000000001501	0.216432865731463\\
-4.00000000001547	0.232464929859719\\
-4.00000000001594	0.248496993987976\\
-4.00000000001643	0.264529058116232\\
-4.00000000001693	0.280561122244489\\
-4.00000000001744	0.296593186372745\\
-4.00000000001796	0.312625250501002\\
-4.0000000000185	0.328657314629258\\
-4.00000000001905	0.344689378757515\\
-4.00000000001962	0.360721442885771\\
-4.0000000000202	0.376753507014028\\
-4.00000000002079	0.392785571142285\\
-4.0000000000214	0.408817635270541\\
-4.00000000002202	0.424849699398798\\
-4.00000000002266	0.440881763527054\\
-4.00000000002332	0.456913827655311\\
-4.00000000002399	0.472945891783567\\
-4.00000000002467	0.488977955911824\\
-4.00000000002538	0.50501002004008\\
-4.0000000000261	0.521042084168337\\
-4.00000000002683	0.537074148296593\\
-4.00000000002759	0.55310621242485\\
-4.00000000002836	0.569138276553106\\
-4.00000000002914	0.585170340681363\\
-4.00000000002995	0.601202404809619\\
-4.00000000003078	0.617234468937876\\
-4.00000000003162	0.633266533066132\\
-4.00000000003248	0.649298597194389\\
-4.00000000003337	0.665330661322646\\
-4.00000000003427	0.681362725450902\\
-4.00000000003519	0.697394789579159\\
-4.00000000003613	0.713426853707415\\
-4.00000000003709	0.729458917835672\\
-4.00000000003808	0.745490981963928\\
-4.00000000003908	0.761523046092185\\
-4.00000000004011	0.777555110220441\\
-4.00000000004115	0.793587174348698\\
-4.00000000004222	0.809619238476954\\
-4.00000000004331	0.825651302605211\\
-4.00000000004442	0.841683366733467\\
-4.00000000004556	0.857715430861724\\
-4.00000000004672	0.87374749498998\\
-4.0000000000479	0.889779559118236\\
-4.00000000004911	0.905811623246493\\
-4.00000000005034	0.921843687374749\\
-4.00000000005159	0.937875751503006\\
-4.00000000005287	0.953907815631262\\
-4.00000000005418	0.969939879759519\\
-4.0000000000555	0.985971943887775\\
-4.00000000005686	1.00200400801603\\
-4.00000000005824	1.01803607214429\\
-4.00000000005964	1.03406813627254\\
-4.00000000006107	1.0501002004008\\
-4.00000000006253	1.06613226452906\\
-4.00000000006401	1.08216432865731\\
-4.00000000006553	1.09819639278557\\
-4.00000000006706	1.11422845691383\\
-4.00000000006863	1.13026052104208\\
-4.00000000007022	1.14629258517034\\
-4.00000000007184	1.1623246492986\\
-4.00000000007349	1.17835671342685\\
-4.00000000007516	1.19438877755511\\
-4.00000000007687	1.21042084168337\\
-4.0000000000786	1.22645290581162\\
-4.00000000008037	1.24248496993988\\
-4.00000000008216	1.25851703406814\\
-4.00000000008398	1.27454909819639\\
-4.00000000008582	1.29058116232465\\
-4.0000000000877	1.30661322645291\\
-4.00000000008961	1.32264529058116\\
-4.00000000009155	1.33867735470942\\
-4.00000000009352	1.35470941883768\\
-4.00000000009552	1.37074148296593\\
-4.00000000009754	1.38677354709419\\
-4.0000000000996	1.40280561122244\\
-4.00000000010169	1.4188376753507\\
-4.00000000010381	1.43486973947896\\
-4.00000000010596	1.45090180360721\\
-4.00000000010814	1.46693386773547\\
-4.00000000011035	1.48296593186373\\
-4.00000000011259	1.49899799599198\\
-4.00000000011487	1.51503006012024\\
-4.00000000011717	1.5310621242485\\
-4.00000000011951	1.54709418837675\\
-4.00000000012187	1.56312625250501\\
-4.00000000012427	1.57915831663327\\
-4.00000000012669	1.59519038076152\\
-4.00000000012915	1.61122244488978\\
-4.00000000013164	1.62725450901804\\
-4.00000000013416	1.64328657314629\\
-4.00000000013671	1.65931863727455\\
-4.00000000013929	1.67535070140281\\
-4.0000000001419	1.69138276553106\\
-4.00000000014454	1.70741482965932\\
-4.00000000014721	1.72344689378758\\
-4.00000000014991	1.73947895791583\\
-4.00000000015265	1.75551102204409\\
-4.00000000015541	1.77154308617235\\
-4.0000000001582	1.7875751503006\\
-4.00000000016102	1.80360721442886\\
-4.00000000016387	1.81963927855711\\
-4.00000000016675	1.83567134268537\\
-4.00000000016965	1.85170340681363\\
-4.00000000017259	1.86773547094188\\
-4.00000000017555	1.88376753507014\\
-4.00000000017854	1.8997995991984\\
-4.00000000018156	1.91583166332665\\
-4.00000000018461	1.93186372745491\\
-4.00000000018769	1.94789579158317\\
-4.00000000019079	1.96392785571142\\
-4.00000000019391	1.97995991983968\\
-4.00000000019707	1.99599198396794\\
-4.00000000020024	2.01202404809619\\
-4.00000000020345	2.02805611222445\\
-4.00000000020668	2.04408817635271\\
-4.00000000020993	2.06012024048096\\
-4.0000000002132	2.07615230460922\\
-4.0000000002165	2.09218436873747\\
-4.00000000021983	2.10821643286573\\
-4.00000000022317	2.12424849699399\\
-4.00000000022654	2.14028056112224\\
-4.00000000022992	2.1563126252505\\
-4.00000000023333	2.17234468937876\\
-4.00000000023676	2.18837675350701\\
-4.00000000024021	2.20440881763527\\
-4.00000000024368	2.22044088176353\\
-4.00000000024716	2.23647294589178\\
-4.00000000025066	2.25250501002004\\
-4.00000000025418	2.2685370741483\\
-4.00000000025772	2.28456913827655\\
-4.00000000026127	2.30060120240481\\
-4.00000000026484	2.31663326653307\\
-4.00000000026842	2.33266533066132\\
-4.00000000027201	2.34869739478958\\
-4.00000000027562	2.36472945891784\\
-4.00000000027924	2.38076152304609\\
-4.00000000028287	2.39679358717435\\
-4.00000000028651	2.41282565130261\\
-4.00000000029016	2.42885771543086\\
-4.00000000029382	2.44488977955912\\
-4.00000000029748	2.46092184368737\\
-4.00000000030116	2.47695390781563\\
-4.00000000030484	2.49298597194389\\
-4.00000000030852	2.50901803607214\\
-4.00000000031221	2.5250501002004\\
-4.0000000003159	2.54108216432866\\
-4.0000000003196	2.55711422845691\\
-4.0000000003233	2.57314629258517\\
-4.00000000032699	2.58917835671343\\
-4.00000000033069	2.60521042084168\\
-4.00000000033439	2.62124248496994\\
-4.00000000033808	2.6372745490982\\
-4.00000000034177	2.65330661322645\\
-4.00000000034546	2.66933867735471\\
-4.00000000034914	2.68537074148297\\
-4.00000000035282	2.70140280561122\\
-4.00000000035649	2.71743486973948\\
-4.00000000036015	2.73346693386774\\
-4.0000000003638	2.74949899799599\\
-4.00000000036744	2.76553106212425\\
-4.00000000037107	2.7815631262525\\
-4.00000000037469	2.79759519038076\\
-4.0000000003783	2.81362725450902\\
-4.00000000038189	2.82965931863727\\
-4.00000000038546	2.84569138276553\\
-4.00000000038902	2.86172344689379\\
-4.00000000039256	2.87775551102204\\
-4.00000000039608	2.8937875751503\\
-4.00000000039958	2.90981963927856\\
-4.00000000040307	2.92585170340681\\
-4.00000000040653	2.94188376753507\\
-4.00000000040996	2.95791583166333\\
-4.00000000041337	2.97394789579158\\
-4.00000000041676	2.98997995991984\\
-4.00000000042012	3.0060120240481\\
-4.00000000042346	3.02204408817635\\
-4.00000000042676	3.03807615230461\\
-4.00000000043004	3.05410821643287\\
-4.00000000043328	3.07014028056112\\
-4.0000000004365	3.08617234468938\\
-4.00000000043968	3.10220440881764\\
-4.00000000044282	3.11823647294589\\
-4.00000000044594	3.13426853707415\\
-4.00000000044901	3.1503006012024\\
-4.00000000045205	3.16633266533066\\
-4.00000000045505	3.18236472945892\\
-4.00000000045802	3.19839679358717\\
-4.00000000046094	3.21442885771543\\
-4.00000000046382	3.23046092184369\\
-4.00000000046666	3.24649298597194\\
-4.00000000046946	3.2625250501002\\
-4.00000000047221	3.27855711422846\\
-4.00000000047492	3.29458917835671\\
-4.00000000047758	3.31062124248497\\
-4.0000000004802	3.32665330661323\\
-4.00000000048277	3.34268537074148\\
-4.00000000048529	3.35871743486974\\
-4.00000000048776	3.374749498998\\
-4.00000000049017	3.39078156312625\\
-4.00000000049254	3.40681362725451\\
-4.00000000049486	3.42284569138277\\
-4.00000000049712	3.43887775551102\\
-4.00000000049933	3.45490981963928\\
-4.00000000050148	3.47094188376754\\
-4.00000000050358	3.48697394789579\\
-4.00000000050563	3.50300601202405\\
-4.00000000050761	3.51903807615231\\
-4.00000000050954	3.53507014028056\\
-4.00000000051141	3.55110220440882\\
-4.00000000051322	3.56713426853707\\
-4.00000000051497	3.58316633266533\\
-4.00000000051666	3.59919839679359\\
-4.00000000051829	3.61523046092184\\
-4.00000000051986	3.6312625250501\\
-4.00000000052136	3.64729458917836\\
-4.00000000052281	3.66332665330661\\
-4.00000000052419	3.67935871743487\\
-4.0000000005255	3.69539078156313\\
-4.00000000052675	3.71142284569138\\
-4.00000000052794	3.72745490981964\\
-4.00000000052906	3.7434869739479\\
-4.00000000053011	3.75951903807615\\
-4.0000000005311	3.77555110220441\\
-4.00000000053202	3.79158316633267\\
-4.00000000053288	3.80761523046092\\
-4.00000000053367	3.82364729458918\\
-4.00000000053439	3.83967935871743\\
-4.00000000053504	3.85571142284569\\
-4.00000000053563	3.87174348697395\\
-4.00000000053614	3.8877755511022\\
-4.00000000053659	3.90380761523046\\
-4.00000000053697	3.91983967935872\\
-4.00000000053728	3.93587174348697\\
-4.00000000053752	3.95190380761523\\
-4.0000000005377	3.96793587174349\\
-4.0000000005378	3.98396793587174\\
-4.00000000053783	4\\
-4	4.00000000053783\\
}--cycle;


\addplot[area legend,solid,fill=mycolor2,draw=black,forget plot]
table[row sep=crcr] {%
x	y\\
-2.18837675350701	2.51218829191193\\
-2.18733967634176	2.5250501002004\\
-2.18597120113148	2.54108216432866\\
-2.18452376321569	2.55711422845691\\
-2.18299582363611	2.57314629258517\\
-2.18138579001696	2.58917835671343\\
-2.179692015072	2.60521042084168\\
-2.17791279505507	2.62124248496994\\
-2.17604636815184	2.6372745490982\\
-2.17409091281085	2.65330661322645\\
-2.17234468937876	2.66700321609212\\
-2.17205000108829	2.66933867735471\\
-2.169949223652	2.68537074148297\\
-2.16775450582966	2.70140280561122\\
-2.1654637519929	2.71743486973948\\
-2.1630747968375	2.73346693386774\\
-2.16058540325749	2.74949899799599\\
-2.15799326013909	2.76553106212425\\
-2.1563126252505	2.77555497956145\\
-2.15531263058558	2.7815631262525\\
-2.1525529883469	2.79759519038076\\
-2.14968393265089	2.81362725450902\\
-2.14670280909838	2.82965931863727\\
-2.1436068759187	2.84569138276553\\
-2.14039330115298	2.86172344689379\\
-2.14028056112224	2.86227095588746\\
-2.13710656700386	2.87775551102204\\
-2.13369847096792	2.8937875751503\\
-2.13016418146547	2.90981963927856\\
-2.12650044307887	2.92585170340681\\
-2.12424849699399	2.9354039450745\\
-2.12272457763207	2.94188376753507\\
-2.11884335835732	2.95791583166333\\
-2.11482256942741	2.97394789579158\\
-2.11065846187091	2.98997995991984\\
-2.10821643286573	2.99910755738531\\
-2.10636965934285	3.0060120240481\\
-2.10195962779741	3.02204408817635\\
-2.09739431105722	3.03807615230461\\
-2.09266938106967	3.05410821643287\\
-2.09218436873747	3.05571690098827\\
-2.08782724256108	3.07014028056112\\
-2.08282144178327	3.08617234468938\\
-2.07764187201208	3.10220440881764\\
-2.07615230460922	3.10670417198628\\
-2.0723199728081	3.11823647294589\\
-2.0668279322443	3.13426853707415\\
-2.06114596808467	3.1503006012024\\
-2.06012024048096	3.15313095886327\\
-2.05530808263616	3.16633266533066\\
-2.04927652476668	3.18236472945892\\
-2.04408817635271	3.19572743614558\\
-2.04304403780948	3.19839679358717\\
-2.03663353865448	3.21442885771543\\
-2.03000039851701	3.23046092184369\\
-2.02805611222445	3.23505272624421\\
-2.02316612665281	3.24649298597194\\
-2.01610504929561	3.2625250501002\\
-2.01202404809619	3.27154119214257\\
-2.00881293353938	3.27855711422846\\
-2.00128462031002	3.29458917835671\\
-1.99599198396794	3.30553466284577\\
-1.99350063178841	3.31062124248497\\
-1.98546057570419	3.32665330661323\\
-1.97995991983968	3.33730692442868\\
-1.97714210753077	3.34268537074148\\
-1.96853981202308	3.35871743486974\\
-1.96392785571142	3.36708567002893\\
-1.95963426252646	3.374749498998\\
-1.9504123103506	3.39078156312625\\
-1.94789579158317	3.39506013426401\\
-1.9408553430388	3.40681362725451\\
-1.93186372745491	3.42138894870863\\
-1.93094713526962	3.42284569138277\\
-1.92066168164344	3.43887775551102\\
-1.91583166332665	3.44621303528219\\
-1.90997946975807	3.45490981963928\\
-1.8997995991984	3.46964207855806\\
-1.89888029101296	3.47094188376754\\
-1.8873210583821	3.48697394789579\\
-1.88376753507014	3.4917892515538\\
-1.87527724755564	3.50300601202405\\
-1.86773547094188	3.51273640977956\\
-1.86271726148988	3.51903807615231\\
-1.85170340681363	3.53256061063652\\
-1.84959972635276	3.53507014028056\\
-1.8358740682194	3.55110220440882\\
-1.83567134268537	3.55133486431535\\
-1.82146947117734	3.56713426853707\\
-1.81963927855711	3.56913032217164\\
-1.80633755317204	3.58316633266533\\
-1.80360721442886	3.58599389569402\\
-1.79040271332929	3.59919839679359\\
-1.7875751503006	3.6019764910627\\
-1.77357601502004	3.61523046092184\\
-1.77154308617234	3.6171235514231\\
-1.75575261167513	3.6312625250501\\
-1.75551102204409	3.63147552060405\\
-1.73947895791583	3.64508353874812\\
-1.73675998401568	3.64729458917836\\
-1.72344689378758	3.65797252578098\\
-1.71645946836329	3.66332665330661\\
-1.70741482965932	3.67016942626541\\
-1.69466730704812	3.67935871743487\\
-1.69138276553106	3.68169890685915\\
-1.67535070140281	3.69260532634724\\
-1.67104927101087	3.69539078156313\\
-1.65931863727455	3.70291133356174\\
-1.64527066679244	3.71142284569138\\
-1.64328657314629	3.71261420294734\\
-1.62725450901804	3.72177920887112\\
-1.61670775334827	3.72745490981964\\
-1.61122244488978	3.7303851331788\\
-1.59519038076152	3.73847872555087\\
-1.5846136617279	3.7434869739479\\
-1.57915831663327	3.74605536787514\\
-1.56312625250501	3.75315746679068\\
-1.54767777357342	3.75951903807615\\
-1.54709418837675	3.75975836220358\\
-1.5310621242485	3.76593753192175\\
-1.51503006012024	3.77164679629925\\
-1.50316156487214	3.77555110220441\\
-1.49899799599198	3.77691813026863\\
-1.48296593186373	3.7817922824152\\
-1.46693386773547	3.78624079510059\\
-1.45090180360721	3.7902776572105\\
-1.44519057425423	3.79158316633267\\
-1.43486973947896	3.79394442068093\\
-1.4188376753507	3.79723728898389\\
-1.40280561122244	3.80015222065808\\
-1.38677354709419	3.80270011516812\\
-1.37074148296593	3.80489119658113\\
-1.35470941883768	3.80673504142009\\
-1.34536965782306	3.80761523046092\\
-1.33867735470942	3.80824897603755\\
-1.32264529058116	3.80944066381979\\
-1.30661322645291	3.81030619218637\\
-1.29058116232465	3.81085272362304\\
-1.27454909819639	3.81108688513842\\
-1.25851703406814	3.81101478617959\\
-1.24248496993988	3.81064203520423\\
-1.22645290581162	3.80997375496874\\
-1.21042084168337	3.80901459658709\\
-1.19438877755511	3.80776875241063\\
-1.19278466694236	3.80761523046092\\
-1.17835671342685	3.80626041352538\\
-1.1623246492986	3.80447940168065\\
-1.14629258517034	3.80242643223183\\
-1.13026052104208	3.80010406348015\\
-1.11422845691383	3.79751444724034\\
-1.09819639278557	3.79465933573617\\
-1.08238610784513	3.79158316633267\\
-1.08216432865731	3.79154076292006\\
-1.06613226452906	3.78821192551388\\
-1.0501002004008	3.7846257924548\\
-1.03406813627255	3.78078268667116\\
-1.01803607214429	3.77668255792351\\
-1.01387250326413	3.77555110220441\\
-1.00200400801603	3.7723776273203\\
-0.985971943887776	3.76783838447288\\
-0.969939879759519	3.76304552242583\\
-0.958735177018894	3.75951903807615\\
-0.953907815631263	3.75802296121048\\
-0.937875751503006	3.75280765424696\\
-0.92184368737475	3.74733952723095\\
-0.91104327932147	3.7434869739479\\
-0.905811623246493	3.74164795998862\\
-0.889779559118236	3.73576945779861\\
-0.87374749498998	3.72963642279804\\
-0.868262186531491	3.72745490981964\\
-0.857715430861723	3.7233182685757\\
-0.841683366733467	3.71678256208697\\
-0.829025370107675	3.71142284569138\\
-0.82565130260521	3.71001314770832\\
-0.809619238476954	3.70308011609804\\
-0.793587174348697	3.69588653505746\\
-0.792516323769336	3.69539078156313\\
-0.777555110220441	3.68855112833831\\
-0.761523046092184	3.68096111671956\\
-0.758238504575129	3.67935871743487\\
-0.745490981963928	3.6732143754226\\
-0.729458917835671	3.66522904433948\\
-0.725746813913297	3.66332665330661\\
-0.713426853707415	3.65708548600821\\
-0.697394789579158	3.64870434829676\\
-0.69476868281116	3.64729458917836\\
-0.681362725450902	3.64017704005774\\
-0.665330661322646	3.63139803025136\\
-0.665089071691607	3.6312625250501\\
-0.649298597194389	3.62249864928523\\
-0.636584308250442	3.61523046092184\\
-0.633266533066132	3.61335323719904\\
-0.617234468937876	3.60405701947759\\
-0.609073991017453	3.59919839679359\\
-0.601202404809619	3.59455732732911\\
-0.585170340681363	3.58485599522903\\
-0.582440001938185	3.58316633266533\\
-0.569138276553106	3.57501041351771\\
-0.556626181278272	3.56713426853707\\
-0.55310621242485	3.56493835321383\\
-0.537074148296593	3.55471227402176\\
-0.531542492394103	3.55110220440882\\
-0.521042084168337	3.5443073060698\\
-0.507113700500945	3.53507014028056\\
-0.50501002004008	3.53368642806086\\
-0.488977955911824	3.52292736966742\\
-0.483308498413444	3.51903807615231\\
-0.472945891783567	3.51198384860222\\
-0.46005621128414	3.50300601202405\\
-0.456913827655311	3.50083355292382\\
-0.440881763527054	3.48952899261935\\
-0.437328240215089	3.48697394789579\\
-0.424849699398798	3.47806371092255\\
-0.415086534647536	3.47094188376754\\
-0.408817635270541	3.46639943697746\\
-0.393288980010621	3.45490981963928\\
-0.392785571142285	3.454539734248\\
-0.376753507014028	3.44255838458529\\
-0.371923488697241	3.43887775551102\\
-0.360721442885771	3.43039175147638\\
-0.350951914584676	3.42284569138277\\
-0.344689378757515	3.41803573511502\\
-0.330353172608619	3.40681362725451\\
-0.328657314629258	3.40549328796713\\
-0.312625250501002	3.39280623394943\\
-0.310108731733566	3.39078156312625\\
-0.296593186372745	3.37996057446672\\
-0.29019815203592	3.374749498998\\
-0.280561122244489	3.36693288932542\\
-0.270605055257431	3.35871743486974\\
-0.264529058116232	3.35372555976707\\
-0.251314806296885	3.34268537074148\\
-0.248496993987976	3.34034081811575\\
-0.232464929859719	3.32678327605539\\
-0.232313243736369	3.32665330661323\\
-0.216432865731463	3.31309537364555\\
-0.213579442999436	3.31062124248497\\
-0.200400801603207	3.2992324911091\\
-0.195108165261117	3.29458917835671\\
-0.18436873747495	3.28519632550059\\
-0.17688838962525	3.27855711422846\\
-0.168336673346694	3.27098843500974\\
-0.158909757082702	3.2625250501002\\
-0.152304609218437	3.25661024098888\\
-0.141162530661818	3.24649298597194\\
-0.13627254509018	3.24206302979891\\
-0.12363755846904	3.23046092184369\\
-0.120240480961924	3.22734795447686\\
-0.10632623939084	3.21442885771543\\
-0.104208416833667	3.21246603622831\\
-0.0892204912486374	3.19839679358717\\
-0.0881763527054109	3.19741816574723\\
-0.0723127212168553	3.18236472945892\\
-0.0721442885771544	3.18220510436568\\
-0.0561122244488979	3.16683736834846\\
-0.0555916565288081	3.16633266533066\\
-0.0400801603206413	3.15130549133715\\
-0.0390544327169323	3.1503006012024\\
-0.0240480961923848	3.13560667108172\\
-0.0226962325176485	3.13426853707415\\
-0.00801603206412826	3.11974116466135\\
-0.00651134034866899	3.11823647294589\\
0.00801603206412782	3.10370910053309\\
0.00950559946698457	3.10220440881764\\
0.0240480961923843	3.08751047869695\\
0.0253596009134551	3.08617234468938\\
0.0400801603206409	3.07114517069587\\
0.0410553532443366	3.07014028056112\\
0.0561122244488974	3.05461291945066\\
0.0565972367810957	3.05410821643287\\
0.0719910800676468	3.03807615230461\\
0.0721442885771539	3.03791652721137\\
0.087246414967761	3.02204408817635\\
0.0881763527054105	3.0210654603364\\
0.102361643310788	3.0060120240481\\
0.104208416833667	3.00404920256098\\
0.117340166624626	2.98997995991984\\
0.120240480961924	2.98686699255302\\
0.1321851375395	2.97394789579158\\
0.13627254509018	2.96951793961855\\
0.146899470581772	2.95791583166333\\
0.152304609218437	2.95200102255201\\
0.161485852207839	2.94188376753507\\
0.168336673346693	2.93431508831635\\
0.175946750125508	2.92585170340681\\
0.18436873747495	2.91645885055069\\
0.190284421946428	2.90981963927856\\
0.200400801603206	2.89843088790269\\
0.204500923209734	2.8937875751503\\
0.216432865731463	2.88022964218263\\
0.218598114813846	2.87775551102204\\
0.232464929859719	2.86185341633595\\
0.232577669890459	2.86172344689379\\
0.246472484612783	2.84569138276553\\
0.248496993987976	2.84334683013979\\
0.260257551260827	2.82965931863727\\
0.264529058116232	2.8246674435346\\
0.273932429644883	2.81362725450902\\
0.280561122244489	2.80581064483644\\
0.28749819455161	2.79759519038076\\
0.296593186372745	2.78677420172124\\
0.300955760353324	2.7815631262525\\
0.312625250501002	2.76755573294743\\
0.314305885389588	2.76553106212425\\
0.327567928056745	2.74949899799599\\
0.328657314629258	2.74817865870861\\
0.340755276724305	2.73346693386774\\
0.344689378757515	2.72865697759999\\
0.353840505499915	2.71743486973948\\
0.360721442885771	2.70894886570484\\
0.366823878678944	2.70140280561122\\
0.376753507014028	2.68905137055723\\
0.37970552252767	2.68537074148297\\
0.39249088285182	2.66933867735471\\
0.392785571142285	2.66896859196343\\
0.405231158769934	2.65330661322645\\
0.408817635270541	2.64876416643638\\
0.417873568327678	2.6372745490982\\
0.424849699398798	2.62836431212495\\
0.430417805075108	2.62124248496994\\
0.440881763527054	2.60776546556524\\
0.442863436587081	2.60521042084168\\
0.45524297062471	2.58917835671343\\
0.456913827655311	2.5870058976132\\
0.467564961912667	2.57314629258517\\
0.472945891783567	2.56609206503508\\
0.479790945645323	2.55711422845691\\
0.488977955911824	2.54497145784377\\
0.491920100199112	2.54108216432866\\
0.503972942874826	2.5250501002004\\
0.50501002004008	2.5236663879807\\
0.515990373212707	2.50901803607214\\
0.521042084168337	2.50222313773313\\
0.527912695607077	2.49298597194389\\
0.537074148296593	2.48056397742858\\
0.539738724534669	2.47695390781563\\
0.551501787790471	2.46092184368737\\
0.55310621242485	2.45872592836413\\
0.563226279677954	2.44488977955912\\
0.569138276553106	2.43673386041149\\
0.574855395806361	2.42885771543086\\
0.585170340681363	2.4145153138663\\
0.586387604466831	2.41282565130261\\
0.597895556901213	2.39679358717435\\
0.601202404809619	2.39215251770987\\
0.609334863426597	2.38076152304609\\
0.617234468937876	2.36958808160184\\
0.620677433211101	2.36472945891784\\
0.631951634761235	2.34869739478958\\
0.633266533066132	2.34682017106677\\
0.643206984131002	2.33266533066132\\
0.649298597194389	2.32390145489646\\
0.654365277727699	2.31663326653307\\
0.665330661322646	2.30073670760607\\
0.665424433461496	2.30060120240481\\
0.676498350458976	2.28456913827655\\
0.681362725450902	2.27745158915594\\
0.687476579602111	2.2685370741483\\
0.697394789579159	2.25391476913844\\
0.698354673058552	2.25250501002004\\
0.709232506939536	2.23647294589178\\
0.713426853707415	2.23023177859338\\
0.720033795682041	2.22044088176353\\
0.729458917835672	2.20631120866813\\
0.730733503473829	2.20440881763527\\
0.7414301412799	2.18837675350701\\
0.745490981963928	2.18223241149474\\
0.752056579978297	2.17234468937876\\
0.761523046092185	2.15791502453519\\
0.762579557512944	2.1563126252505\\
0.773109252646421	2.14028056112224\\
0.777555110220441	2.13344090789742\\
0.783561943357505	2.12424849699399\\
0.793587174348698	2.10871218636007\\
0.793908866757872	2.10821643286573\\
0.804285298864114	2.09218436873747\\
0.809619238476954	2.08384163914413\\
0.814564392751905	2.07615230460922\\
0.824758302252478	2.06012024048096\\
0.825651302605211	2.0587105424979\\
0.834971324123861	2.04408817635271\\
0.841683366733467	2.03341582862003\\
0.845076051383763	2.02805611222445\\
0.855137738611259	2.01202404809619\\
0.857715430861724	2.00788740685225\\
0.865178068571839	1.99599198396794\\
0.87374749498998	1.98214143281808\\
0.875106761295386	1.97995991983968\\
0.885048557799838	1.96392785571142\\
0.889779559118236	1.95621033956214\\
0.89491406096065	1.94789579158317\\
0.904694649819828	1.93186372745491\\
0.905811623246493	1.93002471349564\\
0.914498021268976	1.91583166332665\\
0.921843687374749	1.90365215248145\\
0.924185695363177	1.8997995991984\\
0.933865262670698	1.88376753507014\\
0.937875751503006	1.87705615124094\\
0.943491311068582	1.86773547094188\\
0.953022059741463	1.85170340681363\\
0.953907815631262	1.85020732994795\\
0.962587153792168	1.83567134268537\\
0.969939879759519	1.8231657629068\\
0.972031583136088	1.81963927855711\\
0.981479033512833	1.80360721442886\\
0.985971943887775	1.79589449669732\\
0.990862346907323	1.7875751503006\\
1.00017244570538	1.77154308617235\\
1.00200400801603	1.76836961128823\\
1.00949495205705	1.75551102204409\\
1.01803607214429	1.74061041363493\\
1.01869091350944	1.73947895791583\\
1.02793446764556	1.72344689378758\\
1.03406813627254	1.71264641412607\\
1.03706881500154	1.70741482965932\\
1.04618567492923	1.69138276553106\\
1.0501002004008	1.6844253916532\\
1.05525837210245	1.67535070140281\\
1.06425307777468	1.65931863727455\\
1.06613226452906	1.65594739645577\\
1.07326397108547	1.64328657314629\\
1.08214091242481	1.62725450901804\\
1.08216432865731	1.62721210560544\\
1.09108973294102	1.61122244488978\\
1.09819639278557	1.59826655016503\\
1.09990251902498	1.59519038076152\\
1.10873952224162	1.57915831663327\\
1.11422845691383	1.56905753341269\\
1.11748836383966	1.56312625250501\\
1.12621695542465	1.54709418837675\\
1.13026052104208	1.53958302139598\\
1.13490125784064	1.5310621242485\\
1.14352540851714	1.51503006012024\\
1.14629258517034	1.50984126189114\\
1.15214446671673	1.49899799599198\\
1.16066802432549	1.48296593186373\\
1.1623246492986	1.47983010308346\\
1.16922102401272	1.46693386773547\\
1.17764771911111	1.45090180360721\\
1.17835671342685	1.44954698667167\\
1.18613373755529	1.43486973947896\\
1.19438877755511	1.41899119730041\\
1.19446961549923	1.4188376753507\\
1.2028851953898	1.40280561122244\\
1.21042084168337	1.38817291322035\\
1.21115094474724	1.38677354709419\\
1.21947777124482	1.37074148296593\\
1.22645290581162	1.35706794334549\\
1.22767214575648	1.35470941883768\\
1.23591362954052	1.33867735470942\\
1.24248496993988	1.32567209532447\\
1.24403527399133	1.32264529058116\\
1.25219472995507	1.30661322645291\\
1.25851703406814	1.29398071804332\\
1.26024217985847	1.29058116232465\\
1.26832283156246	1.27454909819639\\
1.27454909819639	1.26198868874564\\
1.27629451261218	1.25851703406814\\
1.28429949655346	1.24248496993988\\
1.29058116232465	1.22969039897374\\
1.29219372384407	1.22645290581162\\
1.30012609355036	1.21042084168337\\
1.30661322645291	1.19707973928056\\
1.30794107056714	1.19438877755511\\
1.31580380052522	1.17835671342685\\
1.32264529058116	1.16415008265747\\
1.32353761790295	1.1623246492986\\
1.33133360732971	1.14629258517034\\
1.33867735470942	1.13089426661871\\
1.33898424137981	1.13026052104208\\
1.34671631784397	1.11422845691383\\
1.35429512759832	1.09819639278557\\
1.35470941883768	1.09731620374474\\
1.36195255175066	1.08216432865731\\
1.36947185626311	1.06613226452906\\
1.37074148296593	1.06340823064926\\
1.37704274593921	1.0501002004008\\
1.38450516462339	1.03406813627254\\
1.38677354709419	1.02915302097975\\
1.39198715554453	1.01803607214429\\
1.39939524815508	1.00200400801603\\
1.40280561122244	0.994540998213194\\
1.40678585462307	0.985971943887775\\
1.41414212425145	0.969939879759519\\
1.4188376753507	0.959561938282485\\
1.42143873646836	0.953907815631262\\
1.42874563242771	0.937875751503006\\
1.43486973947896	0.924204941723015\\
1.43594551356707	0.921843687374749\\
1.44320543417176	0.905811623246493\\
1.45032497946464	0.889779559118236\\
1.45090180360721	0.888474049996076\\
1.4575210124409	0.87374749498998\\
1.46459710752866	0.857715430861724\\
1.46693386773547	0.852373059629644\\
1.47169167080326	0.841683366733467\\
1.4787265569285	0.825651302605211\\
1.48296593186373	0.815860418687749\\
1.48571653222177	0.809619238476954\\
1.49271240871521	0.793587174348698\\
1.49899799599198	0.778922138284666\\
1.49959453747732	0.777555110220441\\
1.50655356408688	0.761523046092185\\
1.51338033327571	0.745490981963928\\
1.51503006012024	0.741586676058771\\
1.52024874236831	0.729458917835672\\
1.52704167434769	0.713426853707415\\
1.5310621242485	0.703813283424752\\
1.53379647862441	0.697394789579159\\
1.540557604138	0.681362725450902\\
1.54709418837675	0.665569985450078\\
1.54719512090086	0.665330661322646\\
1.55392643861466	0.649298597194389\\
1.56053156729777	0.633266533066132\\
1.56312625250501	0.626904961780657\\
1.56714630555442	0.617234468937876\\
1.57372428420662	0.601202404809619\\
1.57915831663327	0.587738734608609\\
1.58021514095198	0.585170340681363\\
1.58676785760723	0.569138276553106\\
1.59319910861719	0.55310621242485\\
1.59519038076152	0.548097964027826\\
1.59966000209789	0.537074148296593\\
1.60606840850847	0.521042084168337\\
1.61122244488978	0.507940243399239\\
1.61239823384089	0.50501002004008\\
1.61878558460074	0.488977955911824\\
1.62505577178471	0.472945891783567\\
1.62725450901804	0.467270190835045\\
1.63134792809836	0.456913827655311\\
1.63759925671641	0.440881763527054\\
1.64328657314629	0.42604105665475\\
1.64375252317298	0.424849699398798\\
1.64998669320543	0.408817635270541\\
1.65610761012657	0.392785571142285\\
1.65931863727455	0.384274059012638\\
1.66221493444902	0.376753507014028\\
1.66832068705787	0.360721442885771\\
1.67431648597408	0.344689378757515\\
1.67535070140281	0.34190392354163\\
1.68037282410443	0.328657314629258\\
1.6863553242951	0.312625250501002\\
1.69138276553106	0.298933375797022\\
1.6922604173067	0.296593186372745\\
1.69823117961289	0.280561122244489\\
1.70409529753085	0.264529058116232\\
1.70741482965932	0.255339766946777\\
1.70994020395826	0.248496993987976\\
1.71579427084906	0.232464929859719\\
1.72154459568374	0.216432865731463\\
1.72344689378758	0.211078738205828\\
1.72732382048331	0.200400801603206\\
1.73306566673049	0.18436873747495\\
1.73870651950803	0.168336673346693\\
1.73947895791583	0.166125622916457\\
1.74441441207152	0.152304609218437\\
1.75004824908652	0.13627254509018\\
1.75551102204409	0.12045347651587\\
1.75558622659312	0.120240480961924\\
1.76121443676716	0.104208416833667\\
1.76674422382966	0.0881763527054105\\
1.77154308617235	0.0740373790784134\\
1.77220020625581	0.0721442885771539\\
1.77772567498475	0.0561122244488974\\
1.78315512949002	0.0400801603206409\\
1.7875751503006	0.0268261904615026\\
1.78852289179611	0.0240480961923843\\
1.7939492363902	0.00801603206412782\\
1.79928184315911	-0.00801603206412826\\
1.80360721442886	-0.0212205331636935\\
1.80455495592437	-0.0240480961923848\\
1.80988556418485	-0.0400801603206413\\
1.81512458387508	-0.0561122244488979\\
1.81963927855711	-0.0701482349425848\\
1.82029639864058	-0.0721442885771544\\
1.82553443681551	-0.0881763527054109\\
1.83068291346776	-0.104208416833667\\
1.83567134268537	-0.120007821055388\\
1.8357465472344	-0.120240480961924\\
1.84089496711966	-0.13627254509018\\
1.84595573486811	-0.152304609218437\\
1.85093096840582	-0.168336673346694\\
1.85170340681363	-0.17084620299073\\
1.85596559890135	-0.18436873747495\\
1.86094128787276	-0.200400801603207\\
1.86583317283805	-0.216432865731463\\
1.86773547094188	-0.222734532104207\\
1.87074410091915	-0.232464929859719\\
1.875637142341	-0.248496993987976\\
1.88044800294167	-0.264529058116232\\
1.88376753507014	-0.275745818586478\\
1.88522755825375	-0.280561122244489\\
1.89004018878716	-0.296593186372745\\
1.89477215796244	-0.312625250501002\\
1.89942534149261	-0.328657314629258\\
1.8997995991984	-0.329957119838733\\
1.90414662631756	-0.344689378757515\\
1.90880164898172	-0.360721442885771\\
1.9133792480818	-0.376753507014028\\
1.91583166332665	-0.385450291371113\\
1.91795194784638	-0.392785571142285\\
1.92253178338579	-0.408817635270541\\
1.92703546142494	-0.424849699398798\\
1.93146469803591	-0.440881763527054\\
1.93186372745491	-0.44233850620119\\
1.93595714653524	-0.456913827655311\\
1.94038838530739	-0.472945891783567\\
1.94474630364003	-0.488977955911824\\
1.94789579158317	-0.500731448902317\\
1.94907158053428	-0.50501002004008\\
1.95343168096556	-0.521042084168337\\
1.95771949174352	-0.537074148296593\\
1.96193658356709	-0.55310621242485\\
1.96392785571142	-0.560770041393917\\
1.96615824208962	-0.569138276553106\\
1.97037697660293	-0.585170340681363\\
1.97452588741304	-0.601202404809619\\
1.97860645615369	-0.617234468937876\\
1.97995991983968	-0.622612915250674\\
1.98271067567622	-0.633266533066132\\
1.98679217007759	-0.649298597194389\\
1.99080608711256	-0.665330661322646\\
1.99475382388421	-0.681362725450902\\
1.99599198396794	-0.686449305090096\\
1.99872633834385	-0.697394789579158\\
2.00267410246425	-0.713426853707415\\
2.0065563245599	-0.729458917835671\\
2.01037432125166	-0.745490981963928\\
2.01202404809619	-0.752506904049819\\
2.0142003569969	-0.761523046092184\\
2.01801729618351	-0.777555110220441\\
2.02177052494767	-0.793587174348697\\
2.02546128301537	-0.809619238476954\\
2.02805611222445	-0.821059498204689\\
2.02912549303033	-0.82565130260521\\
2.03281391529224	-0.841683366733467\\
2.03644026093137	-0.857715430861723\\
2.04000569592562	-0.87374749498998\\
2.04351135221013	-0.889779559118236\\
2.04408817635271	-0.892448916559835\\
2.04705370876431	-0.905811623246493\\
2.05055469160957	-0.92184368737475\\
2.05399613342972	-0.937875751503006\\
2.05737909650658	-0.953907815631263\\
2.06012024048096	-0.96710952209865\\
2.06072393619058	-0.969939879759519\\
2.06410048388159	-0.985971943887776\\
2.06741867270145	-1.00200400801603\\
2.07067949744408	-1.01803607214429\\
2.07388392213842	-1.03406813627255\\
2.07615230460922	-1.04560043723216\\
2.0770617153881	-1.0501002004008\\
2.08025679315444	-1.06613226452906\\
2.0833954375222	-1.08216432865731\\
2.08647854783053	-1.09819639278557\\
2.08950699428547	-1.11422845691383\\
2.09218436873747	-1.12865183648668\\
2.09249125540787	-1.13026052104208\\
2.09550706610889	-1.14629258517034\\
2.09846802604284	-1.1623246492986\\
2.10137494280979	-1.17835671342685\\
2.10422859624857	-1.19438877755511\\
2.10702973894416	-1.21042084168337\\
2.10821643286573	-1.21732530834615\\
2.10982899438516	-1.22645290581162\\
2.11261272985613	-1.24248496993988\\
2.11534357785558	-1.25851703406814\\
2.11802223036006	-1.27454909819639\\
2.12064935314585	-1.29058116232465\\
2.1232255862079	-1.30661322645291\\
2.12424849699399	-1.31309304891348\\
2.12579880104544	-1.32264529058116\\
2.12835227738831	-1.33867735470942\\
2.13085429246183	-1.35470941883768\\
2.13330542655544	-1.37074148296593\\
2.13570623492495	-1.38677354709419\\
2.13805724812759	-1.40280561122244\\
2.14028056112224	-1.41829016635703\\
2.14036139906636	-1.4188376753507\\
2.14268365816133	-1.43486973947896\\
2.14495534686681	-1.45090180360721\\
2.14717693583636	-1.46693386773547\\
2.14934887147192	-1.48296593186373\\
2.15147157617786	-1.49899799599198\\
2.1535454485973	-1.51503006012024\\
2.15557086383083	-1.5310621242485\\
2.1563126252505	-1.53707027093956\\
2.15758539475005	-1.54709418837675\\
2.15957253217634	-1.56312625250501\\
2.16151014528104	-1.57915831663327\\
2.1633985476971	-1.59519038076152\\
2.1652380295342	-1.61122244488978\\
2.16702885751804	-1.62725450901804\\
2.16877127511243	-1.64328657314629\\
2.17046550262438	-1.65931863727455\\
2.17211173729216	-1.67535070140281\\
2.17234468937876	-1.67768616266539\\
2.17374993028851	-1.69138276553106\\
2.17534536810776	-1.70741482965932\\
2.17689138091722	-1.72344689378758\\
2.17838810008491	-1.73947895791583\\
2.17983563341978	-1.75551102204409\\
2.18123406517753	-1.77154308617234\\
2.18258345604918	-1.7875751503006\\
2.18388384313207	-1.80360721442886\\
2.1851352398834	-1.81963927855711\\
2.18633763605613	-1.83567134268537\\
2.18749099761721	-1.85170340681363\\
2.18837675350701	-1.8645652151021\\
2.18860133989036	-1.86773547094188\\
2.18968554512439	-1.88376753507014\\
2.19071876149544	-1.8997995991984\\
2.19170087930555	-1.91583166332665\\
2.19263176387134	-1.93186372745491\\
2.19351125534943	-1.94789579158317\\
2.19433916854262	-1.96392785571142\\
2.19511529268665	-1.97995991983968\\
2.19583939121713	-1.99599198396794\\
2.19651120151655	-2.01202404809619\\
2.19713043464091	-2.02805611222445\\
2.19769677502566	-2.04408817635271\\
2.19820988017068	-2.06012024048096\\
2.19866938030378	-2.07615230460922\\
2.19907487802243	-2.09218436873747\\
2.19942594791327	-2.10821643286573\\
2.19972213614885	-2.12424849699399\\
2.19996296006125	-2.14028056112224\\
2.20014790769201	-2.1563126252505\\
2.20027643731779	-2.17234468937876\\
2.20034797695124	-2.18837675350701\\
2.20036192381657	-2.20440881763527\\
2.20031764379895	-2.22044088176353\\
2.20021447086739	-2.23647294589178\\
2.20005170647016	-2.25250501002004\\
2.19982861890213	-2.2685370741483\\
2.19954444264335	-2.28456913827655\\
2.19919837766783	-2.30060120240481\\
2.19878958872204	-2.31663326653307\\
2.19831720457188	-2.33266533066132\\
2.19778031721757	-2.34869739478958\\
2.19717798107516	-2.36472945891784\\
2.19650921212399	-2.38076152304609\\
2.19577298701883	-2.39679358717435\\
2.1949682421657	-2.41282565130261\\
2.19409387276027	-2.42885771543086\\
2.19314873178755	-2.44488977955912\\
2.19213162898178	-2.46092184368737\\
2.19104132974509	-2.47695390781563\\
2.1898765540237	-2.49298597194389\\
2.18863597514014	-2.50901803607214\\
2.18837675350701	-2.51218829191193\\
2.18733967634176	-2.5250501002004\\
2.18597120113148	-2.54108216432866\\
2.18452376321569	-2.55711422845691\\
2.18299582363611	-2.57314629258517\\
2.18138579001696	-2.58917835671343\\
2.179692015072	-2.60521042084168\\
2.17791279505507	-2.62124248496994\\
2.17604636815184	-2.6372745490982\\
2.17409091281085	-2.65330661322645\\
2.17234468937876	-2.66700321609212\\
2.17205000108829	-2.66933867735471\\
2.169949223652	-2.68537074148297\\
2.16775450582966	-2.70140280561122\\
2.1654637519929	-2.71743486973948\\
2.1630747968375	-2.73346693386774\\
2.16058540325749	-2.74949899799599\\
2.15799326013909	-2.76553106212425\\
2.1563126252505	-2.77555497956145\\
2.15531263058558	-2.7815631262525\\
2.1525529883469	-2.79759519038076\\
2.14968393265089	-2.81362725450902\\
2.14670280909838	-2.82965931863727\\
2.1436068759187	-2.84569138276553\\
2.14039330115298	-2.86172344689379\\
2.14028056112224	-2.86227095588746\\
2.13710656700386	-2.87775551102204\\
2.13369847096792	-2.8937875751503\\
2.13016418146547	-2.90981963927856\\
2.12650044307887	-2.92585170340681\\
2.12424849699399	-2.9354039450745\\
2.12272457763207	-2.94188376753507\\
2.11884335835732	-2.95791583166333\\
2.11482256942741	-2.97394789579158\\
2.11065846187091	-2.98997995991984\\
2.10821643286573	-2.99910755738531\\
2.10636965934285	-3.0060120240481\\
2.10195962779741	-3.02204408817635\\
2.09739431105722	-3.03807615230461\\
2.09266938106967	-3.05410821643287\\
2.09218436873747	-3.05571690098827\\
2.08782724256107	-3.07014028056112\\
2.08282144178327	-3.08617234468938\\
2.07764187201208	-3.10220440881764\\
2.07615230460922	-3.10670417198628\\
2.0723199728081	-3.11823647294589\\
2.0668279322443	-3.13426853707415\\
2.06114596808467	-3.1503006012024\\
2.06012024048096	-3.15313095886327\\
2.05530808263616	-3.16633266533066\\
2.04927652476668	-3.18236472945892\\
2.04408817635271	-3.19572743614558\\
2.04304403780948	-3.19839679358717\\
2.03663353865448	-3.21442885771543\\
2.03000039851701	-3.23046092184369\\
2.02805611222445	-3.23505272624421\\
2.02316612665281	-3.24649298597194\\
2.01610504929561	-3.2625250501002\\
2.01202404809619	-3.27154119214257\\
2.00881293353938	-3.27855711422846\\
2.00128462031003	-3.29458917835671\\
1.99599198396794	-3.30553466284577\\
1.99350063178841	-3.31062124248497\\
1.98546057570419	-3.32665330661323\\
1.97995991983968	-3.33730692442868\\
1.97714210753077	-3.34268537074148\\
1.96853981202308	-3.35871743486974\\
1.96392785571142	-3.36708567002893\\
1.95963426252646	-3.374749498998\\
1.9504123103506	-3.39078156312625\\
1.94789579158317	-3.39506013426401\\
1.94085534303881	-3.40681362725451\\
1.93186372745491	-3.42138894870863\\
1.93094713526962	-3.42284569138277\\
1.92066168164344	-3.43887775551102\\
1.91583166332665	-3.44621303528219\\
1.90997946975807	-3.45490981963928\\
1.8997995991984	-3.46964207855806\\
1.89888029101296	-3.47094188376753\\
1.88732105838211	-3.48697394789579\\
1.88376753507014	-3.4917892515538\\
1.87527724755564	-3.50300601202405\\
1.86773547094188	-3.51273640977956\\
1.86271726148988	-3.5190380761523\\
1.85170340681363	-3.53256061063652\\
1.84959972635276	-3.53507014028056\\
1.8358740682194	-3.55110220440882\\
1.83567134268537	-3.55133486431535\\
1.82146947117734	-3.56713426853707\\
1.81963927855711	-3.56913032217164\\
1.80633755317203	-3.58316633266533\\
1.80360721442886	-3.58599389569402\\
1.79040271332929	-3.59919839679359\\
1.7875751503006	-3.6019764910627\\
1.77357601502004	-3.61523046092184\\
1.77154308617235	-3.6171235514231\\
1.75575261167513	-3.6312625250501\\
1.75551102204409	-3.63147552060405\\
1.73947895791583	-3.64508353874812\\
1.73675998401568	-3.64729458917836\\
1.72344689378758	-3.65797252578098\\
1.71645946836328	-3.66332665330661\\
1.70741482965932	-3.67016942626541\\
1.69466730704812	-3.67935871743487\\
1.69138276553106	-3.68169890685915\\
1.67535070140281	-3.69260532634724\\
1.67104927101087	-3.69539078156313\\
1.65931863727455	-3.70291133356174\\
1.64527066679244	-3.71142284569138\\
1.64328657314629	-3.71261420294734\\
1.62725450901804	-3.72177920887112\\
1.61670775334827	-3.72745490981964\\
1.61122244488978	-3.7303851331788\\
1.59519038076152	-3.73847872555087\\
1.5846136617279	-3.7434869739479\\
1.57915831663327	-3.74605536787514\\
1.56312625250501	-3.75315746679068\\
1.54767777357342	-3.75951903807615\\
1.54709418837675	-3.75975836220358\\
1.5310621242485	-3.76593753192175\\
1.51503006012024	-3.77164679629925\\
1.50316156487214	-3.77555110220441\\
1.49899799599198	-3.77691813026864\\
1.48296593186373	-3.7817922824152\\
1.46693386773547	-3.78624079510059\\
1.45090180360721	-3.7902776572105\\
1.44519057425423	-3.79158316633267\\
1.43486973947896	-3.79394442068093\\
1.4188376753507	-3.79723728898389\\
1.40280561122244	-3.80015222065808\\
1.38677354709419	-3.80270011516812\\
1.37074148296593	-3.80489119658113\\
1.35470941883768	-3.80673504142009\\
1.34536965782306	-3.80761523046092\\
1.33867735470942	-3.80824897603755\\
1.32264529058116	-3.80944066381979\\
1.30661322645291	-3.81030619218637\\
1.29058116232465	-3.81085272362304\\
1.27454909819639	-3.81108688513842\\
1.25851703406814	-3.81101478617959\\
1.24248496993988	-3.81064203520423\\
1.22645290581162	-3.80997375496874\\
1.21042084168337	-3.80901459658709\\
1.19438877755511	-3.80776875241063\\
1.19278466694236	-3.80761523046092\\
1.17835671342685	-3.80626041352538\\
1.1623246492986	-3.80447940168065\\
1.14629258517034	-3.80242643223183\\
1.13026052104208	-3.80010406348015\\
1.11422845691383	-3.79751444724034\\
1.09819639278557	-3.79465933573617\\
1.08238610784513	-3.79158316633267\\
1.08216432865731	-3.79154076292006\\
1.06613226452906	-3.78821192551389\\
1.0501002004008	-3.7846257924548\\
1.03406813627254	-3.78078268667116\\
1.01803607214429	-3.77668255792351\\
1.01387250326413	-3.77555110220441\\
1.00200400801603	-3.7723776273203\\
0.985971943887775	-3.76783838447288\\
0.969939879759519	-3.76304552242583\\
0.958735177018894	-3.75951903807615\\
0.953907815631262	-3.75802296121048\\
0.937875751503006	-3.75280765424696\\
0.921843687374749	-3.74733952723095\\
0.911043279321471	-3.7434869739479\\
0.905811623246493	-3.74164795998862\\
0.889779559118236	-3.73576945779861\\
0.87374749498998	-3.72963642279804\\
0.868262186531492	-3.72745490981964\\
0.857715430861724	-3.7233182685757\\
0.841683366733467	-3.71678256208697\\
0.829025370107676	-3.71142284569138\\
0.825651302605211	-3.71001314770832\\
0.809619238476954	-3.70308011609804\\
0.793587174348698	-3.69588653505746\\
0.792516323769337	-3.69539078156313\\
0.777555110220441	-3.68855112833831\\
0.761523046092185	-3.68096111671956\\
0.75823850457513	-3.67935871743487\\
0.745490981963928	-3.6732143754226\\
0.729458917835672	-3.66522904433948\\
0.725746813913297	-3.66332665330661\\
0.713426853707415	-3.65708548600821\\
0.697394789579159	-3.64870434829676\\
0.694768682811162	-3.64729458917836\\
0.681362725450902	-3.64017704005774\\
0.665330661322646	-3.63139803025136\\
0.665089071691608	-3.6312625250501\\
0.649298597194389	-3.62249864928523\\
0.636584308250443	-3.61523046092184\\
0.633266533066132	-3.61335323719904\\
0.617234468937876	-3.60405701947759\\
0.609073991017455	-3.59919839679359\\
0.601202404809619	-3.59455732732911\\
0.585170340681363	-3.58485599522903\\
0.582440001938186	-3.58316633266533\\
0.569138276553106	-3.57501041351771\\
0.556626181278274	-3.56713426853707\\
0.55310621242485	-3.56493835321383\\
0.537074148296593	-3.55471227402176\\
0.531542492394102	-3.55110220440882\\
0.521042084168337	-3.5443073060698\\
0.507113700500945	-3.53507014028056\\
0.50501002004008	-3.53368642806086\\
0.488977955911824	-3.52292736966742\\
0.483308498413443	-3.5190380761523\\
0.472945891783567	-3.51198384860222\\
0.460056211284139	-3.50300601202405\\
0.456913827655311	-3.50083355292383\\
0.440881763527054	-3.48952899261935\\
0.437328240215089	-3.48697394789579\\
0.424849699398798	-3.47806371092255\\
0.415086534647535	-3.47094188376754\\
0.408817635270541	-3.46639943697746\\
0.393288980010621	-3.45490981963928\\
0.392785571142285	-3.454539734248\\
0.376753507014028	-3.44255838458529\\
0.371923488697241	-3.43887775551102\\
0.360721442885771	-3.43039175147638\\
0.350951914584676	-3.42284569138276\\
0.344689378757515	-3.41803573511502\\
0.330353172608618	-3.40681362725451\\
0.328657314629258	-3.40549328796713\\
0.312625250501002	-3.39280623394943\\
0.310108731733566	-3.39078156312625\\
0.296593186372745	-3.37996057446673\\
0.29019815203592	-3.374749498998\\
0.280561122244489	-3.36693288932542\\
0.270605055257431	-3.35871743486974\\
0.264529058116232	-3.35372555976707\\
0.251314806296885	-3.34268537074148\\
0.248496993987976	-3.34034081811575\\
0.232464929859719	-3.32678327605539\\
0.232313243736369	-3.32665330661323\\
0.216432865731463	-3.31309537364555\\
0.213579442999435	-3.31062124248497\\
0.200400801603206	-3.2992324911091\\
0.195108165261117	-3.29458917835671\\
0.18436873747495	-3.28519632550059\\
0.176888389625251	-3.27855711422846\\
0.168336673346693	-3.27098843500974\\
0.158909757082701	-3.2625250501002\\
0.152304609218437	-3.25661024098888\\
0.141162530661818	-3.24649298597194\\
0.13627254509018	-3.24206302979891\\
0.123637558469039	-3.23046092184369\\
0.120240480961924	-3.22734795447686\\
0.10632623939084	-3.21442885771543\\
0.104208416833667	-3.21246603622831\\
0.0892204912486378	-3.19839679358717\\
0.0881763527054105	-3.19741816574722\\
0.0723127212168553	-3.18236472945892\\
0.0721442885771539	-3.18220510436568\\
0.0561122244488974	-3.16683736834846\\
0.055591656528809	-3.16633266533066\\
0.0400801603206409	-3.15130549133715\\
0.0390544327169332	-3.1503006012024\\
0.0240480961923843	-3.13560667108172\\
0.0226962325176485	-3.13426853707415\\
0.00801603206412782	-3.11974116466135\\
0.00651134034866946	-3.11823647294589\\
-0.00801603206412826	-3.10370910053309\\
-0.00950559946698502	-3.10220440881764\\
-0.0240480961923848	-3.08751047869695\\
-0.0253596009134542	-3.08617234468938\\
-0.0400801603206413	-3.07114517069587\\
-0.0410553532443371	-3.07014028056112\\
-0.0561122244488979	-3.05461291945066\\
-0.0565972367810961	-3.05410821643287\\
-0.0719910800676459	-3.03807615230461\\
-0.0721442885771544	-3.03791652721137\\
-0.087246414967761	-3.02204408817635\\
-0.0881763527054109	-3.0210654603364\\
-0.102361643310788	-3.0060120240481\\
-0.104208416833667	-3.00404920256098\\
-0.117340166624625	-2.98997995991984\\
-0.120240480961924	-2.98686699255302\\
-0.1321851375395	-2.97394789579158\\
-0.13627254509018	-2.96951793961855\\
-0.146899470581772	-2.95791583166333\\
-0.152304609218437	-2.95200102255201\\
-0.161485852207838	-2.94188376753507\\
-0.168336673346694	-2.93431508831635\\
-0.175946750125508	-2.92585170340681\\
-0.18436873747495	-2.91645885055069\\
-0.190284421946428	-2.90981963927856\\
-0.200400801603207	-2.89843088790269\\
-0.204500923209734	-2.8937875751503\\
-0.216432865731463	-2.88022964218263\\
-0.218598114813846	-2.87775551102204\\
-0.232464929859719	-2.86185341633595\\
-0.232577669890459	-2.86172344689379\\
-0.246472484612783	-2.84569138276553\\
-0.248496993987976	-2.84334683013979\\
-0.260257551260827	-2.82965931863727\\
-0.264529058116232	-2.8246674435346\\
-0.273932429644883	-2.81362725450902\\
-0.280561122244489	-2.80581064483644\\
-0.28749819455161	-2.79759519038076\\
-0.296593186372745	-2.78677420172123\\
-0.300955760353324	-2.7815631262525\\
-0.312625250501002	-2.76755573294743\\
-0.314305885389588	-2.76553106212425\\
-0.327567928056745	-2.74949899799599\\
-0.328657314629258	-2.74817865870861\\
-0.340755276724304	-2.73346693386774\\
-0.344689378757515	-2.72865697759999\\
-0.353840505499914	-2.71743486973948\\
-0.360721442885771	-2.70894886570484\\
-0.366823878678944	-2.70140280561122\\
-0.376753507014028	-2.68905137055723\\
-0.379705522527669	-2.68537074148297\\
-0.39249088285182	-2.66933867735471\\
-0.392785571142285	-2.66896859196343\\
-0.405231158769934	-2.65330661322645\\
-0.408817635270541	-2.64876416643638\\
-0.417873568327679	-2.6372745490982\\
-0.424849699398798	-2.62836431212495\\
-0.430417805075109	-2.62124248496994\\
-0.440881763527054	-2.60776546556524\\
-0.442863436587081	-2.60521042084168\\
-0.455242970624711	-2.58917835671343\\
-0.456913827655311	-2.5870058976132\\
-0.467564961912668	-2.57314629258517\\
-0.472945891783567	-2.56609206503508\\
-0.479790945645323	-2.55711422845691\\
-0.488977955911824	-2.54497145784377\\
-0.491920100199112	-2.54108216432866\\
-0.503972942874826	-2.5250501002004\\
-0.50501002004008	-2.5236663879807\\
-0.515990373212706	-2.50901803607214\\
-0.521042084168337	-2.50222313773313\\
-0.527912695607077	-2.49298597194389\\
-0.537074148296593	-2.48056397742858\\
-0.539738724534669	-2.47695390781563\\
-0.551501787790471	-2.46092184368737\\
-0.55310621242485	-2.45872592836413\\
-0.563226279677955	-2.44488977955912\\
-0.569138276553106	-2.43673386041149\\
-0.574855395806361	-2.42885771543086\\
-0.585170340681363	-2.4145153138663\\
-0.586387604466832	-2.41282565130261\\
-0.597895556901214	-2.39679358717435\\
-0.601202404809619	-2.39215251770987\\
-0.609334863426596	-2.38076152304609\\
-0.617234468937876	-2.36958808160184\\
-0.620677433211101	-2.36472945891784\\
-0.631951634761235	-2.34869739478958\\
-0.633266533066132	-2.34682017106677\\
-0.643206984131002	-2.33266533066132\\
-0.649298597194389	-2.32390145489646\\
-0.654365277727698	-2.31663326653307\\
-0.665330661322646	-2.30073670760607\\
-0.665424433461496	-2.30060120240481\\
-0.676498350458976	-2.28456913827655\\
-0.681362725450902	-2.27745158915594\\
-0.687476579602111	-2.2685370741483\\
-0.697394789579158	-2.25391476913844\\
-0.698354673058551	-2.25250501002004\\
-0.709232506939537	-2.23647294589178\\
-0.713426853707415	-2.23023177859338\\
-0.720033795682041	-2.22044088176353\\
-0.729458917835671	-2.20631120866813\\
-0.730733503473829	-2.20440881763527\\
-0.741430141279901	-2.18837675350701\\
-0.745490981963928	-2.18223241149475\\
-0.752056579978297	-2.17234468937876\\
-0.761523046092184	-2.15791502453519\\
-0.762579557512944	-2.1563126252505\\
-0.773109252646421	-2.14028056112224\\
-0.777555110220441	-2.13344090789742\\
-0.783561943357505	-2.12424849699399\\
-0.793587174348697	-2.10871218636007\\
-0.793908866757872	-2.10821643286573\\
-0.804285298864114	-2.09218436873747\\
-0.809619238476954	-2.08384163914414\\
-0.814564392751905	-2.07615230460922\\
-0.824758302252477	-2.06012024048096\\
-0.82565130260521	-2.0587105424979\\
-0.834971324123861	-2.04408817635271\\
-0.841683366733467	-2.03341582862003\\
-0.845076051383762	-2.02805611222445\\
-0.855137738611258	-2.01202404809619\\
-0.857715430861723	-2.00788740685225\\
-0.865178068571839	-1.99599198396794\\
-0.87374749498998	-1.98214143281808\\
-0.875106761295386	-1.97995991983968\\
-0.885048557799838	-1.96392785571142\\
-0.889779559118236	-1.95621033956214\\
-0.89491406096065	-1.94789579158317\\
-0.904694649819828	-1.93186372745491\\
-0.905811623246493	-1.93002471349564\\
-0.914498021268976	-1.91583166332665\\
-0.92184368737475	-1.90365215248145\\
-0.924185695363177	-1.8997995991984\\
-0.933865262670698	-1.88376753507014\\
-0.937875751503006	-1.87705615124094\\
-0.943491311068581	-1.86773547094188\\
-0.953022059741462	-1.85170340681363\\
-0.953907815631263	-1.85020732994795\\
-0.962587153792167	-1.83567134268537\\
-0.969939879759519	-1.8231657629068\\
-0.972031583136088	-1.81963927855711\\
-0.981479033512832	-1.80360721442886\\
-0.985971943887776	-1.79589449669732\\
-0.990862346907323	-1.7875751503006\\
-1.00017244570538	-1.77154308617234\\
-1.00200400801603	-1.76836961128823\\
-1.00949495205705	-1.75551102204409\\
-1.01803607214429	-1.74061041363493\\
-1.01869091350944	-1.73947895791583\\
-1.02793446764556	-1.72344689378758\\
-1.03406813627255	-1.71264641412607\\
-1.03706881500154	-1.70741482965932\\
-1.04618567492923	-1.69138276553106\\
-1.0501002004008	-1.68442539165319\\
-1.05525837210245	-1.67535070140281\\
-1.06425307777468	-1.65931863727455\\
-1.06613226452906	-1.65594739645577\\
-1.07326397108547	-1.64328657314629\\
-1.08214091242481	-1.62725450901804\\
-1.08216432865731	-1.62721210560543\\
-1.09108973294102	-1.61122244488978\\
-1.09819639278557	-1.59826655016503\\
-1.09990251902498	-1.59519038076152\\
-1.10873952224162	-1.57915831663327\\
-1.11422845691383	-1.56905753341268\\
-1.11748836383966	-1.56312625250501\\
-1.12621695542465	-1.54709418837675\\
-1.13026052104208	-1.53958302139598\\
-1.13490125784064	-1.5310621242485\\
-1.14352540851714	-1.51503006012024\\
-1.14629258517034	-1.50984126189114\\
-1.15214446671673	-1.49899799599198\\
-1.16066802432549	-1.48296593186373\\
-1.1623246492986	-1.47983010308346\\
-1.16922102401271	-1.46693386773547\\
-1.17764771911111	-1.45090180360721\\
-1.17835671342685	-1.44954698667167\\
-1.18613373755529	-1.43486973947896\\
-1.19438877755511	-1.41899119730041\\
-1.19446961549923	-1.4188376753507\\
-1.2028851953898	-1.40280561122244\\
-1.21042084168337	-1.38817291322035\\
-1.21115094474724	-1.38677354709419\\
-1.21947777124482	-1.37074148296593\\
-1.22645290581162	-1.35706794334549\\
-1.22767214575648	-1.35470941883768\\
-1.23591362954052	-1.33867735470942\\
-1.24248496993988	-1.32567209532447\\
-1.24403527399133	-1.32264529058116\\
-1.25219472995507	-1.30661322645291\\
-1.25851703406814	-1.29398071804332\\
-1.26024217985847	-1.29058116232465\\
-1.26832283156246	-1.27454909819639\\
-1.27454909819639	-1.26198868874564\\
-1.27629451261218	-1.25851703406814\\
-1.28429949655346	-1.24248496993988\\
-1.29058116232465	-1.22969039897374\\
-1.29219372384407	-1.22645290581162\\
-1.30012609355036	-1.21042084168337\\
-1.30661322645291	-1.19707973928056\\
-1.30794107056714	-1.19438877755511\\
-1.31580380052522	-1.17835671342685\\
-1.32264529058116	-1.16415008265747\\
-1.32353761790295	-1.1623246492986\\
-1.33133360732971	-1.14629258517034\\
-1.33867735470942	-1.13089426661871\\
-1.33898424137981	-1.13026052104208\\
-1.34671631784397	-1.11422845691383\\
-1.35429512759832	-1.09819639278557\\
-1.35470941883768	-1.09731620374474\\
-1.36195255175066	-1.08216432865731\\
-1.36947185626311	-1.06613226452906\\
-1.37074148296593	-1.06340823064926\\
-1.37704274593921	-1.0501002004008\\
-1.38450516462339	-1.03406813627255\\
-1.38677354709419	-1.02915302097975\\
-1.39198715554453	-1.01803607214429\\
-1.39939524815508	-1.00200400801603\\
-1.40280561122244	-0.994540998213195\\
-1.40678585462307	-0.985971943887776\\
-1.41414212425145	-0.969939879759519\\
-1.4188376753507	-0.959561938282485\\
-1.42143873646836	-0.953907815631263\\
-1.42874563242771	-0.937875751503006\\
-1.43486973947896	-0.924204941723015\\
-1.43594551356707	-0.92184368737475\\
-1.44320543417176	-0.905811623246493\\
-1.45032497946464	-0.889779559118236\\
-1.45090180360721	-0.888474049996076\\
-1.4575210124409	-0.87374749498998\\
-1.46459710752866	-0.857715430861723\\
-1.46693386773547	-0.852373059629644\\
-1.47169167080326	-0.841683366733467\\
-1.4787265569285	-0.82565130260521\\
-1.48296593186373	-0.815860418687748\\
-1.48571653222177	-0.809619238476954\\
-1.49271240871521	-0.793587174348697\\
-1.49899799599198	-0.778922138284666\\
-1.49959453747732	-0.777555110220441\\
-1.50655356408688	-0.761523046092184\\
-1.51338033327571	-0.745490981963928\\
-1.51503006012024	-0.74158667605877\\
-1.52024874236831	-0.729458917835671\\
-1.52704167434769	-0.713426853707415\\
-1.5310621242485	-0.703813283424752\\
-1.53379647862441	-0.697394789579158\\
-1.54055760413801	-0.681362725450902\\
-1.54709418837675	-0.665569985450078\\
-1.54719512090086	-0.665330661322646\\
-1.55392643861466	-0.649298597194389\\
-1.56053156729777	-0.633266533066132\\
-1.56312625250501	-0.626904961780658\\
-1.56714630555442	-0.617234468937876\\
-1.57372428420663	-0.601202404809619\\
-1.57915831663327	-0.587738734608611\\
-1.58021514095198	-0.585170340681363\\
-1.58676785760723	-0.569138276553106\\
-1.59319910861719	-0.55310621242485\\
-1.59519038076152	-0.548097964027827\\
-1.59966000209789	-0.537074148296593\\
-1.60606840850847	-0.521042084168337\\
-1.61122244488978	-0.50794024339924\\
-1.61239823384089	-0.50501002004008\\
-1.61878558460074	-0.488977955911824\\
-1.62505577178471	-0.472945891783567\\
-1.62725450901804	-0.467270190835046\\
-1.63134792809836	-0.456913827655311\\
-1.63759925671641	-0.440881763527054\\
-1.64328657314629	-0.426041056654751\\
-1.64375252317298	-0.424849699398798\\
-1.64998669320543	-0.408817635270541\\
-1.65610761012657	-0.392785571142285\\
-1.65931863727455	-0.38427405901264\\
-1.66221493444902	-0.376753507014028\\
-1.66832068705787	-0.360721442885771\\
-1.67431648597408	-0.344689378757515\\
-1.67535070140281	-0.341903923541631\\
-1.68037282410443	-0.328657314629258\\
-1.6863553242951	-0.312625250501002\\
-1.69138276553106	-0.298933375797024\\
-1.69226041730671	-0.296593186372745\\
-1.69823117961289	-0.280561122244489\\
-1.70409529753085	-0.264529058116232\\
-1.70741482965932	-0.255339766946777\\
-1.70994020395826	-0.248496993987976\\
-1.71579427084906	-0.232464929859719\\
-1.72154459568374	-0.216432865731463\\
-1.72344689378758	-0.211078738205828\\
-1.72732382048331	-0.200400801603207\\
-1.73306566673049	-0.18436873747495\\
-1.73870651950802	-0.168336673346694\\
-1.73947895791583	-0.166125622916459\\
-1.74441441207152	-0.152304609218437\\
-1.75004824908651	-0.13627254509018\\
-1.75551102204409	-0.120453476515871\\
-1.75558622659312	-0.120240480961924\\
-1.76121443676716	-0.104208416833667\\
-1.76674422382965	-0.0881763527054109\\
-1.77154308617234	-0.0740373790784148\\
-1.77220020625581	-0.0721442885771544\\
-1.77772567498475	-0.0561122244488979\\
-1.78315512949002	-0.0400801603206413\\
-1.7875751503006	-0.0268261904615012\\
-1.78852289179611	-0.0240480961923848\\
-1.7939492363902	-0.00801603206412826\\
-1.79928184315911	0.00801603206412782\\
-1.80360721442886	0.0212205331636953\\
-1.80455495592437	0.0240480961923843\\
-1.80988556418485	0.0400801603206409\\
-1.81512458387508	0.0561122244488974\\
-1.81963927855711	0.0701482349425853\\
-1.82029639864058	0.0721442885771539\\
-1.82553443681551	0.0881763527054105\\
-1.83068291346776	0.104208416833667\\
-1.83567134268537	0.120007821055389\\
-1.8357465472344	0.120240480961924\\
-1.84089496711966	0.13627254509018\\
-1.84595573486811	0.152304609218437\\
-1.85093096840582	0.168336673346693\\
-1.85170340681363	0.170846202990731\\
-1.85596559890135	0.18436873747495\\
-1.86094128787276	0.200400801603206\\
-1.86583317283805	0.216432865731463\\
-1.86773547094188	0.222734532104208\\
-1.87074410091915	0.232464929859719\\
-1.875637142341	0.248496993987976\\
-1.88044800294167	0.264529058116232\\
-1.88376753507014	0.27574581858648\\
-1.88522755825375	0.280561122244489\\
-1.89004018878716	0.296593186372745\\
-1.89477215796244	0.312625250501002\\
-1.89942534149261	0.328657314629258\\
-1.8997995991984	0.329957119838733\\
-1.90414662631756	0.344689378757515\\
-1.90880164898172	0.360721442885771\\
-1.9133792480818	0.376753507014028\\
-1.91583166332665	0.385450291371114\\
-1.91795194784638	0.392785571142285\\
-1.92253178338579	0.408817635270541\\
-1.92703546142494	0.424849699398798\\
-1.93146469803591	0.440881763527054\\
-1.93186372745491	0.442338506201192\\
-1.93595714653524	0.456913827655311\\
-1.94038838530739	0.472945891783567\\
-1.94474630364003	0.488977955911824\\
-1.94789579158317	0.500731448902319\\
-1.94907158053428	0.50501002004008\\
-1.95343168096556	0.521042084168337\\
-1.95771949174352	0.537074148296593\\
-1.96193658356709	0.55310621242485\\
-1.96392785571142	0.560770041393919\\
-1.96615824208962	0.569138276553106\\
-1.97037697660293	0.585170340681363\\
-1.97452588741304	0.601202404809619\\
-1.97860645615369	0.617234468937876\\
-1.97995991983968	0.622612915250676\\
-1.98271067567622	0.633266533066132\\
-1.98679217007759	0.649298597194389\\
-1.99080608711256	0.665330661322646\\
-1.99475382388421	0.681362725450902\\
-1.99599198396794	0.686449305090098\\
-1.99872633834385	0.697394789579159\\
-2.00267410246425	0.713426853707415\\
-2.0065563245599	0.729458917835672\\
-2.01037432125166	0.745490981963928\\
-2.01202404809619	0.752506904049819\\
-2.0142003569969	0.761523046092185\\
-2.01801729618351	0.777555110220441\\
-2.02177052494767	0.793587174348698\\
-2.02546128301537	0.809619238476954\\
-2.02805611222445	0.821059498204689\\
-2.02912549303033	0.825651302605211\\
-2.03281391529224	0.841683366733467\\
-2.03644026093137	0.857715430861724\\
-2.04000569592562	0.87374749498998\\
-2.04351135221013	0.889779559118236\\
-2.04408817635271	0.892448916559834\\
-2.04705370876431	0.905811623246493\\
-2.05055469160957	0.921843687374749\\
-2.05399613342972	0.937875751503006\\
-2.05737909650658	0.953907815631262\\
-2.06012024048096	0.96710952209865\\
-2.06072393619058	0.969939879759519\\
-2.06410048388159	0.985971943887775\\
-2.06741867270145	1.00200400801603\\
-2.07067949744408	1.01803607214429\\
-2.07388392213842	1.03406813627254\\
-2.07615230460922	1.04560043723216\\
-2.0770617153881	1.0501002004008\\
-2.08025679315444	1.06613226452906\\
-2.0833954375222	1.08216432865731\\
-2.08647854783053	1.09819639278557\\
-2.08950699428547	1.11422845691383\\
-2.09218436873747	1.12865183648668\\
-2.09249125540787	1.13026052104208\\
-2.09550706610889	1.14629258517034\\
-2.09846802604284	1.1623246492986\\
-2.10137494280979	1.17835671342685\\
-2.10422859624857	1.19438877755511\\
-2.10702973894416	1.21042084168337\\
-2.10821643286573	1.21732530834615\\
-2.10982899438516	1.22645290581162\\
-2.11261272985613	1.24248496993988\\
-2.11534357785558	1.25851703406814\\
-2.11802223036006	1.27454909819639\\
-2.12064935314585	1.29058116232465\\
-2.1232255862079	1.30661322645291\\
-2.12424849699399	1.31309304891348\\
-2.12579880104544	1.32264529058116\\
-2.12835227738831	1.33867735470942\\
-2.13085429246183	1.35470941883768\\
-2.13330542655544	1.37074148296593\\
-2.13570623492495	1.38677354709419\\
-2.13805724812759	1.40280561122244\\
-2.14028056112224	1.41829016635702\\
-2.14036139906636	1.4188376753507\\
-2.14268365816133	1.43486973947896\\
-2.14495534686681	1.45090180360721\\
-2.14717693583636	1.46693386773547\\
-2.14934887147192	1.48296593186373\\
-2.15147157617786	1.49899799599198\\
-2.1535454485973	1.51503006012024\\
-2.15557086383083	1.5310621242485\\
-2.1563126252505	1.53707027093956\\
-2.15758539475005	1.54709418837675\\
-2.15957253217634	1.56312625250501\\
-2.16151014528104	1.57915831663327\\
-2.1633985476971	1.59519038076152\\
-2.1652380295342	1.61122244488978\\
-2.16702885751804	1.62725450901804\\
-2.16877127511243	1.64328657314629\\
-2.17046550262438	1.65931863727455\\
-2.17211173729216	1.67535070140281\\
-2.17234468937876	1.67768616266539\\
-2.17374993028851	1.69138276553106\\
-2.17534536810776	1.70741482965932\\
-2.17689138091722	1.72344689378758\\
-2.17838810008491	1.73947895791583\\
-2.17983563341978	1.75551102204409\\
-2.18123406517753	1.77154308617235\\
-2.18258345604918	1.7875751503006\\
-2.18388384313207	1.80360721442886\\
-2.1851352398834	1.81963927855711\\
-2.18633763605613	1.83567134268537\\
-2.18749099761721	1.85170340681363\\
-2.18837675350701	1.8645652151021\\
-2.18860133989036	1.86773547094188\\
-2.18968554512439	1.88376753507014\\
-2.19071876149544	1.8997995991984\\
-2.19170087930555	1.91583166332665\\
-2.19263176387134	1.93186372745491\\
-2.19351125534943	1.94789579158317\\
-2.19433916854262	1.96392785571142\\
-2.19511529268665	1.97995991983968\\
-2.19583939121713	1.99599198396794\\
-2.19651120151655	2.01202404809619\\
-2.19713043464091	2.02805611222445\\
-2.19769677502566	2.04408817635271\\
-2.19820988017068	2.06012024048096\\
-2.19866938030378	2.07615230460922\\
-2.19907487802243	2.09218436873747\\
-2.19942594791327	2.10821643286573\\
-2.19972213614885	2.12424849699399\\
-2.19996296006125	2.14028056112224\\
-2.20014790769201	2.1563126252505\\
-2.20027643731779	2.17234468937876\\
-2.20034797695124	2.18837675350701\\
-2.20036192381657	2.20440881763527\\
-2.20031764379895	2.22044088176353\\
-2.20021447086739	2.23647294589178\\
-2.20005170647016	2.25250501002004\\
-2.19982861890213	2.2685370741483\\
-2.19954444264334	2.28456913827655\\
-2.19919837766783	2.30060120240481\\
-2.19878958872204	2.31663326653307\\
-2.19831720457188	2.33266533066132\\
-2.19778031721757	2.34869739478958\\
-2.19717798107516	2.36472945891784\\
-2.19650921212399	2.38076152304609\\
-2.19577298701883	2.39679358717435\\
-2.1949682421657	2.41282565130261\\
-2.19409387276027	2.42885771543086\\
-2.19314873178755	2.44488977955912\\
-2.19213162898178	2.46092184368737\\
-2.19104132974509	2.47695390781563\\
-2.1898765540237	2.49298597194389\\
-2.18863597514014	2.50901803607214\\
-2.18837675350701	2.51218829191193\\
}--cycle;


\addplot[area legend,solid,fill=mycolor3,draw=black,forget plot]
table[row sep=crcr] {%
x	y\\
-1.85170340681363	2.08110478474631\\
-1.85096402425816	2.09218436873747\\
-1.84980986194504	2.10821643286573\\
-1.84856756467358	2.12424849699399\\
-1.84723538393176	2.14028056112224\\
-1.84581150654969	2.1563126252505\\
-1.84429405280818	2.17234468937876\\
-1.84268107447124	2.18837675350701\\
-1.8409705527399	2.20440881763527\\
-1.83916039612411	2.22044088176353\\
-1.83724843822968	2.23647294589178\\
-1.83567134268537	2.2490366856574\\
-1.83523737101778	2.25250501002004\\
-1.83313835568119	2.2685370741483\\
-1.83093089661092	2.28456913827655\\
-1.82861245985542	2.30060120240481\\
-1.82618042239087	2.31663326653307\\
-1.82363206922794	2.33266533066132\\
-1.82096459040213	2.34869739478958\\
-1.81963927855711	2.35635177958395\\
-1.81818857080282	2.36472945891784\\
-1.81529996453305	2.38076152304609\\
-1.8122831122404	2.39679358717435\\
-1.80913473079258	2.41282565130261\\
-1.80585142247804	2.42885771543086\\
-1.80360721442886	2.43941001692876\\
-1.8024385369892	2.44488977955912\\
-1.79890051672139	2.46092184368737\\
-1.79521614279353	2.47695390781563\\
-1.79138143889474	2.49298597194389\\
-1.7875751503006	2.50829095273659\\
-1.78739338660917	2.50901803607214\\
-1.78326946333912	2.5250501002004\\
-1.77898147439258	2.54108216432866\\
-1.7745247538334	2.55711422845691\\
-1.77154308617234	2.56748238129583\\
-1.76990175838651	2.57314629258517\\
-1.76511267720294	2.58917835671343\\
-1.7601383404456	2.60521042084168\\
-1.75551102204409	2.61959233253044\\
-1.75497487581311	2.62124248496994\\
-1.7496281134607	2.6372745490982\\
-1.74407694612616	2.65330661322645\\
-1.73947895791583	2.66613514873682\\
-1.73831692392196	2.66933867735471\\
-1.73234548685442	2.68537074148297\\
-1.72614739434424	2.70140280561122\\
-1.72344689378758	2.70818987746438\\
-1.71971566625049	2.71743486973948\\
-1.71304013232617	2.73346693386774\\
-1.70741482965932	2.74652047249504\\
-1.7061103722011	2.74949899799599\\
-1.69891012383408	2.76553106212425\\
-1.69143610370506	2.7815631262525\\
-1.69138276553106	2.78167509565841\\
-1.68365691788803	2.79759519038076\\
-1.67557984389159	2.81362725450902\\
-1.67535070140281	2.81407237793107\\
-1.66715968749157	2.82965931863727\\
-1.65931863727455	2.84405724240068\\
-1.65840866361447	2.84569138276553\\
-1.64927335199124	2.86172344689379\\
-1.64328657314629	2.87190089013585\\
-1.6397574132464	2.87775551102204\\
-1.62982373029581	2.8937875751503\\
-1.62725450901804	2.89783144559011\\
-1.61943156879626	2.90981963927856\\
-1.61122244488978	2.92202643431353\\
-1.60857531675665	2.92585170340681\\
-1.59719561432098	2.94188376753507\\
-1.59519038076152	2.94464271289486\\
-1.58523875260076	2.95791583166333\\
-1.57915831663327	2.9658107404219\\
-1.5726802746891	2.97394789579158\\
-1.56312625250501	2.98564600747063\\
-1.55945965112415	2.98997995991984\\
-1.54709418837675	3.00424549977693\\
-1.5455044615911	3.0060120240481\\
-1.5310621242485	3.02169538090901\\
-1.5307279572433	3.02204408817635\\
-1.51503006012024	3.03807216018363\\
-1.51502598115792	3.03807615230461\\
-1.49899799599198	3.05344370640559\\
-1.49827304395488	3.05410821643287\\
-1.48296593186373	3.06787012941409\\
-1.48031727468776	3.07014028056112\\
-1.46693386773547	3.08140454787408\\
-1.46097385579681	3.08617234468938\\
-1.45090180360721	3.09409375882733\\
-1.44001638721062	3.10220440881764\\
-1.43486973947896	3.10597882209824\\
-1.4188376753507	3.11710033903772\\
-1.41710903881226	3.11823647294589\\
-1.40280561122244	3.12750636962587\\
-1.39167734969043	3.13426853707415\\
-1.38677354709419	3.1372101514459\\
-1.37074148296593	3.14625626530284\\
-1.36307656718187	3.1503006012024\\
-1.35470941883768	3.15466661391826\\
-1.33867735470942	3.16246635164411\\
-1.33012237525181	3.16633266533066\\
-1.32264529058116	3.16968031772411\\
-1.30661322645291	3.17633463965954\\
-1.29075325444766	3.18236472945892\\
-1.29058116232465	3.18242966225713\\
-1.27454909819639	3.18802432785573\\
-1.25851703406814	3.19309930910985\\
-1.24248496993988	3.19767335714685\\
-1.23967836172005	3.19839679358717\\
-1.22645290581162	3.2017900554151\\
-1.21042084168337	3.20544337144136\\
-1.19438877755511	3.20864327520901\\
-1.17835671342685	3.21140421553854\\
-1.1623246492986	3.21373973178891\\
-1.15661502706466	3.21442885771543\\
-1.14629258517034	3.215673431823\\
-1.13026052104208	3.21720929975852\\
-1.11422845691383	3.21835227700143\\
-1.09819639278557	3.21911252681467\\
-1.08216432865731	3.21949948728803\\
-1.06613226452906	3.21952189989994\\
-1.0501002004008	3.21918783601784\\
-1.03406813627255	3.21850472144465\\
-1.01803607214429	3.21747935911114\\
-1.00200400801603	3.21611795000566\\
-0.985998095749708	3.21442885771543\\
-0.985971943887776	3.21442614155896\\
-0.969939879759519	3.21243069533663\\
-0.953907815631263	3.21011861233941\\
-0.937875751503006	3.2074939838205\\
-0.92184368737475	3.20456038185535\\
-0.905811623246493	3.20132087162849\\
-0.892586167338064	3.19839679358717\\
-0.889779559118236	3.19778513500252\\
-0.87374749498998	3.19398599487374\\
-0.857715430861723	3.18989208776193\\
-0.841683366733467	3.18550468284531\\
-0.830931707074993	3.18236472945892\\
-0.82565130260521	3.18084309491341\\
-0.809619238476954	3.17593153597101\\
-0.793587174348697	3.17073286629194\\
-0.780729021259947	3.16633266533066\\
-0.777555110220441	3.16526011168692\\
-0.761523046092184	3.15955813639492\\
-0.745490981963928	3.15357218820123\\
-0.737123833619732	3.1503006012024\\
-0.729458917835671	3.14733881557017\\
-0.713426853707415	3.14086475809335\\
-0.697777947004	3.13426853707415\\
-0.697394789579158	3.13410884736716\\
-0.681362725450902	3.12715612856513\\
-0.665330661322646	3.11991987857457\\
-0.661735583422272	3.11823647294589\\
-0.649298597194389	3.11247386624583\\
-0.633266533066132	3.10476552161437\\
-0.628119885334472	3.10220440881764\\
-0.617234468937876	3.09684155551826\\
-0.601202404809619	3.08866524528832\\
-0.596468517478442	3.08617234468938\\
-0.585170340681363	3.08027895349741\\
-0.569138276553106	3.07163682472855\\
-0.566442776085726	3.07014028056112\\
-0.55310621242485	3.06280212052261\\
-0.537799100333697	3.05410821643287\\
-0.537074148296593	3.05370005159222\\
-0.521042084168337	3.0444235273177\\
-0.510393744430537	3.03807615230461\\
-0.50501002004008	3.03489325179561\\
-0.488977955911824	3.02515213976752\\
-0.483995866906375	3.02204408817635\\
-0.472945891783567	3.01520356570113\\
-0.458503554440966	3.0060120240481\\
-0.456913827655311	3.00500778702454\\
-0.440881763527054	2.99463500693657\\
-0.433868346994613	2.98997995991984\\
-0.424849699398798	2.98403536906207\\
-0.409932488193913	2.97394789579158\\
-0.408817635270541	2.97319901951363\\
-0.392785571142285	2.96219094658839\\
-0.386705135174792	2.95791583166333\\
-0.376753507014028	2.95096193545089\\
-0.364065662450606	2.94188376753507\\
-0.360721442885771	2.93950507801007\\
-0.344689378757515	2.92785282855784\\
-0.341992196266275	2.92585170340681\\
-0.328657314629258	2.91601138957255\\
-0.32044819072278	2.90981963927856\\
-0.312625250501002	2.90394942268957\\
-0.299368989727407	2.8937875751503\\
-0.296593186372745	2.89167009450756\\
-0.280561122244489	2.87919735205685\\
-0.278741749184476	2.87775551102204\\
-0.264529058116232	2.86654115588278\\
-0.258542279271288	2.86172344689379\\
-0.248496993987976	2.85367293277856\\
-0.238727092594649	2.84569138276553\\
-0.232464929859719	2.84059517615725\\
-0.219276137482059	2.82965931863727\\
-0.216432865731463	2.82731020824735\\
-0.200400801603207	2.81382306215925\\
-0.200171659114423	2.81362725450902\\
-0.18436873747495	2.80016487245945\\
-0.181403709616206	2.79759519038076\\
-0.168336673346694	2.78630243725173\\
-0.162943743482435	2.7815631262525\\
-0.152304609218437	2.77223744705891\\
-0.14477725091542	2.76553106212425\\
-0.13627254509018	2.75797143184814\\
-0.126890685934725	2.74949899799599\\
-0.120240480961924	2.74350576301438\\
-0.109271399852711	2.73346693386774\\
-0.104208416833667	2.72884165515535\\
-0.0919075802425003	2.71743486973948\\
-0.0881763527054109	2.71398016764136\\
-0.0747881948941846	2.70140280561122\\
-0.0721442885771544	2.69892220598275\\
-0.0579029403073767	2.68537074148297\\
-0.0561122244488979	2.68366852299741\\
-0.0412421943145111	2.66933867735471\\
-0.0400801603206413	2.66821971978054\\
-0.0247969724703408	2.65330661322645\\
-0.0240480961923848	2.65257624647837\\
-0.00855888787954408	2.6372745490982\\
-0.00801603206412826	2.63673840286722\\
0.00747988583314737	2.62124248496994\\
0.00801603206412782	2.62070633873896\\
0.0233266486770152	2.60521042084168\\
0.0240480961923843	2.6044800540936\\
0.0389882058144757	2.58917835671343\\
0.0400801603206409	2.58805939913926\\
0.0544708966630596	2.57314629258517\\
0.0561122244488974	2.57144407409961\\
0.0697806217774118	2.55711422845691\\
0.0721442885771539	2.55463362882844\\
0.0849228676899447	2.54108216432866\\
0.0881763527054105	2.53762746223054\\
0.0999027298721873	2.5250501002004\\
0.104208416833667	2.52042482148801\\
0.114724933963928	2.50901803607214\\
0.120240480961924	2.50302480109053\\
0.129393855403837	2.49298597194389\\
0.13627254509018	2.48542634166778\\
0.143913537583111	2.47695390781563\\
0.152304609218437	2.46762822862203\\
0.158287708632811	2.46092184368737\\
0.168336673346693	2.44962909055835\\
0.172519796945656	2.44488977955912\\
0.18436873747495	2.43142739750955\\
0.186612945524132	2.42885771543086\\
0.200400801603206	2.41302145895284\\
0.200570025238683	2.41282565130261\\
0.214411163717496	2.39679358717435\\
0.216432865731463	2.39444447678443\\
0.22812561583566	2.38076152304609\\
0.232464929859719	2.37566531643781\\
0.241714242995323	2.36472945891784\\
0.248496993987976	2.35667894480261\\
0.255179057332447	2.34869739478958\\
0.264529058116232	2.33748303965031\\
0.268521848787055	2.33266533066132\\
0.280561122244489	2.31807510756787\\
0.281744193712368	2.31663326653307\\
0.294865693023195	2.30060120240481\\
0.296593186372745	2.29848372176207\\
0.307884804426556	2.28456913827655\\
0.312625250501002	2.27869892168757\\
0.320790737016945	2.2685370741483\\
0.328657314629258	2.25869676031403\\
0.333584388029848	2.25250501002004\\
0.344689378757515	2.23847407104281\\
0.346266474301825	2.23647294589178\\
0.358859841022653	2.22044088176353\\
0.360721442885771	2.21806219223852\\
0.37136467745965	2.20440881763527\\
0.376753507014028	2.19745492142283\\
0.383763238799898	2.18837675350701\\
0.392785571142285	2.17661980430382\\
0.396055667160588	2.17234468937876\\
0.408249324480346	2.1563126252505\\
0.408817635270541	2.15556374897255\\
0.420381676516927	2.14028056112224\\
0.424849699398798	2.13433597026447\\
0.432411864046547	2.12424849699399\\
0.440881763527054	2.11287147988246\\
0.444339516511447	2.10821643286573\\
0.456174445099843	2.09218436873747\\
0.456913827655311	2.09118013171392\\
0.467956638280627	2.07615230460922\\
0.472945891783567	2.069311782134\\
0.47963909184413	2.06012024048096\\
0.488977955911824	2.04719622794388\\
0.491220970321726	2.04408817635271\\
0.502735279239629	2.02805611222445\\
0.50501002004008	2.02487321171545\\
0.514184056143404	2.01202404809619\\
0.521042084168337	2.00233935898103\\
0.525533998792645	1.99599198396794\\
0.536788276952951	1.97995991983968\\
0.537074148296593	1.97955175499903\\
0.548013140606755	1.96392785571142\\
0.55310621242485	1.95658969567291\\
0.559140227809762	1.94789579158317\\
0.569138276553106	1.93336027162234\\
0.570167995371698	1.93186372745491\\
0.581160780256625	1.91583166332665\\
0.585170340681363	1.90993827213469\\
0.592072506252331	1.8997995991984\\
0.601202404809619	1.88626043566908\\
0.60288506512914	1.88376753507014\\
0.613657428555809	1.86773547094188\\
0.617234468937876	1.86237261764251\\
0.624359807395399	1.85170340681363\\
0.633266533066132	1.8382324554821\\
0.634962578770195	1.83567134268537\\
0.645529835186418	1.81963927855711\\
0.649298597194389	1.81387667185705\\
0.65602749226157	1.80360721442886\\
0.665330661322646	1.78925855592928\\
0.666424536732041	1.7875751503006\\
0.676801290439637	1.77154308617235\\
0.681362725450902	1.76443067766332\\
0.68709754087772	1.75551102204409\\
0.697293488050765	1.73947895791583\\
0.697394789579159	1.73931926820884\\
0.707491839219817	1.72344689378758\\
0.713426853707415	1.71401105067851\\
0.717588754260136	1.70741482965932\\
0.727616224980459	1.69138276553106\\
0.729458917835672	1.68842097989882\\
0.737618467272952	1.67535070140281\\
0.745490981963928	1.66259022427337\\
0.747516929577984	1.65931863727455\\
0.757388887898833	1.64328657314629\\
0.761523046092185	1.63651204421055\\
0.767195261899139	1.62725450901804\\
0.776907704806536	1.61122244488978\\
0.777555110220441	1.61014989124604\\
0.786623917378866	1.59519038076152\\
0.793587174348698	1.58355851759455\\
0.796233549095041	1.57915831663327\\
0.8058099752019	1.56312625250501\\
0.809619238476954	1.5566930590171\\
0.815331077458949	1.54709418837675\\
0.824760127386751	1.5310621242485\\
0.825651302605211	1.52954048970299\\
0.834194099609312	1.51503006012024\\
0.841683366733467	1.50213794937837\\
0.843517559081993	1.49899799599198\\
0.852828753689511	1.48296593186373\\
0.857715430861724	1.47446122603849\\
0.862065825221398	1.46693386773547\\
0.871240825915441	1.45090180360721\\
0.87374749498998	1.44649100489378\\
0.880392419679554	1.43486973947896\\
0.889435763559499	1.4188376753507\\
0.889779559118236	1.41822601676604\\
0.898502620912698	1.40280561122244\\
0.905811623246493	1.38969762513551\\
0.907453086973702	1.38677354709419\\
0.916401385107147	1.37074148296593\\
0.921843687374749	1.36087300710585\\
0.925266721059645	1.35470941883768\\
0.934093357167528	1.33867735470942\\
0.937875751503006	1.33174248081449\\
0.942873854060321	1.32264529058116\\
0.951582880981771	1.30661322645291\\
0.953907815631262	1.30230298107688\\
0.960278674444841	1.29058116232465\\
0.968874008993936	1.27454909819639\\
0.969939879759519	1.27255093581759\\
0.977485082284088	1.25851703406814\\
0.985970511113727	1.24248496993988\\
0.985971943887775	1.24248225378341\\
0.99449669779832	1.22645290581162\\
1.00200400801603	1.21210993397359\\
1.00289562534923	1.21042084168337\\
1.01131686927922	1.19438877755511\\
1.01803607214429	1.18140721482257\\
1.0196288895712	1.17835671342685\\
1.02794868041008	1.1623246492986\\
1.03406813627254	1.15036844889956\\
1.03617320976842	1.14629258517034\\
1.04439495700578	1.13026052104208\\
1.0501002004008	1.11898743521624\\
1.05253126766382	1.11422845691383\\
1.06065827319264	1.09819639278557\\
1.06613226452906	1.08725737084183\\
1.06870549438125	1.08216432865731\\
1.07674095704631	1.06613226452906\\
1.08216432865731	1.0551708299734\\
1.0846980757674	1.0501002004008\\
1.09264509570445	1.03406813627254\\
1.09819639278557	1.02271974124353\\
1.10051095719356	1.01803607214429\\
1.10837253996919	1.00200400801603\\
1.11422845691383	0.989895363173775\\
1.11614584785142	0.985971943887775\\
1.12392490841295	0.969939879759519\\
1.13026052104208	0.956688257674352\\
1.13160422455531	0.953907815631262\\
1.13930359099971	0.937875751503006\\
1.14629258517034	0.92308826148232\\
1.14688733506242	0.921843687374749\\
1.15450975223249	0.905811623246493\\
1.16200401996448	0.889779559118236\\
1.1623246492986	0.889090433191721\\
1.16954433383618	0.87374749498998\\
1.17696578019152	0.857715430861724\\
1.17835671342685	0.854690788684829\\
1.18440805698392	0.841683366733467\\
1.19175920173928	0.825651302605211\\
1.19438877755511	0.819865720098788\\
1.19910142407344	0.809619238476954\\
1.20638471539175	0.793587174348698\\
1.21042084168337	0.784601688074628\\
1.21362472005891	0.777555110220441\\
1.22084253737021	0.761523046092185\\
1.22645290581162	0.748884243791853\\
1.22797801334242	0.745490981963928\\
1.23513267009027	0.729458917835672\\
1.24216907752019	0.713426853707415\\
1.24248496993988	0.712703417267096\\
1.24925490248686	0.697394789579159\\
1.2562314894554	0.681362725450902\\
1.25851703406814	0.676065240973576\\
1.26320880990842	0.665330661322646\\
1.27012770666234	0.649298597194389\\
1.27454909819639	0.63892613146295\\
1.27699375358044	0.633266533066132\\
1.28385703316352	0.617234468937876\\
1.29058116232465	0.60126733760783\\
1.29060887963702	0.601202404809619\\
1.29741856006962	0.585170340681363\\
1.30411699672397	0.569138276553106\\
1.30661322645291	0.56310818675373\\
1.31081116406888	0.55310621242485\\
1.31745878918003	0.537074148296593\\
1.32264529058116	0.524389736561787\\
1.32403350548239	0.521042084168337\\
1.33063218027391	0.50501002004008\\
1.33712404097122	0.488977955911824\\
1.33867735470942	0.485111642225269\\
1.34363556339904	0.472945891783567\\
1.3500809692004	0.456913827655311\\
1.35470941883768	0.445247776242905\\
1.35646711206576	0.440881763527054\\
1.36286777974099	0.424849699398798\\
1.3691656409182	0.408817635270541\\
1.37074148296593	0.404773299370973\\
1.37548237873076	0.392785571142285\\
1.38173772597015	0.376753507014028\\
1.38677354709419	0.363663057257521\\
1.3879224466588	0.360721442885771\\
1.39413686273957	0.344689378757515\\
1.40025208659613	0.328657314629258\\
1.40280561122244	0.32189514718098\\
1.40636046095951	0.312625250501002\\
1.41243672444534	0.296593186372745\\
1.41841668542161	0.280561122244489\\
1.4188376753507	0.279424988336313\\
1.42444445727106	0.264529058116232\\
1.43038727181292	0.248496993987976\\
1.43486973947896	0.236239343140326\\
1.43627218400033	0.232464929859719\\
1.44217922799749	0.216432865731463\\
1.44799298112941	0.200400801603206\\
1.45090180360721	0.192290151612905\\
1.4537891721569	0.18436873747495\\
1.45956873876195	0.168336673346693\\
1.46525751441528	0.152304609218437\\
1.46693386773547	0.147536812403136\\
1.47096011403291	0.13627254509018\\
1.47661614380373	0.120240480961924\\
1.48218373363104	0.104208416833667\\
1.48296593186373	0.101938265686634\\
1.48778765267569	0.0881763527054105\\
1.49332379819697	0.0721442885771539\\
1.49877371175641	0.0561122244488974\\
1.49899799599198	0.0554477144216261\\
1.50427363149831	0.0400801603206409\\
1.50969326695727	0.0240480961923843\\
1.51502874080229	0.00801603206412782\\
1.51503006012024	0.00801203994315158\\
1.52041910140116	-0.00801603206412826\\
1.52572533108552	-0.0240480961923848\\
1.53094933679874	-0.0400801603206413\\
1.5310621242485	-0.0404288675879876\\
1.53622432482812	-0.0561122244488979\\
1.54141999058174	-0.0721442885771544\\
1.546535241786	-0.0881763527054109\\
1.54709418837675	-0.0899428769765749\\
1.551688776779	-0.104208416833667\\
1.55677646444501	-0.120240480961924\\
1.56178542281927	-0.13627254509018\\
1.56312625250501	-0.140606497539393\\
1.56681114278396	-0.152304609218437\\
1.57179318765974	-0.168336673346694\\
1.57669806798192	-0.18436873747495\\
1.57915831663327	-0.192505892844636\\
1.58158931382882	-0.200400801603207\\
1.5864678051518	-0.216432865731463\\
1.59127057938552	-0.232464929859719\\
1.59519038076152	-0.245738048628188\\
1.59602037820382	-0.248496993987976\\
1.60079716268189	-0.264529058116232\\
1.6054995631189	-0.280561122244489\\
1.61012914769989	-0.296593186372745\\
1.61122244488978	-0.300418455466027\\
1.61477729462685	-0.312625250501002\\
1.61938081609172	-0.328657314629258\\
1.62391268479772	-0.344689378757515\\
1.62725450901804	-0.356677572445968\\
1.62840340858265	-0.360721442885771\\
1.63290930970114	-0.376753507014028\\
1.63734461298571	-0.392785571142285\\
1.64171073109856	-0.408817635270541\\
1.64328657314629	-0.414672256156732\\
1.64607917023272	-0.424849699398798\\
1.65041882425813	-0.440881763527054\\
1.65469018763728	-0.456913827655311\\
1.6588945800742	-0.472945891783567\\
1.65931863727455	-0.474580032148421\\
1.66312834154604	-0.488977955911824\\
1.66730552696729	-0.50501002004008\\
1.67141648023444	-0.521042084168337\\
1.67535070140281	-0.536629024874543\\
1.67546529128451	-0.537074148296593\\
1.67954863901878	-0.55310621242485\\
1.68356639100834	-0.569138276553106\\
1.68751972819312	-0.585170340681363\\
1.69138276553106	-0.601090435403715\\
1.69141048284343	-0.601202404809619\\
1.69533503186013	-0.617234468937876\\
1.69919565132172	-0.633266533066132\\
1.70299343812526	-0.649298597194389\\
1.70672945483409	-0.665330661322646\\
1.70741482965932	-0.668309186823601\\
1.71047970488146	-0.681362725450902\\
1.7141847622063	-0.697394789579158\\
1.71782833817753	-0.713426853707415\\
1.7214114151401	-0.729458917835671\\
1.72344689378758	-0.738703910110774\\
1.72497200131837	-0.745490981963928\\
1.72852233483592	-0.761523046092184\\
1.73201231025909	-0.777555110220441\\
1.73544283162421	-0.793587174348697\\
1.7388147715124	-0.809619238476954\\
1.73947895791583	-0.812822767094841\\
1.74219405406295	-0.82565130260521\\
1.7455303014729	-0.841683366733467\\
1.74880791438502	-0.857715430861723\\
1.7520276893305	-0.87374749498998\\
1.75519039270997	-0.889779559118236\\
1.75551102204409	-0.891429711557734\\
1.75836421385233	-0.905811623246493\\
1.7614878128815	-0.92184368737475\\
1.76455409200172	-0.937875751503006\\
1.76756374294779	-0.953907815631263\\
1.77051742834487	-0.969939879759519\\
1.77154308617235	-0.975603791048864\\
1.77346047710993	-0.985971943887776\\
1.77637115270639	-1.00200400801603\\
1.77922541525254	-1.01803607214429\\
1.78202385321948	-1.03406813627255\\
1.78476702669091	-1.0501002004008\\
1.78745546766551	-1.06613226452906\\
1.7875751503006	-1.06685934786461\\
1.7901483801528	-1.08216432865731\\
1.79278818988801	-1.09819639278557\\
1.79537256765055	-1.11422845691383\\
1.79790197103384	-1.13026052104208\\
1.80037682989223	-1.14629258517034\\
1.80279754654409	-1.1623246492986\\
1.80360721442886	-1.16780441192895\\
1.80520003185577	-1.17835671342685\\
1.80756567487247	-1.19438877755511\\
1.80987621161925	-1.21042084168337\\
1.8121319683394	-1.22645290581162\\
1.81433324370427	-1.24248496993988\\
1.81648030891641	-1.25851703406814\\
1.81857340779153	-1.27454909819639\\
1.81963927855711	-1.28292677753027\\
1.82063428513926	-1.29058116232465\\
1.82266346550923	-1.30661322645291\\
1.82463738111443	-1.32264529058116\\
1.82655619601566	-1.33867735470942\\
1.82842004637171	-1.35470941883768\\
1.83022904041777	-1.37074148296593\\
1.83198325842196	-1.38677354709419\\
1.83368275261984	-1.40280561122244\\
1.83532754712663	-1.4188376753507\\
1.83567134268537	-1.42230599971333\\
1.83694387071105	-1.43486973947896\\
1.83851104478601	-1.45090180360721\\
1.84002173704505	-1.46693386773547\\
1.84147588385069	-1.48296593186373\\
1.84287339223335	-1.49899799599198\\
1.84421413968947	-1.51503006012024\\
1.84549797395561	-1.5310621242485\\
1.84672471275821	-1.54709418837675\\
1.84789414353857	-1.56312625250501\\
1.84900602315273	-1.57915831663327\\
1.8500600775457	-1.59519038076152\\
1.85105600139972	-1.61122244488978\\
1.85170340681363	-1.62230202888094\\
1.85199909229475	-1.62725450901804\\
1.85289456737126	-1.64328657314629\\
1.85372935442768	-1.65931863727455\\
1.85450304230896	-1.67535070140281\\
1.85521518624693	-1.69138276553106\\
1.85586530736635	-1.70741482965932\\
1.85645289216113	-1.72344689378758\\
1.85697739193992	-1.73947895791583\\
1.85743822224045	-1.75551102204409\\
1.85783476221172	-1.77154308617234\\
1.8581663539633	-1.7875751503006\\
1.85843230188081	-1.80360721442886\\
1.85863187190667	-1.81963927855711\\
1.85876429078523	-1.83567134268537\\
1.85882874527115	-1.85170340681363\\
1.85882438130004	-1.86773547094188\\
1.85875030312024	-1.88376753507014\\
1.8586055723846	-1.8997995991984\\
1.85838920720086	-1.91583166332665\\
1.85810018113967	-1.93186372745491\\
1.85773742219854	-1.94789579158317\\
1.85729981172059	-1.96392785571142\\
1.85678618326647	-1.97995991983968\\
1.85619532143793	-1.99599198396794\\
1.85552596065151	-2.01202404809619\\
1.85477678386049	-2.02805611222445\\
1.85394642122353	-2.04408817635271\\
1.85303344871797	-2.06012024048096\\
1.85203638669594	-2.07615230460922\\
1.85170340681363	-2.08110478474632\\
1.85096402425816	-2.09218436873747\\
1.84980986194504	-2.10821643286573\\
1.84856756467358	-2.12424849699399\\
1.84723538393176	-2.14028056112224\\
1.84581150654969	-2.1563126252505\\
1.84429405280818	-2.17234468937876\\
1.84268107447124	-2.18837675350701\\
1.8409705527399	-2.20440881763527\\
1.83916039612411	-2.22044088176353\\
1.83724843822968	-2.23647294589178\\
1.83567134268537	-2.24903668565741\\
1.83523737101778	-2.25250501002004\\
1.83313835568119	-2.2685370741483\\
1.83093089661092	-2.28456913827655\\
1.82861245985542	-2.30060120240481\\
1.82618042239087	-2.31663326653307\\
1.82363206922794	-2.33266533066132\\
1.82096459040213	-2.34869739478958\\
1.81963927855711	-2.35635177958396\\
1.81818857080282	-2.36472945891784\\
1.81529996453305	-2.38076152304609\\
1.8122831122404	-2.39679358717435\\
1.80913473079258	-2.41282565130261\\
1.80585142247804	-2.42885771543086\\
1.80360721442886	-2.43941001692876\\
1.8024385369892	-2.44488977955912\\
1.79890051672139	-2.46092184368737\\
1.79521614279353	-2.47695390781563\\
1.79138143889474	-2.49298597194389\\
1.7875751503006	-2.50829095273659\\
1.78739338660917	-2.50901803607214\\
1.78326946333912	-2.5250501002004\\
1.77898147439258	-2.54108216432866\\
1.7745247538334	-2.55711422845691\\
1.77154308617235	-2.56748238129583\\
1.76990175838651	-2.57314629258517\\
1.76511267720294	-2.58917835671343\\
1.7601383404456	-2.60521042084168\\
1.75551102204409	-2.61959233253044\\
1.75497487581311	-2.62124248496994\\
1.7496281134607	-2.6372745490982\\
1.74407694612616	-2.65330661322645\\
1.73947895791583	-2.66613514873682\\
1.73831692392196	-2.66933867735471\\
1.73234548685442	-2.68537074148297\\
1.72614739434424	-2.70140280561122\\
1.72344689378758	-2.70818987746438\\
1.71971566625049	-2.71743486973948\\
1.71304013232617	-2.73346693386774\\
1.70741482965932	-2.74652047249504\\
1.7061103722011	-2.74949899799599\\
1.69891012383408	-2.76553106212425\\
1.69143610370506	-2.7815631262525\\
1.69138276553106	-2.78167509565841\\
1.68365691788803	-2.79759519038076\\
1.67557984389159	-2.81362725450902\\
1.67535070140281	-2.81407237793107\\
1.66715968749157	-2.82965931863727\\
1.65931863727455	-2.84405724240068\\
1.65840866361447	-2.84569138276553\\
1.64927335199124	-2.86172344689379\\
1.64328657314629	-2.87190089013585\\
1.6397574132464	-2.87775551102204\\
1.62982373029581	-2.8937875751503\\
1.62725450901804	-2.8978314455901\\
1.61943156879626	-2.90981963927856\\
1.61122244488978	-2.92202643431353\\
1.60857531675665	-2.92585170340681\\
1.59719561432098	-2.94188376753507\\
1.59519038076152	-2.94464271289486\\
1.58523875260076	-2.95791583166333\\
1.57915831663327	-2.9658107404219\\
1.5726802746891	-2.97394789579158\\
1.56312625250501	-2.98564600747063\\
1.55945965112415	-2.98997995991984\\
1.54709418837675	-3.00424549977693\\
1.5455044615911	-3.0060120240481\\
1.5310621242485	-3.02169538090901\\
1.5307279572433	-3.02204408817635\\
1.51503006012024	-3.03807216018363\\
1.51502598115792	-3.03807615230461\\
1.49899799599198	-3.05344370640559\\
1.49827304395488	-3.05410821643287\\
1.48296593186373	-3.06787012941409\\
1.48031727468776	-3.07014028056112\\
1.46693386773547	-3.08140454787408\\
1.46097385579681	-3.08617234468938\\
1.45090180360721	-3.09409375882733\\
1.44001638721062	-3.10220440881764\\
1.43486973947896	-3.10597882209824\\
1.4188376753507	-3.11710033903772\\
1.41710903881226	-3.11823647294589\\
1.40280561122244	-3.12750636962587\\
1.39167734969043	-3.13426853707415\\
1.38677354709419	-3.1372101514459\\
1.37074148296593	-3.14625626530284\\
1.36307656718187	-3.1503006012024\\
1.35470941883768	-3.15466661391826\\
1.33867735470942	-3.16246635164411\\
1.33012237525181	-3.16633266533066\\
1.32264529058116	-3.16968031772411\\
1.30661322645291	-3.17633463965954\\
1.29075325444766	-3.18236472945892\\
1.29058116232465	-3.18242966225713\\
1.27454909819639	-3.18802432785573\\
1.25851703406814	-3.19309930910985\\
1.24248496993988	-3.19767335714685\\
1.23967836172005	-3.19839679358717\\
1.22645290581162	-3.2017900554151\\
1.21042084168337	-3.20544337144136\\
1.19438877755511	-3.20864327520901\\
1.17835671342685	-3.21140421553854\\
1.1623246492986	-3.21373973178892\\
1.15661502706466	-3.21442885771543\\
1.14629258517034	-3.215673431823\\
1.13026052104208	-3.21720929975852\\
1.11422845691383	-3.21835227700143\\
1.09819639278557	-3.21911252681467\\
1.08216432865731	-3.21949948728802\\
1.06613226452906	-3.21952189989994\\
1.0501002004008	-3.21918783601784\\
1.03406813627254	-3.21850472144465\\
1.01803607214429	-3.21747935911114\\
1.00200400801603	-3.21611795000566\\
0.985998095749708	-3.21442885771543\\
0.985971943887775	-3.21442614155896\\
0.969939879759519	-3.21243069533663\\
0.953907815631262	-3.21011861233941\\
0.937875751503006	-3.2074939838205\\
0.921843687374749	-3.20456038185535\\
0.905811623246493	-3.20132087162849\\
0.892586167338063	-3.19839679358717\\
0.889779559118236	-3.19778513500252\\
0.87374749498998	-3.19398599487374\\
0.857715430861724	-3.18989208776193\\
0.841683366733467	-3.18550468284531\\
0.830931707074993	-3.18236472945892\\
0.825651302605211	-3.18084309491341\\
0.809619238476954	-3.17593153597101\\
0.793587174348698	-3.17073286629194\\
0.780729021259948	-3.16633266533066\\
0.777555110220441	-3.16526011168692\\
0.761523046092185	-3.15955813639492\\
0.745490981963928	-3.15357218820123\\
0.737123833619732	-3.1503006012024\\
0.729458917835672	-3.14733881557017\\
0.713426853707415	-3.14086475809335\\
0.697777947004	-3.13426853707415\\
0.697394789579159	-3.13410884736716\\
0.681362725450902	-3.12715612856513\\
0.665330661322646	-3.11991987857457\\
0.661735583422272	-3.11823647294589\\
0.649298597194389	-3.11247386624583\\
0.633266533066132	-3.10476552161437\\
0.628119885334472	-3.10220440881764\\
0.617234468937876	-3.09684155551826\\
0.601202404809619	-3.08866524528832\\
0.596468517478442	-3.08617234468938\\
0.585170340681363	-3.08027895349741\\
0.569138276553106	-3.07163682472855\\
0.566442776085726	-3.07014028056112\\
0.55310621242485	-3.06280212052261\\
0.537799100333697	-3.05410821643287\\
0.537074148296593	-3.05370005159222\\
0.521042084168337	-3.0444235273177\\
0.510393744430537	-3.03807615230461\\
0.50501002004008	-3.03489325179561\\
0.488977955911824	-3.02515213976752\\
0.483995866906375	-3.02204408817635\\
0.472945891783567	-3.01520356570113\\
0.458503554440966	-3.0060120240481\\
0.456913827655311	-3.00500778702454\\
0.440881763527054	-2.99463500693657\\
0.433868346994613	-2.98997995991984\\
0.424849699398798	-2.98403536906207\\
0.409932488193913	-2.97394789579158\\
0.408817635270541	-2.97319901951363\\
0.392785571142285	-2.96219094658839\\
0.386705135174793	-2.95791583166333\\
0.376753507014028	-2.95096193545089\\
0.364065662450606	-2.94188376753507\\
0.360721442885771	-2.93950507801007\\
0.344689378757515	-2.92785282855784\\
0.341992196266275	-2.92585170340681\\
0.328657314629258	-2.91601138957255\\
0.32044819072278	-2.90981963927856\\
0.312625250501002	-2.90394942268957\\
0.299368989727407	-2.8937875751503\\
0.296593186372745	-2.89167009450756\\
0.280561122244489	-2.87919735205685\\
0.278741749184475	-2.87775551102204\\
0.264529058116232	-2.86654115588278\\
0.258542279271288	-2.86172344689379\\
0.248496993987976	-2.85367293277856\\
0.238727092594649	-2.84569138276553\\
0.232464929859719	-2.84059517615724\\
0.219276137482059	-2.82965931863727\\
0.216432865731463	-2.82731020824735\\
0.200400801603206	-2.81382306215925\\
0.200171659114422	-2.81362725450902\\
0.18436873747495	-2.80016487245945\\
0.181403709616206	-2.79759519038076\\
0.168336673346693	-2.78630243725173\\
0.162943743482434	-2.7815631262525\\
0.152304609218437	-2.77223744705891\\
0.144777250915421	-2.76553106212425\\
0.13627254509018	-2.75797143184814\\
0.126890685934725	-2.74949899799599\\
0.120240480961924	-2.74350576301438\\
0.109271399852711	-2.73346693386774\\
0.104208416833667	-2.72884165515535\\
0.0919075802425008	-2.71743486973948\\
0.0881763527054105	-2.71398016764136\\
0.0747881948941855	-2.70140280561122\\
0.0721442885771539	-2.69892220598275\\
0.0579029403073767	-2.68537074148297\\
0.0561122244488974	-2.68366852299741\\
0.0412421943145115	-2.66933867735471\\
0.0400801603206409	-2.66821971978054\\
0.0247969724703404	-2.65330661322645\\
0.0240480961923843	-2.65257624647837\\
0.00855888787954364	-2.6372745490982\\
0.00801603206412782	-2.63673840286722\\
-0.00747988583314781	-2.62124248496994\\
-0.00801603206412826	-2.62070633873896\\
-0.0233266486770157	-2.60521042084168\\
-0.0240480961923848	-2.6044800540936\\
-0.0389882058144761	-2.58917835671343\\
-0.0400801603206413	-2.58805939913926\\
-0.05447089666306	-2.57314629258517\\
-0.0561122244488979	-2.57144407409961\\
-0.0697806217774123	-2.55711422845691\\
-0.0721442885771544	-2.55463362882844\\
-0.0849228676899451	-2.54108216432866\\
-0.0881763527054109	-2.53762746223054\\
-0.0999027298721868	-2.5250501002004\\
-0.104208416833667	-2.52042482148801\\
-0.114724933963928	-2.50901803607214\\
-0.120240480961924	-2.50302480109053\\
-0.129393855403837	-2.49298597194389\\
-0.13627254509018	-2.48542634166778\\
-0.14391353758311	-2.47695390781563\\
-0.152304609218437	-2.46762822862203\\
-0.158287708632811	-2.46092184368737\\
-0.168336673346694	-2.44962909055835\\
-0.172519796945657	-2.44488977955912\\
-0.18436873747495	-2.43142739750955\\
-0.186612945524132	-2.42885771543086\\
-0.200400801603207	-2.41302145895284\\
-0.200570025238682	-2.41282565130261\\
-0.214411163717496	-2.39679358717435\\
-0.216432865731463	-2.39444447678443\\
-0.228125615835659	-2.38076152304609\\
-0.232464929859719	-2.37566531643781\\
-0.241714242995323	-2.36472945891784\\
-0.248496993987976	-2.35667894480261\\
-0.255179057332446	-2.34869739478958\\
-0.264529058116232	-2.33748303965031\\
-0.268521848787055	-2.33266533066132\\
-0.280561122244489	-2.31807510756787\\
-0.281744193712368	-2.31663326653307\\
-0.294865693023194	-2.30060120240481\\
-0.296593186372745	-2.29848372176207\\
-0.307884804426556	-2.28456913827655\\
-0.312625250501002	-2.27869892168757\\
-0.320790737016945	-2.2685370741483\\
-0.328657314629258	-2.25869676031403\\
-0.333584388029848	-2.25250501002004\\
-0.344689378757515	-2.23847407104281\\
-0.346266474301825	-2.23647294589178\\
-0.358859841022652	-2.22044088176353\\
-0.360721442885771	-2.21806219223852\\
-0.371364677459651	-2.20440881763527\\
-0.376753507014028	-2.19745492142283\\
-0.383763238799898	-2.18837675350701\\
-0.392785571142285	-2.17661980430382\\
-0.396055667160588	-2.17234468937876\\
-0.408249324480346	-2.1563126252505\\
-0.408817635270541	-2.15556374897255\\
-0.420381676516927	-2.14028056112224\\
-0.424849699398798	-2.13433597026447\\
-0.432411864046547	-2.12424849699399\\
-0.440881763527054	-2.11287147988246\\
-0.444339516511446	-2.10821643286573\\
-0.456174445099843	-2.09218436873747\\
-0.456913827655311	-2.09118013171392\\
-0.467956638280628	-2.07615230460922\\
-0.472945891783567	-2.069311782134\\
-0.47963909184413	-2.06012024048096\\
-0.488977955911824	-2.04719622794388\\
-0.491220970321727	-2.04408817635271\\
-0.502735279239629	-2.02805611222445\\
-0.50501002004008	-2.02487321171545\\
-0.514184056143404	-2.01202404809619\\
-0.521042084168337	-2.00233935898103\\
-0.525533998792644	-1.99599198396794\\
-0.53678827695295	-1.97995991983968\\
-0.537074148296593	-1.97955175499903\\
-0.548013140606755	-1.96392785571142\\
-0.55310621242485	-1.95658969567291\\
-0.559140227809761	-1.94789579158317\\
-0.569138276553106	-1.93336027162234\\
-0.570167995371698	-1.93186372745491\\
-0.581160780256625	-1.91583166332665\\
-0.585170340681363	-1.90993827213469\\
-0.592072506252331	-1.8997995991984\\
-0.601202404809619	-1.88626043566908\\
-0.60288506512914	-1.88376753507014\\
-0.613657428555808	-1.86773547094188\\
-0.617234468937876	-1.86237261764251\\
-0.624359807395399	-1.85170340681363\\
-0.633266533066132	-1.8382324554821\\
-0.634962578770196	-1.83567134268537\\
-0.645529835186418	-1.81963927855711\\
-0.649298597194389	-1.81387667185705\\
-0.656027492261569	-1.80360721442886\\
-0.665330661322646	-1.78925855592928\\
-0.666424536732041	-1.7875751503006\\
-0.676801290439637	-1.77154308617234\\
-0.681362725450902	-1.76443067766332\\
-0.687097540877721	-1.75551102204409\\
-0.697293488050765	-1.73947895791583\\
-0.697394789579158	-1.73931926820884\\
-0.707491839219817	-1.72344689378758\\
-0.713426853707415	-1.71401105067852\\
-0.717588754260136	-1.70741482965932\\
-0.727616224980459	-1.69138276553106\\
-0.729458917835671	-1.68842097989882\\
-0.737618467272952	-1.67535070140281\\
-0.745490981963928	-1.66259022427337\\
-0.747516929577984	-1.65931863727455\\
-0.757388887898833	-1.64328657314629\\
-0.761523046092184	-1.63651204421055\\
-0.767195261899139	-1.62725450901804\\
-0.776907704806536	-1.61122244488978\\
-0.777555110220441	-1.61014989124604\\
-0.786623917378867	-1.59519038076152\\
-0.793587174348697	-1.58355851759455\\
-0.796233549095041	-1.57915831663327\\
-0.8058099752019	-1.56312625250501\\
-0.809619238476954	-1.5566930590171\\
-0.815331077458949	-1.54709418837675\\
-0.824760127386751	-1.5310621242485\\
-0.82565130260521	-1.52954048970299\\
-0.834194099609312	-1.51503006012024\\
-0.841683366733467	-1.50213794937837\\
-0.843517559081993	-1.49899799599198\\
-0.852828753689511	-1.48296593186373\\
-0.857715430861723	-1.47446122603849\\
-0.862065825221398	-1.46693386773547\\
-0.871240825915441	-1.45090180360721\\
-0.87374749498998	-1.44649100489378\\
-0.880392419679554	-1.43486973947896\\
-0.889435763559499	-1.4188376753507\\
-0.889779559118236	-1.41822601676604\\
-0.898502620912698	-1.40280561122244\\
-0.905811623246493	-1.38969762513551\\
-0.907453086973702	-1.38677354709419\\
-0.916401385107147	-1.37074148296593\\
-0.92184368737475	-1.36087300710585\\
-0.925266721059645	-1.35470941883768\\
-0.934093357167528	-1.33867735470942\\
-0.937875751503006	-1.33174248081448\\
-0.942873854060321	-1.32264529058116\\
-0.951582880981771	-1.30661322645291\\
-0.953907815631263	-1.30230298107688\\
-0.96027867444484	-1.29058116232465\\
-0.968874008993936	-1.27454909819639\\
-0.969939879759519	-1.27255093581759\\
-0.977485082284089	-1.25851703406814\\
-0.985970511113727	-1.24248496993988\\
-0.985971943887776	-1.24248225378341\\
-0.99449669779832	-1.22645290581162\\
-1.00200400801603	-1.21210993397359\\
-1.00289562534923	-1.21042084168337\\
-1.01131686927922	-1.19438877755511\\
-1.01803607214429	-1.18140721482257\\
-1.0196288895712	-1.17835671342685\\
-1.02794868041008	-1.1623246492986\\
-1.03406813627255	-1.15036844889956\\
-1.03617320976842	-1.14629258517034\\
-1.04439495700578	-1.13026052104208\\
-1.0501002004008	-1.11898743521624\\
-1.05253126766383	-1.11422845691383\\
-1.06065827319264	-1.09819639278557\\
-1.06613226452906	-1.08725737084183\\
-1.06870549438125	-1.08216432865731\\
-1.07674095704631	-1.06613226452906\\
-1.08216432865731	-1.0551708299734\\
-1.0846980757674	-1.0501002004008\\
-1.09264509570445	-1.03406813627255\\
-1.09819639278557	-1.02271974124353\\
-1.10051095719356	-1.01803607214429\\
-1.10837253996919	-1.00200400801603\\
-1.11422845691383	-0.989895363173774\\
-1.11614584785142	-0.985971943887776\\
-1.12392490841295	-0.969939879759519\\
-1.13026052104208	-0.956688257674352\\
-1.13160422455531	-0.953907815631263\\
-1.13930359099971	-0.937875751503006\\
-1.14629258517034	-0.92308826148232\\
-1.14688733506242	-0.92184368737475\\
-1.15450975223249	-0.905811623246493\\
-1.16200401996448	-0.889779559118236\\
-1.1623246492986	-0.889090433191721\\
-1.16954433383618	-0.87374749498998\\
-1.17696578019152	-0.857715430861723\\
-1.17835671342685	-0.854690788684829\\
-1.18440805698392	-0.841683366733467\\
-1.19175920173928	-0.82565130260521\\
-1.19438877755511	-0.819865720098788\\
-1.19910142407344	-0.809619238476954\\
-1.20638471539175	-0.793587174348697\\
-1.21042084168337	-0.784601688074629\\
-1.21362472005891	-0.777555110220441\\
-1.22084253737021	-0.761523046092184\\
-1.22645290581162	-0.748884243791853\\
-1.22797801334242	-0.745490981963928\\
-1.23513267009027	-0.729458917835671\\
-1.24216907752019	-0.713426853707415\\
-1.24248496993988	-0.712703417267096\\
-1.24925490248686	-0.697394789579158\\
-1.2562314894554	-0.681362725450902\\
-1.25851703406814	-0.676065240973576\\
-1.26320880990842	-0.665330661322646\\
-1.27012770666234	-0.649298597194389\\
-1.27454909819639	-0.638926131462949\\
-1.27699375358044	-0.633266533066132\\
-1.28385703316352	-0.617234468937876\\
-1.29058116232465	-0.601267337607829\\
-1.29060887963702	-0.601202404809619\\
-1.29741856006962	-0.585170340681363\\
-1.30411699672397	-0.569138276553106\\
-1.30661322645291	-0.56310818675373\\
-1.31081116406888	-0.55310621242485\\
-1.31745878918003	-0.537074148296593\\
-1.32264529058116	-0.524389736561787\\
-1.32403350548239	-0.521042084168337\\
-1.33063218027391	-0.50501002004008\\
-1.33712404097122	-0.488977955911824\\
-1.33867735470942	-0.48511164222527\\
-1.34363556339904	-0.472945891783567\\
-1.3500809692004	-0.456913827655311\\
-1.35470941883768	-0.445247776242905\\
-1.35646711206576	-0.440881763527054\\
-1.36286777974099	-0.424849699398798\\
-1.3691656409182	-0.408817635270541\\
-1.37074148296593	-0.404773299370972\\
-1.37548237873076	-0.392785571142285\\
-1.38173772597015	-0.376753507014028\\
-1.38677354709419	-0.363663057257521\\
-1.3879224466588	-0.360721442885771\\
-1.39413686273957	-0.344689378757515\\
-1.40025208659613	-0.328657314629258\\
-1.40280561122244	-0.32189514718098\\
-1.40636046095951	-0.312625250501002\\
-1.41243672444534	-0.296593186372745\\
-1.41841668542161	-0.280561122244489\\
-1.4188376753507	-0.279424988336314\\
-1.42444445727107	-0.264529058116232\\
-1.43038727181292	-0.248496993987976\\
-1.43486973947896	-0.236239343140326\\
-1.43627218400033	-0.232464929859719\\
-1.44217922799749	-0.216432865731463\\
-1.44799298112941	-0.200400801603207\\
-1.45090180360721	-0.192290151612905\\
-1.4537891721569	-0.18436873747495\\
-1.45956873876195	-0.168336673346694\\
-1.46525751441528	-0.152304609218437\\
-1.46693386773547	-0.147536812403136\\
-1.47096011403291	-0.13627254509018\\
-1.47661614380372	-0.120240480961924\\
-1.48218373363104	-0.104208416833667\\
-1.48296593186373	-0.101938265686635\\
-1.48778765267569	-0.0881763527054109\\
-1.49332379819697	-0.0721442885771544\\
-1.49877371175641	-0.0561122244488979\\
-1.49899799599198	-0.0554477144216257\\
-1.50427363149831	-0.0400801603206413\\
-1.50969326695727	-0.0240480961923848\\
-1.51502874080229	-0.00801603206412826\\
-1.51503006012024	-0.00801203994315157\\
-1.52041910140116	0.00801603206412782\\
-1.52572533108552	0.0240480961923843\\
-1.53094933679874	0.0400801603206409\\
-1.5310621242485	0.0404288675879877\\
-1.53622432482812	0.0561122244488974\\
-1.54141999058174	0.0721442885771539\\
-1.546535241786	0.0881763527054105\\
-1.54709418837675	0.0899428769765749\\
-1.551688776779	0.104208416833667\\
-1.55677646444501	0.120240480961924\\
-1.56178542281927	0.13627254509018\\
-1.56312625250501	0.140606497539392\\
-1.56681114278396	0.152304609218437\\
-1.57179318765974	0.168336673346693\\
-1.57669806798192	0.18436873747495\\
-1.57915831663327	0.192505892844634\\
-1.58158931382882	0.200400801603206\\
-1.5864678051518	0.216432865731463\\
-1.59127057938552	0.232464929859719\\
-1.59519038076152	0.245738048628186\\
-1.59602037820382	0.248496993987976\\
-1.60079716268189	0.264529058116232\\
-1.6054995631189	0.280561122244489\\
-1.61012914769989	0.296593186372745\\
-1.61122244488978	0.300418455466026\\
-1.61477729462685	0.312625250501002\\
-1.61938081609172	0.328657314629258\\
-1.62391268479772	0.344689378757515\\
-1.62725450901804	0.356677572445966\\
-1.62840340858265	0.360721442885771\\
-1.63290930970114	0.376753507014028\\
-1.63734461298571	0.392785571142285\\
-1.64171073109856	0.408817635270541\\
-1.64328657314629	0.414672256156731\\
-1.64607917023272	0.424849699398798\\
-1.65041882425813	0.440881763527054\\
-1.65469018763728	0.456913827655311\\
-1.6588945800742	0.472945891783567\\
-1.65931863727455	0.474580032148419\\
-1.66312834154604	0.488977955911824\\
-1.66730552696729	0.50501002004008\\
-1.67141648023444	0.521042084168337\\
-1.67535070140281	0.536629024874541\\
-1.67546529128451	0.537074148296593\\
-1.67954863901878	0.55310621242485\\
-1.68356639100834	0.569138276553106\\
-1.68751972819312	0.585170340681363\\
-1.69138276553106	0.601090435403712\\
-1.69141048284343	0.601202404809619\\
-1.69533503186013	0.617234468937876\\
-1.69919565132171	0.633266533066132\\
-1.70299343812526	0.649298597194389\\
-1.70672945483409	0.665330661322646\\
-1.70741482965932	0.668309186823599\\
-1.71047970488146	0.681362725450902\\
-1.7141847622063	0.697394789579159\\
-1.71782833817753	0.713426853707415\\
-1.7214114151401	0.729458917835672\\
-1.72344689378758	0.738703910110772\\
-1.72497200131837	0.745490981963928\\
-1.72852233483592	0.761523046092185\\
-1.73201231025909	0.777555110220441\\
-1.73544283162421	0.793587174348698\\
-1.7388147715124	0.809619238476954\\
-1.73947895791583	0.812822767094839\\
-1.74219405406295	0.825651302605211\\
-1.7455303014729	0.841683366733467\\
-1.74880791438502	0.857715430861724\\
-1.7520276893305	0.87374749498998\\
-1.75519039270997	0.889779559118236\\
-1.75551102204409	0.891429711557732\\
-1.75836421385233	0.905811623246493\\
-1.7614878128815	0.921843687374749\\
-1.76455409200172	0.937875751503006\\
-1.76756374294779	0.953907815631262\\
-1.77051742834487	0.969939879759519\\
-1.77154308617234	0.975603791048862\\
-1.77346047710993	0.985971943887775\\
-1.77637115270639	1.00200400801603\\
-1.77922541525254	1.01803607214429\\
-1.78202385321948	1.03406813627254\\
-1.78476702669091	1.0501002004008\\
-1.78745546766551	1.06613226452906\\
-1.7875751503006	1.06685934786461\\
-1.7901483801528	1.08216432865731\\
-1.79278818988801	1.09819639278557\\
-1.79537256765055	1.11422845691383\\
-1.79790197103384	1.13026052104208\\
-1.80037682989223	1.14629258517034\\
-1.80279754654409	1.1623246492986\\
-1.80360721442886	1.16780441192895\\
-1.80520003185577	1.17835671342685\\
-1.80756567487247	1.19438877755511\\
-1.80987621161925	1.21042084168337\\
-1.8121319683394	1.22645290581162\\
-1.81433324370427	1.24248496993988\\
-1.81648030891641	1.25851703406814\\
-1.81857340779153	1.27454909819639\\
-1.81963927855711	1.28292677753027\\
-1.82063428513926	1.29058116232465\\
-1.82266346550923	1.30661322645291\\
-1.82463738111443	1.32264529058116\\
-1.82655619601566	1.33867735470942\\
-1.82842004637171	1.35470941883768\\
-1.83022904041777	1.37074148296593\\
-1.83198325842196	1.38677354709419\\
-1.83368275261984	1.40280561122244\\
-1.83532754712663	1.4188376753507\\
-1.83567134268537	1.42230599971334\\
-1.83694387071105	1.43486973947896\\
-1.83851104478601	1.45090180360721\\
-1.84002173704505	1.46693386773547\\
-1.84147588385069	1.48296593186373\\
-1.84287339223335	1.49899799599198\\
-1.84421413968947	1.51503006012024\\
-1.84549797395561	1.5310621242485\\
-1.84672471275821	1.54709418837675\\
-1.84789414353857	1.56312625250501\\
-1.84900602315273	1.57915831663327\\
-1.8500600775457	1.59519038076152\\
-1.85105600139972	1.61122244488978\\
-1.85170340681363	1.62230202888095\\
-1.85199909229475	1.62725450901804\\
-1.85289456737126	1.64328657314629\\
-1.85372935442768	1.65931863727455\\
-1.85450304230896	1.67535070140281\\
-1.85521518624693	1.69138276553106\\
-1.85586530736635	1.70741482965932\\
-1.85645289216113	1.72344689378758\\
-1.85697739193992	1.73947895791583\\
-1.85743822224045	1.75551102204409\\
-1.85783476221172	1.77154308617235\\
-1.8581663539633	1.7875751503006\\
-1.85843230188081	1.80360721442886\\
-1.85863187190667	1.81963927855711\\
-1.85876429078523	1.83567134268537\\
-1.85882874527115	1.85170340681363\\
-1.85882438130004	1.86773547094188\\
-1.85875030312025	1.88376753507014\\
-1.8586055723846	1.8997995991984\\
-1.85838920720086	1.91583166332665\\
-1.85810018113967	1.93186372745491\\
-1.85773742219854	1.94789579158317\\
-1.85729981172059	1.96392785571142\\
-1.85678618326647	1.97995991983968\\
-1.85619532143793	1.99599198396794\\
-1.85552596065151	2.01202404809619\\
-1.85477678386049	2.02805611222445\\
-1.85394642122353	2.04408817635271\\
-1.85303344871797	2.06012024048096\\
-1.85203638669594	2.07615230460922\\
-1.85170340681363	2.08110478474631\\
}--cycle;


\addplot[area legend,solid,fill=mycolor4,draw=black,forget plot]
table[row sep=crcr] {%
x	y\\
-1.61122244488978	1.92154062635485\\
-1.61011248878115	1.93186372745491\\
-1.60828067438538	1.94789579158317\\
-1.60633322202377	1.96392785571142\\
-1.60426738422376	1.97995991983968\\
-1.60208031121155	1.99599198396794\\
-1.59976904743487	2.01202404809619\\
-1.59733052793776	2.02805611222445\\
-1.59519038076152	2.04143268533249\\
-1.59476347033196	2.04408817635271\\
-1.59207201391084	2.06012024048096\\
-1.58924270019754	2.07615230460922\\
-1.58627189389273	2.09218436873747\\
-1.58315582550126	2.10821643286573\\
-1.57989058644274	2.12424849699399\\
-1.57915831663327	2.1277110612458\\
-1.5764782936904	2.14028056112224\\
-1.57290909694685	2.1563126252505\\
-1.56917665717801	2.17234468937876\\
-1.56527632738903	2.18837675350701\\
-1.56312625250501	2.19688521514885\\
-1.56120437078298	2.20440881763527\\
-1.55695444683608	2.22044088176353\\
-1.55251945803753	2.23647294589178\\
-1.5478938590338	2.25250501002004\\
-1.54709418837675	2.25519119398684\\
-1.54306685727617	2.2685370741483\\
-1.53803384825403	2.28456913827655\\
-1.53278921849702	2.30060120240481\\
-1.5310621242485	2.30571529601306\\
-1.52731554649612	2.31663326653307\\
-1.52160843687928	2.33266533066132\\
-1.5156649653976	2.34869739478958\\
-1.51503006012024	2.35036312387492\\
-1.50945191118913	2.36472945891784\\
-1.50297916329548	2.38076152304609\\
-1.49899799599198	2.39029006435056\\
-1.49622401985135	2.39679358717435\\
-1.48916536249221	2.41282565130261\\
-1.48296593186373	2.42638221231704\\
-1.48180767585431	2.42885771543086\\
-1.47409706992632	2.44488977955912\\
-1.46693386773547	2.4592173411896\\
-1.4660599246591	2.46092184368737\\
-1.45762008602662	2.47695390781563\\
-1.45090180360721	2.48925412846316\\
-1.4488063422536	2.49298597194389\\
-1.43954643592755	2.50901803607214\\
-1.43486973947896	2.51685864994032\\
-1.42983385031263	2.5250501002004\\
-1.41964657378708	2.54108216432866\\
-1.4188376753507	2.54232403453555\\
-1.40888141830834	2.55711422845691\\
-1.40280561122244	2.56585381624442\\
-1.39755434365011	2.57314629258517\\
-1.38677354709419	2.58766528351743\\
-1.38560707639808	2.58917835671343\\
-1.37292503730557	2.60521042084168\\
-1.37074148296593	2.6078980187191\\
-1.35944897462705	2.62124248496994\\
-1.35470941883768	2.62669324850733\\
-1.34510050105639	2.6372745490982\\
-1.33867735470942	2.64416791174315\\
-1.32975869254771	2.65330661322645\\
-1.32264529058116	2.660420015193\\
-1.31327547851173	2.66933867735471\\
-1.30661322645291	2.67553578751015\\
-1.29546845826235	2.68537074148297\\
-1.29058116232465	2.68959102694447\\
-1.27611129417885	2.70140280561122\\
-1.27454909819639	2.70265225874734\\
-1.25851703406814	2.71477960525793\\
-1.25478519058741	2.71743486973948\\
-1.24248496993988	2.7260271997387\\
-1.23108671830602	2.73346693386774\\
-1.22645290581162	2.73644003213387\\
-1.21042084168337	2.74606380126248\\
-1.20427225725145	2.74949899799599\\
-1.19438877755511	2.75493757065414\\
-1.17835671342685	2.76309270994689\\
-1.17317428761382	2.76553106212425\\
-1.1623246492986	2.7705684654317\\
-1.14629258517034	2.77738784786229\\
-1.13552081220309	2.7815631262525\\
-1.13026052104208	2.78357893720756\\
-1.11422845691383	2.78917463166277\\
-1.09819639278557	2.79418489914963\\
-1.08594047435841	2.79759519038076\\
-1.08216432865731	2.79863663408102\\
-1.06613226452906	2.80255991511815\\
-1.0501002004008	2.80595976525476\\
-1.03406813627255	2.80885428964474\\
-1.01803607214429	2.81126042517164\\
-1.00200400801603	2.81319399879284\\
-0.997331417501761	2.81362725450902\\
-0.985971943887776	2.814675928256\\
-0.969939879759519	2.81571419710024\\
-0.953907815631263	2.81631895101368\\
-0.937875751503006	2.81650189349093\\
-0.92184368737475	2.8162738607838\\
-0.905811623246493	2.81564485848873\\
-0.889779559118236	2.81462409541754\\
-0.878420085504249	2.81362725450902\\
-0.87374749498998	2.81322297105368\\
-0.857715430861723	2.81145662464762\\
-0.841683366733467	2.80932517220469\\
-0.82565130260521	2.80683508253215\\
-0.809619238476954	2.80399215955377\\
-0.793587174348697	2.80080156334239\\
-0.779046159184035	2.79759519038076\\
-0.777555110220441	2.79727052082353\\
-0.761523046092184	2.79343021580375\\
-0.745490981963928	2.78925838468812\\
-0.729458917835671	2.78475793913733\\
-0.718857090061259	2.7815631262525\\
-0.713426853707415	2.7799455667311\\
-0.697394789579158	2.77484091227589\\
-0.681362725450902	2.76941790865144\\
-0.670513087135676	2.76553106212425\\
-0.665330661322646	2.76369440978076\\
-0.649298597194389	2.75769375545286\\
-0.633266533066132	2.75138106085277\\
-0.628713705911684	2.74949899799599\\
-0.617234468937876	2.74480100256469\\
-0.601202404809619	2.73793042458416\\
-0.591235090961048	2.73346693386774\\
-0.585170340681363	2.7307767665998\\
-0.569138276553106	2.72336002335721\\
-0.556838055905579	2.71743486973948\\
-0.55310621242485	2.71565329829787\\
-0.537074148296593	2.70769959745392\\
-0.524849509067469	2.70140280561122\\
-0.521042084168337	2.69945820243091\\
-0.50501002004008	2.69097432997933\\
-0.494783771844951	2.68537074148297\\
-0.488977955911824	2.68221462676709\\
-0.472945891783567	2.67320501962568\\
-0.466283639724746	2.66933867735471\\
-0.456913827655311	2.66394142181914\\
-0.440881763527054	2.65440822496289\\
-0.439082402647768	2.65330661322645\\
-0.424849699398798	2.64465327283536\\
-0.413074526782555	2.6372745490982\\
-0.408817635270541	2.63462480283205\\
-0.392785571142285	2.6243608052396\\
-0.388046015352913	2.62124248496994\\
-0.376753507014028	2.6138584373754\\
-0.363888837953899	2.60521042084168\\
-0.360721442885771	2.60309375284415\\
-0.344689378757515	2.59210255044851\\
-0.340529533563966	2.58917835671343\\
-0.328657314629258	2.5808776181457\\
-0.317876518073337	2.57314629258517\\
-0.312625250501002	2.5693997148328\\
-0.296593186372745	2.55767866334882\\
-0.295837892673827	2.55711422845691\\
-0.280561122244489	2.54575051469963\\
-0.274429644828789	2.54108216432866\\
-0.264529058116232	2.53357697573616\\
-0.253532883154308	2.5250501002004\\
-0.248496993987976	2.52116109576389\\
-0.233116340787461	2.50901803607214\\
-0.232464929859719	2.50850572468606\\
-0.216432865731463	2.49564312778937\\
-0.213188106974364	2.49298597194389\\
-0.200400801603207	2.48254928829144\\
-0.193682519183802	2.47695390781563\\
-0.18436873747495	2.46922073142719\\
-0.174568524887323	2.46092184368737\\
-0.168336673346694	2.45565956781352\\
-0.155824596672748	2.44488977955912\\
-0.152304609218437	2.44186772271686\\
-0.137430801099601	2.42885771543086\\
-0.13627254509018	2.42784693855345\\
-0.120240480961924	2.41360762885972\\
-0.119374742175998	2.41282565130261\\
-0.104208416833667	2.39915131035514\\
-0.101637517500693	2.39679358717435\\
-0.0881763527054109	2.38446763068815\\
-0.0841951854019193	2.38076152304609\\
-0.0721442885771544	2.36955763291399\\
-0.0670336762908581	2.36472945891784\\
-0.0561122244488979	2.35442218420181\\
-0.0501399221213309	2.34869739478958\\
-0.0400801603206413	2.33906197695177\\
-0.0335017835616004	2.33266533066132\\
-0.0240480961923848	2.32347752968626\\
-0.0171079835705498	2.31663326653307\\
-0.00801603206412826	2.30766918771692\\
-0.000948046752016809	2.30060120240481\\
0.00801603206412782	2.29163712358867\\
0.0149877560696653	2.28456913827655\\
0.0240480961923843	2.27538133730149\\
0.0307084585490926	2.2685370741483\\
0.0400801603206409	2.25890165631049\\
0.046222445236839	2.25250501002004\\
0.0561122244488974	2.24219773530401\\
0.0615374941096744	2.23647294589178\\
0.0721442885771539	2.22526905575968\\
0.0766608155678959	2.22044088176353\\
0.0881763527054105	2.20811492527733\\
0.0915990882169153	2.20440881763527\\
0.104208416833667	2.19073447668781\\
0.106358491717682	2.18837675350701\\
0.120240480961924	2.17312666693587\\
0.120944736961574	2.17234468937876\\
0.135364739286021	2.1563126252505\\
0.13627254509018	2.15530184837309\\
0.149624586275568	2.14028056112224\\
0.152304609218437	2.13725850427999\\
0.163728008132806	2.12424849699399\\
0.168336673346693	2.11898622112013\\
0.177679279651716	2.10821643286573\\
0.18436873747495	2.10048325647729\\
0.191482314734416	2.09218436873747\\
0.200400801603206	2.08174768508503\\
0.205140685797516	2.07615230460922\\
0.216432865731463	2.06277739632644\\
0.218657641608958	2.06012024048096\\
0.232038019430153	2.04408817635271\\
0.232464929859719	2.04357586496662\\
0.245294482052848	2.02805611222445\\
0.248496993987976	2.02416710778794\\
0.258420727231885	2.01202404809619\\
0.264529058116232	2.0045188595037\\
0.271418988566259	1.99599198396794\\
0.280561122244489	1.98462827021065\\
0.284291241551083	1.97995991983968\\
0.296593186372745	1.96449229060333\\
0.297039214899656	1.96392785571142\\
0.309683479996603	1.94789579158317\\
0.312625250501002	1.94414921383079\\
0.322212717668509	1.93186372745491\\
0.328657314629258	1.92356298888719\\
0.334625405535997	1.91583166332665\\
0.344689378757515	1.90272379293348\\
0.346922486628156	1.8997995991984\\
0.359116872563305	1.88376753507014\\
0.360721442885771	1.88165086707261\\
0.371216990235094	1.86773547094188\\
0.376753507014028	1.86035142334735\\
0.383207437170299	1.85170340681363\\
0.392785571142285	1.83878966295503\\
0.395088476620054	1.83567134268537\\
0.406877006331786	1.81963927855711\\
0.408817635270541	1.81698953229097\\
0.418579485774504	1.80360721442886\\
0.424849699398798	1.79495387403776\\
0.430176934846032	1.7875751503006\\
0.440881763527054	1.77264469790878\\
0.441669019392435	1.77154308617235\\
0.453092393214548	1.75551102204409\\
0.456913827655311	1.75011376650852\\
0.464421380727239	1.73947895791583\\
0.472945891783567	1.72731323605855\\
0.475648009534167	1.72344689378758\\
0.486794006752962	1.70741482965932\\
0.488977955911824	1.70425871494344\\
0.497867416095764	1.69138276553106\\
0.50501002004008	1.68095428989917\\
0.508840595929194	1.67535070140281\\
0.519726753421353	1.65931863727455\\
0.521042084168337	1.65737403409424\\
0.530555404980253	1.64328657314629\\
0.537074148296593	1.63355130086073\\
0.54128514558374	1.62725450901804\\
0.551928074821154	1.61122244488978\\
0.55310621242485	1.60944087344817\\
0.562520884658426	1.59519038076152\\
0.569138276553106	1.585083470251\\
0.573015347633412	1.57915831663327\\
0.583430809687441	1.56312625250501\\
0.585170340681363	1.56043608523708\\
0.593794926995524	1.54709418837675\\
0.601202404809619	1.53552561496492\\
0.604060554477165	1.5310621242485\\
0.614263529626301	1.51503006012024\\
0.617234468937876	1.51033206468894\\
0.62440445512755	1.49899799599198\\
0.633266533066132	1.4848479947205\\
0.634446080277041	1.48296593186373\\
0.644450832711952	1.46693386773547\\
0.649298597194389	1.45909656106408\\
0.654372519869492	1.45090180360721\\
0.664208689869032	1.43486973947896\\
0.665330661322646	1.43303308713547\\
0.67401359899695	1.4188376753507\\
0.681362725450902	1.40669245774963\\
0.683718539634568	1.40280561122244\\
0.69337756591569	1.38677354709419\\
0.697394789579159	1.38005133311758\\
0.702969211433921	1.37074148296593\\
0.712472211428738	1.35470941883768\\
0.713426853707415	1.35309185931627\\
0.721953017014737	1.33867735470942\\
0.729458917835672	1.32584010346599\\
0.7313317452183	1.32264529058116\\
0.740677067007647	1.30661322645291\\
0.745490981963928	1.29827642076026\\
0.749947267981695	1.29058116232465\\
0.759148058525284	1.27454909819639\\
0.761523046092185	1.27038412361938\\
0.76831154460939	1.25851703406814\\
0.777372290686372	1.24248496993988\\
0.777555110220441	1.24216030038265\\
0.786430653651559	1.22645290581162\\
0.793587174348698	1.213627214645\\
0.795383112446186	1.21042084168337\\
0.804310296988349	1.19438877755511\\
0.809619238476954	1.18475368259986\\
0.813158487753533	1.17835671342685\\
0.821955812894223	1.1623246492986\\
0.825651302605211	1.15553247732173\\
0.830700749518797	1.14629258517034\\
0.839372188236657	1.13026052104208\\
0.841683366733467	1.12595843873776\\
0.848014685479281	1.11422845691383\\
0.856564069847169	1.09819639278557\\
0.857715430861724	1.09602576292418\\
0.865104748160702	1.08216432865731\\
0.873535775099905	1.06613226452906\\
0.87374749498998	1.06572798107372\\
0.881975064979866	1.0501002004008\\
0.889779559118236	1.03506497718107\\
0.890299918799153	1.03406813627254\\
0.898629447604153	1.01803607214429\\
0.905811623246493	1.00402161199574\\
0.906851749946105	1.00200400801603\\
0.915071400596516	0.985971943887775\\
0.921843687374749	0.972586486034307\\
0.923191066272462	0.969939879759519\\
0.931304129372398	0.953907815631262\\
0.937875751503006	0.94075039048492\\
0.939320894316716	0.937875751503006\\
0.947330547492817	0.921843687374749\\
0.953907815631262	0.908503319751152\\
0.955243971041258	0.905811623246493\\
0.96315328331582	0.889779559118236\\
0.969939879759519	0.8758344375812\\
0.970962750081582	0.87374749498998\\
0.978774686026535	0.857715430861724\\
0.985971943887775	0.842732040480446\\
0.986479407382706	0.841683366733467\\
0.994196831064096	0.825651302605211\\
1.00179955554845	0.809619238476954\\
1.00200400801603	0.809185982760772\\
1.00942152496197	0.793587174348698\\
1.01693390146419	0.777555110220441\\
1.01803607214429	0.775188280883066\\
1.02445030961641	0.761523046092185\\
1.03187494369613	0.745490981963928\\
1.03406813627254	0.740718017099647\\
1.03928446599602	0.729458917835672\\
1.04662387524361	0.713426853707415\\
1.0501002004008	0.705759364453157\\
1.05392501730387	0.697394789579159\\
1.0611816344267	0.681362725450902\\
1.06613226452906	0.670295386060036\\
1.06837273160182	0.665330661322646\\
1.07554890734466	0.649298597194389\\
1.08216432865731	0.634307976766395\\
1.0826281239054	0.633266533066132\\
1.08972612981108	0.617234468937876\\
1.09671960727013	0.601202404809619\\
1.09819639278557	0.597792113578492\\
1.10371348877188	0.585170340681363\\
1.11063195689279	0.569138276553106\\
1.11422845691383	0.560717717835118\\
1.11751092321085	0.55310621242485\\
1.12435644545861	0.537074148296593\\
1.13026052104208	0.523057895123392\\
1.13111812454601	0.521042084168337\\
1.13789269380047	0.50501002004008\\
1.14456803763354	0.488977955911824\\
1.14629258517034	0.484802677521606\\
1.15124007716546	0.472945891783567\\
1.1578471485081	0.456913827655311\\
1.1623246492986	0.445919166834502\\
1.16439772446391	0.440881763527054\\
1.17093832854351	0.424849699398798\\
1.17738359767137	0.408817635270541\\
1.17835671342685	0.406379283093181\\
1.18384039529476	0.392785571142285\\
1.19022155276666	0.376753507014028\\
1.19438877755511	0.366160015543915\\
1.19655191858787	0.360721442885771\\
1.20287059404896	0.344689378757515\\
1.20909740780311	0.328657314629258\\
1.21042084168337	0.325222117895742\\
1.21532898133727	0.312625250501002\\
1.22149536367499	0.296593186372745\\
1.22645290581162	0.283534220510624\\
1.2275947227263	0.280561122244489\\
1.2337021375955	0.264529058116232\\
1.23972077261114	0.248496993987976\\
1.24248496993988	0.241057259858942\\
1.24571542558249	0.232464929859719\\
1.25167685511604	0.216432865731463\\
1.25755193846325	0.200400801603206\\
1.25851703406814	0.197745537121656\\
1.2634381954357	0.18436873747495\\
1.26925765836088	0.168336673346693\\
1.27454909819639	0.153554062354551\\
1.27500191367548	0.152304609218437\\
1.28076696369929	0.13627254509018\\
1.28644814606728	0.120240480961924\\
1.29058116232465	0.108428702295168\\
1.29207665346788	0.104208416833667\\
1.29770473969571	0.0881763527054105\\
1.30325100518834	0.0721442885771539\\
1.30661322645291	0.0623093346043338\\
1.30875930199207	0.0561122244488974\\
1.31425362229937	0.0400801603206409\\
1.3196680212829	0.0240480961923843\\
1.32264529058116	0.0151294340306766\\
1.32505136586401	0.00801603206412782\\
1.3304148080451	-0.00801603206412826\\
1.33570008541116	-0.0240480961923848\\
1.33867735470942	-0.0331867976756923\\
1.34095344682717	-0.0400801603206413\\
1.34618859574639	-0.0561122244488979\\
1.35134719757311	-0.0721442885771544\\
1.35470941883768	-0.0827255891680218\\
1.35646524407426	-0.0881763527054109\\
1.36157438764822	-0.104208416833667\\
1.36660846670857	-0.120240480961924\\
1.37074148296593	-0.133584947212763\\
1.37158555308672	-0.13627254509018\\
1.3765706871207	-0.152304609218437\\
1.38148210725868	-0.168336673346694\\
1.38632129259891	-0.18436873747495\\
1.38677354709419	-0.185881810670942\\
1.39117509288114	-0.200400801603207\\
1.39596543227035	-0.216432865731463\\
1.40068469352238	-0.232464929859719\\
1.40280561122244	-0.23975740620047\\
1.40538428971324	-0.248496993987976\\
1.41005484300632	-0.264529058116232\\
1.41465534948129	-0.280561122244489\\
1.4188376753507	-0.295351316165849\\
1.4191940356332	-0.296593186372745\\
1.4237458150046	-0.312625250501002\\
1.4282284552075	-0.328657314629258\\
1.43264320670604	-0.344689378757515\\
1.43486973947896	-0.352880829017598\\
1.43703288051172	-0.360721442885771\\
1.44139826164899	-0.376753507014028\\
1.44569647964082	-0.392785571142285\\
1.44992868785174	-0.408817635270541\\
1.45090180360721	-0.412549478751272\\
1.45415805356282	-0.424849699398798\\
1.45833994169558	-0.440881763527054\\
1.46245636694497	-0.456913827655311\\
1.46650838862278	-0.472945891783567\\
1.46693386773547	-0.474650394281346\\
1.47056559973766	-0.488977955911824\\
1.47456604049386	-0.50501002004008\\
1.47850244808027	-0.521042084168337\\
1.48237579019866	-0.537074148296593\\
1.48296593186373	-0.539549651410414\\
1.48624839816075	-0.55310621242485\\
1.49006858026906	-0.569138276553106\\
1.49382589172678	-0.585170340681363\\
1.49752121047654	-0.601202404809619\\
1.49899799599198	-0.607705927633412\\
1.50119610909336	-0.617234468937876\\
1.50483636086096	-0.633266533066132\\
1.50841463880758	-0.649298597194389\\
1.51193173238232	-0.665330661322646\\
1.51503006012024	-0.679696996365562\\
1.5153950975408	-0.681362725450902\\
1.51885487702331	-0.697394789579158\\
1.52225331347104	-0.713426853707415\\
1.52559110852095	-0.729458917835671\\
1.52886893167208	-0.745490981963928\\
1.5310621242485	-0.756408952483932\\
1.53210621144258	-0.761523046092184\\
1.53532310962606	-0.777555110220441\\
1.53847964119444	-0.793587174348697\\
1.54157638824944	-0.809619238476954\\
1.54461390135166	-0.82565130260521\\
1.54709418837675	-0.838997182766671\\
1.54760165187168	-0.841683366733467\\
1.5505740365611	-0.857715430861723\\
1.55348655045289	-0.87374749498998\\
1.55633965606131	-0.889779559118236\\
1.55913378459536	-0.905811623246493\\
1.56186933611373	-0.92184368737475\\
1.56312625250501	-0.929367289861169\\
1.56457139531872	-0.937875751503006\\
1.56723569397651	-0.953907815631263\\
1.56984046882894	-0.969939879759519\\
1.57238602985503	-0.985971943887776\\
1.57487265565427	-1.00200400801603\\
1.57730059347975	-1.01803607214429\\
1.57915831663327	-1.03060557202073\\
1.57967867631418	-1.03406813627255\\
1.58202824084217	-1.0501002004008\\
1.58431784226701	-1.06613226452906\\
1.58654763393225	-1.08216432865731\\
1.58871773711821	-1.09819639278557\\
1.59082824094773	-1.11422845691383\\
1.59287920226471	-1.13026052104208\\
1.59487064548539	-1.14629258517034\\
1.59519038076152	-1.14894807619056\\
1.59682818575385	-1.1623246492986\\
1.5987296300381	-1.17835671342685\\
1.6005697919074	-1.19438877755511\\
1.60234859378189	-1.21042084168337\\
1.60406592419264	-1.22645290581162\\
1.60572163749197	-1.24248496993988\\
1.60731555353372	-1.25851703406814\\
1.60884745732288	-1.27454909819639\\
1.61031709863407	-1.29058116232465\\
1.61122244488978	-1.30090426342471\\
1.61173158890952	-1.30661322645291\\
1.61309527227241	-1.32264529058116\\
1.61439429698395	-1.33867735470942\\
1.61562829653605	-1.35470941883768\\
1.61679686674454	-1.37074148296593\\
1.61789956520109	-1.38677354709419\\
1.61893591068928	-1.40280561122244\\
1.61990538256408	-1.4188376753507\\
1.62080742009368	-1.43486973947896\\
1.62164142176267	-1.45090180360721\\
1.6224067445356	-1.46693386773547\\
1.62310270307978	-1.48296593186373\\
1.62372856894605	-1.49899799599198\\
1.62428356970646	-1.51503006012024\\
1.62476688804736	-1.5310621242485\\
1.62517766081672	-1.54709418837675\\
1.62551497802411	-1.56312625250501\\
1.62577788179188	-1.57915831663327\\
1.62596536525592	-1.59519038076152\\
1.62607637141434	-1.61122244488978\\
1.62610979192237	-1.62725450901804\\
1.62606446583149	-1.64328657314629\\
1.62593917827105	-1.65931863727455\\
1.62573265907016	-1.67535070140281\\
1.62544358131785	-1.69138276553106\\
1.62507055985917	-1.70741482965932\\
1.62461214972503	-1.72344689378758\\
1.62406684449307	-1.73947895791583\\
1.62343307457727	-1.75551102204409\\
1.62270920544335	-1.77154308617234\\
1.62189353574725	-1.7875751503006\\
1.62098429539374	-1.80360721442886\\
1.61997964351187	-1.81963927855711\\
1.61887766634416	-1.83567134268537\\
1.61767637504605	-1.85170340681363\\
1.61637370339194	-1.86773547094188\\
1.61496750538409	-1.88376753507014\\
1.61345555276042	-1.8997995991984\\
1.61183553239709	-1.91583166332665\\
1.61122244488978	-1.92154062635485\\
1.61011248878115	-1.93186372745491\\
1.60828067438538	-1.94789579158317\\
1.60633322202377	-1.96392785571142\\
1.60426738422377	-1.97995991983968\\
1.60208031121155	-1.99599198396794\\
1.59976904743487	-2.01202404809619\\
1.59733052793776	-2.02805611222445\\
1.59519038076152	-2.04143268533248\\
1.59476347033196	-2.04408817635271\\
1.59207201391084	-2.06012024048096\\
1.58924270019754	-2.07615230460922\\
1.58627189389273	-2.09218436873747\\
1.58315582550126	-2.10821643286573\\
1.57989058644274	-2.12424849699399\\
1.57915831663327	-2.1277110612458\\
1.5764782936904	-2.14028056112224\\
1.57290909694685	-2.1563126252505\\
1.56917665717801	-2.17234468937876\\
1.56527632738902	-2.18837675350701\\
1.56312625250501	-2.19688521514885\\
1.56120437078298	-2.20440881763527\\
1.55695444683608	-2.22044088176353\\
1.55251945803753	-2.23647294589178\\
1.5478938590338	-2.25250501002004\\
1.54709418837675	-2.25519119398684\\
1.54306685727617	-2.2685370741483\\
1.53803384825403	-2.28456913827655\\
1.53278921849702	-2.30060120240481\\
1.5310621242485	-2.30571529601306\\
1.52731554649612	-2.31663326653307\\
1.52160843687928	-2.33266533066132\\
1.5156649653976	-2.34869739478958\\
1.51503006012024	-2.35036312387492\\
1.50945191118913	-2.36472945891784\\
1.50297916329547	-2.38076152304609\\
1.49899799599198	-2.39029006435056\\
1.49622401985135	-2.39679358717435\\
1.48916536249221	-2.41282565130261\\
1.48296593186373	-2.42638221231704\\
1.48180767585431	-2.42885771543086\\
1.47409706992632	-2.44488977955912\\
1.46693386773547	-2.4592173411896\\
1.4660599246591	-2.46092184368737\\
1.45762008602662	-2.47695390781563\\
1.45090180360721	-2.48925412846316\\
1.4488063422536	-2.49298597194389\\
1.43954643592755	-2.50901803607214\\
1.43486973947896	-2.51685864994032\\
1.42983385031263	-2.5250501002004\\
1.41964657378708	-2.54108216432866\\
1.4188376753507	-2.54232403453555\\
1.40888141830834	-2.55711422845691\\
1.40280561122244	-2.56585381624442\\
1.39755434365011	-2.57314629258517\\
1.38677354709419	-2.58766528351743\\
1.38560707639808	-2.58917835671343\\
1.37292503730557	-2.60521042084168\\
1.37074148296593	-2.6078980187191\\
1.35944897462705	-2.62124248496994\\
1.35470941883768	-2.62669324850733\\
1.34510050105639	-2.6372745490982\\
1.33867735470942	-2.64416791174315\\
1.32975869254771	-2.65330661322645\\
1.32264529058116	-2.660420015193\\
1.31327547851173	-2.66933867735471\\
1.30661322645291	-2.67553578751015\\
1.29546845826235	-2.68537074148297\\
1.29058116232465	-2.68959102694447\\
1.27611129417885	-2.70140280561122\\
1.27454909819639	-2.70265225874734\\
1.25851703406814	-2.71477960525793\\
1.2547851905874	-2.71743486973948\\
1.24248496993988	-2.7260271997387\\
1.23108671830602	-2.73346693386774\\
1.22645290581162	-2.73644003213387\\
1.21042084168337	-2.74606380126247\\
1.20427225725145	-2.74949899799599\\
1.19438877755511	-2.75493757065414\\
1.17835671342685	-2.76309270994689\\
1.17317428761382	-2.76553106212425\\
1.1623246492986	-2.7705684654317\\
1.14629258517034	-2.77738784786229\\
1.13552081220309	-2.7815631262525\\
1.13026052104208	-2.78357893720756\\
1.11422845691383	-2.78917463166277\\
1.09819639278557	-2.79418489914963\\
1.08594047435841	-2.79759519038076\\
1.08216432865731	-2.79863663408102\\
1.06613226452906	-2.80255991511815\\
1.0501002004008	-2.80595976525476\\
1.03406813627254	-2.80885428964474\\
1.01803607214429	-2.81126042517164\\
1.00200400801603	-2.81319399879284\\
0.997331417501759	-2.81362725450902\\
0.985971943887775	-2.814675928256\\
0.969939879759519	-2.81571419710024\\
0.953907815631262	-2.81631895101368\\
0.937875751503006	-2.81650189349093\\
0.921843687374749	-2.8162738607838\\
0.905811623246493	-2.81564485848873\\
0.889779559118236	-2.81462409541754\\
0.878420085504253	-2.81362725450902\\
0.87374749498998	-2.81322297105368\\
0.857715430861724	-2.81145662464762\\
0.841683366733467	-2.80932517220469\\
0.825651302605211	-2.80683508253215\\
0.809619238476954	-2.80399215955377\\
0.793587174348698	-2.80080156334239\\
0.779046159184034	-2.79759519038076\\
0.777555110220441	-2.79727052082353\\
0.761523046092185	-2.79343021580375\\
0.745490981963928	-2.78925838468812\\
0.729458917835672	-2.78475793913733\\
0.718857090061259	-2.7815631262525\\
0.713426853707415	-2.7799455667311\\
0.697394789579159	-2.77484091227589\\
0.681362725450902	-2.76941790865144\\
0.670513087135677	-2.76553106212425\\
0.665330661322646	-2.76369440978076\\
0.649298597194389	-2.75769375545286\\
0.633266533066132	-2.75138106085277\\
0.628713705911686	-2.74949899799599\\
0.617234468937876	-2.74480100256469\\
0.601202404809619	-2.73793042458416\\
0.591235090961049	-2.73346693386774\\
0.585170340681363	-2.7307767665998\\
0.569138276553106	-2.72336002335721\\
0.556838055905581	-2.71743486973948\\
0.55310621242485	-2.71565329829787\\
0.537074148296593	-2.70769959745392\\
0.524849509067469	-2.70140280561122\\
0.521042084168337	-2.69945820243091\\
0.50501002004008	-2.69097432997933\\
0.494783771844953	-2.68537074148297\\
0.488977955911824	-2.68221462676709\\
0.472945891783567	-2.67320501962568\\
0.466283639724747	-2.66933867735471\\
0.456913827655311	-2.66394142181914\\
0.440881763527054	-2.65440822496289\\
0.439082402647767	-2.65330661322645\\
0.424849699398798	-2.64465327283535\\
0.413074526782554	-2.6372745490982\\
0.408817635270541	-2.63462480283205\\
0.392785571142285	-2.6243608052396\\
0.388046015352912	-2.62124248496994\\
0.376753507014028	-2.6138584373754\\
0.363888837953899	-2.60521042084168\\
0.360721442885771	-2.60309375284415\\
0.344689378757515	-2.59210255044851\\
0.340529533563966	-2.58917835671343\\
0.328657314629258	-2.5808776181457\\
0.317876518073336	-2.57314629258517\\
0.312625250501002	-2.5693997148328\\
0.296593186372745	-2.55767866334882\\
0.295837892673827	-2.55711422845691\\
0.280561122244489	-2.54575051469963\\
0.274429644828789	-2.54108216432866\\
0.264529058116232	-2.53357697573616\\
0.253532883154308	-2.5250501002004\\
0.248496993987976	-2.52116109576389\\
0.233116340787461	-2.50901803607214\\
0.232464929859719	-2.50850572468606\\
0.216432865731463	-2.49564312778937\\
0.213188106974364	-2.49298597194389\\
0.200400801603206	-2.48254928829144\\
0.193682519183801	-2.47695390781563\\
0.18436873747495	-2.46922073142719\\
0.174568524887323	-2.46092184368737\\
0.168336673346693	-2.45565956781352\\
0.155824596672748	-2.44488977955912\\
0.152304609218437	-2.44186772271686\\
0.137430801099602	-2.42885771543086\\
0.13627254509018	-2.42784693855345\\
0.120240480961924	-2.41360762885972\\
0.119374742175997	-2.41282565130261\\
0.104208416833667	-2.39915131035514\\
0.101637517500693	-2.39679358717435\\
0.0881763527054105	-2.38446763068815\\
0.0841951854019198	-2.38076152304609\\
0.0721442885771539	-2.36955763291399\\
0.0670336762908576	-2.36472945891784\\
0.0561122244488974	-2.35442218420181\\
0.0501399221213309	-2.34869739478958\\
0.0400801603206409	-2.33906197695177\\
0.0335017835616004	-2.33266533066132\\
0.0240480961923843	-2.32347752968626\\
0.0171079835705493	-2.31663326653307\\
0.00801603206412782	-2.30766918771692\\
0.00094804675201682	-2.30060120240481\\
-0.00801603206412826	-2.29163712358866\\
-0.0149877560696652	-2.28456913827655\\
-0.0240480961923848	-2.27538133730149\\
-0.0307084585490926	-2.2685370741483\\
-0.0400801603206413	-2.25890165631049\\
-0.046222445236839	-2.25250501002004\\
-0.0561122244488979	-2.24219773530401\\
-0.0615374941096744	-2.23647294589178\\
-0.0721442885771544	-2.22526905575968\\
-0.0766608155678955	-2.22044088176353\\
-0.0881763527054109	-2.20811492527733\\
-0.0915990882169157	-2.20440881763527\\
-0.104208416833667	-2.19073447668781\\
-0.106358491717682	-2.18837675350701\\
-0.120240480961924	-2.17312666693587\\
-0.120944736961575	-2.17234468937876\\
-0.13536473928602	-2.1563126252505\\
-0.13627254509018	-2.15530184837309\\
-0.149624586275567	-2.14028056112224\\
-0.152304609218437	-2.13725850427999\\
-0.163728008132807	-2.12424849699399\\
-0.168336673346694	-2.11898622112013\\
-0.177679279651716	-2.10821643286573\\
-0.18436873747495	-2.10048325647729\\
-0.191482314734416	-2.09218436873747\\
-0.200400801603207	-2.08174768508503\\
-0.205140685797516	-2.07615230460922\\
-0.216432865731463	-2.06277739632644\\
-0.218657641608958	-2.06012024048096\\
-0.232038019430153	-2.04408817635271\\
-0.232464929859719	-2.04357586496662\\
-0.245294482052848	-2.02805611222445\\
-0.248496993987976	-2.02416710778794\\
-0.258420727231884	-2.01202404809619\\
-0.264529058116232	-2.0045188595037\\
-0.271418988566259	-1.99599198396794\\
-0.280561122244489	-1.98462827021066\\
-0.284291241551083	-1.97995991983968\\
-0.296593186372745	-1.96449229060333\\
-0.297039214899656	-1.96392785571142\\
-0.309683479996603	-1.94789579158317\\
-0.312625250501002	-1.94414921383079\\
-0.322212717668508	-1.93186372745491\\
-0.328657314629258	-1.92356298888719\\
-0.334625405535997	-1.91583166332665\\
-0.344689378757515	-1.90272379293348\\
-0.346922486628155	-1.8997995991984\\
-0.359116872563304	-1.88376753507014\\
-0.360721442885771	-1.88165086707261\\
-0.371216990235094	-1.86773547094188\\
-0.376753507014028	-1.86035142334735\\
-0.383207437170299	-1.85170340681363\\
-0.392785571142285	-1.83878966295503\\
-0.395088476620055	-1.83567134268537\\
-0.406877006331786	-1.81963927855711\\
-0.408817635270541	-1.81698953229097\\
-0.418579485774505	-1.80360721442886\\
-0.424849699398798	-1.79495387403776\\
-0.430176934846032	-1.7875751503006\\
-0.440881763527054	-1.77264469790878\\
-0.441669019392435	-1.77154308617234\\
-0.453092393214548	-1.75551102204409\\
-0.456913827655311	-1.75011376650852\\
-0.46442138072724	-1.73947895791583\\
-0.472945891783567	-1.72731323605855\\
-0.475648009534168	-1.72344689378758\\
-0.486794006752962	-1.70741482965932\\
-0.488977955911824	-1.70425871494344\\
-0.497867416095764	-1.69138276553106\\
-0.50501002004008	-1.68095428989917\\
-0.508840595929195	-1.67535070140281\\
-0.519726753421353	-1.65931863727455\\
-0.521042084168337	-1.65737403409424\\
-0.530555404980253	-1.64328657314629\\
-0.537074148296593	-1.63355130086073\\
-0.54128514558374	-1.62725450901804\\
-0.551928074821154	-1.61122244488978\\
-0.55310621242485	-1.60944087344817\\
-0.562520884658426	-1.59519038076152\\
-0.569138276553106	-1.585083470251\\
-0.573015347633412	-1.57915831663327\\
-0.583430809687441	-1.56312625250501\\
-0.585170340681363	-1.56043608523708\\
-0.593794926995524	-1.54709418837675\\
-0.601202404809619	-1.53552561496492\\
-0.604060554477165	-1.5310621242485\\
-0.614263529626301	-1.51503006012024\\
-0.617234468937876	-1.51033206468894\\
-0.62440445512755	-1.49899799599198\\
-0.633266533066132	-1.4848479947205\\
-0.634446080277042	-1.48296593186373\\
-0.644450832711952	-1.46693386773547\\
-0.649298597194389	-1.45909656106408\\
-0.654372519869492	-1.45090180360721\\
-0.664208689869032	-1.43486973947896\\
-0.665330661322646	-1.43303308713547\\
-0.67401359899695	-1.4188376753507\\
-0.681362725450902	-1.40669245774963\\
-0.683718539634568	-1.40280561122244\\
-0.693377565915689	-1.38677354709419\\
-0.697394789579158	-1.38005133311758\\
-0.702969211433921	-1.37074148296593\\
-0.712472211428738	-1.35470941883768\\
-0.713426853707415	-1.35309185931627\\
-0.721953017014737	-1.33867735470942\\
-0.729458917835671	-1.32584010346599\\
-0.7313317452183	-1.32264529058116\\
-0.740677067007647	-1.30661322645291\\
-0.745490981963928	-1.29827642076026\\
-0.749947267981696	-1.29058116232465\\
-0.759148058525283	-1.27454909819639\\
-0.761523046092184	-1.27038412361938\\
-0.76831154460939	-1.25851703406814\\
-0.777372290686371	-1.24248496993988\\
-0.777555110220441	-1.24216030038265\\
-0.786430653651558	-1.22645290581162\\
-0.793587174348697	-1.213627214645\\
-0.795383112446185	-1.21042084168337\\
-0.804310296988349	-1.19438877755511\\
-0.809619238476954	-1.18475368259986\\
-0.813158487753533	-1.17835671342685\\
-0.821955812894222	-1.1623246492986\\
-0.82565130260521	-1.15553247732173\\
-0.830700749518797	-1.14629258517034\\
-0.839372188236656	-1.13026052104208\\
-0.841683366733467	-1.12595843873776\\
-0.848014685479281	-1.11422845691383\\
-0.856564069847169	-1.09819639278557\\
-0.857715430861723	-1.09602576292418\\
-0.865104748160702	-1.08216432865731\\
-0.873535775099906	-1.06613226452906\\
-0.87374749498998	-1.06572798107372\\
-0.881975064979865	-1.0501002004008\\
-0.889779559118236	-1.03506497718107\\
-0.890299918799153	-1.03406813627255\\
-0.898629447604153	-1.01803607214429\\
-0.905811623246493	-1.00402161199574\\
-0.906851749946105	-1.00200400801603\\
-0.915071400596516	-0.985971943887776\\
-0.92184368737475	-0.972586486034305\\
-0.923191066272462	-0.969939879759519\\
-0.931304129372398	-0.953907815631263\\
-0.937875751503006	-0.94075039048492\\
-0.939320894316716	-0.937875751503006\\
-0.947330547492816	-0.92184368737475\\
-0.953907815631263	-0.908503319751151\\
-0.955243971041259	-0.905811623246493\\
-0.96315328331582	-0.889779559118236\\
-0.969939879759519	-0.875834437581199\\
-0.970962750081582	-0.87374749498998\\
-0.978774686026536	-0.857715430861723\\
-0.985971943887776	-0.842732040480445\\
-0.986479407382707	-0.841683366733467\\
-0.994196831064096	-0.82565130260521\\
-1.00179955554845	-0.809619238476954\\
-1.00200400801603	-0.809185982760772\\
-1.00942152496197	-0.793587174348697\\
-1.01693390146419	-0.777555110220441\\
-1.01803607214429	-0.775188280883064\\
-1.02445030961641	-0.761523046092184\\
-1.03187494369613	-0.745490981963928\\
-1.03406813627255	-0.740718017099646\\
-1.03928446599602	-0.729458917835671\\
-1.04662387524361	-0.713426853707415\\
-1.0501002004008	-0.705759364453155\\
-1.05392501730387	-0.697394789579158\\
-1.0611816344267	-0.681362725450902\\
-1.06613226452906	-0.670295386060035\\
-1.06837273160182	-0.665330661322646\\
-1.07554890734466	-0.649298597194389\\
-1.08216432865731	-0.634307976766393\\
-1.0826281239054	-0.633266533066132\\
-1.08972612981108	-0.617234468937876\\
-1.09671960727013	-0.601202404809619\\
-1.09819639278557	-0.59779211357849\\
-1.10371348877188	-0.585170340681363\\
-1.11063195689279	-0.569138276553106\\
-1.11422845691383	-0.560717717835118\\
-1.11751092321085	-0.55310621242485\\
-1.12435644545861	-0.537074148296593\\
-1.13026052104208	-0.523057895123392\\
-1.13111812454601	-0.521042084168337\\
-1.13789269380047	-0.50501002004008\\
-1.14456803763354	-0.488977955911824\\
-1.14629258517034	-0.484802677521605\\
-1.15124007716546	-0.472945891783567\\
-1.1578471485081	-0.456913827655311\\
-1.1623246492986	-0.445919166834502\\
-1.16439772446391	-0.440881763527054\\
-1.17093832854351	-0.424849699398798\\
-1.17738359767137	-0.408817635270541\\
-1.17835671342685	-0.406379283093181\\
-1.18384039529476	-0.392785571142285\\
-1.19022155276666	-0.376753507014028\\
-1.19438877755511	-0.366160015543915\\
-1.19655191858787	-0.360721442885771\\
-1.20287059404896	-0.344689378757515\\
-1.20909740780311	-0.328657314629258\\
-1.21042084168337	-0.325222117895742\\
-1.21532898133727	-0.312625250501002\\
-1.22149536367499	-0.296593186372745\\
-1.22645290581162	-0.283534220510624\\
-1.2275947227263	-0.280561122244489\\
-1.2337021375955	-0.264529058116232\\
-1.23972077261114	-0.248496993987976\\
-1.24248496993988	-0.241057259858942\\
-1.24571542558249	-0.232464929859719\\
-1.25167685511604	-0.216432865731463\\
-1.25755193846325	-0.200400801603207\\
-1.25851703406814	-0.197745537121656\\
-1.2634381954357	-0.18436873747495\\
-1.26925765836088	-0.168336673346694\\
-1.27454909819639	-0.153554062354551\\
-1.27500191367548	-0.152304609218437\\
-1.28076696369929	-0.13627254509018\\
-1.28644814606728	-0.120240480961924\\
-1.29058116232465	-0.108428702295168\\
-1.29207665346788	-0.104208416833667\\
-1.29770473969571	-0.0881763527054109\\
-1.30325100518834	-0.0721442885771544\\
-1.30661322645291	-0.0623093346043342\\
-1.30875930199207	-0.0561122244488979\\
-1.31425362229937	-0.0400801603206413\\
-1.3196680212829	-0.0240480961923848\\
-1.32264529058116	-0.0151294340306766\\
-1.32505136586401	-0.00801603206412826\\
-1.3304148080451	0.00801603206412782\\
-1.33570008541116	0.0240480961923843\\
-1.33867735470942	0.0331867976756918\\
-1.34095344682717	0.0400801603206409\\
-1.34618859574639	0.0561122244488974\\
-1.35134719757311	0.0721442885771539\\
-1.35470941883768	0.0827255891680218\\
-1.35646524407426	0.0881763527054105\\
-1.36157438764822	0.104208416833667\\
-1.36660846670857	0.120240480961924\\
-1.37074148296593	0.133584947212763\\
-1.37158555308672	0.13627254509018\\
-1.3765706871207	0.152304609218437\\
-1.38148210725868	0.168336673346693\\
-1.38632129259891	0.18436873747495\\
-1.38677354709419	0.185881810670943\\
-1.39117509288114	0.200400801603206\\
-1.39596543227035	0.216432865731463\\
-1.40068469352238	0.232464929859719\\
-1.40280561122244	0.23975740620047\\
-1.40538428971324	0.248496993987976\\
-1.41005484300632	0.264529058116232\\
-1.41465534948129	0.280561122244489\\
-1.4188376753507	0.29535131616585\\
-1.4191940356332	0.296593186372745\\
-1.4237458150046	0.312625250501002\\
-1.4282284552075	0.328657314629258\\
-1.43264320670604	0.344689378757515\\
-1.43486973947896	0.352880829017598\\
-1.43703288051172	0.360721442885771\\
-1.44139826164899	0.376753507014028\\
-1.44569647964082	0.392785571142285\\
-1.44992868785174	0.408817635270541\\
-1.45090180360721	0.412549478751272\\
-1.45415805356282	0.424849699398798\\
-1.45833994169558	0.440881763527054\\
-1.46245636694497	0.456913827655311\\
-1.46650838862278	0.472945891783567\\
-1.46693386773547	0.474650394281346\\
-1.47056559973766	0.488977955911824\\
-1.47456604049386	0.50501002004008\\
-1.47850244808027	0.521042084168337\\
-1.48237579019867	0.537074148296593\\
-1.48296593186373	0.539549651410414\\
-1.48624839816075	0.55310621242485\\
-1.49006858026906	0.569138276553106\\
-1.49382589172678	0.585170340681363\\
-1.49752121047654	0.601202404809619\\
-1.49899799599198	0.607705927633412\\
-1.50119610909336	0.617234468937876\\
-1.50483636086096	0.633266533066132\\
-1.50841463880758	0.649298597194389\\
-1.51193173238232	0.665330661322646\\
-1.51503006012024	0.679696996365563\\
-1.5153950975408	0.681362725450902\\
-1.51885487702331	0.697394789579159\\
-1.52225331347104	0.713426853707415\\
-1.52559110852095	0.729458917835672\\
-1.52886893167208	0.745490981963928\\
-1.5310621242485	0.756408952483931\\
-1.53210621144258	0.761523046092185\\
-1.53532310962606	0.777555110220441\\
-1.53847964119444	0.793587174348698\\
-1.54157638824944	0.809619238476954\\
-1.54461390135166	0.825651302605211\\
-1.54709418837675	0.838997182766671\\
-1.54760165187168	0.841683366733467\\
-1.5505740365611	0.857715430861724\\
-1.55348655045289	0.87374749498998\\
-1.55633965606131	0.889779559118236\\
-1.55913378459536	0.905811623246493\\
-1.56186933611373	0.921843687374749\\
-1.56312625250501	0.929367289861167\\
-1.56457139531872	0.937875751503006\\
-1.56723569397651	0.953907815631262\\
-1.56984046882894	0.969939879759519\\
-1.57238602985503	0.985971943887775\\
-1.57487265565427	1.00200400801603\\
-1.57730059347975	1.01803607214429\\
-1.57915831663327	1.03060557202073\\
-1.57967867631418	1.03406813627254\\
-1.58202824084217	1.0501002004008\\
-1.58431784226701	1.06613226452906\\
-1.58654763393225	1.08216432865731\\
-1.58871773711821	1.09819639278557\\
-1.59082824094773	1.11422845691383\\
-1.59287920226471	1.13026052104208\\
-1.59487064548539	1.14629258517034\\
-1.59519038076152	1.14894807619056\\
-1.59682818575385	1.1623246492986\\
-1.5987296300381	1.17835671342685\\
-1.6005697919074	1.19438877755511\\
-1.60234859378189	1.21042084168337\\
-1.60406592419264	1.22645290581162\\
-1.60572163749197	1.24248496993988\\
-1.60731555353372	1.25851703406814\\
-1.60884745732288	1.27454909819639\\
-1.61031709863407	1.29058116232465\\
-1.61122244488978	1.30090426342471\\
-1.61173158890952	1.30661322645291\\
-1.61309527227241	1.32264529058116\\
-1.61439429698395	1.33867735470942\\
-1.61562829653605	1.35470941883768\\
-1.61679686674454	1.37074148296593\\
-1.61789956520109	1.38677354709419\\
-1.61893591068928	1.40280561122244\\
-1.61990538256409	1.4188376753507\\
-1.62080742009368	1.43486973947896\\
-1.62164142176267	1.45090180360721\\
-1.6224067445356	1.46693386773547\\
-1.62310270307978	1.48296593186373\\
-1.62372856894605	1.49899799599198\\
-1.62428356970646	1.51503006012024\\
-1.62476688804736	1.5310621242485\\
-1.62517766081672	1.54709418837675\\
-1.62551497802411	1.56312625250501\\
-1.62577788179189	1.57915831663327\\
-1.62596536525592	1.59519038076152\\
-1.62607637141434	1.61122244488978\\
-1.62610979192237	1.62725450901804\\
-1.62606446583149	1.64328657314629\\
-1.62593917827105	1.65931863727455\\
-1.62573265907016	1.67535070140281\\
-1.62544358131785	1.69138276553106\\
-1.62507055985917	1.70741482965932\\
-1.62461214972503	1.72344689378758\\
-1.62406684449307	1.73947895791583\\
-1.62343307457727	1.75551102204409\\
-1.62270920544335	1.77154308617235\\
-1.62189353574725	1.7875751503006\\
-1.62098429539374	1.80360721442886\\
-1.61997964351187	1.81963927855711\\
-1.61887766634416	1.83567134268537\\
-1.61767637504605	1.85170340681363\\
-1.61637370339194	1.86773547094188\\
-1.61496750538409	1.88376753507014\\
-1.61345555276042	1.8997995991984\\
-1.61183553239709	1.91583166332665\\
-1.61122244488978	1.92154062635485\\
}--cycle;


\addplot[area legend,solid,fill=mycolor5,draw=black,forget plot]
table[row sep=crcr] {%
x	y\\
-1.43486973947896	1.57614782301339\\
-1.43471943107953	1.57915831663327\\
-1.43381826014665	1.59519038076152\\
-1.43280952025882	1.61122244488978\\
-1.43169085204059	1.62725450901804\\
-1.4304598001939	1.64328657314629\\
-1.42911381027157	1.65931863727455\\
-1.4276502253059	1.67535070140281\\
-1.42606628228583	1.69138276553106\\
-1.42435910847566	1.70741482965932\\
-1.4225257175681	1.72344689378758\\
-1.42056300566399	1.73947895791583\\
-1.4188376753507	1.75270116524991\\
-1.41846804884804	1.75551102204409\\
-1.41623805524873	1.77154308617235\\
-1.41386758907841	1.7875751503006\\
-1.41135291389638	1.80360721442886\\
-1.40869014715433	1.81963927855711\\
-1.40587525464174	1.83567134268537\\
-1.40290404467388	1.85170340681363\\
-1.40280561122244	1.85221236475073\\
-1.39976673853409	1.86773547094188\\
-1.39646204917176	1.88376753507014\\
-1.39298525898563	1.8997995991984\\
-1.38933130901527	1.91583166332665\\
-1.38677354709419	1.92656076872958\\
-1.38549017444567	1.93186372745491\\
-1.38144940352448	1.94789579158317\\
-1.37721190834076	1.96392785571142\\
-1.37277155563299	1.97995991983968\\
-1.37074148296593	1.98700845858851\\
-1.36810682049629	1.99599198396794\\
-1.36321095195326	2.01202404809619\\
-1.35808788971171	2.02805611222445\\
-1.35470941883768	2.03821482835606\\
-1.35271501593282	2.04408817635271\\
-1.34706954954247	2.06012024048096\\
-1.34116778578055	2.07615230460922\\
-1.33867735470942	2.08268121363305\\
-1.33496551159711	2.09218436873747\\
-1.3284588477304	2.10821643286573\\
-1.32264529058116	2.12195623172791\\
-1.32164969012634	2.12424849699399\\
-1.31446628330718	2.14028056112224\\
-1.30696119426195	2.1563126252505\\
-1.30661322645291	2.1570351362201\\
-1.29902049060337	2.17234468937876\\
-1.29071804569593	2.18837675350701\\
-1.29058116232465	2.18863394640411\\
-1.28191145320043	2.20440881763527\\
-1.27454909819639	2.2172723129112\\
-1.27267209655242	2.22044088176353\\
-1.26287839719742	2.23647294589178\\
-1.25851703406814	2.2433713531395\\
-1.25252169552617	2.25250501002004\\
-1.24248496993988	2.26725508244213\\
-1.24157692121844	2.2685370741483\\
-1.2298926066466	2.28456913827655\\
-1.22645290581162	2.28914032936524\\
-1.2174438451519	2.30060120240481\\
-1.21042084168337	2.3092557276721\\
-1.20414790670024	2.31663326653307\\
-1.19438877755511	2.32776952279695\\
-1.18987869651441	2.33266533066132\\
-1.17835671342685	2.34481988990147\\
-1.17447920853874	2.34869739478958\\
-1.1623246492986	2.36052717493355\\
-1.15775251694412	2.36472945891784\\
-1.14629258517034	2.37499623458497\\
-1.13944940232465	2.38076152304609\\
-1.13026052104208	2.38831840762883\\
-1.11925105511328	2.39679358717435\\
-1.11422845691383	2.40057318173519\\
-1.09819639278557	2.41182795528761\\
-1.09667379717982	2.41282565130261\\
-1.08216432865731	2.42214004892704\\
-1.07080768800794	2.42885771543086\\
-1.06613226452906	2.4315707992682\\
-1.0501002004008	2.4401641685971\\
-1.0404549230817	2.44488977955912\\
-1.03406813627255	2.4479659469966\\
-1.01803607214429	2.45501471015489\\
-1.00309160510989	2.46092184368737\\
-1.00200400801603	2.46134532663581\\
-0.985971943887776	2.46699364640741\\
-0.969939879759519	2.47198542314756\\
-0.953907815631263	2.47634725722471\\
-0.951348053022255	2.47695390781563\\
-0.937875751503006	2.48011007091374\\
-0.92184368737475	2.48329258491622\\
-0.905811623246493	2.48591508836163\\
-0.889779559118236	2.48799746033493\\
-0.87374749498998	2.4895582254375\\
-0.857715430861723	2.49061462510628\\
-0.841683366733467	2.49118268369602\\
-0.82565130260521	2.49127726968805\\
-0.809619238476954	2.49091215235856\\
-0.793587174348697	2.49010005421127\\
-0.777555110220441	2.48885269945374\\
-0.761523046092184	2.48718085877326\\
-0.745490981963928	2.48509439064656\\
-0.729458917835671	2.48260227939784\\
-0.713426853707415	2.47971267020162\\
-0.699954552188166	2.47695390781563\\
-0.697394789579158	2.47643563232263\\
-0.681362725450902	2.47279233708659\\
-0.665330661322646	2.46877490054165\\
-0.649298597194389	2.46438859305215\\
-0.637613172394768	2.46092184368737\\
-0.633266533066132	2.45964550325484\\
-0.617234468937876	2.45456571746513\\
-0.601202404809619	2.44913301662021\\
-0.589448840589472	2.44488977955912\\
-0.585170340681363	2.44335978778237\\
-0.569138276553106	2.43726879886481\\
-0.55310621242485	2.43083597158032\\
-0.548430788945969	2.42885771543086\\
-0.537074148296593	2.42409432489073\\
-0.521042084168337	2.41703015422659\\
-0.511937632205333	2.41282565130261\\
-0.50501002004008	2.40965263427974\\
-0.488977955911824	2.40197261032635\\
-0.47861633688666	2.39679358717435\\
-0.472945891783567	2.39398111250709\\
-0.456913827655311	2.38569774828604\\
-0.447724946372745	2.38076152304609\\
-0.440881763527054	2.37711175756346\\
-0.424849699398798	2.36823470705649\\
-0.418742219050504	2.36472945891784\\
-0.408817635270541	2.35907140958386\\
-0.392785571142285	2.34960756594041\\
-0.391293580145418	2.34869739478958\\
-0.376753507014028	2.33988194566119\\
-0.365231523926469	2.33266533066132\\
-0.360721442885771	2.3298571615125\\
-0.344689378757515	2.31956044169135\\
-0.340265649230689	2.31663326653307\\
-0.328657314629258	2.30899339723786\\
-0.316276698443884	2.30060120240481\\
-0.312625250501002	2.298138761015\\
-0.296593186372745	2.28701987652264\\
-0.293153485537765	2.28456913827655\\
-0.280561122244489	2.27563866976863\\
-0.270812795449883	2.2685370741483\\
-0.264529058116232	2.26397936875109\\
-0.249116841161576	2.25250501002004\\
-0.248496993987976	2.25204542916615\\
-0.232464929859719	2.23986734300685\\
-0.228103566730434	2.23647294589178\\
-0.216432865731463	2.22742240388654\\
-0.20764169030169	2.22044088176353\\
-0.200400801603207	2.21470970355198\\
-0.187683817264239	2.20440881763527\\
-0.18436873747495	2.20173183890034\\
-0.168336673346694	2.18849214306497\\
-0.16819978997541	2.18837675350701\\
-0.152304609218437	2.17501187444054\\
-0.149198527430876	2.17234468937876\\
-0.13627254509018	2.16127082520807\\
-0.13060223360748	2.1563126252505\\
-0.120240480961924	2.14727077441581\\
-0.112387424107648	2.14028056112224\\
-0.104208416833667	2.13301329770614\\
-0.0945324789276519	2.12424849699399\\
-0.0881763527054109	2.11849976935194\\
-0.0770174803552489	2.10821643286573\\
-0.0721442885771544	2.10373136403057\\
-0.0598240675612109	2.09218436873747\\
-0.0561122244488979	2.08870905833842\\
-0.042935305942095	2.07615230460922\\
-0.0400801603206413	2.07343363204895\\
-0.0263355695258069	2.06012024048096\\
-0.0240480961923848	2.05790566911636\\
-0.0100104349689827	2.04408817635271\\
-0.00801603206412826	2.04212555842685\\
0.00605341413827421	2.02805611222445\\
0.00801603206412782	2.02609349429859\\
0.0218682737844108	2.01202404809619\\
0.0240480961923843	2.0098094767316\\
0.0374454978510013	1.99599198396794\\
0.0400801603206409	1.99327331140766\\
0.0527955690962666	1.97995991983968\\
0.0561122244488974	1.97648460944062\\
0.0679281635703945	1.96392785571142\\
0.0721442885771539	1.95944278687626\\
0.082852209135698	1.94789579158317\\
0.0881763527054105	1.94214706394112\\
0.0975759386864971	1.93186372745491\\
0.104208416833667	1.9245964640388\\
0.112106938595429	1.91583166332665\\
0.120240480961924	1.90678981249196\\
0.126452192853365	1.8997995991984\\
0.13627254509018	1.88872573502771\\
0.14061812331798	1.88376753507014\\
0.152304609218437	1.87040265600366\\
0.154610626440267	1.86773547094188\\
0.168336673346693	1.85181879637159\\
0.16843510679813	1.85170340681363\\
0.182094263918885	1.83567134268537\\
0.18436873747495	1.83299436395044\\
0.195597114389196	1.81963927855711\\
0.200400801603206	1.81390810034557\\
0.208948104277145	1.80360721442886\\
0.216432865731463	1.79455667242361\\
0.222151133760788	1.7875751503006\\
0.232464929859719	1.77493748328741\\
0.235209729534101	1.77154308617235\\
0.248127367485314	1.75551102204409\\
0.248496993987976	1.7550514411902\\
0.260910749817074	1.73947895791583\\
0.264529058116232	1.73492125251862\\
0.273561692861503	1.72344689378758\\
0.280561122244489	1.71451642527965\\
0.286082555369444	1.70741482965932\\
0.296593186372745	1.69383350377715\\
0.298475405129103	1.69138276553106\\
0.31074681341069	1.67535070140281\\
0.312625250501002	1.672888260013\\
0.322901385421872	1.65931863727455\\
0.328657314629258	1.65167876797935\\
0.334936639563524	1.64328657314629\\
0.344689378757515	1.63018168417632\\
0.346853715889065	1.62725450901804\\
0.358661223665631	1.61122244488978\\
0.360721442885771	1.60841427574096\\
0.370363623606131	1.59519038076152\\
0.376753507014028	1.58637493163314\\
0.381954431452667	1.57915831663327\\
0.392785571142285	1.56403642365584\\
0.393433997955205	1.56312625250501\\
0.404822061190901	1.54709418837675\\
0.408817635270541	1.54143613904278\\
0.416106673164823	1.5310621242485\\
0.424849699398798	1.51853530825889\\
0.427284817757179	1.51503006012024\\
0.438370195933025	1.49899799599198\\
0.440881763527054	1.49534823050935\\
0.44936577793113	1.48296593186373\\
0.456913827655311	1.47187009297541\\
0.460258527737727	1.46693386773547\\
0.47105948890644	1.45090180360721\\
0.472945891783567	1.44808932893996\\
0.481780563993476	1.43486973947896\\
0.488977955911824	1.42401669850271\\
0.492401403989198	1.4188376753507\\
0.502935212529043	1.40280561122244\\
0.50501002004008	1.39963259419958\\
0.513393889713681	1.38677354709419\\
0.521042084168337	1.37494598588992\\
0.523753976742535	1.37074148296593\\
0.534036970397823	1.35470941883768\\
0.537074148296593	1.34994602829754\\
0.54424314245709	1.33867735470942\\
0.55310621242485	1.32462354673062\\
0.554351490046198	1.32264529058116\\
0.564399151523312	1.30661322645291\\
0.569138276553106	1.29899224575859\\
0.574360665418839	1.29058116232465\\
0.584232230108001	1.27454909819639\\
0.585170340681363	1.27301910641964\\
0.594051326219487	1.25851703406814\\
0.601202404809619	1.24672820700097\\
0.603774129129969	1.24248496993988\\
0.613432791996306	1.22645290581162\\
0.617234468937876	1.22009677958938\\
0.62301858987926	1.21042084168337\\
0.632513854057083	1.19438877755511\\
0.633266533066132	1.19311243712257\\
0.641966158922827	1.17835671342685\\
0.649298597194389	1.16579139866338\\
0.651321859979967	1.1623246492986\\
0.660624815805226	1.14629258517034\\
0.665330661322646	1.13811357789636\\
0.669850938713261	1.13026052104208\\
0.679002064889276	1.11422845691383\\
0.681362725450902	1.11006688618478\\
0.688101307851878	1.09819639278557\\
0.69710495342446	1.08216432865731\\
0.697394789579159	1.08164605316432\\
0.706079738533738	1.06613226452906\\
0.713426853707415	1.05285896278679\\
0.714956057633846	1.0501002004008\\
0.723792570976108	1.03406813627254\\
0.729458917835672	1.0236844437265\\
0.732546518759912	1.01803607214429\\
0.741245729576007	1.00200400801603\\
0.745490981963928	0.994112426718699\\
0.749879009420214	0.985971943887775\\
0.75844473699409	0.969939879759519\\
0.761523046092185	0.964134766588891\\
0.766958806572806	0.953907815631262\\
0.775394727267325	0.937875751503006\\
0.777555110220441	0.933742479012859\\
0.783790805290208	0.921843687374749\\
0.79210045799249	0.905811623246493\\
0.793587174348698	0.902925705513879\\
0.800379530336358	0.889779559118236\\
0.808566321619408	0.87374749498998\\
0.809619238476954	0.871673675404656\\
0.816729146876445	0.857715430861724\\
0.824796355889865	0.841683366733467\\
0.825651302605211	0.839974664477626\\
0.832843470353536	0.825651302605211\\
0.840794253455332	0.809619238476954\\
0.841683366733467	0.807815950229083\\
0.848725975563253	0.793587174348698\\
0.856563370704004	0.777555110220441\\
0.857715430861724	0.775183763382832\\
0.864379804955127	0.761523046092185\\
0.872106735825037	0.745490981963928\\
0.87374749498998	0.742063235457539\\
0.879807776186825	0.729458917835672\\
0.887427056135494	0.713426853707415\\
0.889779559118236	0.708438342098458\\
0.895012388955042	0.697394789579159\\
0.902526724693145	0.681362725450902\\
0.905811623246493	0.674291841868642\\
0.909995831124626	0.665330661322646\\
0.917407826216055	0.649298597194389\\
0.921843687374749	0.639605210166722\\
0.924759984175293	0.633266533066132\\
0.932072142327726	0.617234468937876\\
0.937875751503006	0.604358567907728\\
0.939306427983218	0.601202404809619\\
0.946521156144502	0.585170340681363\\
0.953640079044779	0.569138276553106\\
0.953907815631262	0.568531625962186\\
0.960756056219921	0.55310621242485\\
0.967780754132099	0.537074148296593\\
0.969939879759519	0.532105663628523\\
0.974777739858769	0.521042084168337\\
0.981710376076964	0.50501002004008\\
0.985971943887775	0.495049758631863\\
0.988586815811709	0.488977955911824\\
0.99542946636084	0.472945891783567\\
1.00200400801603	0.45733731060375\\
1.00218360635947	0.456913827655311\\
1.00893826228546	0.440881763527054\\
1.01560260849886	0.424849699398798\\
1.01803607214429	0.418942565866309\\
1.02223671829137	0.408817635270541\\
1.02881559873643	0.392785571142285\\
1.03406813627254	0.379829674451507\\
1.03532450668279	0.376753507014028\\
1.04181969582964	0.360721442885771\\
1.04822789550428	0.344689378757515\\
1.0501002004008	0.339963767795497\\
1.05461421190746	0.328657314629258\\
1.06094089149501	0.312625250501002\\
1.06613226452906	0.299306270210088\\
1.06719817822434	0.296593186372745\\
1.07344490733363	0.280561122244489\\
1.07960754402498	0.264529058116232\\
1.08216432865731	0.257811391612406\\
1.08573861773676	0.248496993987976\\
1.09182313616956	0.232464929859719\\
1.09782580665023	0.216432865731463\\
1.09819639278557	0.215435169716469\\
1.10382818894802	0.200400801603206\\
1.10975436746823	0.18436873747495\\
1.11422845691383	0.172116267907533\\
1.11562073759942	0.168336673346693\\
1.12147167987081	0.152304609218437\\
1.12724301738029	0.13627254509018\\
1.13026052104208	0.127797365544665\\
1.13297542122262	0.120240480961924\\
1.13867283245662	0.104208416833667\\
1.14429246627308	0.0881763527054105\\
1.14629258517034	0.0824110642442843\\
1.14988753346395	0.0721442885771539\\
1.15543435708166	0.0561122244488974\\
1.16090507180902	0.0400801603206409\\
1.1623246492986	0.0358778763363548\\
1.16635911296884	0.0240480961923843\\
1.17175794302227	0.00801603206412782\\
1.17708217805388	-0.00801603206412826\\
1.17835671342685	-0.0118935369522394\\
1.1823911770971	-0.0240480961923848\\
1.1876442649766	-0.0400801603206413\\
1.19282412066439	-0.0561122244488979\\
1.19438877755511	-0.0610080323132715\\
1.19798372584872	-0.0721442885771544\\
1.20309298566784	-0.0881763527054109\\
1.20813022818325	-0.104208416833667\\
1.21042084168337	-0.111585955694638\\
1.21313574186391	-0.120240480961924\\
1.21810275454298	-0.13627254509018\\
1.22299881944837	-0.152304609218437\\
1.22645290581162	-0.163765482258007\\
1.22784518649722	-0.168336673346694\\
1.23267120256016	-0.18436873747495\\
1.23742719710257	-0.200400801603207\\
1.24211438380454	-0.216432865731463\\
1.24248496993988	-0.217714857437629\\
1.24679493303642	-0.232464929859719\\
1.25141163716841	-0.248496993987976\\
1.2559602494358	-0.264529058116232\\
1.25851703406814	-0.273662714996768\\
1.26046950850622	-0.280561122244489\\
1.26494737463194	-0.296593186372745\\
1.26935772516235	-0.312625250501002\\
1.27370159972485	-0.328657314629258\\
1.27454909819639	-0.331825883481582\\
1.2780285849571	-0.344689378757515\\
1.28230065775349	-0.360721442885771\\
1.28650662502303	-0.376753507014028\\
1.29058116232465	-0.392528378245189\\
1.29064836143032	-0.392785571142285\\
1.29478180847173	-0.408817635270541\\
1.29884938139001	-0.424849699398798\\
1.30285195075063	-0.440881763527054\\
1.30661322645291	-0.456191316685715\\
1.30679282479634	-0.456913827655311\\
1.31072117881957	-0.472945891783567\\
1.31458455512255	-0.488977955911824\\
1.31838372277035	-0.50501002004008\\
1.32211941541806	-0.521042084168337\\
1.32264529058116	-0.523334349434413\\
1.32583504765005	-0.537074148296593\\
1.32949353116982	-0.55310621242485\\
1.33308829272707	-0.569138276553106\\
1.33661996515266	-0.585170340681363\\
1.33867735470942	-0.594673495785783\\
1.34010803118963	-0.601202404809619\\
1.34355990970964	-0.617234468937876\\
1.34694824139766	-0.633266533066132\\
1.35027355767898	-0.649298597194389\\
1.35353635514788	-0.665330661322646\\
1.35470941883768	-0.671204009319294\\
1.35676341650596	-0.681362725450902\\
1.35994224867448	-0.697394789579158\\
1.36305782068158	-0.713426853707415\\
1.36611052617763	-0.729458917835671\\
1.36910072380099	-0.745490981963928\\
1.37074148296593	-0.754474507343357\\
1.37204493428342	-0.761523046092184\\
1.37494628488202	-0.777555110220441\\
1.37778409179572	-0.793587174348697\\
1.380558607853	-0.809619238476954\\
1.38327005023437	-0.82565130260521\\
1.38591860037884	-0.841683366733467\\
1.38677354709419	-0.846986325458802\\
1.38852514086549	-0.857715430861723\\
1.39107767773152	-0.87374749498998\\
1.39356590308185	-0.889779559118236\\
1.39598988725695	-0.905811623246493\\
1.39834966361338	-0.92184368737475\\
1.40064522826933	-0.937875751503006\\
1.40280561122244	-0.953398857694159\\
1.40287734648678	-0.953907815631263\\
1.40506851430965	-0.969939879759519\\
1.40719363867873	-0.985971943887776\\
1.40925259695101	-1.00200400801603\\
1.41124522734124	-1.01803607214429\\
1.41317132849114	-1.03406813627255\\
1.41503065900035	-1.0501002004008\\
1.4168229369183	-1.06613226452906\\
1.418547839196	-1.08216432865731\\
1.4188376753507	-1.0849741854515\\
1.42021886398182	-1.09819639278557\\
1.42182337322298	-1.11422845691383\\
1.42335795274132	-1.13026052104208\\
1.42482214355233	-1.14629258517034\\
1.42621544244241	-1.1623246492986\\
1.4275373012074	-1.17835671342685\\
1.42878712584424	-1.19438877755511\\
1.4299642756944	-1.21042084168337\\
1.43106806253739	-1.22645290581162\\
1.43209774963311	-1.24248496993988\\
1.43305255071117	-1.25851703406814\\
1.4339316289056	-1.27454909819639\\
1.43473409563297	-1.29058116232465\\
1.43486973947896	-1.29359165594453\\
1.435463735431	-1.30661322645291\\
1.43611501710031	-1.32264529058116\\
1.43668581376201	-1.33867735470942\\
1.43717505973574	-1.35470941883768\\
1.43758163205316	-1.37074148296593\\
1.43790434902837	-1.38677354709419\\
1.43814196875914	-1.40280561122244\\
1.43829318755633	-1.4188376753507\\
1.43835663829887	-1.43486973947896\\
1.43833088871135	-1.45090180360721\\
1.43821443956137	-1.46693386773547\\
1.43800572277337	-1.48296593186373\\
1.43770309945568	-1.49899799599198\\
1.43730485783734	-1.51503006012024\\
1.43680921111101	-1.5310621242485\\
1.43621429517806	-1.54709418837675\\
1.43551816629188	-1.56312625250501\\
1.43486973947896	-1.57614782301339\\
1.43471943107953	-1.57915831663327\\
1.43381826014665	-1.59519038076152\\
1.43280952025882	-1.61122244488978\\
1.43169085204059	-1.62725450901804\\
1.4304598001939	-1.64328657314629\\
1.42911381027157	-1.65931863727455\\
1.4276502253059	-1.67535070140281\\
1.42606628228583	-1.69138276553106\\
1.42435910847566	-1.70741482965932\\
1.4225257175681	-1.72344689378758\\
1.42056300566399	-1.73947895791583\\
1.4188376753507	-1.75270116524991\\
1.41846804884804	-1.75551102204409\\
1.41623805524873	-1.77154308617234\\
1.41386758907841	-1.7875751503006\\
1.41135291389638	-1.80360721442886\\
1.40869014715433	-1.81963927855711\\
1.40587525464174	-1.83567134268537\\
1.40290404467388	-1.85170340681363\\
1.40280561122244	-1.85221236475073\\
1.39976673853409	-1.86773547094188\\
1.39646204917176	-1.88376753507014\\
1.39298525898563	-1.8997995991984\\
1.38933130901527	-1.91583166332665\\
1.38677354709419	-1.92656076872958\\
1.38549017444567	-1.93186372745491\\
1.38144940352448	-1.94789579158317\\
1.37721190834076	-1.96392785571142\\
1.37277155563299	-1.97995991983968\\
1.37074148296593	-1.98700845858851\\
1.36810682049629	-1.99599198396794\\
1.36321095195326	-2.01202404809619\\
1.35808788971171	-2.02805611222445\\
1.35470941883768	-2.03821482835606\\
1.35271501593282	-2.04408817635271\\
1.34706954954247	-2.06012024048096\\
1.34116778578055	-2.07615230460922\\
1.33867735470942	-2.08268121363305\\
1.33496551159711	-2.09218436873747\\
1.3284588477304	-2.10821643286573\\
1.32264529058116	-2.12195623172791\\
1.32164969012635	-2.12424849699399\\
1.31446628330718	-2.14028056112224\\
1.30696119426195	-2.1563126252505\\
1.30661322645291	-2.1570351362201\\
1.29902049060337	-2.17234468937876\\
1.29071804569593	-2.18837675350701\\
1.29058116232465	-2.18863394640411\\
1.28191145320043	-2.20440881763527\\
1.27454909819639	-2.2172723129112\\
1.27267209655242	-2.22044088176353\\
1.26287839719742	-2.23647294589178\\
1.25851703406814	-2.2433713531395\\
1.25252169552617	-2.25250501002004\\
1.24248496993988	-2.26725508244213\\
1.24157692121844	-2.2685370741483\\
1.2298926066466	-2.28456913827655\\
1.22645290581162	-2.28914032936524\\
1.2174438451519	-2.30060120240481\\
1.21042084168337	-2.3092557276721\\
1.20414790670024	-2.31663326653307\\
1.19438877755511	-2.32776952279695\\
1.18987869651441	-2.33266533066132\\
1.17835671342685	-2.34481988990147\\
1.17447920853874	-2.34869739478958\\
1.1623246492986	-2.36052717493355\\
1.15775251694412	-2.36472945891784\\
1.14629258517034	-2.37499623458497\\
1.13944940232465	-2.38076152304609\\
1.13026052104208	-2.38831840762883\\
1.11925105511328	-2.39679358717435\\
1.11422845691383	-2.40057318173519\\
1.09819639278557	-2.41182795528761\\
1.09667379717982	-2.41282565130261\\
1.08216432865731	-2.42214004892704\\
1.07080768800794	-2.42885771543086\\
1.06613226452906	-2.4315707992682\\
1.0501002004008	-2.4401641685971\\
1.0404549230817	-2.44488977955912\\
1.03406813627254	-2.4479659469966\\
1.01803607214429	-2.45501471015489\\
1.00309160510989	-2.46092184368737\\
1.00200400801603	-2.46134532663581\\
0.985971943887775	-2.46699364640741\\
0.969939879759519	-2.47198542314756\\
0.953907815631262	-2.47634725722471\\
0.951348053022254	-2.47695390781563\\
0.937875751503006	-2.48011007091374\\
0.921843687374749	-2.48329258491622\\
0.905811623246493	-2.48591508836163\\
0.889779559118236	-2.48799746033493\\
0.87374749498998	-2.4895582254375\\
0.857715430861724	-2.49061462510628\\
0.841683366733467	-2.49118268369602\\
0.825651302605211	-2.49127726968805\\
0.809619238476954	-2.49091215235856\\
0.793587174348698	-2.49010005421127\\
0.777555110220441	-2.48885269945374\\
0.761523046092185	-2.48718085877326\\
0.745490981963928	-2.48509439064656\\
0.729458917835672	-2.48260227939784\\
0.713426853707415	-2.47971267020162\\
0.699954552188167	-2.47695390781563\\
0.697394789579159	-2.47643563232263\\
0.681362725450902	-2.47279233708659\\
0.665330661322646	-2.46877490054165\\
0.649298597194389	-2.46438859305215\\
0.637613172394769	-2.46092184368737\\
0.633266533066132	-2.45964550325484\\
0.617234468937876	-2.45456571746513\\
0.601202404809619	-2.44913301662021\\
0.589448840589471	-2.44488977955912\\
0.585170340681363	-2.44335978778237\\
0.569138276553106	-2.43726879886481\\
0.55310621242485	-2.43083597158032\\
0.548430788945969	-2.42885771543086\\
0.537074148296593	-2.42409432489073\\
0.521042084168337	-2.41703015422659\\
0.511937632205333	-2.41282565130261\\
0.50501002004008	-2.40965263427974\\
0.488977955911824	-2.40197261032635\\
0.47861633688666	-2.39679358717435\\
0.472945891783567	-2.39398111250709\\
0.456913827655311	-2.38569774828604\\
0.447724946372745	-2.38076152304609\\
0.440881763527054	-2.37711175756346\\
0.424849699398798	-2.36823470705649\\
0.418742219050504	-2.36472945891784\\
0.408817635270541	-2.35907140958386\\
0.392785571142285	-2.34960756594041\\
0.391293580145418	-2.34869739478958\\
0.376753507014028	-2.33988194566119\\
0.365231523926469	-2.33266533066132\\
0.360721442885771	-2.3298571615125\\
0.344689378757515	-2.31956044169135\\
0.340265649230688	-2.31663326653307\\
0.328657314629258	-2.30899339723786\\
0.316276698443884	-2.30060120240481\\
0.312625250501002	-2.298138761015\\
0.296593186372745	-2.28701987652264\\
0.293153485537765	-2.28456913827655\\
0.280561122244489	-2.27563866976863\\
0.270812795449883	-2.2685370741483\\
0.264529058116232	-2.26397936875109\\
0.249116841161576	-2.25250501002004\\
0.248496993987976	-2.25204542916615\\
0.232464929859719	-2.23986734300685\\
0.228103566730433	-2.23647294589178\\
0.216432865731463	-2.22742240388654\\
0.20764169030169	-2.22044088176353\\
0.200400801603206	-2.21470970355198\\
0.187683817264239	-2.20440881763527\\
0.18436873747495	-2.20173183890034\\
0.168336673346693	-2.18849214306497\\
0.168199789975409	-2.18837675350701\\
0.152304609218437	-2.17501187444054\\
0.149198527430876	-2.17234468937876\\
0.13627254509018	-2.16127082520807\\
0.13060223360748	-2.1563126252505\\
0.120240480961924	-2.14727077441581\\
0.112387424107648	-2.14028056112224\\
0.104208416833667	-2.13301329770613\\
0.0945324789276524	-2.12424849699399\\
0.0881763527054105	-2.11849976935194\\
0.0770174803552489	-2.10821643286573\\
0.0721442885771539	-2.10373136403057\\
0.0598240675612114	-2.09218436873747\\
0.0561122244488974	-2.08870905833842\\
0.0429353059420955	-2.07615230460922\\
0.0400801603206409	-2.07343363204895\\
0.0263355695258069	-2.06012024048096\\
0.0240480961923843	-2.05790566911636\\
0.0100104349689827	-2.04408817635271\\
0.00801603206412782	-2.04212555842685\\
-0.0060534141382742	-2.02805611222445\\
-0.00801603206412826	-2.02609349429859\\
-0.0218682737844103	-2.01202404809619\\
-0.0240480961923848	-2.00980947673159\\
-0.0374454978510004	-1.99599198396794\\
-0.0400801603206413	-1.99327331140766\\
-0.0527955690962666	-1.97995991983968\\
-0.0561122244488979	-1.97648460944062\\
-0.067928163570394	-1.96392785571142\\
-0.0721442885771544	-1.95944278687626\\
-0.082852209135697	-1.94789579158317\\
-0.0881763527054109	-1.94214706394111\\
-0.0975759386864961	-1.93186372745491\\
-0.104208416833667	-1.9245964640388\\
-0.112106938595428	-1.91583166332665\\
-0.120240480961924	-1.90678981249196\\
-0.126452192853364	-1.8997995991984\\
-0.13627254509018	-1.8887257350277\\
-0.140618123317979	-1.88376753507014\\
-0.152304609218437	-1.87040265600366\\
-0.154610626440267	-1.86773547094188\\
-0.168336673346694	-1.85181879637159\\
-0.168435106798128	-1.85170340681363\\
-0.182094263918885	-1.83567134268537\\
-0.18436873747495	-1.83299436395044\\
-0.195597114389196	-1.81963927855711\\
-0.200400801603207	-1.81390810034557\\
-0.208948104277145	-1.80360721442886\\
-0.216432865731463	-1.79455667242361\\
-0.222151133760787	-1.7875751503006\\
-0.232464929859719	-1.77493748328741\\
-0.235209729534102	-1.77154308617234\\
-0.248127367485314	-1.75551102204409\\
-0.248496993987976	-1.7550514411902\\
-0.260910749817074	-1.73947895791583\\
-0.264529058116232	-1.73492125251862\\
-0.273561692861503	-1.72344689378758\\
-0.280561122244489	-1.71451642527965\\
-0.286082555369444	-1.70741482965932\\
-0.296593186372745	-1.69383350377715\\
-0.298475405129103	-1.69138276553106\\
-0.31074681341069	-1.67535070140281\\
-0.312625250501002	-1.672888260013\\
-0.322901385421872	-1.65931863727455\\
-0.328657314629258	-1.65167876797935\\
-0.334936639563524	-1.64328657314629\\
-0.344689378757515	-1.63018168417632\\
-0.346853715889066	-1.62725450901804\\
-0.358661223665631	-1.61122244488978\\
-0.360721442885771	-1.60841427574096\\
-0.370363623606131	-1.59519038076152\\
-0.376753507014028	-1.58637493163314\\
-0.381954431452668	-1.57915831663327\\
-0.392785571142285	-1.56403642365584\\
-0.393433997955205	-1.56312625250501\\
-0.404822061190901	-1.54709418837675\\
-0.408817635270541	-1.54143613904278\\
-0.416106673164823	-1.5310621242485\\
-0.424849699398798	-1.51853530825889\\
-0.427284817757179	-1.51503006012024\\
-0.438370195933025	-1.49899799599198\\
-0.440881763527054	-1.49534823050935\\
-0.44936577793113	-1.48296593186373\\
-0.456913827655311	-1.47187009297541\\
-0.460258527737727	-1.46693386773547\\
-0.47105948890644	-1.45090180360721\\
-0.472945891783567	-1.44808932893996\\
-0.481780563993476	-1.43486973947896\\
-0.488977955911824	-1.42401669850271\\
-0.492401403989197	-1.4188376753507\\
-0.502935212529043	-1.40280561122244\\
-0.50501002004008	-1.39963259419958\\
-0.513393889713682	-1.38677354709419\\
-0.521042084168337	-1.37494598588992\\
-0.523753976742535	-1.37074148296593\\
-0.534036970397823	-1.35470941883768\\
-0.537074148296593	-1.34994602829754\\
-0.54424314245709	-1.33867735470942\\
-0.55310621242485	-1.32462354673062\\
-0.554351490046198	-1.32264529058116\\
-0.564399151523312	-1.30661322645291\\
-0.569138276553106	-1.29899224575859\\
-0.57436066541884	-1.29058116232465\\
-0.584232230108001	-1.27454909819639\\
-0.585170340681363	-1.27301910641964\\
-0.594051326219487	-1.25851703406814\\
-0.601202404809619	-1.24672820700097\\
-0.603774129129969	-1.24248496993988\\
-0.613432791996307	-1.22645290581162\\
-0.617234468937876	-1.22009677958938\\
-0.62301858987926	-1.21042084168337\\
-0.632513854057083	-1.19438877755511\\
-0.633266533066132	-1.19311243712257\\
-0.641966158922827	-1.17835671342685\\
-0.649298597194389	-1.16579139866338\\
-0.651321859979967	-1.1623246492986\\
-0.660624815805227	-1.14629258517034\\
-0.665330661322646	-1.13811357789636\\
-0.669850938713261	-1.13026052104208\\
-0.679002064889275	-1.11422845691383\\
-0.681362725450902	-1.11006688618478\\
-0.688101307851878	-1.09819639278557\\
-0.69710495342446	-1.08216432865731\\
-0.697394789579158	-1.08164605316432\\
-0.706079738533738	-1.06613226452906\\
-0.713426853707415	-1.05285896278679\\
-0.714956057633846	-1.0501002004008\\
-0.723792570976108	-1.03406813627255\\
-0.729458917835671	-1.0236844437265\\
-0.732546518759912	-1.01803607214429\\
-0.741245729576007	-1.00200400801603\\
-0.745490981963928	-0.9941124267187\\
-0.749879009420214	-0.985971943887776\\
-0.75844473699409	-0.969939879759519\\
-0.761523046092184	-0.964134766588893\\
-0.766958806572806	-0.953907815631263\\
-0.775394727267324	-0.937875751503006\\
-0.777555110220441	-0.93374247901286\\
-0.783790805290207	-0.92184368737475\\
-0.792100457992489	-0.905811623246493\\
-0.793587174348697	-0.90292570551388\\
-0.800379530336358	-0.889779559118236\\
-0.808566321619409	-0.87374749498998\\
-0.809619238476954	-0.871673675404657\\
-0.816729146876445	-0.857715430861723\\
-0.824796355889865	-0.841683366733467\\
-0.82565130260521	-0.839974664477627\\
-0.832843470353537	-0.82565130260521\\
-0.840794253455332	-0.809619238476954\\
-0.841683366733467	-0.807815950229083\\
-0.848725975563253	-0.793587174348697\\
-0.856563370704004	-0.777555110220441\\
-0.857715430861723	-0.775183763382833\\
-0.864379804955127	-0.761523046092184\\
-0.872106735825038	-0.745490981963928\\
-0.87374749498998	-0.74206323545754\\
-0.879807776186825	-0.729458917835671\\
-0.887427056135495	-0.713426853707415\\
-0.889779559118236	-0.708438342098458\\
-0.895012388955043	-0.697394789579158\\
-0.902526724693145	-0.681362725450902\\
-0.905811623246493	-0.67429184186864\\
-0.909995831124627	-0.665330661322646\\
-0.917407826216054	-0.649298597194389\\
-0.92184368737475	-0.639605210166721\\
-0.924759984175294	-0.633266533066132\\
-0.932072142327726	-0.617234468937876\\
-0.937875751503006	-0.604358567907728\\
-0.939306427983218	-0.601202404809619\\
-0.946521156144503	-0.585170340681363\\
-0.953640079044779	-0.569138276553106\\
-0.953907815631263	-0.568531625962185\\
-0.960756056219921	-0.55310621242485\\
-0.967780754132098	-0.537074148296593\\
-0.969939879759519	-0.532105663628523\\
-0.97477773985877	-0.521042084168337\\
-0.981710376076964	-0.50501002004008\\
-0.985971943887776	-0.495049758631863\\
-0.988586815811709	-0.488977955911824\\
-0.99542946636084	-0.472945891783567\\
-1.00200400801603	-0.45733731060375\\
-1.00218360635947	-0.456913827655311\\
-1.00893826228546	-0.440881763527054\\
-1.01560260849885	-0.424849699398798\\
-1.01803607214429	-0.418942565866308\\
-1.02223671829137	-0.408817635270541\\
-1.02881559873643	-0.392785571142285\\
-1.03406813627255	-0.379829674451506\\
-1.03532450668279	-0.376753507014028\\
-1.04181969582964	-0.360721442885771\\
-1.04822789550428	-0.344689378757515\\
-1.0501002004008	-0.339963767795496\\
-1.05461421190746	-0.328657314629258\\
-1.06094089149501	-0.312625250501002\\
-1.06613226452906	-0.299306270210088\\
-1.06719817822434	-0.296593186372745\\
-1.07344490733363	-0.280561122244489\\
-1.07960754402498	-0.264529058116232\\
-1.08216432865731	-0.257811391612405\\
-1.08573861773676	-0.248496993987976\\
-1.09182313616956	-0.232464929859719\\
-1.09782580665023	-0.216432865731463\\
-1.09819639278557	-0.215435169716468\\
-1.10382818894802	-0.200400801603207\\
-1.10975436746823	-0.18436873747495\\
-1.11422845691383	-0.172116267907533\\
-1.11562073759942	-0.168336673346694\\
-1.12147167987081	-0.152304609218437\\
-1.12724301738029	-0.13627254509018\\
-1.13026052104208	-0.127797365544665\\
-1.13297542122262	-0.120240480961924\\
-1.13867283245662	-0.104208416833667\\
-1.14429246627308	-0.0881763527054109\\
-1.14629258517034	-0.0824110642442843\\
-1.14988753346395	-0.0721442885771544\\
-1.15543435708166	-0.0561122244488979\\
-1.16090507180902	-0.0400801603206413\\
-1.1623246492986	-0.0358778763363548\\
-1.16635911296884	-0.0240480961923848\\
-1.17175794302227	-0.00801603206412826\\
-1.17708217805388	0.00801603206412782\\
-1.17835671342685	0.0118935369522394\\
-1.1823911770971	0.0240480961923843\\
-1.1876442649766	0.0400801603206409\\
-1.19282412066439	0.0561122244488974\\
-1.19438877755511	0.0610080323132719\\
-1.19798372584872	0.0721442885771539\\
-1.20309298566784	0.0881763527054105\\
-1.20813022818325	0.104208416833667\\
-1.21042084168337	0.111585955694638\\
-1.21313574186391	0.120240480961924\\
-1.21810275454298	0.13627254509018\\
-1.22299881944837	0.152304609218437\\
-1.22645290581162	0.163765482258007\\
-1.22784518649722	0.168336673346693\\
-1.23267120256016	0.18436873747495\\
-1.23742719710257	0.200400801603206\\
-1.24211438380454	0.216432865731463\\
-1.24248496993988	0.217714857437629\\
-1.24679493303642	0.232464929859719\\
-1.25141163716841	0.248496993987976\\
-1.2559602494358	0.264529058116232\\
-1.25851703406814	0.273662714996768\\
-1.26046950850622	0.280561122244489\\
-1.26494737463194	0.296593186372745\\
-1.26935772516235	0.312625250501002\\
-1.27370159972485	0.328657314629258\\
-1.27454909819639	0.331825883481581\\
-1.2780285849571	0.344689378757515\\
-1.28230065775349	0.360721442885771\\
-1.28650662502303	0.376753507014028\\
-1.29058116232465	0.392528378245189\\
-1.29064836143032	0.392785571142285\\
-1.29478180847173	0.408817635270541\\
-1.29884938139001	0.424849699398798\\
-1.30285195075063	0.440881763527054\\
-1.30661322645291	0.456191316685715\\
-1.30679282479634	0.456913827655311\\
-1.31072117881957	0.472945891783567\\
-1.31458455512255	0.488977955911824\\
-1.31838372277035	0.50501002004008\\
-1.32211941541806	0.521042084168337\\
-1.32264529058116	0.523334349434413\\
-1.32583504765005	0.537074148296593\\
-1.32949353116982	0.55310621242485\\
-1.33308829272707	0.569138276553106\\
-1.33661996515266	0.585170340681363\\
-1.33867735470942	0.594673495785783\\
-1.34010803118963	0.601202404809619\\
-1.34355990970964	0.617234468937876\\
-1.34694824139766	0.633266533066132\\
-1.35027355767898	0.649298597194389\\
-1.35353635514788	0.665330661322646\\
-1.35470941883768	0.671204009319294\\
-1.35676341650596	0.681362725450902\\
-1.35994224867448	0.697394789579159\\
-1.36305782068158	0.713426853707415\\
-1.36611052617763	0.729458917835672\\
-1.36910072380099	0.745490981963928\\
-1.37074148296593	0.754474507343357\\
-1.37204493428342	0.761523046092185\\
-1.37494628488202	0.777555110220441\\
-1.37778409179572	0.793587174348698\\
-1.380558607853	0.809619238476954\\
-1.38327005023437	0.825651302605211\\
-1.38591860037884	0.841683366733467\\
-1.38677354709419	0.846986325458802\\
-1.38852514086549	0.857715430861724\\
-1.39107767773152	0.87374749498998\\
-1.39356590308185	0.889779559118236\\
-1.39598988725695	0.905811623246493\\
-1.39834966361338	0.921843687374749\\
-1.40064522826933	0.937875751503006\\
-1.40280561122244	0.953398857694158\\
-1.40287734648678	0.953907815631262\\
-1.40506851430965	0.969939879759519\\
-1.40719363867873	0.985971943887775\\
-1.40925259695101	1.00200400801603\\
-1.41124522734124	1.01803607214429\\
-1.41317132849114	1.03406813627254\\
-1.41503065900035	1.0501002004008\\
-1.4168229369183	1.06613226452906\\
-1.418547839196	1.08216432865731\\
-1.4188376753507	1.0849741854515\\
-1.42021886398182	1.09819639278557\\
-1.42182337322298	1.11422845691383\\
-1.42335795274132	1.13026052104208\\
-1.42482214355234	1.14629258517034\\
-1.42621544244241	1.1623246492986\\
-1.4275373012074	1.17835671342685\\
-1.42878712584424	1.19438877755511\\
-1.4299642756944	1.21042084168337\\
-1.43106806253739	1.22645290581162\\
-1.43209774963311	1.24248496993988\\
-1.43305255071117	1.25851703406814\\
-1.4339316289056	1.27454909819639\\
-1.43473409563298	1.29058116232465\\
-1.43486973947896	1.29359165594453\\
-1.435463735431	1.30661322645291\\
-1.43611501710031	1.32264529058116\\
-1.43668581376201	1.33867735470942\\
-1.43717505973574	1.35470941883768\\
-1.43758163205316	1.37074148296593\\
-1.43790434902837	1.38677354709419\\
-1.43814196875914	1.40280561122244\\
-1.43829318755633	1.4188376753507\\
-1.43835663829887	1.43486973947896\\
-1.43833088871135	1.45090180360721\\
-1.43821443956137	1.46693386773547\\
-1.43800572277337	1.48296593186373\\
-1.43770309945568	1.49899799599198\\
-1.43730485783734	1.51503006012024\\
-1.43680921111101	1.5310621242485\\
-1.43621429517806	1.54709418837675\\
-1.43551816629188	1.56312625250501\\
-1.43486973947896	1.57614782301339\\
}--cycle;


\addplot[area legend,solid,fill=mycolor6,draw=black,forget plot]
table[row sep=crcr] {%
x	y\\
-1.25851703406814	1.5366182780483\\
-1.2571939945978	1.54709418837675\\
-1.25502463969909	1.56312625250501\\
-1.25269977496859	1.57915831663327\\
-1.25021500021237	1.59519038076152\\
-1.24756573216555	1.61122244488978\\
-1.24474719693108	1.62725450901804\\
-1.24248496993988	1.63940453353308\\
-1.24174975232708	1.64328657314629\\
-1.23855607385912	1.65931863727455\\
-1.23517430122925	1.67535070140281\\
-1.23159857003181	1.69138276553106\\
-1.22782276802783	1.70741482965932\\
-1.22645290581162	1.71297294500933\\
-1.22381713524383	1.72344689378758\\
-1.21958196768543	1.73947895791583\\
-1.21512182688343	1.75551102204409\\
-1.21042914983683	1.77154308617235\\
-1.21042084168337	1.77157044071457\\
-1.20544003002271	1.7875751503006\\
-1.2001956701131	1.80360721442886\\
-1.19468712586423	1.81963927855711\\
-1.19438877755511	1.82047757060002\\
-1.18883032863662	1.83567134268537\\
-1.18267763612223	1.85170340681363\\
-1.17835671342685	1.86248676464928\\
-1.17618880789234	1.86773547094188\\
-1.16930974881892	1.88376753507014\\
-1.1623246492986	1.89929379840314\\
-1.16208939750908	1.8997995991984\\
-1.15438504264813	1.91583166332665\\
-1.14630279283667	1.93186372745491\\
-1.14629258517034	1.93188336299896\\
-1.13765350560881	1.94789579158317\\
-1.13026052104208	1.96099723421974\\
-1.12853963234662	1.96392785571142\\
-1.11879866336283	1.97995991983968\\
-1.11422845691383	1.98719867302287\\
-1.10842999965929	1.99599198396794\\
-1.09819639278557	2.01090995449779\\
-1.09739566362948	2.01202404809619\\
-1.08549980486119	2.02805611222445\\
-1.08216432865731	2.03239562534614\\
-1.07270745353508	2.04408817635271\\
-1.06613226452906	2.05194028088218\\
-1.05889760819504	2.06012024048096\\
-1.0501002004008	2.06974519316727\\
-1.04388952525939	2.07615230460922\\
-1.03406813627255	2.08597369359606\\
-1.02745181211216	2.09218436873747\\
-1.01803607214429	2.10076645757832\\
-1.00928481999218	2.10821643286573\\
-1.00200400801603	2.11424476645324\\
-0.988995122323532	2.12424849699399\\
-0.985971943887776	2.12651320994356\\
-0.969939879759519	2.1376506054179\\
-0.965832546162879	2.14028056112224\\
-0.953907815631263	2.14773689968831\\
-0.93884409198786	2.1563126252505\\
-0.937875751503006	2.15685176728282\\
-0.92184368737475	2.16503328233086\\
-0.905838977788721	2.17234468937876\\
-0.905811623246493	2.17235693884157\\
-0.889779559118236	2.1788492796017\\
-0.87374749498998	2.18456798436975\\
-0.861530437300388	2.18837675350701\\
-0.857715430861723	2.18954616042773\\
-0.841683366733467	2.19381497273679\\
-0.82565130260521	2.19741021755645\\
-0.809619238476954	2.20035935280452\\
-0.793587174348697	2.20268792893981\\
-0.777657159274506	2.20440881763527\\
-0.777555110220441	2.20441970542425\\
-0.761523046092184	2.20557727650906\\
-0.745490981963928	2.20618078314376\\
-0.729458917835671	2.20624879964812\\
-0.713426853707415	2.20579850771998\\
-0.697394789579158	2.20484577011933\\
-0.692563647657071	2.20440881763527\\
-0.681362725450902	2.20340688550526\\
-0.665330661322646	2.20149581598421\\
-0.649298597194389	2.19912452309507\\
-0.633266533066132	2.19630450520987\\
-0.617234468937876	2.1930462183839\\
-0.601202404809619	2.18935912257919\\
-0.597387398370955	2.18837675350701\\
-0.585170340681363	2.18526062119864\\
-0.569138276553106	2.18075511445476\\
-0.55310621242485	2.17584697356038\\
-0.542541012275099	2.17234468937876\\
-0.537074148296593	2.17054823307935\\
-0.521042084168337	2.16487272763267\\
-0.50501002004008	2.15881437208123\\
-0.498795492688808	2.1563126252505\\
-0.488977955911824	2.15239180075642\\
-0.472945891783567	2.14560656143752\\
-0.461021161251951	2.14028056112224\\
-0.456913827655311	2.13845970574222\\
-0.440881763527054	2.1309704822989\\
-0.427159101112417	2.12424849699399\\
-0.424849699398798	2.12312507020458\\
-0.408817635270541	2.11495108169751\\
-0.396163769225858	2.10821643286573\\
-0.392785571142285	2.10643000206709\\
-0.376753507014028	2.09758696454167\\
-0.367337767046153	2.09218436873747\\
-0.360721442885771	2.0884104077147\\
-0.344689378757515	2.07891068588303\\
-0.340200968789764	2.07615230460922\\
-0.328657314629258	2.06909626596799\\
-0.314476241216867	2.06012024048096\\
-0.312625250501002	2.05895464809079\\
-0.296593186372745	2.04851186060859\\
-0.290017997366719	2.04408817635271\\
-0.280561122244489	2.03775510652911\\
-0.266525006674562	2.02805611222445\\
-0.264529058116232	2.02668291954234\\
-0.248496993987976	2.01531578552215\\
-0.24398720621718	2.01202404809619\\
-0.232464929859719	2.00364619026176\\
-0.222231322986003	1.99599198396794\\
-0.216432865731463	1.99167061710524\\
-0.201144387012885	1.97995991983968\\
-0.200400801603207	1.97939224687704\\
-0.18436873747495	1.96682922522964\\
-0.180759277272084	1.96392785571142\\
-0.168336673346694	1.95397046507693\\
-0.160943688779969	1.94789579158317\\
-0.152304609218437	1.94081544756269\\
-0.141641876092624	1.93186372745491\\
-0.13627254509018	1.92736641042022\\
-0.122819082958197	1.91583166332665\\
-0.120240480961924	1.91362535945474\\
-0.104443668623181	1.8997995991984\\
-0.104208416833667	1.89959407113954\\
-0.0881763527054109	1.88528235039355\\
-0.0865148160541831	1.88376753507014\\
-0.0721442885771544	1.8706828177036\\
-0.0689729344898847	1.86773547094188\\
-0.0561122244488979	1.85579545394615\\
-0.0517913017535205	1.85170340681363\\
-0.0400801603206413	1.8406211431019\\
-0.0349491946272899	1.83567134268537\\
-0.0240480961923848	1.82516054716343\\
-0.0184276157516171	1.81963927855711\\
-0.00801603206412826	1.80941410698685\\
-0.00220913950613561	1.80360721442886\\
0.00801603206412782	1.79338204285859\\
0.0137222256779407	1.7875751503006\\
0.0240480961923843	1.77706435477866\\
0.0293811605494023	1.77154308617235\\
0.0400801603206409	1.76046082246061\\
0.044781145520701	1.75551102204409\\
0.0561122244488974	1.74357100504835\\
0.0599345598845288	1.73947895791583\\
0.0721442885771539	1.72639424054929\\
0.0748527710502992	1.72344689378758\\
0.0881763527054105	1.70892964498273\\
0.0895462149216145	1.70741482965932\\
0.104023002497253	1.69138276553106\\
0.104208416833667	1.6911772374722\\
0.118282065685979	1.67535070140281\\
0.120240480961924	1.67314439753089\\
0.132343649009419	1.65931863727455\\
0.13627254509018	1.65482132023986\\
0.146215470783112	1.64328657314629\\
0.152304609218437	1.63620622912581\\
0.159904586324255	1.62725450901804\\
0.168336673346693	1.61729711838354\\
0.173417435572365	1.61122244488978\\
0.18436873747495	1.59809175027974\\
0.186759886243764	1.59519038076152\\
0.199935144005117	1.57915831663327\\
0.200400801603206	1.57859064367063\\
0.212940471362413	1.56312625250501\\
0.216432865731463	1.55880488564231\\
0.225792837809639	1.54709418837675\\
0.232464929859719	1.53871633054232\\
0.238496370489154	1.5310621242485\\
0.248496993987976	1.5183217975462\\
0.251054773962973	1.51503006012024\\
0.263468599425322	1.49899799599198\\
0.264529058116232	1.49762480330988\\
0.275739145546932	1.48296593186373\\
0.280561122244489	1.47663286204013\\
0.287877223636193	1.46693386773547\\
0.296593186372745	1.4553254878631\\
0.299885216061356	1.45090180360721\\
0.311764107097419	1.43486973947896\\
0.312625250501002	1.43370414708879\\
0.323515162650359	1.4188376753507\\
0.328657314629258	1.41178163670947\\
0.335145841858041	1.40280561122244\\
0.344689378757515	1.389531928368\\
0.346657440132767	1.38677354709419\\
0.358051131200599	1.37074148296593\\
0.360721442885771	1.36696752194316\\
0.369331448980606	1.35470941883768\\
0.376753507014028	1.34407995051361\\
0.380499925436401	1.33867735470942\\
0.391557950811915	1.32264529058116\\
0.392785571142285	1.32085885978252\\
0.402510992260561	1.30661322645291\\
0.408817635270541	1.29731581115643\\
0.41335791184124	1.29058116232465\\
0.424099699060212	1.27454909819639\\
0.424849699398798	1.27342567140698\\
0.434745077414153	1.25851703406814\\
0.440881763527054	1.2492069552448\\
0.445288689743263	1.24248496993988\\
0.455732556726431	1.22645290581162\\
0.456913827655311	1.2246320504316\\
0.46608675354468	1.21042084168337\\
0.472945891783567	1.19971477787038\\
0.476342309784794	1.19438877755511\\
0.486505544007245	1.17835671342685\\
0.488977955911824	1.17443588893277\\
0.496582201926817	1.1623246492986\\
0.50501002004008	1.14879433200106\\
0.506562227335664	1.14629258517034\\
0.516461159868609	1.13026052104208\\
0.521042084168337	1.12278855929599\\
0.526271334030238	1.11422845691383\\
0.535990169562516	1.09819639278557\\
0.537074148296593	1.09639993648616\\
0.545635936496428	1.08216432865731\\
0.55310621242485	1.06963454871068\\
0.555188309789687	1.06613226452906\\
0.564665976417858	1.0501002004008\\
0.569138276553106	1.04247856134855\\
0.574060918999784	1.03406813627254\\
0.583370807925782	1.01803607214429\\
0.585170340681363	1.01491993983592\\
0.592612661348774	1.00200400801603\\
0.601202404809619	0.986954312959955\\
0.601762078998257	0.985971943887775\\
0.610851960394462	0.969939879759519\\
0.617234468937876	0.95857728050815\\
0.619853429263515	0.953907815631262\\
0.628786712597152	0.937875751503006\\
0.633266533066132	0.929771439077608\\
0.637643590364386	0.921843687374749\\
0.646424307191761	0.905811623246493\\
0.649298597194389	0.900527328706294\\
0.655139621623025	0.889779559118236\\
0.663771644745461	0.87374749498998\\
0.665330661322646	0.870834493338914\\
0.672348105225822	0.857715430861724\\
0.680835154463941	0.841683366733467\\
0.681362725450902	0.840681434603457\\
0.689275162647607	0.825651302605211\\
0.697394789579159	0.810056190961008\\
0.697622326369228	0.809619238476954\\
0.705926469923734	0.793587174348698\\
0.713426853707415	0.778944800305149\\
0.714139046579946	0.777555110220441\\
0.722307271821182	0.761523046092185\\
0.729458917835672	0.747330963976777\\
0.730386855299095	0.745490981963928\\
0.738422394955954	0.729458917835672\\
0.745490981963928	0.715198819215905\\
0.746370303804596	0.713426853707415\\
0.754276259900431	0.697394789579159\\
0.761523046092185	0.682531184324693\\
0.762093544238379	0.681362725450902\\
0.76987289232088	0.665330661322646\\
0.777555110220441	0.649309484983373\\
0.77756034005763	0.649298597194389\\
0.785215933182017	0.633266533066132\\
0.792780544847256	0.617234468937876\\
0.793587174348698	0.615513580242413\\
0.800308648052462	0.601202404809619\\
0.807753133342204	0.585170340681363\\
0.809619238476954	0.581120875850607\\
0.815153935541872	0.569138276553106\\
0.822480850118346	0.55310621242485\\
0.825651302605211	0.546107612346024\\
0.829754334897784	0.537074148296593\\
0.836966124067748	0.521042084168337\\
0.841683366733467	0.510448239269849\\
0.844112032787387	0.50501002004008\\
0.851211034475311	0.488977955911824\\
0.857715430861724	0.474115298704278\\
0.858228869286926	0.472945891783567\\
0.865217316805395	0.456913827655311\\
0.872120783966104	0.440881763527054\\
0.87374749498998	0.437072994389789\\
0.878986367722158	0.424849699398798\\
0.885781952461454	0.408817635270541\\
0.889779559118236	0.399290161365227\\
0.892519249360777	0.392785571142285\\
0.899208927971631	0.376753507014028\\
0.905811623246493	0.36073369234858\\
0.905816692864383	0.360721442885771\\
0.912402345178452	0.344689378757515\\
0.918906420569259	0.328657314629258\\
0.921843687374749	0.321345907581362\\
0.925362512507619	0.312625250501002\\
0.931764677049471	0.296593186372745\\
0.937875751503006	0.281100264276807\\
0.93808941405656	0.280561122244489\\
0.944391308699561	0.264529058116232\\
0.950614039736601	0.248496993987976\\
0.953907815631262	0.239921268425783\\
0.956785885963635	0.232464929859719\\
0.962910080477966	0.216432865731463\\
0.96895700668126	0.200400801603206\\
0.969939879759519	0.197770845898868\\
0.974974727817517	0.18436873747495\\
0.980924619413404	0.168336673346693\\
0.985971943887775	0.154569322168009\\
0.986806821734876	0.152304609218437\\
0.992660954792984	0.13627254509018\\
0.998439532095766	0.120240480961924\\
1.00200400801603	0.110236750421178\\
1.00416444469553	0.104208416833667\\
1.00984840578382	0.0881763527054105\\
1.01545825797897	0.0721442885771539\\
1.01803607214429	0.0646943132897401\\
1.02102360004871	0.0561122244488974\\
1.02653975825509	0.0400801603206409\\
1.03198308268065	0.0240480961923843\\
1.03406813627254	0.0178374210509739\\
1.03738620316287	0.00801603206412782\\
1.04273653858697	-0.00801603206412826\\
1.04801514680891	-0.0240480961923848\\
1.0501002004008	-0.0304552076343362\\
1.05325301903754	-0.0400801603206413\\
1.05843912796602	-0.0561122244488979\\
1.06355445036374	-0.0721442885771544\\
1.06613226452906	-0.0803242481759373\\
1.06862366525489	-0.0881763527054109\\
1.07364676385577	-0.104208416833667\\
1.07859985273705	-0.120240480961924\\
1.08216432865731	-0.131933031968492\\
1.08349661050439	-0.13627254509018\\
1.08835753705469	-0.152304609218437\\
1.0931490683112	-0.168336673346694\\
1.09787225175611	-0.18436873747495\\
1.09819639278557	-0.18548283107335\\
1.10256838432536	-0.200400801603207\\
1.10719865763227	-0.216432865731463\\
1.1117609582029	-0.232464929859719\\
1.11422845691383	-0.241258240804784\\
1.11627507655532	-0.248496993987976\\
1.12074401411038	-0.264529058116232\\
1.12514519320558	-0.280561122244489\\
1.12947946693003	-0.296593186372745\\
1.13026052104208	-0.29952380786443\\
1.13377934617495	-0.312625250501002\\
1.13801878451301	-0.328657314629258\\
1.14219129312577	-0.344689378757515\\
1.14629258517034	-0.360701807341723\\
1.14629765478823	-0.360721442885771\\
1.15037434760639	-0.376753507014028\\
1.15438392376633	-0.392785571142285\\
1.15832704264181	-0.408817635270541\\
1.162204325077	-0.424849699398798\\
1.1623246492986	-0.425355500194049\\
1.16604833959957	-0.440881763527054\\
1.16982653524227	-0.456913827655311\\
1.17353838602857	-0.472945891783567\\
1.17718439541235	-0.488977955911824\\
1.17835671342685	-0.494226662204424\\
1.18078537948077	-0.50501002004008\\
1.18432949877743	-0.521042084168337\\
1.1878070192803	-0.537074148296593\\
1.19121832506825	-0.55310621242485\\
1.19438877755511	-0.568299984510198\\
1.19456520206247	-0.569138276553106\\
1.19787134074402	-0.585170340681363\\
1.20111025125887	-0.601202404809619\\
1.20428219543686	-0.617234468937876\\
1.20738739493193	-0.633266533066132\\
1.21042084168337	-0.649271242652159\\
1.21042607152056	-0.649298597194389\\
1.21342083185755	-0.665330661322646\\
1.21634748309878	-0.681362725450902\\
1.21920611961987	-0.697394789579158\\
1.22199679407919	-0.713426853707415\\
1.2247195170619	-0.729458917835671\\
1.22645290581162	-0.739932866613912\\
1.22738084327505	-0.745490981963928\\
1.22998524429604	-0.761523046092184\\
1.23251986702919	-0.777555110220441\\
1.2349845861562	-0.793587174348697\\
1.23737923194097	-0.809619238476954\\
1.23970358965474	-0.82565130260521\\
1.24195739895292	-0.841683366733467\\
1.24248496993988	-0.845565406346682\\
1.24415072011282	-0.857715430861723\\
1.24627484657362	-0.87374749498998\\
1.24832599436852	-0.889779559118236\\
1.25030375289917	-0.905811623246493\\
1.25220766250782	-0.92184368737475\\
1.25403721359916	-0.937875751503006\\
1.25579184570513	-0.953907815631263\\
1.25747094649076	-0.969939879759519\\
1.25851703406814	-0.980415790087977\\
1.25907670825677	-0.985971943887776\\
1.26061005025889	-1.00200400801603\\
1.26206463497995	-1.01803607214429\\
1.26343967651481	-1.03406813627255\\
1.2647343321079	-1.0501002004008\\
1.26594770084412	-1.06613226452906\\
1.26707882226797	-1.08216432865731\\
1.26812667492813	-1.09819639278557\\
1.2690901748448	-1.11422845691383\\
1.26996817389666	-1.13026052104208\\
1.2707594581245	-1.14629258517034\\
1.2714627459479	-1.1623246492986\\
1.27207668629181	-1.17835671342685\\
1.27259985661893	-1.19438877755511\\
1.27303076086424	-1.21042084168337\\
1.27336782726751	-1.22645290581162\\
1.27360940609925	-1.24248496993988\\
1.27375376727568	-1.25851703406814\\
1.27379909785781	-1.27454909819639\\
1.27374349942935	-1.29058116232465\\
1.27358498534827	-1.30661322645291\\
1.27332147786602	-1.32264529058116\\
1.27295080510859	-1.33867735470942\\
1.27247069791279	-1.35470941883768\\
1.27187878651122	-1.37074148296593\\
1.27117259705861	-1.38677354709419\\
1.27034954799212	-1.40280561122244\\
1.26940694621749	-1.4188376753507\\
1.2683419831128	-1.43486973947896\\
1.2671517303407	-1.45090180360721\\
1.26583313545984	-1.46693386773547\\
1.26438301732549	-1.48296593186373\\
1.26279806126875	-1.49899799599198\\
1.26107481404313	-1.51503006012024\\
1.25920967852684	-1.5310621242485\\
1.25851703406814	-1.5366182780483\\
1.2571939945978	-1.54709418837675\\
1.25502463969909	-1.56312625250501\\
1.25269977496859	-1.57915831663327\\
1.25021500021237	-1.59519038076152\\
1.24756573216555	-1.61122244488978\\
1.24474719693108	-1.62725450901804\\
1.24248496993988	-1.63940453353308\\
1.24174975232708	-1.64328657314629\\
1.23855607385912	-1.65931863727455\\
1.23517430122925	-1.67535070140281\\
1.23159857003181	-1.69138276553106\\
1.22782276802783	-1.70741482965932\\
1.22645290581162	-1.71297294500933\\
1.22381713524383	-1.72344689378758\\
1.21958196768543	-1.73947895791583\\
1.21512182688343	-1.75551102204409\\
1.21042914983683	-1.77154308617234\\
1.21042084168337	-1.77157044071457\\
1.20544003002271	-1.7875751503006\\
1.2001956701131	-1.80360721442886\\
1.19468712586423	-1.81963927855711\\
1.19438877755511	-1.82047757060002\\
1.18883032863661	-1.83567134268537\\
1.18267763612223	-1.85170340681363\\
1.17835671342685	-1.86248676464928\\
1.17618880789234	-1.86773547094188\\
1.16930974881892	-1.88376753507014\\
1.1623246492986	-1.89929379840315\\
1.16208939750908	-1.8997995991984\\
1.15438504264813	-1.91583166332665\\
1.14630279283667	-1.93186372745491\\
1.14629258517034	-1.93188336299896\\
1.13765350560881	-1.94789579158317\\
1.13026052104208	-1.96099723421974\\
1.12853963234662	-1.96392785571142\\
1.11879866336283	-1.97995991983968\\
1.11422845691383	-1.98719867302287\\
1.10842999965929	-1.99599198396794\\
1.09819639278557	-2.01090995449779\\
1.09739566362948	-2.01202404809619\\
1.08549980486119	-2.02805611222445\\
1.08216432865731	-2.03239562534614\\
1.07270745353508	-2.04408817635271\\
1.06613226452906	-2.05194028088218\\
1.05889760819504	-2.06012024048096\\
1.0501002004008	-2.06974519316727\\
1.04388952525939	-2.07615230460922\\
1.03406813627254	-2.08597369359606\\
1.02745181211216	-2.09218436873747\\
1.01803607214429	-2.10076645757832\\
1.00928481999218	-2.10821643286573\\
1.00200400801603	-2.11424476645324\\
0.988995122323532	-2.12424849699399\\
0.985971943887775	-2.12651320994356\\
0.969939879759519	-2.1376506054179\\
0.965832546162879	-2.14028056112224\\
0.953907815631262	-2.14773689968831\\
0.938844091987861	-2.1563126252505\\
0.937875751503006	-2.15685176728282\\
0.921843687374749	-2.16503328233086\\
0.905838977788721	-2.17234468937876\\
0.905811623246493	-2.17235693884156\\
0.889779559118236	-2.1788492796017\\
0.87374749498998	-2.18456798436975\\
0.861530437300388	-2.18837675350701\\
0.857715430861724	-2.18954616042773\\
0.841683366733467	-2.19381497273678\\
0.825651302605211	-2.19741021755645\\
0.809619238476954	-2.20035935280452\\
0.793587174348698	-2.20268792893981\\
0.777657159274506	-2.20440881763527\\
0.777555110220441	-2.20441970542426\\
0.761523046092185	-2.20557727650906\\
0.745490981963928	-2.20618078314376\\
0.729458917835672	-2.20624879964812\\
0.713426853707415	-2.20579850771998\\
0.697394789579159	-2.20484577011933\\
0.692563647657066	-2.20440881763527\\
0.681362725450902	-2.20340688550526\\
0.665330661322646	-2.20149581598421\\
0.649298597194389	-2.19912452309507\\
0.633266533066132	-2.19630450520987\\
0.617234468937876	-2.1930462183839\\
0.601202404809619	-2.18935912257919\\
0.597387398370955	-2.18837675350701\\
0.585170340681363	-2.18526062119864\\
0.569138276553106	-2.18075511445476\\
0.55310621242485	-2.17584697356038\\
0.542541012275099	-2.17234468937876\\
0.537074148296593	-2.17054823307935\\
0.521042084168337	-2.16487272763267\\
0.50501002004008	-2.15881437208122\\
0.498795492688808	-2.1563126252505\\
0.488977955911824	-2.15239180075642\\
0.472945891783567	-2.14560656143752\\
0.46102116125195	-2.14028056112224\\
0.456913827655311	-2.13845970574222\\
0.440881763527054	-2.1309704822989\\
0.427159101112417	-2.12424849699399\\
0.424849699398798	-2.12312507020458\\
0.408817635270541	-2.11495108169751\\
0.396163769225858	-2.10821643286573\\
0.392785571142285	-2.10643000206709\\
0.376753507014028	-2.09758696454167\\
0.367337767046153	-2.09218436873747\\
0.360721442885771	-2.0884104077147\\
0.344689378757515	-2.07891068588303\\
0.340200968789764	-2.07615230460922\\
0.328657314629258	-2.06909626596799\\
0.314476241216868	-2.06012024048096\\
0.312625250501002	-2.05895464809079\\
0.296593186372745	-2.04851186060859\\
0.29001799736672	-2.04408817635271\\
0.280561122244489	-2.03775510652911\\
0.266525006674562	-2.02805611222445\\
0.264529058116232	-2.02668291954234\\
0.248496993987976	-2.01531578552215\\
0.243987206217179	-2.01202404809619\\
0.232464929859719	-2.00364619026176\\
0.222231322986004	-1.99599198396794\\
0.216432865731463	-1.99167061710524\\
0.201144387012887	-1.97995991983968\\
0.200400801603206	-1.97939224687704\\
0.18436873747495	-1.96682922522964\\
0.180759277272084	-1.96392785571142\\
0.168336673346693	-1.95397046507693\\
0.160943688779969	-1.94789579158317\\
0.152304609218437	-1.94081544756269\\
0.141641876092625	-1.93186372745491\\
0.13627254509018	-1.92736641042022\\
0.122819082958197	-1.91583166332665\\
0.120240480961924	-1.91362535945474\\
0.104443668623181	-1.8997995991984\\
0.104208416833667	-1.89959407113954\\
0.0881763527054105	-1.88528235039355\\
0.0865148160541835	-1.88376753507014\\
0.0721442885771539	-1.8706828177036\\
0.0689729344898856	-1.86773547094188\\
0.0561122244488974	-1.85579545394615\\
0.051791301753521	-1.85170340681363\\
0.0400801603206409	-1.8406211431019\\
0.0349491946272904	-1.83567134268537\\
0.0240480961923843	-1.82516054716343\\
0.0184276157516176	-1.81963927855711\\
0.00801603206412782	-1.80941410698685\\
0.00220913950613607	-1.80360721442886\\
-0.00801603206412826	-1.79338204285859\\
-0.0137222256779403	-1.7875751503006\\
-0.0240480961923848	-1.77706435477866\\
-0.0293811605494027	-1.77154308617234\\
-0.0400801603206413	-1.76046082246061\\
-0.0447811455207014	-1.75551102204409\\
-0.0561122244488979	-1.74357100504835\\
-0.0599345598845293	-1.73947895791583\\
-0.0721442885771544	-1.72639424054929\\
-0.0748527710502996	-1.72344689378758\\
-0.0881763527054109	-1.70892964498273\\
-0.089546214921614	-1.70741482965932\\
-0.104023002497253	-1.69138276553106\\
-0.104208416833667	-1.6911772374722\\
-0.11828206568598	-1.67535070140281\\
-0.120240480961924	-1.67314439753089\\
-0.13234364900942	-1.65931863727455\\
-0.13627254509018	-1.65482132023986\\
-0.146215470783112	-1.64328657314629\\
-0.152304609218437	-1.63620622912581\\
-0.159904586324256	-1.62725450901804\\
-0.168336673346694	-1.61729711838354\\
-0.173417435572365	-1.61122244488978\\
-0.18436873747495	-1.59809175027974\\
-0.186759886243765	-1.59519038076152\\
-0.199935144005117	-1.57915831663327\\
-0.200400801603207	-1.57859064367062\\
-0.212940471362413	-1.56312625250501\\
-0.216432865731463	-1.55880488564231\\
-0.225792837809638	-1.54709418837675\\
-0.232464929859719	-1.53871633054232\\
-0.238496370489154	-1.5310621242485\\
-0.248496993987976	-1.5183217975462\\
-0.251054773962973	-1.51503006012024\\
-0.263468599425322	-1.49899799599198\\
-0.264529058116232	-1.49762480330988\\
-0.275739145546932	-1.48296593186373\\
-0.280561122244489	-1.47663286204013\\
-0.287877223636193	-1.46693386773547\\
-0.296593186372745	-1.4553254878631\\
-0.299885216061356	-1.45090180360721\\
-0.311764107097419	-1.43486973947896\\
-0.312625250501002	-1.43370414708879\\
-0.323515162650359	-1.4188376753507\\
-0.328657314629258	-1.41178163670947\\
-0.335145841858041	-1.40280561122244\\
-0.344689378757515	-1.389531928368\\
-0.346657440132767	-1.38677354709419\\
-0.358051131200599	-1.37074148296593\\
-0.360721442885771	-1.36696752194316\\
-0.369331448980606	-1.35470941883768\\
-0.376753507014028	-1.34407995051361\\
-0.380499925436401	-1.33867735470942\\
-0.391557950811915	-1.32264529058116\\
-0.392785571142285	-1.32085885978252\\
-0.402510992260561	-1.30661322645291\\
-0.408817635270541	-1.29731581115643\\
-0.41335791184124	-1.29058116232465\\
-0.424099699060212	-1.27454909819639\\
-0.424849699398798	-1.27342567140698\\
-0.434745077414153	-1.25851703406814\\
-0.440881763527054	-1.2492069552448\\
-0.445288689743263	-1.24248496993988\\
-0.455732556726431	-1.22645290581162\\
-0.456913827655311	-1.2246320504316\\
-0.466086753544681	-1.21042084168337\\
-0.472945891783567	-1.19971477787038\\
-0.476342309784793	-1.19438877755511\\
-0.486505544007245	-1.17835671342685\\
-0.488977955911824	-1.17443588893277\\
-0.496582201926817	-1.1623246492986\\
-0.50501002004008	-1.14879433200106\\
-0.506562227335664	-1.14629258517034\\
-0.516461159868609	-1.13026052104208\\
-0.521042084168337	-1.12278855929599\\
-0.526271334030238	-1.11422845691383\\
-0.535990169562516	-1.09819639278557\\
-0.537074148296593	-1.09639993648616\\
-0.545635936496428	-1.08216432865731\\
-0.55310621242485	-1.06963454871068\\
-0.555188309789687	-1.06613226452906\\
-0.564665976417857	-1.0501002004008\\
-0.569138276553106	-1.04247856134855\\
-0.574060918999784	-1.03406813627255\\
-0.583370807925782	-1.01803607214429\\
-0.585170340681363	-1.01491993983592\\
-0.592612661348773	-1.00200400801603\\
-0.601202404809619	-0.986954312959955\\
-0.601762078998257	-0.985971943887776\\
-0.610851960394462	-0.969939879759519\\
-0.617234468937876	-0.95857728050815\\
-0.619853429263515	-0.953907815631263\\
-0.628786712597152	-0.937875751503006\\
-0.633266533066132	-0.929771439077608\\
-0.637643590364386	-0.92184368737475\\
-0.646424307191761	-0.905811623246493\\
-0.649298597194389	-0.900527328706294\\
-0.655139621623025	-0.889779559118236\\
-0.663771644745462	-0.87374749498998\\
-0.665330661322646	-0.870834493338914\\
-0.672348105225823	-0.857715430861723\\
-0.680835154463942	-0.841683366733467\\
-0.681362725450902	-0.840681434603458\\
-0.689275162647607	-0.82565130260521\\
-0.697394789579158	-0.81005619096101\\
-0.697622326369228	-0.809619238476954\\
-0.705926469923734	-0.793587174348697\\
-0.713426853707415	-0.77894480030515\\
-0.714139046579946	-0.777555110220441\\
-0.722307271821182	-0.761523046092184\\
-0.729458917835671	-0.747330963976778\\
-0.730386855299095	-0.745490981963928\\
-0.738422394955954	-0.729458917835671\\
-0.745490981963928	-0.715198819215906\\
-0.746370303804596	-0.713426853707415\\
-0.754276259900432	-0.697394789579158\\
-0.761523046092184	-0.682531184324695\\
-0.76209354423838	-0.681362725450902\\
-0.76987289232088	-0.665330661322646\\
-0.777555110220441	-0.649309484983374\\
-0.77756034005763	-0.649298597194389\\
-0.785215933182017	-0.633266533066132\\
-0.792780544847256	-0.617234468937876\\
-0.793587174348697	-0.615513580242414\\
-0.800308648052461	-0.601202404809619\\
-0.807753133342203	-0.585170340681363\\
-0.809619238476954	-0.581120875850608\\
-0.815153935541872	-0.569138276553106\\
-0.822480850118345	-0.55310621242485\\
-0.82565130260521	-0.546107612346026\\
-0.829754334897784	-0.537074148296593\\
-0.836966124067748	-0.521042084168337\\
-0.841683366733467	-0.510448239269851\\
-0.844112032787387	-0.50501002004008\\
-0.851211034475311	-0.488977955911824\\
-0.857715430861723	-0.474115298704279\\
-0.858228869286925	-0.472945891783567\\
-0.865217316805395	-0.456913827655311\\
-0.872120783966104	-0.440881763527054\\
-0.87374749498998	-0.437072994389791\\
-0.878986367722158	-0.424849699398798\\
-0.885781952461454	-0.408817635270541\\
-0.889779559118236	-0.399290161365226\\
-0.892519249360777	-0.392785571142285\\
-0.899208927971631	-0.376753507014028\\
-0.905811623246493	-0.360733692348578\\
-0.905816692864383	-0.360721442885771\\
-0.912402345178453	-0.344689378757515\\
-0.918906420569258	-0.328657314629258\\
-0.92184368737475	-0.321345907581361\\
-0.925362512507619	-0.312625250501002\\
-0.93176467704947	-0.296593186372745\\
-0.937875751503006	-0.281100264276806\\
-0.938089414056559	-0.280561122244489\\
-0.944391308699561	-0.264529058116232\\
-0.950614039736601	-0.248496993987976\\
-0.953907815631263	-0.239921268425782\\
-0.956785885963635	-0.232464929859719\\
-0.962910080477966	-0.216432865731463\\
-0.968957006681261	-0.200400801603207\\
-0.969939879759519	-0.197770845898866\\
-0.974974727817517	-0.18436873747495\\
-0.980924619413404	-0.168336673346694\\
-0.985971943887776	-0.154569322168008\\
-0.986806821734876	-0.152304609218437\\
-0.992660954792983	-0.13627254509018\\
-0.998439532095766	-0.120240480961924\\
-1.00200400801603	-0.110236750421177\\
-1.00416444469553	-0.104208416833667\\
-1.00984840578382	-0.0881763527054109\\
-1.01545825797897	-0.0721442885771544\\
-1.01803607214429	-0.0646943132897392\\
-1.02102360004871	-0.0561122244488979\\
-1.02653975825509	-0.0400801603206413\\
-1.03198308268065	-0.0240480961923848\\
-1.03406813627255	-0.0178374210509725\\
-1.03738620316287	-0.00801603206412826\\
-1.04273653858697	0.00801603206412782\\
-1.04801514680891	0.0240480961923843\\
-1.0501002004008	0.0304552076343385\\
-1.05325301903754	0.0400801603206409\\
-1.05843912796601	0.0561122244488974\\
-1.06355445036374	0.0721442885771539\\
-1.06613226452906	0.0803242481759391\\
-1.06862366525489	0.0881763527054105\\
-1.07364676385577	0.104208416833667\\
-1.07859985273705	0.120240480961924\\
-1.08216432865731	0.131933031968493\\
-1.08349661050439	0.13627254509018\\
-1.08835753705469	0.152304609218437\\
-1.0931490683112	0.168336673346693\\
-1.09787225175611	0.18436873747495\\
-1.09819639278557	0.185482831073351\\
-1.10256838432536	0.200400801603206\\
-1.10719865763227	0.216432865731463\\
-1.1117609582029	0.232464929859719\\
-1.11422845691383	0.241258240804785\\
-1.11627507655532	0.248496993987976\\
-1.12074401411038	0.264529058116232\\
-1.12514519320558	0.280561122244489\\
-1.12947946693003	0.296593186372745\\
-1.13026052104208	0.299523807864431\\
-1.13377934617495	0.312625250501002\\
-1.13801878451301	0.328657314629258\\
-1.14219129312577	0.344689378757515\\
-1.14629258517034	0.360701807341723\\
-1.14629765478823	0.360721442885771\\
-1.15037434760639	0.376753507014028\\
-1.15438392376633	0.392785571142285\\
-1.15832704264181	0.408817635270541\\
-1.162204325077	0.424849699398798\\
-1.1623246492986	0.425355500194049\\
-1.16604833959957	0.440881763527054\\
-1.16982653524227	0.456913827655311\\
-1.17353838602857	0.472945891783567\\
-1.17718439541235	0.488977955911824\\
-1.17835671342685	0.494226662204425\\
-1.18078537948077	0.50501002004008\\
-1.18432949877743	0.521042084168337\\
-1.1878070192803	0.537074148296593\\
-1.19121832506825	0.55310621242485\\
-1.19438877755511	0.568299984510198\\
-1.19456520206247	0.569138276553106\\
-1.19787134074402	0.585170340681363\\
-1.20111025125887	0.601202404809619\\
-1.20428219543685	0.617234468937876\\
-1.20738739493193	0.633266533066132\\
-1.21042084168337	0.649271242652159\\
-1.21042607152056	0.649298597194389\\
-1.21342083185755	0.665330661322646\\
-1.21634748309878	0.681362725450902\\
-1.21920611961987	0.697394789579159\\
-1.22199679407919	0.713426853707415\\
-1.2247195170619	0.729458917835672\\
-1.22645290581162	0.739932866613911\\
-1.22738084327505	0.745490981963928\\
-1.22998524429604	0.761523046092185\\
-1.23251986702919	0.777555110220441\\
-1.2349845861562	0.793587174348698\\
-1.23737923194097	0.809619238476954\\
-1.23970358965474	0.825651302605211\\
-1.24195739895292	0.841683366733467\\
-1.24248496993988	0.845565406346681\\
-1.24415072011282	0.857715430861724\\
-1.24627484657362	0.87374749498998\\
-1.24832599436852	0.889779559118236\\
-1.25030375289917	0.905811623246493\\
-1.25220766250782	0.921843687374749\\
-1.25403721359916	0.937875751503006\\
-1.25579184570513	0.953907815631262\\
-1.25747094649076	0.969939879759519\\
-1.25851703406814	0.980415790087977\\
-1.25907670825677	0.985971943887775\\
-1.26061005025888	1.00200400801603\\
-1.26206463497995	1.01803607214429\\
-1.26343967651481	1.03406813627254\\
-1.2647343321079	1.0501002004008\\
-1.26594770084412	1.06613226452906\\
-1.26707882226797	1.08216432865731\\
-1.26812667492813	1.09819639278557\\
-1.2690901748448	1.11422845691383\\
-1.26996817389666	1.13026052104208\\
-1.2707594581245	1.14629258517034\\
-1.2714627459479	1.1623246492986\\
-1.27207668629181	1.17835671342685\\
-1.27259985661893	1.19438877755511\\
-1.27303076086424	1.21042084168337\\
-1.27336782726751	1.22645290581162\\
-1.27360940609925	1.24248496993988\\
-1.27375376727568	1.25851703406814\\
-1.27379909785781	1.27454909819639\\
-1.27374349942935	1.29058116232465\\
-1.27358498534827	1.30661322645291\\
-1.27332147786602	1.32264529058116\\
-1.27295080510859	1.33867735470942\\
-1.27247069791279	1.35470941883768\\
-1.27187878651122	1.37074148296593\\
-1.27117259705861	1.38677354709419\\
-1.27034954799212	1.40280561122244\\
-1.26940694621749	1.4188376753507\\
-1.2683419831128	1.43486973947896\\
-1.2671517303407	1.45090180360721\\
-1.26583313545984	1.46693386773547\\
-1.26438301732549	1.48296593186373\\
-1.26279806126875	1.49899799599198\\
-1.26107481404313	1.51503006012024\\
-1.25920967852684	1.5310621242485\\
-1.25851703406814	1.5366182780483\\
}--cycle;


\addplot[area legend,solid,fill=mycolor7,draw=black,forget plot]
table[row sep=crcr] {%
x	y\\
-1.11422845691383	1.29527490685972\\
-1.11319708140095	1.30661322645291\\
-1.11158897321772	1.32264529058116\\
-1.1098226773115	1.33867735470942\\
-1.10789361042715	1.35470941883768\\
-1.10579698497173	1.37074148296593\\
-1.10352780003294	1.38677354709419\\
-1.1010808319104	1.40280561122244\\
-1.0984506241303	1.4188376753507\\
-1.09819639278557	1.42029804511158\\
-1.09560322010549	1.43486973947896\\
-1.09255398071092	1.45090180360721\\
-1.08929907735243	1.46693386773547\\
-1.08583167329455	1.48296593186373\\
-1.08216432865731	1.49891333546637\\
-1.0821443520793	1.49899799599198\\
-1.07817537844418	1.51503006012024\\
-1.07396500034531	1.5310621242485\\
-1.06950457834855	1.54709418837675\\
-1.06613226452906	1.55859377815824\\
-1.0647629193205	1.56312625250501\\
-1.05969013932373	1.57915831663327\\
-1.05432972957213	1.59519038076152\\
-1.0501002004008	1.60721650529931\\
-1.04864334990002	1.61122244488978\\
-1.04255657480825	1.62725450901804\\
-1.0361352655747	1.64328657314629\\
-1.03406813627255	1.64823587357896\\
-1.0292623072434	1.65931863727455\\
-1.02196731801052	1.67535070140281\\
-1.01803607214429	1.68361504268286\\
-1.01418588617855	1.69138276553106\\
-1.00587990953774	1.70741482965932\\
-1.00200400801603	1.71458298935537\\
-0.996990227098053	1.72344689378758\\
-0.987502407962091	1.73947895791583\\
-0.985971943887776	1.74197122833202\\
-0.977236362534941	1.75551102204409\\
-0.969939879759519	1.7663285202369\\
-0.966230684303345	1.77154308617235\\
-0.954346209377059	1.7875751503006\\
-0.953907815631263	1.78814656274547\\
-0.94133327615024	1.80360721442886\\
-0.937875751503006	1.8076965221177\\
-0.927121552749067	1.81963927855711\\
-0.92184368737475	1.82528934752995\\
-0.911461692219334	1.83567134268537\\
-0.905811623246493	1.84112894737434\\
-0.89402410506561	1.85170340681363\\
-0.889779559118236	1.85538858208849\\
-0.874365875445441	1.86773547094188\\
-0.87374749498998	1.86821576553839\\
-0.857715430861723	1.87970629675285\\
-0.851492446151481	1.88376753507014\\
-0.841683366733467	1.8899920287837\\
-0.82565130260521	1.89916697663983\\
-0.824429387048633	1.8997995991984\\
-0.809619238476954	1.90727502109806\\
-0.793587174348697	1.91442224207739\\
-0.790015546419613	1.91583166332665\\
-0.777555110220441	1.92063769330967\\
-0.761523046092184	1.92599125684182\\
-0.745490981963928	1.93053135242795\\
-0.739874814442953	1.93186372745491\\
-0.729458917835671	1.93428701368306\\
-0.713426853707415	1.9373019131192\\
-0.697394789579158	1.9396132115098\\
-0.681362725450902	1.94125040392693\\
-0.665330661322646	1.94224079591663\\
-0.649298597194389	1.94260964305413\\
-0.633266533066132	1.94238027890235\\
-0.617234468937876	1.94157423236356\\
-0.601202404809619	1.94021133531567\\
-0.585170340681363	1.93830982133684\\
-0.569138276553106	1.93588641624351\\
-0.55310621242485	1.93295642109648\\
-0.54801938157678	1.93186372745491\\
-0.537074148296593	1.92953305696762\\
-0.521042084168337	1.92563085023898\\
-0.50501002004008	1.92126222414596\\
-0.488977955911824	1.91643779725655\\
-0.487146074453037	1.91583166332665\\
-0.472945891783567	1.91116918961357\\
-0.456913827655311	1.9054657821794\\
-0.442103679083631	1.8997995991984\\
-0.440881763527054	1.89933546148151\\
-0.424849699398798	1.89279061863763\\
-0.408817635270541	1.8858346643723\\
-0.404332999824192	1.88376753507014\\
-0.392785571142285	1.8784790028984\\
-0.376753507014028	1.87072758265841\\
-0.370881128353451	1.86773547094188\\
-0.360721442885771	1.86258938978954\\
-0.344689378757515	1.85406785039725\\
-0.340444832810142	1.85170340681363\\
-0.328657314629258	1.84517246283947\\
-0.312625250501002	1.83590190159421\\
-0.312243254575273	1.83567134268537\\
-0.296593186372745	1.82627154452174\\
-0.285972676222855	1.81963927855711\\
-0.280561122244489	1.81627548532647\\
-0.264529058116232	1.80592295097609\\
-0.261071533468998	1.80360721442886\\
-0.248496993987976	1.795219764379\\
-0.237425008585017	1.7875751503006\\
-0.232464929859719	1.78416365914826\\
-0.216432865731463	1.77276117036572\\
-0.21477460953327	1.77154308617235\\
-0.200400801603207	1.76101992105668\\
-0.193104318827785	1.75551102204409\\
-0.18436873747495	1.7489359039608\\
-0.172173345241227	1.73947895791583\\
-0.168336673346694	1.73651220130204\\
-0.152304609218437	1.72375235381353\\
-0.151931278795002	1.72344689378758\\
-0.13627254509018	1.71066447591853\\
-0.132396643568474	1.70741482965932\\
-0.120240480961924	1.69724375478096\\
-0.113414571508778	1.69138276553106\\
-0.104208416833667	1.68349221526087\\
-0.0949445761708533	1.67535070140281\\
-0.0881763527054109	1.66941162343815\\
-0.0769501176063007	1.65931863727455\\
-0.0721442885771544	1.6550034888922\\
-0.0593981639867885	1.64328657314629\\
-0.0561122244488979	1.64026906664833\\
-0.0422588104508495	1.62725450901804\\
-0.0400801603206413	1.62520935879441\\
-0.025504946693164	1.61122244488978\\
-0.0240480961923848	1.60982511577077\\
-0.00911196622221128	1.59519038076152\\
-0.00801603206412826	1.59411683733554\\
0.00694248863814595	1.57915831663327\\
0.00801603206412782	1.57808477320728\\
0.0226787509838308	1.56312625250501\\
0.0240480961923843	1.561728923386\\
0.0381153270050993	1.54709418837675\\
0.0400801603206409	1.54504903815313\\
0.0532690706591171	1.5310621242485\\
0.0561122244488974	1.52804461775053\\
0.0681553383640204	1.51503006012024\\
0.0721442885771539	1.5107149117379\\
0.082788126291551	1.49899799599198\\
0.0881763527054105	1.49305891802733\\
0.0971801924616865	1.48296593186373\\
0.104208416833667	1.47507538159354\\
0.111343165528784	1.46693386773547\\
0.120240480961924	1.45676279285712\\
0.125287641884237	1.45090180360721\\
0.13627254509018	1.43811938573817\\
0.139023272476939	1.43486973947896\\
0.152304609218437	1.41914313537665\\
0.152558840563168	1.4188376753507\\
0.165878065541669	1.40280561122244\\
0.168336673346693	1.39983885460865\\
0.179011697335403	1.38677354709419\\
0.18436873747495	1.3801984290109\\
0.191969329661104	1.37074148296593\\
0.200400801603206	1.36021831785027\\
0.20475718723836	1.35470941883768\\
0.216432865731463	1.33989543890279\\
0.21738088618959	1.33867735470942\\
0.229825446163612	1.32264529058116\\
0.232464929859719	1.31923379942882\\
0.242111014129311	1.30661322645291\\
0.248496993987976	1.29822577640305\\
0.254249679111395	1.29058116232465\\
0.264529058116232	1.27686483474363\\
0.266245197116871	1.27454909819639\\
0.278085913715293	1.25851703406814\\
0.280561122244489	1.25515324083749\\
0.289782528321357	1.24248496993988\\
0.296593186372745	1.23308517177625\\
0.301349186753381	1.22645290581162\\
0.312625250501002	1.2106514005922\\
0.312788164902552	1.21042084168337\\
0.324080466620025	1.19438877755511\\
0.328657314629258	1.18785783358095\\
0.335252744928811	1.17835671342685\\
0.344689378757515	1.16468909288222\\
0.346307219645143	1.1623246492986\\
0.357232011798649	1.14629258517034\\
0.360721442885771	1.141146504018\\
0.368039327265191	1.13026052104208\\
0.376753507014028	1.11722056863035\\
0.378736718869431	1.11422845691383\\
0.389314546384562	1.09819639278557\\
0.392785571142285	1.09290786061383\\
0.399781828147799	1.08216432865731\\
0.408817635270541	1.06819939383122\\
0.410145301707386	1.06613226452906\\
0.420395503992964	1.0501002004008\\
0.424849699398798	1.04309121984003\\
0.430543682829265	1.03406813627254\\
0.440592084630359	1.01803607214429\\
0.440881763527054	1.0175719344274\\
0.450533390184912	1.00200400801603\\
0.456913827655311	0.991638126868775\\
0.460379816391221	0.985971943887775\\
0.470127348333343	0.969939879759519\\
0.472945891783567	0.965277406046434\\
0.479778661095603	0.953907815631262\\
0.488977955911824	0.938481885432903\\
0.489337460739446	0.937875751503006\\
0.498800763532824	0.921843687374749\\
0.50501002004008	0.911242184065805\\
0.508174479282237	0.905811623246493\\
0.517457049057759	0.889779559118236\\
0.521042084168337	0.883546681902311\\
0.526651618119988	0.87374749498998\\
0.535757770323694	0.857715430861724\\
0.537074148296593	0.855384760374438\\
0.544778659699726	0.841683366733467\\
0.55310621242485	0.826743996246779\\
0.553712876870233	0.825651302605211\\
0.562564761636584	0.809619238476954\\
0.569138276553106	0.797609863137293\\
0.571332121958972	0.793587174348698\\
0.5800184813075	0.777555110220441\\
0.585170340681363	0.767969139974112\\
0.588623238808741	0.761523046092185\\
0.597147798796265	0.745490981963928\\
0.601202404809619	0.737806525696426\\
0.605593806770739	0.729458917835672\\
0.613960138273222	0.713426853707415\\
0.617234468937876	0.707105294487803\\
0.62225086427468	0.697394789579159\\
0.630462387886829	0.681362725450902\\
0.633266533066132	0.675847212770085\\
0.638600927584093	0.665330661322646\\
0.64666091823862	0.649298597194389\\
0.649298597194389	0.644012448665355\\
0.654650008190866	0.633266533066132\\
0.662561599507746	0.617234468937876\\
0.665330661322646	0.611579473271343\\
0.670403628910887	0.601202404809619\\
0.678169817286189	0.585170340681363\\
0.681362725450902	0.578524953025124\\
0.685866838737787	0.569138276553106\\
0.69349048718108	0.55310621242485\\
0.697394789579159	0.544823632351482\\
0.701044226507351	0.537074148296593\\
0.708528068235948	0.521042084168337\\
0.713426853707415	0.510448205704373\\
0.715939933420816	0.50501002004008\\
0.723286575218568	0.488977955911824\\
0.729458917835672	0.475369178011716\\
0.7305576644713	0.472945891783567\\
0.737769589818998	0.456913827655311\\
0.744902759144508	0.440881763527054\\
0.745490981963928	0.439549388500092\\
0.751980270797614	0.424849699398798\\
0.758981402159146	0.408817635270541\\
0.761523046092185	0.402945164657454\\
0.765921363119241	0.392785571142285\\
0.772792646047509	0.376753507014028\\
0.777555110220441	0.365527472868786\\
0.779595206106097	0.360721442885771\\
0.78633871591627	0.344689378757515\\
0.79300605003884	0.328657314629258\\
0.793587174348698	0.327247893379989\\
0.799621440602657	0.312625250501002\\
0.806163058730893	0.296593186372745\\
0.809619238476954	0.288036544144152\\
0.81264225875116	0.280561122244489\\
0.819059962112337	0.264529058116232\\
0.825403267700109	0.248496993987976\\
0.825651302605211	0.247864371429403\\
0.831697706153768	0.232464929859719\\
0.837918726609621	0.216432865731463\\
0.841683366733467	0.206625295316764\\
0.84407685759974	0.200400801603206\\
0.850177130537457	0.18436873747495\\
0.856204317512007	0.168336673346693\\
0.857715430861724	0.164275435029403\\
0.862178565765762	0.152304609218437\\
0.868086222771197	0.13627254509018\\
0.87374749498998	0.120720775558425\\
0.873922746025479	0.120240480961924\\
0.879712154483288	0.104208416833667\\
0.88542933896269	0.0881763527054105\\
0.889779559118236	0.075829463852014\\
0.891081352733647	0.0721442885771539\\
0.896681104837513	0.0561122244488974\\
0.902209517311712	0.0400801603206409\\
0.905811623246493	0.0295057008813569\\
0.907676036994777	0.0240480961923843\\
0.913087586458924	0.00801603206412782\\
0.918428489847537	-0.00801603206412826\\
0.921843687374749	-0.0183980272195427\\
0.923708101123034	-0.0240480961923848\\
0.928932465899462	-0.0400801603206413\\
0.934086689113959	-0.0561122244488979\\
0.937875751503006	-0.0680549808883071\\
0.939177545118417	-0.0721442885771544\\
0.944215309975964	-0.0881763527054109\\
0.949183249838047	-0.104208416833667\\
0.953907815631262	-0.119669068517051\\
0.954083066666762	-0.120240480961924\\
0.958934384098946	-0.13627254509018\\
0.963716005591637	-0.152304609218437\\
0.968428766409803	-0.168336673346694\\
0.969939879759519	-0.173551239282138\\
0.973086645577692	-0.18436873747495\\
0.977681480427534	-0.200400801603207\\
0.982207303763929	-0.216432865731463\\
0.985971943887775	-0.229972659443533\\
0.986667731865482	-0.232464929859719\\
0.991074875445203	-0.248496993987976\\
0.995412667523159	-0.264529058116232\\
0.99968171568218	-0.280561122244489\\
1.00200400801603	-0.289425026676692\\
1.00389005021323	-0.296593186372745\\
1.00803827426999	-0.312625250501002\\
1.01211712930511	-0.328657314629258\\
1.01612708530091	-0.344689378757515\\
1.01803607214429	-0.352457101605711\\
1.02007616802994	-0.360721442885771\\
1.0239626482088	-0.376753507014028\\
1.02777931323051	-0.392785571142285\\
1.03152649233951	-0.408817635270541\\
1.03406813627254	-0.419900398966124\\
1.03520845019296	-0.424849699398798\\
1.03882897978193	-0.440881763527054\\
1.04237880825587	-0.456913827655311\\
1.04585811988899	-0.472945891783567\\
1.04926705264233	-0.488977955911824\\
1.0501002004008	-0.492983895502286\\
1.0526132801142	-0.50501002004008\\
1.05589018644029	-0.521042084168337\\
1.0590950852347	-0.537074148296593\\
1.06222796213098	-0.55310621242485\\
1.06528875386917	-0.569138276553106\\
1.06613226452906	-0.573670750899876\\
1.06828280652761	-0.585170340681363\\
1.0712052321173	-0.601202404809619\\
1.07405350670123	-0.617234468937876\\
1.07682740507046	-0.633266533066132\\
1.07952664970155	-0.649298597194389\\
1.08215090986002	-0.665330661322646\\
1.08216432865731	-0.66541532184826\\
1.08470439642679	-0.681362725450902\\
1.08718072399412	-0.697394789579158\\
1.08957941337184	-0.713426853707415\\
1.09189995632039	-0.729458917835671\\
1.09414178677222	-0.745490981963928\\
1.09630427959414	-0.761523046092184\\
1.09819639278557	-0.776094740459558\\
1.09838695251236	-0.777555110220441\\
1.10039023819144	-0.793587174348697\\
1.10231088468654	-0.809619238476954\\
1.10414806928106	-0.82565130260521\\
1.1059009041887	-0.841683366733467\\
1.10756843491448	-0.857715430861723\\
1.10914963852203	-0.87374749498998\\
1.11064342180325	-0.889779559118236\\
1.11204861934615	-0.905811623246493\\
1.11336399149656	-0.92184368737475\\
1.11422845691383	-0.933182006967936\\
1.11458796174145	-0.937875751503006\\
1.1157184454276	-0.953907815631263\\
1.11675443604705	-0.969939879759519\\
1.11769444564974	-0.985971943887776\\
1.11853690276846	-1.00200400801603\\
1.11928014981559	-1.01803607214429\\
1.11992244034429	-1.03406813627255\\
1.12046193616776	-1.0501002004008\\
1.12089670432974	-1.06613226452906\\
1.12122471391934	-1.08216432865731\\
1.12144383272227	-1.09819639278557\\
1.12155182370072	-1.11422845691383\\
1.12154634129325	-1.13026052104208\\
1.12142492752561	-1.14629258517034\\
1.12118500792299	-1.1623246492986\\
1.12082388721338	-1.17835671342685\\
1.12033874481139	-1.19438877755511\\
1.1197266300709	-1.21042084168337\\
1.11898445729446	-1.22645290581162\\
1.11810900048634	-1.24248496993988\\
1.11709688783561	-1.25851703406814\\
1.11594459591447	-1.27454909819639\\
1.11464844357632	-1.29058116232465\\
1.11422845691383	-1.29527490685972\\
1.11319708140095	-1.30661322645291\\
1.11158897321772	-1.32264529058116\\
1.1098226773115	-1.33867735470942\\
1.10789361042715	-1.35470941883768\\
1.10579698497173	-1.37074148296593\\
1.10352780003294	-1.38677354709419\\
1.1010808319104	-1.40280561122244\\
1.0984506241303	-1.4188376753507\\
1.09819639278557	-1.42029804511158\\
1.09560322010549	-1.43486973947896\\
1.09255398071092	-1.45090180360721\\
1.08929907735243	-1.46693386773547\\
1.08583167329455	-1.48296593186373\\
1.08216432865731	-1.49891333546637\\
1.0821443520793	-1.49899799599198\\
1.07817537844418	-1.51503006012024\\
1.07396500034531	-1.5310621242485\\
1.06950457834855	-1.54709418837675\\
1.06613226452906	-1.55859377815824\\
1.0647629193205	-1.56312625250501\\
1.05969013932373	-1.57915831663327\\
1.05432972957213	-1.59519038076152\\
1.0501002004008	-1.60721650529932\\
1.04864334990002	-1.61122244488978\\
1.04255657480825	-1.62725450901804\\
1.0361352655747	-1.64328657314629\\
1.03406813627254	-1.64823587357897\\
1.0292623072434	-1.65931863727455\\
1.02196731801052	-1.67535070140281\\
1.01803607214429	-1.68361504268287\\
1.01418588617855	-1.69138276553106\\
1.00587990953774	-1.70741482965932\\
1.00200400801603	-1.71458298935537\\
0.996990227098052	-1.72344689378758\\
0.987502407962091	-1.73947895791583\\
0.985971943887775	-1.74197122833202\\
0.977236362534941	-1.75551102204409\\
0.969939879759519	-1.7663285202369\\
0.966230684303344	-1.77154308617234\\
0.954346209377058	-1.7875751503006\\
0.953907815631262	-1.78814656274547\\
0.94133327615024	-1.80360721442886\\
0.937875751503006	-1.8076965221177\\
0.927121552749066	-1.81963927855711\\
0.921843687374749	-1.82528934752995\\
0.911461692219333	-1.83567134268537\\
0.905811623246493	-1.84112894737434\\
0.894024105065609	-1.85170340681363\\
0.889779559118236	-1.85538858208849\\
0.874365875445442	-1.86773547094188\\
0.87374749498998	-1.86821576553839\\
0.857715430861724	-1.87970629675285\\
0.851492446151481	-1.88376753507014\\
0.841683366733467	-1.8899920287837\\
0.825651302605211	-1.89916697663982\\
0.824429387048632	-1.8997995991984\\
0.809619238476954	-1.90727502109806\\
0.793587174348698	-1.91442224207739\\
0.790015546419612	-1.91583166332665\\
0.777555110220441	-1.92063769330967\\
0.761523046092185	-1.92599125684182\\
0.745490981963928	-1.93053135242795\\
0.739874814442951	-1.93186372745491\\
0.729458917835672	-1.93428701368306\\
0.713426853707415	-1.9373019131192\\
0.697394789579159	-1.9396132115098\\
0.681362725450902	-1.94125040392693\\
0.665330661322646	-1.94224079591663\\
0.649298597194389	-1.94260964305413\\
0.633266533066132	-1.94238027890235\\
0.617234468937876	-1.94157423236356\\
0.601202404809619	-1.94021133531566\\
0.585170340681363	-1.93830982133684\\
0.569138276553106	-1.93588641624351\\
0.55310621242485	-1.93295642109648\\
0.548019381576782	-1.93186372745491\\
0.537074148296593	-1.92953305696762\\
0.521042084168337	-1.92563085023898\\
0.50501002004008	-1.92126222414596\\
0.488977955911824	-1.91643779725655\\
0.487146074453037	-1.91583166332665\\
0.472945891783567	-1.91116918961357\\
0.456913827655311	-1.9054657821794\\
0.442103679083632	-1.8997995991984\\
0.440881763527054	-1.89933546148151\\
0.424849699398798	-1.89279061863763\\
0.408817635270541	-1.8858346643723\\
0.404332999824194	-1.88376753507014\\
0.392785571142285	-1.8784790028984\\
0.376753507014028	-1.87072758265841\\
0.370881128353451	-1.86773547094188\\
0.360721442885771	-1.86258938978954\\
0.344689378757515	-1.85406785039725\\
0.340444832810142	-1.85170340681363\\
0.328657314629258	-1.84517246283947\\
0.312625250501002	-1.83590190159421\\
0.312243254575274	-1.83567134268537\\
0.296593186372745	-1.82627154452174\\
0.285972676222856	-1.81963927855711\\
0.280561122244489	-1.81627548532647\\
0.264529058116232	-1.8059229509761\\
0.261071533468998	-1.80360721442886\\
0.248496993987976	-1.795219764379\\
0.237425008585017	-1.7875751503006\\
0.232464929859719	-1.78416365914826\\
0.216432865731463	-1.77276117036572\\
0.21477460953327	-1.77154308617234\\
0.200400801603206	-1.76101992105668\\
0.193104318827784	-1.75551102204409\\
0.18436873747495	-1.7489359039608\\
0.172173345241226	-1.73947895791583\\
0.168336673346693	-1.73651220130204\\
0.152304609218437	-1.72375235381353\\
0.151931278795001	-1.72344689378758\\
0.13627254509018	-1.71066447591853\\
0.132396643568474	-1.70741482965932\\
0.120240480961924	-1.69724375478096\\
0.113414571508777	-1.69138276553106\\
0.104208416833667	-1.68349221526087\\
0.0949445761708529	-1.67535070140281\\
0.0881763527054105	-1.66941162343815\\
0.0769501176063007	-1.65931863727455\\
0.0721442885771539	-1.6550034888922\\
0.0593981639867881	-1.64328657314629\\
0.0561122244488974	-1.64026906664833\\
0.042258810450849	-1.62725450901804\\
0.0400801603206409	-1.62520935879441\\
0.0255049466931636	-1.61122244488978\\
0.0240480961923843	-1.60982511577077\\
0.00911196622221084	-1.59519038076152\\
0.00801603206412782	-1.59411683733554\\
-0.0069424886381464	-1.57915831663327\\
-0.00801603206412826	-1.57808477320728\\
-0.0226787509838312	-1.56312625250501\\
-0.0240480961923848	-1.561728923386\\
-0.0381153270050997	-1.54709418837675\\
-0.0400801603206413	-1.54504903815313\\
-0.0532690706591176	-1.5310621242485\\
-0.0561122244488979	-1.52804461775053\\
-0.0681553383640199	-1.51503006012024\\
-0.0721442885771544	-1.5107149117379\\
-0.082788126291551	-1.49899799599198\\
-0.0881763527054109	-1.49305891802733\\
-0.0971801924616865	-1.48296593186373\\
-0.104208416833667	-1.47507538159353\\
-0.111343165528784	-1.46693386773547\\
-0.120240480961924	-1.45676279285711\\
-0.125287641884237	-1.45090180360721\\
-0.13627254509018	-1.43811938573817\\
-0.13902327247694	-1.43486973947896\\
-0.152304609218437	-1.41914313537665\\
-0.152558840563168	-1.4188376753507\\
-0.165878065541669	-1.40280561122244\\
-0.168336673346694	-1.39983885460865\\
-0.179011697335402	-1.38677354709419\\
-0.18436873747495	-1.3801984290109\\
-0.191969329661105	-1.37074148296593\\
-0.200400801603207	-1.36021831785027\\
-0.204757187238359	-1.35470941883768\\
-0.216432865731463	-1.33989543890279\\
-0.21738088618959	-1.33867735470942\\
-0.229825446163611	-1.32264529058116\\
-0.232464929859719	-1.31923379942883\\
-0.242111014129311	-1.30661322645291\\
-0.248496993987976	-1.29822577640305\\
-0.254249679111395	-1.29058116232465\\
-0.264529058116232	-1.27686483474363\\
-0.266245197116871	-1.27454909819639\\
-0.278085913715293	-1.25851703406814\\
-0.280561122244489	-1.25515324083749\\
-0.289782528321357	-1.24248496993988\\
-0.296593186372745	-1.23308517177625\\
-0.301349186753381	-1.22645290581162\\
-0.312625250501002	-1.2106514005922\\
-0.312788164902552	-1.21042084168337\\
-0.324080466620025	-1.19438877755511\\
-0.328657314629258	-1.18785783358095\\
-0.33525274492881	-1.17835671342685\\
-0.344689378757515	-1.16468909288222\\
-0.346307219645143	-1.1623246492986\\
-0.357232011798649	-1.14629258517034\\
-0.360721442885771	-1.141146504018\\
-0.368039327265191	-1.13026052104208\\
-0.376753507014028	-1.11722056863035\\
-0.37873671886943	-1.11422845691383\\
-0.389314546384562	-1.09819639278557\\
-0.392785571142285	-1.09290786061383\\
-0.399781828147799	-1.08216432865731\\
-0.408817635270541	-1.06819939383122\\
-0.410145301707386	-1.06613226452906\\
-0.420395503992964	-1.0501002004008\\
-0.424849699398798	-1.04309121984003\\
-0.430543682829265	-1.03406813627255\\
-0.440592084630359	-1.01803607214429\\
-0.440881763527054	-1.0175719344274\\
-0.450533390184912	-1.00200400801603\\
-0.456913827655311	-0.991638126868775\\
-0.46037981639122	-0.985971943887776\\
-0.470127348333342	-0.969939879759519\\
-0.472945891783567	-0.965277406046435\\
-0.479778661095603	-0.953907815631263\\
-0.488977955911824	-0.938481885432904\\
-0.489337460739446	-0.937875751503006\\
-0.498800763532823	-0.92184368737475\\
-0.50501002004008	-0.911242184065805\\
-0.508174479282237	-0.905811623246493\\
-0.517457049057759	-0.889779559118236\\
-0.521042084168337	-0.883546681902311\\
-0.526651618119988	-0.87374749498998\\
-0.535757770323694	-0.857715430861723\\
-0.537074148296593	-0.855384760374438\\
-0.544778659699726	-0.841683366733467\\
-0.55310621242485	-0.826743996246779\\
-0.553712876870233	-0.82565130260521\\
-0.562564761636584	-0.809619238476954\\
-0.569138276553106	-0.797609863137293\\
-0.571332121958972	-0.793587174348697\\
-0.5800184813075	-0.777555110220441\\
-0.585170340681363	-0.767969139974112\\
-0.588623238808741	-0.761523046092184\\
-0.597147798796266	-0.745490981963928\\
-0.601202404809619	-0.737806525696426\\
-0.60559380677074	-0.729458917835671\\
-0.613960138273222	-0.713426853707415\\
-0.617234468937876	-0.707105294487803\\
-0.622250864274681	-0.697394789579158\\
-0.630462387886829	-0.681362725450902\\
-0.633266533066132	-0.675847212770085\\
-0.638600927584093	-0.665330661322646\\
-0.646660918238621	-0.649298597194389\\
-0.649298597194389	-0.644012448665355\\
-0.654650008190866	-0.633266533066132\\
-0.662561599507746	-0.617234468937876\\
-0.665330661322646	-0.611579473271343\\
-0.670403628910887	-0.601202404809619\\
-0.678169817286189	-0.585170340681363\\
-0.681362725450902	-0.578524953025125\\
-0.685866838737788	-0.569138276553106\\
-0.693490487181081	-0.55310621242485\\
-0.697394789579158	-0.544823632351483\\
-0.701044226507351	-0.537074148296593\\
-0.708528068235948	-0.521042084168337\\
-0.713426853707415	-0.510448205704374\\
-0.715939933420816	-0.50501002004008\\
-0.723286575218568	-0.488977955911824\\
-0.729458917835671	-0.475369178011717\\
-0.730557664471301	-0.472945891783567\\
-0.737769589818998	-0.456913827655311\\
-0.744902759144508	-0.440881763527054\\
-0.745490981963928	-0.439549388500093\\
-0.751980270797614	-0.424849699398798\\
-0.758981402159147	-0.408817635270541\\
-0.761523046092184	-0.402945164657456\\
-0.765921363119241	-0.392785571142285\\
-0.772792646047509	-0.376753507014028\\
-0.777555110220441	-0.365527472868787\\
-0.779595206106096	-0.360721442885771\\
-0.78633871591627	-0.344689378757515\\
-0.79300605003884	-0.328657314629258\\
-0.793587174348697	-0.327247893379991\\
-0.799621440602657	-0.312625250501002\\
-0.806163058730893	-0.296593186372745\\
-0.809619238476954	-0.288036544144153\\
-0.81264225875116	-0.280561122244489\\
-0.819059962112338	-0.264529058116232\\
-0.825403267700109	-0.248496993987976\\
-0.82565130260521	-0.247864371429404\\
-0.831697706153768	-0.232464929859719\\
-0.83791872660962	-0.216432865731463\\
-0.841683366733467	-0.206625295316765\\
-0.84407685759974	-0.200400801603207\\
-0.850177130537458	-0.18436873747495\\
-0.856204317512007	-0.168336673346694\\
-0.857715430861723	-0.164275435029404\\
-0.862178565765761	-0.152304609218437\\
-0.868086222771196	-0.13627254509018\\
-0.87374749498998	-0.120720775558426\\
-0.873922746025479	-0.120240480961924\\
-0.879712154483288	-0.104208416833667\\
-0.885429338962689	-0.0881763527054109\\
-0.889779559118236	-0.075829463852013\\
-0.891081352733647	-0.0721442885771544\\
-0.896681104837513	-0.0561122244488979\\
-0.902209517311712	-0.0400801603206413\\
-0.905811623246493	-0.0295057008813555\\
-0.907676036994777	-0.0240480961923848\\
-0.913087586458924	-0.00801603206412826\\
-0.918428489847537	0.00801603206412782\\
-0.92184368737475	0.0183980272195441\\
-0.923708101123034	0.0240480961923843\\
-0.928932465899462	0.0400801603206409\\
-0.934086689113959	0.0561122244488974\\
-0.937875751503006	0.0680549808883089\\
-0.939177545118416	0.0721442885771539\\
-0.944215309975965	0.0881763527054105\\
-0.949183249838047	0.104208416833667\\
-0.953907815631263	0.119669068517051\\
-0.954083066666762	0.120240480961924\\
-0.958934384098946	0.13627254509018\\
-0.963716005591637	0.152304609218437\\
-0.968428766409803	0.168336673346693\\
-0.969939879759519	0.173551239282139\\
-0.973086645577692	0.18436873747495\\
-0.977681480427534	0.200400801603206\\
-0.982207303763929	0.216432865731463\\
-0.985971943887776	0.229972659443534\\
-0.986667731865482	0.232464929859719\\
-0.991074875445203	0.248496993987976\\
-0.995412667523159	0.264529058116232\\
-0.99968171568218	0.280561122244489\\
-1.00200400801603	0.289425026676694\\
-1.00389005021323	0.296593186372745\\
-1.00803827426999	0.312625250501002\\
-1.01211712930511	0.328657314629258\\
-1.01612708530091	0.344689378757515\\
-1.01803607214429	0.352457101605714\\
-1.02007616802994	0.360721442885771\\
-1.0239626482088	0.376753507014028\\
-1.02777931323051	0.392785571142285\\
-1.03152649233951	0.408817635270541\\
-1.03406813627255	0.419900398966127\\
-1.03520845019296	0.424849699398798\\
-1.03882897978193	0.440881763527054\\
-1.04237880825587	0.456913827655311\\
-1.04585811988899	0.472945891783567\\
-1.04926705264232	0.488977955911824\\
-1.0501002004008	0.492983895502289\\
-1.0526132801142	0.50501002004008\\
-1.05589018644029	0.521042084168337\\
-1.0590950852347	0.537074148296593\\
-1.06222796213098	0.55310621242485\\
-1.06528875386917	0.569138276553106\\
-1.06613226452906	0.57367075089988\\
-1.06828280652761	0.585170340681363\\
-1.0712052321173	0.601202404809619\\
-1.07405350670123	0.617234468937876\\
-1.07682740507046	0.633266533066132\\
-1.07952664970155	0.649298597194389\\
-1.08215090986002	0.665330661322646\\
-1.08216432865731	0.665415321848262\\
-1.08470439642679	0.681362725450902\\
-1.08718072399412	0.697394789579159\\
-1.08957941337184	0.713426853707415\\
-1.09189995632039	0.729458917835672\\
-1.09414178677222	0.745490981963928\\
-1.09630427959414	0.761523046092185\\
-1.09819639278557	0.776094740459559\\
-1.09838695251236	0.777555110220441\\
-1.10039023819144	0.793587174348698\\
-1.10231088468654	0.809619238476954\\
-1.10414806928106	0.825651302605211\\
-1.1059009041887	0.841683366733467\\
-1.10756843491448	0.857715430861724\\
-1.10914963852203	0.87374749498998\\
-1.11064342180325	0.889779559118236\\
-1.11204861934615	0.905811623246493\\
-1.11336399149656	0.921843687374749\\
-1.11422845691383	0.933182006967935\\
-1.11458796174145	0.937875751503006\\
-1.1157184454276	0.953907815631262\\
-1.11675443604705	0.969939879759519\\
-1.11769444564974	0.985971943887775\\
-1.11853690276846	1.00200400801603\\
-1.11928014981559	1.01803607214429\\
-1.11992244034429	1.03406813627254\\
-1.12046193616776	1.0501002004008\\
-1.12089670432974	1.06613226452906\\
-1.12122471391934	1.08216432865731\\
-1.12144383272227	1.09819639278557\\
-1.12155182370072	1.11422845691383\\
-1.12154634129325	1.13026052104208\\
-1.12142492752561	1.14629258517034\\
-1.12118500792299	1.1623246492986\\
-1.12082388721338	1.17835671342685\\
-1.12033874481139	1.19438877755511\\
-1.1197266300709	1.21042084168337\\
-1.11898445729446	1.22645290581162\\
-1.11810900048634	1.24248496993988\\
-1.11709688783561	1.25851703406814\\
-1.11594459591447	1.27454909819639\\
-1.11464844357632	1.29058116232465\\
-1.11422845691383	1.29527490685972\\
}--cycle;


\addplot[area legend,solid,fill=mycolor8,draw=black,forget plot]
table[row sep=crcr] {%
x	y\\
-0.969939879759519	1.10267832172919\\
-0.969057219367736	1.11422845691383\\
-0.967666023500817	1.13026052104208\\
-0.96610008118631	1.14629258517034\\
-0.964353882669638	1.1623246492986\\
-0.962421651452387	1.17835671342685\\
-0.960297331194264	1.19438877755511\\
-0.957974571822777	1.21042084168337\\
-0.955446714796597	1.22645290581162\\
-0.953907815631263	1.23550430311173\\
-0.95268652032031	1.24248496993988\\
-0.949675465068525	1.25851703406814\\
-0.946430689407037	1.27454909819639\\
-0.942943626489223	1.29058116232465\\
-0.939205278892995	1.30661322645291\\
-0.937875751503006	1.31199129212065\\
-0.935151423287963	1.32264529058116\\
-0.930790674073424	1.33867735470942\\
-0.926138506431433	1.35470941883768\\
-0.92184368737475	1.36863159698073\\
-0.921167596555201	1.37074148296593\\
-0.91576922275681	1.38677354709419\\
-0.910028348601568	1.40280561122244\\
-0.905811623246493	1.4139460396668\\
-0.903879403874429	1.4188376753507\\
-0.89722860759457	1.43486973947896\\
-0.890169387584158	1.45090180360721\\
-0.889779559118236	1.45175054669336\\
-0.882464023832977	1.46693386773547\\
-0.874275126038673	1.48296593186373\\
-0.87374749498998	1.4839573570158\\
-0.865310295946134	1.49899799599198\\
-0.857715430861723	1.51182056797403\\
-0.85570327730774	1.51503006012024\\
-0.845197697745205	1.5310621242485\\
-0.841683366733467	1.5361856585203\\
-0.83371642719017	1.54709418837675\\
-0.82565130260521	1.55763141692518\\
-0.821149424312379	1.56312625250501\\
-0.809619238476954	1.57659144839573\\
-0.807252347724585	1.57915831663327\\
-0.793587174348697	1.59337379320406\\
-0.791693855339852	1.59519038076152\\
-0.777555110220441	1.60823431958121\\
-0.774015623754877	1.61122244488978\\
-0.761523046092184	1.62138648351191\\
-0.753568053904924	1.62725450901804\\
-0.745490981963928	1.6330089628635\\
-0.729458917835671	1.64325132499136\\
-0.729397545610944	1.64328657314629\\
-0.713426853707415	1.65217400092951\\
-0.698733223512349	1.65931863727455\\
-0.697394789579158	1.65995043575415\\
-0.681362725450902	1.66660099071443\\
-0.665330661322646	1.67225062416767\\
-0.654827748809892	1.67535070140281\\
-0.649298597194389	1.67694127241858\\
-0.633266533066132	1.68072001757187\\
-0.617234468937876	1.6836551377129\\
-0.601202404809619	1.68578743348757\\
-0.585170340681363	1.68715450310275\\
-0.569138276553106	1.6877909773566\\
-0.55310621242485	1.68772873313515\\
-0.537074148296593	1.68699708749149\\
-0.521042084168337	1.68562297418631\\
-0.50501002004008	1.68363110436025\\
-0.488977955911824	1.681044112825\\
-0.472945891783567	1.67788269129889\\
-0.462062759340567	1.67535070140281\\
-0.456913827655311	1.67416168445254\\
-0.440881763527054	1.66989348069114\\
-0.424849699398798	1.66510347263781\\
-0.408817635270541	1.65980612744266\\
-0.40747920133735	1.65931863727455\\
-0.392785571142285	1.65400113858226\\
-0.376753507014028	1.64771432457223\\
-0.366276833404529	1.64328657314629\\
-0.360721442885771	1.64095253996619\\
-0.344689378757515	1.63372219754025\\
-0.331202431040213	1.62725450901804\\
-0.328657314629258	1.62604056421676\\
-0.312625250501002	1.61790654165211\\
-0.300132672838309	1.61122244488978\\
-0.296593186372745	1.60933788456875\\
-0.280561122244489	1.60033434295526\\
-0.271830152307451	1.59519038076152\\
-0.264529058116232	1.59090760245724\\
-0.248496993987976	1.5810633694857\\
-0.245529404938141	1.57915831663327\\
-0.232464929859719	1.57080380915392\\
-0.220934744024294	1.56312625250501\\
-0.216432865731463	1.56013937290885\\
-0.200400801603207	1.54907207324439\\
-0.19764255158661	1.54709418837675\\
-0.18436873747495	1.53760511938098\\
-0.175535084883226	1.5310621242485\\
-0.168336673346694	1.52574538852246\\
-0.15431676277242	1.51503006012024\\
-0.152304609218437	1.51349613165084\\
-0.13627254509018	1.50085872488534\\
-0.133985795284029	1.49899799599198\\
-0.120240480961924	1.48783640452733\\
-0.11442157751627	1.48296593186373\\
-0.104208416833667	1.47443279432367\\
-0.0954918879906701	1.46693386773547\\
-0.0881763527054109	1.46064992979705\\
-0.0771441629499948	1.45090180360721\\
-0.0721442885771544	1.44648955083659\\
-0.0593315894323775	1.43486973947896\\
-0.0561122244488979	1.4319531039411\\
-0.0420123796927054	1.4188376753507\\
-0.0400801603206413	1.41704174408248\\
-0.0251491528545071	1.40280561122244\\
-0.0240480961923848	1.40175633619233\\
-0.00870840581046833	1.38677354709419\\
-0.00801603206412826	1.38609745627464\\
0.00733994124458037	1.37074148296593\\
0.00801603206412782	1.37006539214638\\
0.0230229365910116	1.35470941883768\\
0.0240480961923843	1.35366014380756\\
0.0383649358792694	1.33867735470942\\
0.0400801603206409	1.33688142344119\\
0.0533878962338543	1.32264529058116\\
0.0561122244488974	1.3197286550433\\
0.0681116322727954	1.30661322645291\\
0.0721442885771539	1.30220097368228\\
0.0825540396247736	1.29058116232465\\
0.0881763527054105	1.28429722438623\\
0.0967312906094415	1.27454909819639\\
0.104208416833667	1.26601596065633\\
0.110658005995674	1.25851703406814\\
0.120240480961924	1.24735544260349\\
0.124347406134783	1.24248496993988\\
0.13627254509018	1.22831363470498\\
0.137811444255516	1.22645290581162\\
0.151041518246361	1.21042084168337\\
0.152304609218437	1.20888691321397\\
0.164043027344138	1.19438877755511\\
0.168336673346693	1.18907204182907\\
0.176850509167818	1.17835671342685\\
0.18436873747495	1.16886764443108\\
0.189472282332026	1.1623246492986\\
0.200400801603206	1.14827047003798\\
0.201915825520064	1.14629258517034\\
0.21415900947276	1.13026052104208\\
0.216432865731463	1.12727364144593\\
0.226219685290747	1.11422845691383\\
0.232464929859719	1.10587394943448\\
0.238124052009175	1.09819639278557\\
0.248496993987976	1.08406938150975\\
0.249877068771455	1.08216432865731\\
0.261449565491779	1.06613226452906\\
0.264529058116232	1.06184948622477\\
0.272867526498327	1.0501002004008\\
0.280561122244489	1.03921209846628\\
0.284150637741463	1.03406813627254\\
0.295289049977966	1.01803607214429\\
0.296593186372745	1.01615151182326\\
0.306265440129244	1.00200400801603\\
0.312625250501002	0.992656040650101\\
0.317120335276949	0.985971943887775\\
0.327848425282381	0.969939879759519\\
0.328657314629258	0.968725934958247\\
0.338421193847936	0.953907815631262\\
0.344689378757515	0.944343440025218\\
0.348883112072032	0.937875751503006\\
0.359223219238986	0.921843687374749\\
0.360721442885771	0.919509654194643\\
0.369424311362763	0.905811623246493\\
0.376753507014028	0.894207310544178\\
0.37952284977305	0.889779559118236\\
0.389495963689498	0.87374749498998\\
0.392785571142285	0.86842999629769\\
0.399352061568488	0.857715430861724\\
0.408817635270541	0.842170856901581\\
0.409111855743252	0.841683366733467\\
0.418737689933317	0.825651302605211\\
0.424849699398798	0.81540407384021\\
0.428270836853321	0.809619238476954\\
0.437693808705884	0.793587174348698\\
0.440881763527054	0.788129953637036\\
0.447009199199949	0.777555110220441\\
0.456233617637009	0.761523046092185\\
0.456913827655311	0.760334029141922\\
0.465339483189213	0.745490981963928\\
0.472945891783567	0.731990907731753\\
0.474362142761551	0.729458917835672\\
0.48327342673314	0.713426853707415\\
0.488977955911824	0.703088201001349\\
0.492097725170469	0.697394789579159\\
0.500821999933126	0.681362725450902\\
0.50501002004008	0.673611064280093\\
0.509454470646171	0.665330661322646\\
0.51799543742193	0.649298597194389\\
0.521042084168337	0.643538805849636\\
0.526442067692356	0.633266533066132\\
0.534803268494776	0.617234468937876\\
0.537074148296593	0.612848790898299\\
0.543069523425128	0.601202404809619\\
0.551254345150408	0.585170340681363\\
0.55310621242485	0.581516308285451\\
0.559345189869164	0.569138276553106\\
0.567356868153898	0.55310621242485\\
0.569138276553106	0.549514424250385\\
0.575276788374131	0.537074148296593\\
0.583118411224556	0.521042084168337\\
0.585170340681363	0.516813821740019\\
0.590871432247094	0.50501002004008\\
0.598545943444414	0.488977955911824\\
0.601202404809619	0.483382623868328\\
0.606135647689298	0.472945891783567\\
0.613645849975363	0.456913827655311\\
0.617234468937876	0.44918619983715\\
0.621075393118784	0.440881763527054\\
0.62842395116611	0.424849699398798\\
0.633266533066132	0.414186951439601\\
0.635696076953843	0.408817635270541\\
0.642885520123645	0.392785571142285\\
0.649298597194389	0.378344078029799\\
0.650002573926212	0.376753507014028\\
0.657035298817787	0.360721442885771\\
0.663996859551543	0.344689378757515\\
0.665330661322646	0.341589301522375\\
0.670877512781642	0.328657314629258\\
0.677684346364305	0.312625250501002\\
0.681362725450902	0.303875539812623\\
0.684415884465412	0.296593186372745\\
0.691070078445277	0.280561122244489\\
0.697394789579159	0.265160856595837\\
0.697653645295825	0.264529058116232\\
0.704157161442419	0.248496993987976\\
0.71058999818202	0.232464929859719\\
0.713426853707415	0.225320293514676\\
0.716948222425681	0.216432865731463\\
0.723231725412169	0.200400801603206\\
0.729445402395971	0.18436873747495\\
0.729458917835672	0.184333489320014\\
0.7355812556841	0.168336673346693\\
0.741646656558198	0.152304609218437\\
0.745490981963928	0.142026998935648\\
0.747640164685551	0.13627254509018\\
0.753558788829143	0.120240480961924\\
0.759407488220804	0.104208416833667\\
0.761523046092185	0.0983403913275378\\
0.765182801678966	0.0881763527054105\\
0.77088533435874	0.0721442885771539\\
0.776518228317745	0.0561122244488974\\
0.777555110220441	0.053124099140325\\
0.782076830845543	0.0400801603206409\\
0.787563890959639	0.0240480961923843\\
0.792981374834583	0.00801603206412782\\
0.793587174348698	0.00619944450666274\\
0.798324247801242	-0.00801603206412826\\
0.803595955087895	-0.0240480961923848\\
0.80879792546851	-0.0400801603206413\\
0.809619238476954	-0.0426470285581829\\
0.81392555091599	-0.0561122244488979\\
0.818981526743509	-0.0721442885771544\\
0.823967381477785	-0.0881763527054109\\
0.825651302605211	-0.0936711882852378\\
0.828879743236386	-0.104208416833667\\
0.833719109470425	-0.120240480961924\\
0.838487745757591	-0.13627254509018\\
0.841683366733467	-0.14718107494664\\
0.843184328639133	-0.152304609218437\\
0.847805704581895	-0.168336673346694\\
0.852355515131915	-0.18436873747495\\
0.856834199960817	-0.200400801603207\\
0.857715430861724	-0.203610293749422\\
0.86123679957999	-0.216432865731463\\
0.865565666819784	-0.232464929859719\\
0.869822225003994	-0.248496993987976\\
0.87374749498998	-0.263537632964165\\
0.874006350706646	-0.264529058116232\\
0.878111639002643	-0.280561122244489\\
0.8821431997404	-0.296593186372745\\
0.88610118003164	-0.312625250501002\\
0.889779559118236	-0.32780857154311\\
0.889985305387594	-0.328657314629258\\
0.893788465297644	-0.344689378757515\\
0.897516260741634	-0.360721442885771\\
0.901168657328444	-0.376753507014028\\
0.904745563887746	-0.392785571142285\\
0.905811623246493	-0.397677206826184\\
0.908241167134204	-0.408817635270541\\
0.911657017285245	-0.424849699398798\\
0.914995079503241	-0.440881763527054\\
0.918255068412237	-0.456913827655311\\
0.921436637000515	-0.472945891783567\\
0.921843687374749	-0.47505577776877\\
0.924531625014911	-0.488977955911824\\
0.92754477894048	-0.50501002004008\\
0.93047666597542	-0.521042084168337\\
0.933326728992571	-0.537074148296593\\
0.936094343103798	-0.55310621242485\\
0.937875751503006	-0.563760210885365\\
0.938775714503375	-0.569138276553106\\
0.941366364051953	-0.585170340681363\\
0.943871126631541	-0.601202404809619\\
0.946289147749201	-0.617234468937876\\
0.94861949758841	-0.633266533066132\\
0.950861168884856	-0.649298597194389\\
0.953013074674022	-0.665330661322646\\
0.953907815631262	-0.672311328150791\\
0.955068713952377	-0.681362725450902\\
0.957027584889907	-0.697394789579158\\
0.958892252810273	-0.713426853707415\\
0.960661363087028	-0.729458917835671\\
0.962333471165166	-0.745490981963928\\
0.963907039627009	-0.761523046092184\\
0.965380435091499	-0.777555110220441\\
0.966751924938348	-0.793587174348697\\
0.968019673847939	-0.809619238476954\\
0.969181740147287	-0.82565130260521\\
0.969939879759519	-0.837201437789848\\
0.97023410023223	-0.841683366733467\\
0.971171862150369	-0.857715430861723\\
0.971997666920664	-0.87374749498998\\
0.972709222518539	-0.889779559118236\\
0.973304114743161	-0.905811623246493\\
0.973779802541407	-0.92184368737475\\
0.974133613074036	-0.937875751503006\\
0.974362736509592	-0.953907815631263\\
0.974464220530582	-0.969939879759519\\
0.974434964535466	-0.985971943887776\\
0.974271713518829	-1.00200400801603\\
0.973971051610933	-1.01803607214429\\
0.973529395256493	-1.03406813627255\\
0.972942986011204	-1.0501002004008\\
0.97220788293294	-1.06613226452906\\
0.971319954542998	-1.08216432865731\\
0.970274870330963	-1.09819639278557\\
0.969939879759519	-1.10267832172919\\
0.969057219367736	-1.11422845691383\\
0.967666023500816	-1.13026052104208\\
0.96610008118631	-1.14629258517034\\
0.964353882669638	-1.1623246492986\\
0.962421651452387	-1.17835671342685\\
0.960297331194265	-1.19438877755511\\
0.957974571822777	-1.21042084168337\\
0.955446714796598	-1.22645290581162\\
0.953907815631262	-1.23550430311173\\
0.95268652032031	-1.24248496993988\\
0.949675465068525	-1.25851703406814\\
0.946430689407037	-1.27454909819639\\
0.942943626489224	-1.29058116232465\\
0.939205278892995	-1.30661322645291\\
0.937875751503006	-1.31199129212065\\
0.935151423287963	-1.32264529058116\\
0.930790674073424	-1.33867735470942\\
0.926138506431433	-1.35470941883768\\
0.921843687374749	-1.36863159698073\\
0.921167596555201	-1.37074148296593\\
0.91576922275681	-1.38677354709419\\
0.910028348601568	-1.40280561122244\\
0.905811623246493	-1.4139460396668\\
0.903879403874429	-1.4188376753507\\
0.897228607594571	-1.43486973947896\\
0.890169387584157	-1.45090180360721\\
0.889779559118236	-1.45175054669336\\
0.882464023832978	-1.46693386773547\\
0.874275126038673	-1.48296593186373\\
0.87374749498998	-1.4839573570158\\
0.865310295946134	-1.49899799599198\\
0.857715430861724	-1.51182056797403\\
0.85570327730774	-1.51503006012024\\
0.845197697745205	-1.5310621242485\\
0.841683366733467	-1.53618565852029\\
0.83371642719017	-1.54709418837675\\
0.825651302605211	-1.55763141692518\\
0.821149424312379	-1.56312625250501\\
0.809619238476954	-1.57659144839572\\
0.807252347724586	-1.57915831663327\\
0.793587174348698	-1.59337379320406\\
0.791693855339852	-1.59519038076152\\
0.777555110220441	-1.60823431958121\\
0.774015623754877	-1.61122244488978\\
0.761523046092185	-1.62138648351191\\
0.753568053904925	-1.62725450901804\\
0.745490981963928	-1.6330089628635\\
0.729458917835672	-1.64325132499136\\
0.729397545610944	-1.64328657314629\\
0.713426853707415	-1.65217400092951\\
0.698733223512349	-1.65931863727455\\
0.697394789579159	-1.65995043575415\\
0.681362725450902	-1.66660099071443\\
0.665330661322646	-1.67225062416767\\
0.654827748809895	-1.67535070140281\\
0.649298597194389	-1.67694127241858\\
0.633266533066132	-1.68072001757187\\
0.617234468937876	-1.6836551377129\\
0.601202404809619	-1.68578743348757\\
0.585170340681363	-1.68715450310274\\
0.569138276553106	-1.6877909773566\\
0.55310621242485	-1.68772873313515\\
0.537074148296593	-1.68699708749149\\
0.521042084168337	-1.68562297418631\\
0.50501002004008	-1.68363110436025\\
0.488977955911824	-1.681044112825\\
0.472945891783567	-1.67788269129889\\
0.462062759340563	-1.67535070140281\\
0.456913827655311	-1.67416168445254\\
0.440881763527054	-1.66989348069114\\
0.424849699398798	-1.66510347263781\\
0.408817635270541	-1.65980612744266\\
0.407479201337349	-1.65931863727455\\
0.392785571142285	-1.65400113858226\\
0.376753507014028	-1.64771432457223\\
0.366276833404527	-1.64328657314629\\
0.360721442885771	-1.64095253996619\\
0.344689378757515	-1.63372219754025\\
0.331202431040212	-1.62725450901804\\
0.328657314629258	-1.62604056421676\\
0.312625250501002	-1.61790654165211\\
0.300132672838309	-1.61122244488978\\
0.296593186372745	-1.60933788456875\\
0.280561122244489	-1.60033434295526\\
0.27183015230745	-1.59519038076152\\
0.264529058116232	-1.59090760245724\\
0.248496993987976	-1.5810633694857\\
0.245529404938141	-1.57915831663327\\
0.232464929859719	-1.57080380915392\\
0.220934744024294	-1.56312625250501\\
0.216432865731463	-1.56013937290885\\
0.200400801603206	-1.54907207324439\\
0.19764255158661	-1.54709418837675\\
0.18436873747495	-1.53760511938098\\
0.175535084883226	-1.5310621242485\\
0.168336673346693	-1.52574538852245\\
0.15431676277242	-1.51503006012024\\
0.152304609218437	-1.51349613165084\\
0.13627254509018	-1.50085872488534\\
0.133985795284029	-1.49899799599198\\
0.120240480961924	-1.48783640452733\\
0.11442157751627	-1.48296593186373\\
0.104208416833667	-1.47443279432367\\
0.0954918879906697	-1.46693386773547\\
0.0881763527054105	-1.46064992979705\\
0.0771441629499948	-1.45090180360721\\
0.0721442885771539	-1.44648955083659\\
0.0593315894323771	-1.43486973947896\\
0.0561122244488974	-1.4319531039411\\
0.042012379692705	-1.4188376753507\\
0.0400801603206409	-1.41704174408247\\
0.025149152854508	-1.40280561122244\\
0.0240480961923843	-1.40175633619233\\
0.0087084058104688	-1.38677354709419\\
0.00801603206412782	-1.38609745627464\\
-0.0073399412445799	-1.37074148296593\\
-0.00801603206412826	-1.37006539214638\\
-0.023022936591012	-1.35470941883768\\
-0.0240480961923848	-1.35366014380756\\
-0.0383649358792694	-1.33867735470942\\
-0.0400801603206413	-1.33688142344119\\
-0.0533878962338543	-1.32264529058116\\
-0.0561122244488979	-1.3197286550433\\
-0.0681116322727958	-1.30661322645291\\
-0.0721442885771544	-1.30220097368228\\
-0.0825540396247731	-1.29058116232465\\
-0.0881763527054109	-1.28429722438623\\
-0.0967312906094415	-1.27454909819639\\
-0.104208416833667	-1.26601596065633\\
-0.110658005995674	-1.25851703406814\\
-0.120240480961924	-1.24735544260349\\
-0.124347406134783	-1.24248496993988\\
-0.13627254509018	-1.22831363470498\\
-0.137811444255515	-1.22645290581162\\
-0.151041518246361	-1.21042084168337\\
-0.152304609218437	-1.20888691321397\\
-0.164043027344137	-1.19438877755511\\
-0.168336673346694	-1.18907204182907\\
-0.176850509167818	-1.17835671342685\\
-0.18436873747495	-1.16886764443108\\
-0.189472282332025	-1.1623246492986\\
-0.200400801603207	-1.14827047003798\\
-0.201915825520063	-1.14629258517034\\
-0.214159009472761	-1.13026052104208\\
-0.216432865731463	-1.12727364144593\\
-0.226219685290747	-1.11422845691383\\
-0.232464929859719	-1.10587394943448\\
-0.238124052009175	-1.09819639278557\\
-0.248496993987976	-1.08406938150975\\
-0.249877068771455	-1.08216432865731\\
-0.261449565491779	-1.06613226452906\\
-0.264529058116232	-1.06184948622477\\
-0.272867526498328	-1.0501002004008\\
-0.280561122244489	-1.03921209846628\\
-0.284150637741463	-1.03406813627255\\
-0.295289049977966	-1.01803607214429\\
-0.296593186372745	-1.01615151182326\\
-0.306265440129243	-1.00200400801603\\
-0.312625250501002	-0.992656040650101\\
-0.317120335276949	-0.985971943887776\\
-0.327848425282381	-0.969939879759519\\
-0.328657314629258	-0.968725934958247\\
-0.338421193847936	-0.953907815631263\\
-0.344689378757515	-0.944343440025219\\
-0.348883112072033	-0.937875751503006\\
-0.359223219238986	-0.92184368737475\\
-0.360721442885771	-0.919509654194644\\
-0.369424311362763	-0.905811623246493\\
-0.376753507014028	-0.894207310544178\\
-0.379522849773049	-0.889779559118236\\
-0.389495963689499	-0.87374749498998\\
-0.392785571142285	-0.86842999629769\\
-0.399352061568489	-0.857715430861723\\
-0.408817635270541	-0.84217085690158\\
-0.409111855743253	-0.841683366733467\\
-0.418737689933318	-0.82565130260521\\
-0.424849699398798	-0.81540407384021\\
-0.428270836853321	-0.809619238476954\\
-0.437693808705884	-0.793587174348697\\
-0.440881763527054	-0.788129953637036\\
-0.447009199199949	-0.777555110220441\\
-0.45623361763701	-0.761523046092184\\
-0.456913827655311	-0.760334029141922\\
-0.465339483189214	-0.745490981963928\\
-0.472945891783567	-0.731990907731753\\
-0.474362142761551	-0.729458917835671\\
-0.48327342673314	-0.713426853707415\\
-0.488977955911824	-0.703088201001348\\
-0.49209772517047	-0.697394789579158\\
-0.500821999933126	-0.681362725450902\\
-0.50501002004008	-0.673611064280093\\
-0.509454470646172	-0.665330661322646\\
-0.51799543742193	-0.649298597194389\\
-0.521042084168337	-0.643538805849636\\
-0.526442067692356	-0.633266533066132\\
-0.534803268494776	-0.617234468937876\\
-0.537074148296593	-0.612848790898299\\
-0.543069523425128	-0.601202404809619\\
-0.551254345150408	-0.585170340681363\\
-0.55310621242485	-0.581516308285452\\
-0.559345189869164	-0.569138276553106\\
-0.567356868153898	-0.55310621242485\\
-0.569138276553106	-0.549514424250385\\
-0.57527678837413	-0.537074148296593\\
-0.583118411224556	-0.521042084168337\\
-0.585170340681363	-0.516813821740019\\
-0.590871432247094	-0.50501002004008\\
-0.598545943444414	-0.488977955911824\\
-0.601202404809619	-0.483382623868328\\
-0.606135647689298	-0.472945891783567\\
-0.613645849975363	-0.456913827655311\\
-0.617234468937876	-0.449186199837151\\
-0.621075393118784	-0.440881763527054\\
-0.62842395116611	-0.424849699398798\\
-0.633266533066132	-0.414186951439601\\
-0.635696076953843	-0.408817635270541\\
-0.642885520123645	-0.392785571142285\\
-0.649298597194389	-0.378344078029798\\
-0.650002573926212	-0.376753507014028\\
-0.657035298817786	-0.360721442885771\\
-0.663996859551543	-0.344689378757515\\
-0.665330661322646	-0.341589301522375\\
-0.670877512781641	-0.328657314629258\\
-0.677684346364305	-0.312625250501002\\
-0.681362725450902	-0.303875539812625\\
-0.684415884465412	-0.296593186372745\\
-0.691070078445277	-0.280561122244489\\
-0.697394789579158	-0.265160856595838\\
-0.697653645295824	-0.264529058116232\\
-0.704157161442419	-0.248496993987976\\
-0.710589998182019	-0.232464929859719\\
-0.713426853707415	-0.225320293514677\\
-0.716948222425681	-0.216432865731463\\
-0.723231725412169	-0.200400801603207\\
-0.729445402395971	-0.18436873747495\\
-0.729458917835671	-0.184333489320016\\
-0.735581255684099	-0.168336673346694\\
-0.741646656558198	-0.152304609218437\\
-0.745490981963928	-0.142026998935649\\
-0.747640164685551	-0.13627254509018\\
-0.753558788829143	-0.120240480961924\\
-0.759407488220804	-0.104208416833667\\
-0.761523046092184	-0.0983403913275397\\
-0.765182801678965	-0.0881763527054109\\
-0.770885334358739	-0.0721442885771544\\
-0.776518228317744	-0.0561122244488979\\
-0.777555110220441	-0.0531240991403255\\
-0.782076830845542	-0.0400801603206413\\
-0.787563890959638	-0.0240480961923848\\
-0.792981374834583	-0.00801603206412826\\
-0.793587174348697	-0.00619944450666455\\
-0.798324247801242	0.00801603206412782\\
-0.803595955087895	0.0240480961923843\\
-0.80879792546851	0.0400801603206409\\
-0.809619238476954	0.0426470285581824\\
-0.81392555091599	0.0561122244488974\\
-0.818981526743509	0.0721442885771539\\
-0.823967381477785	0.0881763527054105\\
-0.82565130260521	0.0936711882852369\\
-0.828879743236386	0.104208416833667\\
-0.833719109470425	0.120240480961924\\
-0.838487745757591	0.13627254509018\\
-0.841683366733467	0.147181074946638\\
-0.843184328639133	0.152304609218437\\
-0.847805704581895	0.168336673346693\\
-0.852355515131915	0.18436873747495\\
-0.856834199960817	0.200400801603206\\
-0.857715430861723	0.20361029374942\\
-0.861236799579989	0.216432865731463\\
-0.865565666819784	0.232464929859719\\
-0.869822225003994	0.248496993987976\\
-0.87374749498998	0.263537632964162\\
-0.874006350706646	0.264529058116232\\
-0.878111639002643	0.280561122244489\\
-0.8821431997404	0.296593186372745\\
-0.88610118003164	0.312625250501002\\
-0.889779559118236	0.327808571543111\\
-0.889985305387594	0.328657314629258\\
-0.893788465297644	0.344689378757515\\
-0.897516260741634	0.360721442885771\\
-0.901168657328444	0.376753507014028\\
-0.904745563887746	0.392785571142285\\
-0.905811623246493	0.397677206826187\\
-0.908241167134204	0.408817635270541\\
-0.911657017285245	0.424849699398798\\
-0.914995079503241	0.440881763527054\\
-0.918255068412237	0.456913827655311\\
-0.921436637000515	0.472945891783567\\
-0.92184368737475	0.475055777768773\\
-0.924531625014912	0.488977955911824\\
-0.927544778940481	0.50501002004008\\
-0.930476665975421	0.521042084168337\\
-0.933326728992572	0.537074148296593\\
-0.936094343103798	0.55310621242485\\
-0.937875751503006	0.563760210885367\\
-0.938775714503375	0.569138276553106\\
-0.941366364051952	0.585170340681363\\
-0.943871126631541	0.601202404809619\\
-0.9462891477492	0.617234468937876\\
-0.94861949758841	0.633266533066132\\
-0.950861168884856	0.649298597194389\\
-0.953013074674022	0.665330661322646\\
-0.953907815631263	0.672311328150798\\
-0.955068713952377	0.681362725450902\\
-0.957027584889908	0.697394789579159\\
-0.958892252810273	0.713426853707415\\
-0.960661363087028	0.729458917835672\\
-0.962333471165166	0.745490981963928\\
-0.963907039627009	0.761523046092185\\
-0.965380435091499	0.777555110220441\\
-0.966751924938348	0.793587174348698\\
-0.968019673847939	0.809619238476954\\
-0.969181740147288	0.825651302605211\\
-0.969939879759519	0.837201437789848\\
-0.970234100232231	0.841683366733467\\
-0.971171862150368	0.857715430861724\\
-0.971997666920663	0.87374749498998\\
-0.97270922251854	0.889779559118236\\
-0.973304114743161	0.905811623246493\\
-0.973779802541406	0.921843687374749\\
-0.974133613074036	0.937875751503006\\
-0.974362736509593	0.953907815631262\\
-0.974464220530582	0.969939879759519\\
-0.974434964535466	0.985971943887775\\
-0.97427171351883	1.00200400801603\\
-0.973971051610933	1.01803607214429\\
-0.973529395256494	1.03406813627254\\
-0.972942986011204	1.0501002004008\\
-0.97220788293294	1.06613226452906\\
-0.971319954542998	1.08216432865731\\
-0.970274870330964	1.09819639278557\\
-0.969939879759519	1.10267832172919\\
}--cycle;


\addplot[area legend,solid,fill=mycolor9,draw=black,forget plot]
table[row sep=crcr] {%
x	y\\
-0.82565130260521	0.866245169292337\\
-0.825462060494201	0.87374749498998\\
-0.824887980644687	0.889779559118236\\
-0.824136376831622	0.905811623246493\\
-0.823201350070427	0.921843687374749\\
-0.822076686092449	0.937875751503006\\
-0.820755838230285	0.953907815631262\\
-0.819231909149522	0.969939879759519\\
-0.817497631338341	0.985971943887775\\
-0.815545346258589	1.00200400801603\\
-0.813366982053292	1.01803607214429\\
-0.810954029695988	1.03406813627254\\
-0.809619238476954	1.042176361739\\
-0.80826563518729	1.0501002004008\\
-0.805283581699165	1.06613226452906\\
-0.802028601208808	1.08216432865731\\
-0.79848934370967	1.09819639278557\\
-0.794653812367824	1.11422845691383\\
-0.793587174348697	1.11840086240547\\
-0.79042052310809	1.13026052104208\\
-0.785819903998319	1.14629258517034\\
-0.780866613022514	1.1623246492986\\
-0.777555110220441	1.17236205496192\\
-0.775478014964858	1.17835671342685\\
-0.769575432731745	1.19438877755511\\
-0.763245346289689	1.21042084168337\\
-0.761523046092184	1.21454979020275\\
-0.75627569446411	1.22645290581162\\
-0.748740773012284	1.24248496993988\\
-0.745490981963928	1.24902269292138\\
-0.7404736102047	1.25851703406814\\
-0.731474334627966	1.27454909819639\\
-0.729458917835671	1.27796128310748\\
-0.721480456734734	1.29058116232465\\
-0.713426853707415	1.30259251812186\\
-0.710523412722237	1.30661322645291\\
-0.698314123476152	1.32264529058116\\
-0.697394789579158	1.32379662789258\\
-0.684481340561341	1.33867735470942\\
-0.681362725450902	1.34209568133149\\
-0.66874898794471	1.35470941883768\\
-0.665330661322646	1.35797099976621\\
-0.650511183447168	1.37074148296593\\
-0.649298597194389	1.37174138065302\\
-0.633266533066132	1.38361953634077\\
-0.62845300893667	1.38677354709419\\
-0.617234468937876	1.39383666296571\\
-0.601202404809619	1.4025892937035\\
-0.600745994797927	1.40280561122244\\
-0.585170340681363	1.40992573720582\\
-0.569138276553106	1.4160452009309\\
-0.560232538149745	1.4188376753507\\
-0.55310621242485	1.42100045734942\\
-0.537074148296593	1.42486816501932\\
-0.521042084168337	1.42774204570758\\
-0.50501002004008	1.42967718233409\\
-0.488977955911824	1.43072393472941\\
-0.472945891783567	1.43092834074326\\
-0.456913827655311	1.43033247579527\\
-0.440881763527054	1.42897477560459\\
-0.424849699398798	1.42689032622384\\
-0.408817635270541	1.42411112497851\\
-0.392785571142285	1.42066631546198\\
-0.38565924541739	1.4188376753507\\
-0.376753507014028	1.41656648519904\\
-0.360721442885771	1.41183688755087\\
-0.344689378757515	1.40651358659104\\
-0.334651973094187	1.40280561122244\\
-0.328657314629258	1.40060301102082\\
-0.312625250501002	1.39411330148802\\
-0.296593186372745	1.38708617343689\\
-0.295935787071416	1.38677354709419\\
-0.280561122244489	1.37949587086625\\
-0.264529058116232	1.37139710129056\\
-0.263316471863453	1.37074148296593\\
-0.248496993987976	1.36276180547724\\
-0.234356124466997	1.35470941883768\\
-0.232464929859719	1.35363663963936\\
-0.216432865731463	1.3439988072933\\
-0.208016305591172	1.33867735470942\\
-0.200400801603207	1.33387843940809\\
-0.18436873747495	1.32328265171227\\
-0.183449403577956	1.32264529058116\\
-0.168336673346694	1.31219743040739\\
-0.160593790422456	1.30661322645291\\
-0.152304609218437	1.30065037543827\\
-0.138863158302592	1.29058116232465\\
-0.13627254509018	1.28864497519273\\
-0.120240480961924	1.27617679261093\\
-0.118225064169629	1.27454909819639\\
-0.104208416833667	1.26324929788337\\
-0.0985418869905856	1.25851703406814\\
-0.0881763527054109	1.24987385108974\\
-0.0796107483583451	1.24248496993988\\
-0.0721442885771544	1.23605250728087\\
-0.0613595760769719	1.22645290581162\\
-0.0561122244488979	1.22178697497293\\
-0.043725344827961	1.21042084168337\\
-0.0400801603206413	1.20707861834823\\
-0.0266527533644603	1.19438877755511\\
-0.0240480961923848	1.19192845901123\\
-0.0100931273197109	1.17835671342685\\
-0.00801603206412826	1.17633717730267\\
0.00599649593994431	1.1623246492986\\
0.00801603206412782	1.16030511317441\\
0.0216541346291227	1.14629258517034\\
0.0240480961923843	1.14383226662646\\
0.0369135090800333	1.13026052104208\\
0.0400801603206409	1.12691829770695\\
0.0518045890419841	1.11422845691383\\
0.0561122244488974	1.10956252607513\\
0.0663540590020255	1.09819639278557\\
0.0721442885771539	1.09176393012657\\
0.0805857154372647	1.08216432865731\\
0.0881763527054105	1.07352114567892\\
0.0945208072747595	1.06613226452906\\
0.104208416833667	1.05483246421604\\
0.108178328748497	1.0501002004008\\
0.120240480961924	1.03569583068708\\
0.121575272180958	1.03406813627254\\
0.134692032149139	1.01803607214429\\
0.13627254509018	1.01609988501237\\
0.147546506819147	1.00200400801603\\
0.152304609218437	0.9960411570014\\
0.160183002079825	0.985971943887775\\
0.168336673346693	0.975524083714005\\
0.172612977213375	0.969939879759519\\
0.18436873747495	0.954545176762373\\
0.184846659229807	0.953907815631262\\
0.196826185090445	0.937875751503006\\
0.200400801603206	0.933076836201674\\
0.208621100028441	0.921843687374749\\
0.216432865731463	0.91113307583037\\
0.220248467550492	0.905811623246493\\
0.231701607899197	0.889779559118236\\
0.232464929859719	0.888706779919921\\
0.242934002994659	0.87374749498998\\
0.248496993987976	0.865767817501285\\
0.254022025382232	0.857715430861724\\
0.264529058116232	0.842338985058091\\
0.264970265322122	0.841683366733467\\
0.275708137982974	0.825651302605211\\
0.280561122244489	0.818373626377271\\
0.286312909771483	0.809619238476954\\
0.296593186372745	0.793899800691397\\
0.296794755951398	0.793587174348698\\
0.307077408297698	0.777555110220441\\
0.312625250501002	0.768862800486012\\
0.317245772034545	0.761523046092185\\
0.327285519079521	0.745490981963928\\
0.328657314629258	0.743288381762303\\
0.337156300817286	0.729458917835672\\
0.344689378757515	0.717134829076012\\
0.346927181285696	0.713426853707415\\
0.356545602273577	0.697394789579159\\
0.360721442885771	0.690394001779324\\
0.366042486545644	0.681362725450902\\
0.375431709792827	0.665330661322646\\
0.376753507014028	0.663059471170983\\
0.384667400047787	0.649298597194389\\
0.392785571142285	0.635095173177409\\
0.393818986979759	0.633266533066132\\
0.402817842545186	0.617234468937876\\
0.408817635270541	0.60647585443743\\
0.411726576130805	0.601202404809619\\
0.42050868119499	0.585170340681363\\
0.424849699398798	0.577190927426249\\
0.429184778407611	0.569138276553106\\
0.437753779839534	0.55310621242485\\
0.440881763527054	0.547211248550484\\
0.446206654881929	0.537074148296593\\
0.454566045756848	0.521042084168337\\
0.456913827655311	0.516504820484648\\
0.462804351611308	0.50501002004008\\
0.470957473100776	0.488977955911824\\
0.472945891783567	0.48503655717613\\
0.478989140028764	0.472945891783567\\
0.486939183233613	0.456913827655311\\
0.488977955911824	0.452768022905766\\
0.494771454425069	0.440881763527054\\
0.502521462138104	0.424849699398798\\
0.50501002004008	0.419657142253924\\
0.510160926694367	0.408817635270541\\
0.517713795080866	0.392785571142285\\
0.521042084168337	0.385657877370906\\
0.525166418500278	0.376753507014028\\
0.532524898685601	0.360721442885771\\
0.537074148296593	0.350719868426134\\
0.539796051007148	0.344689378757515\\
0.546962750561728	0.328657314629258\\
0.55310621242485	0.314788032499723\\
0.554057232309417	0.312625250501002\\
0.561034616622361	0.296593186372745\\
0.567946882818409	0.280561122244489\\
0.569138276553106	0.277768647824683\\
0.574747076214539	0.264529058116232\\
0.581471212238471	0.248496993987976\\
0.585170340681363	0.23958505584309\\
0.588106045174488	0.232464929859719\\
0.594644196129463	0.216432865731463\\
0.601116114696609	0.200400801603206\\
0.601202404809619	0.200184484084263\\
0.607470969037548	0.18436873747495\\
0.613757593813339	0.168336673346693\\
0.617234468937876	0.159367725089954\\
0.619956043021835	0.152304609218437\\
0.626059204199459	0.13627254509018\\
0.632094429226009	0.120240480961924\\
0.633266533066132	0.117086470208509\\
0.63802471474842	0.104208416833667\\
0.643876682886971	0.0881763527054105\\
0.649298597194389	0.0731441862642451\\
0.649657305025131	0.0721442885771539\\
0.655327533694953	0.0561122244488974\\
0.660927542575827	0.0400801603206409\\
0.665330661322646	0.0273096771209221\\
0.666449447450091	0.0240480961923843\\
0.671867442661018	0.00801603206412782\\
0.677214141975015	-0.00801603206412826\\
0.681362725450902	-0.0206297695703205\\
0.682481511578348	-0.0240480961923848\\
0.687645617002839	-0.0400801603206413\\
0.692737070130544	-0.0561122244488979\\
0.697394789579159	-0.0709929512657397\\
0.6977534974099	-0.0721442885771544\\
0.702661469168185	-0.0881763527054109\\
0.707495148697988	-0.104208416833667\\
0.712254749867292	-0.120240480961924\\
0.713426853707415	-0.124261189292971\\
0.7169126464296	-0.13627254509018\\
0.721485427396051	-0.152304609218437\\
0.725982042711135	-0.168336673346694\\
0.729458917835672	-0.180956552563859\\
0.730395021803703	-0.18436873747495\\
0.734703172616581	-0.200400801603207\\
0.738932773283772	-0.216432865731463\\
0.743083645568415	-0.232464929859719\\
0.745490981963928	-0.241959271006479\\
0.747141849129499	-0.248496993987976\\
0.751099781625361	-0.264529058116232\\
0.754976124731029	-0.280561122244489\\
0.758770453206528	-0.296593186372745\\
0.761523046092185	-0.308496301981619\\
0.762474065976751	-0.312625250501002\\
0.766070269536165	-0.328657314629258\\
0.769581097463983	-0.344689378757515\\
0.773005860609449	-0.360721442885771\\
0.776343786249278	-0.376753507014028\\
0.777555110220441	-0.382748165478959\\
0.779575252649355	-0.392785571142285\\
0.782706016874727	-0.408817635270541\\
0.785745855233877	-0.424849699398798\\
0.788693703292883	-0.440881763527054\\
0.791548401670487	-0.456913827655311\\
0.793587174348698	-0.468773486291924\\
0.794301571550162	-0.472945891783567\\
0.796938650811452	-0.488977955911824\\
0.799477698304696	-0.50501002004008\\
0.801917229327064	-0.521042084168337\\
0.804255650319159	-0.537074148296593\\
0.806491254789434	-0.55310621242485\\
0.808622218986226	-0.569138276553106\\
0.809619238476954	-0.577062115214904\\
0.810635043483407	-0.585170340681363\\
0.812528179337217	-0.601202404809619\\
0.814310405770573	-0.617234468937876\\
0.815979507096244	-0.633266533066132\\
0.817533131510713	-0.649298597194389\\
0.818968785424053	-0.665330661322646\\
0.820283827444078	-0.681362725450902\\
0.821475461993017	-0.697394789579158\\
0.822540732533354	-0.713426853707415\\
0.82347651437768	-0.729458917835671\\
0.824279507055472	-0.745490981963928\\
0.824946226207663	-0.761523046092184\\
0.825472994977549	-0.777555110220441\\
0.825651302605211	-0.785057435918098\\
0.825852872183863	-0.793587174348697\\
0.826084201198702	-0.809619238476954\\
0.826165509867008	-0.82565130260521\\
0.826092509811099	-0.841683366733467\\
0.825860684610487	-0.857715430861723\\
0.825651302605211	-0.866245169292336\\
0.825462060494201	-0.87374749498998\\
0.824887980644688	-0.889779559118236\\
0.824136376831622	-0.905811623246493\\
0.823201350070428	-0.92184368737475\\
0.822076686092449	-0.937875751503006\\
0.820755838230285	-0.953907815631263\\
0.819231909149522	-0.969939879759519\\
0.817497631338341	-0.985971943887776\\
0.815545346258589	-1.00200400801603\\
0.813366982053292	-1.01803607214429\\
0.810954029695989	-1.03406813627255\\
0.809619238476954	-1.042176361739\\
0.80826563518729	-1.0501002004008\\
0.805283581699165	-1.06613226452906\\
0.802028601208808	-1.08216432865731\\
0.79848934370967	-1.09819639278557\\
0.794653812367823	-1.11422845691383\\
0.793587174348698	-1.11840086240547\\
0.790420523108089	-1.13026052104208\\
0.785819903998319	-1.14629258517034\\
0.780866613022514	-1.1623246492986\\
0.777555110220441	-1.17236205496192\\
0.775478014964857	-1.17835671342685\\
0.769575432731745	-1.19438877755511\\
0.763245346289689	-1.21042084168337\\
0.761523046092185	-1.21454979020275\\
0.75627569446411	-1.22645290581162\\
0.748740773012284	-1.24248496993988\\
0.745490981963928	-1.24902269292138\\
0.7404736102047	-1.25851703406814\\
0.731474334627967	-1.27454909819639\\
0.729458917835672	-1.27796128310748\\
0.721480456734734	-1.29058116232465\\
0.713426853707415	-1.30259251812186\\
0.710523412722236	-1.30661322645291\\
0.698314123476151	-1.32264529058116\\
0.697394789579159	-1.32379662789258\\
0.68448134056134	-1.33867735470942\\
0.681362725450902	-1.34209568133148\\
0.66874898794471	-1.35470941883768\\
0.665330661322646	-1.35797099976621\\
0.650511183447168	-1.37074148296593\\
0.649298597194389	-1.37174138065302\\
0.633266533066132	-1.38361953634077\\
0.62845300893667	-1.38677354709419\\
0.617234468937876	-1.39383666296571\\
0.601202404809619	-1.4025892937035\\
0.600745994797927	-1.40280561122244\\
0.585170340681363	-1.40992573720582\\
0.569138276553106	-1.4160452009309\\
0.560232538149745	-1.4188376753507\\
0.55310621242485	-1.42100045734942\\
0.537074148296593	-1.42486816501932\\
0.521042084168337	-1.42774204570758\\
0.50501002004008	-1.42967718233408\\
0.488977955911824	-1.43072393472941\\
0.472945891783567	-1.43092834074326\\
0.456913827655311	-1.43033247579527\\
0.440881763527054	-1.42897477560459\\
0.424849699398798	-1.42689032622384\\
0.408817635270541	-1.42411112497851\\
0.392785571142285	-1.42066631546198\\
0.385659245417389	-1.4188376753507\\
0.376753507014028	-1.41656648519904\\
0.360721442885771	-1.41183688755087\\
0.344689378757515	-1.40651358659104\\
0.334651973094188	-1.40280561122244\\
0.328657314629258	-1.40060301102082\\
0.312625250501002	-1.39411330148802\\
0.296593186372745	-1.38708617343689\\
0.295935787071416	-1.38677354709419\\
0.280561122244489	-1.37949587086625\\
0.264529058116232	-1.37139710129056\\
0.263316471863453	-1.37074148296593\\
0.248496993987976	-1.36276180547724\\
0.234356124466996	-1.35470941883768\\
0.232464929859719	-1.35363663963936\\
0.216432865731463	-1.3439988072933\\
0.208016305591172	-1.33867735470942\\
0.200400801603206	-1.33387843940809\\
0.18436873747495	-1.32328265171227\\
0.183449403577957	-1.32264529058116\\
0.168336673346693	-1.31219743040739\\
0.160593790422456	-1.30661322645291\\
0.152304609218437	-1.30065037543827\\
0.138863158302593	-1.29058116232465\\
0.13627254509018	-1.28864497519273\\
0.120240480961924	-1.27617679261093\\
0.118225064169628	-1.27454909819639\\
0.104208416833667	-1.26324929788337\\
0.098541886990586	-1.25851703406814\\
0.0881763527054105	-1.24987385108974\\
0.0796107483583451	-1.24248496993988\\
0.0721442885771539	-1.23605250728087\\
0.0613595760769714	-1.22645290581162\\
0.0561122244488974	-1.22178697497293\\
0.043725344827961	-1.21042084168337\\
0.0400801603206409	-1.20707861834823\\
0.0266527533644607	-1.19438877755511\\
0.0240480961923843	-1.19192845901123\\
0.0100931273197114	-1.17835671342685\\
0.00801603206412782	-1.17633717730267\\
-0.00599649593994475	-1.1623246492986\\
-0.00801603206412826	-1.16030511317441\\
-0.0216541346291227	-1.14629258517034\\
-0.0240480961923848	-1.14383226662646\\
-0.0369135090800337	-1.13026052104208\\
-0.0400801603206413	-1.12691829770695\\
-0.0518045890419846	-1.11422845691383\\
-0.0561122244488979	-1.10956252607513\\
-0.0663540590020255	-1.09819639278557\\
-0.0721442885771544	-1.09176393012657\\
-0.0805857154372647	-1.08216432865731\\
-0.0881763527054109	-1.07352114567892\\
-0.0945208072747585	-1.06613226452906\\
-0.104208416833667	-1.05483246421604\\
-0.108178328748496	-1.0501002004008\\
-0.120240480961924	-1.03569583068708\\
-0.121575272180958	-1.03406813627255\\
-0.134692032149139	-1.01803607214429\\
-0.13627254509018	-1.01609988501237\\
-0.147546506819146	-1.00200400801603\\
-0.152304609218437	-0.996041157001399\\
-0.160183002079824	-0.985971943887776\\
-0.168336673346694	-0.975524083714005\\
-0.172612977213376	-0.969939879759519\\
-0.18436873747495	-0.954545176762373\\
-0.184846659229807	-0.953907815631263\\
-0.196826185090445	-0.937875751503006\\
-0.200400801603207	-0.933076836201674\\
-0.208621100028441	-0.92184368737475\\
-0.216432865731463	-0.91113307583037\\
-0.220248467550491	-0.905811623246493\\
-0.231701607899196	-0.889779559118236\\
-0.232464929859719	-0.888706779919921\\
-0.242934002994659	-0.87374749498998\\
-0.248496993987976	-0.865767817501284\\
-0.254022025382232	-0.857715430861723\\
-0.264529058116232	-0.842338985058091\\
-0.264970265322122	-0.841683366733467\\
-0.275708137982974	-0.82565130260521\\
-0.280561122244489	-0.818373626377271\\
-0.286312909771483	-0.809619238476954\\
-0.296593186372745	-0.793899800691397\\
-0.296794755951398	-0.793587174348697\\
-0.307077408297698	-0.777555110220441\\
-0.312625250501002	-0.768862800486012\\
-0.317245772034545	-0.761523046092184\\
-0.327285519079521	-0.745490981963928\\
-0.328657314629258	-0.743288381762303\\
-0.337156300817286	-0.729458917835671\\
-0.344689378757515	-0.717134829076012\\
-0.346927181285696	-0.713426853707415\\
-0.356545602273577	-0.697394789579158\\
-0.360721442885771	-0.690394001779324\\
-0.366042486545644	-0.681362725450902\\
-0.375431709792827	-0.665330661322646\\
-0.376753507014028	-0.663059471170983\\
-0.384667400047787	-0.649298597194389\\
-0.392785571142285	-0.635095173177409\\
-0.393818986979759	-0.633266533066132\\
-0.402817842545187	-0.617234468937876\\
-0.408817635270541	-0.60647585443743\\
-0.411726576130805	-0.601202404809619\\
-0.42050868119499	-0.585170340681363\\
-0.424849699398798	-0.577190927426249\\
-0.429184778407611	-0.569138276553106\\
-0.437753779839535	-0.55310621242485\\
-0.440881763527054	-0.547211248550484\\
-0.446206654881929	-0.537074148296593\\
-0.454566045756848	-0.521042084168337\\
-0.456913827655311	-0.516504820484648\\
-0.462804351611308	-0.50501002004008\\
-0.470957473100776	-0.488977955911824\\
-0.472945891783567	-0.485036557176129\\
-0.478989140028764	-0.472945891783567\\
-0.486939183233613	-0.456913827655311\\
-0.488977955911824	-0.452768022905766\\
-0.494771454425069	-0.440881763527054\\
-0.502521462138104	-0.424849699398798\\
-0.50501002004008	-0.419657142253924\\
-0.510160926694366	-0.408817635270541\\
-0.517713795080866	-0.392785571142285\\
-0.521042084168337	-0.385657877370906\\
-0.525166418500279	-0.376753507014028\\
-0.532524898685601	-0.360721442885771\\
-0.537074148296593	-0.350719868426134\\
-0.539796051007148	-0.344689378757515\\
-0.546962750561728	-0.328657314629258\\
-0.55310621242485	-0.314788032499723\\
-0.554057232309416	-0.312625250501002\\
-0.561034616622361	-0.296593186372745\\
-0.56794688281841	-0.280561122244489\\
-0.569138276553106	-0.277768647824683\\
-0.574747076214539	-0.264529058116232\\
-0.581471212238471	-0.248496993987976\\
-0.585170340681363	-0.23958505584309\\
-0.588106045174488	-0.232464929859719\\
-0.594644196129463	-0.216432865731463\\
-0.601116114696608	-0.200400801603207\\
-0.601202404809619	-0.200184484084262\\
-0.607470969037547	-0.18436873747495\\
-0.613757593813339	-0.168336673346694\\
-0.617234468937876	-0.159367725089954\\
-0.619956043021835	-0.152304609218437\\
-0.626059204199459	-0.13627254509018\\
-0.632094429226009	-0.120240480961924\\
-0.633266533066132	-0.117086470208509\\
-0.638024714748419	-0.104208416833667\\
-0.643876682886971	-0.0881763527054109\\
-0.649298597194389	-0.0731441862642456\\
-0.649657305025131	-0.0721442885771544\\
-0.655327533694952	-0.0561122244488979\\
-0.660927542575827	-0.0400801603206413\\
-0.665330661322646	-0.0273096771209217\\
-0.666449447450091	-0.0240480961923848\\
-0.671867442661018	-0.00801603206412826\\
-0.677214141975015	0.00801603206412782\\
-0.681362725450902	0.0206297695703186\\
-0.682481511578348	0.0240480961923843\\
-0.687645617002839	0.0400801603206409\\
-0.692737070130543	0.0561122244488974\\
-0.697394789579158	0.0709929512657383\\
-0.6977534974099	0.0721442885771539\\
-0.702661469168184	0.0881763527054105\\
-0.707495148697989	0.104208416833667\\
-0.712254749867292	0.120240480961924\\
-0.713426853707415	0.124261189292968\\
-0.7169126464296	0.13627254509018\\
-0.72148542739605	0.152304609218437\\
-0.725982042711135	0.168336673346693\\
-0.729458917835671	0.180956552563858\\
-0.730395021803702	0.18436873747495\\
-0.734703172616582	0.200400801603206\\
-0.738932773283771	0.216432865731463\\
-0.743083645568415	0.232464929859719\\
-0.745490981963928	0.241959271006478\\
-0.747141849129499	0.248496993987976\\
-0.75109978162536	0.264529058116232\\
-0.754976124731029	0.280561122244489\\
-0.758770453206528	0.296593186372745\\
-0.761523046092184	0.308496301981619\\
-0.762474065976751	0.312625250501002\\
-0.766070269536166	0.328657314629258\\
-0.769581097463983	0.344689378757515\\
-0.773005860609449	0.360721442885771\\
-0.776343786249279	0.376753507014028\\
-0.777555110220441	0.382748165478957\\
-0.779575252649355	0.392785571142285\\
-0.782706016874727	0.408817635270541\\
-0.785745855233877	0.424849699398798\\
-0.788693703292883	0.440881763527054\\
-0.791548401670487	0.456913827655311\\
-0.793587174348697	0.468773486291922\\
-0.794301571550162	0.472945891783567\\
-0.796938650811452	0.488977955911824\\
-0.799477698304695	0.50501002004008\\
-0.801917229327064	0.521042084168337\\
-0.804255650319159	0.537074148296593\\
-0.806491254789434	0.55310621242485\\
-0.808622218986225	0.569138276553106\\
-0.809619238476954	0.577062115214904\\
-0.810635043483407	0.585170340681363\\
-0.812528179337218	0.601202404809619\\
-0.814310405770573	0.617234468937876\\
-0.815979507096244	0.633266533066132\\
-0.817533131510713	0.649298597194389\\
-0.818968785424053	0.665330661322646\\
-0.820283827444077	0.681362725450902\\
-0.821475461993016	0.697394789579159\\
-0.822540732533354	0.713426853707415\\
-0.82347651437768	0.729458917835672\\
-0.824279507055473	0.745490981963928\\
-0.824946226207662	0.761523046092185\\
-0.825472994977549	0.777555110220441\\
-0.82565130260521	0.785057435918085\\
-0.825852872183862	0.793587174348698\\
-0.826084201198703	0.809619238476954\\
-0.826165509867008	0.825651302605211\\
-0.8260925098111	0.841683366733467\\
-0.825860684610486	0.857715430861724\\
-0.82565130260521	0.866245169292337\\
}--cycle;


\addplot[area legend,solid,fill=mycolor10,draw=black,forget plot]
table[row sep=crcr] {%
x	y\\
-0.665330661322646	0.758105025403793\\
-0.665082784862698	0.761523046092185\\
-0.663686564675468	0.777555110220441\\
-0.662033485917416	0.793587174348698\\
-0.660112238187751	0.809619238476954\\
-0.657910790516671	0.825651302605211\\
-0.655416339751623	0.841683366733467\\
-0.652615254428646	0.857715430861724\\
-0.649493013666892	0.87374749498998\\
-0.649298597194389	0.87466276666894\\
-0.645911158536872	0.889779559118236\\
-0.641949453571151	0.905811623246493\\
-0.637595573522108	0.921843687374749\\
-0.633266533066132	0.936427568599341\\
-0.632809737264636	0.937875751503006\\
-0.62737690649475	0.953907815631262\\
-0.621452974746538	0.969939879759519\\
-0.617234468937876	0.980509918222073\\
-0.614898112565123	0.985971943887775\\
-0.607542947700778	1.00200400801603\\
-0.601202404809619	1.0147880599505\\
-0.599461200225299	1.01803607214429\\
-0.590292763112442	1.03406813627254\\
-0.585170340681363	1.04241611958894\\
-0.58002212340773	1.0501002004008\\
-0.569138276553106	1.06525453592202\\
-0.568443290727499	1.06613226452906\\
-0.55496007841122	1.08216432865731\\
-0.55310621242485	1.08423901886186\\
-0.539148838501137	1.09819639278557\\
-0.537074148296593	1.10015445046976\\
-0.521042084168337	1.11349529538232\\
-0.520031608379621	1.11422845691383\\
-0.50501002004008	1.1245759618867\\
-0.495273095705868	1.13026052104208\\
-0.488977955911824	1.13376186558903\\
-0.472945891783567	1.14121295074715\\
-0.459322825664768	1.14629258517034\\
-0.456913827655311	1.1471525085698\\
-0.440881763527054	1.15164454166159\\
-0.424849699398798	1.1548759669644\\
-0.408817635270541	1.15693170935783\\
-0.392785571142285	1.15788836523612\\
-0.376753507014028	1.15781506554153\\
-0.360721442885771	1.15677423201626\\
-0.344689378757515	1.15482224148008\\
-0.328657314629258	1.15201001063494\\
-0.312625250501002	1.14838351198749\\
-0.305073103182926	1.14629258517034\\
-0.296593186372745	1.14395622879759\\
-0.280561122244489	1.13876287781614\\
-0.264529058116232	1.13286477697221\\
-0.258233918322189	1.13026052104208\\
-0.248496993987976	1.12624895384855\\
-0.232464929859719	1.11895667271266\\
-0.222939668618048	1.11422845691383\\
-0.216432865731463	1.11101016607699\\
-0.200400801603207	1.10241489859423\\
-0.193078381798149	1.09819639278557\\
-0.18436873747495	1.09319414797708\\
-0.168336673346694	1.0833749248919\\
-0.166482807360323	1.08216432865731\\
-0.152304609218437	1.07292956533563\\
-0.142421230303577	1.06613226452906\\
-0.13627254509018	1.06191327279785\\
-0.120240480961924	1.05032770442359\\
-0.119941671271634	1.0501002004008\\
-0.104208416833667	1.03814294264968\\
-0.0990859944025878	1.03406813627254\\
-0.0881763527054109	1.02540306136772\\
-0.0792960980873046	1.01803607214429\\
-0.0721442885771544	1.0121106072761\\
-0.0604426536847802	1.00200400801603\\
-0.0561122244488979	0.998267697634059\\
-0.0424165166933938	0.985971943887775\\
-0.0400801603206413	0.983876022989052\\
-0.0251250206077705	0.969939879759519\\
-0.0240480961923848	0.968936849352776\\
-0.00848905848976323	0.953907815631262\\
-0.00801603206412826	0.953451019829765\\
0.007559236262631	0.937875751503006\\
0.00801603206412782	0.937418955701508\\
0.023078348608166	0.921843687374749\\
0.0240480961923843	0.920840656968006\\
0.038118883178968	0.905811623246493\\
0.0400801603206409	0.903715702347771\\
0.052724785791382	0.889779559118236\\
0.0561122244488974	0.886043248736265\\
0.0669343440570812	0.87374749498998\\
0.0721442885771539	0.867822030121793\\
0.0807810122405978	0.857715430861724\\
0.0881763527054105	0.849050355956901\\
0.0942940952626445	0.841683366733467\\
0.104208416833667	0.829726108982349\\
0.107499319045926	0.825651302605211\\
0.120240480961924	0.809846742499742\\
0.12041930855841	0.809619238476954\\
0.132975369684951	0.793587174348698\\
0.13627254509018	0.789368182617485\\
0.145279652829894	0.777555110220441\\
0.152304609218437	0.768320346898754\\
0.157354243807146	0.761523046092185\\
0.168336673346693	0.746701578198513\\
0.169213498649643	0.745490981963928\\
0.180773489041761	0.729458917835672\\
0.18436873747495	0.724456673027176\\
0.192122093660993	0.713426853707415\\
0.200400801603206	0.70161329538782\\
0.203294662397068	0.697394789579159\\
0.214244932658426	0.681362725450902\\
0.216432865731463	0.67814443461406\\
0.224965119394612	0.665330661322646\\
0.232464929859719	0.654026812993221\\
0.235540035818519	0.649298597194389\\
0.245914642067158	0.633266533066132\\
0.248496993987976	0.629254965872603\\
0.256086167319485	0.617234468937876\\
0.264529058116232	0.603806660739741\\
0.266136311158703	0.601202404809619\\
0.275967306441132	0.585170340681363\\
0.280561122244489	0.577640633327164\\
0.285655289054162	0.569138276553106\\
0.295210419473976	0.55310621242485\\
0.296593186372745	0.550769856052097\\
0.30455760395013	0.537074148296593\\
0.312625250501002	0.523133010985484\\
0.313815042172045	0.521042084168337\\
0.322866732980162	0.50501002004008\\
0.328657314629258	0.494695381376427\\
0.331814981981456	0.488977955911824\\
0.340604525579819	0.472945891783567\\
0.344689378757515	0.465443483965051\\
0.349260440368795	0.456913827655311\\
0.357791295522285	0.440881763527054\\
0.360721442885771	0.435331346244714\\
0.366170395731711	0.424849699398798\\
0.374445905957461	0.408817635270541\\
0.376753507014028	0.404308051513471\\
0.382562459543581	0.392785571142285\\
0.39058584797749	0.376753507014028\\
0.392785571142285	0.372317222951554\\
0.398452948292133	0.360721442885771\\
0.406227313185292	0.344689378757515\\
0.408817635270541	0.339296438816749\\
0.413856949942716	0.328657314629258\\
0.421385260700624	0.312625250501002\\
0.424849699398798	0.305176568166802\\
0.428788385313387	0.296593186372745\\
0.436073478999931	0.280561122244489\\
0.440881763527054	0.269881014607484\\
0.443260064715237	0.264529058116232\\
0.450304642951515	0.248496993987976\\
0.456913827655311	0.233324853259175\\
0.457283740180685	0.232464929859719\\
0.464090366375889	0.216432865731463\\
0.470835143423245	0.200400801603206\\
0.472945891783567	0.195321167180018\\
0.477441250432316	0.18436873747495\\
0.483947723367293	0.168336673346693\\
0.488977955911824	0.155805953765381\\
0.490366928106163	0.152304609218437\\
0.496637367872314	0.13627254509018\\
0.502841024354953	0.120240480961924\\
0.50501002004008	0.114555921806544\\
0.508912636015967	0.104208416833667\\
0.514879435468517	0.0881763527054105\\
0.520776905952092	0.0721442885771539\\
0.521042084168337	0.0714111270456462\\
0.526513104852846	0.0561122244488974\\
0.532172574628555	0.0400801603206409\\
0.537074148296593	0.0260061538765745\\
0.537748689204562	0.0240480961923843\\
0.54317196671062	0.00801603206412782\\
0.5485197049853	-0.00801603206412826\\
0.55310621242485	-0.0219734059878415\\
0.553780753332819	-0.0240480961923848\\
0.558890672250114	-0.0400801603206413\\
0.56392155566562	-0.0561122244488979\\
0.568873098336861	-0.0721442885771544\\
0.569138276553106	-0.0730220171841901\\
0.573668496296285	-0.0881763527054109\\
0.57837666810335	-0.104208416833667\\
0.583001344996236	-0.120240480961924\\
0.585170340681363	-0.127924561773787\\
0.587502535040489	-0.13627254509018\\
0.591881390411667	-0.152304609218437\\
0.596172172265089	-0.168336673346694\\
0.60037395363977	-0.18436873747495\\
0.601202404809619	-0.187616749668742\\
0.60442986328447	-0.200400801603207\\
0.608378943530198	-0.216432865731463\\
0.612233727171058	-0.232464929859719\\
0.615992884122811	-0.248496993987976\\
0.617234468937876	-0.25395901965368\\
0.619612770126059	-0.264529058116232\\
0.623111489402566	-0.280561122244489\\
0.626508521446038	-0.296593186372745\\
0.629802094367958	-0.312625250501002\\
0.632990291864623	-0.328657314629258\\
0.633266533066132	-0.330105497532926\\
0.6360197366617	-0.344689378757515\\
0.638933910215981	-0.360721442885771\\
0.641735488262777	-0.376753507014028\\
0.644422048509105	-0.392785571142285\\
0.646990996137822	-0.408817635270541\\
0.649298597194389	-0.423934427719837\\
0.649436850751654	-0.424849699398798\\
0.651716318887212	-0.440881763527054\\
0.653869658805669	-0.456913827655311\\
0.655893692198952	-0.472945891783567\\
0.657785033616345	-0.488977955911824\\
0.659540079673549	-0.50501002004008\\
0.661154997500123	-0.521042084168337\\
0.662625712366185	-0.537074148296593\\
0.663947894423875	-0.55310621242485\\
0.665116944493114	-0.569138276553106\\
0.665330661322646	-0.572556297241498\\
0.666110017040159	-0.585170340681363\\
0.666937914365116	-0.601202404809619\\
0.667600197623856	-0.617234468937876\\
0.668091410176274	-0.633266533066132\\
0.668405767281445	-0.649298597194389\\
0.668537136825009	-0.665330661322646\\
0.66847901861384	-0.681362725450902\\
0.668224522116507	-0.697394789579158\\
0.6677663425161	-0.713426853707415\\
0.66709673492865	-0.729458917835671\\
0.666207486625594	-0.745490981963928\\
0.665330661322646	-0.758105025403793\\
0.665082784862698	-0.761523046092184\\
0.663686564675469	-0.777555110220441\\
0.662033485917416	-0.793587174348697\\
0.660112238187751	-0.809619238476954\\
0.657910790516671	-0.82565130260521\\
0.655416339751623	-0.841683366733467\\
0.652615254428647	-0.857715430861723\\
0.649493013666891	-0.87374749498998\\
0.649298597194389	-0.87466276666894\\
0.645911158536872	-0.889779559118236\\
0.641949453571151	-0.905811623246493\\
0.637595573522108	-0.92184368737475\\
0.633266533066132	-0.936427568599342\\
0.632809737264635	-0.937875751503006\\
0.627376906494749	-0.953907815631263\\
0.621452974746537	-0.969939879759519\\
0.617234468937876	-0.980509918222073\\
0.614898112565123	-0.985971943887776\\
0.607542947700777	-1.00200400801603\\
0.601202404809619	-1.0147880599505\\
0.599461200225299	-1.01803607214429\\
0.590292763112442	-1.03406813627255\\
0.585170340681363	-1.04241611958894\\
0.580022123407729	-1.0501002004008\\
0.569138276553106	-1.06525453592202\\
0.5684432907275	-1.06613226452906\\
0.55496007841122	-1.08216432865731\\
0.55310621242485	-1.08423901886186\\
0.539148838501136	-1.09819639278557\\
0.537074148296593	-1.10015445046976\\
0.521042084168337	-1.11349529538232\\
0.520031608379621	-1.11422845691383\\
0.50501002004008	-1.12457596188671\\
0.495273095705867	-1.13026052104208\\
0.488977955911824	-1.13376186558903\\
0.472945891783567	-1.14121295074715\\
0.459322825664769	-1.14629258517034\\
0.456913827655311	-1.1471525085698\\
0.440881763527054	-1.15164454166159\\
0.424849699398798	-1.1548759669644\\
0.408817635270541	-1.15693170935783\\
0.392785571142285	-1.15788836523612\\
0.376753507014028	-1.15781506554153\\
0.360721442885771	-1.15677423201626\\
0.344689378757515	-1.15482224148008\\
0.328657314629258	-1.15201001063494\\
0.312625250501002	-1.14838351198749\\
0.305073103182926	-1.14629258517034\\
0.296593186372745	-1.14395622879759\\
0.280561122244489	-1.13876287781614\\
0.264529058116232	-1.13286477697221\\
0.258233918322189	-1.13026052104208\\
0.248496993987976	-1.12624895384855\\
0.232464929859719	-1.11895667271266\\
0.222939668618048	-1.11422845691383\\
0.216432865731463	-1.11101016607699\\
0.200400801603206	-1.10241489859423\\
0.19307838179815	-1.09819639278557\\
0.18436873747495	-1.09319414797708\\
0.168336673346693	-1.0833749248919\\
0.166482807360324	-1.08216432865731\\
0.152304609218437	-1.07292956533563\\
0.142421230303577	-1.06613226452906\\
0.13627254509018	-1.06191327279785\\
0.120240480961924	-1.05032770442359\\
0.119941671271634	-1.0501002004008\\
0.104208416833667	-1.03814294264968\\
0.0990859944025882	-1.03406813627255\\
0.0881763527054105	-1.02540306136772\\
0.079296098087305	-1.01803607214429\\
0.0721442885771539	-1.0121106072761\\
0.0604426536847797	-1.00200400801603\\
0.0561122244488974	-0.99826769763406\\
0.0424165166933943	-0.985971943887776\\
0.0400801603206409	-0.983876022989052\\
0.0251250206077718	-0.969939879759519\\
0.0240480961923843	-0.968936849352774\\
0.00848905848976415	-0.953907815631263\\
0.00801603206412782	-0.953451019829765\\
-0.00755923626263144	-0.937875751503006\\
-0.00801603206412826	-0.937418955701509\\
-0.0230783486081656	-0.92184368737475\\
-0.0240480961923848	-0.920840656968006\\
-0.0381188831789675	-0.905811623246493\\
-0.0400801603206413	-0.90371570234777\\
-0.0527247857913811	-0.889779559118236\\
-0.0561122244488979	-0.886043248736264\\
-0.0669343440570817	-0.87374749498998\\
-0.0721442885771544	-0.867822030121793\\
-0.0807810122405983	-0.857715430861723\\
-0.0881763527054109	-0.8490503559569\\
-0.0942940952626444	-0.841683366733467\\
-0.104208416833667	-0.829726108982349\\
-0.107499319045926	-0.82565130260521\\
-0.120240480961924	-0.809846742499741\\
-0.120419308558411	-0.809619238476954\\
-0.132975369684952	-0.793587174348697\\
-0.13627254509018	-0.789368182617485\\
-0.145279652829894	-0.777555110220441\\
-0.152304609218437	-0.768320346898754\\
-0.157354243807146	-0.761523046092184\\
-0.168336673346694	-0.746701578198513\\
-0.169213498649643	-0.745490981963928\\
-0.180773489041761	-0.729458917835671\\
-0.18436873747495	-0.724456673027176\\
-0.192122093660993	-0.713426853707415\\
-0.200400801603207	-0.70161329538782\\
-0.203294662397068	-0.697394789579158\\
-0.214244932658426	-0.681362725450902\\
-0.216432865731463	-0.678144434614059\\
-0.224965119394612	-0.665330661322646\\
-0.232464929859719	-0.654026812993221\\
-0.235540035818519	-0.649298597194389\\
-0.245914642067158	-0.633266533066132\\
-0.248496993987976	-0.629254965872603\\
-0.256086167319485	-0.617234468937876\\
-0.264529058116232	-0.603806660739742\\
-0.266136311158703	-0.601202404809619\\
-0.275967306441132	-0.585170340681363\\
-0.280561122244489	-0.577640633327163\\
-0.285655289054161	-0.569138276553106\\
-0.295210419473976	-0.55310621242485\\
-0.296593186372745	-0.550769856052097\\
-0.30455760395013	-0.537074148296593\\
-0.312625250501002	-0.523133010985485\\
-0.313815042172045	-0.521042084168337\\
-0.322866732980163	-0.50501002004008\\
-0.328657314629258	-0.494695381376427\\
-0.331814981981456	-0.488977955911824\\
-0.34060452557982	-0.472945891783567\\
-0.344689378757515	-0.465443483965052\\
-0.349260440368795	-0.456913827655311\\
-0.357791295522285	-0.440881763527054\\
-0.360721442885771	-0.435331346244714\\
-0.366170395731711	-0.424849699398798\\
-0.374445905957461	-0.408817635270541\\
-0.376753507014028	-0.404308051513471\\
-0.382562459543581	-0.392785571142285\\
-0.390585847977489	-0.376753507014028\\
-0.392785571142285	-0.372317222951554\\
-0.398452948292133	-0.360721442885771\\
-0.406227313185292	-0.344689378757515\\
-0.408817635270541	-0.339296438816749\\
-0.413856949942716	-0.328657314629258\\
-0.421385260700624	-0.312625250501002\\
-0.424849699398798	-0.305176568166802\\
-0.428788385313387	-0.296593186372745\\
-0.436073478999931	-0.280561122244489\\
-0.440881763527054	-0.269881014607484\\
-0.443260064715237	-0.264529058116232\\
-0.450304642951515	-0.248496993987976\\
-0.456913827655311	-0.233324853259175\\
-0.457283740180684	-0.232464929859719\\
-0.46409036637589	-0.216432865731463\\
-0.470835143423245	-0.200400801603207\\
-0.472945891783567	-0.195321167180017\\
-0.477441250432316	-0.18436873747495\\
-0.483947723367294	-0.168336673346694\\
-0.488977955911824	-0.155805953765381\\
-0.490366928106163	-0.152304609218437\\
-0.496637367872314	-0.13627254509018\\
-0.502841024354953	-0.120240480961924\\
-0.50501002004008	-0.114555921806545\\
-0.508912636015967	-0.104208416833667\\
-0.514879435468517	-0.0881763527054109\\
-0.520776905952091	-0.0721442885771544\\
-0.521042084168337	-0.0714111270456461\\
-0.526513104852846	-0.0561122244488979\\
-0.532172574628555	-0.0400801603206413\\
-0.537074148296593	-0.026006153876575\\
-0.537748689204563	-0.0240480961923848\\
-0.54317196671062	-0.00801603206412826\\
-0.548519704985299	0.00801603206412782\\
-0.55310621242485	0.0219734059878419\\
-0.553780753332819	0.0240480961923843\\
-0.558890672250114	0.0400801603206409\\
-0.563921555665619	0.0561122244488974\\
-0.56887309833686	0.0721442885771539\\
-0.569138276553106	0.073022017184192\\
-0.573668496296284	0.0881763527054105\\
-0.57837666810335	0.104208416833667\\
-0.583001344996236	0.120240480961924\\
-0.585170340681363	0.127924561773787\\
-0.58750253504049	0.13627254509018\\
-0.591881390411667	0.152304609218437\\
-0.596172172265089	0.168336673346693\\
-0.60037395363977	0.18436873747495\\
-0.601202404809619	0.187616749668742\\
-0.60442986328447	0.200400801603206\\
-0.608378943530199	0.216432865731463\\
-0.612233727171058	0.232464929859719\\
-0.615992884122811	0.248496993987976\\
-0.617234468937876	0.25395901965368\\
-0.619612770126059	0.264529058116232\\
-0.623111489402566	0.280561122244489\\
-0.626508521446038	0.296593186372745\\
-0.629802094367959	0.312625250501002\\
-0.632990291864623	0.328657314629258\\
-0.633266533066132	0.330105497532926\\
-0.6360197366617	0.344689378757515\\
-0.638933910215981	0.360721442885771\\
-0.641735488262777	0.376753507014028\\
-0.644422048509105	0.392785571142285\\
-0.646990996137822	0.408817635270541\\
-0.649298597194389	0.423934427719838\\
-0.649436850751654	0.424849699398798\\
-0.651716318887212	0.440881763527054\\
-0.653869658805669	0.456913827655311\\
-0.655893692198952	0.472945891783567\\
-0.657785033616345	0.488977955911824\\
-0.659540079673549	0.50501002004008\\
-0.661154997500123	0.521042084168337\\
-0.662625712366185	0.537074148296593\\
-0.663947894423875	0.55310621242485\\
-0.665116944493114	0.569138276553106\\
-0.665330661322646	0.572556297241498\\
-0.666110017040159	0.585170340681363\\
-0.666937914365116	0.601202404809619\\
-0.667600197623856	0.617234468937876\\
-0.668091410176274	0.633266533066132\\
-0.668405767281445	0.649298597194389\\
-0.668537136825008	0.665330661322646\\
-0.668479018613839	0.681362725450902\\
-0.668224522116507	0.697394789579159\\
-0.6677663425161	0.713426853707415\\
-0.66709673492865	0.729458917835672\\
-0.666207486625595	0.745490981963928\\
-0.665330661322646	0.758105025403793\\
}--cycle;


\addplot[area legend,solid,fill=mycolor11,draw=black,forget plot]
table[row sep=crcr] {%
x	y\\
-0.472945891783567	0.631558403398626\\
-0.472605099508047	0.633266533066132\\
-0.468985504333452	0.649298597194389\\
-0.464849004208713	0.665330661322646\\
-0.460159075615718	0.681362725450902\\
-0.456913827655311	0.691303111073879\\
-0.454736499135691	0.697394789579159\\
-0.448395418577774	0.713426853707415\\
-0.441293873286597	0.729458917835672\\
-0.440881763527054	0.730315450863864\\
-0.432752526676281	0.745490981963928\\
-0.424849699398798	0.758778909934585\\
-0.423006465444712	0.761523046092185\\
-0.411260209628666	0.777555110220441\\
-0.408817635270541	0.780620139334836\\
-0.396869159056258	0.793587174348698\\
-0.392785571142285	0.797670762262671\\
-0.378610477861005	0.809619238476954\\
-0.376753507014028	0.811071611595405\\
-0.360721442885771	0.821430877328627\\
-0.352418655016157	0.825651302605211\\
-0.344689378757515	0.829329654544471\\
-0.328657314629258	0.835068349076468\\
-0.312625250501002	0.838968684198747\\
-0.296593186372745	0.841199718536656\\
-0.285769468609448	0.841683366733467\\
-0.280561122244489	0.841903918836987\\
-0.275352775879529	0.841683366733467\\
-0.264529058116232	0.841227025015592\\
-0.248496993987976	0.83926680531526\\
-0.232464929859719	0.836128895371847\\
-0.216432865731463	0.831901692994961\\
-0.200400801603207	0.82666322786356\\
-0.197825995821692	0.825651302605211\\
-0.18436873747495	0.820378748496846\\
-0.168336673346694	0.813184089207794\\
-0.161302958637682	0.809619238476954\\
-0.152304609218437	0.805070371891943\\
-0.13627254509018	0.796098946117739\\
-0.132188957176207	0.793587174348698\\
-0.120240480961924	0.786252889299523\\
-0.107126535443859	0.777555110220441\\
-0.104208416833667	0.77562315720775\\
-0.0881763527054109	0.764164839203355\\
-0.084723917528645	0.761523046092185\\
-0.0721442885771544	0.751909514130287\\
-0.0642414612996724	0.745490981963928\\
-0.0561122244488979	0.738895348584444\\
-0.045128831381008	0.729458917835672\\
-0.0400801603206413	0.725124673100652\\
-0.0271702987996977	0.713426853707415\\
-0.0240480961923848	0.710599232962271\\
-0.0101933605837475	0.697394789579159\\
-0.00801603206412826	0.695320190446341\\
0.00594143293131114	0.681362725450902\\
0.00801603206412782	0.679288126318085\\
0.0213479068822207	0.665330661322646\\
0.0240480961923843	0.662503040577503\\
0.0361197728705266	0.649298597194389\\
0.0400801603206409	0.64496435245937\\
0.0503347757565049	0.633266533066132\\
0.0561122244488974	0.626670899686649\\
0.0640578544475958	0.617234468937876\\
0.0721442885771539	0.607620936975979\\
0.0773435796145164	0.601202404809619\\
0.0881763527054105	0.587812133792533\\
0.0902380540757649	0.585170340681363\\
0.102710955361553	0.569138276553106\\
0.104208416833667	0.567206323540415\\
0.114761222290392	0.55310621242485\\
0.120240480961924	0.545771927375676\\
0.126524458918498	0.537074148296593\\
0.13627254509018	0.523553855937378\\
0.138026478403903	0.521042084168337\\
0.149159038093379	0.50501002004008\\
0.152304609218437	0.50046115345507\\
0.160001583014829	0.488977955911824\\
0.168336673346693	0.476510742514408\\
0.170649960884252	0.472945891783567\\
0.180987089305425	0.456913827655311\\
0.18436873747495	0.451641273546946\\
0.191071979166073	0.440881763527054\\
0.200400801603206	0.425861624657147\\
0.201012065831786	0.424849699398798\\
0.210603949389608	0.408817635270541\\
0.216432865731463	0.399035961532035\\
0.220058021957355	0.392785571142285\\
0.229290611656081	0.376753507014028\\
0.232464929859719	0.371199035652408\\
0.238297033847969	0.360721442885771\\
0.247173167878561	0.344689378757515\\
0.248496993987976	0.342272817339308\\
0.255766886224913	0.328657314629258\\
0.264289549707929	0.312625250501002\\
0.264529058116232	0.312168908783128\\
0.272502383661635	0.296593186372745\\
0.280561122244489	0.28078167434801\\
0.280670880448816	0.280561122244489\\
0.288535564991394	0.264529058116232\\
0.296360600228989	0.248496993987976\\
0.296593186372745	0.248013345791166\\
0.303895898217337	0.232464929859719\\
0.311376856476042	0.216432865731463\\
0.312625250501002	0.213718183196744\\
0.318610457811489	0.200400801603206\\
0.325750495336266	0.18436873747495\\
0.328657314629258	0.177753719817951\\
0.332704086079932	0.168336673346693\\
0.339506322152324	0.152304609218437\\
0.344689378757515	0.139950897029441\\
0.346199540087413	0.13627254509018\\
0.352667032756419	0.120240480961924\\
0.359067068258363	0.104208416833667\\
0.360721442885771	0.0999879915570851\\
0.36525335024532	0.0881763527054105\\
0.371314638075837	0.0721442885771539\\
0.376753507014028	0.0575645975673485\\
0.377284149442735	0.0561122244488974\\
0.383009686993799	0.0400801603206409\\
0.388652960980946	0.0240480961923843\\
0.392785571142285	0.0120996199781027\\
0.394169229744255	0.00801603206412782\\
0.399472542357823	-0.00801603206412826\\
0.404685025109202	-0.0240480961923848\\
0.408817635270541	-0.0370151312062448\\
0.409774840597944	-0.0400801603206413\\
0.414642913275298	-0.0561122244488979\\
0.419410830460606	-0.0721442885771544\\
0.424076362139965	-0.0881763527054109\\
0.424849699398798	-0.0909204888630107\\
0.428518805443235	-0.104208416833667\\
0.432827353397701	-0.120240480961924\\
0.437022813017709	-0.13627254509018\\
0.440881763527054	-0.151448076190245\\
0.441095225623083	-0.152304609218437\\
0.444928534977728	-0.168336673346694\\
0.448637216736993	-0.18436873747495\\
0.452217531449852	-0.200400801603207\\
0.45566543363035	-0.216432865731463\\
0.456913827655311	-0.222524544236743\\
0.458909321024828	-0.232464929859719\\
0.46197191826691	-0.248496993987976\\
0.464888270402216	-0.264529058116232\\
0.467653230225484	-0.280561122244489\\
0.470261256084326	-0.296593186372745\\
0.472706383375263	-0.312625250501002\\
0.472945891783567	-0.314333380168509\\
0.474911396503024	-0.328657314629258\\
0.476934843536039	-0.344689378757515\\
0.478777995771817	-0.360721442885771\\
0.48043346647087	-0.376753507014028\\
0.481893321940788	-0.392785571142285\\
0.483149039569969	-0.408817635270541\\
0.484191461832447	-0.424849699398798\\
0.485010745812909	-0.440881763527054\\
0.485596307742298	-0.456913827655311\\
0.48593676196695	-0.472945891783567\\
0.486019853696573	-0.488977955911824\\
0.485832384786766	-0.50501002004008\\
0.485360131708188	-0.521042084168337\\
0.484587754734332	-0.537074148296593\\
0.483498697240291	-0.55310621242485\\
0.482075073842231	-0.569138276553106\\
0.480297545917277	-0.585170340681363\\
0.47814518282093	-0.601202404809619\\
0.475595306857631	-0.617234468937876\\
0.472945891783567	-0.631558403398626\\
0.472605099508047	-0.633266533066132\\
0.468985504333452	-0.649298597194389\\
0.464849004208713	-0.665330661322646\\
0.46015907561572	-0.681362725450902\\
0.456913827655311	-0.69130311107388\\
0.454736499135691	-0.697394789579158\\
0.448395418577773	-0.713426853707415\\
0.441293873286598	-0.729458917835671\\
0.440881763527054	-0.730315450863864\\
0.432752526676281	-0.745490981963928\\
0.424849699398798	-0.758778909934585\\
0.423006465444711	-0.761523046092184\\
0.411260209628669	-0.777555110220441\\
0.408817635270541	-0.780620139334837\\
0.396869159056258	-0.793587174348697\\
0.392785571142285	-0.797670762262671\\
0.378610477861005	-0.809619238476954\\
0.376753507014028	-0.811071611595405\\
0.360721442885771	-0.821430877328629\\
0.352418655016159	-0.82565130260521\\
0.344689378757515	-0.829329654544471\\
0.328657314629258	-0.835068349076468\\
0.312625250501002	-0.838968684198748\\
0.296593186372745	-0.841199718536656\\
0.285769468609449	-0.841683366733467\\
0.280561122244489	-0.841903918836987\\
0.275352775879529	-0.841683366733467\\
0.264529058116232	-0.841227025015592\\
0.248496993987976	-0.839266805315259\\
0.232464929859719	-0.836128895371846\\
0.216432865731463	-0.83190169299496\\
0.200400801603206	-0.826663227863559\\
0.197825995821691	-0.82565130260521\\
0.18436873747495	-0.820378748496845\\
0.168336673346693	-0.813184089207795\\
0.161302958637679	-0.809619238476954\\
0.152304609218437	-0.805070371891943\\
0.13627254509018	-0.796098946117737\\
0.132188957176207	-0.793587174348697\\
0.120240480961924	-0.786252889299523\\
0.107126535443859	-0.777555110220441\\
0.104208416833667	-0.77562315720775\\
0.0881763527054105	-0.764164839203354\\
0.0847239175286446	-0.761523046092184\\
0.0721442885771539	-0.751909514130287\\
0.064241461299671	-0.745490981963928\\
0.0561122244488974	-0.738895348584444\\
0.0451288313810076	-0.729458917835671\\
0.0400801603206409	-0.725124673100651\\
0.0271702987996958	-0.713426853707415\\
0.0240480961923843	-0.710599232962272\\
0.010193360583747	-0.697394789579158\\
0.00801603206412782	-0.69532019044634\\
-0.00594143293131158	-0.681362725450902\\
-0.00801603206412826	-0.679288126318085\\
-0.0213479068822197	-0.665330661322646\\
-0.0240480961923848	-0.662503040577502\\
-0.0361197728705266	-0.649298597194389\\
-0.0400801603206413	-0.644964352459369\\
-0.050334775756504	-0.633266533066132\\
-0.0561122244488979	-0.626670899686648\\
-0.0640578544475958	-0.617234468937876\\
-0.0721442885771544	-0.607620936975978\\
-0.0773435796145168	-0.601202404809619\\
-0.0881763527054109	-0.587812133792533\\
-0.0902380540757654	-0.585170340681363\\
-0.102710955361554	-0.569138276553106\\
-0.104208416833667	-0.567206323540416\\
-0.114761222290391	-0.55310621242485\\
-0.120240480961924	-0.545771927375675\\
-0.126524458918498	-0.537074148296593\\
-0.13627254509018	-0.523553855937377\\
-0.138026478403903	-0.521042084168337\\
-0.149159038093379	-0.50501002004008\\
-0.152304609218437	-0.500461153455069\\
-0.160001583014829	-0.488977955911824\\
-0.168336673346694	-0.476510742514407\\
-0.170649960884251	-0.472945891783567\\
-0.180987089305424	-0.456913827655311\\
-0.18436873747495	-0.451641273546945\\
-0.191071979166074	-0.440881763527054\\
-0.200400801603207	-0.425861624657147\\
-0.201012065831786	-0.424849699398798\\
-0.210603949389609	-0.408817635270541\\
-0.216432865731463	-0.399035961532034\\
-0.220058021957355	-0.392785571142285\\
-0.229290611656081	-0.376753507014028\\
-0.232464929859719	-0.371199035652408\\
-0.238297033847968	-0.360721442885771\\
-0.247173167878561	-0.344689378757515\\
-0.248496993987976	-0.342272817339309\\
-0.255766886224913	-0.328657314629258\\
-0.264289549707929	-0.312625250501002\\
-0.264529058116232	-0.312168908783128\\
-0.272502383661634	-0.296593186372745\\
-0.280561122244489	-0.28078167434801\\
-0.280670880448816	-0.280561122244489\\
-0.288535564991394	-0.264529058116232\\
-0.296360600228989	-0.248496993987976\\
-0.296593186372745	-0.248013345791166\\
-0.303895898217337	-0.232464929859719\\
-0.311376856476042	-0.216432865731463\\
-0.312625250501002	-0.213718183196745\\
-0.318610457811488	-0.200400801603207\\
-0.325750495336266	-0.18436873747495\\
-0.328657314629258	-0.177753719817951\\
-0.332704086079932	-0.168336673346694\\
-0.339506322152324	-0.152304609218437\\
-0.344689378757515	-0.139950897029441\\
-0.346199540087413	-0.13627254509018\\
-0.352667032756418	-0.120240480961924\\
-0.359067068258364	-0.104208416833667\\
-0.360721442885771	-0.0999879915570855\\
-0.365253350245319	-0.0881763527054109\\
-0.371314638075837	-0.0721442885771544\\
-0.376753507014028	-0.0575645975673489\\
-0.377284149442735	-0.0561122244488979\\
-0.383009686993798	-0.0400801603206413\\
-0.388652960980946	-0.0240480961923848\\
-0.392785571142285	-0.0120996199781018\\
-0.394169229744254	-0.00801603206412826\\
-0.399472542357823	0.00801603206412782\\
-0.404685025109202	0.0240480961923843\\
-0.408817635270541	0.0370151312062466\\
-0.409774840597943	0.0400801603206409\\
-0.414642913275298	0.0561122244488974\\
-0.419410830460605	0.0721442885771539\\
-0.424076362139966	0.0881763527054105\\
-0.424849699398798	0.0909204888630098\\
-0.428518805443234	0.104208416833667\\
-0.432827353397701	0.120240480961924\\
-0.437022813017709	0.13627254509018\\
-0.440881763527054	0.151448076190245\\
-0.441095225623083	0.152304609218437\\
-0.444928534977728	0.168336673346693\\
-0.448637216736993	0.18436873747495\\
-0.452217531449852	0.200400801603206\\
-0.455665433630351	0.216432865731463\\
-0.456913827655311	0.222524544236743\\
-0.458909321024828	0.232464929859719\\
-0.461971918266909	0.248496993987976\\
-0.464888270402216	0.264529058116232\\
-0.467653230225484	0.280561122244489\\
-0.470261256084326	0.296593186372745\\
-0.472706383375263	0.312625250501002\\
-0.472945891783567	0.314333380168508\\
-0.474911396503024	0.328657314629258\\
-0.476934843536038	0.344689378757515\\
-0.478777995771816	0.360721442885771\\
-0.48043346647087	0.376753507014028\\
-0.481893321940788	0.392785571142285\\
-0.483149039569969	0.408817635270541\\
-0.484191461832447	0.424849699398798\\
-0.485010745812909	0.440881763527054\\
-0.485596307742298	0.456913827655311\\
-0.48593676196695	0.472945891783567\\
-0.486019853696573	0.488977955911824\\
-0.485832384786766	0.50501002004008\\
-0.485360131708188	0.521042084168337\\
-0.484587754734332	0.537074148296593\\
-0.483498697240291	0.55310621242485\\
-0.482075073842231	0.569138276553106\\
-0.480297545917278	0.585170340681363\\
-0.47814518282093	0.601202404809619\\
-0.475595306857631	0.617234468937876\\
-0.472945891783567	0.631558403398626\\
}--cycle;


\addplot[area legend,solid,fill=mycolor12,draw=black,forget plot]
table[row sep=crcr] {%
x	y\\
-0.200400801603207	0.304862370138295\\
-0.197344662109513	0.312625250501002\\
-0.189390682595018	0.328657314629258\\
-0.18436873747495	0.336847857012185\\
-0.177891791257849	0.344689378757515\\
-0.168336673346694	0.35424449666867\\
-0.158402802348644	0.360721442885771\\
-0.152304609218437	0.364110634083375\\
-0.13627254509018	0.368853940670599\\
-0.120240480961924	0.369881171754941\\
-0.104208416833667	0.367869101570326\\
-0.0881763527054109	0.363336633675013\\
-0.082078159575204	0.360721442885771\\
-0.0721442885771544	0.356466805104686\\
-0.0561122244488979	0.34759700820266\\
-0.0518105910934108	0.344689378757515\\
-0.0400801603206413	0.336766468226409\\
-0.0297366567970107	0.328657314629258\\
-0.0240480961923848	0.324199859313259\\
-0.0110721715578217	0.312625250501002\\
-0.00801603206412826	0.309899854870135\\
0.00529063643326164	0.296593186372745\\
0.00801603206412782	0.293867790741879\\
0.0200260317046286	0.280561122244489\\
0.0240480961923843	0.276103666928489\\
0.0335674421125732	0.264529058116232\\
0.0400801603206409	0.256606147585127\\
0.0462057588758648	0.248496993987976\\
0.0561122244488974	0.235372559304864\\
0.0581431221874816	0.232464929859719\\
0.0692259403854356	0.216432865731463\\
0.0721442885771539	0.212178227950381\\
0.0796438295284159	0.200400801603206\\
0.0881763527054105	0.186983928264192\\
0.0897289331740241	0.18436873747495\\
0.099068780199145	0.168336673346693\\
0.104208416833667	0.159452267902991\\
0.108078661342575	0.152304609218437\\
0.116632841383845	0.13627254509018\\
0.120240480961924	0.129400209831093\\
0.124752283040789	0.120240480961924\\
0.132520874517804	0.104208416833667\\
0.13627254509018	0.0963088504902381\\
0.139905087192154	0.0881763527054105\\
0.146892501509108	0.0721442885771539\\
0.152304609218437	0.0595014156465018\\
0.153672158159379	0.0561122244488974\\
0.159885987743175	0.0400801603206409\\
0.165975112210303	0.0240480961923843\\
0.168336673346693	0.0175711499752883\\
0.171621426422992	0.00801603206412782\\
0.176904626791464	-0.00801603206412826\\
0.182007176338558	-0.0240480961923848\\
0.18436873747495	-0.0318896179377154\\
0.186695011676916	-0.0400801603206413\\
0.190963287470592	-0.0561122244488979\\
0.194988693893878	-0.0721442885771544\\
0.198748967636206	-0.0881763527054109\\
0.200400801603206	-0.0959392330681169\\
0.202057402535111	-0.104208416833667\\
0.204912603682072	-0.120240480961924\\
0.207426878711987	-0.13627254509018\\
0.209566606502076	-0.152304609218437\\
0.211293229096941	-0.168336673346694\\
0.212562354681171	-0.18436873747495\\
0.213322657300812	-0.200400801603207\\
0.213514517539743	-0.216432865731463\\
0.213068330138315	-0.232464929859719\\
0.211902379345737	-0.248496993987976\\
0.209920147523395	-0.264529058116232\\
0.207006872475959	-0.280561122244489\\
0.203025096977945	-0.296593186372745\\
0.200400801603206	-0.304862370138296\\
0.197344662109512	-0.312625250501002\\
0.189390682595018	-0.328657314629258\\
0.18436873747495	-0.336847857012184\\
0.17789179125785	-0.344689378757515\\
0.168336673346693	-0.354244496668672\\
0.158402802348643	-0.360721442885771\\
0.152304609218437	-0.364110634083375\\
0.13627254509018	-0.3688539406706\\
0.120240480961924	-0.369881171754941\\
0.104208416833667	-0.367869101570326\\
0.0881763527054105	-0.363336633675014\\
0.0820781595752036	-0.360721442885771\\
0.0721442885771539	-0.356466805104686\\
0.0561122244488974	-0.347597008202659\\
0.0518105910934104	-0.344689378757515\\
0.0400801603206409	-0.336766468226409\\
0.0297366567970103	-0.328657314629258\\
0.0240480961923843	-0.324199859313259\\
0.0110721715578212	-0.312625250501002\\
0.00801603206412782	-0.309899854870136\\
-0.00529063643326208	-0.296593186372745\\
-0.00801603206412826	-0.293867790741879\\
-0.020026031704629	-0.280561122244489\\
-0.0240480961923848	-0.276103666928489\\
-0.0335674421125736	-0.264529058116232\\
-0.0400801603206413	-0.256606147585127\\
-0.0462057588758648	-0.248496993987976\\
-0.0561122244488979	-0.235372559304863\\
-0.058143122187482	-0.232464929859719\\
-0.0692259403854342	-0.216432865731463\\
-0.0721442885771544	-0.212178227950378\\
-0.0796438295284154	-0.200400801603207\\
-0.0881763527054109	-0.186983928264193\\
-0.0897289331740245	-0.18436873747495\\
-0.0990687801991455	-0.168336673346694\\
-0.104208416833667	-0.159452267902992\\
-0.108078661342576	-0.152304609218437\\
-0.116632841383844	-0.13627254509018\\
-0.120240480961924	-0.129400209831092\\
-0.124752283040789	-0.120240480961924\\
-0.132520874517804	-0.104208416833667\\
-0.13627254509018	-0.096308850490239\\
-0.139905087192154	-0.0881763527054109\\
-0.146892501509108	-0.0721442885771544\\
-0.152304609218437	-0.0595014156465017\\
-0.15367215815938	-0.0561122244488979\\
-0.159885987743175	-0.0400801603206413\\
-0.165975112210302	-0.0240480961923848\\
-0.168336673346694	-0.0175711499752838\\
-0.171621426422991	-0.00801603206412826\\
-0.176904626791464	0.00801603206412782\\
-0.182007176338559	0.0240480961923843\\
-0.18436873747495	0.031889617937715\\
-0.186695011676916	0.0400801603206409\\
-0.190963287470593	0.0561122244488974\\
-0.194988693893878	0.0721442885771539\\
-0.198748967636206	0.0881763527054105\\
-0.200400801603207	0.0959392330681169\\
-0.202057402535111	0.104208416833667\\
-0.204912603682072	0.120240480961924\\
-0.207426878711987	0.13627254509018\\
-0.209566606502074	0.152304609218437\\
-0.211293229096941	0.168336673346693\\
-0.212562354681169	0.18436873747495\\
-0.213322657300812	0.200400801603206\\
-0.213514517539743	0.216432865731463\\
-0.213068330138315	0.232464929859719\\
-0.211902379345734	0.248496993987976\\
-0.209920147523395	0.264529058116232\\
-0.207006872475959	0.280561122244489\\
-0.203025096977946	0.296593186372745\\
-0.200400801603207	0.304862370138295\\
}--cycle;

\end{axis}
\end{tikzpicture}%
    \caption{$\mat{\Sigma} = \begin{bmatrix} 1 & -1 \\ -1 & 3 \end{bmatrix}$}
    \label{2d_example_3}
  \end{subfigure}
  \caption{Contour plots for example bivariate Gaussian distributions.
    Here $\vec{\mu} = \vec{0}$ for all examples.}
  \label{2d_examples}
\end{figure}

Figure \ref{2d_examples} shows contour plots of the density of three
bivariate (two-dimensional) Gaussian distributions.  The elliptical
shape of the contours is clear.

The Gaussian distribution has a number of convenient analytic
properties, some of which we describe below.

\subsection*{Marginalization}

Often we will have a set of variables $\vec{x}$ with a joint
multivariate Gaussian distribution, but only be interested in
reasoning about a subset of these variables.  Suppose $\vec{x}$
has a multivariate Gaussian distribution:
\begin{equation*}
  p(\vec{x} \given \vec{\mu}, \mat{\Sigma})
  =
  \mc{N}(\vec{x}, \vec{\mu}, \mat{\Sigma}).
\end{equation*}

Let us partition the vector into two components:
\begin{equation*}
  \vec{x}
  =
  \begin{bmatrix}
    \vec{x}_1 \\
    \vec{x}_2
  \end{bmatrix}.
\end{equation*}
We partition the mean vector and covariance matrix in the same
way:
\begin{equation*}
  \vec{\mu}
  =
  \begin{bmatrix}
    \vec{\mu}_1 \\
    \vec{\mu}_2
  \end{bmatrix}
  \qquad
  \mat{\Sigma}
  =
  \begin{bmatrix}
    \mat{\Sigma}_{11} & \mat{\Sigma}_{12} \\
    \mat{\Sigma}_{21} & \mat{\Sigma}_{22}
  \end{bmatrix}.
\end{equation*}
Now the marginal distribution of the subvector $\vec{x}_1$
has a simple form:
\begin{equation*}
  p(\vec{x}_1 \given \vec{\mu}, \mat{\Sigma})
  =
  \mc{N}(\vec{x}_1, \vec{\mu}_1, \mat{\Sigma}_{11}),
\end{equation*}
so we simply pick out the entries of $\vec{\mu}$ and $\mat{\Sigma}$
corresponding to $\vec{x}_1$.

Figure \ref{marginal_example} illustrates the marginal distribution of
$x_1$ for the joint distribution shown in Figure
\ref{2d_examples}(\subref{2d_example_3}).

\begin{figure}
  \centering
  \begin{subfigure}[t]{0.49\textwidth}
    % This file was created by matlab2tikz.
% Minimal pgfplots version: 1.3
%
\tikzsetnextfilename{2d_gaussian_pdf_3_large}
\definecolor{mycolor1}{rgb}{0.01430,0.01430,0.01430}%
\definecolor{mycolor2}{rgb}{0.15932,0.06827,0.17506}%
\definecolor{mycolor3}{rgb}{0.17345,0.11709,0.41691}%
\definecolor{mycolor4}{rgb}{0.10466,0.22842,0.49922}%
\definecolor{mycolor5}{rgb}{0.03136,0.34573,0.47968}%
\definecolor{mycolor6}{rgb}{0.00003,0.46181,0.36160}%
\definecolor{mycolor7}{rgb}{0.00000,0.57116,0.23204}%
\definecolor{mycolor8}{rgb}{0.09251,0.67012,0.06175}%
\definecolor{mycolor9}{rgb}{0.37724,0.75416,0.00000}%
\definecolor{mycolor10}{rgb}{0.74811,0.78629,0.07418}%
\definecolor{mycolor11}{rgb}{0.94890,0.82638,0.64748}%
\definecolor{mycolor12}{rgb}{0.96920,0.92730,0.89610}%
%
\begin{tikzpicture}

\begin{axis}[%
width=0.95092\squarefigurewidth,
height=\squarefigureheight,
at={(0\squarefigurewidth,0\squarefigureheight)},
scale only axis,
xmin=-4,
xmax=4,
xlabel={$x_1$},
ymin=-4,
ymax=4,
ylabel={$x_2$},
axis x line*=bottom,
axis y line*=left
]

\addplot[area legend,solid,fill=mycolor1,draw=black,forget plot]
table[row sep=crcr] {%
x	y\\
-4	4.00000000053783\\
-3.98396793587174	4.00000000057334\\
-3.96793587174349	4.00000000061096\\
-3.95190380761523	4.0000000006508\\
-3.93587174348697	4.00000000069296\\
-3.91983967935872	4.00000000073758\\
-3.90380761523046	4.00000000078476\\
-3.8877755511022	4.00000000083464\\
-3.87174348697395	4.00000000088735\\
-3.85571142284569	4.00000000094302\\
-3.83967935871743	4.0000000010018\\
-3.82364729458918	4.00000000106383\\
-3.80761523046092	4.00000000112927\\
-3.79158316633267	4.00000000119827\\
-3.77555110220441	4.000000001271\\
-3.75951903807615	4.00000000134762\\
-3.7434869739479	4.00000000142832\\
-3.72745490981964	4.00000000151325\\
-3.71142284569138	4.00000000160263\\
-3.69539078156313	4.00000000169662\\
-3.67935871743487	4.00000000179544\\
-3.66332665330661	4.00000000189928\\
-3.64729458917836	4.00000000200835\\
-3.6312625250501	4.00000000212287\\
-3.61523046092184	4.00000000224305\\
-3.59919839679359	4.00000000236912\\
-3.58316633266533	4.00000000250131\\
-3.56713426853707	4.00000000263986\\
-3.55110220440882	4.00000000278501\\
-3.53507014028056	4.00000000293701\\
-3.5190380761523	4.00000000309611\\
-3.50300601202405	4.00000000326257\\
-3.48697394789579	4.00000000343666\\
-3.47094188376753	4.00000000361864\\
-3.45490981963928	4.00000000380879\\
-3.43887775551102	4.00000000400738\\
-3.42284569138277	4.00000000421471\\
-3.40681362725451	4.00000000443105\\
-3.39078156312625	4.0000000046567\\
-3.374749498998	4.00000000489195\\
-3.35871743486974	4.00000000513711\\
-3.34268537074148	4.00000000539247\\
-3.32665330661323	4.00000000565835\\
-3.31062124248497	4.00000000593505\\
-3.29458917835671	4.00000000622288\\
-3.27855711422846	4.00000000652215\\
-3.2625250501002	4.00000000683318\\
-3.24649298597194	4.00000000715629\\
-3.23046092184369	4.00000000749178\\
-3.21442885771543	4.00000000783997\\
-3.19839679358717	4.00000000820119\\
-3.18236472945892	4.00000000857575\\
-3.16633266533066	4.00000000896395\\
-3.1503006012024	4.00000000936611\\
-3.13426853707415	4.00000000978255\\
-3.11823647294589	4.00000001021356\\
-3.10220440881764	4.00000001065945\\
-3.08617234468938	4.00000001112052\\
-3.07014028056112	4.00000001159706\\
-3.05410821643287	4.00000001208936\\
-3.03807615230461	4.0000000125977\\
-3.02204408817635	4.00000001312236\\
-3.0060120240481	4.00000001366359\\
-2.98997995991984	4.00000001422167\\
-2.97394789579158	4.00000001479684\\
-2.95791583166333	4.00000001538933\\
-2.94188376753507	4.00000001599937\\
-2.92585170340681	4.00000001662719\\
-2.90981963927856	4.00000001727298\\
-2.8937875751503	4.00000001793694\\
-2.87775551102204	4.00000001861924\\
-2.86172344689379	4.00000001932004\\
-2.84569138276553	4.00000002003949\\
-2.82965931863727	4.00000002077772\\
-2.81362725450902	4.00000002153485\\
-2.79759519038076	4.00000002231096\\
-2.7815631262525	4.00000002310613\\
-2.76553106212425	4.00000002392041\\
-2.74949899799599	4.00000002475385\\
-2.73346693386774	4.00000002560645\\
-2.71743486973948	4.0000000264782\\
-2.70140280561122	4.00000002736908\\
-2.68537074148297	4.00000002827904\\
-2.66933867735471	4.00000002920798\\
-2.65330661322645	4.0000000301558\\
-2.6372745490982	4.00000003112239\\
-2.62124248496994	4.00000003210757\\
-2.60521042084168	4.00000003311117\\
-2.58917835671343	4.00000003413298\\
-2.57314629258517	4.00000003517276\\
-2.55711422845691	4.00000003623025\\
-2.54108216432866	4.00000003730514\\
-2.5250501002004	4.00000003839712\\
-2.50901803607214	4.00000003950582\\
-2.49298597194389	4.00000004063087\\
-2.47695390781563	4.00000004177185\\
-2.46092184368737	4.00000004292832\\
-2.44488977955912	4.0000000440998\\
-2.42885771543086	4.00000004528579\\
-2.41282565130261	4.00000004648575\\
-2.39679358717435	4.0000000476991\\
-2.38076152304609	4.00000004892527\\
-2.36472945891784	4.0000000501636\\
-2.34869739478958	4.00000005141346\\
-2.33266533066132	4.00000005267415\\
-2.31663326653307	4.00000005394494\\
-2.30060120240481	4.0000000552251\\
-2.28456913827655	4.00000005651385\\
-2.2685370741483	4.00000005781037\\
-2.25250501002004	4.00000005911385\\
-2.23647294589178	4.00000006042342\\
-2.22044088176353	4.00000006173819\\
-2.20440881763527	4.00000006305725\\
-2.18837675350701	4.00000006437967\\
-2.17234468937876	4.00000006570449\\
-2.1563126252505	4.00000006703072\\
-2.14028056112224	4.00000006835736\\
-2.12424849699399	4.00000006968339\\
-2.10821643286573	4.00000007100775\\
-2.09218436873747	4.0000000723294\\
-2.07615230460922	4.00000007364725\\
-2.06012024048096	4.0000000749602\\
-2.04408817635271	4.00000007626715\\
-2.02805611222445	4.00000007756697\\
-2.01202404809619	4.00000007885854\\
-1.99599198396794	4.00000008014071\\
-1.97995991983968	4.00000008141234\\
-1.96392785571142	4.00000008267226\\
-1.94789579158317	4.00000008391932\\
-1.93186372745491	4.00000008515236\\
-1.91583166332665	4.0000000863702\\
-1.8997995991984	4.0000000875717\\
-1.88376753507014	4.00000008875568\\
-1.86773547094188	4.000000089921\\
-1.85170340681363	4.0000000910665\\
-1.83567134268537	4.00000009219105\\
-1.81963927855711	4.0000000932935\\
-1.80360721442886	4.00000009437275\\
-1.7875751503006	4.00000009542768\\
-1.77154308617234	4.00000009645721\\
-1.75551102204409	4.00000009746027\\
-1.73947895791583	4.0000000984358\\
-1.72344689378758	4.00000009938277\\
-1.70741482965932	4.00000010030017\\
-1.69138276553106	4.00000010118702\\
-1.67535070140281	4.00000010204237\\
-1.65931863727455	4.00000010286528\\
-1.64328657314629	4.00000010365485\\
-1.62725450901804	4.00000010441023\\
-1.61122244488978	4.00000010513056\\
-1.59519038076152	4.00000010581507\\
-1.57915831663327	4.00000010646298\\
-1.56312625250501	4.00000010707356\\
-1.54709418837675	4.00000010764614\\
-1.5310621242485	4.00000010818007\\
-1.51503006012024	4.00000010867473\\
-1.49899799599198	4.00000010912958\\
-1.48296593186373	4.00000010954409\\
-1.46693386773547	4.00000010991779\\
-1.45090180360721	4.00000011025024\\
-1.43486973947896	4.00000011054108\\
-1.4188376753507	4.00000011078997\\
-1.40280561122244	4.00000011099661\\
-1.38677354709419	4.00000011116077\\
-1.37074148296593	4.00000011128227\\
-1.35470941883768	4.00000011136095\\
-1.33867735470942	4.00000011139673\\
-1.32264529058116	4.00000011138958\\
-1.30661322645291	4.00000011133949\\
-1.29058116232465	4.00000011124652\\
-1.27454909819639	4.00000011111078\\
-1.25851703406814	4.00000011093244\\
-1.24248496993988	4.00000011071169\\
-1.22645290581162	4.00000011044878\\
-1.21042084168337	4.00000011014403\\
-1.19438877755511	4.00000010979778\\
-1.17835671342685	4.00000010941043\\
-1.1623246492986	4.00000010898242\\
-1.14629258517034	4.00000010851424\\
-1.13026052104208	4.00000010800642\\
-1.11422845691383	4.00000010745955\\
-1.09819639278557	4.00000010687423\\
-1.08216432865731	4.00000010625112\\
-1.06613226452906	4.00000010559093\\
-1.0501002004008	4.00000010489439\\
-1.03406813627255	4.00000010416229\\
-1.01803607214429	4.00000010339542\\
-1.00200400801603	4.00000010259463\\
-0.985971943887776	4.00000010176081\\
-0.969939879759519	4.00000010089486\\
-0.953907815631263	4.00000009999772\\
-0.937875751503006	4.00000009907035\\
-0.92184368737475	4.00000009811374\\
-0.905811623246493	4.00000009712892\\
-0.889779559118236	4.00000009611693\\
-0.87374749498998	4.00000009507881\\
-0.857715430861723	4.00000009401565\\
-0.841683366733467	4.00000009292854\\
-0.82565130260521	4.0000000918186\\
-0.809619238476954	4.00000009068694\\
-0.793587174348697	4.00000008953471\\
-0.777555110220441	4.00000008836304\\
-0.761523046092184	4.00000008717309\\
-0.745490981963928	4.00000008596601\\
-0.729458917835671	4.00000008474298\\
-0.713426853707415	4.00000008350514\\
-0.697394789579158	4.00000008225366\\
-0.681362725450902	4.00000008098971\\
-0.665330661322646	4.00000007971444\\
-0.649298597194389	4.00000007842901\\
-0.633266533066132	4.00000007713456\\
-0.617234468937876	4.00000007583223\\
-0.601202404809619	4.00000007452316\\
-0.585170340681363	4.00000007320845\\
-0.569138276553106	4.00000007188922\\
-0.55310621242485	4.00000007056655\\
-0.537074148296593	4.00000006924151\\
-0.521042084168337	4.00000006791516\\
-0.50501002004008	4.00000006658855\\
-0.488977955911824	4.00000006526268\\
-0.472945891783567	4.00000006393855\\
-0.456913827655311	4.00000006261714\\
-0.440881763527054	4.00000006129941\\
-0.424849699398798	4.00000005998627\\
-0.408817635270541	4.00000005867864\\
-0.392785571142285	4.00000005737738\\
-0.376753507014028	4.00000005608336\\
-0.360721442885771	4.00000005479739\\
-0.344689378757515	4.00000005352027\\
-0.328657314629258	4.00000005225276\\
-0.312625250501002	4.00000005099561\\
-0.296593186372745	4.00000004974951\\
-0.280561122244489	4.00000004851516\\
-0.264529058116232	4.0000000472932\\
-0.248496993987976	4.00000004608424\\
-0.232464929859719	4.00000004488888\\
-0.216432865731463	4.00000004370767\\
-0.200400801603207	4.00000004254114\\
-0.18436873747495	4.00000004138978\\
-0.168336673346694	4.00000004025406\\
-0.152304609218437	4.00000003913442\\
-0.13627254509018	4.00000003803125\\
-0.120240480961924	4.00000003694493\\
-0.104208416833667	4.0000000358758\\
-0.0881763527054109	4.00000003482419\\
-0.0721442885771544	4.00000003379037\\
-0.0561122244488979	4.00000003277461\\
-0.0400801603206413	4.00000003177712\\
-0.0240480961923848	4.00000003079812\\
-0.00801603206412826	4.00000002983777\\
0.00801603206412782	4.00000002889622\\
0.0240480961923843	4.0000000279736\\
0.0400801603206409	4.00000002707\\
0.0561122244488974	4.00000002618549\\
0.0721442885771539	4.00000002532012\\
0.0881763527054105	4.00000002447391\\
0.104208416833667	4.00000002364686\\
0.120240480961924	4.00000002283895\\
0.13627254509018	4.00000002205014\\
0.152304609218437	4.00000002128037\\
0.168336673346693	4.00000002052955\\
0.18436873747495	4.00000001979759\\
0.200400801603206	4.00000001908437\\
0.216432865731463	4.00000001838976\\
0.232464929859719	4.00000001771359\\
0.248496993987976	4.00000001705571\\
0.264529058116232	4.00000001641593\\
0.280561122244489	4.00000001579406\\
0.296593186372745	4.00000001518989\\
0.312625250501002	4.0000000146032\\
0.328657314629258	4.00000001403376\\
0.344689378757515	4.00000001348132\\
0.360721442885771	4.00000001294564\\
0.376753507014028	4.00000001242646\\
0.392785571142285	4.0000000119235\\
0.408817635270541	4.00000001143648\\
0.424849699398798	4.00000001096513\\
0.440881763527054	4.00000001050915\\
0.456913827655311	4.00000001006825\\
0.472945891783567	4.00000000964213\\
0.488977955911824	4.00000000923049\\
0.50501002004008	4.00000000883301\\
0.521042084168337	4.00000000844939\\
0.537074148296593	4.00000000807932\\
0.55310621242485	4.00000000772248\\
0.569138276553106	4.00000000737855\\
0.585170340681363	4.00000000704722\\
0.601202404809619	4.00000000672818\\
0.617234468937876	4.0000000064211\\
0.633266533066132	4.00000000612568\\
0.649298597194389	4.00000000584159\\
0.665330661322646	4.00000000556854\\
0.681362725450902	4.0000000053062\\
0.697394789579159	4.00000000505427\\
0.713426853707415	4.00000000481245\\
0.729458917835672	4.00000000458043\\
0.745490981963928	4.00000000435791\\
0.761523046092185	4.00000000414461\\
0.777555110220441	4.00000000394023\\
0.793587174348698	4.00000000374448\\
0.809619238476954	4.00000000355709\\
0.825651302605211	4.00000000337777\\
0.841683366733467	4.00000000320625\\
0.857715430861724	4.00000000304227\\
0.87374749498998	4.00000000288557\\
0.889779559118236	4.00000000273588\\
0.905811623246493	4.00000000259296\\
0.921843687374749	4.00000000245655\\
0.937875751503006	4.00000000232643\\
0.953907815631262	4.00000000220234\\
0.969939879759519	4.00000000208408\\
0.985971943887775	4.0000000019714\\
1.00200400801603	4.00000000186409\\
1.01803607214429	4.00000000176195\\
1.03406813627254	4.00000000166477\\
1.0501002004008	4.00000000157233\\
1.06613226452906	4.00000000148446\\
1.08216432865731	4.00000000140096\\
1.09819639278557	4.00000000132164\\
1.11422845691383	4.00000000124633\\
1.13026052104208	4.00000000117487\\
1.14629258517034	4.00000000110707\\
1.1623246492986	4.00000000104278\\
1.17835671342685	4.00000000098185\\
1.19438877755511	4.00000000092412\\
1.21042084168337	4.00000000086946\\
1.22645290581162	4.00000000081771\\
1.24248496993988	4.00000000076874\\
1.25851703406814	4.00000000072243\\
1.27454909819639	4.00000000067864\\
1.29058116232465	4.00000000063727\\
1.30661322645291	4.00000000059818\\
1.32264529058116	4.00000000056128\\
1.33867735470942	4.00000000052645\\
1.35470941883768	4.00000000049359\\
1.37074148296593	4.0000000004626\\
1.38677354709419	4.00000000043339\\
1.40280561122244	4.00000000040587\\
1.4188376753507	4.00000000037995\\
1.43486973947896	4.00000000035555\\
1.45090180360721	4.00000000033259\\
1.46693386773547	4.00000000031099\\
1.48296593186373	4.00000000029068\\
1.49899799599198	4.00000000027159\\
1.51503006012024	4.00000000025366\\
1.5310621242485	4.00000000023682\\
1.54709418837675	4.00000000022102\\
1.56312625250501	4.00000000020618\\
1.57915831663327	4.00000000019228\\
1.59519038076152	4.00000000017923\\
1.61122244488978	4.00000000016701\\
1.62725450901804	4.00000000015557\\
1.64328657314629	4.00000000014485\\
1.65931863727455	4.00000000013482\\
1.67535070140281	4.00000000012543\\
1.69138276553106	4.00000000011665\\
1.70741482965932	4.00000000010845\\
1.72344689378758	4.00000000010078\\
1.73947895791583	4.00000000009362\\
1.75551102204409	4.00000000008693\\
1.77154308617235	4.0000000000807\\
1.7875751503006	4.00000000007488\\
1.80360721442886	4.00000000006945\\
1.81963927855711	4.00000000006439\\
1.83567134268537	4.00000000005968\\
1.85170340681363	4.00000000005529\\
1.86773547094188	4.0000000000512\\
1.88376753507014	4.0000000000474\\
1.8997995991984	4.00000000004386\\
1.91583166332665	4.00000000004057\\
1.93186372745491	4.00000000003751\\
1.94789579158317	4.00000000003468\\
1.96392785571142	4.00000000003204\\
1.97995991983968	4.00000000002959\\
1.99599198396794	4.00000000002732\\
2.01202404809619	4.00000000002521\\
2.02805611222445	4.00000000002326\\
2.04408817635271	4.00000000002145\\
2.06012024048096	4.00000000001977\\
2.07615230460922	4.00000000001822\\
2.09218436873747	4.00000000001678\\
2.10821643286573	4.00000000001545\\
2.12424849699399	4.00000000001422\\
2.14028056112224	4.00000000001308\\
2.1563126252505	4.00000000001203\\
2.17234468937876	4.00000000001106\\
2.18837675350701	4.00000000001017\\
2.20440881763527	4.00000000000934\\
2.22044088176353	4.00000000000858\\
2.23647294589178	4.00000000000787\\
2.25250501002004	4.00000000000722\\
2.2685370741483	4.00000000000662\\
2.28456913827655	4.00000000000607\\
2.30060120240481	4.00000000000557\\
2.31663326653307	4.0000000000051\\
2.33266533066132	4.00000000000467\\
2.34869739478958	4.00000000000427\\
2.36472945891784	4.00000000000391\\
2.38076152304609	4.00000000000358\\
2.39679358717435	4.00000000000327\\
2.41282565130261	4.00000000000299\\
2.42885771543086	4.00000000000273\\
2.44488977955912	4.0000000000025\\
2.46092184368737	4.00000000000228\\
2.47695390781563	4.00000000000208\\
2.49298597194389	4.0000000000019\\
2.50901803607214	4.00000000000173\\
2.5250501002004	4.00000000000158\\
2.54108216432866	4.00000000000144\\
2.55711422845691	4.00000000000131\\
2.57314629258517	4.00000000000119\\
2.58917835671343	4.00000000000108\\
2.60521042084168	4.00000000000099\\
2.62124248496994	4.0000000000009\\
2.6372745490982	4.00000000000082\\
2.65330661322645	4.00000000000074\\
2.66933867735471	4.00000000000067\\
2.68537074148297	4.00000000000061\\
2.70140280561122	4.00000000000056\\
2.71743486973948	4.0000000000005\\
2.73346693386774	4.00000000000046\\
2.74949899799599	4.00000000000041\\
2.76553106212425	4.00000000000038\\
2.7815631262525	4.00000000000034\\
2.79759519038076	4.00000000000031\\
2.81362725450902	4.00000000000028\\
2.82965931863727	4.00000000000025\\
2.84569138276553	4.00000000000023\\
2.86172344689379	4.00000000000021\\
2.87775551102204	4.00000000000019\\
2.8937875751503	4.00000000000017\\
2.90981963927856	4.00000000000015\\
2.92585170340681	4.00000000000014\\
2.94188376753507	4.00000000000012\\
2.95791583166333	4.00000000000011\\
2.97394789579158	4.0000000000001\\
2.98997995991984	4.00000000000009\\
3.0060120240481	4.00000000000008\\
3.02204408817635	4.00000000000007\\
3.03807615230461	4.00000000000007\\
3.05410821643287	4.00000000000006\\
3.07014028056112	4.00000000000005\\
3.08617234468938	4.00000000000005\\
3.10220440881764	4.00000000000004\\
3.11823647294589	4.00000000000004\\
3.13426853707415	4.00000000000003\\
3.1503006012024	4.00000000000003\\
3.16633266533066	4.00000000000003\\
3.18236472945892	4.00000000000003\\
3.19839679358717	4.00000000000002\\
3.21442885771543	4.00000000000002\\
3.23046092184369	4.00000000000002\\
3.24649298597194	4.00000000000002\\
3.2625250501002	4.00000000000001\\
3.27855711422846	4.00000000000001\\
3.29458917835671	4.00000000000001\\
3.31062124248497	4.00000000000001\\
3.32665330661323	4.00000000000001\\
3.34268537074148	4.00000000000001\\
3.35871743486974	4.00000000000001\\
3.374749498998	4.00000000000001\\
3.39078156312625	4.00000000000001\\
3.40681362725451	4.00000000000001\\
3.42284569138277	4\\
3.43887775551102	4\\
3.45490981963928	4\\
3.47094188376754	4\\
3.48697394789579	4\\
3.50300601202405	4\\
3.51903807615231	4\\
3.53507014028056	4\\
3.55110220440882	4\\
3.56713426853707	4\\
3.58316633266533	4\\
3.59919839679359	4\\
3.61523046092184	4\\
3.6312625250501	4\\
3.64729458917836	4\\
3.66332665330661	4\\
3.67935871743487	4\\
3.69539078156313	4\\
3.71142284569138	4\\
3.72745490981964	4\\
3.7434869739479	4\\
3.75951903807615	4\\
3.77555110220441	4\\
3.79158316633267	4\\
3.80761523046092	4\\
3.82364729458918	4\\
3.83967935871743	4\\
3.85571142284569	4\\
3.87174348697395	4\\
3.8877755511022	4\\
3.90380761523046	4\\
3.91983967935872	4\\
3.93587174348697	4\\
3.95190380761523	4\\
3.96793587174349	4\\
3.98396793587174	4\\
4	4\\
4	3.98396793587174\\
4	3.96793587174349\\
4	3.95190380761523\\
4	3.93587174348697\\
4	3.91983967935872\\
4	3.90380761523046\\
4	3.8877755511022\\
4	3.87174348697395\\
4	3.85571142284569\\
4	3.83967935871743\\
4	3.82364729458918\\
4	3.80761523046092\\
4	3.79158316633267\\
4	3.77555110220441\\
4	3.75951903807615\\
4	3.7434869739479\\
4	3.72745490981964\\
4	3.71142284569138\\
4	3.69539078156313\\
4	3.67935871743487\\
4	3.66332665330661\\
4	3.64729458917836\\
4	3.6312625250501\\
4	3.61523046092184\\
4	3.59919839679359\\
4	3.58316633266533\\
4	3.56713426853707\\
4	3.55110220440882\\
4	3.53507014028056\\
4	3.51903807615231\\
4	3.50300601202405\\
4	3.48697394789579\\
4	3.47094188376754\\
4	3.45490981963928\\
4	3.43887775551102\\
4	3.42284569138277\\
4	3.40681362725451\\
4	3.39078156312625\\
4	3.374749498998\\
4	3.35871743486974\\
4	3.34268537074148\\
4	3.32665330661323\\
4	3.31062124248497\\
4	3.29458917835671\\
4	3.27855711422846\\
4	3.2625250501002\\
4	3.24649298597194\\
4	3.23046092184369\\
4	3.21442885771543\\
4	3.19839679358717\\
4	3.18236472945892\\
4	3.16633266533066\\
4	3.1503006012024\\
4	3.13426853707415\\
4	3.11823647294589\\
4	3.10220440881764\\
4	3.08617234468938\\
4	3.07014028056112\\
4	3.05410821643287\\
4	3.03807615230461\\
4	3.02204408817635\\
4	3.0060120240481\\
4	2.98997995991984\\
4	2.97394789579158\\
4	2.95791583166333\\
4	2.94188376753507\\
4	2.92585170340681\\
4	2.90981963927856\\
4	2.8937875751503\\
4	2.87775551102204\\
4	2.86172344689379\\
4	2.84569138276553\\
4	2.82965931863727\\
4	2.81362725450902\\
4.00000000000001	2.79759519038076\\
4.00000000000001	2.7815631262525\\
4.00000000000001	2.76553106212425\\
4.00000000000001	2.74949899799599\\
4.00000000000001	2.73346693386774\\
4.00000000000001	2.71743486973948\\
4.00000000000001	2.70140280561122\\
4.00000000000001	2.68537074148297\\
4.00000000000001	2.66933867735471\\
4.00000000000001	2.65330661322645\\
4.00000000000001	2.6372745490982\\
4.00000000000001	2.62124248496994\\
4.00000000000001	2.60521042084168\\
4.00000000000001	2.58917835671343\\
4.00000000000001	2.57314629258517\\
4.00000000000001	2.55711422845691\\
4.00000000000001	2.54108216432866\\
4.00000000000001	2.5250501002004\\
4.00000000000001	2.50901803607214\\
4.00000000000001	2.49298597194389\\
4.00000000000001	2.47695390781563\\
4.00000000000002	2.46092184368737\\
4.00000000000002	2.44488977955912\\
4.00000000000002	2.42885771543086\\
4.00000000000002	2.41282565130261\\
4.00000000000002	2.39679358717435\\
4.00000000000002	2.38076152304609\\
4.00000000000002	2.36472945891784\\
4.00000000000002	2.34869739478958\\
4.00000000000002	2.33266533066132\\
4.00000000000002	2.31663326653307\\
4.00000000000003	2.30060120240481\\
4.00000000000003	2.28456913827655\\
4.00000000000003	2.2685370741483\\
4.00000000000003	2.25250501002004\\
4.00000000000003	2.23647294589178\\
4.00000000000003	2.22044088176353\\
4.00000000000004	2.20440881763527\\
4.00000000000004	2.18837675350701\\
4.00000000000004	2.17234468937876\\
4.00000000000004	2.1563126252505\\
4.00000000000004	2.14028056112224\\
4.00000000000005	2.12424849699399\\
4.00000000000005	2.10821643286573\\
4.00000000000005	2.09218436873747\\
4.00000000000005	2.07615230460922\\
4.00000000000006	2.06012024048096\\
4.00000000000006	2.04408817635271\\
4.00000000000006	2.02805611222445\\
4.00000000000006	2.01202404809619\\
4.00000000000007	1.99599198396794\\
4.00000000000007	1.97995991983968\\
4.00000000000007	1.96392785571142\\
4.00000000000008	1.94789579158317\\
4.00000000000008	1.93186372745491\\
4.00000000000009	1.91583166332665\\
4.00000000000009	1.8997995991984\\
4.00000000000009	1.88376753507014\\
4.0000000000001	1.86773547094188\\
4.0000000000001	1.85170340681363\\
4.00000000000011	1.83567134268537\\
4.00000000000011	1.81963927855711\\
4.00000000000012	1.80360721442886\\
4.00000000000012	1.7875751503006\\
4.00000000000013	1.77154308617235\\
4.00000000000014	1.75551102204409\\
4.00000000000014	1.73947895791583\\
4.00000000000015	1.72344689378758\\
4.00000000000016	1.70741482965932\\
4.00000000000016	1.69138276553106\\
4.00000000000017	1.67535070140281\\
4.00000000000018	1.65931863727455\\
4.00000000000019	1.64328657314629\\
4.0000000000002	1.62725450901804\\
4.00000000000021	1.61122244488978\\
4.00000000000021	1.59519038076152\\
4.00000000000022	1.57915831663327\\
4.00000000000023	1.56312625250501\\
4.00000000000025	1.54709418837675\\
4.00000000000026	1.5310621242485\\
4.00000000000027	1.51503006012024\\
4.00000000000028	1.49899799599198\\
4.00000000000029	1.48296593186373\\
4.00000000000031	1.46693386773547\\
4.00000000000032	1.45090180360721\\
4.00000000000033	1.43486973947896\\
4.00000000000035	1.4188376753507\\
4.00000000000036	1.40280561122244\\
4.00000000000038	1.38677354709419\\
4.0000000000004	1.37074148296593\\
4.00000000000041	1.35470941883768\\
4.00000000000043	1.33867735470942\\
4.00000000000045	1.32264529058116\\
4.00000000000047	1.30661322645291\\
4.00000000000049	1.29058116232465\\
4.00000000000051	1.27454909819639\\
4.00000000000053	1.25851703406814\\
4.00000000000056	1.24248496993988\\
4.00000000000058	1.22645290581162\\
4.00000000000061	1.21042084168337\\
4.00000000000063	1.19438877755511\\
4.00000000000066	1.17835671342685\\
4.00000000000069	1.1623246492986\\
4.00000000000072	1.14629258517034\\
4.00000000000075	1.13026052104208\\
4.00000000000078	1.11422845691383\\
4.00000000000081	1.09819639278557\\
4.00000000000084	1.08216432865731\\
4.00000000000088	1.06613226452906\\
4.00000000000091	1.0501002004008\\
4.00000000000095	1.03406813627254\\
4.00000000000099	1.01803607214429\\
4.00000000000103	1.00200400801603\\
4.00000000000108	0.985971943887775\\
4.00000000000112	0.969939879759519\\
4.00000000000116	0.953907815631262\\
4.00000000000121	0.937875751503006\\
4.00000000000126	0.921843687374749\\
4.00000000000131	0.905811623246493\\
4.00000000000136	0.889779559118236\\
4.00000000000142	0.87374749498998\\
4.00000000000147	0.857715430861724\\
4.00000000000153	0.841683366733467\\
4.00000000000159	0.825651302605211\\
4.00000000000166	0.809619238476954\\
4.00000000000172	0.793587174348698\\
4.00000000000179	0.777555110220441\\
4.00000000000186	0.761523046092185\\
4.00000000000193	0.745490981963928\\
4.000000000002	0.729458917835672\\
4.00000000000208	0.713426853707415\\
4.00000000000216	0.697394789579159\\
4.00000000000225	0.681362725450902\\
4.00000000000233	0.665330661322646\\
4.00000000000242	0.649298597194389\\
4.00000000000251	0.633266533066132\\
4.00000000000261	0.617234468937876\\
4.0000000000027	0.601202404809619\\
4.00000000000281	0.585170340681363\\
4.00000000000291	0.569138276553106\\
4.00000000000302	0.55310621242485\\
4.00000000000313	0.537074148296593\\
4.00000000000325	0.521042084168337\\
4.00000000000337	0.50501002004008\\
4.00000000000349	0.488977955911824\\
4.00000000000362	0.472945891783567\\
4.00000000000375	0.456913827655311\\
4.00000000000388	0.440881763527054\\
4.00000000000403	0.424849699398798\\
4.00000000000417	0.408817635270541\\
4.00000000000432	0.392785571142285\\
4.00000000000448	0.376753507014028\\
4.00000000000464	0.360721442885771\\
4.0000000000048	0.344689378757515\\
4.00000000000497	0.328657314629258\\
4.00000000000514	0.312625250501002\\
4.00000000000532	0.296593186372745\\
4.00000000000551	0.280561122244489\\
4.0000000000057	0.264529058116232\\
4.0000000000059	0.248496993987976\\
4.00000000000611	0.232464929859719\\
4.00000000000632	0.216432865731463\\
4.00000000000653	0.200400801603206\\
4.00000000000676	0.18436873747495\\
4.00000000000699	0.168336673346693\\
4.00000000000722	0.152304609218437\\
4.00000000000747	0.13627254509018\\
4.00000000000772	0.120240480961924\\
4.00000000000798	0.104208416833667\\
4.00000000000824	0.0881763527054105\\
4.00000000000852	0.0721442885771539\\
4.0000000000088	0.0561122244488974\\
4.00000000000909	0.0400801603206409\\
4.00000000000939	0.0240480961923843\\
4.00000000000969	0.00801603206412782\\
4.00000000001001	-0.00801603206412826\\
4.00000000001033	-0.0240480961923848\\
4.00000000001067	-0.0400801603206413\\
4.00000000001101	-0.0561122244488979\\
4.00000000001137	-0.0721442885771544\\
4.00000000001173	-0.0881763527054109\\
4.0000000000121	-0.104208416833667\\
4.00000000001248	-0.120240480961924\\
4.00000000001288	-0.13627254509018\\
4.00000000001328	-0.152304609218437\\
4.0000000000137	-0.168336673346694\\
4.00000000001412	-0.18436873747495\\
4.00000000001456	-0.200400801603207\\
4.00000000001501	-0.216432865731463\\
4.00000000001547	-0.232464929859719\\
4.00000000001594	-0.248496993987976\\
4.00000000001643	-0.264529058116232\\
4.00000000001693	-0.280561122244489\\
4.00000000001744	-0.296593186372745\\
4.00000000001796	-0.312625250501002\\
4.0000000000185	-0.328657314629258\\
4.00000000001905	-0.344689378757515\\
4.00000000001962	-0.360721442885771\\
4.0000000000202	-0.376753507014028\\
4.00000000002079	-0.392785571142285\\
4.0000000000214	-0.408817635270541\\
4.00000000002202	-0.424849699398798\\
4.00000000002266	-0.440881763527054\\
4.00000000002332	-0.456913827655311\\
4.00000000002399	-0.472945891783567\\
4.00000000002467	-0.488977955911824\\
4.00000000002538	-0.50501002004008\\
4.0000000000261	-0.521042084168337\\
4.00000000002683	-0.537074148296593\\
4.00000000002759	-0.55310621242485\\
4.00000000002836	-0.569138276553106\\
4.00000000002914	-0.585170340681363\\
4.00000000002995	-0.601202404809619\\
4.00000000003078	-0.617234468937876\\
4.00000000003162	-0.633266533066132\\
4.00000000003248	-0.649298597194389\\
4.00000000003337	-0.665330661322646\\
4.00000000003427	-0.681362725450902\\
4.00000000003519	-0.697394789579158\\
4.00000000003613	-0.713426853707415\\
4.00000000003709	-0.729458917835671\\
4.00000000003808	-0.745490981963928\\
4.00000000003908	-0.761523046092184\\
4.00000000004011	-0.777555110220441\\
4.00000000004115	-0.793587174348697\\
4.00000000004222	-0.809619238476954\\
4.00000000004331	-0.82565130260521\\
4.00000000004442	-0.841683366733467\\
4.00000000004556	-0.857715430861723\\
4.00000000004672	-0.87374749498998\\
4.0000000000479	-0.889779559118236\\
4.00000000004911	-0.905811623246493\\
4.00000000005034	-0.92184368737475\\
4.00000000005159	-0.937875751503006\\
4.00000000005287	-0.953907815631263\\
4.00000000005418	-0.969939879759519\\
4.0000000000555	-0.985971943887776\\
4.00000000005686	-1.00200400801603\\
4.00000000005824	-1.01803607214429\\
4.00000000005964	-1.03406813627255\\
4.00000000006107	-1.0501002004008\\
4.00000000006253	-1.06613226452906\\
4.00000000006401	-1.08216432865731\\
4.00000000006552	-1.09819639278557\\
4.00000000006706	-1.11422845691383\\
4.00000000006863	-1.13026052104208\\
4.00000000007022	-1.14629258517034\\
4.00000000007184	-1.1623246492986\\
4.00000000007349	-1.17835671342685\\
4.00000000007516	-1.19438877755511\\
4.00000000007687	-1.21042084168337\\
4.0000000000786	-1.22645290581162\\
4.00000000008036	-1.24248496993988\\
4.00000000008216	-1.25851703406814\\
4.00000000008398	-1.27454909819639\\
4.00000000008583	-1.29058116232465\\
4.0000000000877	-1.30661322645291\\
4.00000000008961	-1.32264529058116\\
4.00000000009155	-1.33867735470942\\
4.00000000009352	-1.35470941883768\\
4.00000000009552	-1.37074148296593\\
4.00000000009755	-1.38677354709419\\
4.0000000000996	-1.40280561122244\\
4.00000000010169	-1.4188376753507\\
4.00000000010381	-1.43486973947896\\
4.00000000010596	-1.45090180360721\\
4.00000000010814	-1.46693386773547\\
4.00000000011035	-1.48296593186373\\
4.0000000001126	-1.49899799599198\\
4.00000000011487	-1.51503006012024\\
4.00000000011717	-1.5310621242485\\
4.00000000011951	-1.54709418837675\\
4.00000000012187	-1.56312625250501\\
4.00000000012427	-1.57915831663327\\
4.00000000012669	-1.59519038076152\\
4.00000000012915	-1.61122244488978\\
4.00000000013164	-1.62725450901804\\
4.00000000013416	-1.64328657314629\\
4.00000000013671	-1.65931863727455\\
4.00000000013929	-1.67535070140281\\
4.0000000001419	-1.69138276553106\\
4.00000000014454	-1.70741482965932\\
4.00000000014721	-1.72344689378758\\
4.00000000014991	-1.73947895791583\\
4.00000000015265	-1.75551102204409\\
4.00000000015541	-1.77154308617234\\
4.0000000001582	-1.7875751503006\\
4.00000000016102	-1.80360721442886\\
4.00000000016387	-1.81963927855711\\
4.00000000016675	-1.83567134268537\\
4.00000000016965	-1.85170340681363\\
4.00000000017259	-1.86773547094188\\
4.00000000017555	-1.88376753507014\\
4.00000000017854	-1.8997995991984\\
4.00000000018156	-1.91583166332665\\
4.00000000018461	-1.93186372745491\\
4.00000000018768	-1.94789579158317\\
4.00000000019079	-1.96392785571142\\
4.00000000019391	-1.97995991983968\\
4.00000000019707	-1.99599198396794\\
4.00000000020024	-2.01202404809619\\
4.00000000020345	-2.02805611222445\\
4.00000000020668	-2.04408817635271\\
4.00000000020993	-2.06012024048096\\
4.0000000002132	-2.07615230460922\\
4.0000000002165	-2.09218436873747\\
4.00000000021983	-2.10821643286573\\
4.00000000022317	-2.12424849699399\\
4.00000000022654	-2.14028056112224\\
4.00000000022992	-2.1563126252505\\
4.00000000023333	-2.17234468937876\\
4.00000000023676	-2.18837675350701\\
4.00000000024021	-2.20440881763527\\
4.00000000024368	-2.22044088176353\\
4.00000000024716	-2.23647294589178\\
4.00000000025066	-2.25250501002004\\
4.00000000025418	-2.2685370741483\\
4.00000000025772	-2.28456913827655\\
4.00000000026127	-2.30060120240481\\
4.00000000026484	-2.31663326653307\\
4.00000000026842	-2.33266533066132\\
4.00000000027201	-2.34869739478958\\
4.00000000027562	-2.36472945891784\\
4.00000000027924	-2.38076152304609\\
4.00000000028287	-2.39679358717435\\
4.00000000028651	-2.41282565130261\\
4.00000000029016	-2.42885771543086\\
4.00000000029382	-2.44488977955912\\
4.00000000029748	-2.46092184368737\\
4.00000000030116	-2.47695390781563\\
4.00000000030484	-2.49298597194389\\
4.00000000030852	-2.50901803607214\\
4.00000000031221	-2.5250501002004\\
4.0000000003159	-2.54108216432866\\
4.0000000003196	-2.55711422845691\\
4.0000000003233	-2.57314629258517\\
4.00000000032699	-2.58917835671343\\
4.00000000033069	-2.60521042084168\\
4.00000000033439	-2.62124248496994\\
4.00000000033808	-2.6372745490982\\
4.00000000034177	-2.65330661322645\\
4.00000000034546	-2.66933867735471\\
4.00000000034914	-2.68537074148297\\
4.00000000035282	-2.70140280561122\\
4.00000000035649	-2.71743486973948\\
4.00000000036015	-2.73346693386774\\
4.0000000003638	-2.74949899799599\\
4.00000000036744	-2.76553106212425\\
4.00000000037107	-2.7815631262525\\
4.00000000037469	-2.79759519038076\\
4.0000000003783	-2.81362725450902\\
4.00000000038189	-2.82965931863727\\
4.00000000038546	-2.84569138276553\\
4.00000000038902	-2.86172344689379\\
4.00000000039256	-2.87775551102204\\
4.00000000039608	-2.8937875751503\\
4.00000000039958	-2.90981963927856\\
4.00000000040307	-2.92585170340681\\
4.00000000040652	-2.94188376753507\\
4.00000000040996	-2.95791583166333\\
4.00000000041337	-2.97394789579158\\
4.00000000041676	-2.98997995991984\\
4.00000000042012	-3.0060120240481\\
4.00000000042346	-3.02204408817635\\
4.00000000042676	-3.03807615230461\\
4.00000000043004	-3.05410821643287\\
4.00000000043328	-3.07014028056112\\
4.0000000004365	-3.08617234468938\\
4.00000000043968	-3.10220440881764\\
4.00000000044282	-3.11823647294589\\
4.00000000044594	-3.13426853707415\\
4.00000000044901	-3.1503006012024\\
4.00000000045205	-3.16633266533066\\
4.00000000045505	-3.18236472945892\\
4.00000000045802	-3.19839679358717\\
4.00000000046094	-3.21442885771543\\
4.00000000046382	-3.23046092184369\\
4.00000000046666	-3.24649298597194\\
4.00000000046946	-3.2625250501002\\
4.00000000047221	-3.27855711422846\\
4.00000000047492	-3.29458917835671\\
4.00000000047758	-3.31062124248497\\
4.0000000004802	-3.32665330661323\\
4.00000000048277	-3.34268537074148\\
4.00000000048529	-3.35871743486974\\
4.00000000048776	-3.374749498998\\
4.00000000049018	-3.39078156312625\\
4.00000000049254	-3.40681362725451\\
4.00000000049486	-3.42284569138277\\
4.00000000049712	-3.43887775551102\\
4.00000000049933	-3.45490981963928\\
4.00000000050148	-3.47094188376753\\
4.00000000050358	-3.48697394789579\\
4.00000000050563	-3.50300601202405\\
4.00000000050761	-3.5190380761523\\
4.00000000050954	-3.53507014028056\\
4.00000000051141	-3.55110220440882\\
4.00000000051322	-3.56713426853707\\
4.00000000051497	-3.58316633266533\\
4.00000000051666	-3.59919839679359\\
4.00000000051829	-3.61523046092184\\
4.00000000051986	-3.6312625250501\\
4.00000000052136	-3.64729458917836\\
4.00000000052281	-3.66332665330661\\
4.00000000052419	-3.67935871743487\\
4.0000000005255	-3.69539078156313\\
4.00000000052675	-3.71142284569138\\
4.00000000052794	-3.72745490981964\\
4.00000000052906	-3.7434869739479\\
4.00000000053011	-3.75951903807615\\
4.0000000005311	-3.77555110220441\\
4.00000000053202	-3.79158316633267\\
4.00000000053288	-3.80761523046092\\
4.00000000053367	-3.82364729458918\\
4.00000000053439	-3.83967935871743\\
4.00000000053504	-3.85571142284569\\
4.00000000053563	-3.87174348697395\\
4.00000000053614	-3.8877755511022\\
4.00000000053659	-3.90380761523046\\
4.00000000053697	-3.91983967935872\\
4.00000000053728	-3.93587174348697\\
4.00000000053752	-3.95190380761523\\
4.00000000053769	-3.96793587174349\\
4.0000000005378	-3.98396793587174\\
4.00000000053783	-4\\
4	-4.00000000053783\\
3.98396793587174	-4.00000000057334\\
3.96793587174349	-4.00000000061096\\
3.95190380761523	-4.0000000006508\\
3.93587174348697	-4.00000000069296\\
3.91983967935872	-4.00000000073758\\
3.90380761523046	-4.00000000078476\\
3.8877755511022	-4.00000000083464\\
3.87174348697395	-4.00000000088735\\
3.85571142284569	-4.00000000094302\\
3.83967935871743	-4.0000000010018\\
3.82364729458918	-4.00000000106383\\
3.80761523046092	-4.00000000112927\\
3.79158316633267	-4.00000000119827\\
3.77555110220441	-4.000000001271\\
3.75951903807615	-4.00000000134762\\
3.7434869739479	-4.00000000142832\\
3.72745490981964	-4.00000000151325\\
3.71142284569138	-4.00000000160263\\
3.69539078156313	-4.00000000169662\\
3.67935871743487	-4.00000000179544\\
3.66332665330661	-4.00000000189928\\
3.64729458917836	-4.00000000200835\\
3.6312625250501	-4.00000000212287\\
3.61523046092184	-4.00000000224305\\
3.59919839679359	-4.00000000236912\\
3.58316633266533	-4.00000000250131\\
3.56713426853707	-4.00000000263986\\
3.55110220440882	-4.00000000278501\\
3.53507014028056	-4.00000000293701\\
3.51903807615231	-4.00000000309611\\
3.50300601202405	-4.00000000326257\\
3.48697394789579	-4.00000000343666\\
3.47094188376754	-4.00000000361864\\
3.45490981963928	-4.00000000380879\\
3.43887775551102	-4.00000000400738\\
3.42284569138277	-4.00000000421471\\
3.40681362725451	-4.00000000443105\\
3.39078156312625	-4.0000000046567\\
3.374749498998	-4.00000000489195\\
3.35871743486974	-4.00000000513711\\
3.34268537074148	-4.00000000539247\\
3.32665330661323	-4.00000000565835\\
3.31062124248497	-4.00000000593505\\
3.29458917835671	-4.00000000622288\\
3.27855711422846	-4.00000000652215\\
3.2625250501002	-4.00000000683318\\
3.24649298597194	-4.00000000715629\\
3.23046092184369	-4.00000000749178\\
3.21442885771543	-4.00000000783997\\
3.19839679358717	-4.00000000820119\\
3.18236472945892	-4.00000000857575\\
3.16633266533066	-4.00000000896395\\
3.1503006012024	-4.00000000936611\\
3.13426853707415	-4.00000000978255\\
3.11823647294589	-4.00000001021356\\
3.10220440881764	-4.00000001065945\\
3.08617234468938	-4.00000001112052\\
3.07014028056112	-4.00000001159706\\
3.05410821643287	-4.00000001208936\\
3.03807615230461	-4.0000000125977\\
3.02204408817635	-4.00000001312236\\
3.0060120240481	-4.0000000136636\\
2.98997995991984	-4.00000001422167\\
2.97394789579158	-4.00000001479684\\
2.95791583166333	-4.00000001538933\\
2.94188376753507	-4.00000001599937\\
2.92585170340681	-4.00000001662719\\
2.90981963927856	-4.00000001727298\\
2.8937875751503	-4.00000001793694\\
2.87775551102204	-4.00000001861924\\
2.86172344689379	-4.00000001932004\\
2.84569138276553	-4.00000002003949\\
2.82965931863727	-4.00000002077772\\
2.81362725450902	-4.00000002153485\\
2.79759519038076	-4.00000002231096\\
2.7815631262525	-4.00000002310613\\
2.76553106212425	-4.00000002392041\\
2.74949899799599	-4.00000002475385\\
2.73346693386774	-4.00000002560645\\
2.71743486973948	-4.0000000264782\\
2.70140280561122	-4.00000002736909\\
2.68537074148297	-4.00000002827904\\
2.66933867735471	-4.00000002920798\\
2.65330661322645	-4.0000000301558\\
2.6372745490982	-4.00000003112239\\
2.62124248496994	-4.00000003210757\\
2.60521042084168	-4.00000003311117\\
2.58917835671343	-4.00000003413298\\
2.57314629258517	-4.00000003517276\\
2.55711422845691	-4.00000003623025\\
2.54108216432866	-4.00000003730514\\
2.5250501002004	-4.00000003839712\\
2.50901803607214	-4.00000003950582\\
2.49298597194389	-4.00000004063087\\
2.47695390781563	-4.00000004177185\\
2.46092184368737	-4.00000004292832\\
2.44488977955912	-4.0000000440998\\
2.42885771543086	-4.00000004528579\\
2.41282565130261	-4.00000004648575\\
2.39679358717435	-4.0000000476991\\
2.38076152304609	-4.00000004892527\\
2.36472945891784	-4.0000000501636\\
2.34869739478958	-4.00000005141346\\
2.33266533066132	-4.00000005267415\\
2.31663326653307	-4.00000005394494\\
2.30060120240481	-4.0000000552251\\
2.28456913827655	-4.00000005651385\\
2.2685370741483	-4.00000005781037\\
2.25250501002004	-4.00000005911385\\
2.23647294589178	-4.00000006042342\\
2.22044088176353	-4.00000006173819\\
2.20440881763527	-4.00000006305725\\
2.18837675350701	-4.00000006437967\\
2.17234468937876	-4.00000006570449\\
2.1563126252505	-4.00000006703072\\
2.14028056112224	-4.00000006835736\\
2.12424849699399	-4.00000006968339\\
2.10821643286573	-4.00000007100775\\
2.09218436873747	-4.0000000723294\\
2.07615230460922	-4.00000007364725\\
2.06012024048096	-4.0000000749602\\
2.04408817635271	-4.00000007626715\\
2.02805611222445	-4.00000007756697\\
2.01202404809619	-4.00000007885854\\
1.99599198396794	-4.00000008014071\\
1.97995991983968	-4.00000008141234\\
1.96392785571142	-4.00000008267226\\
1.94789579158317	-4.00000008391932\\
1.93186372745491	-4.00000008515236\\
1.91583166332665	-4.0000000863702\\
1.8997995991984	-4.0000000875717\\
1.88376753507014	-4.00000008875568\\
1.86773547094188	-4.000000089921\\
1.85170340681363	-4.0000000910665\\
1.83567134268537	-4.00000009219105\\
1.81963927855711	-4.0000000932935\\
1.80360721442886	-4.00000009437275\\
1.7875751503006	-4.00000009542768\\
1.77154308617235	-4.00000009645721\\
1.75551102204409	-4.00000009746027\\
1.73947895791583	-4.0000000984358\\
1.72344689378758	-4.00000009938277\\
1.70741482965932	-4.00000010030017\\
1.69138276553106	-4.00000010118702\\
1.67535070140281	-4.00000010204237\\
1.65931863727455	-4.00000010286528\\
1.64328657314629	-4.00000010365485\\
1.62725450901804	-4.00000010441023\\
1.61122244488978	-4.00000010513056\\
1.59519038076152	-4.00000010581507\\
1.57915831663327	-4.00000010646298\\
1.56312625250501	-4.00000010707356\\
1.54709418837675	-4.00000010764614\\
1.5310621242485	-4.00000010818007\\
1.51503006012024	-4.00000010867473\\
1.49899799599198	-4.00000010912958\\
1.48296593186373	-4.00000010954409\\
1.46693386773547	-4.00000010991779\\
1.45090180360721	-4.00000011025025\\
1.43486973947896	-4.00000011054108\\
1.4188376753507	-4.00000011078997\\
1.40280561122244	-4.00000011099661\\
1.38677354709419	-4.00000011116077\\
1.37074148296593	-4.00000011128227\\
1.35470941883768	-4.00000011136095\\
1.33867735470942	-4.00000011139673\\
1.32264529058116	-4.00000011138958\\
1.30661322645291	-4.00000011133948\\
1.29058116232465	-4.00000011124652\\
1.27454909819639	-4.00000011111078\\
1.25851703406814	-4.00000011093244\\
1.24248496993988	-4.00000011071169\\
1.22645290581162	-4.00000011044878\\
1.21042084168337	-4.00000011014403\\
1.19438877755511	-4.00000010979778\\
1.17835671342685	-4.00000010941043\\
1.1623246492986	-4.00000010898242\\
1.14629258517034	-4.00000010851424\\
1.13026052104208	-4.00000010800642\\
1.11422845691383	-4.00000010745955\\
1.09819639278557	-4.00000010687422\\
1.08216432865731	-4.00000010625112\\
1.06613226452906	-4.00000010559093\\
1.0501002004008	-4.00000010489439\\
1.03406813627254	-4.00000010416229\\
1.01803607214429	-4.00000010339542\\
1.00200400801603	-4.00000010259463\\
0.985971943887775	-4.00000010176081\\
0.969939879759519	-4.00000010089486\\
0.953907815631262	-4.00000009999772\\
0.937875751503006	-4.00000009907035\\
0.921843687374749	-4.00000009811374\\
0.905811623246493	-4.00000009712892\\
0.889779559118236	-4.00000009611692\\
0.87374749498998	-4.00000009507881\\
0.857715430861724	-4.00000009401565\\
0.841683366733467	-4.00000009292854\\
0.825651302605211	-4.0000000918186\\
0.809619238476954	-4.00000009068694\\
0.793587174348698	-4.00000008953471\\
0.777555110220441	-4.00000008836304\\
0.761523046092185	-4.00000008717309\\
0.745490981963928	-4.00000008596601\\
0.729458917835672	-4.00000008474298\\
0.713426853707415	-4.00000008350514\\
0.697394789579159	-4.00000008225366\\
0.681362725450902	-4.00000008098971\\
0.665330661322646	-4.00000007971444\\
0.649298597194389	-4.00000007842901\\
0.633266533066132	-4.00000007713456\\
0.617234468937876	-4.00000007583223\\
0.601202404809619	-4.00000007452316\\
0.585170340681363	-4.00000007320845\\
0.569138276553106	-4.00000007188922\\
0.55310621242485	-4.00000007056655\\
0.537074148296593	-4.00000006924151\\
0.521042084168337	-4.00000006791516\\
0.50501002004008	-4.00000006658855\\
0.488977955911824	-4.00000006526268\\
0.472945891783567	-4.00000006393855\\
0.456913827655311	-4.00000006261714\\
0.440881763527054	-4.00000006129941\\
0.424849699398798	-4.00000005998627\\
0.408817635270541	-4.00000005867864\\
0.392785571142285	-4.00000005737738\\
0.376753507014028	-4.00000005608336\\
0.360721442885771	-4.00000005479739\\
0.344689378757515	-4.00000005352027\\
0.328657314629258	-4.00000005225276\\
0.312625250501002	-4.00000005099561\\
0.296593186372745	-4.00000004974951\\
0.280561122244489	-4.00000004851516\\
0.264529058116232	-4.0000000472932\\
0.248496993987976	-4.00000004608424\\
0.232464929859719	-4.00000004488888\\
0.216432865731463	-4.00000004370767\\
0.200400801603206	-4.00000004254114\\
0.18436873747495	-4.00000004138978\\
0.168336673346693	-4.00000004025406\\
0.152304609218437	-4.00000003913442\\
0.13627254509018	-4.00000003803125\\
0.120240480961924	-4.00000003694493\\
0.104208416833667	-4.0000000358758\\
0.0881763527054105	-4.00000003482419\\
0.0721442885771539	-4.00000003379037\\
0.0561122244488974	-4.00000003277461\\
0.0400801603206409	-4.00000003177712\\
0.0240480961923843	-4.00000003079812\\
0.00801603206412782	-4.00000002983777\\
-0.00801603206412826	-4.00000002889623\\
-0.0240480961923848	-4.0000000279736\\
-0.0400801603206413	-4.00000002707\\
-0.0561122244488979	-4.00000002618549\\
-0.0721442885771544	-4.00000002532012\\
-0.0881763527054109	-4.00000002447391\\
-0.104208416833667	-4.00000002364686\\
-0.120240480961924	-4.00000002283895\\
-0.13627254509018	-4.00000002205014\\
-0.152304609218437	-4.00000002128037\\
-0.168336673346694	-4.00000002052955\\
-0.18436873747495	-4.00000001979759\\
-0.200400801603207	-4.00000001908437\\
-0.216432865731463	-4.00000001838976\\
-0.232464929859719	-4.00000001771359\\
-0.248496993987976	-4.00000001705571\\
-0.264529058116232	-4.00000001641593\\
-0.280561122244489	-4.00000001579406\\
-0.296593186372745	-4.00000001518989\\
-0.312625250501002	-4.0000000146032\\
-0.328657314629258	-4.00000001403376\\
-0.344689378757515	-4.00000001348132\\
-0.360721442885771	-4.00000001294564\\
-0.376753507014028	-4.00000001242646\\
-0.392785571142285	-4.00000001192349\\
-0.408817635270541	-4.00000001143648\\
-0.424849699398798	-4.00000001096513\\
-0.440881763527054	-4.00000001050915\\
-0.456913827655311	-4.00000001006825\\
-0.472945891783567	-4.00000000964213\\
-0.488977955911824	-4.00000000923049\\
-0.50501002004008	-4.00000000883301\\
-0.521042084168337	-4.00000000844939\\
-0.537074148296593	-4.00000000807932\\
-0.55310621242485	-4.00000000772248\\
-0.569138276553106	-4.00000000737855\\
-0.585170340681363	-4.00000000704722\\
-0.601202404809619	-4.00000000672818\\
-0.617234468937876	-4.0000000064211\\
-0.633266533066132	-4.00000000612568\\
-0.649298597194389	-4.0000000058416\\
-0.665330661322646	-4.00000000556854\\
-0.681362725450902	-4.0000000053062\\
-0.697394789579158	-4.00000000505427\\
-0.713426853707415	-4.00000000481245\\
-0.729458917835671	-4.00000000458043\\
-0.745490981963928	-4.00000000435791\\
-0.761523046092184	-4.00000000414461\\
-0.777555110220441	-4.00000000394023\\
-0.793587174348697	-4.00000000374448\\
-0.809619238476954	-4.00000000355709\\
-0.82565130260521	-4.00000000337777\\
-0.841683366733467	-4.00000000320625\\
-0.857715430861723	-4.00000000304227\\
-0.87374749498998	-4.00000000288557\\
-0.889779559118236	-4.00000000273588\\
-0.905811623246493	-4.00000000259296\\
-0.92184368737475	-4.00000000245655\\
-0.937875751503006	-4.00000000232643\\
-0.953907815631263	-4.00000000220234\\
-0.969939879759519	-4.00000000208408\\
-0.985971943887776	-4.0000000019714\\
-1.00200400801603	-4.0000000018641\\
-1.01803607214429	-4.00000000176195\\
-1.03406813627255	-4.00000000166477\\
-1.0501002004008	-4.00000000157233\\
-1.06613226452906	-4.00000000148446\\
-1.08216432865731	-4.00000000140096\\
-1.09819639278557	-4.00000000132164\\
-1.11422845691383	-4.00000000124633\\
-1.13026052104208	-4.00000000117487\\
-1.14629258517034	-4.00000000110707\\
-1.1623246492986	-4.00000000104278\\
-1.17835671342685	-4.00000000098185\\
-1.19438877755511	-4.00000000092412\\
-1.21042084168337	-4.00000000086946\\
-1.22645290581162	-4.00000000081771\\
-1.24248496993988	-4.00000000076874\\
-1.25851703406814	-4.00000000072243\\
-1.27454909819639	-4.00000000067864\\
-1.29058116232465	-4.00000000063727\\
-1.30661322645291	-4.00000000059818\\
-1.32264529058116	-4.00000000056128\\
-1.33867735470942	-4.00000000052645\\
-1.35470941883768	-4.00000000049359\\
-1.37074148296593	-4.0000000004626\\
-1.38677354709419	-4.00000000043339\\
-1.40280561122244	-4.00000000040587\\
-1.4188376753507	-4.00000000037995\\
-1.43486973947896	-4.00000000035555\\
-1.45090180360721	-4.00000000033259\\
-1.46693386773547	-4.00000000031099\\
-1.48296593186373	-4.00000000029068\\
-1.49899799599198	-4.00000000027159\\
-1.51503006012024	-4.00000000025366\\
-1.5310621242485	-4.00000000023682\\
-1.54709418837675	-4.00000000022102\\
-1.56312625250501	-4.00000000020618\\
-1.57915831663327	-4.00000000019227\\
-1.59519038076152	-4.00000000017923\\
-1.61122244488978	-4.00000000016701\\
-1.62725450901804	-4.00000000015557\\
-1.64328657314629	-4.00000000014485\\
-1.65931863727455	-4.00000000013482\\
-1.67535070140281	-4.00000000012543\\
-1.69138276553106	-4.00000000011665\\
-1.70741482965932	-4.00000000010845\\
-1.72344689378758	-4.00000000010078\\
-1.73947895791583	-4.00000000009362\\
-1.75551102204409	-4.00000000008693\\
-1.77154308617234	-4.0000000000807\\
-1.7875751503006	-4.00000000007488\\
-1.80360721442886	-4.00000000006945\\
-1.81963927855711	-4.00000000006439\\
-1.83567134268537	-4.00000000005968\\
-1.85170340681363	-4.00000000005529\\
-1.86773547094188	-4.0000000000512\\
-1.88376753507014	-4.0000000000474\\
-1.8997995991984	-4.00000000004386\\
-1.91583166332665	-4.00000000004057\\
-1.93186372745491	-4.00000000003751\\
-1.94789579158317	-4.00000000003468\\
-1.96392785571142	-4.00000000003204\\
-1.97995991983968	-4.00000000002959\\
-1.99599198396794	-4.00000000002732\\
-2.01202404809619	-4.00000000002521\\
-2.02805611222445	-4.00000000002326\\
-2.04408817635271	-4.00000000002145\\
-2.06012024048096	-4.00000000001977\\
-2.07615230460922	-4.00000000001822\\
-2.09218436873747	-4.00000000001678\\
-2.10821643286573	-4.00000000001545\\
-2.12424849699399	-4.00000000001422\\
-2.14028056112224	-4.00000000001308\\
-2.1563126252505	-4.00000000001203\\
-2.17234468937876	-4.00000000001106\\
-2.18837675350701	-4.00000000001017\\
-2.20440881763527	-4.00000000000934\\
-2.22044088176353	-4.00000000000858\\
-2.23647294589178	-4.00000000000787\\
-2.25250501002004	-4.00000000000722\\
-2.2685370741483	-4.00000000000662\\
-2.28456913827655	-4.00000000000607\\
-2.30060120240481	-4.00000000000557\\
-2.31663326653307	-4.0000000000051\\
-2.33266533066132	-4.00000000000467\\
-2.34869739478958	-4.00000000000428\\
-2.36472945891784	-4.00000000000391\\
-2.38076152304609	-4.00000000000358\\
-2.39679358717435	-4.00000000000327\\
-2.41282565130261	-4.00000000000299\\
-2.42885771543086	-4.00000000000273\\
-2.44488977955912	-4.0000000000025\\
-2.46092184368737	-4.00000000000228\\
-2.47695390781563	-4.00000000000208\\
-2.49298597194389	-4.0000000000019\\
-2.50901803607214	-4.00000000000173\\
-2.5250501002004	-4.00000000000158\\
-2.54108216432866	-4.00000000000144\\
-2.55711422845691	-4.00000000000131\\
-2.57314629258517	-4.00000000000119\\
-2.58917835671343	-4.00000000000108\\
-2.60521042084168	-4.00000000000099\\
-2.62124248496994	-4.0000000000009\\
-2.6372745490982	-4.00000000000082\\
-2.65330661322645	-4.00000000000074\\
-2.66933867735471	-4.00000000000067\\
-2.68537074148297	-4.00000000000061\\
-2.70140280561122	-4.00000000000056\\
-2.71743486973948	-4.0000000000005\\
-2.73346693386774	-4.00000000000046\\
-2.74949899799599	-4.00000000000041\\
-2.76553106212425	-4.00000000000038\\
-2.7815631262525	-4.00000000000034\\
-2.79759519038076	-4.00000000000031\\
-2.81362725450902	-4.00000000000028\\
-2.82965931863727	-4.00000000000025\\
-2.84569138276553	-4.00000000000023\\
-2.86172344689379	-4.00000000000021\\
-2.87775551102204	-4.00000000000019\\
-2.8937875751503	-4.00000000000017\\
-2.90981963927856	-4.00000000000015\\
-2.92585170340681	-4.00000000000014\\
-2.94188376753507	-4.00000000000012\\
-2.95791583166333	-4.00000000000011\\
-2.97394789579158	-4.0000000000001\\
-2.98997995991984	-4.00000000000009\\
-3.0060120240481	-4.00000000000008\\
-3.02204408817635	-4.00000000000007\\
-3.03807615230461	-4.00000000000007\\
-3.05410821643287	-4.00000000000006\\
-3.07014028056112	-4.00000000000005\\
-3.08617234468938	-4.00000000000005\\
-3.10220440881764	-4.00000000000004\\
-3.11823647294589	-4.00000000000004\\
-3.13426853707415	-4.00000000000003\\
-3.1503006012024	-4.00000000000003\\
-3.16633266533066	-4.00000000000003\\
-3.18236472945892	-4.00000000000003\\
-3.19839679358717	-4.00000000000002\\
-3.21442885771543	-4.00000000000002\\
-3.23046092184369	-4.00000000000002\\
-3.24649298597194	-4.00000000000002\\
-3.2625250501002	-4.00000000000001\\
-3.27855711422846	-4.00000000000001\\
-3.29458917835671	-4.00000000000001\\
-3.31062124248497	-4.00000000000001\\
-3.32665330661323	-4.00000000000001\\
-3.34268537074148	-4.00000000000001\\
-3.35871743486974	-4.00000000000001\\
-3.374749498998	-4.00000000000001\\
-3.39078156312625	-4.00000000000001\\
-3.40681362725451	-4.00000000000001\\
-3.42284569138277	-4\\
-3.43887775551102	-4\\
-3.45490981963928	-4\\
-3.47094188376753	-4\\
-3.48697394789579	-4\\
-3.50300601202405	-4\\
-3.5190380761523	-4\\
-3.53507014028056	-4\\
-3.55110220440882	-4\\
-3.56713426853707	-4\\
-3.58316633266533	-4\\
-3.59919839679359	-4\\
-3.61523046092184	-4\\
-3.6312625250501	-4\\
-3.64729458917836	-4\\
-3.66332665330661	-4\\
-3.67935871743487	-4\\
-3.69539078156313	-4\\
-3.71142284569138	-4\\
-3.72745490981964	-4\\
-3.7434869739479	-4\\
-3.75951903807615	-4\\
-3.77555110220441	-4\\
-3.79158316633267	-4\\
-3.80761523046092	-4\\
-3.82364729458918	-4\\
-3.83967935871743	-4\\
-3.85571142284569	-4\\
-3.87174348697395	-4\\
-3.8877755511022	-4\\
-3.90380761523046	-4\\
-3.91983967935872	-4\\
-3.93587174348697	-4\\
-3.95190380761523	-4\\
-3.96793587174349	-4\\
-3.98396793587174	-4\\
-4	-4\\
-4	-3.98396793587174\\
-4	-3.96793587174349\\
-4	-3.95190380761523\\
-4	-3.93587174348697\\
-4	-3.91983967935872\\
-4	-3.90380761523046\\
-4	-3.8877755511022\\
-4	-3.87174348697395\\
-4	-3.85571142284569\\
-4	-3.83967935871743\\
-4	-3.82364729458918\\
-4	-3.80761523046092\\
-4	-3.79158316633267\\
-4	-3.77555110220441\\
-4	-3.75951903807615\\
-4	-3.7434869739479\\
-4	-3.72745490981964\\
-4	-3.71142284569138\\
-4	-3.69539078156313\\
-4	-3.67935871743487\\
-4	-3.66332665330661\\
-4	-3.64729458917836\\
-4	-3.6312625250501\\
-4	-3.61523046092184\\
-4	-3.59919839679359\\
-4	-3.58316633266533\\
-4	-3.56713426853707\\
-4	-3.55110220440882\\
-4	-3.53507014028056\\
-4	-3.5190380761523\\
-4	-3.50300601202405\\
-4	-3.48697394789579\\
-4	-3.47094188376753\\
-4	-3.45490981963928\\
-4	-3.43887775551102\\
-4	-3.42284569138277\\
-4	-3.40681362725451\\
-4	-3.39078156312625\\
-4	-3.374749498998\\
-4	-3.35871743486974\\
-4	-3.34268537074148\\
-4	-3.32665330661323\\
-4	-3.31062124248497\\
-4	-3.29458917835671\\
-4	-3.27855711422846\\
-4	-3.2625250501002\\
-4	-3.24649298597194\\
-4	-3.23046092184369\\
-4	-3.21442885771543\\
-4	-3.19839679358717\\
-4	-3.18236472945892\\
-4	-3.16633266533066\\
-4	-3.1503006012024\\
-4	-3.13426853707415\\
-4	-3.11823647294589\\
-4	-3.10220440881764\\
-4	-3.08617234468938\\
-4	-3.07014028056112\\
-4	-3.05410821643287\\
-4	-3.03807615230461\\
-4	-3.02204408817635\\
-4	-3.0060120240481\\
-4	-2.98997995991984\\
-4	-2.97394789579158\\
-4	-2.95791583166333\\
-4	-2.94188376753507\\
-4	-2.92585170340681\\
-4	-2.90981963927856\\
-4	-2.8937875751503\\
-4	-2.87775551102204\\
-4	-2.86172344689379\\
-4	-2.84569138276553\\
-4	-2.82965931863727\\
-4	-2.81362725450902\\
-4.00000000000001	-2.79759519038076\\
-4.00000000000001	-2.7815631262525\\
-4.00000000000001	-2.76553106212425\\
-4.00000000000001	-2.74949899799599\\
-4.00000000000001	-2.73346693386774\\
-4.00000000000001	-2.71743486973948\\
-4.00000000000001	-2.70140280561122\\
-4.00000000000001	-2.68537074148297\\
-4.00000000000001	-2.66933867735471\\
-4.00000000000001	-2.65330661322645\\
-4.00000000000001	-2.6372745490982\\
-4.00000000000001	-2.62124248496994\\
-4.00000000000001	-2.60521042084168\\
-4.00000000000001	-2.58917835671343\\
-4.00000000000001	-2.57314629258517\\
-4.00000000000001	-2.55711422845691\\
-4.00000000000001	-2.54108216432866\\
-4.00000000000001	-2.5250501002004\\
-4.00000000000001	-2.50901803607214\\
-4.00000000000001	-2.49298597194389\\
-4.00000000000002	-2.47695390781563\\
-4.00000000000002	-2.46092184368737\\
-4.00000000000002	-2.44488977955912\\
-4.00000000000002	-2.42885771543086\\
-4.00000000000002	-2.41282565130261\\
-4.00000000000002	-2.39679358717435\\
-4.00000000000002	-2.38076152304609\\
-4.00000000000002	-2.36472945891784\\
-4.00000000000002	-2.34869739478958\\
-4.00000000000002	-2.33266533066132\\
-4.00000000000002	-2.31663326653307\\
-4.00000000000003	-2.30060120240481\\
-4.00000000000003	-2.28456913827655\\
-4.00000000000003	-2.2685370741483\\
-4.00000000000003	-2.25250501002004\\
-4.00000000000003	-2.23647294589178\\
-4.00000000000003	-2.22044088176353\\
-4.00000000000004	-2.20440881763527\\
-4.00000000000004	-2.18837675350701\\
-4.00000000000004	-2.17234468937876\\
-4.00000000000004	-2.1563126252505\\
-4.00000000000004	-2.14028056112224\\
-4.00000000000005	-2.12424849699399\\
-4.00000000000005	-2.10821643286573\\
-4.00000000000005	-2.09218436873747\\
-4.00000000000005	-2.07615230460922\\
-4.00000000000006	-2.06012024048096\\
-4.00000000000006	-2.04408817635271\\
-4.00000000000006	-2.02805611222445\\
-4.00000000000006	-2.01202404809619\\
-4.00000000000007	-1.99599198396794\\
-4.00000000000007	-1.97995991983968\\
-4.00000000000007	-1.96392785571142\\
-4.00000000000008	-1.94789579158317\\
-4.00000000000008	-1.93186372745491\\
-4.00000000000009	-1.91583166332665\\
-4.00000000000009	-1.8997995991984\\
-4.00000000000009	-1.88376753507014\\
-4.0000000000001	-1.86773547094188\\
-4.0000000000001	-1.85170340681363\\
-4.00000000000011	-1.83567134268537\\
-4.00000000000011	-1.81963927855711\\
-4.00000000000012	-1.80360721442886\\
-4.00000000000012	-1.7875751503006\\
-4.00000000000013	-1.77154308617234\\
-4.00000000000014	-1.75551102204409\\
-4.00000000000014	-1.73947895791583\\
-4.00000000000015	-1.72344689378758\\
-4.00000000000016	-1.70741482965932\\
-4.00000000000016	-1.69138276553106\\
-4.00000000000017	-1.67535070140281\\
-4.00000000000018	-1.65931863727455\\
-4.00000000000019	-1.64328657314629\\
-4.0000000000002	-1.62725450901804\\
-4.00000000000021	-1.61122244488978\\
-4.00000000000021	-1.59519038076152\\
-4.00000000000022	-1.57915831663327\\
-4.00000000000023	-1.56312625250501\\
-4.00000000000025	-1.54709418837675\\
-4.00000000000026	-1.5310621242485\\
-4.00000000000027	-1.51503006012024\\
-4.00000000000028	-1.49899799599198\\
-4.00000000000029	-1.48296593186373\\
-4.00000000000031	-1.46693386773547\\
-4.00000000000032	-1.45090180360721\\
-4.00000000000033	-1.43486973947896\\
-4.00000000000035	-1.4188376753507\\
-4.00000000000036	-1.40280561122244\\
-4.00000000000038	-1.38677354709419\\
-4.0000000000004	-1.37074148296593\\
-4.00000000000041	-1.35470941883768\\
-4.00000000000043	-1.33867735470942\\
-4.00000000000045	-1.32264529058116\\
-4.00000000000047	-1.30661322645291\\
-4.00000000000049	-1.29058116232465\\
-4.00000000000051	-1.27454909819639\\
-4.00000000000053	-1.25851703406814\\
-4.00000000000056	-1.24248496993988\\
-4.00000000000058	-1.22645290581162\\
-4.00000000000061	-1.21042084168337\\
-4.00000000000063	-1.19438877755511\\
-4.00000000000066	-1.17835671342685\\
-4.00000000000069	-1.1623246492986\\
-4.00000000000072	-1.14629258517034\\
-4.00000000000075	-1.13026052104208\\
-4.00000000000078	-1.11422845691383\\
-4.00000000000081	-1.09819639278557\\
-4.00000000000084	-1.08216432865731\\
-4.00000000000088	-1.06613226452906\\
-4.00000000000092	-1.0501002004008\\
-4.00000000000095	-1.03406813627255\\
-4.00000000000099	-1.01803607214429\\
-4.00000000000103	-1.00200400801603\\
-4.00000000000108	-0.985971943887776\\
-4.00000000000112	-0.969939879759519\\
-4.00000000000116	-0.953907815631263\\
-4.00000000000121	-0.937875751503006\\
-4.00000000000126	-0.92184368737475\\
-4.00000000000131	-0.905811623246493\\
-4.00000000000136	-0.889779559118236\\
-4.00000000000142	-0.87374749498998\\
-4.00000000000147	-0.857715430861723\\
-4.00000000000153	-0.841683366733467\\
-4.00000000000159	-0.82565130260521\\
-4.00000000000166	-0.809619238476954\\
-4.00000000000172	-0.793587174348697\\
-4.00000000000179	-0.777555110220441\\
-4.00000000000186	-0.761523046092184\\
-4.00000000000193	-0.745490981963928\\
-4.000000000002	-0.729458917835671\\
-4.00000000000208	-0.713426853707415\\
-4.00000000000216	-0.697394789579158\\
-4.00000000000225	-0.681362725450902\\
-4.00000000000233	-0.665330661322646\\
-4.00000000000242	-0.649298597194389\\
-4.00000000000251	-0.633266533066132\\
-4.00000000000261	-0.617234468937876\\
-4.0000000000027	-0.601202404809619\\
-4.00000000000281	-0.585170340681363\\
-4.00000000000291	-0.569138276553106\\
-4.00000000000302	-0.55310621242485\\
-4.00000000000313	-0.537074148296593\\
-4.00000000000325	-0.521042084168337\\
-4.00000000000337	-0.50501002004008\\
-4.00000000000349	-0.488977955911824\\
-4.00000000000362	-0.472945891783567\\
-4.00000000000375	-0.456913827655311\\
-4.00000000000388	-0.440881763527054\\
-4.00000000000403	-0.424849699398798\\
-4.00000000000417	-0.408817635270541\\
-4.00000000000432	-0.392785571142285\\
-4.00000000000448	-0.376753507014028\\
-4.00000000000463	-0.360721442885771\\
-4.0000000000048	-0.344689378757515\\
-4.00000000000497	-0.328657314629258\\
-4.00000000000514	-0.312625250501002\\
-4.00000000000532	-0.296593186372745\\
-4.00000000000551	-0.280561122244489\\
-4.0000000000057	-0.264529058116232\\
-4.0000000000059	-0.248496993987976\\
-4.00000000000611	-0.232464929859719\\
-4.00000000000631	-0.216432865731463\\
-4.00000000000653	-0.200400801603207\\
-4.00000000000676	-0.18436873747495\\
-4.00000000000699	-0.168336673346694\\
-4.00000000000722	-0.152304609218437\\
-4.00000000000747	-0.13627254509018\\
-4.00000000000772	-0.120240480961924\\
-4.00000000000798	-0.104208416833667\\
-4.00000000000824	-0.0881763527054109\\
-4.00000000000852	-0.0721442885771544\\
-4.0000000000088	-0.0561122244488979\\
-4.00000000000909	-0.0400801603206413\\
-4.00000000000939	-0.0240480961923848\\
-4.00000000000969	-0.00801603206412826\\
-4.00000000001001	0.00801603206412782\\
-4.00000000001033	0.0240480961923843\\
-4.00000000001067	0.0400801603206409\\
-4.00000000001101	0.0561122244488974\\
-4.00000000001137	0.0721442885771539\\
-4.00000000001173	0.0881763527054105\\
-4.0000000000121	0.104208416833667\\
-4.00000000001248	0.120240480961924\\
-4.00000000001288	0.13627254509018\\
-4.00000000001328	0.152304609218437\\
-4.0000000000137	0.168336673346693\\
-4.00000000001412	0.18436873747495\\
-4.00000000001456	0.200400801603206\\
-4.00000000001501	0.216432865731463\\
-4.00000000001547	0.232464929859719\\
-4.00000000001594	0.248496993987976\\
-4.00000000001643	0.264529058116232\\
-4.00000000001693	0.280561122244489\\
-4.00000000001744	0.296593186372745\\
-4.00000000001796	0.312625250501002\\
-4.0000000000185	0.328657314629258\\
-4.00000000001905	0.344689378757515\\
-4.00000000001962	0.360721442885771\\
-4.0000000000202	0.376753507014028\\
-4.00000000002079	0.392785571142285\\
-4.0000000000214	0.408817635270541\\
-4.00000000002202	0.424849699398798\\
-4.00000000002266	0.440881763527054\\
-4.00000000002332	0.456913827655311\\
-4.00000000002399	0.472945891783567\\
-4.00000000002467	0.488977955911824\\
-4.00000000002538	0.50501002004008\\
-4.0000000000261	0.521042084168337\\
-4.00000000002683	0.537074148296593\\
-4.00000000002759	0.55310621242485\\
-4.00000000002836	0.569138276553106\\
-4.00000000002914	0.585170340681363\\
-4.00000000002995	0.601202404809619\\
-4.00000000003078	0.617234468937876\\
-4.00000000003162	0.633266533066132\\
-4.00000000003248	0.649298597194389\\
-4.00000000003337	0.665330661322646\\
-4.00000000003427	0.681362725450902\\
-4.00000000003519	0.697394789579159\\
-4.00000000003613	0.713426853707415\\
-4.00000000003709	0.729458917835672\\
-4.00000000003808	0.745490981963928\\
-4.00000000003908	0.761523046092185\\
-4.00000000004011	0.777555110220441\\
-4.00000000004115	0.793587174348698\\
-4.00000000004222	0.809619238476954\\
-4.00000000004331	0.825651302605211\\
-4.00000000004442	0.841683366733467\\
-4.00000000004556	0.857715430861724\\
-4.00000000004672	0.87374749498998\\
-4.0000000000479	0.889779559118236\\
-4.00000000004911	0.905811623246493\\
-4.00000000005034	0.921843687374749\\
-4.00000000005159	0.937875751503006\\
-4.00000000005287	0.953907815631262\\
-4.00000000005418	0.969939879759519\\
-4.0000000000555	0.985971943887775\\
-4.00000000005686	1.00200400801603\\
-4.00000000005824	1.01803607214429\\
-4.00000000005964	1.03406813627254\\
-4.00000000006107	1.0501002004008\\
-4.00000000006253	1.06613226452906\\
-4.00000000006401	1.08216432865731\\
-4.00000000006553	1.09819639278557\\
-4.00000000006706	1.11422845691383\\
-4.00000000006863	1.13026052104208\\
-4.00000000007022	1.14629258517034\\
-4.00000000007184	1.1623246492986\\
-4.00000000007349	1.17835671342685\\
-4.00000000007516	1.19438877755511\\
-4.00000000007687	1.21042084168337\\
-4.0000000000786	1.22645290581162\\
-4.00000000008037	1.24248496993988\\
-4.00000000008216	1.25851703406814\\
-4.00000000008398	1.27454909819639\\
-4.00000000008582	1.29058116232465\\
-4.0000000000877	1.30661322645291\\
-4.00000000008961	1.32264529058116\\
-4.00000000009155	1.33867735470942\\
-4.00000000009352	1.35470941883768\\
-4.00000000009552	1.37074148296593\\
-4.00000000009754	1.38677354709419\\
-4.0000000000996	1.40280561122244\\
-4.00000000010169	1.4188376753507\\
-4.00000000010381	1.43486973947896\\
-4.00000000010596	1.45090180360721\\
-4.00000000010814	1.46693386773547\\
-4.00000000011035	1.48296593186373\\
-4.00000000011259	1.49899799599198\\
-4.00000000011487	1.51503006012024\\
-4.00000000011717	1.5310621242485\\
-4.00000000011951	1.54709418837675\\
-4.00000000012187	1.56312625250501\\
-4.00000000012427	1.57915831663327\\
-4.00000000012669	1.59519038076152\\
-4.00000000012915	1.61122244488978\\
-4.00000000013164	1.62725450901804\\
-4.00000000013416	1.64328657314629\\
-4.00000000013671	1.65931863727455\\
-4.00000000013929	1.67535070140281\\
-4.0000000001419	1.69138276553106\\
-4.00000000014454	1.70741482965932\\
-4.00000000014721	1.72344689378758\\
-4.00000000014991	1.73947895791583\\
-4.00000000015265	1.75551102204409\\
-4.00000000015541	1.77154308617235\\
-4.0000000001582	1.7875751503006\\
-4.00000000016102	1.80360721442886\\
-4.00000000016387	1.81963927855711\\
-4.00000000016675	1.83567134268537\\
-4.00000000016965	1.85170340681363\\
-4.00000000017259	1.86773547094188\\
-4.00000000017555	1.88376753507014\\
-4.00000000017854	1.8997995991984\\
-4.00000000018156	1.91583166332665\\
-4.00000000018461	1.93186372745491\\
-4.00000000018769	1.94789579158317\\
-4.00000000019079	1.96392785571142\\
-4.00000000019391	1.97995991983968\\
-4.00000000019707	1.99599198396794\\
-4.00000000020024	2.01202404809619\\
-4.00000000020345	2.02805611222445\\
-4.00000000020668	2.04408817635271\\
-4.00000000020993	2.06012024048096\\
-4.0000000002132	2.07615230460922\\
-4.0000000002165	2.09218436873747\\
-4.00000000021983	2.10821643286573\\
-4.00000000022317	2.12424849699399\\
-4.00000000022654	2.14028056112224\\
-4.00000000022992	2.1563126252505\\
-4.00000000023333	2.17234468937876\\
-4.00000000023676	2.18837675350701\\
-4.00000000024021	2.20440881763527\\
-4.00000000024368	2.22044088176353\\
-4.00000000024716	2.23647294589178\\
-4.00000000025066	2.25250501002004\\
-4.00000000025418	2.2685370741483\\
-4.00000000025772	2.28456913827655\\
-4.00000000026127	2.30060120240481\\
-4.00000000026484	2.31663326653307\\
-4.00000000026842	2.33266533066132\\
-4.00000000027201	2.34869739478958\\
-4.00000000027562	2.36472945891784\\
-4.00000000027924	2.38076152304609\\
-4.00000000028287	2.39679358717435\\
-4.00000000028651	2.41282565130261\\
-4.00000000029016	2.42885771543086\\
-4.00000000029382	2.44488977955912\\
-4.00000000029748	2.46092184368737\\
-4.00000000030116	2.47695390781563\\
-4.00000000030484	2.49298597194389\\
-4.00000000030852	2.50901803607214\\
-4.00000000031221	2.5250501002004\\
-4.0000000003159	2.54108216432866\\
-4.0000000003196	2.55711422845691\\
-4.0000000003233	2.57314629258517\\
-4.00000000032699	2.58917835671343\\
-4.00000000033069	2.60521042084168\\
-4.00000000033439	2.62124248496994\\
-4.00000000033808	2.6372745490982\\
-4.00000000034177	2.65330661322645\\
-4.00000000034546	2.66933867735471\\
-4.00000000034914	2.68537074148297\\
-4.00000000035282	2.70140280561122\\
-4.00000000035649	2.71743486973948\\
-4.00000000036015	2.73346693386774\\
-4.0000000003638	2.74949899799599\\
-4.00000000036744	2.76553106212425\\
-4.00000000037107	2.7815631262525\\
-4.00000000037469	2.79759519038076\\
-4.0000000003783	2.81362725450902\\
-4.00000000038189	2.82965931863727\\
-4.00000000038546	2.84569138276553\\
-4.00000000038902	2.86172344689379\\
-4.00000000039256	2.87775551102204\\
-4.00000000039608	2.8937875751503\\
-4.00000000039958	2.90981963927856\\
-4.00000000040307	2.92585170340681\\
-4.00000000040653	2.94188376753507\\
-4.00000000040996	2.95791583166333\\
-4.00000000041337	2.97394789579158\\
-4.00000000041676	2.98997995991984\\
-4.00000000042012	3.0060120240481\\
-4.00000000042346	3.02204408817635\\
-4.00000000042676	3.03807615230461\\
-4.00000000043004	3.05410821643287\\
-4.00000000043328	3.07014028056112\\
-4.0000000004365	3.08617234468938\\
-4.00000000043968	3.10220440881764\\
-4.00000000044282	3.11823647294589\\
-4.00000000044594	3.13426853707415\\
-4.00000000044901	3.1503006012024\\
-4.00000000045205	3.16633266533066\\
-4.00000000045505	3.18236472945892\\
-4.00000000045802	3.19839679358717\\
-4.00000000046094	3.21442885771543\\
-4.00000000046382	3.23046092184369\\
-4.00000000046666	3.24649298597194\\
-4.00000000046946	3.2625250501002\\
-4.00000000047221	3.27855711422846\\
-4.00000000047492	3.29458917835671\\
-4.00000000047758	3.31062124248497\\
-4.0000000004802	3.32665330661323\\
-4.00000000048277	3.34268537074148\\
-4.00000000048529	3.35871743486974\\
-4.00000000048776	3.374749498998\\
-4.00000000049017	3.39078156312625\\
-4.00000000049254	3.40681362725451\\
-4.00000000049486	3.42284569138277\\
-4.00000000049712	3.43887775551102\\
-4.00000000049933	3.45490981963928\\
-4.00000000050148	3.47094188376754\\
-4.00000000050358	3.48697394789579\\
-4.00000000050563	3.50300601202405\\
-4.00000000050761	3.51903807615231\\
-4.00000000050954	3.53507014028056\\
-4.00000000051141	3.55110220440882\\
-4.00000000051322	3.56713426853707\\
-4.00000000051497	3.58316633266533\\
-4.00000000051666	3.59919839679359\\
-4.00000000051829	3.61523046092184\\
-4.00000000051986	3.6312625250501\\
-4.00000000052136	3.64729458917836\\
-4.00000000052281	3.66332665330661\\
-4.00000000052419	3.67935871743487\\
-4.0000000005255	3.69539078156313\\
-4.00000000052675	3.71142284569138\\
-4.00000000052794	3.72745490981964\\
-4.00000000052906	3.7434869739479\\
-4.00000000053011	3.75951903807615\\
-4.0000000005311	3.77555110220441\\
-4.00000000053202	3.79158316633267\\
-4.00000000053288	3.80761523046092\\
-4.00000000053367	3.82364729458918\\
-4.00000000053439	3.83967935871743\\
-4.00000000053504	3.85571142284569\\
-4.00000000053563	3.87174348697395\\
-4.00000000053614	3.8877755511022\\
-4.00000000053659	3.90380761523046\\
-4.00000000053697	3.91983967935872\\
-4.00000000053728	3.93587174348697\\
-4.00000000053752	3.95190380761523\\
-4.0000000005377	3.96793587174349\\
-4.0000000005378	3.98396793587174\\
-4.00000000053783	4\\
-4	4.00000000053783\\
}--cycle;


\addplot[area legend,solid,fill=mycolor2,draw=black,forget plot]
table[row sep=crcr] {%
x	y\\
-2.18837675350701	2.51218829191193\\
-2.18733967634176	2.5250501002004\\
-2.18597120113148	2.54108216432866\\
-2.18452376321569	2.55711422845691\\
-2.18299582363611	2.57314629258517\\
-2.18138579001696	2.58917835671343\\
-2.179692015072	2.60521042084168\\
-2.17791279505507	2.62124248496994\\
-2.17604636815184	2.6372745490982\\
-2.17409091281085	2.65330661322645\\
-2.17234468937876	2.66700321609212\\
-2.17205000108829	2.66933867735471\\
-2.169949223652	2.68537074148297\\
-2.16775450582966	2.70140280561122\\
-2.1654637519929	2.71743486973948\\
-2.1630747968375	2.73346693386774\\
-2.16058540325749	2.74949899799599\\
-2.15799326013909	2.76553106212425\\
-2.1563126252505	2.77555497956145\\
-2.15531263058558	2.7815631262525\\
-2.1525529883469	2.79759519038076\\
-2.14968393265089	2.81362725450902\\
-2.14670280909838	2.82965931863727\\
-2.1436068759187	2.84569138276553\\
-2.14039330115298	2.86172344689379\\
-2.14028056112224	2.86227095588746\\
-2.13710656700386	2.87775551102204\\
-2.13369847096792	2.8937875751503\\
-2.13016418146547	2.90981963927856\\
-2.12650044307887	2.92585170340681\\
-2.12424849699399	2.9354039450745\\
-2.12272457763207	2.94188376753507\\
-2.11884335835732	2.95791583166333\\
-2.11482256942741	2.97394789579158\\
-2.11065846187091	2.98997995991984\\
-2.10821643286573	2.99910755738531\\
-2.10636965934285	3.0060120240481\\
-2.10195962779741	3.02204408817635\\
-2.09739431105722	3.03807615230461\\
-2.09266938106967	3.05410821643287\\
-2.09218436873747	3.05571690098827\\
-2.08782724256108	3.07014028056112\\
-2.08282144178327	3.08617234468938\\
-2.07764187201208	3.10220440881764\\
-2.07615230460922	3.10670417198628\\
-2.0723199728081	3.11823647294589\\
-2.0668279322443	3.13426853707415\\
-2.06114596808467	3.1503006012024\\
-2.06012024048096	3.15313095886327\\
-2.05530808263616	3.16633266533066\\
-2.04927652476668	3.18236472945892\\
-2.04408817635271	3.19572743614558\\
-2.04304403780948	3.19839679358717\\
-2.03663353865448	3.21442885771543\\
-2.03000039851701	3.23046092184369\\
-2.02805611222445	3.23505272624421\\
-2.02316612665281	3.24649298597194\\
-2.01610504929561	3.2625250501002\\
-2.01202404809619	3.27154119214257\\
-2.00881293353938	3.27855711422846\\
-2.00128462031002	3.29458917835671\\
-1.99599198396794	3.30553466284577\\
-1.99350063178841	3.31062124248497\\
-1.98546057570419	3.32665330661323\\
-1.97995991983968	3.33730692442868\\
-1.97714210753077	3.34268537074148\\
-1.96853981202308	3.35871743486974\\
-1.96392785571142	3.36708567002893\\
-1.95963426252646	3.374749498998\\
-1.9504123103506	3.39078156312625\\
-1.94789579158317	3.39506013426401\\
-1.9408553430388	3.40681362725451\\
-1.93186372745491	3.42138894870863\\
-1.93094713526962	3.42284569138277\\
-1.92066168164344	3.43887775551102\\
-1.91583166332665	3.44621303528219\\
-1.90997946975807	3.45490981963928\\
-1.8997995991984	3.46964207855806\\
-1.89888029101296	3.47094188376754\\
-1.8873210583821	3.48697394789579\\
-1.88376753507014	3.4917892515538\\
-1.87527724755564	3.50300601202405\\
-1.86773547094188	3.51273640977956\\
-1.86271726148988	3.51903807615231\\
-1.85170340681363	3.53256061063652\\
-1.84959972635276	3.53507014028056\\
-1.8358740682194	3.55110220440882\\
-1.83567134268537	3.55133486431535\\
-1.82146947117734	3.56713426853707\\
-1.81963927855711	3.56913032217164\\
-1.80633755317204	3.58316633266533\\
-1.80360721442886	3.58599389569402\\
-1.79040271332929	3.59919839679359\\
-1.7875751503006	3.6019764910627\\
-1.77357601502004	3.61523046092184\\
-1.77154308617234	3.6171235514231\\
-1.75575261167513	3.6312625250501\\
-1.75551102204409	3.63147552060405\\
-1.73947895791583	3.64508353874812\\
-1.73675998401568	3.64729458917836\\
-1.72344689378758	3.65797252578098\\
-1.71645946836329	3.66332665330661\\
-1.70741482965932	3.67016942626541\\
-1.69466730704812	3.67935871743487\\
-1.69138276553106	3.68169890685915\\
-1.67535070140281	3.69260532634724\\
-1.67104927101087	3.69539078156313\\
-1.65931863727455	3.70291133356174\\
-1.64527066679244	3.71142284569138\\
-1.64328657314629	3.71261420294734\\
-1.62725450901804	3.72177920887112\\
-1.61670775334827	3.72745490981964\\
-1.61122244488978	3.7303851331788\\
-1.59519038076152	3.73847872555087\\
-1.5846136617279	3.7434869739479\\
-1.57915831663327	3.74605536787514\\
-1.56312625250501	3.75315746679068\\
-1.54767777357342	3.75951903807615\\
-1.54709418837675	3.75975836220358\\
-1.5310621242485	3.76593753192175\\
-1.51503006012024	3.77164679629925\\
-1.50316156487214	3.77555110220441\\
-1.49899799599198	3.77691813026863\\
-1.48296593186373	3.7817922824152\\
-1.46693386773547	3.78624079510059\\
-1.45090180360721	3.7902776572105\\
-1.44519057425423	3.79158316633267\\
-1.43486973947896	3.79394442068093\\
-1.4188376753507	3.79723728898389\\
-1.40280561122244	3.80015222065808\\
-1.38677354709419	3.80270011516812\\
-1.37074148296593	3.80489119658113\\
-1.35470941883768	3.80673504142009\\
-1.34536965782306	3.80761523046092\\
-1.33867735470942	3.80824897603755\\
-1.32264529058116	3.80944066381979\\
-1.30661322645291	3.81030619218637\\
-1.29058116232465	3.81085272362304\\
-1.27454909819639	3.81108688513842\\
-1.25851703406814	3.81101478617959\\
-1.24248496993988	3.81064203520423\\
-1.22645290581162	3.80997375496874\\
-1.21042084168337	3.80901459658709\\
-1.19438877755511	3.80776875241063\\
-1.19278466694236	3.80761523046092\\
-1.17835671342685	3.80626041352538\\
-1.1623246492986	3.80447940168065\\
-1.14629258517034	3.80242643223183\\
-1.13026052104208	3.80010406348015\\
-1.11422845691383	3.79751444724034\\
-1.09819639278557	3.79465933573617\\
-1.08238610784513	3.79158316633267\\
-1.08216432865731	3.79154076292006\\
-1.06613226452906	3.78821192551388\\
-1.0501002004008	3.7846257924548\\
-1.03406813627255	3.78078268667116\\
-1.01803607214429	3.77668255792351\\
-1.01387250326413	3.77555110220441\\
-1.00200400801603	3.7723776273203\\
-0.985971943887776	3.76783838447288\\
-0.969939879759519	3.76304552242583\\
-0.958735177018894	3.75951903807615\\
-0.953907815631263	3.75802296121048\\
-0.937875751503006	3.75280765424696\\
-0.92184368737475	3.74733952723095\\
-0.91104327932147	3.7434869739479\\
-0.905811623246493	3.74164795998862\\
-0.889779559118236	3.73576945779861\\
-0.87374749498998	3.72963642279804\\
-0.868262186531491	3.72745490981964\\
-0.857715430861723	3.7233182685757\\
-0.841683366733467	3.71678256208697\\
-0.829025370107675	3.71142284569138\\
-0.82565130260521	3.71001314770832\\
-0.809619238476954	3.70308011609804\\
-0.793587174348697	3.69588653505746\\
-0.792516323769336	3.69539078156313\\
-0.777555110220441	3.68855112833831\\
-0.761523046092184	3.68096111671956\\
-0.758238504575129	3.67935871743487\\
-0.745490981963928	3.6732143754226\\
-0.729458917835671	3.66522904433948\\
-0.725746813913297	3.66332665330661\\
-0.713426853707415	3.65708548600821\\
-0.697394789579158	3.64870434829676\\
-0.69476868281116	3.64729458917836\\
-0.681362725450902	3.64017704005774\\
-0.665330661322646	3.63139803025136\\
-0.665089071691607	3.6312625250501\\
-0.649298597194389	3.62249864928523\\
-0.636584308250442	3.61523046092184\\
-0.633266533066132	3.61335323719904\\
-0.617234468937876	3.60405701947759\\
-0.609073991017453	3.59919839679359\\
-0.601202404809619	3.59455732732911\\
-0.585170340681363	3.58485599522903\\
-0.582440001938185	3.58316633266533\\
-0.569138276553106	3.57501041351771\\
-0.556626181278272	3.56713426853707\\
-0.55310621242485	3.56493835321383\\
-0.537074148296593	3.55471227402176\\
-0.531542492394103	3.55110220440882\\
-0.521042084168337	3.5443073060698\\
-0.507113700500945	3.53507014028056\\
-0.50501002004008	3.53368642806086\\
-0.488977955911824	3.52292736966742\\
-0.483308498413444	3.51903807615231\\
-0.472945891783567	3.51198384860222\\
-0.46005621128414	3.50300601202405\\
-0.456913827655311	3.50083355292382\\
-0.440881763527054	3.48952899261935\\
-0.437328240215089	3.48697394789579\\
-0.424849699398798	3.47806371092255\\
-0.415086534647536	3.47094188376754\\
-0.408817635270541	3.46639943697746\\
-0.393288980010621	3.45490981963928\\
-0.392785571142285	3.454539734248\\
-0.376753507014028	3.44255838458529\\
-0.371923488697241	3.43887775551102\\
-0.360721442885771	3.43039175147638\\
-0.350951914584676	3.42284569138277\\
-0.344689378757515	3.41803573511502\\
-0.330353172608619	3.40681362725451\\
-0.328657314629258	3.40549328796713\\
-0.312625250501002	3.39280623394943\\
-0.310108731733566	3.39078156312625\\
-0.296593186372745	3.37996057446672\\
-0.29019815203592	3.374749498998\\
-0.280561122244489	3.36693288932542\\
-0.270605055257431	3.35871743486974\\
-0.264529058116232	3.35372555976707\\
-0.251314806296885	3.34268537074148\\
-0.248496993987976	3.34034081811575\\
-0.232464929859719	3.32678327605539\\
-0.232313243736369	3.32665330661323\\
-0.216432865731463	3.31309537364555\\
-0.213579442999436	3.31062124248497\\
-0.200400801603207	3.2992324911091\\
-0.195108165261117	3.29458917835671\\
-0.18436873747495	3.28519632550059\\
-0.17688838962525	3.27855711422846\\
-0.168336673346694	3.27098843500974\\
-0.158909757082702	3.2625250501002\\
-0.152304609218437	3.25661024098888\\
-0.141162530661818	3.24649298597194\\
-0.13627254509018	3.24206302979891\\
-0.12363755846904	3.23046092184369\\
-0.120240480961924	3.22734795447686\\
-0.10632623939084	3.21442885771543\\
-0.104208416833667	3.21246603622831\\
-0.0892204912486374	3.19839679358717\\
-0.0881763527054109	3.19741816574723\\
-0.0723127212168553	3.18236472945892\\
-0.0721442885771544	3.18220510436568\\
-0.0561122244488979	3.16683736834846\\
-0.0555916565288081	3.16633266533066\\
-0.0400801603206413	3.15130549133715\\
-0.0390544327169323	3.1503006012024\\
-0.0240480961923848	3.13560667108172\\
-0.0226962325176485	3.13426853707415\\
-0.00801603206412826	3.11974116466135\\
-0.00651134034866899	3.11823647294589\\
0.00801603206412782	3.10370910053309\\
0.00950559946698457	3.10220440881764\\
0.0240480961923843	3.08751047869695\\
0.0253596009134551	3.08617234468938\\
0.0400801603206409	3.07114517069587\\
0.0410553532443366	3.07014028056112\\
0.0561122244488974	3.05461291945066\\
0.0565972367810957	3.05410821643287\\
0.0719910800676468	3.03807615230461\\
0.0721442885771539	3.03791652721137\\
0.087246414967761	3.02204408817635\\
0.0881763527054105	3.0210654603364\\
0.102361643310788	3.0060120240481\\
0.104208416833667	3.00404920256098\\
0.117340166624626	2.98997995991984\\
0.120240480961924	2.98686699255302\\
0.1321851375395	2.97394789579158\\
0.13627254509018	2.96951793961855\\
0.146899470581772	2.95791583166333\\
0.152304609218437	2.95200102255201\\
0.161485852207839	2.94188376753507\\
0.168336673346693	2.93431508831635\\
0.175946750125508	2.92585170340681\\
0.18436873747495	2.91645885055069\\
0.190284421946428	2.90981963927856\\
0.200400801603206	2.89843088790269\\
0.204500923209734	2.8937875751503\\
0.216432865731463	2.88022964218263\\
0.218598114813846	2.87775551102204\\
0.232464929859719	2.86185341633595\\
0.232577669890459	2.86172344689379\\
0.246472484612783	2.84569138276553\\
0.248496993987976	2.84334683013979\\
0.260257551260827	2.82965931863727\\
0.264529058116232	2.8246674435346\\
0.273932429644883	2.81362725450902\\
0.280561122244489	2.80581064483644\\
0.28749819455161	2.79759519038076\\
0.296593186372745	2.78677420172124\\
0.300955760353324	2.7815631262525\\
0.312625250501002	2.76755573294743\\
0.314305885389588	2.76553106212425\\
0.327567928056745	2.74949899799599\\
0.328657314629258	2.74817865870861\\
0.340755276724305	2.73346693386774\\
0.344689378757515	2.72865697759999\\
0.353840505499915	2.71743486973948\\
0.360721442885771	2.70894886570484\\
0.366823878678944	2.70140280561122\\
0.376753507014028	2.68905137055723\\
0.37970552252767	2.68537074148297\\
0.39249088285182	2.66933867735471\\
0.392785571142285	2.66896859196343\\
0.405231158769934	2.65330661322645\\
0.408817635270541	2.64876416643638\\
0.417873568327678	2.6372745490982\\
0.424849699398798	2.62836431212495\\
0.430417805075108	2.62124248496994\\
0.440881763527054	2.60776546556524\\
0.442863436587081	2.60521042084168\\
0.45524297062471	2.58917835671343\\
0.456913827655311	2.5870058976132\\
0.467564961912667	2.57314629258517\\
0.472945891783567	2.56609206503508\\
0.479790945645323	2.55711422845691\\
0.488977955911824	2.54497145784377\\
0.491920100199112	2.54108216432866\\
0.503972942874826	2.5250501002004\\
0.50501002004008	2.5236663879807\\
0.515990373212707	2.50901803607214\\
0.521042084168337	2.50222313773313\\
0.527912695607077	2.49298597194389\\
0.537074148296593	2.48056397742858\\
0.539738724534669	2.47695390781563\\
0.551501787790471	2.46092184368737\\
0.55310621242485	2.45872592836413\\
0.563226279677954	2.44488977955912\\
0.569138276553106	2.43673386041149\\
0.574855395806361	2.42885771543086\\
0.585170340681363	2.4145153138663\\
0.586387604466831	2.41282565130261\\
0.597895556901213	2.39679358717435\\
0.601202404809619	2.39215251770987\\
0.609334863426597	2.38076152304609\\
0.617234468937876	2.36958808160184\\
0.620677433211101	2.36472945891784\\
0.631951634761235	2.34869739478958\\
0.633266533066132	2.34682017106677\\
0.643206984131002	2.33266533066132\\
0.649298597194389	2.32390145489646\\
0.654365277727699	2.31663326653307\\
0.665330661322646	2.30073670760607\\
0.665424433461496	2.30060120240481\\
0.676498350458976	2.28456913827655\\
0.681362725450902	2.27745158915594\\
0.687476579602111	2.2685370741483\\
0.697394789579159	2.25391476913844\\
0.698354673058552	2.25250501002004\\
0.709232506939536	2.23647294589178\\
0.713426853707415	2.23023177859338\\
0.720033795682041	2.22044088176353\\
0.729458917835672	2.20631120866813\\
0.730733503473829	2.20440881763527\\
0.7414301412799	2.18837675350701\\
0.745490981963928	2.18223241149474\\
0.752056579978297	2.17234468937876\\
0.761523046092185	2.15791502453519\\
0.762579557512944	2.1563126252505\\
0.773109252646421	2.14028056112224\\
0.777555110220441	2.13344090789742\\
0.783561943357505	2.12424849699399\\
0.793587174348698	2.10871218636007\\
0.793908866757872	2.10821643286573\\
0.804285298864114	2.09218436873747\\
0.809619238476954	2.08384163914413\\
0.814564392751905	2.07615230460922\\
0.824758302252478	2.06012024048096\\
0.825651302605211	2.0587105424979\\
0.834971324123861	2.04408817635271\\
0.841683366733467	2.03341582862003\\
0.845076051383763	2.02805611222445\\
0.855137738611259	2.01202404809619\\
0.857715430861724	2.00788740685225\\
0.865178068571839	1.99599198396794\\
0.87374749498998	1.98214143281808\\
0.875106761295386	1.97995991983968\\
0.885048557799838	1.96392785571142\\
0.889779559118236	1.95621033956214\\
0.89491406096065	1.94789579158317\\
0.904694649819828	1.93186372745491\\
0.905811623246493	1.93002471349564\\
0.914498021268976	1.91583166332665\\
0.921843687374749	1.90365215248145\\
0.924185695363177	1.8997995991984\\
0.933865262670698	1.88376753507014\\
0.937875751503006	1.87705615124094\\
0.943491311068582	1.86773547094188\\
0.953022059741463	1.85170340681363\\
0.953907815631262	1.85020732994795\\
0.962587153792168	1.83567134268537\\
0.969939879759519	1.8231657629068\\
0.972031583136088	1.81963927855711\\
0.981479033512833	1.80360721442886\\
0.985971943887775	1.79589449669732\\
0.990862346907323	1.7875751503006\\
1.00017244570538	1.77154308617235\\
1.00200400801603	1.76836961128823\\
1.00949495205705	1.75551102204409\\
1.01803607214429	1.74061041363493\\
1.01869091350944	1.73947895791583\\
1.02793446764556	1.72344689378758\\
1.03406813627254	1.71264641412607\\
1.03706881500154	1.70741482965932\\
1.04618567492923	1.69138276553106\\
1.0501002004008	1.6844253916532\\
1.05525837210245	1.67535070140281\\
1.06425307777468	1.65931863727455\\
1.06613226452906	1.65594739645577\\
1.07326397108547	1.64328657314629\\
1.08214091242481	1.62725450901804\\
1.08216432865731	1.62721210560544\\
1.09108973294102	1.61122244488978\\
1.09819639278557	1.59826655016503\\
1.09990251902498	1.59519038076152\\
1.10873952224162	1.57915831663327\\
1.11422845691383	1.56905753341269\\
1.11748836383966	1.56312625250501\\
1.12621695542465	1.54709418837675\\
1.13026052104208	1.53958302139598\\
1.13490125784064	1.5310621242485\\
1.14352540851714	1.51503006012024\\
1.14629258517034	1.50984126189114\\
1.15214446671673	1.49899799599198\\
1.16066802432549	1.48296593186373\\
1.1623246492986	1.47983010308346\\
1.16922102401272	1.46693386773547\\
1.17764771911111	1.45090180360721\\
1.17835671342685	1.44954698667167\\
1.18613373755529	1.43486973947896\\
1.19438877755511	1.41899119730041\\
1.19446961549923	1.4188376753507\\
1.2028851953898	1.40280561122244\\
1.21042084168337	1.38817291322035\\
1.21115094474724	1.38677354709419\\
1.21947777124482	1.37074148296593\\
1.22645290581162	1.35706794334549\\
1.22767214575648	1.35470941883768\\
1.23591362954052	1.33867735470942\\
1.24248496993988	1.32567209532447\\
1.24403527399133	1.32264529058116\\
1.25219472995507	1.30661322645291\\
1.25851703406814	1.29398071804332\\
1.26024217985847	1.29058116232465\\
1.26832283156246	1.27454909819639\\
1.27454909819639	1.26198868874564\\
1.27629451261218	1.25851703406814\\
1.28429949655346	1.24248496993988\\
1.29058116232465	1.22969039897374\\
1.29219372384407	1.22645290581162\\
1.30012609355036	1.21042084168337\\
1.30661322645291	1.19707973928056\\
1.30794107056714	1.19438877755511\\
1.31580380052522	1.17835671342685\\
1.32264529058116	1.16415008265747\\
1.32353761790295	1.1623246492986\\
1.33133360732971	1.14629258517034\\
1.33867735470942	1.13089426661871\\
1.33898424137981	1.13026052104208\\
1.34671631784397	1.11422845691383\\
1.35429512759832	1.09819639278557\\
1.35470941883768	1.09731620374474\\
1.36195255175066	1.08216432865731\\
1.36947185626311	1.06613226452906\\
1.37074148296593	1.06340823064926\\
1.37704274593921	1.0501002004008\\
1.38450516462339	1.03406813627254\\
1.38677354709419	1.02915302097975\\
1.39198715554453	1.01803607214429\\
1.39939524815508	1.00200400801603\\
1.40280561122244	0.994540998213194\\
1.40678585462307	0.985971943887775\\
1.41414212425145	0.969939879759519\\
1.4188376753507	0.959561938282485\\
1.42143873646836	0.953907815631262\\
1.42874563242771	0.937875751503006\\
1.43486973947896	0.924204941723015\\
1.43594551356707	0.921843687374749\\
1.44320543417176	0.905811623246493\\
1.45032497946464	0.889779559118236\\
1.45090180360721	0.888474049996076\\
1.4575210124409	0.87374749498998\\
1.46459710752866	0.857715430861724\\
1.46693386773547	0.852373059629644\\
1.47169167080326	0.841683366733467\\
1.4787265569285	0.825651302605211\\
1.48296593186373	0.815860418687749\\
1.48571653222177	0.809619238476954\\
1.49271240871521	0.793587174348698\\
1.49899799599198	0.778922138284666\\
1.49959453747732	0.777555110220441\\
1.50655356408688	0.761523046092185\\
1.51338033327571	0.745490981963928\\
1.51503006012024	0.741586676058771\\
1.52024874236831	0.729458917835672\\
1.52704167434769	0.713426853707415\\
1.5310621242485	0.703813283424752\\
1.53379647862441	0.697394789579159\\
1.540557604138	0.681362725450902\\
1.54709418837675	0.665569985450078\\
1.54719512090086	0.665330661322646\\
1.55392643861466	0.649298597194389\\
1.56053156729777	0.633266533066132\\
1.56312625250501	0.626904961780657\\
1.56714630555442	0.617234468937876\\
1.57372428420662	0.601202404809619\\
1.57915831663327	0.587738734608609\\
1.58021514095198	0.585170340681363\\
1.58676785760723	0.569138276553106\\
1.59319910861719	0.55310621242485\\
1.59519038076152	0.548097964027826\\
1.59966000209789	0.537074148296593\\
1.60606840850847	0.521042084168337\\
1.61122244488978	0.507940243399239\\
1.61239823384089	0.50501002004008\\
1.61878558460074	0.488977955911824\\
1.62505577178471	0.472945891783567\\
1.62725450901804	0.467270190835045\\
1.63134792809836	0.456913827655311\\
1.63759925671641	0.440881763527054\\
1.64328657314629	0.42604105665475\\
1.64375252317298	0.424849699398798\\
1.64998669320543	0.408817635270541\\
1.65610761012657	0.392785571142285\\
1.65931863727455	0.384274059012638\\
1.66221493444902	0.376753507014028\\
1.66832068705787	0.360721442885771\\
1.67431648597408	0.344689378757515\\
1.67535070140281	0.34190392354163\\
1.68037282410443	0.328657314629258\\
1.6863553242951	0.312625250501002\\
1.69138276553106	0.298933375797022\\
1.6922604173067	0.296593186372745\\
1.69823117961289	0.280561122244489\\
1.70409529753085	0.264529058116232\\
1.70741482965932	0.255339766946777\\
1.70994020395826	0.248496993987976\\
1.71579427084906	0.232464929859719\\
1.72154459568374	0.216432865731463\\
1.72344689378758	0.211078738205828\\
1.72732382048331	0.200400801603206\\
1.73306566673049	0.18436873747495\\
1.73870651950803	0.168336673346693\\
1.73947895791583	0.166125622916457\\
1.74441441207152	0.152304609218437\\
1.75004824908652	0.13627254509018\\
1.75551102204409	0.12045347651587\\
1.75558622659312	0.120240480961924\\
1.76121443676716	0.104208416833667\\
1.76674422382966	0.0881763527054105\\
1.77154308617235	0.0740373790784134\\
1.77220020625581	0.0721442885771539\\
1.77772567498475	0.0561122244488974\\
1.78315512949002	0.0400801603206409\\
1.7875751503006	0.0268261904615026\\
1.78852289179611	0.0240480961923843\\
1.7939492363902	0.00801603206412782\\
1.79928184315911	-0.00801603206412826\\
1.80360721442886	-0.0212205331636935\\
1.80455495592437	-0.0240480961923848\\
1.80988556418485	-0.0400801603206413\\
1.81512458387508	-0.0561122244488979\\
1.81963927855711	-0.0701482349425848\\
1.82029639864058	-0.0721442885771544\\
1.82553443681551	-0.0881763527054109\\
1.83068291346776	-0.104208416833667\\
1.83567134268537	-0.120007821055388\\
1.8357465472344	-0.120240480961924\\
1.84089496711966	-0.13627254509018\\
1.84595573486811	-0.152304609218437\\
1.85093096840582	-0.168336673346694\\
1.85170340681363	-0.17084620299073\\
1.85596559890135	-0.18436873747495\\
1.86094128787276	-0.200400801603207\\
1.86583317283805	-0.216432865731463\\
1.86773547094188	-0.222734532104207\\
1.87074410091915	-0.232464929859719\\
1.875637142341	-0.248496993987976\\
1.88044800294167	-0.264529058116232\\
1.88376753507014	-0.275745818586478\\
1.88522755825375	-0.280561122244489\\
1.89004018878716	-0.296593186372745\\
1.89477215796244	-0.312625250501002\\
1.89942534149261	-0.328657314629258\\
1.8997995991984	-0.329957119838733\\
1.90414662631756	-0.344689378757515\\
1.90880164898172	-0.360721442885771\\
1.9133792480818	-0.376753507014028\\
1.91583166332665	-0.385450291371113\\
1.91795194784638	-0.392785571142285\\
1.92253178338579	-0.408817635270541\\
1.92703546142494	-0.424849699398798\\
1.93146469803591	-0.440881763527054\\
1.93186372745491	-0.44233850620119\\
1.93595714653524	-0.456913827655311\\
1.94038838530739	-0.472945891783567\\
1.94474630364003	-0.488977955911824\\
1.94789579158317	-0.500731448902317\\
1.94907158053428	-0.50501002004008\\
1.95343168096556	-0.521042084168337\\
1.95771949174352	-0.537074148296593\\
1.96193658356709	-0.55310621242485\\
1.96392785571142	-0.560770041393917\\
1.96615824208962	-0.569138276553106\\
1.97037697660293	-0.585170340681363\\
1.97452588741304	-0.601202404809619\\
1.97860645615369	-0.617234468937876\\
1.97995991983968	-0.622612915250674\\
1.98271067567622	-0.633266533066132\\
1.98679217007759	-0.649298597194389\\
1.99080608711256	-0.665330661322646\\
1.99475382388421	-0.681362725450902\\
1.99599198396794	-0.686449305090096\\
1.99872633834385	-0.697394789579158\\
2.00267410246425	-0.713426853707415\\
2.0065563245599	-0.729458917835671\\
2.01037432125166	-0.745490981963928\\
2.01202404809619	-0.752506904049819\\
2.0142003569969	-0.761523046092184\\
2.01801729618351	-0.777555110220441\\
2.02177052494767	-0.793587174348697\\
2.02546128301537	-0.809619238476954\\
2.02805611222445	-0.821059498204689\\
2.02912549303033	-0.82565130260521\\
2.03281391529224	-0.841683366733467\\
2.03644026093137	-0.857715430861723\\
2.04000569592562	-0.87374749498998\\
2.04351135221013	-0.889779559118236\\
2.04408817635271	-0.892448916559835\\
2.04705370876431	-0.905811623246493\\
2.05055469160957	-0.92184368737475\\
2.05399613342972	-0.937875751503006\\
2.05737909650658	-0.953907815631263\\
2.06012024048096	-0.96710952209865\\
2.06072393619058	-0.969939879759519\\
2.06410048388159	-0.985971943887776\\
2.06741867270145	-1.00200400801603\\
2.07067949744408	-1.01803607214429\\
2.07388392213842	-1.03406813627255\\
2.07615230460922	-1.04560043723216\\
2.0770617153881	-1.0501002004008\\
2.08025679315444	-1.06613226452906\\
2.0833954375222	-1.08216432865731\\
2.08647854783053	-1.09819639278557\\
2.08950699428547	-1.11422845691383\\
2.09218436873747	-1.12865183648668\\
2.09249125540787	-1.13026052104208\\
2.09550706610889	-1.14629258517034\\
2.09846802604284	-1.1623246492986\\
2.10137494280979	-1.17835671342685\\
2.10422859624857	-1.19438877755511\\
2.10702973894416	-1.21042084168337\\
2.10821643286573	-1.21732530834615\\
2.10982899438516	-1.22645290581162\\
2.11261272985613	-1.24248496993988\\
2.11534357785558	-1.25851703406814\\
2.11802223036006	-1.27454909819639\\
2.12064935314585	-1.29058116232465\\
2.1232255862079	-1.30661322645291\\
2.12424849699399	-1.31309304891348\\
2.12579880104544	-1.32264529058116\\
2.12835227738831	-1.33867735470942\\
2.13085429246183	-1.35470941883768\\
2.13330542655544	-1.37074148296593\\
2.13570623492495	-1.38677354709419\\
2.13805724812759	-1.40280561122244\\
2.14028056112224	-1.41829016635703\\
2.14036139906636	-1.4188376753507\\
2.14268365816133	-1.43486973947896\\
2.14495534686681	-1.45090180360721\\
2.14717693583636	-1.46693386773547\\
2.14934887147192	-1.48296593186373\\
2.15147157617786	-1.49899799599198\\
2.1535454485973	-1.51503006012024\\
2.15557086383083	-1.5310621242485\\
2.1563126252505	-1.53707027093956\\
2.15758539475005	-1.54709418837675\\
2.15957253217634	-1.56312625250501\\
2.16151014528104	-1.57915831663327\\
2.1633985476971	-1.59519038076152\\
2.1652380295342	-1.61122244488978\\
2.16702885751804	-1.62725450901804\\
2.16877127511243	-1.64328657314629\\
2.17046550262438	-1.65931863727455\\
2.17211173729216	-1.67535070140281\\
2.17234468937876	-1.67768616266539\\
2.17374993028851	-1.69138276553106\\
2.17534536810776	-1.70741482965932\\
2.17689138091722	-1.72344689378758\\
2.17838810008491	-1.73947895791583\\
2.17983563341978	-1.75551102204409\\
2.18123406517753	-1.77154308617234\\
2.18258345604918	-1.7875751503006\\
2.18388384313207	-1.80360721442886\\
2.1851352398834	-1.81963927855711\\
2.18633763605613	-1.83567134268537\\
2.18749099761721	-1.85170340681363\\
2.18837675350701	-1.8645652151021\\
2.18860133989036	-1.86773547094188\\
2.18968554512439	-1.88376753507014\\
2.19071876149544	-1.8997995991984\\
2.19170087930555	-1.91583166332665\\
2.19263176387134	-1.93186372745491\\
2.19351125534943	-1.94789579158317\\
2.19433916854262	-1.96392785571142\\
2.19511529268665	-1.97995991983968\\
2.19583939121713	-1.99599198396794\\
2.19651120151655	-2.01202404809619\\
2.19713043464091	-2.02805611222445\\
2.19769677502566	-2.04408817635271\\
2.19820988017068	-2.06012024048096\\
2.19866938030378	-2.07615230460922\\
2.19907487802243	-2.09218436873747\\
2.19942594791327	-2.10821643286573\\
2.19972213614885	-2.12424849699399\\
2.19996296006125	-2.14028056112224\\
2.20014790769201	-2.1563126252505\\
2.20027643731779	-2.17234468937876\\
2.20034797695124	-2.18837675350701\\
2.20036192381657	-2.20440881763527\\
2.20031764379895	-2.22044088176353\\
2.20021447086739	-2.23647294589178\\
2.20005170647016	-2.25250501002004\\
2.19982861890213	-2.2685370741483\\
2.19954444264335	-2.28456913827655\\
2.19919837766783	-2.30060120240481\\
2.19878958872204	-2.31663326653307\\
2.19831720457188	-2.33266533066132\\
2.19778031721757	-2.34869739478958\\
2.19717798107516	-2.36472945891784\\
2.19650921212399	-2.38076152304609\\
2.19577298701883	-2.39679358717435\\
2.1949682421657	-2.41282565130261\\
2.19409387276027	-2.42885771543086\\
2.19314873178755	-2.44488977955912\\
2.19213162898178	-2.46092184368737\\
2.19104132974509	-2.47695390781563\\
2.1898765540237	-2.49298597194389\\
2.18863597514014	-2.50901803607214\\
2.18837675350701	-2.51218829191193\\
2.18733967634176	-2.5250501002004\\
2.18597120113148	-2.54108216432866\\
2.18452376321569	-2.55711422845691\\
2.18299582363611	-2.57314629258517\\
2.18138579001696	-2.58917835671343\\
2.179692015072	-2.60521042084168\\
2.17791279505507	-2.62124248496994\\
2.17604636815184	-2.6372745490982\\
2.17409091281085	-2.65330661322645\\
2.17234468937876	-2.66700321609212\\
2.17205000108829	-2.66933867735471\\
2.169949223652	-2.68537074148297\\
2.16775450582966	-2.70140280561122\\
2.1654637519929	-2.71743486973948\\
2.1630747968375	-2.73346693386774\\
2.16058540325749	-2.74949899799599\\
2.15799326013909	-2.76553106212425\\
2.1563126252505	-2.77555497956145\\
2.15531263058558	-2.7815631262525\\
2.1525529883469	-2.79759519038076\\
2.14968393265089	-2.81362725450902\\
2.14670280909838	-2.82965931863727\\
2.1436068759187	-2.84569138276553\\
2.14039330115298	-2.86172344689379\\
2.14028056112224	-2.86227095588746\\
2.13710656700386	-2.87775551102204\\
2.13369847096792	-2.8937875751503\\
2.13016418146547	-2.90981963927856\\
2.12650044307887	-2.92585170340681\\
2.12424849699399	-2.9354039450745\\
2.12272457763207	-2.94188376753507\\
2.11884335835732	-2.95791583166333\\
2.11482256942741	-2.97394789579158\\
2.11065846187091	-2.98997995991984\\
2.10821643286573	-2.99910755738531\\
2.10636965934285	-3.0060120240481\\
2.10195962779741	-3.02204408817635\\
2.09739431105722	-3.03807615230461\\
2.09266938106967	-3.05410821643287\\
2.09218436873747	-3.05571690098827\\
2.08782724256107	-3.07014028056112\\
2.08282144178327	-3.08617234468938\\
2.07764187201208	-3.10220440881764\\
2.07615230460922	-3.10670417198628\\
2.0723199728081	-3.11823647294589\\
2.0668279322443	-3.13426853707415\\
2.06114596808467	-3.1503006012024\\
2.06012024048096	-3.15313095886327\\
2.05530808263616	-3.16633266533066\\
2.04927652476668	-3.18236472945892\\
2.04408817635271	-3.19572743614558\\
2.04304403780948	-3.19839679358717\\
2.03663353865448	-3.21442885771543\\
2.03000039851701	-3.23046092184369\\
2.02805611222445	-3.23505272624421\\
2.02316612665281	-3.24649298597194\\
2.01610504929561	-3.2625250501002\\
2.01202404809619	-3.27154119214257\\
2.00881293353938	-3.27855711422846\\
2.00128462031003	-3.29458917835671\\
1.99599198396794	-3.30553466284577\\
1.99350063178841	-3.31062124248497\\
1.98546057570419	-3.32665330661323\\
1.97995991983968	-3.33730692442868\\
1.97714210753077	-3.34268537074148\\
1.96853981202308	-3.35871743486974\\
1.96392785571142	-3.36708567002893\\
1.95963426252646	-3.374749498998\\
1.9504123103506	-3.39078156312625\\
1.94789579158317	-3.39506013426401\\
1.94085534303881	-3.40681362725451\\
1.93186372745491	-3.42138894870863\\
1.93094713526962	-3.42284569138277\\
1.92066168164344	-3.43887775551102\\
1.91583166332665	-3.44621303528219\\
1.90997946975807	-3.45490981963928\\
1.8997995991984	-3.46964207855806\\
1.89888029101296	-3.47094188376753\\
1.88732105838211	-3.48697394789579\\
1.88376753507014	-3.4917892515538\\
1.87527724755564	-3.50300601202405\\
1.86773547094188	-3.51273640977956\\
1.86271726148988	-3.5190380761523\\
1.85170340681363	-3.53256061063652\\
1.84959972635276	-3.53507014028056\\
1.8358740682194	-3.55110220440882\\
1.83567134268537	-3.55133486431535\\
1.82146947117734	-3.56713426853707\\
1.81963927855711	-3.56913032217164\\
1.80633755317203	-3.58316633266533\\
1.80360721442886	-3.58599389569402\\
1.79040271332929	-3.59919839679359\\
1.7875751503006	-3.6019764910627\\
1.77357601502004	-3.61523046092184\\
1.77154308617235	-3.6171235514231\\
1.75575261167513	-3.6312625250501\\
1.75551102204409	-3.63147552060405\\
1.73947895791583	-3.64508353874812\\
1.73675998401568	-3.64729458917836\\
1.72344689378758	-3.65797252578098\\
1.71645946836328	-3.66332665330661\\
1.70741482965932	-3.67016942626541\\
1.69466730704812	-3.67935871743487\\
1.69138276553106	-3.68169890685915\\
1.67535070140281	-3.69260532634724\\
1.67104927101087	-3.69539078156313\\
1.65931863727455	-3.70291133356174\\
1.64527066679244	-3.71142284569138\\
1.64328657314629	-3.71261420294734\\
1.62725450901804	-3.72177920887112\\
1.61670775334827	-3.72745490981964\\
1.61122244488978	-3.7303851331788\\
1.59519038076152	-3.73847872555087\\
1.5846136617279	-3.7434869739479\\
1.57915831663327	-3.74605536787514\\
1.56312625250501	-3.75315746679068\\
1.54767777357342	-3.75951903807615\\
1.54709418837675	-3.75975836220358\\
1.5310621242485	-3.76593753192175\\
1.51503006012024	-3.77164679629925\\
1.50316156487214	-3.77555110220441\\
1.49899799599198	-3.77691813026864\\
1.48296593186373	-3.7817922824152\\
1.46693386773547	-3.78624079510059\\
1.45090180360721	-3.7902776572105\\
1.44519057425423	-3.79158316633267\\
1.43486973947896	-3.79394442068093\\
1.4188376753507	-3.79723728898389\\
1.40280561122244	-3.80015222065808\\
1.38677354709419	-3.80270011516812\\
1.37074148296593	-3.80489119658113\\
1.35470941883768	-3.80673504142009\\
1.34536965782306	-3.80761523046092\\
1.33867735470942	-3.80824897603755\\
1.32264529058116	-3.80944066381979\\
1.30661322645291	-3.81030619218637\\
1.29058116232465	-3.81085272362304\\
1.27454909819639	-3.81108688513842\\
1.25851703406814	-3.81101478617959\\
1.24248496993988	-3.81064203520423\\
1.22645290581162	-3.80997375496874\\
1.21042084168337	-3.80901459658709\\
1.19438877755511	-3.80776875241063\\
1.19278466694236	-3.80761523046092\\
1.17835671342685	-3.80626041352538\\
1.1623246492986	-3.80447940168065\\
1.14629258517034	-3.80242643223183\\
1.13026052104208	-3.80010406348015\\
1.11422845691383	-3.79751444724034\\
1.09819639278557	-3.79465933573617\\
1.08238610784513	-3.79158316633267\\
1.08216432865731	-3.79154076292006\\
1.06613226452906	-3.78821192551389\\
1.0501002004008	-3.7846257924548\\
1.03406813627254	-3.78078268667116\\
1.01803607214429	-3.77668255792351\\
1.01387250326413	-3.77555110220441\\
1.00200400801603	-3.7723776273203\\
0.985971943887775	-3.76783838447288\\
0.969939879759519	-3.76304552242583\\
0.958735177018894	-3.75951903807615\\
0.953907815631262	-3.75802296121048\\
0.937875751503006	-3.75280765424696\\
0.921843687374749	-3.74733952723095\\
0.911043279321471	-3.7434869739479\\
0.905811623246493	-3.74164795998862\\
0.889779559118236	-3.73576945779861\\
0.87374749498998	-3.72963642279804\\
0.868262186531492	-3.72745490981964\\
0.857715430861724	-3.7233182685757\\
0.841683366733467	-3.71678256208697\\
0.829025370107676	-3.71142284569138\\
0.825651302605211	-3.71001314770832\\
0.809619238476954	-3.70308011609804\\
0.793587174348698	-3.69588653505746\\
0.792516323769337	-3.69539078156313\\
0.777555110220441	-3.68855112833831\\
0.761523046092185	-3.68096111671956\\
0.75823850457513	-3.67935871743487\\
0.745490981963928	-3.6732143754226\\
0.729458917835672	-3.66522904433948\\
0.725746813913297	-3.66332665330661\\
0.713426853707415	-3.65708548600821\\
0.697394789579159	-3.64870434829676\\
0.694768682811162	-3.64729458917836\\
0.681362725450902	-3.64017704005774\\
0.665330661322646	-3.63139803025136\\
0.665089071691608	-3.6312625250501\\
0.649298597194389	-3.62249864928523\\
0.636584308250443	-3.61523046092184\\
0.633266533066132	-3.61335323719904\\
0.617234468937876	-3.60405701947759\\
0.609073991017455	-3.59919839679359\\
0.601202404809619	-3.59455732732911\\
0.585170340681363	-3.58485599522903\\
0.582440001938186	-3.58316633266533\\
0.569138276553106	-3.57501041351771\\
0.556626181278274	-3.56713426853707\\
0.55310621242485	-3.56493835321383\\
0.537074148296593	-3.55471227402176\\
0.531542492394102	-3.55110220440882\\
0.521042084168337	-3.5443073060698\\
0.507113700500945	-3.53507014028056\\
0.50501002004008	-3.53368642806086\\
0.488977955911824	-3.52292736966742\\
0.483308498413443	-3.5190380761523\\
0.472945891783567	-3.51198384860222\\
0.460056211284139	-3.50300601202405\\
0.456913827655311	-3.50083355292383\\
0.440881763527054	-3.48952899261935\\
0.437328240215089	-3.48697394789579\\
0.424849699398798	-3.47806371092255\\
0.415086534647535	-3.47094188376754\\
0.408817635270541	-3.46639943697746\\
0.393288980010621	-3.45490981963928\\
0.392785571142285	-3.454539734248\\
0.376753507014028	-3.44255838458529\\
0.371923488697241	-3.43887775551102\\
0.360721442885771	-3.43039175147638\\
0.350951914584676	-3.42284569138276\\
0.344689378757515	-3.41803573511502\\
0.330353172608618	-3.40681362725451\\
0.328657314629258	-3.40549328796713\\
0.312625250501002	-3.39280623394943\\
0.310108731733566	-3.39078156312625\\
0.296593186372745	-3.37996057446673\\
0.29019815203592	-3.374749498998\\
0.280561122244489	-3.36693288932542\\
0.270605055257431	-3.35871743486974\\
0.264529058116232	-3.35372555976707\\
0.251314806296885	-3.34268537074148\\
0.248496993987976	-3.34034081811575\\
0.232464929859719	-3.32678327605539\\
0.232313243736369	-3.32665330661323\\
0.216432865731463	-3.31309537364555\\
0.213579442999435	-3.31062124248497\\
0.200400801603206	-3.2992324911091\\
0.195108165261117	-3.29458917835671\\
0.18436873747495	-3.28519632550059\\
0.176888389625251	-3.27855711422846\\
0.168336673346693	-3.27098843500974\\
0.158909757082701	-3.2625250501002\\
0.152304609218437	-3.25661024098888\\
0.141162530661818	-3.24649298597194\\
0.13627254509018	-3.24206302979891\\
0.123637558469039	-3.23046092184369\\
0.120240480961924	-3.22734795447686\\
0.10632623939084	-3.21442885771543\\
0.104208416833667	-3.21246603622831\\
0.0892204912486378	-3.19839679358717\\
0.0881763527054105	-3.19741816574722\\
0.0723127212168553	-3.18236472945892\\
0.0721442885771539	-3.18220510436568\\
0.0561122244488974	-3.16683736834846\\
0.055591656528809	-3.16633266533066\\
0.0400801603206409	-3.15130549133715\\
0.0390544327169332	-3.1503006012024\\
0.0240480961923843	-3.13560667108172\\
0.0226962325176485	-3.13426853707415\\
0.00801603206412782	-3.11974116466135\\
0.00651134034866946	-3.11823647294589\\
-0.00801603206412826	-3.10370910053309\\
-0.00950559946698502	-3.10220440881764\\
-0.0240480961923848	-3.08751047869695\\
-0.0253596009134542	-3.08617234468938\\
-0.0400801603206413	-3.07114517069587\\
-0.0410553532443371	-3.07014028056112\\
-0.0561122244488979	-3.05461291945066\\
-0.0565972367810961	-3.05410821643287\\
-0.0719910800676459	-3.03807615230461\\
-0.0721442885771544	-3.03791652721137\\
-0.087246414967761	-3.02204408817635\\
-0.0881763527054109	-3.0210654603364\\
-0.102361643310788	-3.0060120240481\\
-0.104208416833667	-3.00404920256098\\
-0.117340166624625	-2.98997995991984\\
-0.120240480961924	-2.98686699255302\\
-0.1321851375395	-2.97394789579158\\
-0.13627254509018	-2.96951793961855\\
-0.146899470581772	-2.95791583166333\\
-0.152304609218437	-2.95200102255201\\
-0.161485852207838	-2.94188376753507\\
-0.168336673346694	-2.93431508831635\\
-0.175946750125508	-2.92585170340681\\
-0.18436873747495	-2.91645885055069\\
-0.190284421946428	-2.90981963927856\\
-0.200400801603207	-2.89843088790269\\
-0.204500923209734	-2.8937875751503\\
-0.216432865731463	-2.88022964218263\\
-0.218598114813846	-2.87775551102204\\
-0.232464929859719	-2.86185341633595\\
-0.232577669890459	-2.86172344689379\\
-0.246472484612783	-2.84569138276553\\
-0.248496993987976	-2.84334683013979\\
-0.260257551260827	-2.82965931863727\\
-0.264529058116232	-2.8246674435346\\
-0.273932429644883	-2.81362725450902\\
-0.280561122244489	-2.80581064483644\\
-0.28749819455161	-2.79759519038076\\
-0.296593186372745	-2.78677420172123\\
-0.300955760353324	-2.7815631262525\\
-0.312625250501002	-2.76755573294743\\
-0.314305885389588	-2.76553106212425\\
-0.327567928056745	-2.74949899799599\\
-0.328657314629258	-2.74817865870861\\
-0.340755276724304	-2.73346693386774\\
-0.344689378757515	-2.72865697759999\\
-0.353840505499914	-2.71743486973948\\
-0.360721442885771	-2.70894886570484\\
-0.366823878678944	-2.70140280561122\\
-0.376753507014028	-2.68905137055723\\
-0.379705522527669	-2.68537074148297\\
-0.39249088285182	-2.66933867735471\\
-0.392785571142285	-2.66896859196343\\
-0.405231158769934	-2.65330661322645\\
-0.408817635270541	-2.64876416643638\\
-0.417873568327679	-2.6372745490982\\
-0.424849699398798	-2.62836431212495\\
-0.430417805075109	-2.62124248496994\\
-0.440881763527054	-2.60776546556524\\
-0.442863436587081	-2.60521042084168\\
-0.455242970624711	-2.58917835671343\\
-0.456913827655311	-2.5870058976132\\
-0.467564961912668	-2.57314629258517\\
-0.472945891783567	-2.56609206503508\\
-0.479790945645323	-2.55711422845691\\
-0.488977955911824	-2.54497145784377\\
-0.491920100199112	-2.54108216432866\\
-0.503972942874826	-2.5250501002004\\
-0.50501002004008	-2.5236663879807\\
-0.515990373212706	-2.50901803607214\\
-0.521042084168337	-2.50222313773313\\
-0.527912695607077	-2.49298597194389\\
-0.537074148296593	-2.48056397742858\\
-0.539738724534669	-2.47695390781563\\
-0.551501787790471	-2.46092184368737\\
-0.55310621242485	-2.45872592836413\\
-0.563226279677955	-2.44488977955912\\
-0.569138276553106	-2.43673386041149\\
-0.574855395806361	-2.42885771543086\\
-0.585170340681363	-2.4145153138663\\
-0.586387604466832	-2.41282565130261\\
-0.597895556901214	-2.39679358717435\\
-0.601202404809619	-2.39215251770987\\
-0.609334863426596	-2.38076152304609\\
-0.617234468937876	-2.36958808160184\\
-0.620677433211101	-2.36472945891784\\
-0.631951634761235	-2.34869739478958\\
-0.633266533066132	-2.34682017106677\\
-0.643206984131002	-2.33266533066132\\
-0.649298597194389	-2.32390145489646\\
-0.654365277727698	-2.31663326653307\\
-0.665330661322646	-2.30073670760607\\
-0.665424433461496	-2.30060120240481\\
-0.676498350458976	-2.28456913827655\\
-0.681362725450902	-2.27745158915594\\
-0.687476579602111	-2.2685370741483\\
-0.697394789579158	-2.25391476913844\\
-0.698354673058551	-2.25250501002004\\
-0.709232506939537	-2.23647294589178\\
-0.713426853707415	-2.23023177859338\\
-0.720033795682041	-2.22044088176353\\
-0.729458917835671	-2.20631120866813\\
-0.730733503473829	-2.20440881763527\\
-0.741430141279901	-2.18837675350701\\
-0.745490981963928	-2.18223241149475\\
-0.752056579978297	-2.17234468937876\\
-0.761523046092184	-2.15791502453519\\
-0.762579557512944	-2.1563126252505\\
-0.773109252646421	-2.14028056112224\\
-0.777555110220441	-2.13344090789742\\
-0.783561943357505	-2.12424849699399\\
-0.793587174348697	-2.10871218636007\\
-0.793908866757872	-2.10821643286573\\
-0.804285298864114	-2.09218436873747\\
-0.809619238476954	-2.08384163914414\\
-0.814564392751905	-2.07615230460922\\
-0.824758302252477	-2.06012024048096\\
-0.82565130260521	-2.0587105424979\\
-0.834971324123861	-2.04408817635271\\
-0.841683366733467	-2.03341582862003\\
-0.845076051383762	-2.02805611222445\\
-0.855137738611258	-2.01202404809619\\
-0.857715430861723	-2.00788740685225\\
-0.865178068571839	-1.99599198396794\\
-0.87374749498998	-1.98214143281808\\
-0.875106761295386	-1.97995991983968\\
-0.885048557799838	-1.96392785571142\\
-0.889779559118236	-1.95621033956214\\
-0.89491406096065	-1.94789579158317\\
-0.904694649819828	-1.93186372745491\\
-0.905811623246493	-1.93002471349564\\
-0.914498021268976	-1.91583166332665\\
-0.92184368737475	-1.90365215248145\\
-0.924185695363177	-1.8997995991984\\
-0.933865262670698	-1.88376753507014\\
-0.937875751503006	-1.87705615124094\\
-0.943491311068581	-1.86773547094188\\
-0.953022059741462	-1.85170340681363\\
-0.953907815631263	-1.85020732994795\\
-0.962587153792167	-1.83567134268537\\
-0.969939879759519	-1.8231657629068\\
-0.972031583136088	-1.81963927855711\\
-0.981479033512832	-1.80360721442886\\
-0.985971943887776	-1.79589449669732\\
-0.990862346907323	-1.7875751503006\\
-1.00017244570538	-1.77154308617234\\
-1.00200400801603	-1.76836961128823\\
-1.00949495205705	-1.75551102204409\\
-1.01803607214429	-1.74061041363493\\
-1.01869091350944	-1.73947895791583\\
-1.02793446764556	-1.72344689378758\\
-1.03406813627255	-1.71264641412607\\
-1.03706881500154	-1.70741482965932\\
-1.04618567492923	-1.69138276553106\\
-1.0501002004008	-1.68442539165319\\
-1.05525837210245	-1.67535070140281\\
-1.06425307777468	-1.65931863727455\\
-1.06613226452906	-1.65594739645577\\
-1.07326397108547	-1.64328657314629\\
-1.08214091242481	-1.62725450901804\\
-1.08216432865731	-1.62721210560543\\
-1.09108973294102	-1.61122244488978\\
-1.09819639278557	-1.59826655016503\\
-1.09990251902498	-1.59519038076152\\
-1.10873952224162	-1.57915831663327\\
-1.11422845691383	-1.56905753341268\\
-1.11748836383966	-1.56312625250501\\
-1.12621695542465	-1.54709418837675\\
-1.13026052104208	-1.53958302139598\\
-1.13490125784064	-1.5310621242485\\
-1.14352540851714	-1.51503006012024\\
-1.14629258517034	-1.50984126189114\\
-1.15214446671673	-1.49899799599198\\
-1.16066802432549	-1.48296593186373\\
-1.1623246492986	-1.47983010308346\\
-1.16922102401271	-1.46693386773547\\
-1.17764771911111	-1.45090180360721\\
-1.17835671342685	-1.44954698667167\\
-1.18613373755529	-1.43486973947896\\
-1.19438877755511	-1.41899119730041\\
-1.19446961549923	-1.4188376753507\\
-1.2028851953898	-1.40280561122244\\
-1.21042084168337	-1.38817291322035\\
-1.21115094474724	-1.38677354709419\\
-1.21947777124482	-1.37074148296593\\
-1.22645290581162	-1.35706794334549\\
-1.22767214575648	-1.35470941883768\\
-1.23591362954052	-1.33867735470942\\
-1.24248496993988	-1.32567209532447\\
-1.24403527399133	-1.32264529058116\\
-1.25219472995507	-1.30661322645291\\
-1.25851703406814	-1.29398071804332\\
-1.26024217985847	-1.29058116232465\\
-1.26832283156246	-1.27454909819639\\
-1.27454909819639	-1.26198868874564\\
-1.27629451261218	-1.25851703406814\\
-1.28429949655346	-1.24248496993988\\
-1.29058116232465	-1.22969039897374\\
-1.29219372384407	-1.22645290581162\\
-1.30012609355036	-1.21042084168337\\
-1.30661322645291	-1.19707973928056\\
-1.30794107056714	-1.19438877755511\\
-1.31580380052522	-1.17835671342685\\
-1.32264529058116	-1.16415008265747\\
-1.32353761790295	-1.1623246492986\\
-1.33133360732971	-1.14629258517034\\
-1.33867735470942	-1.13089426661871\\
-1.33898424137981	-1.13026052104208\\
-1.34671631784397	-1.11422845691383\\
-1.35429512759832	-1.09819639278557\\
-1.35470941883768	-1.09731620374474\\
-1.36195255175066	-1.08216432865731\\
-1.36947185626311	-1.06613226452906\\
-1.37074148296593	-1.06340823064926\\
-1.37704274593921	-1.0501002004008\\
-1.38450516462339	-1.03406813627255\\
-1.38677354709419	-1.02915302097975\\
-1.39198715554453	-1.01803607214429\\
-1.39939524815508	-1.00200400801603\\
-1.40280561122244	-0.994540998213195\\
-1.40678585462307	-0.985971943887776\\
-1.41414212425145	-0.969939879759519\\
-1.4188376753507	-0.959561938282485\\
-1.42143873646836	-0.953907815631263\\
-1.42874563242771	-0.937875751503006\\
-1.43486973947896	-0.924204941723015\\
-1.43594551356707	-0.92184368737475\\
-1.44320543417176	-0.905811623246493\\
-1.45032497946464	-0.889779559118236\\
-1.45090180360721	-0.888474049996076\\
-1.4575210124409	-0.87374749498998\\
-1.46459710752866	-0.857715430861723\\
-1.46693386773547	-0.852373059629644\\
-1.47169167080326	-0.841683366733467\\
-1.4787265569285	-0.82565130260521\\
-1.48296593186373	-0.815860418687748\\
-1.48571653222177	-0.809619238476954\\
-1.49271240871521	-0.793587174348697\\
-1.49899799599198	-0.778922138284666\\
-1.49959453747732	-0.777555110220441\\
-1.50655356408688	-0.761523046092184\\
-1.51338033327571	-0.745490981963928\\
-1.51503006012024	-0.74158667605877\\
-1.52024874236831	-0.729458917835671\\
-1.52704167434769	-0.713426853707415\\
-1.5310621242485	-0.703813283424752\\
-1.53379647862441	-0.697394789579158\\
-1.54055760413801	-0.681362725450902\\
-1.54709418837675	-0.665569985450078\\
-1.54719512090086	-0.665330661322646\\
-1.55392643861466	-0.649298597194389\\
-1.56053156729777	-0.633266533066132\\
-1.56312625250501	-0.626904961780658\\
-1.56714630555442	-0.617234468937876\\
-1.57372428420663	-0.601202404809619\\
-1.57915831663327	-0.587738734608611\\
-1.58021514095198	-0.585170340681363\\
-1.58676785760723	-0.569138276553106\\
-1.59319910861719	-0.55310621242485\\
-1.59519038076152	-0.548097964027827\\
-1.59966000209789	-0.537074148296593\\
-1.60606840850847	-0.521042084168337\\
-1.61122244488978	-0.50794024339924\\
-1.61239823384089	-0.50501002004008\\
-1.61878558460074	-0.488977955911824\\
-1.62505577178471	-0.472945891783567\\
-1.62725450901804	-0.467270190835046\\
-1.63134792809836	-0.456913827655311\\
-1.63759925671641	-0.440881763527054\\
-1.64328657314629	-0.426041056654751\\
-1.64375252317298	-0.424849699398798\\
-1.64998669320543	-0.408817635270541\\
-1.65610761012657	-0.392785571142285\\
-1.65931863727455	-0.38427405901264\\
-1.66221493444902	-0.376753507014028\\
-1.66832068705787	-0.360721442885771\\
-1.67431648597408	-0.344689378757515\\
-1.67535070140281	-0.341903923541631\\
-1.68037282410443	-0.328657314629258\\
-1.6863553242951	-0.312625250501002\\
-1.69138276553106	-0.298933375797024\\
-1.69226041730671	-0.296593186372745\\
-1.69823117961289	-0.280561122244489\\
-1.70409529753085	-0.264529058116232\\
-1.70741482965932	-0.255339766946777\\
-1.70994020395826	-0.248496993987976\\
-1.71579427084906	-0.232464929859719\\
-1.72154459568374	-0.216432865731463\\
-1.72344689378758	-0.211078738205828\\
-1.72732382048331	-0.200400801603207\\
-1.73306566673049	-0.18436873747495\\
-1.73870651950802	-0.168336673346694\\
-1.73947895791583	-0.166125622916459\\
-1.74441441207152	-0.152304609218437\\
-1.75004824908651	-0.13627254509018\\
-1.75551102204409	-0.120453476515871\\
-1.75558622659312	-0.120240480961924\\
-1.76121443676716	-0.104208416833667\\
-1.76674422382965	-0.0881763527054109\\
-1.77154308617234	-0.0740373790784148\\
-1.77220020625581	-0.0721442885771544\\
-1.77772567498475	-0.0561122244488979\\
-1.78315512949002	-0.0400801603206413\\
-1.7875751503006	-0.0268261904615012\\
-1.78852289179611	-0.0240480961923848\\
-1.7939492363902	-0.00801603206412826\\
-1.79928184315911	0.00801603206412782\\
-1.80360721442886	0.0212205331636953\\
-1.80455495592437	0.0240480961923843\\
-1.80988556418485	0.0400801603206409\\
-1.81512458387508	0.0561122244488974\\
-1.81963927855711	0.0701482349425853\\
-1.82029639864058	0.0721442885771539\\
-1.82553443681551	0.0881763527054105\\
-1.83068291346776	0.104208416833667\\
-1.83567134268537	0.120007821055389\\
-1.8357465472344	0.120240480961924\\
-1.84089496711966	0.13627254509018\\
-1.84595573486811	0.152304609218437\\
-1.85093096840582	0.168336673346693\\
-1.85170340681363	0.170846202990731\\
-1.85596559890135	0.18436873747495\\
-1.86094128787276	0.200400801603206\\
-1.86583317283805	0.216432865731463\\
-1.86773547094188	0.222734532104208\\
-1.87074410091915	0.232464929859719\\
-1.875637142341	0.248496993987976\\
-1.88044800294167	0.264529058116232\\
-1.88376753507014	0.27574581858648\\
-1.88522755825375	0.280561122244489\\
-1.89004018878716	0.296593186372745\\
-1.89477215796244	0.312625250501002\\
-1.89942534149261	0.328657314629258\\
-1.8997995991984	0.329957119838733\\
-1.90414662631756	0.344689378757515\\
-1.90880164898172	0.360721442885771\\
-1.9133792480818	0.376753507014028\\
-1.91583166332665	0.385450291371114\\
-1.91795194784638	0.392785571142285\\
-1.92253178338579	0.408817635270541\\
-1.92703546142494	0.424849699398798\\
-1.93146469803591	0.440881763527054\\
-1.93186372745491	0.442338506201192\\
-1.93595714653524	0.456913827655311\\
-1.94038838530739	0.472945891783567\\
-1.94474630364003	0.488977955911824\\
-1.94789579158317	0.500731448902319\\
-1.94907158053428	0.50501002004008\\
-1.95343168096556	0.521042084168337\\
-1.95771949174352	0.537074148296593\\
-1.96193658356709	0.55310621242485\\
-1.96392785571142	0.560770041393919\\
-1.96615824208962	0.569138276553106\\
-1.97037697660293	0.585170340681363\\
-1.97452588741304	0.601202404809619\\
-1.97860645615369	0.617234468937876\\
-1.97995991983968	0.622612915250676\\
-1.98271067567622	0.633266533066132\\
-1.98679217007759	0.649298597194389\\
-1.99080608711256	0.665330661322646\\
-1.99475382388421	0.681362725450902\\
-1.99599198396794	0.686449305090098\\
-1.99872633834385	0.697394789579159\\
-2.00267410246425	0.713426853707415\\
-2.0065563245599	0.729458917835672\\
-2.01037432125166	0.745490981963928\\
-2.01202404809619	0.752506904049819\\
-2.0142003569969	0.761523046092185\\
-2.01801729618351	0.777555110220441\\
-2.02177052494767	0.793587174348698\\
-2.02546128301537	0.809619238476954\\
-2.02805611222445	0.821059498204689\\
-2.02912549303033	0.825651302605211\\
-2.03281391529224	0.841683366733467\\
-2.03644026093137	0.857715430861724\\
-2.04000569592562	0.87374749498998\\
-2.04351135221013	0.889779559118236\\
-2.04408817635271	0.892448916559834\\
-2.04705370876431	0.905811623246493\\
-2.05055469160957	0.921843687374749\\
-2.05399613342972	0.937875751503006\\
-2.05737909650658	0.953907815631262\\
-2.06012024048096	0.96710952209865\\
-2.06072393619058	0.969939879759519\\
-2.06410048388159	0.985971943887775\\
-2.06741867270145	1.00200400801603\\
-2.07067949744408	1.01803607214429\\
-2.07388392213842	1.03406813627254\\
-2.07615230460922	1.04560043723216\\
-2.0770617153881	1.0501002004008\\
-2.08025679315444	1.06613226452906\\
-2.0833954375222	1.08216432865731\\
-2.08647854783053	1.09819639278557\\
-2.08950699428547	1.11422845691383\\
-2.09218436873747	1.12865183648668\\
-2.09249125540787	1.13026052104208\\
-2.09550706610889	1.14629258517034\\
-2.09846802604284	1.1623246492986\\
-2.10137494280979	1.17835671342685\\
-2.10422859624857	1.19438877755511\\
-2.10702973894416	1.21042084168337\\
-2.10821643286573	1.21732530834615\\
-2.10982899438516	1.22645290581162\\
-2.11261272985613	1.24248496993988\\
-2.11534357785558	1.25851703406814\\
-2.11802223036006	1.27454909819639\\
-2.12064935314585	1.29058116232465\\
-2.1232255862079	1.30661322645291\\
-2.12424849699399	1.31309304891348\\
-2.12579880104544	1.32264529058116\\
-2.12835227738831	1.33867735470942\\
-2.13085429246183	1.35470941883768\\
-2.13330542655544	1.37074148296593\\
-2.13570623492495	1.38677354709419\\
-2.13805724812759	1.40280561122244\\
-2.14028056112224	1.41829016635702\\
-2.14036139906636	1.4188376753507\\
-2.14268365816133	1.43486973947896\\
-2.14495534686681	1.45090180360721\\
-2.14717693583636	1.46693386773547\\
-2.14934887147192	1.48296593186373\\
-2.15147157617786	1.49899799599198\\
-2.1535454485973	1.51503006012024\\
-2.15557086383083	1.5310621242485\\
-2.1563126252505	1.53707027093956\\
-2.15758539475005	1.54709418837675\\
-2.15957253217634	1.56312625250501\\
-2.16151014528104	1.57915831663327\\
-2.1633985476971	1.59519038076152\\
-2.1652380295342	1.61122244488978\\
-2.16702885751804	1.62725450901804\\
-2.16877127511243	1.64328657314629\\
-2.17046550262438	1.65931863727455\\
-2.17211173729216	1.67535070140281\\
-2.17234468937876	1.67768616266539\\
-2.17374993028851	1.69138276553106\\
-2.17534536810776	1.70741482965932\\
-2.17689138091722	1.72344689378758\\
-2.17838810008491	1.73947895791583\\
-2.17983563341978	1.75551102204409\\
-2.18123406517753	1.77154308617235\\
-2.18258345604918	1.7875751503006\\
-2.18388384313207	1.80360721442886\\
-2.1851352398834	1.81963927855711\\
-2.18633763605613	1.83567134268537\\
-2.18749099761721	1.85170340681363\\
-2.18837675350701	1.8645652151021\\
-2.18860133989036	1.86773547094188\\
-2.18968554512439	1.88376753507014\\
-2.19071876149544	1.8997995991984\\
-2.19170087930555	1.91583166332665\\
-2.19263176387134	1.93186372745491\\
-2.19351125534943	1.94789579158317\\
-2.19433916854262	1.96392785571142\\
-2.19511529268665	1.97995991983968\\
-2.19583939121713	1.99599198396794\\
-2.19651120151655	2.01202404809619\\
-2.19713043464091	2.02805611222445\\
-2.19769677502566	2.04408817635271\\
-2.19820988017068	2.06012024048096\\
-2.19866938030378	2.07615230460922\\
-2.19907487802243	2.09218436873747\\
-2.19942594791327	2.10821643286573\\
-2.19972213614885	2.12424849699399\\
-2.19996296006125	2.14028056112224\\
-2.20014790769201	2.1563126252505\\
-2.20027643731779	2.17234468937876\\
-2.20034797695124	2.18837675350701\\
-2.20036192381657	2.20440881763527\\
-2.20031764379895	2.22044088176353\\
-2.20021447086739	2.23647294589178\\
-2.20005170647016	2.25250501002004\\
-2.19982861890213	2.2685370741483\\
-2.19954444264334	2.28456913827655\\
-2.19919837766783	2.30060120240481\\
-2.19878958872204	2.31663326653307\\
-2.19831720457188	2.33266533066132\\
-2.19778031721757	2.34869739478958\\
-2.19717798107516	2.36472945891784\\
-2.19650921212399	2.38076152304609\\
-2.19577298701883	2.39679358717435\\
-2.1949682421657	2.41282565130261\\
-2.19409387276027	2.42885771543086\\
-2.19314873178755	2.44488977955912\\
-2.19213162898178	2.46092184368737\\
-2.19104132974509	2.47695390781563\\
-2.1898765540237	2.49298597194389\\
-2.18863597514014	2.50901803607214\\
-2.18837675350701	2.51218829191193\\
}--cycle;


\addplot[area legend,solid,fill=mycolor3,draw=black,forget plot]
table[row sep=crcr] {%
x	y\\
-1.85170340681363	2.08110478474631\\
-1.85096402425816	2.09218436873747\\
-1.84980986194504	2.10821643286573\\
-1.84856756467358	2.12424849699399\\
-1.84723538393176	2.14028056112224\\
-1.84581150654969	2.1563126252505\\
-1.84429405280818	2.17234468937876\\
-1.84268107447124	2.18837675350701\\
-1.8409705527399	2.20440881763527\\
-1.83916039612411	2.22044088176353\\
-1.83724843822968	2.23647294589178\\
-1.83567134268537	2.2490366856574\\
-1.83523737101778	2.25250501002004\\
-1.83313835568119	2.2685370741483\\
-1.83093089661092	2.28456913827655\\
-1.82861245985542	2.30060120240481\\
-1.82618042239087	2.31663326653307\\
-1.82363206922794	2.33266533066132\\
-1.82096459040213	2.34869739478958\\
-1.81963927855711	2.35635177958395\\
-1.81818857080282	2.36472945891784\\
-1.81529996453305	2.38076152304609\\
-1.8122831122404	2.39679358717435\\
-1.80913473079258	2.41282565130261\\
-1.80585142247804	2.42885771543086\\
-1.80360721442886	2.43941001692876\\
-1.8024385369892	2.44488977955912\\
-1.79890051672139	2.46092184368737\\
-1.79521614279353	2.47695390781563\\
-1.79138143889474	2.49298597194389\\
-1.7875751503006	2.50829095273659\\
-1.78739338660917	2.50901803607214\\
-1.78326946333912	2.5250501002004\\
-1.77898147439258	2.54108216432866\\
-1.7745247538334	2.55711422845691\\
-1.77154308617234	2.56748238129583\\
-1.76990175838651	2.57314629258517\\
-1.76511267720294	2.58917835671343\\
-1.7601383404456	2.60521042084168\\
-1.75551102204409	2.61959233253044\\
-1.75497487581311	2.62124248496994\\
-1.7496281134607	2.6372745490982\\
-1.74407694612616	2.65330661322645\\
-1.73947895791583	2.66613514873682\\
-1.73831692392196	2.66933867735471\\
-1.73234548685442	2.68537074148297\\
-1.72614739434424	2.70140280561122\\
-1.72344689378758	2.70818987746438\\
-1.71971566625049	2.71743486973948\\
-1.71304013232617	2.73346693386774\\
-1.70741482965932	2.74652047249504\\
-1.7061103722011	2.74949899799599\\
-1.69891012383408	2.76553106212425\\
-1.69143610370506	2.7815631262525\\
-1.69138276553106	2.78167509565841\\
-1.68365691788803	2.79759519038076\\
-1.67557984389159	2.81362725450902\\
-1.67535070140281	2.81407237793107\\
-1.66715968749157	2.82965931863727\\
-1.65931863727455	2.84405724240068\\
-1.65840866361447	2.84569138276553\\
-1.64927335199124	2.86172344689379\\
-1.64328657314629	2.87190089013585\\
-1.6397574132464	2.87775551102204\\
-1.62982373029581	2.8937875751503\\
-1.62725450901804	2.89783144559011\\
-1.61943156879626	2.90981963927856\\
-1.61122244488978	2.92202643431353\\
-1.60857531675665	2.92585170340681\\
-1.59719561432098	2.94188376753507\\
-1.59519038076152	2.94464271289486\\
-1.58523875260076	2.95791583166333\\
-1.57915831663327	2.9658107404219\\
-1.5726802746891	2.97394789579158\\
-1.56312625250501	2.98564600747063\\
-1.55945965112415	2.98997995991984\\
-1.54709418837675	3.00424549977693\\
-1.5455044615911	3.0060120240481\\
-1.5310621242485	3.02169538090901\\
-1.5307279572433	3.02204408817635\\
-1.51503006012024	3.03807216018363\\
-1.51502598115792	3.03807615230461\\
-1.49899799599198	3.05344370640559\\
-1.49827304395488	3.05410821643287\\
-1.48296593186373	3.06787012941409\\
-1.48031727468776	3.07014028056112\\
-1.46693386773547	3.08140454787408\\
-1.46097385579681	3.08617234468938\\
-1.45090180360721	3.09409375882733\\
-1.44001638721062	3.10220440881764\\
-1.43486973947896	3.10597882209824\\
-1.4188376753507	3.11710033903772\\
-1.41710903881226	3.11823647294589\\
-1.40280561122244	3.12750636962587\\
-1.39167734969043	3.13426853707415\\
-1.38677354709419	3.1372101514459\\
-1.37074148296593	3.14625626530284\\
-1.36307656718187	3.1503006012024\\
-1.35470941883768	3.15466661391826\\
-1.33867735470942	3.16246635164411\\
-1.33012237525181	3.16633266533066\\
-1.32264529058116	3.16968031772411\\
-1.30661322645291	3.17633463965954\\
-1.29075325444766	3.18236472945892\\
-1.29058116232465	3.18242966225713\\
-1.27454909819639	3.18802432785573\\
-1.25851703406814	3.19309930910985\\
-1.24248496993988	3.19767335714685\\
-1.23967836172005	3.19839679358717\\
-1.22645290581162	3.2017900554151\\
-1.21042084168337	3.20544337144136\\
-1.19438877755511	3.20864327520901\\
-1.17835671342685	3.21140421553854\\
-1.1623246492986	3.21373973178891\\
-1.15661502706466	3.21442885771543\\
-1.14629258517034	3.215673431823\\
-1.13026052104208	3.21720929975852\\
-1.11422845691383	3.21835227700143\\
-1.09819639278557	3.21911252681467\\
-1.08216432865731	3.21949948728803\\
-1.06613226452906	3.21952189989994\\
-1.0501002004008	3.21918783601784\\
-1.03406813627255	3.21850472144465\\
-1.01803607214429	3.21747935911114\\
-1.00200400801603	3.21611795000566\\
-0.985998095749708	3.21442885771543\\
-0.985971943887776	3.21442614155896\\
-0.969939879759519	3.21243069533663\\
-0.953907815631263	3.21011861233941\\
-0.937875751503006	3.2074939838205\\
-0.92184368737475	3.20456038185535\\
-0.905811623246493	3.20132087162849\\
-0.892586167338064	3.19839679358717\\
-0.889779559118236	3.19778513500252\\
-0.87374749498998	3.19398599487374\\
-0.857715430861723	3.18989208776193\\
-0.841683366733467	3.18550468284531\\
-0.830931707074993	3.18236472945892\\
-0.82565130260521	3.18084309491341\\
-0.809619238476954	3.17593153597101\\
-0.793587174348697	3.17073286629194\\
-0.780729021259947	3.16633266533066\\
-0.777555110220441	3.16526011168692\\
-0.761523046092184	3.15955813639492\\
-0.745490981963928	3.15357218820123\\
-0.737123833619732	3.1503006012024\\
-0.729458917835671	3.14733881557017\\
-0.713426853707415	3.14086475809335\\
-0.697777947004	3.13426853707415\\
-0.697394789579158	3.13410884736716\\
-0.681362725450902	3.12715612856513\\
-0.665330661322646	3.11991987857457\\
-0.661735583422272	3.11823647294589\\
-0.649298597194389	3.11247386624583\\
-0.633266533066132	3.10476552161437\\
-0.628119885334472	3.10220440881764\\
-0.617234468937876	3.09684155551826\\
-0.601202404809619	3.08866524528832\\
-0.596468517478442	3.08617234468938\\
-0.585170340681363	3.08027895349741\\
-0.569138276553106	3.07163682472855\\
-0.566442776085726	3.07014028056112\\
-0.55310621242485	3.06280212052261\\
-0.537799100333697	3.05410821643287\\
-0.537074148296593	3.05370005159222\\
-0.521042084168337	3.0444235273177\\
-0.510393744430537	3.03807615230461\\
-0.50501002004008	3.03489325179561\\
-0.488977955911824	3.02515213976752\\
-0.483995866906375	3.02204408817635\\
-0.472945891783567	3.01520356570113\\
-0.458503554440966	3.0060120240481\\
-0.456913827655311	3.00500778702454\\
-0.440881763527054	2.99463500693657\\
-0.433868346994613	2.98997995991984\\
-0.424849699398798	2.98403536906207\\
-0.409932488193913	2.97394789579158\\
-0.408817635270541	2.97319901951363\\
-0.392785571142285	2.96219094658839\\
-0.386705135174792	2.95791583166333\\
-0.376753507014028	2.95096193545089\\
-0.364065662450606	2.94188376753507\\
-0.360721442885771	2.93950507801007\\
-0.344689378757515	2.92785282855784\\
-0.341992196266275	2.92585170340681\\
-0.328657314629258	2.91601138957255\\
-0.32044819072278	2.90981963927856\\
-0.312625250501002	2.90394942268957\\
-0.299368989727407	2.8937875751503\\
-0.296593186372745	2.89167009450756\\
-0.280561122244489	2.87919735205685\\
-0.278741749184476	2.87775551102204\\
-0.264529058116232	2.86654115588278\\
-0.258542279271288	2.86172344689379\\
-0.248496993987976	2.85367293277856\\
-0.238727092594649	2.84569138276553\\
-0.232464929859719	2.84059517615725\\
-0.219276137482059	2.82965931863727\\
-0.216432865731463	2.82731020824735\\
-0.200400801603207	2.81382306215925\\
-0.200171659114423	2.81362725450902\\
-0.18436873747495	2.80016487245945\\
-0.181403709616206	2.79759519038076\\
-0.168336673346694	2.78630243725173\\
-0.162943743482435	2.7815631262525\\
-0.152304609218437	2.77223744705891\\
-0.14477725091542	2.76553106212425\\
-0.13627254509018	2.75797143184814\\
-0.126890685934725	2.74949899799599\\
-0.120240480961924	2.74350576301438\\
-0.109271399852711	2.73346693386774\\
-0.104208416833667	2.72884165515535\\
-0.0919075802425003	2.71743486973948\\
-0.0881763527054109	2.71398016764136\\
-0.0747881948941846	2.70140280561122\\
-0.0721442885771544	2.69892220598275\\
-0.0579029403073767	2.68537074148297\\
-0.0561122244488979	2.68366852299741\\
-0.0412421943145111	2.66933867735471\\
-0.0400801603206413	2.66821971978054\\
-0.0247969724703408	2.65330661322645\\
-0.0240480961923848	2.65257624647837\\
-0.00855888787954408	2.6372745490982\\
-0.00801603206412826	2.63673840286722\\
0.00747988583314737	2.62124248496994\\
0.00801603206412782	2.62070633873896\\
0.0233266486770152	2.60521042084168\\
0.0240480961923843	2.6044800540936\\
0.0389882058144757	2.58917835671343\\
0.0400801603206409	2.58805939913926\\
0.0544708966630596	2.57314629258517\\
0.0561122244488974	2.57144407409961\\
0.0697806217774118	2.55711422845691\\
0.0721442885771539	2.55463362882844\\
0.0849228676899447	2.54108216432866\\
0.0881763527054105	2.53762746223054\\
0.0999027298721873	2.5250501002004\\
0.104208416833667	2.52042482148801\\
0.114724933963928	2.50901803607214\\
0.120240480961924	2.50302480109053\\
0.129393855403837	2.49298597194389\\
0.13627254509018	2.48542634166778\\
0.143913537583111	2.47695390781563\\
0.152304609218437	2.46762822862203\\
0.158287708632811	2.46092184368737\\
0.168336673346693	2.44962909055835\\
0.172519796945656	2.44488977955912\\
0.18436873747495	2.43142739750955\\
0.186612945524132	2.42885771543086\\
0.200400801603206	2.41302145895284\\
0.200570025238683	2.41282565130261\\
0.214411163717496	2.39679358717435\\
0.216432865731463	2.39444447678443\\
0.22812561583566	2.38076152304609\\
0.232464929859719	2.37566531643781\\
0.241714242995323	2.36472945891784\\
0.248496993987976	2.35667894480261\\
0.255179057332447	2.34869739478958\\
0.264529058116232	2.33748303965031\\
0.268521848787055	2.33266533066132\\
0.280561122244489	2.31807510756787\\
0.281744193712368	2.31663326653307\\
0.294865693023195	2.30060120240481\\
0.296593186372745	2.29848372176207\\
0.307884804426556	2.28456913827655\\
0.312625250501002	2.27869892168757\\
0.320790737016945	2.2685370741483\\
0.328657314629258	2.25869676031403\\
0.333584388029848	2.25250501002004\\
0.344689378757515	2.23847407104281\\
0.346266474301825	2.23647294589178\\
0.358859841022653	2.22044088176353\\
0.360721442885771	2.21806219223852\\
0.37136467745965	2.20440881763527\\
0.376753507014028	2.19745492142283\\
0.383763238799898	2.18837675350701\\
0.392785571142285	2.17661980430382\\
0.396055667160588	2.17234468937876\\
0.408249324480346	2.1563126252505\\
0.408817635270541	2.15556374897255\\
0.420381676516927	2.14028056112224\\
0.424849699398798	2.13433597026447\\
0.432411864046547	2.12424849699399\\
0.440881763527054	2.11287147988246\\
0.444339516511447	2.10821643286573\\
0.456174445099843	2.09218436873747\\
0.456913827655311	2.09118013171392\\
0.467956638280627	2.07615230460922\\
0.472945891783567	2.069311782134\\
0.47963909184413	2.06012024048096\\
0.488977955911824	2.04719622794388\\
0.491220970321726	2.04408817635271\\
0.502735279239629	2.02805611222445\\
0.50501002004008	2.02487321171545\\
0.514184056143404	2.01202404809619\\
0.521042084168337	2.00233935898103\\
0.525533998792645	1.99599198396794\\
0.536788276952951	1.97995991983968\\
0.537074148296593	1.97955175499903\\
0.548013140606755	1.96392785571142\\
0.55310621242485	1.95658969567291\\
0.559140227809762	1.94789579158317\\
0.569138276553106	1.93336027162234\\
0.570167995371698	1.93186372745491\\
0.581160780256625	1.91583166332665\\
0.585170340681363	1.90993827213469\\
0.592072506252331	1.8997995991984\\
0.601202404809619	1.88626043566908\\
0.60288506512914	1.88376753507014\\
0.613657428555809	1.86773547094188\\
0.617234468937876	1.86237261764251\\
0.624359807395399	1.85170340681363\\
0.633266533066132	1.8382324554821\\
0.634962578770195	1.83567134268537\\
0.645529835186418	1.81963927855711\\
0.649298597194389	1.81387667185705\\
0.65602749226157	1.80360721442886\\
0.665330661322646	1.78925855592928\\
0.666424536732041	1.7875751503006\\
0.676801290439637	1.77154308617235\\
0.681362725450902	1.76443067766332\\
0.68709754087772	1.75551102204409\\
0.697293488050765	1.73947895791583\\
0.697394789579159	1.73931926820884\\
0.707491839219817	1.72344689378758\\
0.713426853707415	1.71401105067851\\
0.717588754260136	1.70741482965932\\
0.727616224980459	1.69138276553106\\
0.729458917835672	1.68842097989882\\
0.737618467272952	1.67535070140281\\
0.745490981963928	1.66259022427337\\
0.747516929577984	1.65931863727455\\
0.757388887898833	1.64328657314629\\
0.761523046092185	1.63651204421055\\
0.767195261899139	1.62725450901804\\
0.776907704806536	1.61122244488978\\
0.777555110220441	1.61014989124604\\
0.786623917378866	1.59519038076152\\
0.793587174348698	1.58355851759455\\
0.796233549095041	1.57915831663327\\
0.8058099752019	1.56312625250501\\
0.809619238476954	1.5566930590171\\
0.815331077458949	1.54709418837675\\
0.824760127386751	1.5310621242485\\
0.825651302605211	1.52954048970299\\
0.834194099609312	1.51503006012024\\
0.841683366733467	1.50213794937837\\
0.843517559081993	1.49899799599198\\
0.852828753689511	1.48296593186373\\
0.857715430861724	1.47446122603849\\
0.862065825221398	1.46693386773547\\
0.871240825915441	1.45090180360721\\
0.87374749498998	1.44649100489378\\
0.880392419679554	1.43486973947896\\
0.889435763559499	1.4188376753507\\
0.889779559118236	1.41822601676604\\
0.898502620912698	1.40280561122244\\
0.905811623246493	1.38969762513551\\
0.907453086973702	1.38677354709419\\
0.916401385107147	1.37074148296593\\
0.921843687374749	1.36087300710585\\
0.925266721059645	1.35470941883768\\
0.934093357167528	1.33867735470942\\
0.937875751503006	1.33174248081449\\
0.942873854060321	1.32264529058116\\
0.951582880981771	1.30661322645291\\
0.953907815631262	1.30230298107688\\
0.960278674444841	1.29058116232465\\
0.968874008993936	1.27454909819639\\
0.969939879759519	1.27255093581759\\
0.977485082284088	1.25851703406814\\
0.985970511113727	1.24248496993988\\
0.985971943887775	1.24248225378341\\
0.99449669779832	1.22645290581162\\
1.00200400801603	1.21210993397359\\
1.00289562534923	1.21042084168337\\
1.01131686927922	1.19438877755511\\
1.01803607214429	1.18140721482257\\
1.0196288895712	1.17835671342685\\
1.02794868041008	1.1623246492986\\
1.03406813627254	1.15036844889956\\
1.03617320976842	1.14629258517034\\
1.04439495700578	1.13026052104208\\
1.0501002004008	1.11898743521624\\
1.05253126766382	1.11422845691383\\
1.06065827319264	1.09819639278557\\
1.06613226452906	1.08725737084183\\
1.06870549438125	1.08216432865731\\
1.07674095704631	1.06613226452906\\
1.08216432865731	1.0551708299734\\
1.0846980757674	1.0501002004008\\
1.09264509570445	1.03406813627254\\
1.09819639278557	1.02271974124353\\
1.10051095719356	1.01803607214429\\
1.10837253996919	1.00200400801603\\
1.11422845691383	0.989895363173775\\
1.11614584785142	0.985971943887775\\
1.12392490841295	0.969939879759519\\
1.13026052104208	0.956688257674352\\
1.13160422455531	0.953907815631262\\
1.13930359099971	0.937875751503006\\
1.14629258517034	0.92308826148232\\
1.14688733506242	0.921843687374749\\
1.15450975223249	0.905811623246493\\
1.16200401996448	0.889779559118236\\
1.1623246492986	0.889090433191721\\
1.16954433383618	0.87374749498998\\
1.17696578019152	0.857715430861724\\
1.17835671342685	0.854690788684829\\
1.18440805698392	0.841683366733467\\
1.19175920173928	0.825651302605211\\
1.19438877755511	0.819865720098788\\
1.19910142407344	0.809619238476954\\
1.20638471539175	0.793587174348698\\
1.21042084168337	0.784601688074628\\
1.21362472005891	0.777555110220441\\
1.22084253737021	0.761523046092185\\
1.22645290581162	0.748884243791853\\
1.22797801334242	0.745490981963928\\
1.23513267009027	0.729458917835672\\
1.24216907752019	0.713426853707415\\
1.24248496993988	0.712703417267096\\
1.24925490248686	0.697394789579159\\
1.2562314894554	0.681362725450902\\
1.25851703406814	0.676065240973576\\
1.26320880990842	0.665330661322646\\
1.27012770666234	0.649298597194389\\
1.27454909819639	0.63892613146295\\
1.27699375358044	0.633266533066132\\
1.28385703316352	0.617234468937876\\
1.29058116232465	0.60126733760783\\
1.29060887963702	0.601202404809619\\
1.29741856006962	0.585170340681363\\
1.30411699672397	0.569138276553106\\
1.30661322645291	0.56310818675373\\
1.31081116406888	0.55310621242485\\
1.31745878918003	0.537074148296593\\
1.32264529058116	0.524389736561787\\
1.32403350548239	0.521042084168337\\
1.33063218027391	0.50501002004008\\
1.33712404097122	0.488977955911824\\
1.33867735470942	0.485111642225269\\
1.34363556339904	0.472945891783567\\
1.3500809692004	0.456913827655311\\
1.35470941883768	0.445247776242905\\
1.35646711206576	0.440881763527054\\
1.36286777974099	0.424849699398798\\
1.3691656409182	0.408817635270541\\
1.37074148296593	0.404773299370973\\
1.37548237873076	0.392785571142285\\
1.38173772597015	0.376753507014028\\
1.38677354709419	0.363663057257521\\
1.3879224466588	0.360721442885771\\
1.39413686273957	0.344689378757515\\
1.40025208659613	0.328657314629258\\
1.40280561122244	0.32189514718098\\
1.40636046095951	0.312625250501002\\
1.41243672444534	0.296593186372745\\
1.41841668542161	0.280561122244489\\
1.4188376753507	0.279424988336313\\
1.42444445727106	0.264529058116232\\
1.43038727181292	0.248496993987976\\
1.43486973947896	0.236239343140326\\
1.43627218400033	0.232464929859719\\
1.44217922799749	0.216432865731463\\
1.44799298112941	0.200400801603206\\
1.45090180360721	0.192290151612905\\
1.4537891721569	0.18436873747495\\
1.45956873876195	0.168336673346693\\
1.46525751441528	0.152304609218437\\
1.46693386773547	0.147536812403136\\
1.47096011403291	0.13627254509018\\
1.47661614380373	0.120240480961924\\
1.48218373363104	0.104208416833667\\
1.48296593186373	0.101938265686634\\
1.48778765267569	0.0881763527054105\\
1.49332379819697	0.0721442885771539\\
1.49877371175641	0.0561122244488974\\
1.49899799599198	0.0554477144216261\\
1.50427363149831	0.0400801603206409\\
1.50969326695727	0.0240480961923843\\
1.51502874080229	0.00801603206412782\\
1.51503006012024	0.00801203994315158\\
1.52041910140116	-0.00801603206412826\\
1.52572533108552	-0.0240480961923848\\
1.53094933679874	-0.0400801603206413\\
1.5310621242485	-0.0404288675879876\\
1.53622432482812	-0.0561122244488979\\
1.54141999058174	-0.0721442885771544\\
1.546535241786	-0.0881763527054109\\
1.54709418837675	-0.0899428769765749\\
1.551688776779	-0.104208416833667\\
1.55677646444501	-0.120240480961924\\
1.56178542281927	-0.13627254509018\\
1.56312625250501	-0.140606497539393\\
1.56681114278396	-0.152304609218437\\
1.57179318765974	-0.168336673346694\\
1.57669806798192	-0.18436873747495\\
1.57915831663327	-0.192505892844636\\
1.58158931382882	-0.200400801603207\\
1.5864678051518	-0.216432865731463\\
1.59127057938552	-0.232464929859719\\
1.59519038076152	-0.245738048628188\\
1.59602037820382	-0.248496993987976\\
1.60079716268189	-0.264529058116232\\
1.6054995631189	-0.280561122244489\\
1.61012914769989	-0.296593186372745\\
1.61122244488978	-0.300418455466027\\
1.61477729462685	-0.312625250501002\\
1.61938081609172	-0.328657314629258\\
1.62391268479772	-0.344689378757515\\
1.62725450901804	-0.356677572445968\\
1.62840340858265	-0.360721442885771\\
1.63290930970114	-0.376753507014028\\
1.63734461298571	-0.392785571142285\\
1.64171073109856	-0.408817635270541\\
1.64328657314629	-0.414672256156732\\
1.64607917023272	-0.424849699398798\\
1.65041882425813	-0.440881763527054\\
1.65469018763728	-0.456913827655311\\
1.6588945800742	-0.472945891783567\\
1.65931863727455	-0.474580032148421\\
1.66312834154604	-0.488977955911824\\
1.66730552696729	-0.50501002004008\\
1.67141648023444	-0.521042084168337\\
1.67535070140281	-0.536629024874543\\
1.67546529128451	-0.537074148296593\\
1.67954863901878	-0.55310621242485\\
1.68356639100834	-0.569138276553106\\
1.68751972819312	-0.585170340681363\\
1.69138276553106	-0.601090435403715\\
1.69141048284343	-0.601202404809619\\
1.69533503186013	-0.617234468937876\\
1.69919565132172	-0.633266533066132\\
1.70299343812526	-0.649298597194389\\
1.70672945483409	-0.665330661322646\\
1.70741482965932	-0.668309186823601\\
1.71047970488146	-0.681362725450902\\
1.7141847622063	-0.697394789579158\\
1.71782833817753	-0.713426853707415\\
1.7214114151401	-0.729458917835671\\
1.72344689378758	-0.738703910110774\\
1.72497200131837	-0.745490981963928\\
1.72852233483592	-0.761523046092184\\
1.73201231025909	-0.777555110220441\\
1.73544283162421	-0.793587174348697\\
1.7388147715124	-0.809619238476954\\
1.73947895791583	-0.812822767094841\\
1.74219405406295	-0.82565130260521\\
1.7455303014729	-0.841683366733467\\
1.74880791438502	-0.857715430861723\\
1.7520276893305	-0.87374749498998\\
1.75519039270997	-0.889779559118236\\
1.75551102204409	-0.891429711557734\\
1.75836421385233	-0.905811623246493\\
1.7614878128815	-0.92184368737475\\
1.76455409200172	-0.937875751503006\\
1.76756374294779	-0.953907815631263\\
1.77051742834487	-0.969939879759519\\
1.77154308617235	-0.975603791048864\\
1.77346047710993	-0.985971943887776\\
1.77637115270639	-1.00200400801603\\
1.77922541525254	-1.01803607214429\\
1.78202385321948	-1.03406813627255\\
1.78476702669091	-1.0501002004008\\
1.78745546766551	-1.06613226452906\\
1.7875751503006	-1.06685934786461\\
1.7901483801528	-1.08216432865731\\
1.79278818988801	-1.09819639278557\\
1.79537256765055	-1.11422845691383\\
1.79790197103384	-1.13026052104208\\
1.80037682989223	-1.14629258517034\\
1.80279754654409	-1.1623246492986\\
1.80360721442886	-1.16780441192895\\
1.80520003185577	-1.17835671342685\\
1.80756567487247	-1.19438877755511\\
1.80987621161925	-1.21042084168337\\
1.8121319683394	-1.22645290581162\\
1.81433324370427	-1.24248496993988\\
1.81648030891641	-1.25851703406814\\
1.81857340779153	-1.27454909819639\\
1.81963927855711	-1.28292677753027\\
1.82063428513926	-1.29058116232465\\
1.82266346550923	-1.30661322645291\\
1.82463738111443	-1.32264529058116\\
1.82655619601566	-1.33867735470942\\
1.82842004637171	-1.35470941883768\\
1.83022904041777	-1.37074148296593\\
1.83198325842196	-1.38677354709419\\
1.83368275261984	-1.40280561122244\\
1.83532754712663	-1.4188376753507\\
1.83567134268537	-1.42230599971333\\
1.83694387071105	-1.43486973947896\\
1.83851104478601	-1.45090180360721\\
1.84002173704505	-1.46693386773547\\
1.84147588385069	-1.48296593186373\\
1.84287339223335	-1.49899799599198\\
1.84421413968947	-1.51503006012024\\
1.84549797395561	-1.5310621242485\\
1.84672471275821	-1.54709418837675\\
1.84789414353857	-1.56312625250501\\
1.84900602315273	-1.57915831663327\\
1.8500600775457	-1.59519038076152\\
1.85105600139972	-1.61122244488978\\
1.85170340681363	-1.62230202888094\\
1.85199909229475	-1.62725450901804\\
1.85289456737126	-1.64328657314629\\
1.85372935442768	-1.65931863727455\\
1.85450304230896	-1.67535070140281\\
1.85521518624693	-1.69138276553106\\
1.85586530736635	-1.70741482965932\\
1.85645289216113	-1.72344689378758\\
1.85697739193992	-1.73947895791583\\
1.85743822224045	-1.75551102204409\\
1.85783476221172	-1.77154308617234\\
1.8581663539633	-1.7875751503006\\
1.85843230188081	-1.80360721442886\\
1.85863187190667	-1.81963927855711\\
1.85876429078523	-1.83567134268537\\
1.85882874527115	-1.85170340681363\\
1.85882438130004	-1.86773547094188\\
1.85875030312024	-1.88376753507014\\
1.8586055723846	-1.8997995991984\\
1.85838920720086	-1.91583166332665\\
1.85810018113967	-1.93186372745491\\
1.85773742219854	-1.94789579158317\\
1.85729981172059	-1.96392785571142\\
1.85678618326647	-1.97995991983968\\
1.85619532143793	-1.99599198396794\\
1.85552596065151	-2.01202404809619\\
1.85477678386049	-2.02805611222445\\
1.85394642122353	-2.04408817635271\\
1.85303344871797	-2.06012024048096\\
1.85203638669594	-2.07615230460922\\
1.85170340681363	-2.08110478474632\\
1.85096402425816	-2.09218436873747\\
1.84980986194504	-2.10821643286573\\
1.84856756467358	-2.12424849699399\\
1.84723538393176	-2.14028056112224\\
1.84581150654969	-2.1563126252505\\
1.84429405280818	-2.17234468937876\\
1.84268107447124	-2.18837675350701\\
1.8409705527399	-2.20440881763527\\
1.83916039612411	-2.22044088176353\\
1.83724843822968	-2.23647294589178\\
1.83567134268537	-2.24903668565741\\
1.83523737101778	-2.25250501002004\\
1.83313835568119	-2.2685370741483\\
1.83093089661092	-2.28456913827655\\
1.82861245985542	-2.30060120240481\\
1.82618042239087	-2.31663326653307\\
1.82363206922794	-2.33266533066132\\
1.82096459040213	-2.34869739478958\\
1.81963927855711	-2.35635177958396\\
1.81818857080282	-2.36472945891784\\
1.81529996453305	-2.38076152304609\\
1.8122831122404	-2.39679358717435\\
1.80913473079258	-2.41282565130261\\
1.80585142247804	-2.42885771543086\\
1.80360721442886	-2.43941001692876\\
1.8024385369892	-2.44488977955912\\
1.79890051672139	-2.46092184368737\\
1.79521614279353	-2.47695390781563\\
1.79138143889474	-2.49298597194389\\
1.7875751503006	-2.50829095273659\\
1.78739338660917	-2.50901803607214\\
1.78326946333912	-2.5250501002004\\
1.77898147439258	-2.54108216432866\\
1.7745247538334	-2.55711422845691\\
1.77154308617235	-2.56748238129583\\
1.76990175838651	-2.57314629258517\\
1.76511267720294	-2.58917835671343\\
1.7601383404456	-2.60521042084168\\
1.75551102204409	-2.61959233253044\\
1.75497487581311	-2.62124248496994\\
1.7496281134607	-2.6372745490982\\
1.74407694612616	-2.65330661322645\\
1.73947895791583	-2.66613514873682\\
1.73831692392196	-2.66933867735471\\
1.73234548685442	-2.68537074148297\\
1.72614739434424	-2.70140280561122\\
1.72344689378758	-2.70818987746438\\
1.71971566625049	-2.71743486973948\\
1.71304013232617	-2.73346693386774\\
1.70741482965932	-2.74652047249504\\
1.7061103722011	-2.74949899799599\\
1.69891012383408	-2.76553106212425\\
1.69143610370506	-2.7815631262525\\
1.69138276553106	-2.78167509565841\\
1.68365691788803	-2.79759519038076\\
1.67557984389159	-2.81362725450902\\
1.67535070140281	-2.81407237793107\\
1.66715968749157	-2.82965931863727\\
1.65931863727455	-2.84405724240068\\
1.65840866361447	-2.84569138276553\\
1.64927335199124	-2.86172344689379\\
1.64328657314629	-2.87190089013585\\
1.6397574132464	-2.87775551102204\\
1.62982373029581	-2.8937875751503\\
1.62725450901804	-2.8978314455901\\
1.61943156879626	-2.90981963927856\\
1.61122244488978	-2.92202643431353\\
1.60857531675665	-2.92585170340681\\
1.59719561432098	-2.94188376753507\\
1.59519038076152	-2.94464271289486\\
1.58523875260076	-2.95791583166333\\
1.57915831663327	-2.9658107404219\\
1.5726802746891	-2.97394789579158\\
1.56312625250501	-2.98564600747063\\
1.55945965112415	-2.98997995991984\\
1.54709418837675	-3.00424549977693\\
1.5455044615911	-3.0060120240481\\
1.5310621242485	-3.02169538090901\\
1.5307279572433	-3.02204408817635\\
1.51503006012024	-3.03807216018363\\
1.51502598115792	-3.03807615230461\\
1.49899799599198	-3.05344370640559\\
1.49827304395488	-3.05410821643287\\
1.48296593186373	-3.06787012941409\\
1.48031727468776	-3.07014028056112\\
1.46693386773547	-3.08140454787408\\
1.46097385579681	-3.08617234468938\\
1.45090180360721	-3.09409375882733\\
1.44001638721062	-3.10220440881764\\
1.43486973947896	-3.10597882209824\\
1.4188376753507	-3.11710033903772\\
1.41710903881226	-3.11823647294589\\
1.40280561122244	-3.12750636962587\\
1.39167734969043	-3.13426853707415\\
1.38677354709419	-3.1372101514459\\
1.37074148296593	-3.14625626530284\\
1.36307656718187	-3.1503006012024\\
1.35470941883768	-3.15466661391826\\
1.33867735470942	-3.16246635164411\\
1.33012237525181	-3.16633266533066\\
1.32264529058116	-3.16968031772411\\
1.30661322645291	-3.17633463965954\\
1.29075325444766	-3.18236472945892\\
1.29058116232465	-3.18242966225713\\
1.27454909819639	-3.18802432785573\\
1.25851703406814	-3.19309930910985\\
1.24248496993988	-3.19767335714685\\
1.23967836172005	-3.19839679358717\\
1.22645290581162	-3.2017900554151\\
1.21042084168337	-3.20544337144136\\
1.19438877755511	-3.20864327520901\\
1.17835671342685	-3.21140421553854\\
1.1623246492986	-3.21373973178892\\
1.15661502706466	-3.21442885771543\\
1.14629258517034	-3.215673431823\\
1.13026052104208	-3.21720929975852\\
1.11422845691383	-3.21835227700143\\
1.09819639278557	-3.21911252681467\\
1.08216432865731	-3.21949948728802\\
1.06613226452906	-3.21952189989994\\
1.0501002004008	-3.21918783601784\\
1.03406813627254	-3.21850472144465\\
1.01803607214429	-3.21747935911114\\
1.00200400801603	-3.21611795000566\\
0.985998095749708	-3.21442885771543\\
0.985971943887775	-3.21442614155896\\
0.969939879759519	-3.21243069533663\\
0.953907815631262	-3.21011861233941\\
0.937875751503006	-3.2074939838205\\
0.921843687374749	-3.20456038185535\\
0.905811623246493	-3.20132087162849\\
0.892586167338063	-3.19839679358717\\
0.889779559118236	-3.19778513500252\\
0.87374749498998	-3.19398599487374\\
0.857715430861724	-3.18989208776193\\
0.841683366733467	-3.18550468284531\\
0.830931707074993	-3.18236472945892\\
0.825651302605211	-3.18084309491341\\
0.809619238476954	-3.17593153597101\\
0.793587174348698	-3.17073286629194\\
0.780729021259948	-3.16633266533066\\
0.777555110220441	-3.16526011168692\\
0.761523046092185	-3.15955813639492\\
0.745490981963928	-3.15357218820123\\
0.737123833619732	-3.1503006012024\\
0.729458917835672	-3.14733881557017\\
0.713426853707415	-3.14086475809335\\
0.697777947004	-3.13426853707415\\
0.697394789579159	-3.13410884736716\\
0.681362725450902	-3.12715612856513\\
0.665330661322646	-3.11991987857457\\
0.661735583422272	-3.11823647294589\\
0.649298597194389	-3.11247386624583\\
0.633266533066132	-3.10476552161437\\
0.628119885334472	-3.10220440881764\\
0.617234468937876	-3.09684155551826\\
0.601202404809619	-3.08866524528832\\
0.596468517478442	-3.08617234468938\\
0.585170340681363	-3.08027895349741\\
0.569138276553106	-3.07163682472855\\
0.566442776085726	-3.07014028056112\\
0.55310621242485	-3.06280212052261\\
0.537799100333697	-3.05410821643287\\
0.537074148296593	-3.05370005159222\\
0.521042084168337	-3.0444235273177\\
0.510393744430537	-3.03807615230461\\
0.50501002004008	-3.03489325179561\\
0.488977955911824	-3.02515213976752\\
0.483995866906375	-3.02204408817635\\
0.472945891783567	-3.01520356570113\\
0.458503554440966	-3.0060120240481\\
0.456913827655311	-3.00500778702454\\
0.440881763527054	-2.99463500693657\\
0.433868346994613	-2.98997995991984\\
0.424849699398798	-2.98403536906207\\
0.409932488193913	-2.97394789579158\\
0.408817635270541	-2.97319901951363\\
0.392785571142285	-2.96219094658839\\
0.386705135174793	-2.95791583166333\\
0.376753507014028	-2.95096193545089\\
0.364065662450606	-2.94188376753507\\
0.360721442885771	-2.93950507801007\\
0.344689378757515	-2.92785282855784\\
0.341992196266275	-2.92585170340681\\
0.328657314629258	-2.91601138957255\\
0.32044819072278	-2.90981963927856\\
0.312625250501002	-2.90394942268957\\
0.299368989727407	-2.8937875751503\\
0.296593186372745	-2.89167009450756\\
0.280561122244489	-2.87919735205685\\
0.278741749184475	-2.87775551102204\\
0.264529058116232	-2.86654115588278\\
0.258542279271288	-2.86172344689379\\
0.248496993987976	-2.85367293277856\\
0.238727092594649	-2.84569138276553\\
0.232464929859719	-2.84059517615724\\
0.219276137482059	-2.82965931863727\\
0.216432865731463	-2.82731020824735\\
0.200400801603206	-2.81382306215925\\
0.200171659114422	-2.81362725450902\\
0.18436873747495	-2.80016487245945\\
0.181403709616206	-2.79759519038076\\
0.168336673346693	-2.78630243725173\\
0.162943743482434	-2.7815631262525\\
0.152304609218437	-2.77223744705891\\
0.144777250915421	-2.76553106212425\\
0.13627254509018	-2.75797143184814\\
0.126890685934725	-2.74949899799599\\
0.120240480961924	-2.74350576301438\\
0.109271399852711	-2.73346693386774\\
0.104208416833667	-2.72884165515535\\
0.0919075802425008	-2.71743486973948\\
0.0881763527054105	-2.71398016764136\\
0.0747881948941855	-2.70140280561122\\
0.0721442885771539	-2.69892220598275\\
0.0579029403073767	-2.68537074148297\\
0.0561122244488974	-2.68366852299741\\
0.0412421943145115	-2.66933867735471\\
0.0400801603206409	-2.66821971978054\\
0.0247969724703404	-2.65330661322645\\
0.0240480961923843	-2.65257624647837\\
0.00855888787954364	-2.6372745490982\\
0.00801603206412782	-2.63673840286722\\
-0.00747988583314781	-2.62124248496994\\
-0.00801603206412826	-2.62070633873896\\
-0.0233266486770157	-2.60521042084168\\
-0.0240480961923848	-2.6044800540936\\
-0.0389882058144761	-2.58917835671343\\
-0.0400801603206413	-2.58805939913926\\
-0.05447089666306	-2.57314629258517\\
-0.0561122244488979	-2.57144407409961\\
-0.0697806217774123	-2.55711422845691\\
-0.0721442885771544	-2.55463362882844\\
-0.0849228676899451	-2.54108216432866\\
-0.0881763527054109	-2.53762746223054\\
-0.0999027298721868	-2.5250501002004\\
-0.104208416833667	-2.52042482148801\\
-0.114724933963928	-2.50901803607214\\
-0.120240480961924	-2.50302480109053\\
-0.129393855403837	-2.49298597194389\\
-0.13627254509018	-2.48542634166778\\
-0.14391353758311	-2.47695390781563\\
-0.152304609218437	-2.46762822862203\\
-0.158287708632811	-2.46092184368737\\
-0.168336673346694	-2.44962909055835\\
-0.172519796945657	-2.44488977955912\\
-0.18436873747495	-2.43142739750955\\
-0.186612945524132	-2.42885771543086\\
-0.200400801603207	-2.41302145895284\\
-0.200570025238682	-2.41282565130261\\
-0.214411163717496	-2.39679358717435\\
-0.216432865731463	-2.39444447678443\\
-0.228125615835659	-2.38076152304609\\
-0.232464929859719	-2.37566531643781\\
-0.241714242995323	-2.36472945891784\\
-0.248496993987976	-2.35667894480261\\
-0.255179057332446	-2.34869739478958\\
-0.264529058116232	-2.33748303965031\\
-0.268521848787055	-2.33266533066132\\
-0.280561122244489	-2.31807510756787\\
-0.281744193712368	-2.31663326653307\\
-0.294865693023194	-2.30060120240481\\
-0.296593186372745	-2.29848372176207\\
-0.307884804426556	-2.28456913827655\\
-0.312625250501002	-2.27869892168757\\
-0.320790737016945	-2.2685370741483\\
-0.328657314629258	-2.25869676031403\\
-0.333584388029848	-2.25250501002004\\
-0.344689378757515	-2.23847407104281\\
-0.346266474301825	-2.23647294589178\\
-0.358859841022652	-2.22044088176353\\
-0.360721442885771	-2.21806219223852\\
-0.371364677459651	-2.20440881763527\\
-0.376753507014028	-2.19745492142283\\
-0.383763238799898	-2.18837675350701\\
-0.392785571142285	-2.17661980430382\\
-0.396055667160588	-2.17234468937876\\
-0.408249324480346	-2.1563126252505\\
-0.408817635270541	-2.15556374897255\\
-0.420381676516927	-2.14028056112224\\
-0.424849699398798	-2.13433597026447\\
-0.432411864046547	-2.12424849699399\\
-0.440881763527054	-2.11287147988246\\
-0.444339516511446	-2.10821643286573\\
-0.456174445099843	-2.09218436873747\\
-0.456913827655311	-2.09118013171392\\
-0.467956638280628	-2.07615230460922\\
-0.472945891783567	-2.069311782134\\
-0.47963909184413	-2.06012024048096\\
-0.488977955911824	-2.04719622794388\\
-0.491220970321727	-2.04408817635271\\
-0.502735279239629	-2.02805611222445\\
-0.50501002004008	-2.02487321171545\\
-0.514184056143404	-2.01202404809619\\
-0.521042084168337	-2.00233935898103\\
-0.525533998792644	-1.99599198396794\\
-0.53678827695295	-1.97995991983968\\
-0.537074148296593	-1.97955175499903\\
-0.548013140606755	-1.96392785571142\\
-0.55310621242485	-1.95658969567291\\
-0.559140227809761	-1.94789579158317\\
-0.569138276553106	-1.93336027162234\\
-0.570167995371698	-1.93186372745491\\
-0.581160780256625	-1.91583166332665\\
-0.585170340681363	-1.90993827213469\\
-0.592072506252331	-1.8997995991984\\
-0.601202404809619	-1.88626043566908\\
-0.60288506512914	-1.88376753507014\\
-0.613657428555808	-1.86773547094188\\
-0.617234468937876	-1.86237261764251\\
-0.624359807395399	-1.85170340681363\\
-0.633266533066132	-1.8382324554821\\
-0.634962578770196	-1.83567134268537\\
-0.645529835186418	-1.81963927855711\\
-0.649298597194389	-1.81387667185705\\
-0.656027492261569	-1.80360721442886\\
-0.665330661322646	-1.78925855592928\\
-0.666424536732041	-1.7875751503006\\
-0.676801290439637	-1.77154308617234\\
-0.681362725450902	-1.76443067766332\\
-0.687097540877721	-1.75551102204409\\
-0.697293488050765	-1.73947895791583\\
-0.697394789579158	-1.73931926820884\\
-0.707491839219817	-1.72344689378758\\
-0.713426853707415	-1.71401105067852\\
-0.717588754260136	-1.70741482965932\\
-0.727616224980459	-1.69138276553106\\
-0.729458917835671	-1.68842097989882\\
-0.737618467272952	-1.67535070140281\\
-0.745490981963928	-1.66259022427337\\
-0.747516929577984	-1.65931863727455\\
-0.757388887898833	-1.64328657314629\\
-0.761523046092184	-1.63651204421055\\
-0.767195261899139	-1.62725450901804\\
-0.776907704806536	-1.61122244488978\\
-0.777555110220441	-1.61014989124604\\
-0.786623917378867	-1.59519038076152\\
-0.793587174348697	-1.58355851759455\\
-0.796233549095041	-1.57915831663327\\
-0.8058099752019	-1.56312625250501\\
-0.809619238476954	-1.5566930590171\\
-0.815331077458949	-1.54709418837675\\
-0.824760127386751	-1.5310621242485\\
-0.82565130260521	-1.52954048970299\\
-0.834194099609312	-1.51503006012024\\
-0.841683366733467	-1.50213794937837\\
-0.843517559081993	-1.49899799599198\\
-0.852828753689511	-1.48296593186373\\
-0.857715430861723	-1.47446122603849\\
-0.862065825221398	-1.46693386773547\\
-0.871240825915441	-1.45090180360721\\
-0.87374749498998	-1.44649100489378\\
-0.880392419679554	-1.43486973947896\\
-0.889435763559499	-1.4188376753507\\
-0.889779559118236	-1.41822601676604\\
-0.898502620912698	-1.40280561122244\\
-0.905811623246493	-1.38969762513551\\
-0.907453086973702	-1.38677354709419\\
-0.916401385107147	-1.37074148296593\\
-0.92184368737475	-1.36087300710585\\
-0.925266721059645	-1.35470941883768\\
-0.934093357167528	-1.33867735470942\\
-0.937875751503006	-1.33174248081448\\
-0.942873854060321	-1.32264529058116\\
-0.951582880981771	-1.30661322645291\\
-0.953907815631263	-1.30230298107688\\
-0.96027867444484	-1.29058116232465\\
-0.968874008993936	-1.27454909819639\\
-0.969939879759519	-1.27255093581759\\
-0.977485082284089	-1.25851703406814\\
-0.985970511113727	-1.24248496993988\\
-0.985971943887776	-1.24248225378341\\
-0.99449669779832	-1.22645290581162\\
-1.00200400801603	-1.21210993397359\\
-1.00289562534923	-1.21042084168337\\
-1.01131686927922	-1.19438877755511\\
-1.01803607214429	-1.18140721482257\\
-1.0196288895712	-1.17835671342685\\
-1.02794868041008	-1.1623246492986\\
-1.03406813627255	-1.15036844889956\\
-1.03617320976842	-1.14629258517034\\
-1.04439495700578	-1.13026052104208\\
-1.0501002004008	-1.11898743521624\\
-1.05253126766383	-1.11422845691383\\
-1.06065827319264	-1.09819639278557\\
-1.06613226452906	-1.08725737084183\\
-1.06870549438125	-1.08216432865731\\
-1.07674095704631	-1.06613226452906\\
-1.08216432865731	-1.0551708299734\\
-1.0846980757674	-1.0501002004008\\
-1.09264509570445	-1.03406813627255\\
-1.09819639278557	-1.02271974124353\\
-1.10051095719356	-1.01803607214429\\
-1.10837253996919	-1.00200400801603\\
-1.11422845691383	-0.989895363173774\\
-1.11614584785142	-0.985971943887776\\
-1.12392490841295	-0.969939879759519\\
-1.13026052104208	-0.956688257674352\\
-1.13160422455531	-0.953907815631263\\
-1.13930359099971	-0.937875751503006\\
-1.14629258517034	-0.92308826148232\\
-1.14688733506242	-0.92184368737475\\
-1.15450975223249	-0.905811623246493\\
-1.16200401996448	-0.889779559118236\\
-1.1623246492986	-0.889090433191721\\
-1.16954433383618	-0.87374749498998\\
-1.17696578019152	-0.857715430861723\\
-1.17835671342685	-0.854690788684829\\
-1.18440805698392	-0.841683366733467\\
-1.19175920173928	-0.82565130260521\\
-1.19438877755511	-0.819865720098788\\
-1.19910142407344	-0.809619238476954\\
-1.20638471539175	-0.793587174348697\\
-1.21042084168337	-0.784601688074629\\
-1.21362472005891	-0.777555110220441\\
-1.22084253737021	-0.761523046092184\\
-1.22645290581162	-0.748884243791853\\
-1.22797801334242	-0.745490981963928\\
-1.23513267009027	-0.729458917835671\\
-1.24216907752019	-0.713426853707415\\
-1.24248496993988	-0.712703417267096\\
-1.24925490248686	-0.697394789579158\\
-1.2562314894554	-0.681362725450902\\
-1.25851703406814	-0.676065240973576\\
-1.26320880990842	-0.665330661322646\\
-1.27012770666234	-0.649298597194389\\
-1.27454909819639	-0.638926131462949\\
-1.27699375358044	-0.633266533066132\\
-1.28385703316352	-0.617234468937876\\
-1.29058116232465	-0.601267337607829\\
-1.29060887963702	-0.601202404809619\\
-1.29741856006962	-0.585170340681363\\
-1.30411699672397	-0.569138276553106\\
-1.30661322645291	-0.56310818675373\\
-1.31081116406888	-0.55310621242485\\
-1.31745878918003	-0.537074148296593\\
-1.32264529058116	-0.524389736561787\\
-1.32403350548239	-0.521042084168337\\
-1.33063218027391	-0.50501002004008\\
-1.33712404097122	-0.488977955911824\\
-1.33867735470942	-0.48511164222527\\
-1.34363556339904	-0.472945891783567\\
-1.3500809692004	-0.456913827655311\\
-1.35470941883768	-0.445247776242905\\
-1.35646711206576	-0.440881763527054\\
-1.36286777974099	-0.424849699398798\\
-1.3691656409182	-0.408817635270541\\
-1.37074148296593	-0.404773299370972\\
-1.37548237873076	-0.392785571142285\\
-1.38173772597015	-0.376753507014028\\
-1.38677354709419	-0.363663057257521\\
-1.3879224466588	-0.360721442885771\\
-1.39413686273957	-0.344689378757515\\
-1.40025208659613	-0.328657314629258\\
-1.40280561122244	-0.32189514718098\\
-1.40636046095951	-0.312625250501002\\
-1.41243672444534	-0.296593186372745\\
-1.41841668542161	-0.280561122244489\\
-1.4188376753507	-0.279424988336314\\
-1.42444445727107	-0.264529058116232\\
-1.43038727181292	-0.248496993987976\\
-1.43486973947896	-0.236239343140326\\
-1.43627218400033	-0.232464929859719\\
-1.44217922799749	-0.216432865731463\\
-1.44799298112941	-0.200400801603207\\
-1.45090180360721	-0.192290151612905\\
-1.4537891721569	-0.18436873747495\\
-1.45956873876195	-0.168336673346694\\
-1.46525751441528	-0.152304609218437\\
-1.46693386773547	-0.147536812403136\\
-1.47096011403291	-0.13627254509018\\
-1.47661614380372	-0.120240480961924\\
-1.48218373363104	-0.104208416833667\\
-1.48296593186373	-0.101938265686635\\
-1.48778765267569	-0.0881763527054109\\
-1.49332379819697	-0.0721442885771544\\
-1.49877371175641	-0.0561122244488979\\
-1.49899799599198	-0.0554477144216257\\
-1.50427363149831	-0.0400801603206413\\
-1.50969326695727	-0.0240480961923848\\
-1.51502874080229	-0.00801603206412826\\
-1.51503006012024	-0.00801203994315157\\
-1.52041910140116	0.00801603206412782\\
-1.52572533108552	0.0240480961923843\\
-1.53094933679874	0.0400801603206409\\
-1.5310621242485	0.0404288675879877\\
-1.53622432482812	0.0561122244488974\\
-1.54141999058174	0.0721442885771539\\
-1.546535241786	0.0881763527054105\\
-1.54709418837675	0.0899428769765749\\
-1.551688776779	0.104208416833667\\
-1.55677646444501	0.120240480961924\\
-1.56178542281927	0.13627254509018\\
-1.56312625250501	0.140606497539392\\
-1.56681114278396	0.152304609218437\\
-1.57179318765974	0.168336673346693\\
-1.57669806798192	0.18436873747495\\
-1.57915831663327	0.192505892844634\\
-1.58158931382882	0.200400801603206\\
-1.5864678051518	0.216432865731463\\
-1.59127057938552	0.232464929859719\\
-1.59519038076152	0.245738048628186\\
-1.59602037820382	0.248496993987976\\
-1.60079716268189	0.264529058116232\\
-1.6054995631189	0.280561122244489\\
-1.61012914769989	0.296593186372745\\
-1.61122244488978	0.300418455466026\\
-1.61477729462685	0.312625250501002\\
-1.61938081609172	0.328657314629258\\
-1.62391268479772	0.344689378757515\\
-1.62725450901804	0.356677572445966\\
-1.62840340858265	0.360721442885771\\
-1.63290930970114	0.376753507014028\\
-1.63734461298571	0.392785571142285\\
-1.64171073109856	0.408817635270541\\
-1.64328657314629	0.414672256156731\\
-1.64607917023272	0.424849699398798\\
-1.65041882425813	0.440881763527054\\
-1.65469018763728	0.456913827655311\\
-1.6588945800742	0.472945891783567\\
-1.65931863727455	0.474580032148419\\
-1.66312834154604	0.488977955911824\\
-1.66730552696729	0.50501002004008\\
-1.67141648023444	0.521042084168337\\
-1.67535070140281	0.536629024874541\\
-1.67546529128451	0.537074148296593\\
-1.67954863901878	0.55310621242485\\
-1.68356639100834	0.569138276553106\\
-1.68751972819312	0.585170340681363\\
-1.69138276553106	0.601090435403712\\
-1.69141048284343	0.601202404809619\\
-1.69533503186013	0.617234468937876\\
-1.69919565132171	0.633266533066132\\
-1.70299343812526	0.649298597194389\\
-1.70672945483409	0.665330661322646\\
-1.70741482965932	0.668309186823599\\
-1.71047970488146	0.681362725450902\\
-1.7141847622063	0.697394789579159\\
-1.71782833817753	0.713426853707415\\
-1.7214114151401	0.729458917835672\\
-1.72344689378758	0.738703910110772\\
-1.72497200131837	0.745490981963928\\
-1.72852233483592	0.761523046092185\\
-1.73201231025909	0.777555110220441\\
-1.73544283162421	0.793587174348698\\
-1.7388147715124	0.809619238476954\\
-1.73947895791583	0.812822767094839\\
-1.74219405406295	0.825651302605211\\
-1.7455303014729	0.841683366733467\\
-1.74880791438502	0.857715430861724\\
-1.7520276893305	0.87374749498998\\
-1.75519039270997	0.889779559118236\\
-1.75551102204409	0.891429711557732\\
-1.75836421385233	0.905811623246493\\
-1.7614878128815	0.921843687374749\\
-1.76455409200172	0.937875751503006\\
-1.76756374294779	0.953907815631262\\
-1.77051742834487	0.969939879759519\\
-1.77154308617234	0.975603791048862\\
-1.77346047710993	0.985971943887775\\
-1.77637115270639	1.00200400801603\\
-1.77922541525254	1.01803607214429\\
-1.78202385321948	1.03406813627254\\
-1.78476702669091	1.0501002004008\\
-1.78745546766551	1.06613226452906\\
-1.7875751503006	1.06685934786461\\
-1.7901483801528	1.08216432865731\\
-1.79278818988801	1.09819639278557\\
-1.79537256765055	1.11422845691383\\
-1.79790197103384	1.13026052104208\\
-1.80037682989223	1.14629258517034\\
-1.80279754654409	1.1623246492986\\
-1.80360721442886	1.16780441192895\\
-1.80520003185577	1.17835671342685\\
-1.80756567487247	1.19438877755511\\
-1.80987621161925	1.21042084168337\\
-1.8121319683394	1.22645290581162\\
-1.81433324370427	1.24248496993988\\
-1.81648030891641	1.25851703406814\\
-1.81857340779153	1.27454909819639\\
-1.81963927855711	1.28292677753027\\
-1.82063428513926	1.29058116232465\\
-1.82266346550923	1.30661322645291\\
-1.82463738111443	1.32264529058116\\
-1.82655619601566	1.33867735470942\\
-1.82842004637171	1.35470941883768\\
-1.83022904041777	1.37074148296593\\
-1.83198325842196	1.38677354709419\\
-1.83368275261984	1.40280561122244\\
-1.83532754712663	1.4188376753507\\
-1.83567134268537	1.42230599971334\\
-1.83694387071105	1.43486973947896\\
-1.83851104478601	1.45090180360721\\
-1.84002173704505	1.46693386773547\\
-1.84147588385069	1.48296593186373\\
-1.84287339223335	1.49899799599198\\
-1.84421413968947	1.51503006012024\\
-1.84549797395561	1.5310621242485\\
-1.84672471275821	1.54709418837675\\
-1.84789414353857	1.56312625250501\\
-1.84900602315273	1.57915831663327\\
-1.8500600775457	1.59519038076152\\
-1.85105600139972	1.61122244488978\\
-1.85170340681363	1.62230202888095\\
-1.85199909229475	1.62725450901804\\
-1.85289456737126	1.64328657314629\\
-1.85372935442768	1.65931863727455\\
-1.85450304230896	1.67535070140281\\
-1.85521518624693	1.69138276553106\\
-1.85586530736635	1.70741482965932\\
-1.85645289216113	1.72344689378758\\
-1.85697739193992	1.73947895791583\\
-1.85743822224045	1.75551102204409\\
-1.85783476221172	1.77154308617235\\
-1.8581663539633	1.7875751503006\\
-1.85843230188081	1.80360721442886\\
-1.85863187190667	1.81963927855711\\
-1.85876429078523	1.83567134268537\\
-1.85882874527115	1.85170340681363\\
-1.85882438130004	1.86773547094188\\
-1.85875030312025	1.88376753507014\\
-1.8586055723846	1.8997995991984\\
-1.85838920720086	1.91583166332665\\
-1.85810018113967	1.93186372745491\\
-1.85773742219854	1.94789579158317\\
-1.85729981172059	1.96392785571142\\
-1.85678618326647	1.97995991983968\\
-1.85619532143793	1.99599198396794\\
-1.85552596065151	2.01202404809619\\
-1.85477678386049	2.02805611222445\\
-1.85394642122353	2.04408817635271\\
-1.85303344871797	2.06012024048096\\
-1.85203638669594	2.07615230460922\\
-1.85170340681363	2.08110478474631\\
}--cycle;


\addplot[area legend,solid,fill=mycolor4,draw=black,forget plot]
table[row sep=crcr] {%
x	y\\
-1.61122244488978	1.92154062635485\\
-1.61011248878115	1.93186372745491\\
-1.60828067438538	1.94789579158317\\
-1.60633322202377	1.96392785571142\\
-1.60426738422376	1.97995991983968\\
-1.60208031121155	1.99599198396794\\
-1.59976904743487	2.01202404809619\\
-1.59733052793776	2.02805611222445\\
-1.59519038076152	2.04143268533249\\
-1.59476347033196	2.04408817635271\\
-1.59207201391084	2.06012024048096\\
-1.58924270019754	2.07615230460922\\
-1.58627189389273	2.09218436873747\\
-1.58315582550126	2.10821643286573\\
-1.57989058644274	2.12424849699399\\
-1.57915831663327	2.1277110612458\\
-1.5764782936904	2.14028056112224\\
-1.57290909694685	2.1563126252505\\
-1.56917665717801	2.17234468937876\\
-1.56527632738903	2.18837675350701\\
-1.56312625250501	2.19688521514885\\
-1.56120437078298	2.20440881763527\\
-1.55695444683608	2.22044088176353\\
-1.55251945803753	2.23647294589178\\
-1.5478938590338	2.25250501002004\\
-1.54709418837675	2.25519119398684\\
-1.54306685727617	2.2685370741483\\
-1.53803384825403	2.28456913827655\\
-1.53278921849702	2.30060120240481\\
-1.5310621242485	2.30571529601306\\
-1.52731554649612	2.31663326653307\\
-1.52160843687928	2.33266533066132\\
-1.5156649653976	2.34869739478958\\
-1.51503006012024	2.35036312387492\\
-1.50945191118913	2.36472945891784\\
-1.50297916329548	2.38076152304609\\
-1.49899799599198	2.39029006435056\\
-1.49622401985135	2.39679358717435\\
-1.48916536249221	2.41282565130261\\
-1.48296593186373	2.42638221231704\\
-1.48180767585431	2.42885771543086\\
-1.47409706992632	2.44488977955912\\
-1.46693386773547	2.4592173411896\\
-1.4660599246591	2.46092184368737\\
-1.45762008602662	2.47695390781563\\
-1.45090180360721	2.48925412846316\\
-1.4488063422536	2.49298597194389\\
-1.43954643592755	2.50901803607214\\
-1.43486973947896	2.51685864994032\\
-1.42983385031263	2.5250501002004\\
-1.41964657378708	2.54108216432866\\
-1.4188376753507	2.54232403453555\\
-1.40888141830834	2.55711422845691\\
-1.40280561122244	2.56585381624442\\
-1.39755434365011	2.57314629258517\\
-1.38677354709419	2.58766528351743\\
-1.38560707639808	2.58917835671343\\
-1.37292503730557	2.60521042084168\\
-1.37074148296593	2.6078980187191\\
-1.35944897462705	2.62124248496994\\
-1.35470941883768	2.62669324850733\\
-1.34510050105639	2.6372745490982\\
-1.33867735470942	2.64416791174315\\
-1.32975869254771	2.65330661322645\\
-1.32264529058116	2.660420015193\\
-1.31327547851173	2.66933867735471\\
-1.30661322645291	2.67553578751015\\
-1.29546845826235	2.68537074148297\\
-1.29058116232465	2.68959102694447\\
-1.27611129417885	2.70140280561122\\
-1.27454909819639	2.70265225874734\\
-1.25851703406814	2.71477960525793\\
-1.25478519058741	2.71743486973948\\
-1.24248496993988	2.7260271997387\\
-1.23108671830602	2.73346693386774\\
-1.22645290581162	2.73644003213387\\
-1.21042084168337	2.74606380126248\\
-1.20427225725145	2.74949899799599\\
-1.19438877755511	2.75493757065414\\
-1.17835671342685	2.76309270994689\\
-1.17317428761382	2.76553106212425\\
-1.1623246492986	2.7705684654317\\
-1.14629258517034	2.77738784786229\\
-1.13552081220309	2.7815631262525\\
-1.13026052104208	2.78357893720756\\
-1.11422845691383	2.78917463166277\\
-1.09819639278557	2.79418489914963\\
-1.08594047435841	2.79759519038076\\
-1.08216432865731	2.79863663408102\\
-1.06613226452906	2.80255991511815\\
-1.0501002004008	2.80595976525476\\
-1.03406813627255	2.80885428964474\\
-1.01803607214429	2.81126042517164\\
-1.00200400801603	2.81319399879284\\
-0.997331417501761	2.81362725450902\\
-0.985971943887776	2.814675928256\\
-0.969939879759519	2.81571419710024\\
-0.953907815631263	2.81631895101368\\
-0.937875751503006	2.81650189349093\\
-0.92184368737475	2.8162738607838\\
-0.905811623246493	2.81564485848873\\
-0.889779559118236	2.81462409541754\\
-0.878420085504249	2.81362725450902\\
-0.87374749498998	2.81322297105368\\
-0.857715430861723	2.81145662464762\\
-0.841683366733467	2.80932517220469\\
-0.82565130260521	2.80683508253215\\
-0.809619238476954	2.80399215955377\\
-0.793587174348697	2.80080156334239\\
-0.779046159184035	2.79759519038076\\
-0.777555110220441	2.79727052082353\\
-0.761523046092184	2.79343021580375\\
-0.745490981963928	2.78925838468812\\
-0.729458917835671	2.78475793913733\\
-0.718857090061259	2.7815631262525\\
-0.713426853707415	2.7799455667311\\
-0.697394789579158	2.77484091227589\\
-0.681362725450902	2.76941790865144\\
-0.670513087135676	2.76553106212425\\
-0.665330661322646	2.76369440978076\\
-0.649298597194389	2.75769375545286\\
-0.633266533066132	2.75138106085277\\
-0.628713705911684	2.74949899799599\\
-0.617234468937876	2.74480100256469\\
-0.601202404809619	2.73793042458416\\
-0.591235090961048	2.73346693386774\\
-0.585170340681363	2.7307767665998\\
-0.569138276553106	2.72336002335721\\
-0.556838055905579	2.71743486973948\\
-0.55310621242485	2.71565329829787\\
-0.537074148296593	2.70769959745392\\
-0.524849509067469	2.70140280561122\\
-0.521042084168337	2.69945820243091\\
-0.50501002004008	2.69097432997933\\
-0.494783771844951	2.68537074148297\\
-0.488977955911824	2.68221462676709\\
-0.472945891783567	2.67320501962568\\
-0.466283639724746	2.66933867735471\\
-0.456913827655311	2.66394142181914\\
-0.440881763527054	2.65440822496289\\
-0.439082402647768	2.65330661322645\\
-0.424849699398798	2.64465327283536\\
-0.413074526782555	2.6372745490982\\
-0.408817635270541	2.63462480283205\\
-0.392785571142285	2.6243608052396\\
-0.388046015352913	2.62124248496994\\
-0.376753507014028	2.6138584373754\\
-0.363888837953899	2.60521042084168\\
-0.360721442885771	2.60309375284415\\
-0.344689378757515	2.59210255044851\\
-0.340529533563966	2.58917835671343\\
-0.328657314629258	2.5808776181457\\
-0.317876518073337	2.57314629258517\\
-0.312625250501002	2.5693997148328\\
-0.296593186372745	2.55767866334882\\
-0.295837892673827	2.55711422845691\\
-0.280561122244489	2.54575051469963\\
-0.274429644828789	2.54108216432866\\
-0.264529058116232	2.53357697573616\\
-0.253532883154308	2.5250501002004\\
-0.248496993987976	2.52116109576389\\
-0.233116340787461	2.50901803607214\\
-0.232464929859719	2.50850572468606\\
-0.216432865731463	2.49564312778937\\
-0.213188106974364	2.49298597194389\\
-0.200400801603207	2.48254928829144\\
-0.193682519183802	2.47695390781563\\
-0.18436873747495	2.46922073142719\\
-0.174568524887323	2.46092184368737\\
-0.168336673346694	2.45565956781352\\
-0.155824596672748	2.44488977955912\\
-0.152304609218437	2.44186772271686\\
-0.137430801099601	2.42885771543086\\
-0.13627254509018	2.42784693855345\\
-0.120240480961924	2.41360762885972\\
-0.119374742175998	2.41282565130261\\
-0.104208416833667	2.39915131035514\\
-0.101637517500693	2.39679358717435\\
-0.0881763527054109	2.38446763068815\\
-0.0841951854019193	2.38076152304609\\
-0.0721442885771544	2.36955763291399\\
-0.0670336762908581	2.36472945891784\\
-0.0561122244488979	2.35442218420181\\
-0.0501399221213309	2.34869739478958\\
-0.0400801603206413	2.33906197695177\\
-0.0335017835616004	2.33266533066132\\
-0.0240480961923848	2.32347752968626\\
-0.0171079835705498	2.31663326653307\\
-0.00801603206412826	2.30766918771692\\
-0.000948046752016809	2.30060120240481\\
0.00801603206412782	2.29163712358867\\
0.0149877560696653	2.28456913827655\\
0.0240480961923843	2.27538133730149\\
0.0307084585490926	2.2685370741483\\
0.0400801603206409	2.25890165631049\\
0.046222445236839	2.25250501002004\\
0.0561122244488974	2.24219773530401\\
0.0615374941096744	2.23647294589178\\
0.0721442885771539	2.22526905575968\\
0.0766608155678959	2.22044088176353\\
0.0881763527054105	2.20811492527733\\
0.0915990882169153	2.20440881763527\\
0.104208416833667	2.19073447668781\\
0.106358491717682	2.18837675350701\\
0.120240480961924	2.17312666693587\\
0.120944736961574	2.17234468937876\\
0.135364739286021	2.1563126252505\\
0.13627254509018	2.15530184837309\\
0.149624586275568	2.14028056112224\\
0.152304609218437	2.13725850427999\\
0.163728008132806	2.12424849699399\\
0.168336673346693	2.11898622112013\\
0.177679279651716	2.10821643286573\\
0.18436873747495	2.10048325647729\\
0.191482314734416	2.09218436873747\\
0.200400801603206	2.08174768508503\\
0.205140685797516	2.07615230460922\\
0.216432865731463	2.06277739632644\\
0.218657641608958	2.06012024048096\\
0.232038019430153	2.04408817635271\\
0.232464929859719	2.04357586496662\\
0.245294482052848	2.02805611222445\\
0.248496993987976	2.02416710778794\\
0.258420727231885	2.01202404809619\\
0.264529058116232	2.0045188595037\\
0.271418988566259	1.99599198396794\\
0.280561122244489	1.98462827021065\\
0.284291241551083	1.97995991983968\\
0.296593186372745	1.96449229060333\\
0.297039214899656	1.96392785571142\\
0.309683479996603	1.94789579158317\\
0.312625250501002	1.94414921383079\\
0.322212717668509	1.93186372745491\\
0.328657314629258	1.92356298888719\\
0.334625405535997	1.91583166332665\\
0.344689378757515	1.90272379293348\\
0.346922486628156	1.8997995991984\\
0.359116872563305	1.88376753507014\\
0.360721442885771	1.88165086707261\\
0.371216990235094	1.86773547094188\\
0.376753507014028	1.86035142334735\\
0.383207437170299	1.85170340681363\\
0.392785571142285	1.83878966295503\\
0.395088476620054	1.83567134268537\\
0.406877006331786	1.81963927855711\\
0.408817635270541	1.81698953229097\\
0.418579485774504	1.80360721442886\\
0.424849699398798	1.79495387403776\\
0.430176934846032	1.7875751503006\\
0.440881763527054	1.77264469790878\\
0.441669019392435	1.77154308617235\\
0.453092393214548	1.75551102204409\\
0.456913827655311	1.75011376650852\\
0.464421380727239	1.73947895791583\\
0.472945891783567	1.72731323605855\\
0.475648009534167	1.72344689378758\\
0.486794006752962	1.70741482965932\\
0.488977955911824	1.70425871494344\\
0.497867416095764	1.69138276553106\\
0.50501002004008	1.68095428989917\\
0.508840595929194	1.67535070140281\\
0.519726753421353	1.65931863727455\\
0.521042084168337	1.65737403409424\\
0.530555404980253	1.64328657314629\\
0.537074148296593	1.63355130086073\\
0.54128514558374	1.62725450901804\\
0.551928074821154	1.61122244488978\\
0.55310621242485	1.60944087344817\\
0.562520884658426	1.59519038076152\\
0.569138276553106	1.585083470251\\
0.573015347633412	1.57915831663327\\
0.583430809687441	1.56312625250501\\
0.585170340681363	1.56043608523708\\
0.593794926995524	1.54709418837675\\
0.601202404809619	1.53552561496492\\
0.604060554477165	1.5310621242485\\
0.614263529626301	1.51503006012024\\
0.617234468937876	1.51033206468894\\
0.62440445512755	1.49899799599198\\
0.633266533066132	1.4848479947205\\
0.634446080277041	1.48296593186373\\
0.644450832711952	1.46693386773547\\
0.649298597194389	1.45909656106408\\
0.654372519869492	1.45090180360721\\
0.664208689869032	1.43486973947896\\
0.665330661322646	1.43303308713547\\
0.67401359899695	1.4188376753507\\
0.681362725450902	1.40669245774963\\
0.683718539634568	1.40280561122244\\
0.69337756591569	1.38677354709419\\
0.697394789579159	1.38005133311758\\
0.702969211433921	1.37074148296593\\
0.712472211428738	1.35470941883768\\
0.713426853707415	1.35309185931627\\
0.721953017014737	1.33867735470942\\
0.729458917835672	1.32584010346599\\
0.7313317452183	1.32264529058116\\
0.740677067007647	1.30661322645291\\
0.745490981963928	1.29827642076026\\
0.749947267981695	1.29058116232465\\
0.759148058525284	1.27454909819639\\
0.761523046092185	1.27038412361938\\
0.76831154460939	1.25851703406814\\
0.777372290686372	1.24248496993988\\
0.777555110220441	1.24216030038265\\
0.786430653651559	1.22645290581162\\
0.793587174348698	1.213627214645\\
0.795383112446186	1.21042084168337\\
0.804310296988349	1.19438877755511\\
0.809619238476954	1.18475368259986\\
0.813158487753533	1.17835671342685\\
0.821955812894223	1.1623246492986\\
0.825651302605211	1.15553247732173\\
0.830700749518797	1.14629258517034\\
0.839372188236657	1.13026052104208\\
0.841683366733467	1.12595843873776\\
0.848014685479281	1.11422845691383\\
0.856564069847169	1.09819639278557\\
0.857715430861724	1.09602576292418\\
0.865104748160702	1.08216432865731\\
0.873535775099905	1.06613226452906\\
0.87374749498998	1.06572798107372\\
0.881975064979866	1.0501002004008\\
0.889779559118236	1.03506497718107\\
0.890299918799153	1.03406813627254\\
0.898629447604153	1.01803607214429\\
0.905811623246493	1.00402161199574\\
0.906851749946105	1.00200400801603\\
0.915071400596516	0.985971943887775\\
0.921843687374749	0.972586486034307\\
0.923191066272462	0.969939879759519\\
0.931304129372398	0.953907815631262\\
0.937875751503006	0.94075039048492\\
0.939320894316716	0.937875751503006\\
0.947330547492817	0.921843687374749\\
0.953907815631262	0.908503319751152\\
0.955243971041258	0.905811623246493\\
0.96315328331582	0.889779559118236\\
0.969939879759519	0.8758344375812\\
0.970962750081582	0.87374749498998\\
0.978774686026535	0.857715430861724\\
0.985971943887775	0.842732040480446\\
0.986479407382706	0.841683366733467\\
0.994196831064096	0.825651302605211\\
1.00179955554845	0.809619238476954\\
1.00200400801603	0.809185982760772\\
1.00942152496197	0.793587174348698\\
1.01693390146419	0.777555110220441\\
1.01803607214429	0.775188280883066\\
1.02445030961641	0.761523046092185\\
1.03187494369613	0.745490981963928\\
1.03406813627254	0.740718017099647\\
1.03928446599602	0.729458917835672\\
1.04662387524361	0.713426853707415\\
1.0501002004008	0.705759364453157\\
1.05392501730387	0.697394789579159\\
1.0611816344267	0.681362725450902\\
1.06613226452906	0.670295386060036\\
1.06837273160182	0.665330661322646\\
1.07554890734466	0.649298597194389\\
1.08216432865731	0.634307976766395\\
1.0826281239054	0.633266533066132\\
1.08972612981108	0.617234468937876\\
1.09671960727013	0.601202404809619\\
1.09819639278557	0.597792113578492\\
1.10371348877188	0.585170340681363\\
1.11063195689279	0.569138276553106\\
1.11422845691383	0.560717717835118\\
1.11751092321085	0.55310621242485\\
1.12435644545861	0.537074148296593\\
1.13026052104208	0.523057895123392\\
1.13111812454601	0.521042084168337\\
1.13789269380047	0.50501002004008\\
1.14456803763354	0.488977955911824\\
1.14629258517034	0.484802677521606\\
1.15124007716546	0.472945891783567\\
1.1578471485081	0.456913827655311\\
1.1623246492986	0.445919166834502\\
1.16439772446391	0.440881763527054\\
1.17093832854351	0.424849699398798\\
1.17738359767137	0.408817635270541\\
1.17835671342685	0.406379283093181\\
1.18384039529476	0.392785571142285\\
1.19022155276666	0.376753507014028\\
1.19438877755511	0.366160015543915\\
1.19655191858787	0.360721442885771\\
1.20287059404896	0.344689378757515\\
1.20909740780311	0.328657314629258\\
1.21042084168337	0.325222117895742\\
1.21532898133727	0.312625250501002\\
1.22149536367499	0.296593186372745\\
1.22645290581162	0.283534220510624\\
1.2275947227263	0.280561122244489\\
1.2337021375955	0.264529058116232\\
1.23972077261114	0.248496993987976\\
1.24248496993988	0.241057259858942\\
1.24571542558249	0.232464929859719\\
1.25167685511604	0.216432865731463\\
1.25755193846325	0.200400801603206\\
1.25851703406814	0.197745537121656\\
1.2634381954357	0.18436873747495\\
1.26925765836088	0.168336673346693\\
1.27454909819639	0.153554062354551\\
1.27500191367548	0.152304609218437\\
1.28076696369929	0.13627254509018\\
1.28644814606728	0.120240480961924\\
1.29058116232465	0.108428702295168\\
1.29207665346788	0.104208416833667\\
1.29770473969571	0.0881763527054105\\
1.30325100518834	0.0721442885771539\\
1.30661322645291	0.0623093346043338\\
1.30875930199207	0.0561122244488974\\
1.31425362229937	0.0400801603206409\\
1.3196680212829	0.0240480961923843\\
1.32264529058116	0.0151294340306766\\
1.32505136586401	0.00801603206412782\\
1.3304148080451	-0.00801603206412826\\
1.33570008541116	-0.0240480961923848\\
1.33867735470942	-0.0331867976756923\\
1.34095344682717	-0.0400801603206413\\
1.34618859574639	-0.0561122244488979\\
1.35134719757311	-0.0721442885771544\\
1.35470941883768	-0.0827255891680218\\
1.35646524407426	-0.0881763527054109\\
1.36157438764822	-0.104208416833667\\
1.36660846670857	-0.120240480961924\\
1.37074148296593	-0.133584947212763\\
1.37158555308672	-0.13627254509018\\
1.3765706871207	-0.152304609218437\\
1.38148210725868	-0.168336673346694\\
1.38632129259891	-0.18436873747495\\
1.38677354709419	-0.185881810670942\\
1.39117509288114	-0.200400801603207\\
1.39596543227035	-0.216432865731463\\
1.40068469352238	-0.232464929859719\\
1.40280561122244	-0.23975740620047\\
1.40538428971324	-0.248496993987976\\
1.41005484300632	-0.264529058116232\\
1.41465534948129	-0.280561122244489\\
1.4188376753507	-0.295351316165849\\
1.4191940356332	-0.296593186372745\\
1.4237458150046	-0.312625250501002\\
1.4282284552075	-0.328657314629258\\
1.43264320670604	-0.344689378757515\\
1.43486973947896	-0.352880829017598\\
1.43703288051172	-0.360721442885771\\
1.44139826164899	-0.376753507014028\\
1.44569647964082	-0.392785571142285\\
1.44992868785174	-0.408817635270541\\
1.45090180360721	-0.412549478751272\\
1.45415805356282	-0.424849699398798\\
1.45833994169558	-0.440881763527054\\
1.46245636694497	-0.456913827655311\\
1.46650838862278	-0.472945891783567\\
1.46693386773547	-0.474650394281346\\
1.47056559973766	-0.488977955911824\\
1.47456604049386	-0.50501002004008\\
1.47850244808027	-0.521042084168337\\
1.48237579019866	-0.537074148296593\\
1.48296593186373	-0.539549651410414\\
1.48624839816075	-0.55310621242485\\
1.49006858026906	-0.569138276553106\\
1.49382589172678	-0.585170340681363\\
1.49752121047654	-0.601202404809619\\
1.49899799599198	-0.607705927633412\\
1.50119610909336	-0.617234468937876\\
1.50483636086096	-0.633266533066132\\
1.50841463880758	-0.649298597194389\\
1.51193173238232	-0.665330661322646\\
1.51503006012024	-0.679696996365562\\
1.5153950975408	-0.681362725450902\\
1.51885487702331	-0.697394789579158\\
1.52225331347104	-0.713426853707415\\
1.52559110852095	-0.729458917835671\\
1.52886893167208	-0.745490981963928\\
1.5310621242485	-0.756408952483932\\
1.53210621144258	-0.761523046092184\\
1.53532310962606	-0.777555110220441\\
1.53847964119444	-0.793587174348697\\
1.54157638824944	-0.809619238476954\\
1.54461390135166	-0.82565130260521\\
1.54709418837675	-0.838997182766671\\
1.54760165187168	-0.841683366733467\\
1.5505740365611	-0.857715430861723\\
1.55348655045289	-0.87374749498998\\
1.55633965606131	-0.889779559118236\\
1.55913378459536	-0.905811623246493\\
1.56186933611373	-0.92184368737475\\
1.56312625250501	-0.929367289861169\\
1.56457139531872	-0.937875751503006\\
1.56723569397651	-0.953907815631263\\
1.56984046882894	-0.969939879759519\\
1.57238602985503	-0.985971943887776\\
1.57487265565427	-1.00200400801603\\
1.57730059347975	-1.01803607214429\\
1.57915831663327	-1.03060557202073\\
1.57967867631418	-1.03406813627255\\
1.58202824084217	-1.0501002004008\\
1.58431784226701	-1.06613226452906\\
1.58654763393225	-1.08216432865731\\
1.58871773711821	-1.09819639278557\\
1.59082824094773	-1.11422845691383\\
1.59287920226471	-1.13026052104208\\
1.59487064548539	-1.14629258517034\\
1.59519038076152	-1.14894807619056\\
1.59682818575385	-1.1623246492986\\
1.5987296300381	-1.17835671342685\\
1.6005697919074	-1.19438877755511\\
1.60234859378189	-1.21042084168337\\
1.60406592419264	-1.22645290581162\\
1.60572163749197	-1.24248496993988\\
1.60731555353372	-1.25851703406814\\
1.60884745732288	-1.27454909819639\\
1.61031709863407	-1.29058116232465\\
1.61122244488978	-1.30090426342471\\
1.61173158890952	-1.30661322645291\\
1.61309527227241	-1.32264529058116\\
1.61439429698395	-1.33867735470942\\
1.61562829653605	-1.35470941883768\\
1.61679686674454	-1.37074148296593\\
1.61789956520109	-1.38677354709419\\
1.61893591068928	-1.40280561122244\\
1.61990538256408	-1.4188376753507\\
1.62080742009368	-1.43486973947896\\
1.62164142176267	-1.45090180360721\\
1.6224067445356	-1.46693386773547\\
1.62310270307978	-1.48296593186373\\
1.62372856894605	-1.49899799599198\\
1.62428356970646	-1.51503006012024\\
1.62476688804736	-1.5310621242485\\
1.62517766081672	-1.54709418837675\\
1.62551497802411	-1.56312625250501\\
1.62577788179188	-1.57915831663327\\
1.62596536525592	-1.59519038076152\\
1.62607637141434	-1.61122244488978\\
1.62610979192237	-1.62725450901804\\
1.62606446583149	-1.64328657314629\\
1.62593917827105	-1.65931863727455\\
1.62573265907016	-1.67535070140281\\
1.62544358131785	-1.69138276553106\\
1.62507055985917	-1.70741482965932\\
1.62461214972503	-1.72344689378758\\
1.62406684449307	-1.73947895791583\\
1.62343307457727	-1.75551102204409\\
1.62270920544335	-1.77154308617234\\
1.62189353574725	-1.7875751503006\\
1.62098429539374	-1.80360721442886\\
1.61997964351187	-1.81963927855711\\
1.61887766634416	-1.83567134268537\\
1.61767637504605	-1.85170340681363\\
1.61637370339194	-1.86773547094188\\
1.61496750538409	-1.88376753507014\\
1.61345555276042	-1.8997995991984\\
1.61183553239709	-1.91583166332665\\
1.61122244488978	-1.92154062635485\\
1.61011248878115	-1.93186372745491\\
1.60828067438538	-1.94789579158317\\
1.60633322202377	-1.96392785571142\\
1.60426738422377	-1.97995991983968\\
1.60208031121155	-1.99599198396794\\
1.59976904743487	-2.01202404809619\\
1.59733052793776	-2.02805611222445\\
1.59519038076152	-2.04143268533248\\
1.59476347033196	-2.04408817635271\\
1.59207201391084	-2.06012024048096\\
1.58924270019754	-2.07615230460922\\
1.58627189389273	-2.09218436873747\\
1.58315582550126	-2.10821643286573\\
1.57989058644274	-2.12424849699399\\
1.57915831663327	-2.1277110612458\\
1.5764782936904	-2.14028056112224\\
1.57290909694685	-2.1563126252505\\
1.56917665717801	-2.17234468937876\\
1.56527632738902	-2.18837675350701\\
1.56312625250501	-2.19688521514885\\
1.56120437078298	-2.20440881763527\\
1.55695444683608	-2.22044088176353\\
1.55251945803753	-2.23647294589178\\
1.5478938590338	-2.25250501002004\\
1.54709418837675	-2.25519119398684\\
1.54306685727617	-2.2685370741483\\
1.53803384825403	-2.28456913827655\\
1.53278921849702	-2.30060120240481\\
1.5310621242485	-2.30571529601306\\
1.52731554649612	-2.31663326653307\\
1.52160843687928	-2.33266533066132\\
1.5156649653976	-2.34869739478958\\
1.51503006012024	-2.35036312387492\\
1.50945191118913	-2.36472945891784\\
1.50297916329547	-2.38076152304609\\
1.49899799599198	-2.39029006435056\\
1.49622401985135	-2.39679358717435\\
1.48916536249221	-2.41282565130261\\
1.48296593186373	-2.42638221231704\\
1.48180767585431	-2.42885771543086\\
1.47409706992632	-2.44488977955912\\
1.46693386773547	-2.4592173411896\\
1.4660599246591	-2.46092184368737\\
1.45762008602662	-2.47695390781563\\
1.45090180360721	-2.48925412846316\\
1.4488063422536	-2.49298597194389\\
1.43954643592755	-2.50901803607214\\
1.43486973947896	-2.51685864994032\\
1.42983385031263	-2.5250501002004\\
1.41964657378708	-2.54108216432866\\
1.4188376753507	-2.54232403453555\\
1.40888141830834	-2.55711422845691\\
1.40280561122244	-2.56585381624442\\
1.39755434365011	-2.57314629258517\\
1.38677354709419	-2.58766528351743\\
1.38560707639808	-2.58917835671343\\
1.37292503730557	-2.60521042084168\\
1.37074148296593	-2.6078980187191\\
1.35944897462705	-2.62124248496994\\
1.35470941883768	-2.62669324850733\\
1.34510050105639	-2.6372745490982\\
1.33867735470942	-2.64416791174315\\
1.32975869254771	-2.65330661322645\\
1.32264529058116	-2.660420015193\\
1.31327547851173	-2.66933867735471\\
1.30661322645291	-2.67553578751015\\
1.29546845826235	-2.68537074148297\\
1.29058116232465	-2.68959102694447\\
1.27611129417885	-2.70140280561122\\
1.27454909819639	-2.70265225874734\\
1.25851703406814	-2.71477960525793\\
1.2547851905874	-2.71743486973948\\
1.24248496993988	-2.7260271997387\\
1.23108671830602	-2.73346693386774\\
1.22645290581162	-2.73644003213387\\
1.21042084168337	-2.74606380126247\\
1.20427225725145	-2.74949899799599\\
1.19438877755511	-2.75493757065414\\
1.17835671342685	-2.76309270994689\\
1.17317428761382	-2.76553106212425\\
1.1623246492986	-2.7705684654317\\
1.14629258517034	-2.77738784786229\\
1.13552081220309	-2.7815631262525\\
1.13026052104208	-2.78357893720756\\
1.11422845691383	-2.78917463166277\\
1.09819639278557	-2.79418489914963\\
1.08594047435841	-2.79759519038076\\
1.08216432865731	-2.79863663408102\\
1.06613226452906	-2.80255991511815\\
1.0501002004008	-2.80595976525476\\
1.03406813627254	-2.80885428964474\\
1.01803607214429	-2.81126042517164\\
1.00200400801603	-2.81319399879284\\
0.997331417501759	-2.81362725450902\\
0.985971943887775	-2.814675928256\\
0.969939879759519	-2.81571419710024\\
0.953907815631262	-2.81631895101368\\
0.937875751503006	-2.81650189349093\\
0.921843687374749	-2.8162738607838\\
0.905811623246493	-2.81564485848873\\
0.889779559118236	-2.81462409541754\\
0.878420085504253	-2.81362725450902\\
0.87374749498998	-2.81322297105368\\
0.857715430861724	-2.81145662464762\\
0.841683366733467	-2.80932517220469\\
0.825651302605211	-2.80683508253215\\
0.809619238476954	-2.80399215955377\\
0.793587174348698	-2.80080156334239\\
0.779046159184034	-2.79759519038076\\
0.777555110220441	-2.79727052082353\\
0.761523046092185	-2.79343021580375\\
0.745490981963928	-2.78925838468812\\
0.729458917835672	-2.78475793913733\\
0.718857090061259	-2.7815631262525\\
0.713426853707415	-2.7799455667311\\
0.697394789579159	-2.77484091227589\\
0.681362725450902	-2.76941790865144\\
0.670513087135677	-2.76553106212425\\
0.665330661322646	-2.76369440978076\\
0.649298597194389	-2.75769375545286\\
0.633266533066132	-2.75138106085277\\
0.628713705911686	-2.74949899799599\\
0.617234468937876	-2.74480100256469\\
0.601202404809619	-2.73793042458416\\
0.591235090961049	-2.73346693386774\\
0.585170340681363	-2.7307767665998\\
0.569138276553106	-2.72336002335721\\
0.556838055905581	-2.71743486973948\\
0.55310621242485	-2.71565329829787\\
0.537074148296593	-2.70769959745392\\
0.524849509067469	-2.70140280561122\\
0.521042084168337	-2.69945820243091\\
0.50501002004008	-2.69097432997933\\
0.494783771844953	-2.68537074148297\\
0.488977955911824	-2.68221462676709\\
0.472945891783567	-2.67320501962568\\
0.466283639724747	-2.66933867735471\\
0.456913827655311	-2.66394142181914\\
0.440881763527054	-2.65440822496289\\
0.439082402647767	-2.65330661322645\\
0.424849699398798	-2.64465327283535\\
0.413074526782554	-2.6372745490982\\
0.408817635270541	-2.63462480283205\\
0.392785571142285	-2.6243608052396\\
0.388046015352912	-2.62124248496994\\
0.376753507014028	-2.6138584373754\\
0.363888837953899	-2.60521042084168\\
0.360721442885771	-2.60309375284415\\
0.344689378757515	-2.59210255044851\\
0.340529533563966	-2.58917835671343\\
0.328657314629258	-2.5808776181457\\
0.317876518073336	-2.57314629258517\\
0.312625250501002	-2.5693997148328\\
0.296593186372745	-2.55767866334882\\
0.295837892673827	-2.55711422845691\\
0.280561122244489	-2.54575051469963\\
0.274429644828789	-2.54108216432866\\
0.264529058116232	-2.53357697573616\\
0.253532883154308	-2.5250501002004\\
0.248496993987976	-2.52116109576389\\
0.233116340787461	-2.50901803607214\\
0.232464929859719	-2.50850572468606\\
0.216432865731463	-2.49564312778937\\
0.213188106974364	-2.49298597194389\\
0.200400801603206	-2.48254928829144\\
0.193682519183801	-2.47695390781563\\
0.18436873747495	-2.46922073142719\\
0.174568524887323	-2.46092184368737\\
0.168336673346693	-2.45565956781352\\
0.155824596672748	-2.44488977955912\\
0.152304609218437	-2.44186772271686\\
0.137430801099602	-2.42885771543086\\
0.13627254509018	-2.42784693855345\\
0.120240480961924	-2.41360762885972\\
0.119374742175997	-2.41282565130261\\
0.104208416833667	-2.39915131035514\\
0.101637517500693	-2.39679358717435\\
0.0881763527054105	-2.38446763068815\\
0.0841951854019198	-2.38076152304609\\
0.0721442885771539	-2.36955763291399\\
0.0670336762908576	-2.36472945891784\\
0.0561122244488974	-2.35442218420181\\
0.0501399221213309	-2.34869739478958\\
0.0400801603206409	-2.33906197695177\\
0.0335017835616004	-2.33266533066132\\
0.0240480961923843	-2.32347752968626\\
0.0171079835705493	-2.31663326653307\\
0.00801603206412782	-2.30766918771692\\
0.00094804675201682	-2.30060120240481\\
-0.00801603206412826	-2.29163712358866\\
-0.0149877560696652	-2.28456913827655\\
-0.0240480961923848	-2.27538133730149\\
-0.0307084585490926	-2.2685370741483\\
-0.0400801603206413	-2.25890165631049\\
-0.046222445236839	-2.25250501002004\\
-0.0561122244488979	-2.24219773530401\\
-0.0615374941096744	-2.23647294589178\\
-0.0721442885771544	-2.22526905575968\\
-0.0766608155678955	-2.22044088176353\\
-0.0881763527054109	-2.20811492527733\\
-0.0915990882169157	-2.20440881763527\\
-0.104208416833667	-2.19073447668781\\
-0.106358491717682	-2.18837675350701\\
-0.120240480961924	-2.17312666693587\\
-0.120944736961575	-2.17234468937876\\
-0.13536473928602	-2.1563126252505\\
-0.13627254509018	-2.15530184837309\\
-0.149624586275567	-2.14028056112224\\
-0.152304609218437	-2.13725850427999\\
-0.163728008132807	-2.12424849699399\\
-0.168336673346694	-2.11898622112013\\
-0.177679279651716	-2.10821643286573\\
-0.18436873747495	-2.10048325647729\\
-0.191482314734416	-2.09218436873747\\
-0.200400801603207	-2.08174768508503\\
-0.205140685797516	-2.07615230460922\\
-0.216432865731463	-2.06277739632644\\
-0.218657641608958	-2.06012024048096\\
-0.232038019430153	-2.04408817635271\\
-0.232464929859719	-2.04357586496662\\
-0.245294482052848	-2.02805611222445\\
-0.248496993987976	-2.02416710778794\\
-0.258420727231884	-2.01202404809619\\
-0.264529058116232	-2.0045188595037\\
-0.271418988566259	-1.99599198396794\\
-0.280561122244489	-1.98462827021066\\
-0.284291241551083	-1.97995991983968\\
-0.296593186372745	-1.96449229060333\\
-0.297039214899656	-1.96392785571142\\
-0.309683479996603	-1.94789579158317\\
-0.312625250501002	-1.94414921383079\\
-0.322212717668508	-1.93186372745491\\
-0.328657314629258	-1.92356298888719\\
-0.334625405535997	-1.91583166332665\\
-0.344689378757515	-1.90272379293348\\
-0.346922486628155	-1.8997995991984\\
-0.359116872563304	-1.88376753507014\\
-0.360721442885771	-1.88165086707261\\
-0.371216990235094	-1.86773547094188\\
-0.376753507014028	-1.86035142334735\\
-0.383207437170299	-1.85170340681363\\
-0.392785571142285	-1.83878966295503\\
-0.395088476620055	-1.83567134268537\\
-0.406877006331786	-1.81963927855711\\
-0.408817635270541	-1.81698953229097\\
-0.418579485774505	-1.80360721442886\\
-0.424849699398798	-1.79495387403776\\
-0.430176934846032	-1.7875751503006\\
-0.440881763527054	-1.77264469790878\\
-0.441669019392435	-1.77154308617234\\
-0.453092393214548	-1.75551102204409\\
-0.456913827655311	-1.75011376650852\\
-0.46442138072724	-1.73947895791583\\
-0.472945891783567	-1.72731323605855\\
-0.475648009534168	-1.72344689378758\\
-0.486794006752962	-1.70741482965932\\
-0.488977955911824	-1.70425871494344\\
-0.497867416095764	-1.69138276553106\\
-0.50501002004008	-1.68095428989917\\
-0.508840595929195	-1.67535070140281\\
-0.519726753421353	-1.65931863727455\\
-0.521042084168337	-1.65737403409424\\
-0.530555404980253	-1.64328657314629\\
-0.537074148296593	-1.63355130086073\\
-0.54128514558374	-1.62725450901804\\
-0.551928074821154	-1.61122244488978\\
-0.55310621242485	-1.60944087344817\\
-0.562520884658426	-1.59519038076152\\
-0.569138276553106	-1.585083470251\\
-0.573015347633412	-1.57915831663327\\
-0.583430809687441	-1.56312625250501\\
-0.585170340681363	-1.56043608523708\\
-0.593794926995524	-1.54709418837675\\
-0.601202404809619	-1.53552561496492\\
-0.604060554477165	-1.5310621242485\\
-0.614263529626301	-1.51503006012024\\
-0.617234468937876	-1.51033206468894\\
-0.62440445512755	-1.49899799599198\\
-0.633266533066132	-1.4848479947205\\
-0.634446080277042	-1.48296593186373\\
-0.644450832711952	-1.46693386773547\\
-0.649298597194389	-1.45909656106408\\
-0.654372519869492	-1.45090180360721\\
-0.664208689869032	-1.43486973947896\\
-0.665330661322646	-1.43303308713547\\
-0.67401359899695	-1.4188376753507\\
-0.681362725450902	-1.40669245774963\\
-0.683718539634568	-1.40280561122244\\
-0.693377565915689	-1.38677354709419\\
-0.697394789579158	-1.38005133311758\\
-0.702969211433921	-1.37074148296593\\
-0.712472211428738	-1.35470941883768\\
-0.713426853707415	-1.35309185931627\\
-0.721953017014737	-1.33867735470942\\
-0.729458917835671	-1.32584010346599\\
-0.7313317452183	-1.32264529058116\\
-0.740677067007647	-1.30661322645291\\
-0.745490981963928	-1.29827642076026\\
-0.749947267981696	-1.29058116232465\\
-0.759148058525283	-1.27454909819639\\
-0.761523046092184	-1.27038412361938\\
-0.76831154460939	-1.25851703406814\\
-0.777372290686371	-1.24248496993988\\
-0.777555110220441	-1.24216030038265\\
-0.786430653651558	-1.22645290581162\\
-0.793587174348697	-1.213627214645\\
-0.795383112446185	-1.21042084168337\\
-0.804310296988349	-1.19438877755511\\
-0.809619238476954	-1.18475368259986\\
-0.813158487753533	-1.17835671342685\\
-0.821955812894222	-1.1623246492986\\
-0.82565130260521	-1.15553247732173\\
-0.830700749518797	-1.14629258517034\\
-0.839372188236656	-1.13026052104208\\
-0.841683366733467	-1.12595843873776\\
-0.848014685479281	-1.11422845691383\\
-0.856564069847169	-1.09819639278557\\
-0.857715430861723	-1.09602576292418\\
-0.865104748160702	-1.08216432865731\\
-0.873535775099906	-1.06613226452906\\
-0.87374749498998	-1.06572798107372\\
-0.881975064979865	-1.0501002004008\\
-0.889779559118236	-1.03506497718107\\
-0.890299918799153	-1.03406813627255\\
-0.898629447604153	-1.01803607214429\\
-0.905811623246493	-1.00402161199574\\
-0.906851749946105	-1.00200400801603\\
-0.915071400596516	-0.985971943887776\\
-0.92184368737475	-0.972586486034305\\
-0.923191066272462	-0.969939879759519\\
-0.931304129372398	-0.953907815631263\\
-0.937875751503006	-0.94075039048492\\
-0.939320894316716	-0.937875751503006\\
-0.947330547492816	-0.92184368737475\\
-0.953907815631263	-0.908503319751151\\
-0.955243971041259	-0.905811623246493\\
-0.96315328331582	-0.889779559118236\\
-0.969939879759519	-0.875834437581199\\
-0.970962750081582	-0.87374749498998\\
-0.978774686026536	-0.857715430861723\\
-0.985971943887776	-0.842732040480445\\
-0.986479407382707	-0.841683366733467\\
-0.994196831064096	-0.82565130260521\\
-1.00179955554845	-0.809619238476954\\
-1.00200400801603	-0.809185982760772\\
-1.00942152496197	-0.793587174348697\\
-1.01693390146419	-0.777555110220441\\
-1.01803607214429	-0.775188280883064\\
-1.02445030961641	-0.761523046092184\\
-1.03187494369613	-0.745490981963928\\
-1.03406813627255	-0.740718017099646\\
-1.03928446599602	-0.729458917835671\\
-1.04662387524361	-0.713426853707415\\
-1.0501002004008	-0.705759364453155\\
-1.05392501730387	-0.697394789579158\\
-1.0611816344267	-0.681362725450902\\
-1.06613226452906	-0.670295386060035\\
-1.06837273160182	-0.665330661322646\\
-1.07554890734466	-0.649298597194389\\
-1.08216432865731	-0.634307976766393\\
-1.0826281239054	-0.633266533066132\\
-1.08972612981108	-0.617234468937876\\
-1.09671960727013	-0.601202404809619\\
-1.09819639278557	-0.59779211357849\\
-1.10371348877188	-0.585170340681363\\
-1.11063195689279	-0.569138276553106\\
-1.11422845691383	-0.560717717835118\\
-1.11751092321085	-0.55310621242485\\
-1.12435644545861	-0.537074148296593\\
-1.13026052104208	-0.523057895123392\\
-1.13111812454601	-0.521042084168337\\
-1.13789269380047	-0.50501002004008\\
-1.14456803763354	-0.488977955911824\\
-1.14629258517034	-0.484802677521605\\
-1.15124007716546	-0.472945891783567\\
-1.1578471485081	-0.456913827655311\\
-1.1623246492986	-0.445919166834502\\
-1.16439772446391	-0.440881763527054\\
-1.17093832854351	-0.424849699398798\\
-1.17738359767137	-0.408817635270541\\
-1.17835671342685	-0.406379283093181\\
-1.18384039529476	-0.392785571142285\\
-1.19022155276666	-0.376753507014028\\
-1.19438877755511	-0.366160015543915\\
-1.19655191858787	-0.360721442885771\\
-1.20287059404896	-0.344689378757515\\
-1.20909740780311	-0.328657314629258\\
-1.21042084168337	-0.325222117895742\\
-1.21532898133727	-0.312625250501002\\
-1.22149536367499	-0.296593186372745\\
-1.22645290581162	-0.283534220510624\\
-1.2275947227263	-0.280561122244489\\
-1.2337021375955	-0.264529058116232\\
-1.23972077261114	-0.248496993987976\\
-1.24248496993988	-0.241057259858942\\
-1.24571542558249	-0.232464929859719\\
-1.25167685511604	-0.216432865731463\\
-1.25755193846325	-0.200400801603207\\
-1.25851703406814	-0.197745537121656\\
-1.2634381954357	-0.18436873747495\\
-1.26925765836088	-0.168336673346694\\
-1.27454909819639	-0.153554062354551\\
-1.27500191367548	-0.152304609218437\\
-1.28076696369929	-0.13627254509018\\
-1.28644814606728	-0.120240480961924\\
-1.29058116232465	-0.108428702295168\\
-1.29207665346788	-0.104208416833667\\
-1.29770473969571	-0.0881763527054109\\
-1.30325100518834	-0.0721442885771544\\
-1.30661322645291	-0.0623093346043342\\
-1.30875930199207	-0.0561122244488979\\
-1.31425362229937	-0.0400801603206413\\
-1.3196680212829	-0.0240480961923848\\
-1.32264529058116	-0.0151294340306766\\
-1.32505136586401	-0.00801603206412826\\
-1.3304148080451	0.00801603206412782\\
-1.33570008541116	0.0240480961923843\\
-1.33867735470942	0.0331867976756918\\
-1.34095344682717	0.0400801603206409\\
-1.34618859574639	0.0561122244488974\\
-1.35134719757311	0.0721442885771539\\
-1.35470941883768	0.0827255891680218\\
-1.35646524407426	0.0881763527054105\\
-1.36157438764822	0.104208416833667\\
-1.36660846670857	0.120240480961924\\
-1.37074148296593	0.133584947212763\\
-1.37158555308672	0.13627254509018\\
-1.3765706871207	0.152304609218437\\
-1.38148210725868	0.168336673346693\\
-1.38632129259891	0.18436873747495\\
-1.38677354709419	0.185881810670943\\
-1.39117509288114	0.200400801603206\\
-1.39596543227035	0.216432865731463\\
-1.40068469352238	0.232464929859719\\
-1.40280561122244	0.23975740620047\\
-1.40538428971324	0.248496993987976\\
-1.41005484300632	0.264529058116232\\
-1.41465534948129	0.280561122244489\\
-1.4188376753507	0.29535131616585\\
-1.4191940356332	0.296593186372745\\
-1.4237458150046	0.312625250501002\\
-1.4282284552075	0.328657314629258\\
-1.43264320670604	0.344689378757515\\
-1.43486973947896	0.352880829017598\\
-1.43703288051172	0.360721442885771\\
-1.44139826164899	0.376753507014028\\
-1.44569647964082	0.392785571142285\\
-1.44992868785174	0.408817635270541\\
-1.45090180360721	0.412549478751272\\
-1.45415805356282	0.424849699398798\\
-1.45833994169558	0.440881763527054\\
-1.46245636694497	0.456913827655311\\
-1.46650838862278	0.472945891783567\\
-1.46693386773547	0.474650394281346\\
-1.47056559973766	0.488977955911824\\
-1.47456604049386	0.50501002004008\\
-1.47850244808027	0.521042084168337\\
-1.48237579019867	0.537074148296593\\
-1.48296593186373	0.539549651410414\\
-1.48624839816075	0.55310621242485\\
-1.49006858026906	0.569138276553106\\
-1.49382589172678	0.585170340681363\\
-1.49752121047654	0.601202404809619\\
-1.49899799599198	0.607705927633412\\
-1.50119610909336	0.617234468937876\\
-1.50483636086096	0.633266533066132\\
-1.50841463880758	0.649298597194389\\
-1.51193173238232	0.665330661322646\\
-1.51503006012024	0.679696996365563\\
-1.5153950975408	0.681362725450902\\
-1.51885487702331	0.697394789579159\\
-1.52225331347104	0.713426853707415\\
-1.52559110852095	0.729458917835672\\
-1.52886893167208	0.745490981963928\\
-1.5310621242485	0.756408952483931\\
-1.53210621144258	0.761523046092185\\
-1.53532310962606	0.777555110220441\\
-1.53847964119444	0.793587174348698\\
-1.54157638824944	0.809619238476954\\
-1.54461390135166	0.825651302605211\\
-1.54709418837675	0.838997182766671\\
-1.54760165187168	0.841683366733467\\
-1.5505740365611	0.857715430861724\\
-1.55348655045289	0.87374749498998\\
-1.55633965606131	0.889779559118236\\
-1.55913378459536	0.905811623246493\\
-1.56186933611373	0.921843687374749\\
-1.56312625250501	0.929367289861167\\
-1.56457139531872	0.937875751503006\\
-1.56723569397651	0.953907815631262\\
-1.56984046882894	0.969939879759519\\
-1.57238602985503	0.985971943887775\\
-1.57487265565427	1.00200400801603\\
-1.57730059347975	1.01803607214429\\
-1.57915831663327	1.03060557202073\\
-1.57967867631418	1.03406813627254\\
-1.58202824084217	1.0501002004008\\
-1.58431784226701	1.06613226452906\\
-1.58654763393225	1.08216432865731\\
-1.58871773711821	1.09819639278557\\
-1.59082824094773	1.11422845691383\\
-1.59287920226471	1.13026052104208\\
-1.59487064548539	1.14629258517034\\
-1.59519038076152	1.14894807619056\\
-1.59682818575385	1.1623246492986\\
-1.5987296300381	1.17835671342685\\
-1.6005697919074	1.19438877755511\\
-1.60234859378189	1.21042084168337\\
-1.60406592419264	1.22645290581162\\
-1.60572163749197	1.24248496993988\\
-1.60731555353372	1.25851703406814\\
-1.60884745732288	1.27454909819639\\
-1.61031709863407	1.29058116232465\\
-1.61122244488978	1.30090426342471\\
-1.61173158890952	1.30661322645291\\
-1.61309527227241	1.32264529058116\\
-1.61439429698395	1.33867735470942\\
-1.61562829653605	1.35470941883768\\
-1.61679686674454	1.37074148296593\\
-1.61789956520109	1.38677354709419\\
-1.61893591068928	1.40280561122244\\
-1.61990538256409	1.4188376753507\\
-1.62080742009368	1.43486973947896\\
-1.62164142176267	1.45090180360721\\
-1.6224067445356	1.46693386773547\\
-1.62310270307978	1.48296593186373\\
-1.62372856894605	1.49899799599198\\
-1.62428356970646	1.51503006012024\\
-1.62476688804736	1.5310621242485\\
-1.62517766081672	1.54709418837675\\
-1.62551497802411	1.56312625250501\\
-1.62577788179189	1.57915831663327\\
-1.62596536525592	1.59519038076152\\
-1.62607637141434	1.61122244488978\\
-1.62610979192237	1.62725450901804\\
-1.62606446583149	1.64328657314629\\
-1.62593917827105	1.65931863727455\\
-1.62573265907016	1.67535070140281\\
-1.62544358131785	1.69138276553106\\
-1.62507055985917	1.70741482965932\\
-1.62461214972503	1.72344689378758\\
-1.62406684449307	1.73947895791583\\
-1.62343307457727	1.75551102204409\\
-1.62270920544335	1.77154308617235\\
-1.62189353574725	1.7875751503006\\
-1.62098429539374	1.80360721442886\\
-1.61997964351187	1.81963927855711\\
-1.61887766634416	1.83567134268537\\
-1.61767637504605	1.85170340681363\\
-1.61637370339194	1.86773547094188\\
-1.61496750538409	1.88376753507014\\
-1.61345555276042	1.8997995991984\\
-1.61183553239709	1.91583166332665\\
-1.61122244488978	1.92154062635485\\
}--cycle;


\addplot[area legend,solid,fill=mycolor5,draw=black,forget plot]
table[row sep=crcr] {%
x	y\\
-1.43486973947896	1.57614782301339\\
-1.43471943107953	1.57915831663327\\
-1.43381826014665	1.59519038076152\\
-1.43280952025882	1.61122244488978\\
-1.43169085204059	1.62725450901804\\
-1.4304598001939	1.64328657314629\\
-1.42911381027157	1.65931863727455\\
-1.4276502253059	1.67535070140281\\
-1.42606628228583	1.69138276553106\\
-1.42435910847566	1.70741482965932\\
-1.4225257175681	1.72344689378758\\
-1.42056300566399	1.73947895791583\\
-1.4188376753507	1.75270116524991\\
-1.41846804884804	1.75551102204409\\
-1.41623805524873	1.77154308617235\\
-1.41386758907841	1.7875751503006\\
-1.41135291389638	1.80360721442886\\
-1.40869014715433	1.81963927855711\\
-1.40587525464174	1.83567134268537\\
-1.40290404467388	1.85170340681363\\
-1.40280561122244	1.85221236475073\\
-1.39976673853409	1.86773547094188\\
-1.39646204917176	1.88376753507014\\
-1.39298525898563	1.8997995991984\\
-1.38933130901527	1.91583166332665\\
-1.38677354709419	1.92656076872958\\
-1.38549017444567	1.93186372745491\\
-1.38144940352448	1.94789579158317\\
-1.37721190834076	1.96392785571142\\
-1.37277155563299	1.97995991983968\\
-1.37074148296593	1.98700845858851\\
-1.36810682049629	1.99599198396794\\
-1.36321095195326	2.01202404809619\\
-1.35808788971171	2.02805611222445\\
-1.35470941883768	2.03821482835606\\
-1.35271501593282	2.04408817635271\\
-1.34706954954247	2.06012024048096\\
-1.34116778578055	2.07615230460922\\
-1.33867735470942	2.08268121363305\\
-1.33496551159711	2.09218436873747\\
-1.3284588477304	2.10821643286573\\
-1.32264529058116	2.12195623172791\\
-1.32164969012634	2.12424849699399\\
-1.31446628330718	2.14028056112224\\
-1.30696119426195	2.1563126252505\\
-1.30661322645291	2.1570351362201\\
-1.29902049060337	2.17234468937876\\
-1.29071804569593	2.18837675350701\\
-1.29058116232465	2.18863394640411\\
-1.28191145320043	2.20440881763527\\
-1.27454909819639	2.2172723129112\\
-1.27267209655242	2.22044088176353\\
-1.26287839719742	2.23647294589178\\
-1.25851703406814	2.2433713531395\\
-1.25252169552617	2.25250501002004\\
-1.24248496993988	2.26725508244213\\
-1.24157692121844	2.2685370741483\\
-1.2298926066466	2.28456913827655\\
-1.22645290581162	2.28914032936524\\
-1.2174438451519	2.30060120240481\\
-1.21042084168337	2.3092557276721\\
-1.20414790670024	2.31663326653307\\
-1.19438877755511	2.32776952279695\\
-1.18987869651441	2.33266533066132\\
-1.17835671342685	2.34481988990147\\
-1.17447920853874	2.34869739478958\\
-1.1623246492986	2.36052717493355\\
-1.15775251694412	2.36472945891784\\
-1.14629258517034	2.37499623458497\\
-1.13944940232465	2.38076152304609\\
-1.13026052104208	2.38831840762883\\
-1.11925105511328	2.39679358717435\\
-1.11422845691383	2.40057318173519\\
-1.09819639278557	2.41182795528761\\
-1.09667379717982	2.41282565130261\\
-1.08216432865731	2.42214004892704\\
-1.07080768800794	2.42885771543086\\
-1.06613226452906	2.4315707992682\\
-1.0501002004008	2.4401641685971\\
-1.0404549230817	2.44488977955912\\
-1.03406813627255	2.4479659469966\\
-1.01803607214429	2.45501471015489\\
-1.00309160510989	2.46092184368737\\
-1.00200400801603	2.46134532663581\\
-0.985971943887776	2.46699364640741\\
-0.969939879759519	2.47198542314756\\
-0.953907815631263	2.47634725722471\\
-0.951348053022255	2.47695390781563\\
-0.937875751503006	2.48011007091374\\
-0.92184368737475	2.48329258491622\\
-0.905811623246493	2.48591508836163\\
-0.889779559118236	2.48799746033493\\
-0.87374749498998	2.4895582254375\\
-0.857715430861723	2.49061462510628\\
-0.841683366733467	2.49118268369602\\
-0.82565130260521	2.49127726968805\\
-0.809619238476954	2.49091215235856\\
-0.793587174348697	2.49010005421127\\
-0.777555110220441	2.48885269945374\\
-0.761523046092184	2.48718085877326\\
-0.745490981963928	2.48509439064656\\
-0.729458917835671	2.48260227939784\\
-0.713426853707415	2.47971267020162\\
-0.699954552188166	2.47695390781563\\
-0.697394789579158	2.47643563232263\\
-0.681362725450902	2.47279233708659\\
-0.665330661322646	2.46877490054165\\
-0.649298597194389	2.46438859305215\\
-0.637613172394768	2.46092184368737\\
-0.633266533066132	2.45964550325484\\
-0.617234468937876	2.45456571746513\\
-0.601202404809619	2.44913301662021\\
-0.589448840589472	2.44488977955912\\
-0.585170340681363	2.44335978778237\\
-0.569138276553106	2.43726879886481\\
-0.55310621242485	2.43083597158032\\
-0.548430788945969	2.42885771543086\\
-0.537074148296593	2.42409432489073\\
-0.521042084168337	2.41703015422659\\
-0.511937632205333	2.41282565130261\\
-0.50501002004008	2.40965263427974\\
-0.488977955911824	2.40197261032635\\
-0.47861633688666	2.39679358717435\\
-0.472945891783567	2.39398111250709\\
-0.456913827655311	2.38569774828604\\
-0.447724946372745	2.38076152304609\\
-0.440881763527054	2.37711175756346\\
-0.424849699398798	2.36823470705649\\
-0.418742219050504	2.36472945891784\\
-0.408817635270541	2.35907140958386\\
-0.392785571142285	2.34960756594041\\
-0.391293580145418	2.34869739478958\\
-0.376753507014028	2.33988194566119\\
-0.365231523926469	2.33266533066132\\
-0.360721442885771	2.3298571615125\\
-0.344689378757515	2.31956044169135\\
-0.340265649230689	2.31663326653307\\
-0.328657314629258	2.30899339723786\\
-0.316276698443884	2.30060120240481\\
-0.312625250501002	2.298138761015\\
-0.296593186372745	2.28701987652264\\
-0.293153485537765	2.28456913827655\\
-0.280561122244489	2.27563866976863\\
-0.270812795449883	2.2685370741483\\
-0.264529058116232	2.26397936875109\\
-0.249116841161576	2.25250501002004\\
-0.248496993987976	2.25204542916615\\
-0.232464929859719	2.23986734300685\\
-0.228103566730434	2.23647294589178\\
-0.216432865731463	2.22742240388654\\
-0.20764169030169	2.22044088176353\\
-0.200400801603207	2.21470970355198\\
-0.187683817264239	2.20440881763527\\
-0.18436873747495	2.20173183890034\\
-0.168336673346694	2.18849214306497\\
-0.16819978997541	2.18837675350701\\
-0.152304609218437	2.17501187444054\\
-0.149198527430876	2.17234468937876\\
-0.13627254509018	2.16127082520807\\
-0.13060223360748	2.1563126252505\\
-0.120240480961924	2.14727077441581\\
-0.112387424107648	2.14028056112224\\
-0.104208416833667	2.13301329770614\\
-0.0945324789276519	2.12424849699399\\
-0.0881763527054109	2.11849976935194\\
-0.0770174803552489	2.10821643286573\\
-0.0721442885771544	2.10373136403057\\
-0.0598240675612109	2.09218436873747\\
-0.0561122244488979	2.08870905833842\\
-0.042935305942095	2.07615230460922\\
-0.0400801603206413	2.07343363204895\\
-0.0263355695258069	2.06012024048096\\
-0.0240480961923848	2.05790566911636\\
-0.0100104349689827	2.04408817635271\\
-0.00801603206412826	2.04212555842685\\
0.00605341413827421	2.02805611222445\\
0.00801603206412782	2.02609349429859\\
0.0218682737844108	2.01202404809619\\
0.0240480961923843	2.0098094767316\\
0.0374454978510013	1.99599198396794\\
0.0400801603206409	1.99327331140766\\
0.0527955690962666	1.97995991983968\\
0.0561122244488974	1.97648460944062\\
0.0679281635703945	1.96392785571142\\
0.0721442885771539	1.95944278687626\\
0.082852209135698	1.94789579158317\\
0.0881763527054105	1.94214706394112\\
0.0975759386864971	1.93186372745491\\
0.104208416833667	1.9245964640388\\
0.112106938595429	1.91583166332665\\
0.120240480961924	1.90678981249196\\
0.126452192853365	1.8997995991984\\
0.13627254509018	1.88872573502771\\
0.14061812331798	1.88376753507014\\
0.152304609218437	1.87040265600366\\
0.154610626440267	1.86773547094188\\
0.168336673346693	1.85181879637159\\
0.16843510679813	1.85170340681363\\
0.182094263918885	1.83567134268537\\
0.18436873747495	1.83299436395044\\
0.195597114389196	1.81963927855711\\
0.200400801603206	1.81390810034557\\
0.208948104277145	1.80360721442886\\
0.216432865731463	1.79455667242361\\
0.222151133760788	1.7875751503006\\
0.232464929859719	1.77493748328741\\
0.235209729534101	1.77154308617235\\
0.248127367485314	1.75551102204409\\
0.248496993987976	1.7550514411902\\
0.260910749817074	1.73947895791583\\
0.264529058116232	1.73492125251862\\
0.273561692861503	1.72344689378758\\
0.280561122244489	1.71451642527965\\
0.286082555369444	1.70741482965932\\
0.296593186372745	1.69383350377715\\
0.298475405129103	1.69138276553106\\
0.31074681341069	1.67535070140281\\
0.312625250501002	1.672888260013\\
0.322901385421872	1.65931863727455\\
0.328657314629258	1.65167876797935\\
0.334936639563524	1.64328657314629\\
0.344689378757515	1.63018168417632\\
0.346853715889065	1.62725450901804\\
0.358661223665631	1.61122244488978\\
0.360721442885771	1.60841427574096\\
0.370363623606131	1.59519038076152\\
0.376753507014028	1.58637493163314\\
0.381954431452667	1.57915831663327\\
0.392785571142285	1.56403642365584\\
0.393433997955205	1.56312625250501\\
0.404822061190901	1.54709418837675\\
0.408817635270541	1.54143613904278\\
0.416106673164823	1.5310621242485\\
0.424849699398798	1.51853530825889\\
0.427284817757179	1.51503006012024\\
0.438370195933025	1.49899799599198\\
0.440881763527054	1.49534823050935\\
0.44936577793113	1.48296593186373\\
0.456913827655311	1.47187009297541\\
0.460258527737727	1.46693386773547\\
0.47105948890644	1.45090180360721\\
0.472945891783567	1.44808932893996\\
0.481780563993476	1.43486973947896\\
0.488977955911824	1.42401669850271\\
0.492401403989198	1.4188376753507\\
0.502935212529043	1.40280561122244\\
0.50501002004008	1.39963259419958\\
0.513393889713681	1.38677354709419\\
0.521042084168337	1.37494598588992\\
0.523753976742535	1.37074148296593\\
0.534036970397823	1.35470941883768\\
0.537074148296593	1.34994602829754\\
0.54424314245709	1.33867735470942\\
0.55310621242485	1.32462354673062\\
0.554351490046198	1.32264529058116\\
0.564399151523312	1.30661322645291\\
0.569138276553106	1.29899224575859\\
0.574360665418839	1.29058116232465\\
0.584232230108001	1.27454909819639\\
0.585170340681363	1.27301910641964\\
0.594051326219487	1.25851703406814\\
0.601202404809619	1.24672820700097\\
0.603774129129969	1.24248496993988\\
0.613432791996306	1.22645290581162\\
0.617234468937876	1.22009677958938\\
0.62301858987926	1.21042084168337\\
0.632513854057083	1.19438877755511\\
0.633266533066132	1.19311243712257\\
0.641966158922827	1.17835671342685\\
0.649298597194389	1.16579139866338\\
0.651321859979967	1.1623246492986\\
0.660624815805226	1.14629258517034\\
0.665330661322646	1.13811357789636\\
0.669850938713261	1.13026052104208\\
0.679002064889276	1.11422845691383\\
0.681362725450902	1.11006688618478\\
0.688101307851878	1.09819639278557\\
0.69710495342446	1.08216432865731\\
0.697394789579159	1.08164605316432\\
0.706079738533738	1.06613226452906\\
0.713426853707415	1.05285896278679\\
0.714956057633846	1.0501002004008\\
0.723792570976108	1.03406813627254\\
0.729458917835672	1.0236844437265\\
0.732546518759912	1.01803607214429\\
0.741245729576007	1.00200400801603\\
0.745490981963928	0.994112426718699\\
0.749879009420214	0.985971943887775\\
0.75844473699409	0.969939879759519\\
0.761523046092185	0.964134766588891\\
0.766958806572806	0.953907815631262\\
0.775394727267325	0.937875751503006\\
0.777555110220441	0.933742479012859\\
0.783790805290208	0.921843687374749\\
0.79210045799249	0.905811623246493\\
0.793587174348698	0.902925705513879\\
0.800379530336358	0.889779559118236\\
0.808566321619408	0.87374749498998\\
0.809619238476954	0.871673675404656\\
0.816729146876445	0.857715430861724\\
0.824796355889865	0.841683366733467\\
0.825651302605211	0.839974664477626\\
0.832843470353536	0.825651302605211\\
0.840794253455332	0.809619238476954\\
0.841683366733467	0.807815950229083\\
0.848725975563253	0.793587174348698\\
0.856563370704004	0.777555110220441\\
0.857715430861724	0.775183763382832\\
0.864379804955127	0.761523046092185\\
0.872106735825037	0.745490981963928\\
0.87374749498998	0.742063235457539\\
0.879807776186825	0.729458917835672\\
0.887427056135494	0.713426853707415\\
0.889779559118236	0.708438342098458\\
0.895012388955042	0.697394789579159\\
0.902526724693145	0.681362725450902\\
0.905811623246493	0.674291841868642\\
0.909995831124626	0.665330661322646\\
0.917407826216055	0.649298597194389\\
0.921843687374749	0.639605210166722\\
0.924759984175293	0.633266533066132\\
0.932072142327726	0.617234468937876\\
0.937875751503006	0.604358567907728\\
0.939306427983218	0.601202404809619\\
0.946521156144502	0.585170340681363\\
0.953640079044779	0.569138276553106\\
0.953907815631262	0.568531625962186\\
0.960756056219921	0.55310621242485\\
0.967780754132099	0.537074148296593\\
0.969939879759519	0.532105663628523\\
0.974777739858769	0.521042084168337\\
0.981710376076964	0.50501002004008\\
0.985971943887775	0.495049758631863\\
0.988586815811709	0.488977955911824\\
0.99542946636084	0.472945891783567\\
1.00200400801603	0.45733731060375\\
1.00218360635947	0.456913827655311\\
1.00893826228546	0.440881763527054\\
1.01560260849886	0.424849699398798\\
1.01803607214429	0.418942565866309\\
1.02223671829137	0.408817635270541\\
1.02881559873643	0.392785571142285\\
1.03406813627254	0.379829674451507\\
1.03532450668279	0.376753507014028\\
1.04181969582964	0.360721442885771\\
1.04822789550428	0.344689378757515\\
1.0501002004008	0.339963767795497\\
1.05461421190746	0.328657314629258\\
1.06094089149501	0.312625250501002\\
1.06613226452906	0.299306270210088\\
1.06719817822434	0.296593186372745\\
1.07344490733363	0.280561122244489\\
1.07960754402498	0.264529058116232\\
1.08216432865731	0.257811391612406\\
1.08573861773676	0.248496993987976\\
1.09182313616956	0.232464929859719\\
1.09782580665023	0.216432865731463\\
1.09819639278557	0.215435169716469\\
1.10382818894802	0.200400801603206\\
1.10975436746823	0.18436873747495\\
1.11422845691383	0.172116267907533\\
1.11562073759942	0.168336673346693\\
1.12147167987081	0.152304609218437\\
1.12724301738029	0.13627254509018\\
1.13026052104208	0.127797365544665\\
1.13297542122262	0.120240480961924\\
1.13867283245662	0.104208416833667\\
1.14429246627308	0.0881763527054105\\
1.14629258517034	0.0824110642442843\\
1.14988753346395	0.0721442885771539\\
1.15543435708166	0.0561122244488974\\
1.16090507180902	0.0400801603206409\\
1.1623246492986	0.0358778763363548\\
1.16635911296884	0.0240480961923843\\
1.17175794302227	0.00801603206412782\\
1.17708217805388	-0.00801603206412826\\
1.17835671342685	-0.0118935369522394\\
1.1823911770971	-0.0240480961923848\\
1.1876442649766	-0.0400801603206413\\
1.19282412066439	-0.0561122244488979\\
1.19438877755511	-0.0610080323132715\\
1.19798372584872	-0.0721442885771544\\
1.20309298566784	-0.0881763527054109\\
1.20813022818325	-0.104208416833667\\
1.21042084168337	-0.111585955694638\\
1.21313574186391	-0.120240480961924\\
1.21810275454298	-0.13627254509018\\
1.22299881944837	-0.152304609218437\\
1.22645290581162	-0.163765482258007\\
1.22784518649722	-0.168336673346694\\
1.23267120256016	-0.18436873747495\\
1.23742719710257	-0.200400801603207\\
1.24211438380454	-0.216432865731463\\
1.24248496993988	-0.217714857437629\\
1.24679493303642	-0.232464929859719\\
1.25141163716841	-0.248496993987976\\
1.2559602494358	-0.264529058116232\\
1.25851703406814	-0.273662714996768\\
1.26046950850622	-0.280561122244489\\
1.26494737463194	-0.296593186372745\\
1.26935772516235	-0.312625250501002\\
1.27370159972485	-0.328657314629258\\
1.27454909819639	-0.331825883481582\\
1.2780285849571	-0.344689378757515\\
1.28230065775349	-0.360721442885771\\
1.28650662502303	-0.376753507014028\\
1.29058116232465	-0.392528378245189\\
1.29064836143032	-0.392785571142285\\
1.29478180847173	-0.408817635270541\\
1.29884938139001	-0.424849699398798\\
1.30285195075063	-0.440881763527054\\
1.30661322645291	-0.456191316685715\\
1.30679282479634	-0.456913827655311\\
1.31072117881957	-0.472945891783567\\
1.31458455512255	-0.488977955911824\\
1.31838372277035	-0.50501002004008\\
1.32211941541806	-0.521042084168337\\
1.32264529058116	-0.523334349434413\\
1.32583504765005	-0.537074148296593\\
1.32949353116982	-0.55310621242485\\
1.33308829272707	-0.569138276553106\\
1.33661996515266	-0.585170340681363\\
1.33867735470942	-0.594673495785783\\
1.34010803118963	-0.601202404809619\\
1.34355990970964	-0.617234468937876\\
1.34694824139766	-0.633266533066132\\
1.35027355767898	-0.649298597194389\\
1.35353635514788	-0.665330661322646\\
1.35470941883768	-0.671204009319294\\
1.35676341650596	-0.681362725450902\\
1.35994224867448	-0.697394789579158\\
1.36305782068158	-0.713426853707415\\
1.36611052617763	-0.729458917835671\\
1.36910072380099	-0.745490981963928\\
1.37074148296593	-0.754474507343357\\
1.37204493428342	-0.761523046092184\\
1.37494628488202	-0.777555110220441\\
1.37778409179572	-0.793587174348697\\
1.380558607853	-0.809619238476954\\
1.38327005023437	-0.82565130260521\\
1.38591860037884	-0.841683366733467\\
1.38677354709419	-0.846986325458802\\
1.38852514086549	-0.857715430861723\\
1.39107767773152	-0.87374749498998\\
1.39356590308185	-0.889779559118236\\
1.39598988725695	-0.905811623246493\\
1.39834966361338	-0.92184368737475\\
1.40064522826933	-0.937875751503006\\
1.40280561122244	-0.953398857694159\\
1.40287734648678	-0.953907815631263\\
1.40506851430965	-0.969939879759519\\
1.40719363867873	-0.985971943887776\\
1.40925259695101	-1.00200400801603\\
1.41124522734124	-1.01803607214429\\
1.41317132849114	-1.03406813627255\\
1.41503065900035	-1.0501002004008\\
1.4168229369183	-1.06613226452906\\
1.418547839196	-1.08216432865731\\
1.4188376753507	-1.0849741854515\\
1.42021886398182	-1.09819639278557\\
1.42182337322298	-1.11422845691383\\
1.42335795274132	-1.13026052104208\\
1.42482214355233	-1.14629258517034\\
1.42621544244241	-1.1623246492986\\
1.4275373012074	-1.17835671342685\\
1.42878712584424	-1.19438877755511\\
1.4299642756944	-1.21042084168337\\
1.43106806253739	-1.22645290581162\\
1.43209774963311	-1.24248496993988\\
1.43305255071117	-1.25851703406814\\
1.4339316289056	-1.27454909819639\\
1.43473409563297	-1.29058116232465\\
1.43486973947896	-1.29359165594453\\
1.435463735431	-1.30661322645291\\
1.43611501710031	-1.32264529058116\\
1.43668581376201	-1.33867735470942\\
1.43717505973574	-1.35470941883768\\
1.43758163205316	-1.37074148296593\\
1.43790434902837	-1.38677354709419\\
1.43814196875914	-1.40280561122244\\
1.43829318755633	-1.4188376753507\\
1.43835663829887	-1.43486973947896\\
1.43833088871135	-1.45090180360721\\
1.43821443956137	-1.46693386773547\\
1.43800572277337	-1.48296593186373\\
1.43770309945568	-1.49899799599198\\
1.43730485783734	-1.51503006012024\\
1.43680921111101	-1.5310621242485\\
1.43621429517806	-1.54709418837675\\
1.43551816629188	-1.56312625250501\\
1.43486973947896	-1.57614782301339\\
1.43471943107953	-1.57915831663327\\
1.43381826014665	-1.59519038076152\\
1.43280952025882	-1.61122244488978\\
1.43169085204059	-1.62725450901804\\
1.4304598001939	-1.64328657314629\\
1.42911381027157	-1.65931863727455\\
1.4276502253059	-1.67535070140281\\
1.42606628228583	-1.69138276553106\\
1.42435910847566	-1.70741482965932\\
1.4225257175681	-1.72344689378758\\
1.42056300566399	-1.73947895791583\\
1.4188376753507	-1.75270116524991\\
1.41846804884804	-1.75551102204409\\
1.41623805524873	-1.77154308617234\\
1.41386758907841	-1.7875751503006\\
1.41135291389638	-1.80360721442886\\
1.40869014715433	-1.81963927855711\\
1.40587525464174	-1.83567134268537\\
1.40290404467388	-1.85170340681363\\
1.40280561122244	-1.85221236475073\\
1.39976673853409	-1.86773547094188\\
1.39646204917176	-1.88376753507014\\
1.39298525898563	-1.8997995991984\\
1.38933130901527	-1.91583166332665\\
1.38677354709419	-1.92656076872958\\
1.38549017444567	-1.93186372745491\\
1.38144940352448	-1.94789579158317\\
1.37721190834076	-1.96392785571142\\
1.37277155563299	-1.97995991983968\\
1.37074148296593	-1.98700845858851\\
1.36810682049629	-1.99599198396794\\
1.36321095195326	-2.01202404809619\\
1.35808788971171	-2.02805611222445\\
1.35470941883768	-2.03821482835606\\
1.35271501593282	-2.04408817635271\\
1.34706954954247	-2.06012024048096\\
1.34116778578055	-2.07615230460922\\
1.33867735470942	-2.08268121363305\\
1.33496551159711	-2.09218436873747\\
1.3284588477304	-2.10821643286573\\
1.32264529058116	-2.12195623172791\\
1.32164969012635	-2.12424849699399\\
1.31446628330718	-2.14028056112224\\
1.30696119426195	-2.1563126252505\\
1.30661322645291	-2.1570351362201\\
1.29902049060337	-2.17234468937876\\
1.29071804569593	-2.18837675350701\\
1.29058116232465	-2.18863394640411\\
1.28191145320043	-2.20440881763527\\
1.27454909819639	-2.2172723129112\\
1.27267209655242	-2.22044088176353\\
1.26287839719742	-2.23647294589178\\
1.25851703406814	-2.2433713531395\\
1.25252169552617	-2.25250501002004\\
1.24248496993988	-2.26725508244213\\
1.24157692121844	-2.2685370741483\\
1.2298926066466	-2.28456913827655\\
1.22645290581162	-2.28914032936524\\
1.2174438451519	-2.30060120240481\\
1.21042084168337	-2.3092557276721\\
1.20414790670024	-2.31663326653307\\
1.19438877755511	-2.32776952279695\\
1.18987869651441	-2.33266533066132\\
1.17835671342685	-2.34481988990147\\
1.17447920853874	-2.34869739478958\\
1.1623246492986	-2.36052717493355\\
1.15775251694412	-2.36472945891784\\
1.14629258517034	-2.37499623458497\\
1.13944940232465	-2.38076152304609\\
1.13026052104208	-2.38831840762883\\
1.11925105511328	-2.39679358717435\\
1.11422845691383	-2.40057318173519\\
1.09819639278557	-2.41182795528761\\
1.09667379717982	-2.41282565130261\\
1.08216432865731	-2.42214004892704\\
1.07080768800794	-2.42885771543086\\
1.06613226452906	-2.4315707992682\\
1.0501002004008	-2.4401641685971\\
1.0404549230817	-2.44488977955912\\
1.03406813627254	-2.4479659469966\\
1.01803607214429	-2.45501471015489\\
1.00309160510989	-2.46092184368737\\
1.00200400801603	-2.46134532663581\\
0.985971943887775	-2.46699364640741\\
0.969939879759519	-2.47198542314756\\
0.953907815631262	-2.47634725722471\\
0.951348053022254	-2.47695390781563\\
0.937875751503006	-2.48011007091374\\
0.921843687374749	-2.48329258491622\\
0.905811623246493	-2.48591508836163\\
0.889779559118236	-2.48799746033493\\
0.87374749498998	-2.4895582254375\\
0.857715430861724	-2.49061462510628\\
0.841683366733467	-2.49118268369602\\
0.825651302605211	-2.49127726968805\\
0.809619238476954	-2.49091215235856\\
0.793587174348698	-2.49010005421127\\
0.777555110220441	-2.48885269945374\\
0.761523046092185	-2.48718085877326\\
0.745490981963928	-2.48509439064656\\
0.729458917835672	-2.48260227939784\\
0.713426853707415	-2.47971267020162\\
0.699954552188167	-2.47695390781563\\
0.697394789579159	-2.47643563232263\\
0.681362725450902	-2.47279233708659\\
0.665330661322646	-2.46877490054165\\
0.649298597194389	-2.46438859305215\\
0.637613172394769	-2.46092184368737\\
0.633266533066132	-2.45964550325484\\
0.617234468937876	-2.45456571746513\\
0.601202404809619	-2.44913301662021\\
0.589448840589471	-2.44488977955912\\
0.585170340681363	-2.44335978778237\\
0.569138276553106	-2.43726879886481\\
0.55310621242485	-2.43083597158032\\
0.548430788945969	-2.42885771543086\\
0.537074148296593	-2.42409432489073\\
0.521042084168337	-2.41703015422659\\
0.511937632205333	-2.41282565130261\\
0.50501002004008	-2.40965263427974\\
0.488977955911824	-2.40197261032635\\
0.47861633688666	-2.39679358717435\\
0.472945891783567	-2.39398111250709\\
0.456913827655311	-2.38569774828604\\
0.447724946372745	-2.38076152304609\\
0.440881763527054	-2.37711175756346\\
0.424849699398798	-2.36823470705649\\
0.418742219050504	-2.36472945891784\\
0.408817635270541	-2.35907140958386\\
0.392785571142285	-2.34960756594041\\
0.391293580145418	-2.34869739478958\\
0.376753507014028	-2.33988194566119\\
0.365231523926469	-2.33266533066132\\
0.360721442885771	-2.3298571615125\\
0.344689378757515	-2.31956044169135\\
0.340265649230688	-2.31663326653307\\
0.328657314629258	-2.30899339723786\\
0.316276698443884	-2.30060120240481\\
0.312625250501002	-2.298138761015\\
0.296593186372745	-2.28701987652264\\
0.293153485537765	-2.28456913827655\\
0.280561122244489	-2.27563866976863\\
0.270812795449883	-2.2685370741483\\
0.264529058116232	-2.26397936875109\\
0.249116841161576	-2.25250501002004\\
0.248496993987976	-2.25204542916615\\
0.232464929859719	-2.23986734300685\\
0.228103566730433	-2.23647294589178\\
0.216432865731463	-2.22742240388654\\
0.20764169030169	-2.22044088176353\\
0.200400801603206	-2.21470970355198\\
0.187683817264239	-2.20440881763527\\
0.18436873747495	-2.20173183890034\\
0.168336673346693	-2.18849214306497\\
0.168199789975409	-2.18837675350701\\
0.152304609218437	-2.17501187444054\\
0.149198527430876	-2.17234468937876\\
0.13627254509018	-2.16127082520807\\
0.13060223360748	-2.1563126252505\\
0.120240480961924	-2.14727077441581\\
0.112387424107648	-2.14028056112224\\
0.104208416833667	-2.13301329770613\\
0.0945324789276524	-2.12424849699399\\
0.0881763527054105	-2.11849976935194\\
0.0770174803552489	-2.10821643286573\\
0.0721442885771539	-2.10373136403057\\
0.0598240675612114	-2.09218436873747\\
0.0561122244488974	-2.08870905833842\\
0.0429353059420955	-2.07615230460922\\
0.0400801603206409	-2.07343363204895\\
0.0263355695258069	-2.06012024048096\\
0.0240480961923843	-2.05790566911636\\
0.0100104349689827	-2.04408817635271\\
0.00801603206412782	-2.04212555842685\\
-0.0060534141382742	-2.02805611222445\\
-0.00801603206412826	-2.02609349429859\\
-0.0218682737844103	-2.01202404809619\\
-0.0240480961923848	-2.00980947673159\\
-0.0374454978510004	-1.99599198396794\\
-0.0400801603206413	-1.99327331140766\\
-0.0527955690962666	-1.97995991983968\\
-0.0561122244488979	-1.97648460944062\\
-0.067928163570394	-1.96392785571142\\
-0.0721442885771544	-1.95944278687626\\
-0.082852209135697	-1.94789579158317\\
-0.0881763527054109	-1.94214706394111\\
-0.0975759386864961	-1.93186372745491\\
-0.104208416833667	-1.9245964640388\\
-0.112106938595428	-1.91583166332665\\
-0.120240480961924	-1.90678981249196\\
-0.126452192853364	-1.8997995991984\\
-0.13627254509018	-1.8887257350277\\
-0.140618123317979	-1.88376753507014\\
-0.152304609218437	-1.87040265600366\\
-0.154610626440267	-1.86773547094188\\
-0.168336673346694	-1.85181879637159\\
-0.168435106798128	-1.85170340681363\\
-0.182094263918885	-1.83567134268537\\
-0.18436873747495	-1.83299436395044\\
-0.195597114389196	-1.81963927855711\\
-0.200400801603207	-1.81390810034557\\
-0.208948104277145	-1.80360721442886\\
-0.216432865731463	-1.79455667242361\\
-0.222151133760787	-1.7875751503006\\
-0.232464929859719	-1.77493748328741\\
-0.235209729534102	-1.77154308617234\\
-0.248127367485314	-1.75551102204409\\
-0.248496993987976	-1.7550514411902\\
-0.260910749817074	-1.73947895791583\\
-0.264529058116232	-1.73492125251862\\
-0.273561692861503	-1.72344689378758\\
-0.280561122244489	-1.71451642527965\\
-0.286082555369444	-1.70741482965932\\
-0.296593186372745	-1.69383350377715\\
-0.298475405129103	-1.69138276553106\\
-0.31074681341069	-1.67535070140281\\
-0.312625250501002	-1.672888260013\\
-0.322901385421872	-1.65931863727455\\
-0.328657314629258	-1.65167876797935\\
-0.334936639563524	-1.64328657314629\\
-0.344689378757515	-1.63018168417632\\
-0.346853715889066	-1.62725450901804\\
-0.358661223665631	-1.61122244488978\\
-0.360721442885771	-1.60841427574096\\
-0.370363623606131	-1.59519038076152\\
-0.376753507014028	-1.58637493163314\\
-0.381954431452668	-1.57915831663327\\
-0.392785571142285	-1.56403642365584\\
-0.393433997955205	-1.56312625250501\\
-0.404822061190901	-1.54709418837675\\
-0.408817635270541	-1.54143613904278\\
-0.416106673164823	-1.5310621242485\\
-0.424849699398798	-1.51853530825889\\
-0.427284817757179	-1.51503006012024\\
-0.438370195933025	-1.49899799599198\\
-0.440881763527054	-1.49534823050935\\
-0.44936577793113	-1.48296593186373\\
-0.456913827655311	-1.47187009297541\\
-0.460258527737727	-1.46693386773547\\
-0.47105948890644	-1.45090180360721\\
-0.472945891783567	-1.44808932893996\\
-0.481780563993476	-1.43486973947896\\
-0.488977955911824	-1.42401669850271\\
-0.492401403989197	-1.4188376753507\\
-0.502935212529043	-1.40280561122244\\
-0.50501002004008	-1.39963259419958\\
-0.513393889713682	-1.38677354709419\\
-0.521042084168337	-1.37494598588992\\
-0.523753976742535	-1.37074148296593\\
-0.534036970397823	-1.35470941883768\\
-0.537074148296593	-1.34994602829754\\
-0.54424314245709	-1.33867735470942\\
-0.55310621242485	-1.32462354673062\\
-0.554351490046198	-1.32264529058116\\
-0.564399151523312	-1.30661322645291\\
-0.569138276553106	-1.29899224575859\\
-0.57436066541884	-1.29058116232465\\
-0.584232230108001	-1.27454909819639\\
-0.585170340681363	-1.27301910641964\\
-0.594051326219487	-1.25851703406814\\
-0.601202404809619	-1.24672820700097\\
-0.603774129129969	-1.24248496993988\\
-0.613432791996307	-1.22645290581162\\
-0.617234468937876	-1.22009677958938\\
-0.62301858987926	-1.21042084168337\\
-0.632513854057083	-1.19438877755511\\
-0.633266533066132	-1.19311243712257\\
-0.641966158922827	-1.17835671342685\\
-0.649298597194389	-1.16579139866338\\
-0.651321859979967	-1.1623246492986\\
-0.660624815805227	-1.14629258517034\\
-0.665330661322646	-1.13811357789636\\
-0.669850938713261	-1.13026052104208\\
-0.679002064889275	-1.11422845691383\\
-0.681362725450902	-1.11006688618478\\
-0.688101307851878	-1.09819639278557\\
-0.69710495342446	-1.08216432865731\\
-0.697394789579158	-1.08164605316432\\
-0.706079738533738	-1.06613226452906\\
-0.713426853707415	-1.05285896278679\\
-0.714956057633846	-1.0501002004008\\
-0.723792570976108	-1.03406813627255\\
-0.729458917835671	-1.0236844437265\\
-0.732546518759912	-1.01803607214429\\
-0.741245729576007	-1.00200400801603\\
-0.745490981963928	-0.9941124267187\\
-0.749879009420214	-0.985971943887776\\
-0.75844473699409	-0.969939879759519\\
-0.761523046092184	-0.964134766588893\\
-0.766958806572806	-0.953907815631263\\
-0.775394727267324	-0.937875751503006\\
-0.777555110220441	-0.93374247901286\\
-0.783790805290207	-0.92184368737475\\
-0.792100457992489	-0.905811623246493\\
-0.793587174348697	-0.90292570551388\\
-0.800379530336358	-0.889779559118236\\
-0.808566321619409	-0.87374749498998\\
-0.809619238476954	-0.871673675404657\\
-0.816729146876445	-0.857715430861723\\
-0.824796355889865	-0.841683366733467\\
-0.82565130260521	-0.839974664477627\\
-0.832843470353537	-0.82565130260521\\
-0.840794253455332	-0.809619238476954\\
-0.841683366733467	-0.807815950229083\\
-0.848725975563253	-0.793587174348697\\
-0.856563370704004	-0.777555110220441\\
-0.857715430861723	-0.775183763382833\\
-0.864379804955127	-0.761523046092184\\
-0.872106735825038	-0.745490981963928\\
-0.87374749498998	-0.74206323545754\\
-0.879807776186825	-0.729458917835671\\
-0.887427056135495	-0.713426853707415\\
-0.889779559118236	-0.708438342098458\\
-0.895012388955043	-0.697394789579158\\
-0.902526724693145	-0.681362725450902\\
-0.905811623246493	-0.67429184186864\\
-0.909995831124627	-0.665330661322646\\
-0.917407826216054	-0.649298597194389\\
-0.92184368737475	-0.639605210166721\\
-0.924759984175294	-0.633266533066132\\
-0.932072142327726	-0.617234468937876\\
-0.937875751503006	-0.604358567907728\\
-0.939306427983218	-0.601202404809619\\
-0.946521156144503	-0.585170340681363\\
-0.953640079044779	-0.569138276553106\\
-0.953907815631263	-0.568531625962185\\
-0.960756056219921	-0.55310621242485\\
-0.967780754132098	-0.537074148296593\\
-0.969939879759519	-0.532105663628523\\
-0.97477773985877	-0.521042084168337\\
-0.981710376076964	-0.50501002004008\\
-0.985971943887776	-0.495049758631863\\
-0.988586815811709	-0.488977955911824\\
-0.99542946636084	-0.472945891783567\\
-1.00200400801603	-0.45733731060375\\
-1.00218360635947	-0.456913827655311\\
-1.00893826228546	-0.440881763527054\\
-1.01560260849885	-0.424849699398798\\
-1.01803607214429	-0.418942565866308\\
-1.02223671829137	-0.408817635270541\\
-1.02881559873643	-0.392785571142285\\
-1.03406813627255	-0.379829674451506\\
-1.03532450668279	-0.376753507014028\\
-1.04181969582964	-0.360721442885771\\
-1.04822789550428	-0.344689378757515\\
-1.0501002004008	-0.339963767795496\\
-1.05461421190746	-0.328657314629258\\
-1.06094089149501	-0.312625250501002\\
-1.06613226452906	-0.299306270210088\\
-1.06719817822434	-0.296593186372745\\
-1.07344490733363	-0.280561122244489\\
-1.07960754402498	-0.264529058116232\\
-1.08216432865731	-0.257811391612405\\
-1.08573861773676	-0.248496993987976\\
-1.09182313616956	-0.232464929859719\\
-1.09782580665023	-0.216432865731463\\
-1.09819639278557	-0.215435169716468\\
-1.10382818894802	-0.200400801603207\\
-1.10975436746823	-0.18436873747495\\
-1.11422845691383	-0.172116267907533\\
-1.11562073759942	-0.168336673346694\\
-1.12147167987081	-0.152304609218437\\
-1.12724301738029	-0.13627254509018\\
-1.13026052104208	-0.127797365544665\\
-1.13297542122262	-0.120240480961924\\
-1.13867283245662	-0.104208416833667\\
-1.14429246627308	-0.0881763527054109\\
-1.14629258517034	-0.0824110642442843\\
-1.14988753346395	-0.0721442885771544\\
-1.15543435708166	-0.0561122244488979\\
-1.16090507180902	-0.0400801603206413\\
-1.1623246492986	-0.0358778763363548\\
-1.16635911296884	-0.0240480961923848\\
-1.17175794302227	-0.00801603206412826\\
-1.17708217805388	0.00801603206412782\\
-1.17835671342685	0.0118935369522394\\
-1.1823911770971	0.0240480961923843\\
-1.1876442649766	0.0400801603206409\\
-1.19282412066439	0.0561122244488974\\
-1.19438877755511	0.0610080323132719\\
-1.19798372584872	0.0721442885771539\\
-1.20309298566784	0.0881763527054105\\
-1.20813022818325	0.104208416833667\\
-1.21042084168337	0.111585955694638\\
-1.21313574186391	0.120240480961924\\
-1.21810275454298	0.13627254509018\\
-1.22299881944837	0.152304609218437\\
-1.22645290581162	0.163765482258007\\
-1.22784518649722	0.168336673346693\\
-1.23267120256016	0.18436873747495\\
-1.23742719710257	0.200400801603206\\
-1.24211438380454	0.216432865731463\\
-1.24248496993988	0.217714857437629\\
-1.24679493303642	0.232464929859719\\
-1.25141163716841	0.248496993987976\\
-1.2559602494358	0.264529058116232\\
-1.25851703406814	0.273662714996768\\
-1.26046950850622	0.280561122244489\\
-1.26494737463194	0.296593186372745\\
-1.26935772516235	0.312625250501002\\
-1.27370159972485	0.328657314629258\\
-1.27454909819639	0.331825883481581\\
-1.2780285849571	0.344689378757515\\
-1.28230065775349	0.360721442885771\\
-1.28650662502303	0.376753507014028\\
-1.29058116232465	0.392528378245189\\
-1.29064836143032	0.392785571142285\\
-1.29478180847173	0.408817635270541\\
-1.29884938139001	0.424849699398798\\
-1.30285195075063	0.440881763527054\\
-1.30661322645291	0.456191316685715\\
-1.30679282479634	0.456913827655311\\
-1.31072117881957	0.472945891783567\\
-1.31458455512255	0.488977955911824\\
-1.31838372277035	0.50501002004008\\
-1.32211941541806	0.521042084168337\\
-1.32264529058116	0.523334349434413\\
-1.32583504765005	0.537074148296593\\
-1.32949353116982	0.55310621242485\\
-1.33308829272707	0.569138276553106\\
-1.33661996515266	0.585170340681363\\
-1.33867735470942	0.594673495785783\\
-1.34010803118963	0.601202404809619\\
-1.34355990970964	0.617234468937876\\
-1.34694824139766	0.633266533066132\\
-1.35027355767898	0.649298597194389\\
-1.35353635514788	0.665330661322646\\
-1.35470941883768	0.671204009319294\\
-1.35676341650596	0.681362725450902\\
-1.35994224867448	0.697394789579159\\
-1.36305782068158	0.713426853707415\\
-1.36611052617763	0.729458917835672\\
-1.36910072380099	0.745490981963928\\
-1.37074148296593	0.754474507343357\\
-1.37204493428342	0.761523046092185\\
-1.37494628488202	0.777555110220441\\
-1.37778409179572	0.793587174348698\\
-1.380558607853	0.809619238476954\\
-1.38327005023437	0.825651302605211\\
-1.38591860037884	0.841683366733467\\
-1.38677354709419	0.846986325458802\\
-1.38852514086549	0.857715430861724\\
-1.39107767773152	0.87374749498998\\
-1.39356590308185	0.889779559118236\\
-1.39598988725695	0.905811623246493\\
-1.39834966361338	0.921843687374749\\
-1.40064522826933	0.937875751503006\\
-1.40280561122244	0.953398857694158\\
-1.40287734648678	0.953907815631262\\
-1.40506851430965	0.969939879759519\\
-1.40719363867873	0.985971943887775\\
-1.40925259695101	1.00200400801603\\
-1.41124522734124	1.01803607214429\\
-1.41317132849114	1.03406813627254\\
-1.41503065900035	1.0501002004008\\
-1.4168229369183	1.06613226452906\\
-1.418547839196	1.08216432865731\\
-1.4188376753507	1.0849741854515\\
-1.42021886398182	1.09819639278557\\
-1.42182337322298	1.11422845691383\\
-1.42335795274132	1.13026052104208\\
-1.42482214355234	1.14629258517034\\
-1.42621544244241	1.1623246492986\\
-1.4275373012074	1.17835671342685\\
-1.42878712584424	1.19438877755511\\
-1.4299642756944	1.21042084168337\\
-1.43106806253739	1.22645290581162\\
-1.43209774963311	1.24248496993988\\
-1.43305255071117	1.25851703406814\\
-1.4339316289056	1.27454909819639\\
-1.43473409563298	1.29058116232465\\
-1.43486973947896	1.29359165594453\\
-1.435463735431	1.30661322645291\\
-1.43611501710031	1.32264529058116\\
-1.43668581376201	1.33867735470942\\
-1.43717505973574	1.35470941883768\\
-1.43758163205316	1.37074148296593\\
-1.43790434902837	1.38677354709419\\
-1.43814196875914	1.40280561122244\\
-1.43829318755633	1.4188376753507\\
-1.43835663829887	1.43486973947896\\
-1.43833088871135	1.45090180360721\\
-1.43821443956137	1.46693386773547\\
-1.43800572277337	1.48296593186373\\
-1.43770309945568	1.49899799599198\\
-1.43730485783734	1.51503006012024\\
-1.43680921111101	1.5310621242485\\
-1.43621429517806	1.54709418837675\\
-1.43551816629188	1.56312625250501\\
-1.43486973947896	1.57614782301339\\
}--cycle;


\addplot[area legend,solid,fill=mycolor6,draw=black,forget plot]
table[row sep=crcr] {%
x	y\\
-1.25851703406814	1.5366182780483\\
-1.2571939945978	1.54709418837675\\
-1.25502463969909	1.56312625250501\\
-1.25269977496859	1.57915831663327\\
-1.25021500021237	1.59519038076152\\
-1.24756573216555	1.61122244488978\\
-1.24474719693108	1.62725450901804\\
-1.24248496993988	1.63940453353308\\
-1.24174975232708	1.64328657314629\\
-1.23855607385912	1.65931863727455\\
-1.23517430122925	1.67535070140281\\
-1.23159857003181	1.69138276553106\\
-1.22782276802783	1.70741482965932\\
-1.22645290581162	1.71297294500933\\
-1.22381713524383	1.72344689378758\\
-1.21958196768543	1.73947895791583\\
-1.21512182688343	1.75551102204409\\
-1.21042914983683	1.77154308617235\\
-1.21042084168337	1.77157044071457\\
-1.20544003002271	1.7875751503006\\
-1.2001956701131	1.80360721442886\\
-1.19468712586423	1.81963927855711\\
-1.19438877755511	1.82047757060002\\
-1.18883032863662	1.83567134268537\\
-1.18267763612223	1.85170340681363\\
-1.17835671342685	1.86248676464928\\
-1.17618880789234	1.86773547094188\\
-1.16930974881892	1.88376753507014\\
-1.1623246492986	1.89929379840314\\
-1.16208939750908	1.8997995991984\\
-1.15438504264813	1.91583166332665\\
-1.14630279283667	1.93186372745491\\
-1.14629258517034	1.93188336299896\\
-1.13765350560881	1.94789579158317\\
-1.13026052104208	1.96099723421974\\
-1.12853963234662	1.96392785571142\\
-1.11879866336283	1.97995991983968\\
-1.11422845691383	1.98719867302287\\
-1.10842999965929	1.99599198396794\\
-1.09819639278557	2.01090995449779\\
-1.09739566362948	2.01202404809619\\
-1.08549980486119	2.02805611222445\\
-1.08216432865731	2.03239562534614\\
-1.07270745353508	2.04408817635271\\
-1.06613226452906	2.05194028088218\\
-1.05889760819504	2.06012024048096\\
-1.0501002004008	2.06974519316727\\
-1.04388952525939	2.07615230460922\\
-1.03406813627255	2.08597369359606\\
-1.02745181211216	2.09218436873747\\
-1.01803607214429	2.10076645757832\\
-1.00928481999218	2.10821643286573\\
-1.00200400801603	2.11424476645324\\
-0.988995122323532	2.12424849699399\\
-0.985971943887776	2.12651320994356\\
-0.969939879759519	2.1376506054179\\
-0.965832546162879	2.14028056112224\\
-0.953907815631263	2.14773689968831\\
-0.93884409198786	2.1563126252505\\
-0.937875751503006	2.15685176728282\\
-0.92184368737475	2.16503328233086\\
-0.905838977788721	2.17234468937876\\
-0.905811623246493	2.17235693884157\\
-0.889779559118236	2.1788492796017\\
-0.87374749498998	2.18456798436975\\
-0.861530437300388	2.18837675350701\\
-0.857715430861723	2.18954616042773\\
-0.841683366733467	2.19381497273679\\
-0.82565130260521	2.19741021755645\\
-0.809619238476954	2.20035935280452\\
-0.793587174348697	2.20268792893981\\
-0.777657159274506	2.20440881763527\\
-0.777555110220441	2.20441970542425\\
-0.761523046092184	2.20557727650906\\
-0.745490981963928	2.20618078314376\\
-0.729458917835671	2.20624879964812\\
-0.713426853707415	2.20579850771998\\
-0.697394789579158	2.20484577011933\\
-0.692563647657071	2.20440881763527\\
-0.681362725450902	2.20340688550526\\
-0.665330661322646	2.20149581598421\\
-0.649298597194389	2.19912452309507\\
-0.633266533066132	2.19630450520987\\
-0.617234468937876	2.1930462183839\\
-0.601202404809619	2.18935912257919\\
-0.597387398370955	2.18837675350701\\
-0.585170340681363	2.18526062119864\\
-0.569138276553106	2.18075511445476\\
-0.55310621242485	2.17584697356038\\
-0.542541012275099	2.17234468937876\\
-0.537074148296593	2.17054823307935\\
-0.521042084168337	2.16487272763267\\
-0.50501002004008	2.15881437208123\\
-0.498795492688808	2.1563126252505\\
-0.488977955911824	2.15239180075642\\
-0.472945891783567	2.14560656143752\\
-0.461021161251951	2.14028056112224\\
-0.456913827655311	2.13845970574222\\
-0.440881763527054	2.1309704822989\\
-0.427159101112417	2.12424849699399\\
-0.424849699398798	2.12312507020458\\
-0.408817635270541	2.11495108169751\\
-0.396163769225858	2.10821643286573\\
-0.392785571142285	2.10643000206709\\
-0.376753507014028	2.09758696454167\\
-0.367337767046153	2.09218436873747\\
-0.360721442885771	2.0884104077147\\
-0.344689378757515	2.07891068588303\\
-0.340200968789764	2.07615230460922\\
-0.328657314629258	2.06909626596799\\
-0.314476241216867	2.06012024048096\\
-0.312625250501002	2.05895464809079\\
-0.296593186372745	2.04851186060859\\
-0.290017997366719	2.04408817635271\\
-0.280561122244489	2.03775510652911\\
-0.266525006674562	2.02805611222445\\
-0.264529058116232	2.02668291954234\\
-0.248496993987976	2.01531578552215\\
-0.24398720621718	2.01202404809619\\
-0.232464929859719	2.00364619026176\\
-0.222231322986003	1.99599198396794\\
-0.216432865731463	1.99167061710524\\
-0.201144387012885	1.97995991983968\\
-0.200400801603207	1.97939224687704\\
-0.18436873747495	1.96682922522964\\
-0.180759277272084	1.96392785571142\\
-0.168336673346694	1.95397046507693\\
-0.160943688779969	1.94789579158317\\
-0.152304609218437	1.94081544756269\\
-0.141641876092624	1.93186372745491\\
-0.13627254509018	1.92736641042022\\
-0.122819082958197	1.91583166332665\\
-0.120240480961924	1.91362535945474\\
-0.104443668623181	1.8997995991984\\
-0.104208416833667	1.89959407113954\\
-0.0881763527054109	1.88528235039355\\
-0.0865148160541831	1.88376753507014\\
-0.0721442885771544	1.8706828177036\\
-0.0689729344898847	1.86773547094188\\
-0.0561122244488979	1.85579545394615\\
-0.0517913017535205	1.85170340681363\\
-0.0400801603206413	1.8406211431019\\
-0.0349491946272899	1.83567134268537\\
-0.0240480961923848	1.82516054716343\\
-0.0184276157516171	1.81963927855711\\
-0.00801603206412826	1.80941410698685\\
-0.00220913950613561	1.80360721442886\\
0.00801603206412782	1.79338204285859\\
0.0137222256779407	1.7875751503006\\
0.0240480961923843	1.77706435477866\\
0.0293811605494023	1.77154308617235\\
0.0400801603206409	1.76046082246061\\
0.044781145520701	1.75551102204409\\
0.0561122244488974	1.74357100504835\\
0.0599345598845288	1.73947895791583\\
0.0721442885771539	1.72639424054929\\
0.0748527710502992	1.72344689378758\\
0.0881763527054105	1.70892964498273\\
0.0895462149216145	1.70741482965932\\
0.104023002497253	1.69138276553106\\
0.104208416833667	1.6911772374722\\
0.118282065685979	1.67535070140281\\
0.120240480961924	1.67314439753089\\
0.132343649009419	1.65931863727455\\
0.13627254509018	1.65482132023986\\
0.146215470783112	1.64328657314629\\
0.152304609218437	1.63620622912581\\
0.159904586324255	1.62725450901804\\
0.168336673346693	1.61729711838354\\
0.173417435572365	1.61122244488978\\
0.18436873747495	1.59809175027974\\
0.186759886243764	1.59519038076152\\
0.199935144005117	1.57915831663327\\
0.200400801603206	1.57859064367063\\
0.212940471362413	1.56312625250501\\
0.216432865731463	1.55880488564231\\
0.225792837809639	1.54709418837675\\
0.232464929859719	1.53871633054232\\
0.238496370489154	1.5310621242485\\
0.248496993987976	1.5183217975462\\
0.251054773962973	1.51503006012024\\
0.263468599425322	1.49899799599198\\
0.264529058116232	1.49762480330988\\
0.275739145546932	1.48296593186373\\
0.280561122244489	1.47663286204013\\
0.287877223636193	1.46693386773547\\
0.296593186372745	1.4553254878631\\
0.299885216061356	1.45090180360721\\
0.311764107097419	1.43486973947896\\
0.312625250501002	1.43370414708879\\
0.323515162650359	1.4188376753507\\
0.328657314629258	1.41178163670947\\
0.335145841858041	1.40280561122244\\
0.344689378757515	1.389531928368\\
0.346657440132767	1.38677354709419\\
0.358051131200599	1.37074148296593\\
0.360721442885771	1.36696752194316\\
0.369331448980606	1.35470941883768\\
0.376753507014028	1.34407995051361\\
0.380499925436401	1.33867735470942\\
0.391557950811915	1.32264529058116\\
0.392785571142285	1.32085885978252\\
0.402510992260561	1.30661322645291\\
0.408817635270541	1.29731581115643\\
0.41335791184124	1.29058116232465\\
0.424099699060212	1.27454909819639\\
0.424849699398798	1.27342567140698\\
0.434745077414153	1.25851703406814\\
0.440881763527054	1.2492069552448\\
0.445288689743263	1.24248496993988\\
0.455732556726431	1.22645290581162\\
0.456913827655311	1.2246320504316\\
0.46608675354468	1.21042084168337\\
0.472945891783567	1.19971477787038\\
0.476342309784794	1.19438877755511\\
0.486505544007245	1.17835671342685\\
0.488977955911824	1.17443588893277\\
0.496582201926817	1.1623246492986\\
0.50501002004008	1.14879433200106\\
0.506562227335664	1.14629258517034\\
0.516461159868609	1.13026052104208\\
0.521042084168337	1.12278855929599\\
0.526271334030238	1.11422845691383\\
0.535990169562516	1.09819639278557\\
0.537074148296593	1.09639993648616\\
0.545635936496428	1.08216432865731\\
0.55310621242485	1.06963454871068\\
0.555188309789687	1.06613226452906\\
0.564665976417858	1.0501002004008\\
0.569138276553106	1.04247856134855\\
0.574060918999784	1.03406813627254\\
0.583370807925782	1.01803607214429\\
0.585170340681363	1.01491993983592\\
0.592612661348774	1.00200400801603\\
0.601202404809619	0.986954312959955\\
0.601762078998257	0.985971943887775\\
0.610851960394462	0.969939879759519\\
0.617234468937876	0.95857728050815\\
0.619853429263515	0.953907815631262\\
0.628786712597152	0.937875751503006\\
0.633266533066132	0.929771439077608\\
0.637643590364386	0.921843687374749\\
0.646424307191761	0.905811623246493\\
0.649298597194389	0.900527328706294\\
0.655139621623025	0.889779559118236\\
0.663771644745461	0.87374749498998\\
0.665330661322646	0.870834493338914\\
0.672348105225822	0.857715430861724\\
0.680835154463941	0.841683366733467\\
0.681362725450902	0.840681434603457\\
0.689275162647607	0.825651302605211\\
0.697394789579159	0.810056190961008\\
0.697622326369228	0.809619238476954\\
0.705926469923734	0.793587174348698\\
0.713426853707415	0.778944800305149\\
0.714139046579946	0.777555110220441\\
0.722307271821182	0.761523046092185\\
0.729458917835672	0.747330963976777\\
0.730386855299095	0.745490981963928\\
0.738422394955954	0.729458917835672\\
0.745490981963928	0.715198819215905\\
0.746370303804596	0.713426853707415\\
0.754276259900431	0.697394789579159\\
0.761523046092185	0.682531184324693\\
0.762093544238379	0.681362725450902\\
0.76987289232088	0.665330661322646\\
0.777555110220441	0.649309484983373\\
0.77756034005763	0.649298597194389\\
0.785215933182017	0.633266533066132\\
0.792780544847256	0.617234468937876\\
0.793587174348698	0.615513580242413\\
0.800308648052462	0.601202404809619\\
0.807753133342204	0.585170340681363\\
0.809619238476954	0.581120875850607\\
0.815153935541872	0.569138276553106\\
0.822480850118346	0.55310621242485\\
0.825651302605211	0.546107612346024\\
0.829754334897784	0.537074148296593\\
0.836966124067748	0.521042084168337\\
0.841683366733467	0.510448239269849\\
0.844112032787387	0.50501002004008\\
0.851211034475311	0.488977955911824\\
0.857715430861724	0.474115298704278\\
0.858228869286926	0.472945891783567\\
0.865217316805395	0.456913827655311\\
0.872120783966104	0.440881763527054\\
0.87374749498998	0.437072994389789\\
0.878986367722158	0.424849699398798\\
0.885781952461454	0.408817635270541\\
0.889779559118236	0.399290161365227\\
0.892519249360777	0.392785571142285\\
0.899208927971631	0.376753507014028\\
0.905811623246493	0.36073369234858\\
0.905816692864383	0.360721442885771\\
0.912402345178452	0.344689378757515\\
0.918906420569259	0.328657314629258\\
0.921843687374749	0.321345907581362\\
0.925362512507619	0.312625250501002\\
0.931764677049471	0.296593186372745\\
0.937875751503006	0.281100264276807\\
0.93808941405656	0.280561122244489\\
0.944391308699561	0.264529058116232\\
0.950614039736601	0.248496993987976\\
0.953907815631262	0.239921268425783\\
0.956785885963635	0.232464929859719\\
0.962910080477966	0.216432865731463\\
0.96895700668126	0.200400801603206\\
0.969939879759519	0.197770845898868\\
0.974974727817517	0.18436873747495\\
0.980924619413404	0.168336673346693\\
0.985971943887775	0.154569322168009\\
0.986806821734876	0.152304609218437\\
0.992660954792984	0.13627254509018\\
0.998439532095766	0.120240480961924\\
1.00200400801603	0.110236750421178\\
1.00416444469553	0.104208416833667\\
1.00984840578382	0.0881763527054105\\
1.01545825797897	0.0721442885771539\\
1.01803607214429	0.0646943132897401\\
1.02102360004871	0.0561122244488974\\
1.02653975825509	0.0400801603206409\\
1.03198308268065	0.0240480961923843\\
1.03406813627254	0.0178374210509739\\
1.03738620316287	0.00801603206412782\\
1.04273653858697	-0.00801603206412826\\
1.04801514680891	-0.0240480961923848\\
1.0501002004008	-0.0304552076343362\\
1.05325301903754	-0.0400801603206413\\
1.05843912796602	-0.0561122244488979\\
1.06355445036374	-0.0721442885771544\\
1.06613226452906	-0.0803242481759373\\
1.06862366525489	-0.0881763527054109\\
1.07364676385577	-0.104208416833667\\
1.07859985273705	-0.120240480961924\\
1.08216432865731	-0.131933031968492\\
1.08349661050439	-0.13627254509018\\
1.08835753705469	-0.152304609218437\\
1.0931490683112	-0.168336673346694\\
1.09787225175611	-0.18436873747495\\
1.09819639278557	-0.18548283107335\\
1.10256838432536	-0.200400801603207\\
1.10719865763227	-0.216432865731463\\
1.1117609582029	-0.232464929859719\\
1.11422845691383	-0.241258240804784\\
1.11627507655532	-0.248496993987976\\
1.12074401411038	-0.264529058116232\\
1.12514519320558	-0.280561122244489\\
1.12947946693003	-0.296593186372745\\
1.13026052104208	-0.29952380786443\\
1.13377934617495	-0.312625250501002\\
1.13801878451301	-0.328657314629258\\
1.14219129312577	-0.344689378757515\\
1.14629258517034	-0.360701807341723\\
1.14629765478823	-0.360721442885771\\
1.15037434760639	-0.376753507014028\\
1.15438392376633	-0.392785571142285\\
1.15832704264181	-0.408817635270541\\
1.162204325077	-0.424849699398798\\
1.1623246492986	-0.425355500194049\\
1.16604833959957	-0.440881763527054\\
1.16982653524227	-0.456913827655311\\
1.17353838602857	-0.472945891783567\\
1.17718439541235	-0.488977955911824\\
1.17835671342685	-0.494226662204424\\
1.18078537948077	-0.50501002004008\\
1.18432949877743	-0.521042084168337\\
1.1878070192803	-0.537074148296593\\
1.19121832506825	-0.55310621242485\\
1.19438877755511	-0.568299984510198\\
1.19456520206247	-0.569138276553106\\
1.19787134074402	-0.585170340681363\\
1.20111025125887	-0.601202404809619\\
1.20428219543686	-0.617234468937876\\
1.20738739493193	-0.633266533066132\\
1.21042084168337	-0.649271242652159\\
1.21042607152056	-0.649298597194389\\
1.21342083185755	-0.665330661322646\\
1.21634748309878	-0.681362725450902\\
1.21920611961987	-0.697394789579158\\
1.22199679407919	-0.713426853707415\\
1.2247195170619	-0.729458917835671\\
1.22645290581162	-0.739932866613912\\
1.22738084327505	-0.745490981963928\\
1.22998524429604	-0.761523046092184\\
1.23251986702919	-0.777555110220441\\
1.2349845861562	-0.793587174348697\\
1.23737923194097	-0.809619238476954\\
1.23970358965474	-0.82565130260521\\
1.24195739895292	-0.841683366733467\\
1.24248496993988	-0.845565406346682\\
1.24415072011282	-0.857715430861723\\
1.24627484657362	-0.87374749498998\\
1.24832599436852	-0.889779559118236\\
1.25030375289917	-0.905811623246493\\
1.25220766250782	-0.92184368737475\\
1.25403721359916	-0.937875751503006\\
1.25579184570513	-0.953907815631263\\
1.25747094649076	-0.969939879759519\\
1.25851703406814	-0.980415790087977\\
1.25907670825677	-0.985971943887776\\
1.26061005025889	-1.00200400801603\\
1.26206463497995	-1.01803607214429\\
1.26343967651481	-1.03406813627255\\
1.2647343321079	-1.0501002004008\\
1.26594770084412	-1.06613226452906\\
1.26707882226797	-1.08216432865731\\
1.26812667492813	-1.09819639278557\\
1.2690901748448	-1.11422845691383\\
1.26996817389666	-1.13026052104208\\
1.2707594581245	-1.14629258517034\\
1.2714627459479	-1.1623246492986\\
1.27207668629181	-1.17835671342685\\
1.27259985661893	-1.19438877755511\\
1.27303076086424	-1.21042084168337\\
1.27336782726751	-1.22645290581162\\
1.27360940609925	-1.24248496993988\\
1.27375376727568	-1.25851703406814\\
1.27379909785781	-1.27454909819639\\
1.27374349942935	-1.29058116232465\\
1.27358498534827	-1.30661322645291\\
1.27332147786602	-1.32264529058116\\
1.27295080510859	-1.33867735470942\\
1.27247069791279	-1.35470941883768\\
1.27187878651122	-1.37074148296593\\
1.27117259705861	-1.38677354709419\\
1.27034954799212	-1.40280561122244\\
1.26940694621749	-1.4188376753507\\
1.2683419831128	-1.43486973947896\\
1.2671517303407	-1.45090180360721\\
1.26583313545984	-1.46693386773547\\
1.26438301732549	-1.48296593186373\\
1.26279806126875	-1.49899799599198\\
1.26107481404313	-1.51503006012024\\
1.25920967852684	-1.5310621242485\\
1.25851703406814	-1.5366182780483\\
1.2571939945978	-1.54709418837675\\
1.25502463969909	-1.56312625250501\\
1.25269977496859	-1.57915831663327\\
1.25021500021237	-1.59519038076152\\
1.24756573216555	-1.61122244488978\\
1.24474719693108	-1.62725450901804\\
1.24248496993988	-1.63940453353308\\
1.24174975232708	-1.64328657314629\\
1.23855607385912	-1.65931863727455\\
1.23517430122925	-1.67535070140281\\
1.23159857003181	-1.69138276553106\\
1.22782276802783	-1.70741482965932\\
1.22645290581162	-1.71297294500933\\
1.22381713524383	-1.72344689378758\\
1.21958196768543	-1.73947895791583\\
1.21512182688343	-1.75551102204409\\
1.21042914983683	-1.77154308617234\\
1.21042084168337	-1.77157044071457\\
1.20544003002271	-1.7875751503006\\
1.2001956701131	-1.80360721442886\\
1.19468712586423	-1.81963927855711\\
1.19438877755511	-1.82047757060002\\
1.18883032863661	-1.83567134268537\\
1.18267763612223	-1.85170340681363\\
1.17835671342685	-1.86248676464928\\
1.17618880789234	-1.86773547094188\\
1.16930974881892	-1.88376753507014\\
1.1623246492986	-1.89929379840315\\
1.16208939750908	-1.8997995991984\\
1.15438504264813	-1.91583166332665\\
1.14630279283667	-1.93186372745491\\
1.14629258517034	-1.93188336299896\\
1.13765350560881	-1.94789579158317\\
1.13026052104208	-1.96099723421974\\
1.12853963234662	-1.96392785571142\\
1.11879866336283	-1.97995991983968\\
1.11422845691383	-1.98719867302287\\
1.10842999965929	-1.99599198396794\\
1.09819639278557	-2.01090995449779\\
1.09739566362948	-2.01202404809619\\
1.08549980486119	-2.02805611222445\\
1.08216432865731	-2.03239562534614\\
1.07270745353508	-2.04408817635271\\
1.06613226452906	-2.05194028088218\\
1.05889760819504	-2.06012024048096\\
1.0501002004008	-2.06974519316727\\
1.04388952525939	-2.07615230460922\\
1.03406813627254	-2.08597369359606\\
1.02745181211216	-2.09218436873747\\
1.01803607214429	-2.10076645757832\\
1.00928481999218	-2.10821643286573\\
1.00200400801603	-2.11424476645324\\
0.988995122323532	-2.12424849699399\\
0.985971943887775	-2.12651320994356\\
0.969939879759519	-2.1376506054179\\
0.965832546162879	-2.14028056112224\\
0.953907815631262	-2.14773689968831\\
0.938844091987861	-2.1563126252505\\
0.937875751503006	-2.15685176728282\\
0.921843687374749	-2.16503328233086\\
0.905838977788721	-2.17234468937876\\
0.905811623246493	-2.17235693884156\\
0.889779559118236	-2.1788492796017\\
0.87374749498998	-2.18456798436975\\
0.861530437300388	-2.18837675350701\\
0.857715430861724	-2.18954616042773\\
0.841683366733467	-2.19381497273678\\
0.825651302605211	-2.19741021755645\\
0.809619238476954	-2.20035935280452\\
0.793587174348698	-2.20268792893981\\
0.777657159274506	-2.20440881763527\\
0.777555110220441	-2.20441970542426\\
0.761523046092185	-2.20557727650906\\
0.745490981963928	-2.20618078314376\\
0.729458917835672	-2.20624879964812\\
0.713426853707415	-2.20579850771998\\
0.697394789579159	-2.20484577011933\\
0.692563647657066	-2.20440881763527\\
0.681362725450902	-2.20340688550526\\
0.665330661322646	-2.20149581598421\\
0.649298597194389	-2.19912452309507\\
0.633266533066132	-2.19630450520987\\
0.617234468937876	-2.1930462183839\\
0.601202404809619	-2.18935912257919\\
0.597387398370955	-2.18837675350701\\
0.585170340681363	-2.18526062119864\\
0.569138276553106	-2.18075511445476\\
0.55310621242485	-2.17584697356038\\
0.542541012275099	-2.17234468937876\\
0.537074148296593	-2.17054823307935\\
0.521042084168337	-2.16487272763267\\
0.50501002004008	-2.15881437208122\\
0.498795492688808	-2.1563126252505\\
0.488977955911824	-2.15239180075642\\
0.472945891783567	-2.14560656143752\\
0.46102116125195	-2.14028056112224\\
0.456913827655311	-2.13845970574222\\
0.440881763527054	-2.1309704822989\\
0.427159101112417	-2.12424849699399\\
0.424849699398798	-2.12312507020458\\
0.408817635270541	-2.11495108169751\\
0.396163769225858	-2.10821643286573\\
0.392785571142285	-2.10643000206709\\
0.376753507014028	-2.09758696454167\\
0.367337767046153	-2.09218436873747\\
0.360721442885771	-2.0884104077147\\
0.344689378757515	-2.07891068588303\\
0.340200968789764	-2.07615230460922\\
0.328657314629258	-2.06909626596799\\
0.314476241216868	-2.06012024048096\\
0.312625250501002	-2.05895464809079\\
0.296593186372745	-2.04851186060859\\
0.29001799736672	-2.04408817635271\\
0.280561122244489	-2.03775510652911\\
0.266525006674562	-2.02805611222445\\
0.264529058116232	-2.02668291954234\\
0.248496993987976	-2.01531578552215\\
0.243987206217179	-2.01202404809619\\
0.232464929859719	-2.00364619026176\\
0.222231322986004	-1.99599198396794\\
0.216432865731463	-1.99167061710524\\
0.201144387012887	-1.97995991983968\\
0.200400801603206	-1.97939224687704\\
0.18436873747495	-1.96682922522964\\
0.180759277272084	-1.96392785571142\\
0.168336673346693	-1.95397046507693\\
0.160943688779969	-1.94789579158317\\
0.152304609218437	-1.94081544756269\\
0.141641876092625	-1.93186372745491\\
0.13627254509018	-1.92736641042022\\
0.122819082958197	-1.91583166332665\\
0.120240480961924	-1.91362535945474\\
0.104443668623181	-1.8997995991984\\
0.104208416833667	-1.89959407113954\\
0.0881763527054105	-1.88528235039355\\
0.0865148160541835	-1.88376753507014\\
0.0721442885771539	-1.8706828177036\\
0.0689729344898856	-1.86773547094188\\
0.0561122244488974	-1.85579545394615\\
0.051791301753521	-1.85170340681363\\
0.0400801603206409	-1.8406211431019\\
0.0349491946272904	-1.83567134268537\\
0.0240480961923843	-1.82516054716343\\
0.0184276157516176	-1.81963927855711\\
0.00801603206412782	-1.80941410698685\\
0.00220913950613607	-1.80360721442886\\
-0.00801603206412826	-1.79338204285859\\
-0.0137222256779403	-1.7875751503006\\
-0.0240480961923848	-1.77706435477866\\
-0.0293811605494027	-1.77154308617234\\
-0.0400801603206413	-1.76046082246061\\
-0.0447811455207014	-1.75551102204409\\
-0.0561122244488979	-1.74357100504835\\
-0.0599345598845293	-1.73947895791583\\
-0.0721442885771544	-1.72639424054929\\
-0.0748527710502996	-1.72344689378758\\
-0.0881763527054109	-1.70892964498273\\
-0.089546214921614	-1.70741482965932\\
-0.104023002497253	-1.69138276553106\\
-0.104208416833667	-1.6911772374722\\
-0.11828206568598	-1.67535070140281\\
-0.120240480961924	-1.67314439753089\\
-0.13234364900942	-1.65931863727455\\
-0.13627254509018	-1.65482132023986\\
-0.146215470783112	-1.64328657314629\\
-0.152304609218437	-1.63620622912581\\
-0.159904586324256	-1.62725450901804\\
-0.168336673346694	-1.61729711838354\\
-0.173417435572365	-1.61122244488978\\
-0.18436873747495	-1.59809175027974\\
-0.186759886243765	-1.59519038076152\\
-0.199935144005117	-1.57915831663327\\
-0.200400801603207	-1.57859064367062\\
-0.212940471362413	-1.56312625250501\\
-0.216432865731463	-1.55880488564231\\
-0.225792837809638	-1.54709418837675\\
-0.232464929859719	-1.53871633054232\\
-0.238496370489154	-1.5310621242485\\
-0.248496993987976	-1.5183217975462\\
-0.251054773962973	-1.51503006012024\\
-0.263468599425322	-1.49899799599198\\
-0.264529058116232	-1.49762480330988\\
-0.275739145546932	-1.48296593186373\\
-0.280561122244489	-1.47663286204013\\
-0.287877223636193	-1.46693386773547\\
-0.296593186372745	-1.4553254878631\\
-0.299885216061356	-1.45090180360721\\
-0.311764107097419	-1.43486973947896\\
-0.312625250501002	-1.43370414708879\\
-0.323515162650359	-1.4188376753507\\
-0.328657314629258	-1.41178163670947\\
-0.335145841858041	-1.40280561122244\\
-0.344689378757515	-1.389531928368\\
-0.346657440132767	-1.38677354709419\\
-0.358051131200599	-1.37074148296593\\
-0.360721442885771	-1.36696752194316\\
-0.369331448980606	-1.35470941883768\\
-0.376753507014028	-1.34407995051361\\
-0.380499925436401	-1.33867735470942\\
-0.391557950811915	-1.32264529058116\\
-0.392785571142285	-1.32085885978252\\
-0.402510992260561	-1.30661322645291\\
-0.408817635270541	-1.29731581115643\\
-0.41335791184124	-1.29058116232465\\
-0.424099699060212	-1.27454909819639\\
-0.424849699398798	-1.27342567140698\\
-0.434745077414153	-1.25851703406814\\
-0.440881763527054	-1.2492069552448\\
-0.445288689743263	-1.24248496993988\\
-0.455732556726431	-1.22645290581162\\
-0.456913827655311	-1.2246320504316\\
-0.466086753544681	-1.21042084168337\\
-0.472945891783567	-1.19971477787038\\
-0.476342309784793	-1.19438877755511\\
-0.486505544007245	-1.17835671342685\\
-0.488977955911824	-1.17443588893277\\
-0.496582201926817	-1.1623246492986\\
-0.50501002004008	-1.14879433200106\\
-0.506562227335664	-1.14629258517034\\
-0.516461159868609	-1.13026052104208\\
-0.521042084168337	-1.12278855929599\\
-0.526271334030238	-1.11422845691383\\
-0.535990169562516	-1.09819639278557\\
-0.537074148296593	-1.09639993648616\\
-0.545635936496428	-1.08216432865731\\
-0.55310621242485	-1.06963454871068\\
-0.555188309789687	-1.06613226452906\\
-0.564665976417857	-1.0501002004008\\
-0.569138276553106	-1.04247856134855\\
-0.574060918999784	-1.03406813627255\\
-0.583370807925782	-1.01803607214429\\
-0.585170340681363	-1.01491993983592\\
-0.592612661348773	-1.00200400801603\\
-0.601202404809619	-0.986954312959955\\
-0.601762078998257	-0.985971943887776\\
-0.610851960394462	-0.969939879759519\\
-0.617234468937876	-0.95857728050815\\
-0.619853429263515	-0.953907815631263\\
-0.628786712597152	-0.937875751503006\\
-0.633266533066132	-0.929771439077608\\
-0.637643590364386	-0.92184368737475\\
-0.646424307191761	-0.905811623246493\\
-0.649298597194389	-0.900527328706294\\
-0.655139621623025	-0.889779559118236\\
-0.663771644745462	-0.87374749498998\\
-0.665330661322646	-0.870834493338914\\
-0.672348105225823	-0.857715430861723\\
-0.680835154463942	-0.841683366733467\\
-0.681362725450902	-0.840681434603458\\
-0.689275162647607	-0.82565130260521\\
-0.697394789579158	-0.81005619096101\\
-0.697622326369228	-0.809619238476954\\
-0.705926469923734	-0.793587174348697\\
-0.713426853707415	-0.77894480030515\\
-0.714139046579946	-0.777555110220441\\
-0.722307271821182	-0.761523046092184\\
-0.729458917835671	-0.747330963976778\\
-0.730386855299095	-0.745490981963928\\
-0.738422394955954	-0.729458917835671\\
-0.745490981963928	-0.715198819215906\\
-0.746370303804596	-0.713426853707415\\
-0.754276259900432	-0.697394789579158\\
-0.761523046092184	-0.682531184324695\\
-0.76209354423838	-0.681362725450902\\
-0.76987289232088	-0.665330661322646\\
-0.777555110220441	-0.649309484983374\\
-0.77756034005763	-0.649298597194389\\
-0.785215933182017	-0.633266533066132\\
-0.792780544847256	-0.617234468937876\\
-0.793587174348697	-0.615513580242414\\
-0.800308648052461	-0.601202404809619\\
-0.807753133342203	-0.585170340681363\\
-0.809619238476954	-0.581120875850608\\
-0.815153935541872	-0.569138276553106\\
-0.822480850118345	-0.55310621242485\\
-0.82565130260521	-0.546107612346026\\
-0.829754334897784	-0.537074148296593\\
-0.836966124067748	-0.521042084168337\\
-0.841683366733467	-0.510448239269851\\
-0.844112032787387	-0.50501002004008\\
-0.851211034475311	-0.488977955911824\\
-0.857715430861723	-0.474115298704279\\
-0.858228869286925	-0.472945891783567\\
-0.865217316805395	-0.456913827655311\\
-0.872120783966104	-0.440881763527054\\
-0.87374749498998	-0.437072994389791\\
-0.878986367722158	-0.424849699398798\\
-0.885781952461454	-0.408817635270541\\
-0.889779559118236	-0.399290161365226\\
-0.892519249360777	-0.392785571142285\\
-0.899208927971631	-0.376753507014028\\
-0.905811623246493	-0.360733692348578\\
-0.905816692864383	-0.360721442885771\\
-0.912402345178453	-0.344689378757515\\
-0.918906420569258	-0.328657314629258\\
-0.92184368737475	-0.321345907581361\\
-0.925362512507619	-0.312625250501002\\
-0.93176467704947	-0.296593186372745\\
-0.937875751503006	-0.281100264276806\\
-0.938089414056559	-0.280561122244489\\
-0.944391308699561	-0.264529058116232\\
-0.950614039736601	-0.248496993987976\\
-0.953907815631263	-0.239921268425782\\
-0.956785885963635	-0.232464929859719\\
-0.962910080477966	-0.216432865731463\\
-0.968957006681261	-0.200400801603207\\
-0.969939879759519	-0.197770845898866\\
-0.974974727817517	-0.18436873747495\\
-0.980924619413404	-0.168336673346694\\
-0.985971943887776	-0.154569322168008\\
-0.986806821734876	-0.152304609218437\\
-0.992660954792983	-0.13627254509018\\
-0.998439532095766	-0.120240480961924\\
-1.00200400801603	-0.110236750421177\\
-1.00416444469553	-0.104208416833667\\
-1.00984840578382	-0.0881763527054109\\
-1.01545825797897	-0.0721442885771544\\
-1.01803607214429	-0.0646943132897392\\
-1.02102360004871	-0.0561122244488979\\
-1.02653975825509	-0.0400801603206413\\
-1.03198308268065	-0.0240480961923848\\
-1.03406813627255	-0.0178374210509725\\
-1.03738620316287	-0.00801603206412826\\
-1.04273653858697	0.00801603206412782\\
-1.04801514680891	0.0240480961923843\\
-1.0501002004008	0.0304552076343385\\
-1.05325301903754	0.0400801603206409\\
-1.05843912796601	0.0561122244488974\\
-1.06355445036374	0.0721442885771539\\
-1.06613226452906	0.0803242481759391\\
-1.06862366525489	0.0881763527054105\\
-1.07364676385577	0.104208416833667\\
-1.07859985273705	0.120240480961924\\
-1.08216432865731	0.131933031968493\\
-1.08349661050439	0.13627254509018\\
-1.08835753705469	0.152304609218437\\
-1.0931490683112	0.168336673346693\\
-1.09787225175611	0.18436873747495\\
-1.09819639278557	0.185482831073351\\
-1.10256838432536	0.200400801603206\\
-1.10719865763227	0.216432865731463\\
-1.1117609582029	0.232464929859719\\
-1.11422845691383	0.241258240804785\\
-1.11627507655532	0.248496993987976\\
-1.12074401411038	0.264529058116232\\
-1.12514519320558	0.280561122244489\\
-1.12947946693003	0.296593186372745\\
-1.13026052104208	0.299523807864431\\
-1.13377934617495	0.312625250501002\\
-1.13801878451301	0.328657314629258\\
-1.14219129312577	0.344689378757515\\
-1.14629258517034	0.360701807341723\\
-1.14629765478823	0.360721442885771\\
-1.15037434760639	0.376753507014028\\
-1.15438392376633	0.392785571142285\\
-1.15832704264181	0.408817635270541\\
-1.162204325077	0.424849699398798\\
-1.1623246492986	0.425355500194049\\
-1.16604833959957	0.440881763527054\\
-1.16982653524227	0.456913827655311\\
-1.17353838602857	0.472945891783567\\
-1.17718439541235	0.488977955911824\\
-1.17835671342685	0.494226662204425\\
-1.18078537948077	0.50501002004008\\
-1.18432949877743	0.521042084168337\\
-1.1878070192803	0.537074148296593\\
-1.19121832506825	0.55310621242485\\
-1.19438877755511	0.568299984510198\\
-1.19456520206247	0.569138276553106\\
-1.19787134074402	0.585170340681363\\
-1.20111025125887	0.601202404809619\\
-1.20428219543685	0.617234468937876\\
-1.20738739493193	0.633266533066132\\
-1.21042084168337	0.649271242652159\\
-1.21042607152056	0.649298597194389\\
-1.21342083185755	0.665330661322646\\
-1.21634748309878	0.681362725450902\\
-1.21920611961987	0.697394789579159\\
-1.22199679407919	0.713426853707415\\
-1.2247195170619	0.729458917835672\\
-1.22645290581162	0.739932866613911\\
-1.22738084327505	0.745490981963928\\
-1.22998524429604	0.761523046092185\\
-1.23251986702919	0.777555110220441\\
-1.2349845861562	0.793587174348698\\
-1.23737923194097	0.809619238476954\\
-1.23970358965474	0.825651302605211\\
-1.24195739895292	0.841683366733467\\
-1.24248496993988	0.845565406346681\\
-1.24415072011282	0.857715430861724\\
-1.24627484657362	0.87374749498998\\
-1.24832599436852	0.889779559118236\\
-1.25030375289917	0.905811623246493\\
-1.25220766250782	0.921843687374749\\
-1.25403721359916	0.937875751503006\\
-1.25579184570513	0.953907815631262\\
-1.25747094649076	0.969939879759519\\
-1.25851703406814	0.980415790087977\\
-1.25907670825677	0.985971943887775\\
-1.26061005025888	1.00200400801603\\
-1.26206463497995	1.01803607214429\\
-1.26343967651481	1.03406813627254\\
-1.2647343321079	1.0501002004008\\
-1.26594770084412	1.06613226452906\\
-1.26707882226797	1.08216432865731\\
-1.26812667492813	1.09819639278557\\
-1.2690901748448	1.11422845691383\\
-1.26996817389666	1.13026052104208\\
-1.2707594581245	1.14629258517034\\
-1.2714627459479	1.1623246492986\\
-1.27207668629181	1.17835671342685\\
-1.27259985661893	1.19438877755511\\
-1.27303076086424	1.21042084168337\\
-1.27336782726751	1.22645290581162\\
-1.27360940609925	1.24248496993988\\
-1.27375376727568	1.25851703406814\\
-1.27379909785781	1.27454909819639\\
-1.27374349942935	1.29058116232465\\
-1.27358498534827	1.30661322645291\\
-1.27332147786602	1.32264529058116\\
-1.27295080510859	1.33867735470942\\
-1.27247069791279	1.35470941883768\\
-1.27187878651122	1.37074148296593\\
-1.27117259705861	1.38677354709419\\
-1.27034954799212	1.40280561122244\\
-1.26940694621749	1.4188376753507\\
-1.2683419831128	1.43486973947896\\
-1.2671517303407	1.45090180360721\\
-1.26583313545984	1.46693386773547\\
-1.26438301732549	1.48296593186373\\
-1.26279806126875	1.49899799599198\\
-1.26107481404313	1.51503006012024\\
-1.25920967852684	1.5310621242485\\
-1.25851703406814	1.5366182780483\\
}--cycle;


\addplot[area legend,solid,fill=mycolor7,draw=black,forget plot]
table[row sep=crcr] {%
x	y\\
-1.11422845691383	1.29527490685972\\
-1.11319708140095	1.30661322645291\\
-1.11158897321772	1.32264529058116\\
-1.1098226773115	1.33867735470942\\
-1.10789361042715	1.35470941883768\\
-1.10579698497173	1.37074148296593\\
-1.10352780003294	1.38677354709419\\
-1.1010808319104	1.40280561122244\\
-1.0984506241303	1.4188376753507\\
-1.09819639278557	1.42029804511158\\
-1.09560322010549	1.43486973947896\\
-1.09255398071092	1.45090180360721\\
-1.08929907735243	1.46693386773547\\
-1.08583167329455	1.48296593186373\\
-1.08216432865731	1.49891333546637\\
-1.0821443520793	1.49899799599198\\
-1.07817537844418	1.51503006012024\\
-1.07396500034531	1.5310621242485\\
-1.06950457834855	1.54709418837675\\
-1.06613226452906	1.55859377815824\\
-1.0647629193205	1.56312625250501\\
-1.05969013932373	1.57915831663327\\
-1.05432972957213	1.59519038076152\\
-1.0501002004008	1.60721650529931\\
-1.04864334990002	1.61122244488978\\
-1.04255657480825	1.62725450901804\\
-1.0361352655747	1.64328657314629\\
-1.03406813627255	1.64823587357896\\
-1.0292623072434	1.65931863727455\\
-1.02196731801052	1.67535070140281\\
-1.01803607214429	1.68361504268286\\
-1.01418588617855	1.69138276553106\\
-1.00587990953774	1.70741482965932\\
-1.00200400801603	1.71458298935537\\
-0.996990227098053	1.72344689378758\\
-0.987502407962091	1.73947895791583\\
-0.985971943887776	1.74197122833202\\
-0.977236362534941	1.75551102204409\\
-0.969939879759519	1.7663285202369\\
-0.966230684303345	1.77154308617235\\
-0.954346209377059	1.7875751503006\\
-0.953907815631263	1.78814656274547\\
-0.94133327615024	1.80360721442886\\
-0.937875751503006	1.8076965221177\\
-0.927121552749067	1.81963927855711\\
-0.92184368737475	1.82528934752995\\
-0.911461692219334	1.83567134268537\\
-0.905811623246493	1.84112894737434\\
-0.89402410506561	1.85170340681363\\
-0.889779559118236	1.85538858208849\\
-0.874365875445441	1.86773547094188\\
-0.87374749498998	1.86821576553839\\
-0.857715430861723	1.87970629675285\\
-0.851492446151481	1.88376753507014\\
-0.841683366733467	1.8899920287837\\
-0.82565130260521	1.89916697663983\\
-0.824429387048633	1.8997995991984\\
-0.809619238476954	1.90727502109806\\
-0.793587174348697	1.91442224207739\\
-0.790015546419613	1.91583166332665\\
-0.777555110220441	1.92063769330967\\
-0.761523046092184	1.92599125684182\\
-0.745490981963928	1.93053135242795\\
-0.739874814442953	1.93186372745491\\
-0.729458917835671	1.93428701368306\\
-0.713426853707415	1.9373019131192\\
-0.697394789579158	1.9396132115098\\
-0.681362725450902	1.94125040392693\\
-0.665330661322646	1.94224079591663\\
-0.649298597194389	1.94260964305413\\
-0.633266533066132	1.94238027890235\\
-0.617234468937876	1.94157423236356\\
-0.601202404809619	1.94021133531567\\
-0.585170340681363	1.93830982133684\\
-0.569138276553106	1.93588641624351\\
-0.55310621242485	1.93295642109648\\
-0.54801938157678	1.93186372745491\\
-0.537074148296593	1.92953305696762\\
-0.521042084168337	1.92563085023898\\
-0.50501002004008	1.92126222414596\\
-0.488977955911824	1.91643779725655\\
-0.487146074453037	1.91583166332665\\
-0.472945891783567	1.91116918961357\\
-0.456913827655311	1.9054657821794\\
-0.442103679083631	1.8997995991984\\
-0.440881763527054	1.89933546148151\\
-0.424849699398798	1.89279061863763\\
-0.408817635270541	1.8858346643723\\
-0.404332999824192	1.88376753507014\\
-0.392785571142285	1.8784790028984\\
-0.376753507014028	1.87072758265841\\
-0.370881128353451	1.86773547094188\\
-0.360721442885771	1.86258938978954\\
-0.344689378757515	1.85406785039725\\
-0.340444832810142	1.85170340681363\\
-0.328657314629258	1.84517246283947\\
-0.312625250501002	1.83590190159421\\
-0.312243254575273	1.83567134268537\\
-0.296593186372745	1.82627154452174\\
-0.285972676222855	1.81963927855711\\
-0.280561122244489	1.81627548532647\\
-0.264529058116232	1.80592295097609\\
-0.261071533468998	1.80360721442886\\
-0.248496993987976	1.795219764379\\
-0.237425008585017	1.7875751503006\\
-0.232464929859719	1.78416365914826\\
-0.216432865731463	1.77276117036572\\
-0.21477460953327	1.77154308617235\\
-0.200400801603207	1.76101992105668\\
-0.193104318827785	1.75551102204409\\
-0.18436873747495	1.7489359039608\\
-0.172173345241227	1.73947895791583\\
-0.168336673346694	1.73651220130204\\
-0.152304609218437	1.72375235381353\\
-0.151931278795002	1.72344689378758\\
-0.13627254509018	1.71066447591853\\
-0.132396643568474	1.70741482965932\\
-0.120240480961924	1.69724375478096\\
-0.113414571508778	1.69138276553106\\
-0.104208416833667	1.68349221526087\\
-0.0949445761708533	1.67535070140281\\
-0.0881763527054109	1.66941162343815\\
-0.0769501176063007	1.65931863727455\\
-0.0721442885771544	1.6550034888922\\
-0.0593981639867885	1.64328657314629\\
-0.0561122244488979	1.64026906664833\\
-0.0422588104508495	1.62725450901804\\
-0.0400801603206413	1.62520935879441\\
-0.025504946693164	1.61122244488978\\
-0.0240480961923848	1.60982511577077\\
-0.00911196622221128	1.59519038076152\\
-0.00801603206412826	1.59411683733554\\
0.00694248863814595	1.57915831663327\\
0.00801603206412782	1.57808477320728\\
0.0226787509838308	1.56312625250501\\
0.0240480961923843	1.561728923386\\
0.0381153270050993	1.54709418837675\\
0.0400801603206409	1.54504903815313\\
0.0532690706591171	1.5310621242485\\
0.0561122244488974	1.52804461775053\\
0.0681553383640204	1.51503006012024\\
0.0721442885771539	1.5107149117379\\
0.082788126291551	1.49899799599198\\
0.0881763527054105	1.49305891802733\\
0.0971801924616865	1.48296593186373\\
0.104208416833667	1.47507538159354\\
0.111343165528784	1.46693386773547\\
0.120240480961924	1.45676279285712\\
0.125287641884237	1.45090180360721\\
0.13627254509018	1.43811938573817\\
0.139023272476939	1.43486973947896\\
0.152304609218437	1.41914313537665\\
0.152558840563168	1.4188376753507\\
0.165878065541669	1.40280561122244\\
0.168336673346693	1.39983885460865\\
0.179011697335403	1.38677354709419\\
0.18436873747495	1.3801984290109\\
0.191969329661104	1.37074148296593\\
0.200400801603206	1.36021831785027\\
0.20475718723836	1.35470941883768\\
0.216432865731463	1.33989543890279\\
0.21738088618959	1.33867735470942\\
0.229825446163612	1.32264529058116\\
0.232464929859719	1.31923379942882\\
0.242111014129311	1.30661322645291\\
0.248496993987976	1.29822577640305\\
0.254249679111395	1.29058116232465\\
0.264529058116232	1.27686483474363\\
0.266245197116871	1.27454909819639\\
0.278085913715293	1.25851703406814\\
0.280561122244489	1.25515324083749\\
0.289782528321357	1.24248496993988\\
0.296593186372745	1.23308517177625\\
0.301349186753381	1.22645290581162\\
0.312625250501002	1.2106514005922\\
0.312788164902552	1.21042084168337\\
0.324080466620025	1.19438877755511\\
0.328657314629258	1.18785783358095\\
0.335252744928811	1.17835671342685\\
0.344689378757515	1.16468909288222\\
0.346307219645143	1.1623246492986\\
0.357232011798649	1.14629258517034\\
0.360721442885771	1.141146504018\\
0.368039327265191	1.13026052104208\\
0.376753507014028	1.11722056863035\\
0.378736718869431	1.11422845691383\\
0.389314546384562	1.09819639278557\\
0.392785571142285	1.09290786061383\\
0.399781828147799	1.08216432865731\\
0.408817635270541	1.06819939383122\\
0.410145301707386	1.06613226452906\\
0.420395503992964	1.0501002004008\\
0.424849699398798	1.04309121984003\\
0.430543682829265	1.03406813627254\\
0.440592084630359	1.01803607214429\\
0.440881763527054	1.0175719344274\\
0.450533390184912	1.00200400801603\\
0.456913827655311	0.991638126868775\\
0.460379816391221	0.985971943887775\\
0.470127348333343	0.969939879759519\\
0.472945891783567	0.965277406046434\\
0.479778661095603	0.953907815631262\\
0.488977955911824	0.938481885432903\\
0.489337460739446	0.937875751503006\\
0.498800763532824	0.921843687374749\\
0.50501002004008	0.911242184065805\\
0.508174479282237	0.905811623246493\\
0.517457049057759	0.889779559118236\\
0.521042084168337	0.883546681902311\\
0.526651618119988	0.87374749498998\\
0.535757770323694	0.857715430861724\\
0.537074148296593	0.855384760374438\\
0.544778659699726	0.841683366733467\\
0.55310621242485	0.826743996246779\\
0.553712876870233	0.825651302605211\\
0.562564761636584	0.809619238476954\\
0.569138276553106	0.797609863137293\\
0.571332121958972	0.793587174348698\\
0.5800184813075	0.777555110220441\\
0.585170340681363	0.767969139974112\\
0.588623238808741	0.761523046092185\\
0.597147798796265	0.745490981963928\\
0.601202404809619	0.737806525696426\\
0.605593806770739	0.729458917835672\\
0.613960138273222	0.713426853707415\\
0.617234468937876	0.707105294487803\\
0.62225086427468	0.697394789579159\\
0.630462387886829	0.681362725450902\\
0.633266533066132	0.675847212770085\\
0.638600927584093	0.665330661322646\\
0.64666091823862	0.649298597194389\\
0.649298597194389	0.644012448665355\\
0.654650008190866	0.633266533066132\\
0.662561599507746	0.617234468937876\\
0.665330661322646	0.611579473271343\\
0.670403628910887	0.601202404809619\\
0.678169817286189	0.585170340681363\\
0.681362725450902	0.578524953025124\\
0.685866838737787	0.569138276553106\\
0.69349048718108	0.55310621242485\\
0.697394789579159	0.544823632351482\\
0.701044226507351	0.537074148296593\\
0.708528068235948	0.521042084168337\\
0.713426853707415	0.510448205704373\\
0.715939933420816	0.50501002004008\\
0.723286575218568	0.488977955911824\\
0.729458917835672	0.475369178011716\\
0.7305576644713	0.472945891783567\\
0.737769589818998	0.456913827655311\\
0.744902759144508	0.440881763527054\\
0.745490981963928	0.439549388500092\\
0.751980270797614	0.424849699398798\\
0.758981402159146	0.408817635270541\\
0.761523046092185	0.402945164657454\\
0.765921363119241	0.392785571142285\\
0.772792646047509	0.376753507014028\\
0.777555110220441	0.365527472868786\\
0.779595206106097	0.360721442885771\\
0.78633871591627	0.344689378757515\\
0.79300605003884	0.328657314629258\\
0.793587174348698	0.327247893379989\\
0.799621440602657	0.312625250501002\\
0.806163058730893	0.296593186372745\\
0.809619238476954	0.288036544144152\\
0.81264225875116	0.280561122244489\\
0.819059962112337	0.264529058116232\\
0.825403267700109	0.248496993987976\\
0.825651302605211	0.247864371429403\\
0.831697706153768	0.232464929859719\\
0.837918726609621	0.216432865731463\\
0.841683366733467	0.206625295316764\\
0.84407685759974	0.200400801603206\\
0.850177130537457	0.18436873747495\\
0.856204317512007	0.168336673346693\\
0.857715430861724	0.164275435029403\\
0.862178565765762	0.152304609218437\\
0.868086222771197	0.13627254509018\\
0.87374749498998	0.120720775558425\\
0.873922746025479	0.120240480961924\\
0.879712154483288	0.104208416833667\\
0.88542933896269	0.0881763527054105\\
0.889779559118236	0.075829463852014\\
0.891081352733647	0.0721442885771539\\
0.896681104837513	0.0561122244488974\\
0.902209517311712	0.0400801603206409\\
0.905811623246493	0.0295057008813569\\
0.907676036994777	0.0240480961923843\\
0.913087586458924	0.00801603206412782\\
0.918428489847537	-0.00801603206412826\\
0.921843687374749	-0.0183980272195427\\
0.923708101123034	-0.0240480961923848\\
0.928932465899462	-0.0400801603206413\\
0.934086689113959	-0.0561122244488979\\
0.937875751503006	-0.0680549808883071\\
0.939177545118417	-0.0721442885771544\\
0.944215309975964	-0.0881763527054109\\
0.949183249838047	-0.104208416833667\\
0.953907815631262	-0.119669068517051\\
0.954083066666762	-0.120240480961924\\
0.958934384098946	-0.13627254509018\\
0.963716005591637	-0.152304609218437\\
0.968428766409803	-0.168336673346694\\
0.969939879759519	-0.173551239282138\\
0.973086645577692	-0.18436873747495\\
0.977681480427534	-0.200400801603207\\
0.982207303763929	-0.216432865731463\\
0.985971943887775	-0.229972659443533\\
0.986667731865482	-0.232464929859719\\
0.991074875445203	-0.248496993987976\\
0.995412667523159	-0.264529058116232\\
0.99968171568218	-0.280561122244489\\
1.00200400801603	-0.289425026676692\\
1.00389005021323	-0.296593186372745\\
1.00803827426999	-0.312625250501002\\
1.01211712930511	-0.328657314629258\\
1.01612708530091	-0.344689378757515\\
1.01803607214429	-0.352457101605711\\
1.02007616802994	-0.360721442885771\\
1.0239626482088	-0.376753507014028\\
1.02777931323051	-0.392785571142285\\
1.03152649233951	-0.408817635270541\\
1.03406813627254	-0.419900398966124\\
1.03520845019296	-0.424849699398798\\
1.03882897978193	-0.440881763527054\\
1.04237880825587	-0.456913827655311\\
1.04585811988899	-0.472945891783567\\
1.04926705264233	-0.488977955911824\\
1.0501002004008	-0.492983895502286\\
1.0526132801142	-0.50501002004008\\
1.05589018644029	-0.521042084168337\\
1.0590950852347	-0.537074148296593\\
1.06222796213098	-0.55310621242485\\
1.06528875386917	-0.569138276553106\\
1.06613226452906	-0.573670750899876\\
1.06828280652761	-0.585170340681363\\
1.0712052321173	-0.601202404809619\\
1.07405350670123	-0.617234468937876\\
1.07682740507046	-0.633266533066132\\
1.07952664970155	-0.649298597194389\\
1.08215090986002	-0.665330661322646\\
1.08216432865731	-0.66541532184826\\
1.08470439642679	-0.681362725450902\\
1.08718072399412	-0.697394789579158\\
1.08957941337184	-0.713426853707415\\
1.09189995632039	-0.729458917835671\\
1.09414178677222	-0.745490981963928\\
1.09630427959414	-0.761523046092184\\
1.09819639278557	-0.776094740459558\\
1.09838695251236	-0.777555110220441\\
1.10039023819144	-0.793587174348697\\
1.10231088468654	-0.809619238476954\\
1.10414806928106	-0.82565130260521\\
1.1059009041887	-0.841683366733467\\
1.10756843491448	-0.857715430861723\\
1.10914963852203	-0.87374749498998\\
1.11064342180325	-0.889779559118236\\
1.11204861934615	-0.905811623246493\\
1.11336399149656	-0.92184368737475\\
1.11422845691383	-0.933182006967936\\
1.11458796174145	-0.937875751503006\\
1.1157184454276	-0.953907815631263\\
1.11675443604705	-0.969939879759519\\
1.11769444564974	-0.985971943887776\\
1.11853690276846	-1.00200400801603\\
1.11928014981559	-1.01803607214429\\
1.11992244034429	-1.03406813627255\\
1.12046193616776	-1.0501002004008\\
1.12089670432974	-1.06613226452906\\
1.12122471391934	-1.08216432865731\\
1.12144383272227	-1.09819639278557\\
1.12155182370072	-1.11422845691383\\
1.12154634129325	-1.13026052104208\\
1.12142492752561	-1.14629258517034\\
1.12118500792299	-1.1623246492986\\
1.12082388721338	-1.17835671342685\\
1.12033874481139	-1.19438877755511\\
1.1197266300709	-1.21042084168337\\
1.11898445729446	-1.22645290581162\\
1.11810900048634	-1.24248496993988\\
1.11709688783561	-1.25851703406814\\
1.11594459591447	-1.27454909819639\\
1.11464844357632	-1.29058116232465\\
1.11422845691383	-1.29527490685972\\
1.11319708140095	-1.30661322645291\\
1.11158897321772	-1.32264529058116\\
1.1098226773115	-1.33867735470942\\
1.10789361042715	-1.35470941883768\\
1.10579698497173	-1.37074148296593\\
1.10352780003294	-1.38677354709419\\
1.1010808319104	-1.40280561122244\\
1.0984506241303	-1.4188376753507\\
1.09819639278557	-1.42029804511158\\
1.09560322010549	-1.43486973947896\\
1.09255398071092	-1.45090180360721\\
1.08929907735243	-1.46693386773547\\
1.08583167329455	-1.48296593186373\\
1.08216432865731	-1.49891333546637\\
1.0821443520793	-1.49899799599198\\
1.07817537844418	-1.51503006012024\\
1.07396500034531	-1.5310621242485\\
1.06950457834855	-1.54709418837675\\
1.06613226452906	-1.55859377815824\\
1.0647629193205	-1.56312625250501\\
1.05969013932373	-1.57915831663327\\
1.05432972957213	-1.59519038076152\\
1.0501002004008	-1.60721650529932\\
1.04864334990002	-1.61122244488978\\
1.04255657480825	-1.62725450901804\\
1.0361352655747	-1.64328657314629\\
1.03406813627254	-1.64823587357897\\
1.0292623072434	-1.65931863727455\\
1.02196731801052	-1.67535070140281\\
1.01803607214429	-1.68361504268287\\
1.01418588617855	-1.69138276553106\\
1.00587990953774	-1.70741482965932\\
1.00200400801603	-1.71458298935537\\
0.996990227098052	-1.72344689378758\\
0.987502407962091	-1.73947895791583\\
0.985971943887775	-1.74197122833202\\
0.977236362534941	-1.75551102204409\\
0.969939879759519	-1.7663285202369\\
0.966230684303344	-1.77154308617234\\
0.954346209377058	-1.7875751503006\\
0.953907815631262	-1.78814656274547\\
0.94133327615024	-1.80360721442886\\
0.937875751503006	-1.8076965221177\\
0.927121552749066	-1.81963927855711\\
0.921843687374749	-1.82528934752995\\
0.911461692219333	-1.83567134268537\\
0.905811623246493	-1.84112894737434\\
0.894024105065609	-1.85170340681363\\
0.889779559118236	-1.85538858208849\\
0.874365875445442	-1.86773547094188\\
0.87374749498998	-1.86821576553839\\
0.857715430861724	-1.87970629675285\\
0.851492446151481	-1.88376753507014\\
0.841683366733467	-1.8899920287837\\
0.825651302605211	-1.89916697663982\\
0.824429387048632	-1.8997995991984\\
0.809619238476954	-1.90727502109806\\
0.793587174348698	-1.91442224207739\\
0.790015546419612	-1.91583166332665\\
0.777555110220441	-1.92063769330967\\
0.761523046092185	-1.92599125684182\\
0.745490981963928	-1.93053135242795\\
0.739874814442951	-1.93186372745491\\
0.729458917835672	-1.93428701368306\\
0.713426853707415	-1.9373019131192\\
0.697394789579159	-1.9396132115098\\
0.681362725450902	-1.94125040392693\\
0.665330661322646	-1.94224079591663\\
0.649298597194389	-1.94260964305413\\
0.633266533066132	-1.94238027890235\\
0.617234468937876	-1.94157423236356\\
0.601202404809619	-1.94021133531566\\
0.585170340681363	-1.93830982133684\\
0.569138276553106	-1.93588641624351\\
0.55310621242485	-1.93295642109648\\
0.548019381576782	-1.93186372745491\\
0.537074148296593	-1.92953305696762\\
0.521042084168337	-1.92563085023898\\
0.50501002004008	-1.92126222414596\\
0.488977955911824	-1.91643779725655\\
0.487146074453037	-1.91583166332665\\
0.472945891783567	-1.91116918961357\\
0.456913827655311	-1.9054657821794\\
0.442103679083632	-1.8997995991984\\
0.440881763527054	-1.89933546148151\\
0.424849699398798	-1.89279061863763\\
0.408817635270541	-1.8858346643723\\
0.404332999824194	-1.88376753507014\\
0.392785571142285	-1.8784790028984\\
0.376753507014028	-1.87072758265841\\
0.370881128353451	-1.86773547094188\\
0.360721442885771	-1.86258938978954\\
0.344689378757515	-1.85406785039725\\
0.340444832810142	-1.85170340681363\\
0.328657314629258	-1.84517246283947\\
0.312625250501002	-1.83590190159421\\
0.312243254575274	-1.83567134268537\\
0.296593186372745	-1.82627154452174\\
0.285972676222856	-1.81963927855711\\
0.280561122244489	-1.81627548532647\\
0.264529058116232	-1.8059229509761\\
0.261071533468998	-1.80360721442886\\
0.248496993987976	-1.795219764379\\
0.237425008585017	-1.7875751503006\\
0.232464929859719	-1.78416365914826\\
0.216432865731463	-1.77276117036572\\
0.21477460953327	-1.77154308617234\\
0.200400801603206	-1.76101992105668\\
0.193104318827784	-1.75551102204409\\
0.18436873747495	-1.7489359039608\\
0.172173345241226	-1.73947895791583\\
0.168336673346693	-1.73651220130204\\
0.152304609218437	-1.72375235381353\\
0.151931278795001	-1.72344689378758\\
0.13627254509018	-1.71066447591853\\
0.132396643568474	-1.70741482965932\\
0.120240480961924	-1.69724375478096\\
0.113414571508777	-1.69138276553106\\
0.104208416833667	-1.68349221526087\\
0.0949445761708529	-1.67535070140281\\
0.0881763527054105	-1.66941162343815\\
0.0769501176063007	-1.65931863727455\\
0.0721442885771539	-1.6550034888922\\
0.0593981639867881	-1.64328657314629\\
0.0561122244488974	-1.64026906664833\\
0.042258810450849	-1.62725450901804\\
0.0400801603206409	-1.62520935879441\\
0.0255049466931636	-1.61122244488978\\
0.0240480961923843	-1.60982511577077\\
0.00911196622221084	-1.59519038076152\\
0.00801603206412782	-1.59411683733554\\
-0.0069424886381464	-1.57915831663327\\
-0.00801603206412826	-1.57808477320728\\
-0.0226787509838312	-1.56312625250501\\
-0.0240480961923848	-1.561728923386\\
-0.0381153270050997	-1.54709418837675\\
-0.0400801603206413	-1.54504903815313\\
-0.0532690706591176	-1.5310621242485\\
-0.0561122244488979	-1.52804461775053\\
-0.0681553383640199	-1.51503006012024\\
-0.0721442885771544	-1.5107149117379\\
-0.082788126291551	-1.49899799599198\\
-0.0881763527054109	-1.49305891802733\\
-0.0971801924616865	-1.48296593186373\\
-0.104208416833667	-1.47507538159353\\
-0.111343165528784	-1.46693386773547\\
-0.120240480961924	-1.45676279285711\\
-0.125287641884237	-1.45090180360721\\
-0.13627254509018	-1.43811938573817\\
-0.13902327247694	-1.43486973947896\\
-0.152304609218437	-1.41914313537665\\
-0.152558840563168	-1.4188376753507\\
-0.165878065541669	-1.40280561122244\\
-0.168336673346694	-1.39983885460865\\
-0.179011697335402	-1.38677354709419\\
-0.18436873747495	-1.3801984290109\\
-0.191969329661105	-1.37074148296593\\
-0.200400801603207	-1.36021831785027\\
-0.204757187238359	-1.35470941883768\\
-0.216432865731463	-1.33989543890279\\
-0.21738088618959	-1.33867735470942\\
-0.229825446163611	-1.32264529058116\\
-0.232464929859719	-1.31923379942883\\
-0.242111014129311	-1.30661322645291\\
-0.248496993987976	-1.29822577640305\\
-0.254249679111395	-1.29058116232465\\
-0.264529058116232	-1.27686483474363\\
-0.266245197116871	-1.27454909819639\\
-0.278085913715293	-1.25851703406814\\
-0.280561122244489	-1.25515324083749\\
-0.289782528321357	-1.24248496993988\\
-0.296593186372745	-1.23308517177625\\
-0.301349186753381	-1.22645290581162\\
-0.312625250501002	-1.2106514005922\\
-0.312788164902552	-1.21042084168337\\
-0.324080466620025	-1.19438877755511\\
-0.328657314629258	-1.18785783358095\\
-0.33525274492881	-1.17835671342685\\
-0.344689378757515	-1.16468909288222\\
-0.346307219645143	-1.1623246492986\\
-0.357232011798649	-1.14629258517034\\
-0.360721442885771	-1.141146504018\\
-0.368039327265191	-1.13026052104208\\
-0.376753507014028	-1.11722056863035\\
-0.37873671886943	-1.11422845691383\\
-0.389314546384562	-1.09819639278557\\
-0.392785571142285	-1.09290786061383\\
-0.399781828147799	-1.08216432865731\\
-0.408817635270541	-1.06819939383122\\
-0.410145301707386	-1.06613226452906\\
-0.420395503992964	-1.0501002004008\\
-0.424849699398798	-1.04309121984003\\
-0.430543682829265	-1.03406813627255\\
-0.440592084630359	-1.01803607214429\\
-0.440881763527054	-1.0175719344274\\
-0.450533390184912	-1.00200400801603\\
-0.456913827655311	-0.991638126868775\\
-0.46037981639122	-0.985971943887776\\
-0.470127348333342	-0.969939879759519\\
-0.472945891783567	-0.965277406046435\\
-0.479778661095603	-0.953907815631263\\
-0.488977955911824	-0.938481885432904\\
-0.489337460739446	-0.937875751503006\\
-0.498800763532823	-0.92184368737475\\
-0.50501002004008	-0.911242184065805\\
-0.508174479282237	-0.905811623246493\\
-0.517457049057759	-0.889779559118236\\
-0.521042084168337	-0.883546681902311\\
-0.526651618119988	-0.87374749498998\\
-0.535757770323694	-0.857715430861723\\
-0.537074148296593	-0.855384760374438\\
-0.544778659699726	-0.841683366733467\\
-0.55310621242485	-0.826743996246779\\
-0.553712876870233	-0.82565130260521\\
-0.562564761636584	-0.809619238476954\\
-0.569138276553106	-0.797609863137293\\
-0.571332121958972	-0.793587174348697\\
-0.5800184813075	-0.777555110220441\\
-0.585170340681363	-0.767969139974112\\
-0.588623238808741	-0.761523046092184\\
-0.597147798796266	-0.745490981963928\\
-0.601202404809619	-0.737806525696426\\
-0.60559380677074	-0.729458917835671\\
-0.613960138273222	-0.713426853707415\\
-0.617234468937876	-0.707105294487803\\
-0.622250864274681	-0.697394789579158\\
-0.630462387886829	-0.681362725450902\\
-0.633266533066132	-0.675847212770085\\
-0.638600927584093	-0.665330661322646\\
-0.646660918238621	-0.649298597194389\\
-0.649298597194389	-0.644012448665355\\
-0.654650008190866	-0.633266533066132\\
-0.662561599507746	-0.617234468937876\\
-0.665330661322646	-0.611579473271343\\
-0.670403628910887	-0.601202404809619\\
-0.678169817286189	-0.585170340681363\\
-0.681362725450902	-0.578524953025125\\
-0.685866838737788	-0.569138276553106\\
-0.693490487181081	-0.55310621242485\\
-0.697394789579158	-0.544823632351483\\
-0.701044226507351	-0.537074148296593\\
-0.708528068235948	-0.521042084168337\\
-0.713426853707415	-0.510448205704374\\
-0.715939933420816	-0.50501002004008\\
-0.723286575218568	-0.488977955911824\\
-0.729458917835671	-0.475369178011717\\
-0.730557664471301	-0.472945891783567\\
-0.737769589818998	-0.456913827655311\\
-0.744902759144508	-0.440881763527054\\
-0.745490981963928	-0.439549388500093\\
-0.751980270797614	-0.424849699398798\\
-0.758981402159147	-0.408817635270541\\
-0.761523046092184	-0.402945164657456\\
-0.765921363119241	-0.392785571142285\\
-0.772792646047509	-0.376753507014028\\
-0.777555110220441	-0.365527472868787\\
-0.779595206106096	-0.360721442885771\\
-0.78633871591627	-0.344689378757515\\
-0.79300605003884	-0.328657314629258\\
-0.793587174348697	-0.327247893379991\\
-0.799621440602657	-0.312625250501002\\
-0.806163058730893	-0.296593186372745\\
-0.809619238476954	-0.288036544144153\\
-0.81264225875116	-0.280561122244489\\
-0.819059962112338	-0.264529058116232\\
-0.825403267700109	-0.248496993987976\\
-0.82565130260521	-0.247864371429404\\
-0.831697706153768	-0.232464929859719\\
-0.83791872660962	-0.216432865731463\\
-0.841683366733467	-0.206625295316765\\
-0.84407685759974	-0.200400801603207\\
-0.850177130537458	-0.18436873747495\\
-0.856204317512007	-0.168336673346694\\
-0.857715430861723	-0.164275435029404\\
-0.862178565765761	-0.152304609218437\\
-0.868086222771196	-0.13627254509018\\
-0.87374749498998	-0.120720775558426\\
-0.873922746025479	-0.120240480961924\\
-0.879712154483288	-0.104208416833667\\
-0.885429338962689	-0.0881763527054109\\
-0.889779559118236	-0.075829463852013\\
-0.891081352733647	-0.0721442885771544\\
-0.896681104837513	-0.0561122244488979\\
-0.902209517311712	-0.0400801603206413\\
-0.905811623246493	-0.0295057008813555\\
-0.907676036994777	-0.0240480961923848\\
-0.913087586458924	-0.00801603206412826\\
-0.918428489847537	0.00801603206412782\\
-0.92184368737475	0.0183980272195441\\
-0.923708101123034	0.0240480961923843\\
-0.928932465899462	0.0400801603206409\\
-0.934086689113959	0.0561122244488974\\
-0.937875751503006	0.0680549808883089\\
-0.939177545118416	0.0721442885771539\\
-0.944215309975965	0.0881763527054105\\
-0.949183249838047	0.104208416833667\\
-0.953907815631263	0.119669068517051\\
-0.954083066666762	0.120240480961924\\
-0.958934384098946	0.13627254509018\\
-0.963716005591637	0.152304609218437\\
-0.968428766409803	0.168336673346693\\
-0.969939879759519	0.173551239282139\\
-0.973086645577692	0.18436873747495\\
-0.977681480427534	0.200400801603206\\
-0.982207303763929	0.216432865731463\\
-0.985971943887776	0.229972659443534\\
-0.986667731865482	0.232464929859719\\
-0.991074875445203	0.248496993987976\\
-0.995412667523159	0.264529058116232\\
-0.99968171568218	0.280561122244489\\
-1.00200400801603	0.289425026676694\\
-1.00389005021323	0.296593186372745\\
-1.00803827426999	0.312625250501002\\
-1.01211712930511	0.328657314629258\\
-1.01612708530091	0.344689378757515\\
-1.01803607214429	0.352457101605714\\
-1.02007616802994	0.360721442885771\\
-1.0239626482088	0.376753507014028\\
-1.02777931323051	0.392785571142285\\
-1.03152649233951	0.408817635270541\\
-1.03406813627255	0.419900398966127\\
-1.03520845019296	0.424849699398798\\
-1.03882897978193	0.440881763527054\\
-1.04237880825587	0.456913827655311\\
-1.04585811988899	0.472945891783567\\
-1.04926705264232	0.488977955911824\\
-1.0501002004008	0.492983895502289\\
-1.0526132801142	0.50501002004008\\
-1.05589018644029	0.521042084168337\\
-1.0590950852347	0.537074148296593\\
-1.06222796213098	0.55310621242485\\
-1.06528875386917	0.569138276553106\\
-1.06613226452906	0.57367075089988\\
-1.06828280652761	0.585170340681363\\
-1.0712052321173	0.601202404809619\\
-1.07405350670123	0.617234468937876\\
-1.07682740507046	0.633266533066132\\
-1.07952664970155	0.649298597194389\\
-1.08215090986002	0.665330661322646\\
-1.08216432865731	0.665415321848262\\
-1.08470439642679	0.681362725450902\\
-1.08718072399412	0.697394789579159\\
-1.08957941337184	0.713426853707415\\
-1.09189995632039	0.729458917835672\\
-1.09414178677222	0.745490981963928\\
-1.09630427959414	0.761523046092185\\
-1.09819639278557	0.776094740459559\\
-1.09838695251236	0.777555110220441\\
-1.10039023819144	0.793587174348698\\
-1.10231088468654	0.809619238476954\\
-1.10414806928106	0.825651302605211\\
-1.1059009041887	0.841683366733467\\
-1.10756843491448	0.857715430861724\\
-1.10914963852203	0.87374749498998\\
-1.11064342180325	0.889779559118236\\
-1.11204861934615	0.905811623246493\\
-1.11336399149656	0.921843687374749\\
-1.11422845691383	0.933182006967935\\
-1.11458796174145	0.937875751503006\\
-1.1157184454276	0.953907815631262\\
-1.11675443604705	0.969939879759519\\
-1.11769444564974	0.985971943887775\\
-1.11853690276846	1.00200400801603\\
-1.11928014981559	1.01803607214429\\
-1.11992244034429	1.03406813627254\\
-1.12046193616776	1.0501002004008\\
-1.12089670432974	1.06613226452906\\
-1.12122471391934	1.08216432865731\\
-1.12144383272227	1.09819639278557\\
-1.12155182370072	1.11422845691383\\
-1.12154634129325	1.13026052104208\\
-1.12142492752561	1.14629258517034\\
-1.12118500792299	1.1623246492986\\
-1.12082388721338	1.17835671342685\\
-1.12033874481139	1.19438877755511\\
-1.1197266300709	1.21042084168337\\
-1.11898445729446	1.22645290581162\\
-1.11810900048634	1.24248496993988\\
-1.11709688783561	1.25851703406814\\
-1.11594459591447	1.27454909819639\\
-1.11464844357632	1.29058116232465\\
-1.11422845691383	1.29527490685972\\
}--cycle;


\addplot[area legend,solid,fill=mycolor8,draw=black,forget plot]
table[row sep=crcr] {%
x	y\\
-0.969939879759519	1.10267832172919\\
-0.969057219367736	1.11422845691383\\
-0.967666023500817	1.13026052104208\\
-0.96610008118631	1.14629258517034\\
-0.964353882669638	1.1623246492986\\
-0.962421651452387	1.17835671342685\\
-0.960297331194264	1.19438877755511\\
-0.957974571822777	1.21042084168337\\
-0.955446714796597	1.22645290581162\\
-0.953907815631263	1.23550430311173\\
-0.95268652032031	1.24248496993988\\
-0.949675465068525	1.25851703406814\\
-0.946430689407037	1.27454909819639\\
-0.942943626489223	1.29058116232465\\
-0.939205278892995	1.30661322645291\\
-0.937875751503006	1.31199129212065\\
-0.935151423287963	1.32264529058116\\
-0.930790674073424	1.33867735470942\\
-0.926138506431433	1.35470941883768\\
-0.92184368737475	1.36863159698073\\
-0.921167596555201	1.37074148296593\\
-0.91576922275681	1.38677354709419\\
-0.910028348601568	1.40280561122244\\
-0.905811623246493	1.4139460396668\\
-0.903879403874429	1.4188376753507\\
-0.89722860759457	1.43486973947896\\
-0.890169387584158	1.45090180360721\\
-0.889779559118236	1.45175054669336\\
-0.882464023832977	1.46693386773547\\
-0.874275126038673	1.48296593186373\\
-0.87374749498998	1.4839573570158\\
-0.865310295946134	1.49899799599198\\
-0.857715430861723	1.51182056797403\\
-0.85570327730774	1.51503006012024\\
-0.845197697745205	1.5310621242485\\
-0.841683366733467	1.5361856585203\\
-0.83371642719017	1.54709418837675\\
-0.82565130260521	1.55763141692518\\
-0.821149424312379	1.56312625250501\\
-0.809619238476954	1.57659144839573\\
-0.807252347724585	1.57915831663327\\
-0.793587174348697	1.59337379320406\\
-0.791693855339852	1.59519038076152\\
-0.777555110220441	1.60823431958121\\
-0.774015623754877	1.61122244488978\\
-0.761523046092184	1.62138648351191\\
-0.753568053904924	1.62725450901804\\
-0.745490981963928	1.6330089628635\\
-0.729458917835671	1.64325132499136\\
-0.729397545610944	1.64328657314629\\
-0.713426853707415	1.65217400092951\\
-0.698733223512349	1.65931863727455\\
-0.697394789579158	1.65995043575415\\
-0.681362725450902	1.66660099071443\\
-0.665330661322646	1.67225062416767\\
-0.654827748809892	1.67535070140281\\
-0.649298597194389	1.67694127241858\\
-0.633266533066132	1.68072001757187\\
-0.617234468937876	1.6836551377129\\
-0.601202404809619	1.68578743348757\\
-0.585170340681363	1.68715450310275\\
-0.569138276553106	1.6877909773566\\
-0.55310621242485	1.68772873313515\\
-0.537074148296593	1.68699708749149\\
-0.521042084168337	1.68562297418631\\
-0.50501002004008	1.68363110436025\\
-0.488977955911824	1.681044112825\\
-0.472945891783567	1.67788269129889\\
-0.462062759340567	1.67535070140281\\
-0.456913827655311	1.67416168445254\\
-0.440881763527054	1.66989348069114\\
-0.424849699398798	1.66510347263781\\
-0.408817635270541	1.65980612744266\\
-0.40747920133735	1.65931863727455\\
-0.392785571142285	1.65400113858226\\
-0.376753507014028	1.64771432457223\\
-0.366276833404529	1.64328657314629\\
-0.360721442885771	1.64095253996619\\
-0.344689378757515	1.63372219754025\\
-0.331202431040213	1.62725450901804\\
-0.328657314629258	1.62604056421676\\
-0.312625250501002	1.61790654165211\\
-0.300132672838309	1.61122244488978\\
-0.296593186372745	1.60933788456875\\
-0.280561122244489	1.60033434295526\\
-0.271830152307451	1.59519038076152\\
-0.264529058116232	1.59090760245724\\
-0.248496993987976	1.5810633694857\\
-0.245529404938141	1.57915831663327\\
-0.232464929859719	1.57080380915392\\
-0.220934744024294	1.56312625250501\\
-0.216432865731463	1.56013937290885\\
-0.200400801603207	1.54907207324439\\
-0.19764255158661	1.54709418837675\\
-0.18436873747495	1.53760511938098\\
-0.175535084883226	1.5310621242485\\
-0.168336673346694	1.52574538852246\\
-0.15431676277242	1.51503006012024\\
-0.152304609218437	1.51349613165084\\
-0.13627254509018	1.50085872488534\\
-0.133985795284029	1.49899799599198\\
-0.120240480961924	1.48783640452733\\
-0.11442157751627	1.48296593186373\\
-0.104208416833667	1.47443279432367\\
-0.0954918879906701	1.46693386773547\\
-0.0881763527054109	1.46064992979705\\
-0.0771441629499948	1.45090180360721\\
-0.0721442885771544	1.44648955083659\\
-0.0593315894323775	1.43486973947896\\
-0.0561122244488979	1.4319531039411\\
-0.0420123796927054	1.4188376753507\\
-0.0400801603206413	1.41704174408248\\
-0.0251491528545071	1.40280561122244\\
-0.0240480961923848	1.40175633619233\\
-0.00870840581046833	1.38677354709419\\
-0.00801603206412826	1.38609745627464\\
0.00733994124458037	1.37074148296593\\
0.00801603206412782	1.37006539214638\\
0.0230229365910116	1.35470941883768\\
0.0240480961923843	1.35366014380756\\
0.0383649358792694	1.33867735470942\\
0.0400801603206409	1.33688142344119\\
0.0533878962338543	1.32264529058116\\
0.0561122244488974	1.3197286550433\\
0.0681116322727954	1.30661322645291\\
0.0721442885771539	1.30220097368228\\
0.0825540396247736	1.29058116232465\\
0.0881763527054105	1.28429722438623\\
0.0967312906094415	1.27454909819639\\
0.104208416833667	1.26601596065633\\
0.110658005995674	1.25851703406814\\
0.120240480961924	1.24735544260349\\
0.124347406134783	1.24248496993988\\
0.13627254509018	1.22831363470498\\
0.137811444255516	1.22645290581162\\
0.151041518246361	1.21042084168337\\
0.152304609218437	1.20888691321397\\
0.164043027344138	1.19438877755511\\
0.168336673346693	1.18907204182907\\
0.176850509167818	1.17835671342685\\
0.18436873747495	1.16886764443108\\
0.189472282332026	1.1623246492986\\
0.200400801603206	1.14827047003798\\
0.201915825520064	1.14629258517034\\
0.21415900947276	1.13026052104208\\
0.216432865731463	1.12727364144593\\
0.226219685290747	1.11422845691383\\
0.232464929859719	1.10587394943448\\
0.238124052009175	1.09819639278557\\
0.248496993987976	1.08406938150975\\
0.249877068771455	1.08216432865731\\
0.261449565491779	1.06613226452906\\
0.264529058116232	1.06184948622477\\
0.272867526498327	1.0501002004008\\
0.280561122244489	1.03921209846628\\
0.284150637741463	1.03406813627254\\
0.295289049977966	1.01803607214429\\
0.296593186372745	1.01615151182326\\
0.306265440129244	1.00200400801603\\
0.312625250501002	0.992656040650101\\
0.317120335276949	0.985971943887775\\
0.327848425282381	0.969939879759519\\
0.328657314629258	0.968725934958247\\
0.338421193847936	0.953907815631262\\
0.344689378757515	0.944343440025218\\
0.348883112072032	0.937875751503006\\
0.359223219238986	0.921843687374749\\
0.360721442885771	0.919509654194643\\
0.369424311362763	0.905811623246493\\
0.376753507014028	0.894207310544178\\
0.37952284977305	0.889779559118236\\
0.389495963689498	0.87374749498998\\
0.392785571142285	0.86842999629769\\
0.399352061568488	0.857715430861724\\
0.408817635270541	0.842170856901581\\
0.409111855743252	0.841683366733467\\
0.418737689933317	0.825651302605211\\
0.424849699398798	0.81540407384021\\
0.428270836853321	0.809619238476954\\
0.437693808705884	0.793587174348698\\
0.440881763527054	0.788129953637036\\
0.447009199199949	0.777555110220441\\
0.456233617637009	0.761523046092185\\
0.456913827655311	0.760334029141922\\
0.465339483189213	0.745490981963928\\
0.472945891783567	0.731990907731753\\
0.474362142761551	0.729458917835672\\
0.48327342673314	0.713426853707415\\
0.488977955911824	0.703088201001349\\
0.492097725170469	0.697394789579159\\
0.500821999933126	0.681362725450902\\
0.50501002004008	0.673611064280093\\
0.509454470646171	0.665330661322646\\
0.51799543742193	0.649298597194389\\
0.521042084168337	0.643538805849636\\
0.526442067692356	0.633266533066132\\
0.534803268494776	0.617234468937876\\
0.537074148296593	0.612848790898299\\
0.543069523425128	0.601202404809619\\
0.551254345150408	0.585170340681363\\
0.55310621242485	0.581516308285451\\
0.559345189869164	0.569138276553106\\
0.567356868153898	0.55310621242485\\
0.569138276553106	0.549514424250385\\
0.575276788374131	0.537074148296593\\
0.583118411224556	0.521042084168337\\
0.585170340681363	0.516813821740019\\
0.590871432247094	0.50501002004008\\
0.598545943444414	0.488977955911824\\
0.601202404809619	0.483382623868328\\
0.606135647689298	0.472945891783567\\
0.613645849975363	0.456913827655311\\
0.617234468937876	0.44918619983715\\
0.621075393118784	0.440881763527054\\
0.62842395116611	0.424849699398798\\
0.633266533066132	0.414186951439601\\
0.635696076953843	0.408817635270541\\
0.642885520123645	0.392785571142285\\
0.649298597194389	0.378344078029799\\
0.650002573926212	0.376753507014028\\
0.657035298817787	0.360721442885771\\
0.663996859551543	0.344689378757515\\
0.665330661322646	0.341589301522375\\
0.670877512781642	0.328657314629258\\
0.677684346364305	0.312625250501002\\
0.681362725450902	0.303875539812623\\
0.684415884465412	0.296593186372745\\
0.691070078445277	0.280561122244489\\
0.697394789579159	0.265160856595837\\
0.697653645295825	0.264529058116232\\
0.704157161442419	0.248496993987976\\
0.71058999818202	0.232464929859719\\
0.713426853707415	0.225320293514676\\
0.716948222425681	0.216432865731463\\
0.723231725412169	0.200400801603206\\
0.729445402395971	0.18436873747495\\
0.729458917835672	0.184333489320014\\
0.7355812556841	0.168336673346693\\
0.741646656558198	0.152304609218437\\
0.745490981963928	0.142026998935648\\
0.747640164685551	0.13627254509018\\
0.753558788829143	0.120240480961924\\
0.759407488220804	0.104208416833667\\
0.761523046092185	0.0983403913275378\\
0.765182801678966	0.0881763527054105\\
0.77088533435874	0.0721442885771539\\
0.776518228317745	0.0561122244488974\\
0.777555110220441	0.053124099140325\\
0.782076830845543	0.0400801603206409\\
0.787563890959639	0.0240480961923843\\
0.792981374834583	0.00801603206412782\\
0.793587174348698	0.00619944450666274\\
0.798324247801242	-0.00801603206412826\\
0.803595955087895	-0.0240480961923848\\
0.80879792546851	-0.0400801603206413\\
0.809619238476954	-0.0426470285581829\\
0.81392555091599	-0.0561122244488979\\
0.818981526743509	-0.0721442885771544\\
0.823967381477785	-0.0881763527054109\\
0.825651302605211	-0.0936711882852378\\
0.828879743236386	-0.104208416833667\\
0.833719109470425	-0.120240480961924\\
0.838487745757591	-0.13627254509018\\
0.841683366733467	-0.14718107494664\\
0.843184328639133	-0.152304609218437\\
0.847805704581895	-0.168336673346694\\
0.852355515131915	-0.18436873747495\\
0.856834199960817	-0.200400801603207\\
0.857715430861724	-0.203610293749422\\
0.86123679957999	-0.216432865731463\\
0.865565666819784	-0.232464929859719\\
0.869822225003994	-0.248496993987976\\
0.87374749498998	-0.263537632964165\\
0.874006350706646	-0.264529058116232\\
0.878111639002643	-0.280561122244489\\
0.8821431997404	-0.296593186372745\\
0.88610118003164	-0.312625250501002\\
0.889779559118236	-0.32780857154311\\
0.889985305387594	-0.328657314629258\\
0.893788465297644	-0.344689378757515\\
0.897516260741634	-0.360721442885771\\
0.901168657328444	-0.376753507014028\\
0.904745563887746	-0.392785571142285\\
0.905811623246493	-0.397677206826184\\
0.908241167134204	-0.408817635270541\\
0.911657017285245	-0.424849699398798\\
0.914995079503241	-0.440881763527054\\
0.918255068412237	-0.456913827655311\\
0.921436637000515	-0.472945891783567\\
0.921843687374749	-0.47505577776877\\
0.924531625014911	-0.488977955911824\\
0.92754477894048	-0.50501002004008\\
0.93047666597542	-0.521042084168337\\
0.933326728992571	-0.537074148296593\\
0.936094343103798	-0.55310621242485\\
0.937875751503006	-0.563760210885365\\
0.938775714503375	-0.569138276553106\\
0.941366364051953	-0.585170340681363\\
0.943871126631541	-0.601202404809619\\
0.946289147749201	-0.617234468937876\\
0.94861949758841	-0.633266533066132\\
0.950861168884856	-0.649298597194389\\
0.953013074674022	-0.665330661322646\\
0.953907815631262	-0.672311328150791\\
0.955068713952377	-0.681362725450902\\
0.957027584889907	-0.697394789579158\\
0.958892252810273	-0.713426853707415\\
0.960661363087028	-0.729458917835671\\
0.962333471165166	-0.745490981963928\\
0.963907039627009	-0.761523046092184\\
0.965380435091499	-0.777555110220441\\
0.966751924938348	-0.793587174348697\\
0.968019673847939	-0.809619238476954\\
0.969181740147287	-0.82565130260521\\
0.969939879759519	-0.837201437789848\\
0.97023410023223	-0.841683366733467\\
0.971171862150369	-0.857715430861723\\
0.971997666920664	-0.87374749498998\\
0.972709222518539	-0.889779559118236\\
0.973304114743161	-0.905811623246493\\
0.973779802541407	-0.92184368737475\\
0.974133613074036	-0.937875751503006\\
0.974362736509592	-0.953907815631263\\
0.974464220530582	-0.969939879759519\\
0.974434964535466	-0.985971943887776\\
0.974271713518829	-1.00200400801603\\
0.973971051610933	-1.01803607214429\\
0.973529395256493	-1.03406813627255\\
0.972942986011204	-1.0501002004008\\
0.97220788293294	-1.06613226452906\\
0.971319954542998	-1.08216432865731\\
0.970274870330963	-1.09819639278557\\
0.969939879759519	-1.10267832172919\\
0.969057219367736	-1.11422845691383\\
0.967666023500816	-1.13026052104208\\
0.96610008118631	-1.14629258517034\\
0.964353882669638	-1.1623246492986\\
0.962421651452387	-1.17835671342685\\
0.960297331194265	-1.19438877755511\\
0.957974571822777	-1.21042084168337\\
0.955446714796598	-1.22645290581162\\
0.953907815631262	-1.23550430311173\\
0.95268652032031	-1.24248496993988\\
0.949675465068525	-1.25851703406814\\
0.946430689407037	-1.27454909819639\\
0.942943626489224	-1.29058116232465\\
0.939205278892995	-1.30661322645291\\
0.937875751503006	-1.31199129212065\\
0.935151423287963	-1.32264529058116\\
0.930790674073424	-1.33867735470942\\
0.926138506431433	-1.35470941883768\\
0.921843687374749	-1.36863159698073\\
0.921167596555201	-1.37074148296593\\
0.91576922275681	-1.38677354709419\\
0.910028348601568	-1.40280561122244\\
0.905811623246493	-1.4139460396668\\
0.903879403874429	-1.4188376753507\\
0.897228607594571	-1.43486973947896\\
0.890169387584157	-1.45090180360721\\
0.889779559118236	-1.45175054669336\\
0.882464023832978	-1.46693386773547\\
0.874275126038673	-1.48296593186373\\
0.87374749498998	-1.4839573570158\\
0.865310295946134	-1.49899799599198\\
0.857715430861724	-1.51182056797403\\
0.85570327730774	-1.51503006012024\\
0.845197697745205	-1.5310621242485\\
0.841683366733467	-1.53618565852029\\
0.83371642719017	-1.54709418837675\\
0.825651302605211	-1.55763141692518\\
0.821149424312379	-1.56312625250501\\
0.809619238476954	-1.57659144839572\\
0.807252347724586	-1.57915831663327\\
0.793587174348698	-1.59337379320406\\
0.791693855339852	-1.59519038076152\\
0.777555110220441	-1.60823431958121\\
0.774015623754877	-1.61122244488978\\
0.761523046092185	-1.62138648351191\\
0.753568053904925	-1.62725450901804\\
0.745490981963928	-1.6330089628635\\
0.729458917835672	-1.64325132499136\\
0.729397545610944	-1.64328657314629\\
0.713426853707415	-1.65217400092951\\
0.698733223512349	-1.65931863727455\\
0.697394789579159	-1.65995043575415\\
0.681362725450902	-1.66660099071443\\
0.665330661322646	-1.67225062416767\\
0.654827748809895	-1.67535070140281\\
0.649298597194389	-1.67694127241858\\
0.633266533066132	-1.68072001757187\\
0.617234468937876	-1.6836551377129\\
0.601202404809619	-1.68578743348757\\
0.585170340681363	-1.68715450310274\\
0.569138276553106	-1.6877909773566\\
0.55310621242485	-1.68772873313515\\
0.537074148296593	-1.68699708749149\\
0.521042084168337	-1.68562297418631\\
0.50501002004008	-1.68363110436025\\
0.488977955911824	-1.681044112825\\
0.472945891783567	-1.67788269129889\\
0.462062759340563	-1.67535070140281\\
0.456913827655311	-1.67416168445254\\
0.440881763527054	-1.66989348069114\\
0.424849699398798	-1.66510347263781\\
0.408817635270541	-1.65980612744266\\
0.407479201337349	-1.65931863727455\\
0.392785571142285	-1.65400113858226\\
0.376753507014028	-1.64771432457223\\
0.366276833404527	-1.64328657314629\\
0.360721442885771	-1.64095253996619\\
0.344689378757515	-1.63372219754025\\
0.331202431040212	-1.62725450901804\\
0.328657314629258	-1.62604056421676\\
0.312625250501002	-1.61790654165211\\
0.300132672838309	-1.61122244488978\\
0.296593186372745	-1.60933788456875\\
0.280561122244489	-1.60033434295526\\
0.27183015230745	-1.59519038076152\\
0.264529058116232	-1.59090760245724\\
0.248496993987976	-1.5810633694857\\
0.245529404938141	-1.57915831663327\\
0.232464929859719	-1.57080380915392\\
0.220934744024294	-1.56312625250501\\
0.216432865731463	-1.56013937290885\\
0.200400801603206	-1.54907207324439\\
0.19764255158661	-1.54709418837675\\
0.18436873747495	-1.53760511938098\\
0.175535084883226	-1.5310621242485\\
0.168336673346693	-1.52574538852245\\
0.15431676277242	-1.51503006012024\\
0.152304609218437	-1.51349613165084\\
0.13627254509018	-1.50085872488534\\
0.133985795284029	-1.49899799599198\\
0.120240480961924	-1.48783640452733\\
0.11442157751627	-1.48296593186373\\
0.104208416833667	-1.47443279432367\\
0.0954918879906697	-1.46693386773547\\
0.0881763527054105	-1.46064992979705\\
0.0771441629499948	-1.45090180360721\\
0.0721442885771539	-1.44648955083659\\
0.0593315894323771	-1.43486973947896\\
0.0561122244488974	-1.4319531039411\\
0.042012379692705	-1.4188376753507\\
0.0400801603206409	-1.41704174408247\\
0.025149152854508	-1.40280561122244\\
0.0240480961923843	-1.40175633619233\\
0.0087084058104688	-1.38677354709419\\
0.00801603206412782	-1.38609745627464\\
-0.0073399412445799	-1.37074148296593\\
-0.00801603206412826	-1.37006539214638\\
-0.023022936591012	-1.35470941883768\\
-0.0240480961923848	-1.35366014380756\\
-0.0383649358792694	-1.33867735470942\\
-0.0400801603206413	-1.33688142344119\\
-0.0533878962338543	-1.32264529058116\\
-0.0561122244488979	-1.3197286550433\\
-0.0681116322727958	-1.30661322645291\\
-0.0721442885771544	-1.30220097368228\\
-0.0825540396247731	-1.29058116232465\\
-0.0881763527054109	-1.28429722438623\\
-0.0967312906094415	-1.27454909819639\\
-0.104208416833667	-1.26601596065633\\
-0.110658005995674	-1.25851703406814\\
-0.120240480961924	-1.24735544260349\\
-0.124347406134783	-1.24248496993988\\
-0.13627254509018	-1.22831363470498\\
-0.137811444255515	-1.22645290581162\\
-0.151041518246361	-1.21042084168337\\
-0.152304609218437	-1.20888691321397\\
-0.164043027344137	-1.19438877755511\\
-0.168336673346694	-1.18907204182907\\
-0.176850509167818	-1.17835671342685\\
-0.18436873747495	-1.16886764443108\\
-0.189472282332025	-1.1623246492986\\
-0.200400801603207	-1.14827047003798\\
-0.201915825520063	-1.14629258517034\\
-0.214159009472761	-1.13026052104208\\
-0.216432865731463	-1.12727364144593\\
-0.226219685290747	-1.11422845691383\\
-0.232464929859719	-1.10587394943448\\
-0.238124052009175	-1.09819639278557\\
-0.248496993987976	-1.08406938150975\\
-0.249877068771455	-1.08216432865731\\
-0.261449565491779	-1.06613226452906\\
-0.264529058116232	-1.06184948622477\\
-0.272867526498328	-1.0501002004008\\
-0.280561122244489	-1.03921209846628\\
-0.284150637741463	-1.03406813627255\\
-0.295289049977966	-1.01803607214429\\
-0.296593186372745	-1.01615151182326\\
-0.306265440129243	-1.00200400801603\\
-0.312625250501002	-0.992656040650101\\
-0.317120335276949	-0.985971943887776\\
-0.327848425282381	-0.969939879759519\\
-0.328657314629258	-0.968725934958247\\
-0.338421193847936	-0.953907815631263\\
-0.344689378757515	-0.944343440025219\\
-0.348883112072033	-0.937875751503006\\
-0.359223219238986	-0.92184368737475\\
-0.360721442885771	-0.919509654194644\\
-0.369424311362763	-0.905811623246493\\
-0.376753507014028	-0.894207310544178\\
-0.379522849773049	-0.889779559118236\\
-0.389495963689499	-0.87374749498998\\
-0.392785571142285	-0.86842999629769\\
-0.399352061568489	-0.857715430861723\\
-0.408817635270541	-0.84217085690158\\
-0.409111855743253	-0.841683366733467\\
-0.418737689933318	-0.82565130260521\\
-0.424849699398798	-0.81540407384021\\
-0.428270836853321	-0.809619238476954\\
-0.437693808705884	-0.793587174348697\\
-0.440881763527054	-0.788129953637036\\
-0.447009199199949	-0.777555110220441\\
-0.45623361763701	-0.761523046092184\\
-0.456913827655311	-0.760334029141922\\
-0.465339483189214	-0.745490981963928\\
-0.472945891783567	-0.731990907731753\\
-0.474362142761551	-0.729458917835671\\
-0.48327342673314	-0.713426853707415\\
-0.488977955911824	-0.703088201001348\\
-0.49209772517047	-0.697394789579158\\
-0.500821999933126	-0.681362725450902\\
-0.50501002004008	-0.673611064280093\\
-0.509454470646172	-0.665330661322646\\
-0.51799543742193	-0.649298597194389\\
-0.521042084168337	-0.643538805849636\\
-0.526442067692356	-0.633266533066132\\
-0.534803268494776	-0.617234468937876\\
-0.537074148296593	-0.612848790898299\\
-0.543069523425128	-0.601202404809619\\
-0.551254345150408	-0.585170340681363\\
-0.55310621242485	-0.581516308285452\\
-0.559345189869164	-0.569138276553106\\
-0.567356868153898	-0.55310621242485\\
-0.569138276553106	-0.549514424250385\\
-0.57527678837413	-0.537074148296593\\
-0.583118411224556	-0.521042084168337\\
-0.585170340681363	-0.516813821740019\\
-0.590871432247094	-0.50501002004008\\
-0.598545943444414	-0.488977955911824\\
-0.601202404809619	-0.483382623868328\\
-0.606135647689298	-0.472945891783567\\
-0.613645849975363	-0.456913827655311\\
-0.617234468937876	-0.449186199837151\\
-0.621075393118784	-0.440881763527054\\
-0.62842395116611	-0.424849699398798\\
-0.633266533066132	-0.414186951439601\\
-0.635696076953843	-0.408817635270541\\
-0.642885520123645	-0.392785571142285\\
-0.649298597194389	-0.378344078029798\\
-0.650002573926212	-0.376753507014028\\
-0.657035298817786	-0.360721442885771\\
-0.663996859551543	-0.344689378757515\\
-0.665330661322646	-0.341589301522375\\
-0.670877512781641	-0.328657314629258\\
-0.677684346364305	-0.312625250501002\\
-0.681362725450902	-0.303875539812625\\
-0.684415884465412	-0.296593186372745\\
-0.691070078445277	-0.280561122244489\\
-0.697394789579158	-0.265160856595838\\
-0.697653645295824	-0.264529058116232\\
-0.704157161442419	-0.248496993987976\\
-0.710589998182019	-0.232464929859719\\
-0.713426853707415	-0.225320293514677\\
-0.716948222425681	-0.216432865731463\\
-0.723231725412169	-0.200400801603207\\
-0.729445402395971	-0.18436873747495\\
-0.729458917835671	-0.184333489320016\\
-0.735581255684099	-0.168336673346694\\
-0.741646656558198	-0.152304609218437\\
-0.745490981963928	-0.142026998935649\\
-0.747640164685551	-0.13627254509018\\
-0.753558788829143	-0.120240480961924\\
-0.759407488220804	-0.104208416833667\\
-0.761523046092184	-0.0983403913275397\\
-0.765182801678965	-0.0881763527054109\\
-0.770885334358739	-0.0721442885771544\\
-0.776518228317744	-0.0561122244488979\\
-0.777555110220441	-0.0531240991403255\\
-0.782076830845542	-0.0400801603206413\\
-0.787563890959638	-0.0240480961923848\\
-0.792981374834583	-0.00801603206412826\\
-0.793587174348697	-0.00619944450666455\\
-0.798324247801242	0.00801603206412782\\
-0.803595955087895	0.0240480961923843\\
-0.80879792546851	0.0400801603206409\\
-0.809619238476954	0.0426470285581824\\
-0.81392555091599	0.0561122244488974\\
-0.818981526743509	0.0721442885771539\\
-0.823967381477785	0.0881763527054105\\
-0.82565130260521	0.0936711882852369\\
-0.828879743236386	0.104208416833667\\
-0.833719109470425	0.120240480961924\\
-0.838487745757591	0.13627254509018\\
-0.841683366733467	0.147181074946638\\
-0.843184328639133	0.152304609218437\\
-0.847805704581895	0.168336673346693\\
-0.852355515131915	0.18436873747495\\
-0.856834199960817	0.200400801603206\\
-0.857715430861723	0.20361029374942\\
-0.861236799579989	0.216432865731463\\
-0.865565666819784	0.232464929859719\\
-0.869822225003994	0.248496993987976\\
-0.87374749498998	0.263537632964162\\
-0.874006350706646	0.264529058116232\\
-0.878111639002643	0.280561122244489\\
-0.8821431997404	0.296593186372745\\
-0.88610118003164	0.312625250501002\\
-0.889779559118236	0.327808571543111\\
-0.889985305387594	0.328657314629258\\
-0.893788465297644	0.344689378757515\\
-0.897516260741634	0.360721442885771\\
-0.901168657328444	0.376753507014028\\
-0.904745563887746	0.392785571142285\\
-0.905811623246493	0.397677206826187\\
-0.908241167134204	0.408817635270541\\
-0.911657017285245	0.424849699398798\\
-0.914995079503241	0.440881763527054\\
-0.918255068412237	0.456913827655311\\
-0.921436637000515	0.472945891783567\\
-0.92184368737475	0.475055777768773\\
-0.924531625014912	0.488977955911824\\
-0.927544778940481	0.50501002004008\\
-0.930476665975421	0.521042084168337\\
-0.933326728992572	0.537074148296593\\
-0.936094343103798	0.55310621242485\\
-0.937875751503006	0.563760210885367\\
-0.938775714503375	0.569138276553106\\
-0.941366364051952	0.585170340681363\\
-0.943871126631541	0.601202404809619\\
-0.9462891477492	0.617234468937876\\
-0.94861949758841	0.633266533066132\\
-0.950861168884856	0.649298597194389\\
-0.953013074674022	0.665330661322646\\
-0.953907815631263	0.672311328150798\\
-0.955068713952377	0.681362725450902\\
-0.957027584889908	0.697394789579159\\
-0.958892252810273	0.713426853707415\\
-0.960661363087028	0.729458917835672\\
-0.962333471165166	0.745490981963928\\
-0.963907039627009	0.761523046092185\\
-0.965380435091499	0.777555110220441\\
-0.966751924938348	0.793587174348698\\
-0.968019673847939	0.809619238476954\\
-0.969181740147288	0.825651302605211\\
-0.969939879759519	0.837201437789848\\
-0.970234100232231	0.841683366733467\\
-0.971171862150368	0.857715430861724\\
-0.971997666920663	0.87374749498998\\
-0.97270922251854	0.889779559118236\\
-0.973304114743161	0.905811623246493\\
-0.973779802541406	0.921843687374749\\
-0.974133613074036	0.937875751503006\\
-0.974362736509593	0.953907815631262\\
-0.974464220530582	0.969939879759519\\
-0.974434964535466	0.985971943887775\\
-0.97427171351883	1.00200400801603\\
-0.973971051610933	1.01803607214429\\
-0.973529395256494	1.03406813627254\\
-0.972942986011204	1.0501002004008\\
-0.97220788293294	1.06613226452906\\
-0.971319954542998	1.08216432865731\\
-0.970274870330964	1.09819639278557\\
-0.969939879759519	1.10267832172919\\
}--cycle;


\addplot[area legend,solid,fill=mycolor9,draw=black,forget plot]
table[row sep=crcr] {%
x	y\\
-0.82565130260521	0.866245169292337\\
-0.825462060494201	0.87374749498998\\
-0.824887980644687	0.889779559118236\\
-0.824136376831622	0.905811623246493\\
-0.823201350070427	0.921843687374749\\
-0.822076686092449	0.937875751503006\\
-0.820755838230285	0.953907815631262\\
-0.819231909149522	0.969939879759519\\
-0.817497631338341	0.985971943887775\\
-0.815545346258589	1.00200400801603\\
-0.813366982053292	1.01803607214429\\
-0.810954029695988	1.03406813627254\\
-0.809619238476954	1.042176361739\\
-0.80826563518729	1.0501002004008\\
-0.805283581699165	1.06613226452906\\
-0.802028601208808	1.08216432865731\\
-0.79848934370967	1.09819639278557\\
-0.794653812367824	1.11422845691383\\
-0.793587174348697	1.11840086240547\\
-0.79042052310809	1.13026052104208\\
-0.785819903998319	1.14629258517034\\
-0.780866613022514	1.1623246492986\\
-0.777555110220441	1.17236205496192\\
-0.775478014964858	1.17835671342685\\
-0.769575432731745	1.19438877755511\\
-0.763245346289689	1.21042084168337\\
-0.761523046092184	1.21454979020275\\
-0.75627569446411	1.22645290581162\\
-0.748740773012284	1.24248496993988\\
-0.745490981963928	1.24902269292138\\
-0.7404736102047	1.25851703406814\\
-0.731474334627966	1.27454909819639\\
-0.729458917835671	1.27796128310748\\
-0.721480456734734	1.29058116232465\\
-0.713426853707415	1.30259251812186\\
-0.710523412722237	1.30661322645291\\
-0.698314123476152	1.32264529058116\\
-0.697394789579158	1.32379662789258\\
-0.684481340561341	1.33867735470942\\
-0.681362725450902	1.34209568133149\\
-0.66874898794471	1.35470941883768\\
-0.665330661322646	1.35797099976621\\
-0.650511183447168	1.37074148296593\\
-0.649298597194389	1.37174138065302\\
-0.633266533066132	1.38361953634077\\
-0.62845300893667	1.38677354709419\\
-0.617234468937876	1.39383666296571\\
-0.601202404809619	1.4025892937035\\
-0.600745994797927	1.40280561122244\\
-0.585170340681363	1.40992573720582\\
-0.569138276553106	1.4160452009309\\
-0.560232538149745	1.4188376753507\\
-0.55310621242485	1.42100045734942\\
-0.537074148296593	1.42486816501932\\
-0.521042084168337	1.42774204570758\\
-0.50501002004008	1.42967718233409\\
-0.488977955911824	1.43072393472941\\
-0.472945891783567	1.43092834074326\\
-0.456913827655311	1.43033247579527\\
-0.440881763527054	1.42897477560459\\
-0.424849699398798	1.42689032622384\\
-0.408817635270541	1.42411112497851\\
-0.392785571142285	1.42066631546198\\
-0.38565924541739	1.4188376753507\\
-0.376753507014028	1.41656648519904\\
-0.360721442885771	1.41183688755087\\
-0.344689378757515	1.40651358659104\\
-0.334651973094187	1.40280561122244\\
-0.328657314629258	1.40060301102082\\
-0.312625250501002	1.39411330148802\\
-0.296593186372745	1.38708617343689\\
-0.295935787071416	1.38677354709419\\
-0.280561122244489	1.37949587086625\\
-0.264529058116232	1.37139710129056\\
-0.263316471863453	1.37074148296593\\
-0.248496993987976	1.36276180547724\\
-0.234356124466997	1.35470941883768\\
-0.232464929859719	1.35363663963936\\
-0.216432865731463	1.3439988072933\\
-0.208016305591172	1.33867735470942\\
-0.200400801603207	1.33387843940809\\
-0.18436873747495	1.32328265171227\\
-0.183449403577956	1.32264529058116\\
-0.168336673346694	1.31219743040739\\
-0.160593790422456	1.30661322645291\\
-0.152304609218437	1.30065037543827\\
-0.138863158302592	1.29058116232465\\
-0.13627254509018	1.28864497519273\\
-0.120240480961924	1.27617679261093\\
-0.118225064169629	1.27454909819639\\
-0.104208416833667	1.26324929788337\\
-0.0985418869905856	1.25851703406814\\
-0.0881763527054109	1.24987385108974\\
-0.0796107483583451	1.24248496993988\\
-0.0721442885771544	1.23605250728087\\
-0.0613595760769719	1.22645290581162\\
-0.0561122244488979	1.22178697497293\\
-0.043725344827961	1.21042084168337\\
-0.0400801603206413	1.20707861834823\\
-0.0266527533644603	1.19438877755511\\
-0.0240480961923848	1.19192845901123\\
-0.0100931273197109	1.17835671342685\\
-0.00801603206412826	1.17633717730267\\
0.00599649593994431	1.1623246492986\\
0.00801603206412782	1.16030511317441\\
0.0216541346291227	1.14629258517034\\
0.0240480961923843	1.14383226662646\\
0.0369135090800333	1.13026052104208\\
0.0400801603206409	1.12691829770695\\
0.0518045890419841	1.11422845691383\\
0.0561122244488974	1.10956252607513\\
0.0663540590020255	1.09819639278557\\
0.0721442885771539	1.09176393012657\\
0.0805857154372647	1.08216432865731\\
0.0881763527054105	1.07352114567892\\
0.0945208072747595	1.06613226452906\\
0.104208416833667	1.05483246421604\\
0.108178328748497	1.0501002004008\\
0.120240480961924	1.03569583068708\\
0.121575272180958	1.03406813627254\\
0.134692032149139	1.01803607214429\\
0.13627254509018	1.01609988501237\\
0.147546506819147	1.00200400801603\\
0.152304609218437	0.9960411570014\\
0.160183002079825	0.985971943887775\\
0.168336673346693	0.975524083714005\\
0.172612977213375	0.969939879759519\\
0.18436873747495	0.954545176762373\\
0.184846659229807	0.953907815631262\\
0.196826185090445	0.937875751503006\\
0.200400801603206	0.933076836201674\\
0.208621100028441	0.921843687374749\\
0.216432865731463	0.91113307583037\\
0.220248467550492	0.905811623246493\\
0.231701607899197	0.889779559118236\\
0.232464929859719	0.888706779919921\\
0.242934002994659	0.87374749498998\\
0.248496993987976	0.865767817501285\\
0.254022025382232	0.857715430861724\\
0.264529058116232	0.842338985058091\\
0.264970265322122	0.841683366733467\\
0.275708137982974	0.825651302605211\\
0.280561122244489	0.818373626377271\\
0.286312909771483	0.809619238476954\\
0.296593186372745	0.793899800691397\\
0.296794755951398	0.793587174348698\\
0.307077408297698	0.777555110220441\\
0.312625250501002	0.768862800486012\\
0.317245772034545	0.761523046092185\\
0.327285519079521	0.745490981963928\\
0.328657314629258	0.743288381762303\\
0.337156300817286	0.729458917835672\\
0.344689378757515	0.717134829076012\\
0.346927181285696	0.713426853707415\\
0.356545602273577	0.697394789579159\\
0.360721442885771	0.690394001779324\\
0.366042486545644	0.681362725450902\\
0.375431709792827	0.665330661322646\\
0.376753507014028	0.663059471170983\\
0.384667400047787	0.649298597194389\\
0.392785571142285	0.635095173177409\\
0.393818986979759	0.633266533066132\\
0.402817842545186	0.617234468937876\\
0.408817635270541	0.60647585443743\\
0.411726576130805	0.601202404809619\\
0.42050868119499	0.585170340681363\\
0.424849699398798	0.577190927426249\\
0.429184778407611	0.569138276553106\\
0.437753779839534	0.55310621242485\\
0.440881763527054	0.547211248550484\\
0.446206654881929	0.537074148296593\\
0.454566045756848	0.521042084168337\\
0.456913827655311	0.516504820484648\\
0.462804351611308	0.50501002004008\\
0.470957473100776	0.488977955911824\\
0.472945891783567	0.48503655717613\\
0.478989140028764	0.472945891783567\\
0.486939183233613	0.456913827655311\\
0.488977955911824	0.452768022905766\\
0.494771454425069	0.440881763527054\\
0.502521462138104	0.424849699398798\\
0.50501002004008	0.419657142253924\\
0.510160926694367	0.408817635270541\\
0.517713795080866	0.392785571142285\\
0.521042084168337	0.385657877370906\\
0.525166418500278	0.376753507014028\\
0.532524898685601	0.360721442885771\\
0.537074148296593	0.350719868426134\\
0.539796051007148	0.344689378757515\\
0.546962750561728	0.328657314629258\\
0.55310621242485	0.314788032499723\\
0.554057232309417	0.312625250501002\\
0.561034616622361	0.296593186372745\\
0.567946882818409	0.280561122244489\\
0.569138276553106	0.277768647824683\\
0.574747076214539	0.264529058116232\\
0.581471212238471	0.248496993987976\\
0.585170340681363	0.23958505584309\\
0.588106045174488	0.232464929859719\\
0.594644196129463	0.216432865731463\\
0.601116114696609	0.200400801603206\\
0.601202404809619	0.200184484084263\\
0.607470969037548	0.18436873747495\\
0.613757593813339	0.168336673346693\\
0.617234468937876	0.159367725089954\\
0.619956043021835	0.152304609218437\\
0.626059204199459	0.13627254509018\\
0.632094429226009	0.120240480961924\\
0.633266533066132	0.117086470208509\\
0.63802471474842	0.104208416833667\\
0.643876682886971	0.0881763527054105\\
0.649298597194389	0.0731441862642451\\
0.649657305025131	0.0721442885771539\\
0.655327533694953	0.0561122244488974\\
0.660927542575827	0.0400801603206409\\
0.665330661322646	0.0273096771209221\\
0.666449447450091	0.0240480961923843\\
0.671867442661018	0.00801603206412782\\
0.677214141975015	-0.00801603206412826\\
0.681362725450902	-0.0206297695703205\\
0.682481511578348	-0.0240480961923848\\
0.687645617002839	-0.0400801603206413\\
0.692737070130544	-0.0561122244488979\\
0.697394789579159	-0.0709929512657397\\
0.6977534974099	-0.0721442885771544\\
0.702661469168185	-0.0881763527054109\\
0.707495148697988	-0.104208416833667\\
0.712254749867292	-0.120240480961924\\
0.713426853707415	-0.124261189292971\\
0.7169126464296	-0.13627254509018\\
0.721485427396051	-0.152304609218437\\
0.725982042711135	-0.168336673346694\\
0.729458917835672	-0.180956552563859\\
0.730395021803703	-0.18436873747495\\
0.734703172616581	-0.200400801603207\\
0.738932773283772	-0.216432865731463\\
0.743083645568415	-0.232464929859719\\
0.745490981963928	-0.241959271006479\\
0.747141849129499	-0.248496993987976\\
0.751099781625361	-0.264529058116232\\
0.754976124731029	-0.280561122244489\\
0.758770453206528	-0.296593186372745\\
0.761523046092185	-0.308496301981619\\
0.762474065976751	-0.312625250501002\\
0.766070269536165	-0.328657314629258\\
0.769581097463983	-0.344689378757515\\
0.773005860609449	-0.360721442885771\\
0.776343786249278	-0.376753507014028\\
0.777555110220441	-0.382748165478959\\
0.779575252649355	-0.392785571142285\\
0.782706016874727	-0.408817635270541\\
0.785745855233877	-0.424849699398798\\
0.788693703292883	-0.440881763527054\\
0.791548401670487	-0.456913827655311\\
0.793587174348698	-0.468773486291924\\
0.794301571550162	-0.472945891783567\\
0.796938650811452	-0.488977955911824\\
0.799477698304696	-0.50501002004008\\
0.801917229327064	-0.521042084168337\\
0.804255650319159	-0.537074148296593\\
0.806491254789434	-0.55310621242485\\
0.808622218986226	-0.569138276553106\\
0.809619238476954	-0.577062115214904\\
0.810635043483407	-0.585170340681363\\
0.812528179337217	-0.601202404809619\\
0.814310405770573	-0.617234468937876\\
0.815979507096244	-0.633266533066132\\
0.817533131510713	-0.649298597194389\\
0.818968785424053	-0.665330661322646\\
0.820283827444078	-0.681362725450902\\
0.821475461993017	-0.697394789579158\\
0.822540732533354	-0.713426853707415\\
0.82347651437768	-0.729458917835671\\
0.824279507055472	-0.745490981963928\\
0.824946226207663	-0.761523046092184\\
0.825472994977549	-0.777555110220441\\
0.825651302605211	-0.785057435918098\\
0.825852872183863	-0.793587174348697\\
0.826084201198702	-0.809619238476954\\
0.826165509867008	-0.82565130260521\\
0.826092509811099	-0.841683366733467\\
0.825860684610487	-0.857715430861723\\
0.825651302605211	-0.866245169292336\\
0.825462060494201	-0.87374749498998\\
0.824887980644688	-0.889779559118236\\
0.824136376831622	-0.905811623246493\\
0.823201350070428	-0.92184368737475\\
0.822076686092449	-0.937875751503006\\
0.820755838230285	-0.953907815631263\\
0.819231909149522	-0.969939879759519\\
0.817497631338341	-0.985971943887776\\
0.815545346258589	-1.00200400801603\\
0.813366982053292	-1.01803607214429\\
0.810954029695989	-1.03406813627255\\
0.809619238476954	-1.042176361739\\
0.80826563518729	-1.0501002004008\\
0.805283581699165	-1.06613226452906\\
0.802028601208808	-1.08216432865731\\
0.79848934370967	-1.09819639278557\\
0.794653812367823	-1.11422845691383\\
0.793587174348698	-1.11840086240547\\
0.790420523108089	-1.13026052104208\\
0.785819903998319	-1.14629258517034\\
0.780866613022514	-1.1623246492986\\
0.777555110220441	-1.17236205496192\\
0.775478014964857	-1.17835671342685\\
0.769575432731745	-1.19438877755511\\
0.763245346289689	-1.21042084168337\\
0.761523046092185	-1.21454979020275\\
0.75627569446411	-1.22645290581162\\
0.748740773012284	-1.24248496993988\\
0.745490981963928	-1.24902269292138\\
0.7404736102047	-1.25851703406814\\
0.731474334627967	-1.27454909819639\\
0.729458917835672	-1.27796128310748\\
0.721480456734734	-1.29058116232465\\
0.713426853707415	-1.30259251812186\\
0.710523412722236	-1.30661322645291\\
0.698314123476151	-1.32264529058116\\
0.697394789579159	-1.32379662789258\\
0.68448134056134	-1.33867735470942\\
0.681362725450902	-1.34209568133148\\
0.66874898794471	-1.35470941883768\\
0.665330661322646	-1.35797099976621\\
0.650511183447168	-1.37074148296593\\
0.649298597194389	-1.37174138065302\\
0.633266533066132	-1.38361953634077\\
0.62845300893667	-1.38677354709419\\
0.617234468937876	-1.39383666296571\\
0.601202404809619	-1.4025892937035\\
0.600745994797927	-1.40280561122244\\
0.585170340681363	-1.40992573720582\\
0.569138276553106	-1.4160452009309\\
0.560232538149745	-1.4188376753507\\
0.55310621242485	-1.42100045734942\\
0.537074148296593	-1.42486816501932\\
0.521042084168337	-1.42774204570758\\
0.50501002004008	-1.42967718233408\\
0.488977955911824	-1.43072393472941\\
0.472945891783567	-1.43092834074326\\
0.456913827655311	-1.43033247579527\\
0.440881763527054	-1.42897477560459\\
0.424849699398798	-1.42689032622384\\
0.408817635270541	-1.42411112497851\\
0.392785571142285	-1.42066631546198\\
0.385659245417389	-1.4188376753507\\
0.376753507014028	-1.41656648519904\\
0.360721442885771	-1.41183688755087\\
0.344689378757515	-1.40651358659104\\
0.334651973094188	-1.40280561122244\\
0.328657314629258	-1.40060301102082\\
0.312625250501002	-1.39411330148802\\
0.296593186372745	-1.38708617343689\\
0.295935787071416	-1.38677354709419\\
0.280561122244489	-1.37949587086625\\
0.264529058116232	-1.37139710129056\\
0.263316471863453	-1.37074148296593\\
0.248496993987976	-1.36276180547724\\
0.234356124466996	-1.35470941883768\\
0.232464929859719	-1.35363663963936\\
0.216432865731463	-1.3439988072933\\
0.208016305591172	-1.33867735470942\\
0.200400801603206	-1.33387843940809\\
0.18436873747495	-1.32328265171227\\
0.183449403577957	-1.32264529058116\\
0.168336673346693	-1.31219743040739\\
0.160593790422456	-1.30661322645291\\
0.152304609218437	-1.30065037543827\\
0.138863158302593	-1.29058116232465\\
0.13627254509018	-1.28864497519273\\
0.120240480961924	-1.27617679261093\\
0.118225064169628	-1.27454909819639\\
0.104208416833667	-1.26324929788337\\
0.098541886990586	-1.25851703406814\\
0.0881763527054105	-1.24987385108974\\
0.0796107483583451	-1.24248496993988\\
0.0721442885771539	-1.23605250728087\\
0.0613595760769714	-1.22645290581162\\
0.0561122244488974	-1.22178697497293\\
0.043725344827961	-1.21042084168337\\
0.0400801603206409	-1.20707861834823\\
0.0266527533644607	-1.19438877755511\\
0.0240480961923843	-1.19192845901123\\
0.0100931273197114	-1.17835671342685\\
0.00801603206412782	-1.17633717730267\\
-0.00599649593994475	-1.1623246492986\\
-0.00801603206412826	-1.16030511317441\\
-0.0216541346291227	-1.14629258517034\\
-0.0240480961923848	-1.14383226662646\\
-0.0369135090800337	-1.13026052104208\\
-0.0400801603206413	-1.12691829770695\\
-0.0518045890419846	-1.11422845691383\\
-0.0561122244488979	-1.10956252607513\\
-0.0663540590020255	-1.09819639278557\\
-0.0721442885771544	-1.09176393012657\\
-0.0805857154372647	-1.08216432865731\\
-0.0881763527054109	-1.07352114567892\\
-0.0945208072747585	-1.06613226452906\\
-0.104208416833667	-1.05483246421604\\
-0.108178328748496	-1.0501002004008\\
-0.120240480961924	-1.03569583068708\\
-0.121575272180958	-1.03406813627255\\
-0.134692032149139	-1.01803607214429\\
-0.13627254509018	-1.01609988501237\\
-0.147546506819146	-1.00200400801603\\
-0.152304609218437	-0.996041157001399\\
-0.160183002079824	-0.985971943887776\\
-0.168336673346694	-0.975524083714005\\
-0.172612977213376	-0.969939879759519\\
-0.18436873747495	-0.954545176762373\\
-0.184846659229807	-0.953907815631263\\
-0.196826185090445	-0.937875751503006\\
-0.200400801603207	-0.933076836201674\\
-0.208621100028441	-0.92184368737475\\
-0.216432865731463	-0.91113307583037\\
-0.220248467550491	-0.905811623246493\\
-0.231701607899196	-0.889779559118236\\
-0.232464929859719	-0.888706779919921\\
-0.242934002994659	-0.87374749498998\\
-0.248496993987976	-0.865767817501284\\
-0.254022025382232	-0.857715430861723\\
-0.264529058116232	-0.842338985058091\\
-0.264970265322122	-0.841683366733467\\
-0.275708137982974	-0.82565130260521\\
-0.280561122244489	-0.818373626377271\\
-0.286312909771483	-0.809619238476954\\
-0.296593186372745	-0.793899800691397\\
-0.296794755951398	-0.793587174348697\\
-0.307077408297698	-0.777555110220441\\
-0.312625250501002	-0.768862800486012\\
-0.317245772034545	-0.761523046092184\\
-0.327285519079521	-0.745490981963928\\
-0.328657314629258	-0.743288381762303\\
-0.337156300817286	-0.729458917835671\\
-0.344689378757515	-0.717134829076012\\
-0.346927181285696	-0.713426853707415\\
-0.356545602273577	-0.697394789579158\\
-0.360721442885771	-0.690394001779324\\
-0.366042486545644	-0.681362725450902\\
-0.375431709792827	-0.665330661322646\\
-0.376753507014028	-0.663059471170983\\
-0.384667400047787	-0.649298597194389\\
-0.392785571142285	-0.635095173177409\\
-0.393818986979759	-0.633266533066132\\
-0.402817842545187	-0.617234468937876\\
-0.408817635270541	-0.60647585443743\\
-0.411726576130805	-0.601202404809619\\
-0.42050868119499	-0.585170340681363\\
-0.424849699398798	-0.577190927426249\\
-0.429184778407611	-0.569138276553106\\
-0.437753779839535	-0.55310621242485\\
-0.440881763527054	-0.547211248550484\\
-0.446206654881929	-0.537074148296593\\
-0.454566045756848	-0.521042084168337\\
-0.456913827655311	-0.516504820484648\\
-0.462804351611308	-0.50501002004008\\
-0.470957473100776	-0.488977955911824\\
-0.472945891783567	-0.485036557176129\\
-0.478989140028764	-0.472945891783567\\
-0.486939183233613	-0.456913827655311\\
-0.488977955911824	-0.452768022905766\\
-0.494771454425069	-0.440881763527054\\
-0.502521462138104	-0.424849699398798\\
-0.50501002004008	-0.419657142253924\\
-0.510160926694366	-0.408817635270541\\
-0.517713795080866	-0.392785571142285\\
-0.521042084168337	-0.385657877370906\\
-0.525166418500279	-0.376753507014028\\
-0.532524898685601	-0.360721442885771\\
-0.537074148296593	-0.350719868426134\\
-0.539796051007148	-0.344689378757515\\
-0.546962750561728	-0.328657314629258\\
-0.55310621242485	-0.314788032499723\\
-0.554057232309416	-0.312625250501002\\
-0.561034616622361	-0.296593186372745\\
-0.56794688281841	-0.280561122244489\\
-0.569138276553106	-0.277768647824683\\
-0.574747076214539	-0.264529058116232\\
-0.581471212238471	-0.248496993987976\\
-0.585170340681363	-0.23958505584309\\
-0.588106045174488	-0.232464929859719\\
-0.594644196129463	-0.216432865731463\\
-0.601116114696608	-0.200400801603207\\
-0.601202404809619	-0.200184484084262\\
-0.607470969037547	-0.18436873747495\\
-0.613757593813339	-0.168336673346694\\
-0.617234468937876	-0.159367725089954\\
-0.619956043021835	-0.152304609218437\\
-0.626059204199459	-0.13627254509018\\
-0.632094429226009	-0.120240480961924\\
-0.633266533066132	-0.117086470208509\\
-0.638024714748419	-0.104208416833667\\
-0.643876682886971	-0.0881763527054109\\
-0.649298597194389	-0.0731441862642456\\
-0.649657305025131	-0.0721442885771544\\
-0.655327533694952	-0.0561122244488979\\
-0.660927542575827	-0.0400801603206413\\
-0.665330661322646	-0.0273096771209217\\
-0.666449447450091	-0.0240480961923848\\
-0.671867442661018	-0.00801603206412826\\
-0.677214141975015	0.00801603206412782\\
-0.681362725450902	0.0206297695703186\\
-0.682481511578348	0.0240480961923843\\
-0.687645617002839	0.0400801603206409\\
-0.692737070130543	0.0561122244488974\\
-0.697394789579158	0.0709929512657383\\
-0.6977534974099	0.0721442885771539\\
-0.702661469168184	0.0881763527054105\\
-0.707495148697989	0.104208416833667\\
-0.712254749867292	0.120240480961924\\
-0.713426853707415	0.124261189292968\\
-0.7169126464296	0.13627254509018\\
-0.72148542739605	0.152304609218437\\
-0.725982042711135	0.168336673346693\\
-0.729458917835671	0.180956552563858\\
-0.730395021803702	0.18436873747495\\
-0.734703172616582	0.200400801603206\\
-0.738932773283771	0.216432865731463\\
-0.743083645568415	0.232464929859719\\
-0.745490981963928	0.241959271006478\\
-0.747141849129499	0.248496993987976\\
-0.75109978162536	0.264529058116232\\
-0.754976124731029	0.280561122244489\\
-0.758770453206528	0.296593186372745\\
-0.761523046092184	0.308496301981619\\
-0.762474065976751	0.312625250501002\\
-0.766070269536166	0.328657314629258\\
-0.769581097463983	0.344689378757515\\
-0.773005860609449	0.360721442885771\\
-0.776343786249279	0.376753507014028\\
-0.777555110220441	0.382748165478957\\
-0.779575252649355	0.392785571142285\\
-0.782706016874727	0.408817635270541\\
-0.785745855233877	0.424849699398798\\
-0.788693703292883	0.440881763527054\\
-0.791548401670487	0.456913827655311\\
-0.793587174348697	0.468773486291922\\
-0.794301571550162	0.472945891783567\\
-0.796938650811452	0.488977955911824\\
-0.799477698304695	0.50501002004008\\
-0.801917229327064	0.521042084168337\\
-0.804255650319159	0.537074148296593\\
-0.806491254789434	0.55310621242485\\
-0.808622218986225	0.569138276553106\\
-0.809619238476954	0.577062115214904\\
-0.810635043483407	0.585170340681363\\
-0.812528179337218	0.601202404809619\\
-0.814310405770573	0.617234468937876\\
-0.815979507096244	0.633266533066132\\
-0.817533131510713	0.649298597194389\\
-0.818968785424053	0.665330661322646\\
-0.820283827444077	0.681362725450902\\
-0.821475461993016	0.697394789579159\\
-0.822540732533354	0.713426853707415\\
-0.82347651437768	0.729458917835672\\
-0.824279507055473	0.745490981963928\\
-0.824946226207662	0.761523046092185\\
-0.825472994977549	0.777555110220441\\
-0.82565130260521	0.785057435918085\\
-0.825852872183862	0.793587174348698\\
-0.826084201198703	0.809619238476954\\
-0.826165509867008	0.825651302605211\\
-0.8260925098111	0.841683366733467\\
-0.825860684610486	0.857715430861724\\
-0.82565130260521	0.866245169292337\\
}--cycle;


\addplot[area legend,solid,fill=mycolor10,draw=black,forget plot]
table[row sep=crcr] {%
x	y\\
-0.665330661322646	0.758105025403793\\
-0.665082784862698	0.761523046092185\\
-0.663686564675468	0.777555110220441\\
-0.662033485917416	0.793587174348698\\
-0.660112238187751	0.809619238476954\\
-0.657910790516671	0.825651302605211\\
-0.655416339751623	0.841683366733467\\
-0.652615254428646	0.857715430861724\\
-0.649493013666892	0.87374749498998\\
-0.649298597194389	0.87466276666894\\
-0.645911158536872	0.889779559118236\\
-0.641949453571151	0.905811623246493\\
-0.637595573522108	0.921843687374749\\
-0.633266533066132	0.936427568599341\\
-0.632809737264636	0.937875751503006\\
-0.62737690649475	0.953907815631262\\
-0.621452974746538	0.969939879759519\\
-0.617234468937876	0.980509918222073\\
-0.614898112565123	0.985971943887775\\
-0.607542947700778	1.00200400801603\\
-0.601202404809619	1.0147880599505\\
-0.599461200225299	1.01803607214429\\
-0.590292763112442	1.03406813627254\\
-0.585170340681363	1.04241611958894\\
-0.58002212340773	1.0501002004008\\
-0.569138276553106	1.06525453592202\\
-0.568443290727499	1.06613226452906\\
-0.55496007841122	1.08216432865731\\
-0.55310621242485	1.08423901886186\\
-0.539148838501137	1.09819639278557\\
-0.537074148296593	1.10015445046976\\
-0.521042084168337	1.11349529538232\\
-0.520031608379621	1.11422845691383\\
-0.50501002004008	1.1245759618867\\
-0.495273095705868	1.13026052104208\\
-0.488977955911824	1.13376186558903\\
-0.472945891783567	1.14121295074715\\
-0.459322825664768	1.14629258517034\\
-0.456913827655311	1.1471525085698\\
-0.440881763527054	1.15164454166159\\
-0.424849699398798	1.1548759669644\\
-0.408817635270541	1.15693170935783\\
-0.392785571142285	1.15788836523612\\
-0.376753507014028	1.15781506554153\\
-0.360721442885771	1.15677423201626\\
-0.344689378757515	1.15482224148008\\
-0.328657314629258	1.15201001063494\\
-0.312625250501002	1.14838351198749\\
-0.305073103182926	1.14629258517034\\
-0.296593186372745	1.14395622879759\\
-0.280561122244489	1.13876287781614\\
-0.264529058116232	1.13286477697221\\
-0.258233918322189	1.13026052104208\\
-0.248496993987976	1.12624895384855\\
-0.232464929859719	1.11895667271266\\
-0.222939668618048	1.11422845691383\\
-0.216432865731463	1.11101016607699\\
-0.200400801603207	1.10241489859423\\
-0.193078381798149	1.09819639278557\\
-0.18436873747495	1.09319414797708\\
-0.168336673346694	1.0833749248919\\
-0.166482807360323	1.08216432865731\\
-0.152304609218437	1.07292956533563\\
-0.142421230303577	1.06613226452906\\
-0.13627254509018	1.06191327279785\\
-0.120240480961924	1.05032770442359\\
-0.119941671271634	1.0501002004008\\
-0.104208416833667	1.03814294264968\\
-0.0990859944025878	1.03406813627254\\
-0.0881763527054109	1.02540306136772\\
-0.0792960980873046	1.01803607214429\\
-0.0721442885771544	1.0121106072761\\
-0.0604426536847802	1.00200400801603\\
-0.0561122244488979	0.998267697634059\\
-0.0424165166933938	0.985971943887775\\
-0.0400801603206413	0.983876022989052\\
-0.0251250206077705	0.969939879759519\\
-0.0240480961923848	0.968936849352776\\
-0.00848905848976323	0.953907815631262\\
-0.00801603206412826	0.953451019829765\\
0.007559236262631	0.937875751503006\\
0.00801603206412782	0.937418955701508\\
0.023078348608166	0.921843687374749\\
0.0240480961923843	0.920840656968006\\
0.038118883178968	0.905811623246493\\
0.0400801603206409	0.903715702347771\\
0.052724785791382	0.889779559118236\\
0.0561122244488974	0.886043248736265\\
0.0669343440570812	0.87374749498998\\
0.0721442885771539	0.867822030121793\\
0.0807810122405978	0.857715430861724\\
0.0881763527054105	0.849050355956901\\
0.0942940952626445	0.841683366733467\\
0.104208416833667	0.829726108982349\\
0.107499319045926	0.825651302605211\\
0.120240480961924	0.809846742499742\\
0.12041930855841	0.809619238476954\\
0.132975369684951	0.793587174348698\\
0.13627254509018	0.789368182617485\\
0.145279652829894	0.777555110220441\\
0.152304609218437	0.768320346898754\\
0.157354243807146	0.761523046092185\\
0.168336673346693	0.746701578198513\\
0.169213498649643	0.745490981963928\\
0.180773489041761	0.729458917835672\\
0.18436873747495	0.724456673027176\\
0.192122093660993	0.713426853707415\\
0.200400801603206	0.70161329538782\\
0.203294662397068	0.697394789579159\\
0.214244932658426	0.681362725450902\\
0.216432865731463	0.67814443461406\\
0.224965119394612	0.665330661322646\\
0.232464929859719	0.654026812993221\\
0.235540035818519	0.649298597194389\\
0.245914642067158	0.633266533066132\\
0.248496993987976	0.629254965872603\\
0.256086167319485	0.617234468937876\\
0.264529058116232	0.603806660739741\\
0.266136311158703	0.601202404809619\\
0.275967306441132	0.585170340681363\\
0.280561122244489	0.577640633327164\\
0.285655289054162	0.569138276553106\\
0.295210419473976	0.55310621242485\\
0.296593186372745	0.550769856052097\\
0.30455760395013	0.537074148296593\\
0.312625250501002	0.523133010985484\\
0.313815042172045	0.521042084168337\\
0.322866732980162	0.50501002004008\\
0.328657314629258	0.494695381376427\\
0.331814981981456	0.488977955911824\\
0.340604525579819	0.472945891783567\\
0.344689378757515	0.465443483965051\\
0.349260440368795	0.456913827655311\\
0.357791295522285	0.440881763527054\\
0.360721442885771	0.435331346244714\\
0.366170395731711	0.424849699398798\\
0.374445905957461	0.408817635270541\\
0.376753507014028	0.404308051513471\\
0.382562459543581	0.392785571142285\\
0.39058584797749	0.376753507014028\\
0.392785571142285	0.372317222951554\\
0.398452948292133	0.360721442885771\\
0.406227313185292	0.344689378757515\\
0.408817635270541	0.339296438816749\\
0.413856949942716	0.328657314629258\\
0.421385260700624	0.312625250501002\\
0.424849699398798	0.305176568166802\\
0.428788385313387	0.296593186372745\\
0.436073478999931	0.280561122244489\\
0.440881763527054	0.269881014607484\\
0.443260064715237	0.264529058116232\\
0.450304642951515	0.248496993987976\\
0.456913827655311	0.233324853259175\\
0.457283740180685	0.232464929859719\\
0.464090366375889	0.216432865731463\\
0.470835143423245	0.200400801603206\\
0.472945891783567	0.195321167180018\\
0.477441250432316	0.18436873747495\\
0.483947723367293	0.168336673346693\\
0.488977955911824	0.155805953765381\\
0.490366928106163	0.152304609218437\\
0.496637367872314	0.13627254509018\\
0.502841024354953	0.120240480961924\\
0.50501002004008	0.114555921806544\\
0.508912636015967	0.104208416833667\\
0.514879435468517	0.0881763527054105\\
0.520776905952092	0.0721442885771539\\
0.521042084168337	0.0714111270456462\\
0.526513104852846	0.0561122244488974\\
0.532172574628555	0.0400801603206409\\
0.537074148296593	0.0260061538765745\\
0.537748689204562	0.0240480961923843\\
0.54317196671062	0.00801603206412782\\
0.5485197049853	-0.00801603206412826\\
0.55310621242485	-0.0219734059878415\\
0.553780753332819	-0.0240480961923848\\
0.558890672250114	-0.0400801603206413\\
0.56392155566562	-0.0561122244488979\\
0.568873098336861	-0.0721442885771544\\
0.569138276553106	-0.0730220171841901\\
0.573668496296285	-0.0881763527054109\\
0.57837666810335	-0.104208416833667\\
0.583001344996236	-0.120240480961924\\
0.585170340681363	-0.127924561773787\\
0.587502535040489	-0.13627254509018\\
0.591881390411667	-0.152304609218437\\
0.596172172265089	-0.168336673346694\\
0.60037395363977	-0.18436873747495\\
0.601202404809619	-0.187616749668742\\
0.60442986328447	-0.200400801603207\\
0.608378943530198	-0.216432865731463\\
0.612233727171058	-0.232464929859719\\
0.615992884122811	-0.248496993987976\\
0.617234468937876	-0.25395901965368\\
0.619612770126059	-0.264529058116232\\
0.623111489402566	-0.280561122244489\\
0.626508521446038	-0.296593186372745\\
0.629802094367958	-0.312625250501002\\
0.632990291864623	-0.328657314629258\\
0.633266533066132	-0.330105497532926\\
0.6360197366617	-0.344689378757515\\
0.638933910215981	-0.360721442885771\\
0.641735488262777	-0.376753507014028\\
0.644422048509105	-0.392785571142285\\
0.646990996137822	-0.408817635270541\\
0.649298597194389	-0.423934427719837\\
0.649436850751654	-0.424849699398798\\
0.651716318887212	-0.440881763527054\\
0.653869658805669	-0.456913827655311\\
0.655893692198952	-0.472945891783567\\
0.657785033616345	-0.488977955911824\\
0.659540079673549	-0.50501002004008\\
0.661154997500123	-0.521042084168337\\
0.662625712366185	-0.537074148296593\\
0.663947894423875	-0.55310621242485\\
0.665116944493114	-0.569138276553106\\
0.665330661322646	-0.572556297241498\\
0.666110017040159	-0.585170340681363\\
0.666937914365116	-0.601202404809619\\
0.667600197623856	-0.617234468937876\\
0.668091410176274	-0.633266533066132\\
0.668405767281445	-0.649298597194389\\
0.668537136825009	-0.665330661322646\\
0.66847901861384	-0.681362725450902\\
0.668224522116507	-0.697394789579158\\
0.6677663425161	-0.713426853707415\\
0.66709673492865	-0.729458917835671\\
0.666207486625594	-0.745490981963928\\
0.665330661322646	-0.758105025403793\\
0.665082784862698	-0.761523046092184\\
0.663686564675469	-0.777555110220441\\
0.662033485917416	-0.793587174348697\\
0.660112238187751	-0.809619238476954\\
0.657910790516671	-0.82565130260521\\
0.655416339751623	-0.841683366733467\\
0.652615254428647	-0.857715430861723\\
0.649493013666891	-0.87374749498998\\
0.649298597194389	-0.87466276666894\\
0.645911158536872	-0.889779559118236\\
0.641949453571151	-0.905811623246493\\
0.637595573522108	-0.92184368737475\\
0.633266533066132	-0.936427568599342\\
0.632809737264635	-0.937875751503006\\
0.627376906494749	-0.953907815631263\\
0.621452974746537	-0.969939879759519\\
0.617234468937876	-0.980509918222073\\
0.614898112565123	-0.985971943887776\\
0.607542947700777	-1.00200400801603\\
0.601202404809619	-1.0147880599505\\
0.599461200225299	-1.01803607214429\\
0.590292763112442	-1.03406813627255\\
0.585170340681363	-1.04241611958894\\
0.580022123407729	-1.0501002004008\\
0.569138276553106	-1.06525453592202\\
0.5684432907275	-1.06613226452906\\
0.55496007841122	-1.08216432865731\\
0.55310621242485	-1.08423901886186\\
0.539148838501136	-1.09819639278557\\
0.537074148296593	-1.10015445046976\\
0.521042084168337	-1.11349529538232\\
0.520031608379621	-1.11422845691383\\
0.50501002004008	-1.12457596188671\\
0.495273095705867	-1.13026052104208\\
0.488977955911824	-1.13376186558903\\
0.472945891783567	-1.14121295074715\\
0.459322825664769	-1.14629258517034\\
0.456913827655311	-1.1471525085698\\
0.440881763527054	-1.15164454166159\\
0.424849699398798	-1.1548759669644\\
0.408817635270541	-1.15693170935783\\
0.392785571142285	-1.15788836523612\\
0.376753507014028	-1.15781506554153\\
0.360721442885771	-1.15677423201626\\
0.344689378757515	-1.15482224148008\\
0.328657314629258	-1.15201001063494\\
0.312625250501002	-1.14838351198749\\
0.305073103182926	-1.14629258517034\\
0.296593186372745	-1.14395622879759\\
0.280561122244489	-1.13876287781614\\
0.264529058116232	-1.13286477697221\\
0.258233918322189	-1.13026052104208\\
0.248496993987976	-1.12624895384855\\
0.232464929859719	-1.11895667271266\\
0.222939668618048	-1.11422845691383\\
0.216432865731463	-1.11101016607699\\
0.200400801603206	-1.10241489859423\\
0.19307838179815	-1.09819639278557\\
0.18436873747495	-1.09319414797708\\
0.168336673346693	-1.0833749248919\\
0.166482807360324	-1.08216432865731\\
0.152304609218437	-1.07292956533563\\
0.142421230303577	-1.06613226452906\\
0.13627254509018	-1.06191327279785\\
0.120240480961924	-1.05032770442359\\
0.119941671271634	-1.0501002004008\\
0.104208416833667	-1.03814294264968\\
0.0990859944025882	-1.03406813627255\\
0.0881763527054105	-1.02540306136772\\
0.079296098087305	-1.01803607214429\\
0.0721442885771539	-1.0121106072761\\
0.0604426536847797	-1.00200400801603\\
0.0561122244488974	-0.99826769763406\\
0.0424165166933943	-0.985971943887776\\
0.0400801603206409	-0.983876022989052\\
0.0251250206077718	-0.969939879759519\\
0.0240480961923843	-0.968936849352774\\
0.00848905848976415	-0.953907815631263\\
0.00801603206412782	-0.953451019829765\\
-0.00755923626263144	-0.937875751503006\\
-0.00801603206412826	-0.937418955701509\\
-0.0230783486081656	-0.92184368737475\\
-0.0240480961923848	-0.920840656968006\\
-0.0381188831789675	-0.905811623246493\\
-0.0400801603206413	-0.90371570234777\\
-0.0527247857913811	-0.889779559118236\\
-0.0561122244488979	-0.886043248736264\\
-0.0669343440570817	-0.87374749498998\\
-0.0721442885771544	-0.867822030121793\\
-0.0807810122405983	-0.857715430861723\\
-0.0881763527054109	-0.8490503559569\\
-0.0942940952626444	-0.841683366733467\\
-0.104208416833667	-0.829726108982349\\
-0.107499319045926	-0.82565130260521\\
-0.120240480961924	-0.809846742499741\\
-0.120419308558411	-0.809619238476954\\
-0.132975369684952	-0.793587174348697\\
-0.13627254509018	-0.789368182617485\\
-0.145279652829894	-0.777555110220441\\
-0.152304609218437	-0.768320346898754\\
-0.157354243807146	-0.761523046092184\\
-0.168336673346694	-0.746701578198513\\
-0.169213498649643	-0.745490981963928\\
-0.180773489041761	-0.729458917835671\\
-0.18436873747495	-0.724456673027176\\
-0.192122093660993	-0.713426853707415\\
-0.200400801603207	-0.70161329538782\\
-0.203294662397068	-0.697394789579158\\
-0.214244932658426	-0.681362725450902\\
-0.216432865731463	-0.678144434614059\\
-0.224965119394612	-0.665330661322646\\
-0.232464929859719	-0.654026812993221\\
-0.235540035818519	-0.649298597194389\\
-0.245914642067158	-0.633266533066132\\
-0.248496993987976	-0.629254965872603\\
-0.256086167319485	-0.617234468937876\\
-0.264529058116232	-0.603806660739742\\
-0.266136311158703	-0.601202404809619\\
-0.275967306441132	-0.585170340681363\\
-0.280561122244489	-0.577640633327163\\
-0.285655289054161	-0.569138276553106\\
-0.295210419473976	-0.55310621242485\\
-0.296593186372745	-0.550769856052097\\
-0.30455760395013	-0.537074148296593\\
-0.312625250501002	-0.523133010985485\\
-0.313815042172045	-0.521042084168337\\
-0.322866732980163	-0.50501002004008\\
-0.328657314629258	-0.494695381376427\\
-0.331814981981456	-0.488977955911824\\
-0.34060452557982	-0.472945891783567\\
-0.344689378757515	-0.465443483965052\\
-0.349260440368795	-0.456913827655311\\
-0.357791295522285	-0.440881763527054\\
-0.360721442885771	-0.435331346244714\\
-0.366170395731711	-0.424849699398798\\
-0.374445905957461	-0.408817635270541\\
-0.376753507014028	-0.404308051513471\\
-0.382562459543581	-0.392785571142285\\
-0.390585847977489	-0.376753507014028\\
-0.392785571142285	-0.372317222951554\\
-0.398452948292133	-0.360721442885771\\
-0.406227313185292	-0.344689378757515\\
-0.408817635270541	-0.339296438816749\\
-0.413856949942716	-0.328657314629258\\
-0.421385260700624	-0.312625250501002\\
-0.424849699398798	-0.305176568166802\\
-0.428788385313387	-0.296593186372745\\
-0.436073478999931	-0.280561122244489\\
-0.440881763527054	-0.269881014607484\\
-0.443260064715237	-0.264529058116232\\
-0.450304642951515	-0.248496993987976\\
-0.456913827655311	-0.233324853259175\\
-0.457283740180684	-0.232464929859719\\
-0.46409036637589	-0.216432865731463\\
-0.470835143423245	-0.200400801603207\\
-0.472945891783567	-0.195321167180017\\
-0.477441250432316	-0.18436873747495\\
-0.483947723367294	-0.168336673346694\\
-0.488977955911824	-0.155805953765381\\
-0.490366928106163	-0.152304609218437\\
-0.496637367872314	-0.13627254509018\\
-0.502841024354953	-0.120240480961924\\
-0.50501002004008	-0.114555921806545\\
-0.508912636015967	-0.104208416833667\\
-0.514879435468517	-0.0881763527054109\\
-0.520776905952091	-0.0721442885771544\\
-0.521042084168337	-0.0714111270456461\\
-0.526513104852846	-0.0561122244488979\\
-0.532172574628555	-0.0400801603206413\\
-0.537074148296593	-0.026006153876575\\
-0.537748689204563	-0.0240480961923848\\
-0.54317196671062	-0.00801603206412826\\
-0.548519704985299	0.00801603206412782\\
-0.55310621242485	0.0219734059878419\\
-0.553780753332819	0.0240480961923843\\
-0.558890672250114	0.0400801603206409\\
-0.563921555665619	0.0561122244488974\\
-0.56887309833686	0.0721442885771539\\
-0.569138276553106	0.073022017184192\\
-0.573668496296284	0.0881763527054105\\
-0.57837666810335	0.104208416833667\\
-0.583001344996236	0.120240480961924\\
-0.585170340681363	0.127924561773787\\
-0.58750253504049	0.13627254509018\\
-0.591881390411667	0.152304609218437\\
-0.596172172265089	0.168336673346693\\
-0.60037395363977	0.18436873747495\\
-0.601202404809619	0.187616749668742\\
-0.60442986328447	0.200400801603206\\
-0.608378943530199	0.216432865731463\\
-0.612233727171058	0.232464929859719\\
-0.615992884122811	0.248496993987976\\
-0.617234468937876	0.25395901965368\\
-0.619612770126059	0.264529058116232\\
-0.623111489402566	0.280561122244489\\
-0.626508521446038	0.296593186372745\\
-0.629802094367959	0.312625250501002\\
-0.632990291864623	0.328657314629258\\
-0.633266533066132	0.330105497532926\\
-0.6360197366617	0.344689378757515\\
-0.638933910215981	0.360721442885771\\
-0.641735488262777	0.376753507014028\\
-0.644422048509105	0.392785571142285\\
-0.646990996137822	0.408817635270541\\
-0.649298597194389	0.423934427719838\\
-0.649436850751654	0.424849699398798\\
-0.651716318887212	0.440881763527054\\
-0.653869658805669	0.456913827655311\\
-0.655893692198952	0.472945891783567\\
-0.657785033616345	0.488977955911824\\
-0.659540079673549	0.50501002004008\\
-0.661154997500123	0.521042084168337\\
-0.662625712366185	0.537074148296593\\
-0.663947894423875	0.55310621242485\\
-0.665116944493114	0.569138276553106\\
-0.665330661322646	0.572556297241498\\
-0.666110017040159	0.585170340681363\\
-0.666937914365116	0.601202404809619\\
-0.667600197623856	0.617234468937876\\
-0.668091410176274	0.633266533066132\\
-0.668405767281445	0.649298597194389\\
-0.668537136825008	0.665330661322646\\
-0.668479018613839	0.681362725450902\\
-0.668224522116507	0.697394789579159\\
-0.6677663425161	0.713426853707415\\
-0.66709673492865	0.729458917835672\\
-0.666207486625595	0.745490981963928\\
-0.665330661322646	0.758105025403793\\
}--cycle;


\addplot[area legend,solid,fill=mycolor11,draw=black,forget plot]
table[row sep=crcr] {%
x	y\\
-0.472945891783567	0.631558403398626\\
-0.472605099508047	0.633266533066132\\
-0.468985504333452	0.649298597194389\\
-0.464849004208713	0.665330661322646\\
-0.460159075615718	0.681362725450902\\
-0.456913827655311	0.691303111073879\\
-0.454736499135691	0.697394789579159\\
-0.448395418577774	0.713426853707415\\
-0.441293873286597	0.729458917835672\\
-0.440881763527054	0.730315450863864\\
-0.432752526676281	0.745490981963928\\
-0.424849699398798	0.758778909934585\\
-0.423006465444712	0.761523046092185\\
-0.411260209628666	0.777555110220441\\
-0.408817635270541	0.780620139334836\\
-0.396869159056258	0.793587174348698\\
-0.392785571142285	0.797670762262671\\
-0.378610477861005	0.809619238476954\\
-0.376753507014028	0.811071611595405\\
-0.360721442885771	0.821430877328627\\
-0.352418655016157	0.825651302605211\\
-0.344689378757515	0.829329654544471\\
-0.328657314629258	0.835068349076468\\
-0.312625250501002	0.838968684198747\\
-0.296593186372745	0.841199718536656\\
-0.285769468609448	0.841683366733467\\
-0.280561122244489	0.841903918836987\\
-0.275352775879529	0.841683366733467\\
-0.264529058116232	0.841227025015592\\
-0.248496993987976	0.83926680531526\\
-0.232464929859719	0.836128895371847\\
-0.216432865731463	0.831901692994961\\
-0.200400801603207	0.82666322786356\\
-0.197825995821692	0.825651302605211\\
-0.18436873747495	0.820378748496846\\
-0.168336673346694	0.813184089207794\\
-0.161302958637682	0.809619238476954\\
-0.152304609218437	0.805070371891943\\
-0.13627254509018	0.796098946117739\\
-0.132188957176207	0.793587174348698\\
-0.120240480961924	0.786252889299523\\
-0.107126535443859	0.777555110220441\\
-0.104208416833667	0.77562315720775\\
-0.0881763527054109	0.764164839203355\\
-0.084723917528645	0.761523046092185\\
-0.0721442885771544	0.751909514130287\\
-0.0642414612996724	0.745490981963928\\
-0.0561122244488979	0.738895348584444\\
-0.045128831381008	0.729458917835672\\
-0.0400801603206413	0.725124673100652\\
-0.0271702987996977	0.713426853707415\\
-0.0240480961923848	0.710599232962271\\
-0.0101933605837475	0.697394789579159\\
-0.00801603206412826	0.695320190446341\\
0.00594143293131114	0.681362725450902\\
0.00801603206412782	0.679288126318085\\
0.0213479068822207	0.665330661322646\\
0.0240480961923843	0.662503040577503\\
0.0361197728705266	0.649298597194389\\
0.0400801603206409	0.64496435245937\\
0.0503347757565049	0.633266533066132\\
0.0561122244488974	0.626670899686649\\
0.0640578544475958	0.617234468937876\\
0.0721442885771539	0.607620936975979\\
0.0773435796145164	0.601202404809619\\
0.0881763527054105	0.587812133792533\\
0.0902380540757649	0.585170340681363\\
0.102710955361553	0.569138276553106\\
0.104208416833667	0.567206323540415\\
0.114761222290392	0.55310621242485\\
0.120240480961924	0.545771927375676\\
0.126524458918498	0.537074148296593\\
0.13627254509018	0.523553855937378\\
0.138026478403903	0.521042084168337\\
0.149159038093379	0.50501002004008\\
0.152304609218437	0.50046115345507\\
0.160001583014829	0.488977955911824\\
0.168336673346693	0.476510742514408\\
0.170649960884252	0.472945891783567\\
0.180987089305425	0.456913827655311\\
0.18436873747495	0.451641273546946\\
0.191071979166073	0.440881763527054\\
0.200400801603206	0.425861624657147\\
0.201012065831786	0.424849699398798\\
0.210603949389608	0.408817635270541\\
0.216432865731463	0.399035961532035\\
0.220058021957355	0.392785571142285\\
0.229290611656081	0.376753507014028\\
0.232464929859719	0.371199035652408\\
0.238297033847969	0.360721442885771\\
0.247173167878561	0.344689378757515\\
0.248496993987976	0.342272817339308\\
0.255766886224913	0.328657314629258\\
0.264289549707929	0.312625250501002\\
0.264529058116232	0.312168908783128\\
0.272502383661635	0.296593186372745\\
0.280561122244489	0.28078167434801\\
0.280670880448816	0.280561122244489\\
0.288535564991394	0.264529058116232\\
0.296360600228989	0.248496993987976\\
0.296593186372745	0.248013345791166\\
0.303895898217337	0.232464929859719\\
0.311376856476042	0.216432865731463\\
0.312625250501002	0.213718183196744\\
0.318610457811489	0.200400801603206\\
0.325750495336266	0.18436873747495\\
0.328657314629258	0.177753719817951\\
0.332704086079932	0.168336673346693\\
0.339506322152324	0.152304609218437\\
0.344689378757515	0.139950897029441\\
0.346199540087413	0.13627254509018\\
0.352667032756419	0.120240480961924\\
0.359067068258363	0.104208416833667\\
0.360721442885771	0.0999879915570851\\
0.36525335024532	0.0881763527054105\\
0.371314638075837	0.0721442885771539\\
0.376753507014028	0.0575645975673485\\
0.377284149442735	0.0561122244488974\\
0.383009686993799	0.0400801603206409\\
0.388652960980946	0.0240480961923843\\
0.392785571142285	0.0120996199781027\\
0.394169229744255	0.00801603206412782\\
0.399472542357823	-0.00801603206412826\\
0.404685025109202	-0.0240480961923848\\
0.408817635270541	-0.0370151312062448\\
0.409774840597944	-0.0400801603206413\\
0.414642913275298	-0.0561122244488979\\
0.419410830460606	-0.0721442885771544\\
0.424076362139965	-0.0881763527054109\\
0.424849699398798	-0.0909204888630107\\
0.428518805443235	-0.104208416833667\\
0.432827353397701	-0.120240480961924\\
0.437022813017709	-0.13627254509018\\
0.440881763527054	-0.151448076190245\\
0.441095225623083	-0.152304609218437\\
0.444928534977728	-0.168336673346694\\
0.448637216736993	-0.18436873747495\\
0.452217531449852	-0.200400801603207\\
0.45566543363035	-0.216432865731463\\
0.456913827655311	-0.222524544236743\\
0.458909321024828	-0.232464929859719\\
0.46197191826691	-0.248496993987976\\
0.464888270402216	-0.264529058116232\\
0.467653230225484	-0.280561122244489\\
0.470261256084326	-0.296593186372745\\
0.472706383375263	-0.312625250501002\\
0.472945891783567	-0.314333380168509\\
0.474911396503024	-0.328657314629258\\
0.476934843536039	-0.344689378757515\\
0.478777995771817	-0.360721442885771\\
0.48043346647087	-0.376753507014028\\
0.481893321940788	-0.392785571142285\\
0.483149039569969	-0.408817635270541\\
0.484191461832447	-0.424849699398798\\
0.485010745812909	-0.440881763527054\\
0.485596307742298	-0.456913827655311\\
0.48593676196695	-0.472945891783567\\
0.486019853696573	-0.488977955911824\\
0.485832384786766	-0.50501002004008\\
0.485360131708188	-0.521042084168337\\
0.484587754734332	-0.537074148296593\\
0.483498697240291	-0.55310621242485\\
0.482075073842231	-0.569138276553106\\
0.480297545917277	-0.585170340681363\\
0.47814518282093	-0.601202404809619\\
0.475595306857631	-0.617234468937876\\
0.472945891783567	-0.631558403398626\\
0.472605099508047	-0.633266533066132\\
0.468985504333452	-0.649298597194389\\
0.464849004208713	-0.665330661322646\\
0.46015907561572	-0.681362725450902\\
0.456913827655311	-0.69130311107388\\
0.454736499135691	-0.697394789579158\\
0.448395418577773	-0.713426853707415\\
0.441293873286598	-0.729458917835671\\
0.440881763527054	-0.730315450863864\\
0.432752526676281	-0.745490981963928\\
0.424849699398798	-0.758778909934585\\
0.423006465444711	-0.761523046092184\\
0.411260209628669	-0.777555110220441\\
0.408817635270541	-0.780620139334837\\
0.396869159056258	-0.793587174348697\\
0.392785571142285	-0.797670762262671\\
0.378610477861005	-0.809619238476954\\
0.376753507014028	-0.811071611595405\\
0.360721442885771	-0.821430877328629\\
0.352418655016159	-0.82565130260521\\
0.344689378757515	-0.829329654544471\\
0.328657314629258	-0.835068349076468\\
0.312625250501002	-0.838968684198748\\
0.296593186372745	-0.841199718536656\\
0.285769468609449	-0.841683366733467\\
0.280561122244489	-0.841903918836987\\
0.275352775879529	-0.841683366733467\\
0.264529058116232	-0.841227025015592\\
0.248496993987976	-0.839266805315259\\
0.232464929859719	-0.836128895371846\\
0.216432865731463	-0.83190169299496\\
0.200400801603206	-0.826663227863559\\
0.197825995821691	-0.82565130260521\\
0.18436873747495	-0.820378748496845\\
0.168336673346693	-0.813184089207795\\
0.161302958637679	-0.809619238476954\\
0.152304609218437	-0.805070371891943\\
0.13627254509018	-0.796098946117737\\
0.132188957176207	-0.793587174348697\\
0.120240480961924	-0.786252889299523\\
0.107126535443859	-0.777555110220441\\
0.104208416833667	-0.77562315720775\\
0.0881763527054105	-0.764164839203354\\
0.0847239175286446	-0.761523046092184\\
0.0721442885771539	-0.751909514130287\\
0.064241461299671	-0.745490981963928\\
0.0561122244488974	-0.738895348584444\\
0.0451288313810076	-0.729458917835671\\
0.0400801603206409	-0.725124673100651\\
0.0271702987996958	-0.713426853707415\\
0.0240480961923843	-0.710599232962272\\
0.010193360583747	-0.697394789579158\\
0.00801603206412782	-0.69532019044634\\
-0.00594143293131158	-0.681362725450902\\
-0.00801603206412826	-0.679288126318085\\
-0.0213479068822197	-0.665330661322646\\
-0.0240480961923848	-0.662503040577502\\
-0.0361197728705266	-0.649298597194389\\
-0.0400801603206413	-0.644964352459369\\
-0.050334775756504	-0.633266533066132\\
-0.0561122244488979	-0.626670899686648\\
-0.0640578544475958	-0.617234468937876\\
-0.0721442885771544	-0.607620936975978\\
-0.0773435796145168	-0.601202404809619\\
-0.0881763527054109	-0.587812133792533\\
-0.0902380540757654	-0.585170340681363\\
-0.102710955361554	-0.569138276553106\\
-0.104208416833667	-0.567206323540416\\
-0.114761222290391	-0.55310621242485\\
-0.120240480961924	-0.545771927375675\\
-0.126524458918498	-0.537074148296593\\
-0.13627254509018	-0.523553855937377\\
-0.138026478403903	-0.521042084168337\\
-0.149159038093379	-0.50501002004008\\
-0.152304609218437	-0.500461153455069\\
-0.160001583014829	-0.488977955911824\\
-0.168336673346694	-0.476510742514407\\
-0.170649960884251	-0.472945891783567\\
-0.180987089305424	-0.456913827655311\\
-0.18436873747495	-0.451641273546945\\
-0.191071979166074	-0.440881763527054\\
-0.200400801603207	-0.425861624657147\\
-0.201012065831786	-0.424849699398798\\
-0.210603949389609	-0.408817635270541\\
-0.216432865731463	-0.399035961532034\\
-0.220058021957355	-0.392785571142285\\
-0.229290611656081	-0.376753507014028\\
-0.232464929859719	-0.371199035652408\\
-0.238297033847968	-0.360721442885771\\
-0.247173167878561	-0.344689378757515\\
-0.248496993987976	-0.342272817339309\\
-0.255766886224913	-0.328657314629258\\
-0.264289549707929	-0.312625250501002\\
-0.264529058116232	-0.312168908783128\\
-0.272502383661634	-0.296593186372745\\
-0.280561122244489	-0.28078167434801\\
-0.280670880448816	-0.280561122244489\\
-0.288535564991394	-0.264529058116232\\
-0.296360600228989	-0.248496993987976\\
-0.296593186372745	-0.248013345791166\\
-0.303895898217337	-0.232464929859719\\
-0.311376856476042	-0.216432865731463\\
-0.312625250501002	-0.213718183196745\\
-0.318610457811488	-0.200400801603207\\
-0.325750495336266	-0.18436873747495\\
-0.328657314629258	-0.177753719817951\\
-0.332704086079932	-0.168336673346694\\
-0.339506322152324	-0.152304609218437\\
-0.344689378757515	-0.139950897029441\\
-0.346199540087413	-0.13627254509018\\
-0.352667032756418	-0.120240480961924\\
-0.359067068258364	-0.104208416833667\\
-0.360721442885771	-0.0999879915570855\\
-0.365253350245319	-0.0881763527054109\\
-0.371314638075837	-0.0721442885771544\\
-0.376753507014028	-0.0575645975673489\\
-0.377284149442735	-0.0561122244488979\\
-0.383009686993798	-0.0400801603206413\\
-0.388652960980946	-0.0240480961923848\\
-0.392785571142285	-0.0120996199781018\\
-0.394169229744254	-0.00801603206412826\\
-0.399472542357823	0.00801603206412782\\
-0.404685025109202	0.0240480961923843\\
-0.408817635270541	0.0370151312062466\\
-0.409774840597943	0.0400801603206409\\
-0.414642913275298	0.0561122244488974\\
-0.419410830460605	0.0721442885771539\\
-0.424076362139966	0.0881763527054105\\
-0.424849699398798	0.0909204888630098\\
-0.428518805443234	0.104208416833667\\
-0.432827353397701	0.120240480961924\\
-0.437022813017709	0.13627254509018\\
-0.440881763527054	0.151448076190245\\
-0.441095225623083	0.152304609218437\\
-0.444928534977728	0.168336673346693\\
-0.448637216736993	0.18436873747495\\
-0.452217531449852	0.200400801603206\\
-0.455665433630351	0.216432865731463\\
-0.456913827655311	0.222524544236743\\
-0.458909321024828	0.232464929859719\\
-0.461971918266909	0.248496993987976\\
-0.464888270402216	0.264529058116232\\
-0.467653230225484	0.280561122244489\\
-0.470261256084326	0.296593186372745\\
-0.472706383375263	0.312625250501002\\
-0.472945891783567	0.314333380168508\\
-0.474911396503024	0.328657314629258\\
-0.476934843536038	0.344689378757515\\
-0.478777995771816	0.360721442885771\\
-0.48043346647087	0.376753507014028\\
-0.481893321940788	0.392785571142285\\
-0.483149039569969	0.408817635270541\\
-0.484191461832447	0.424849699398798\\
-0.485010745812909	0.440881763527054\\
-0.485596307742298	0.456913827655311\\
-0.48593676196695	0.472945891783567\\
-0.486019853696573	0.488977955911824\\
-0.485832384786766	0.50501002004008\\
-0.485360131708188	0.521042084168337\\
-0.484587754734332	0.537074148296593\\
-0.483498697240291	0.55310621242485\\
-0.482075073842231	0.569138276553106\\
-0.480297545917278	0.585170340681363\\
-0.47814518282093	0.601202404809619\\
-0.475595306857631	0.617234468937876\\
-0.472945891783567	0.631558403398626\\
}--cycle;


\addplot[area legend,solid,fill=mycolor12,draw=black,forget plot]
table[row sep=crcr] {%
x	y\\
-0.200400801603207	0.304862370138295\\
-0.197344662109513	0.312625250501002\\
-0.189390682595018	0.328657314629258\\
-0.18436873747495	0.336847857012185\\
-0.177891791257849	0.344689378757515\\
-0.168336673346694	0.35424449666867\\
-0.158402802348644	0.360721442885771\\
-0.152304609218437	0.364110634083375\\
-0.13627254509018	0.368853940670599\\
-0.120240480961924	0.369881171754941\\
-0.104208416833667	0.367869101570326\\
-0.0881763527054109	0.363336633675013\\
-0.082078159575204	0.360721442885771\\
-0.0721442885771544	0.356466805104686\\
-0.0561122244488979	0.34759700820266\\
-0.0518105910934108	0.344689378757515\\
-0.0400801603206413	0.336766468226409\\
-0.0297366567970107	0.328657314629258\\
-0.0240480961923848	0.324199859313259\\
-0.0110721715578217	0.312625250501002\\
-0.00801603206412826	0.309899854870135\\
0.00529063643326164	0.296593186372745\\
0.00801603206412782	0.293867790741879\\
0.0200260317046286	0.280561122244489\\
0.0240480961923843	0.276103666928489\\
0.0335674421125732	0.264529058116232\\
0.0400801603206409	0.256606147585127\\
0.0462057588758648	0.248496993987976\\
0.0561122244488974	0.235372559304864\\
0.0581431221874816	0.232464929859719\\
0.0692259403854356	0.216432865731463\\
0.0721442885771539	0.212178227950381\\
0.0796438295284159	0.200400801603206\\
0.0881763527054105	0.186983928264192\\
0.0897289331740241	0.18436873747495\\
0.099068780199145	0.168336673346693\\
0.104208416833667	0.159452267902991\\
0.108078661342575	0.152304609218437\\
0.116632841383845	0.13627254509018\\
0.120240480961924	0.129400209831093\\
0.124752283040789	0.120240480961924\\
0.132520874517804	0.104208416833667\\
0.13627254509018	0.0963088504902381\\
0.139905087192154	0.0881763527054105\\
0.146892501509108	0.0721442885771539\\
0.152304609218437	0.0595014156465018\\
0.153672158159379	0.0561122244488974\\
0.159885987743175	0.0400801603206409\\
0.165975112210303	0.0240480961923843\\
0.168336673346693	0.0175711499752883\\
0.171621426422992	0.00801603206412782\\
0.176904626791464	-0.00801603206412826\\
0.182007176338558	-0.0240480961923848\\
0.18436873747495	-0.0318896179377154\\
0.186695011676916	-0.0400801603206413\\
0.190963287470592	-0.0561122244488979\\
0.194988693893878	-0.0721442885771544\\
0.198748967636206	-0.0881763527054109\\
0.200400801603206	-0.0959392330681169\\
0.202057402535111	-0.104208416833667\\
0.204912603682072	-0.120240480961924\\
0.207426878711987	-0.13627254509018\\
0.209566606502076	-0.152304609218437\\
0.211293229096941	-0.168336673346694\\
0.212562354681171	-0.18436873747495\\
0.213322657300812	-0.200400801603207\\
0.213514517539743	-0.216432865731463\\
0.213068330138315	-0.232464929859719\\
0.211902379345737	-0.248496993987976\\
0.209920147523395	-0.264529058116232\\
0.207006872475959	-0.280561122244489\\
0.203025096977945	-0.296593186372745\\
0.200400801603206	-0.304862370138296\\
0.197344662109512	-0.312625250501002\\
0.189390682595018	-0.328657314629258\\
0.18436873747495	-0.336847857012184\\
0.17789179125785	-0.344689378757515\\
0.168336673346693	-0.354244496668672\\
0.158402802348643	-0.360721442885771\\
0.152304609218437	-0.364110634083375\\
0.13627254509018	-0.3688539406706\\
0.120240480961924	-0.369881171754941\\
0.104208416833667	-0.367869101570326\\
0.0881763527054105	-0.363336633675014\\
0.0820781595752036	-0.360721442885771\\
0.0721442885771539	-0.356466805104686\\
0.0561122244488974	-0.347597008202659\\
0.0518105910934104	-0.344689378757515\\
0.0400801603206409	-0.336766468226409\\
0.0297366567970103	-0.328657314629258\\
0.0240480961923843	-0.324199859313259\\
0.0110721715578212	-0.312625250501002\\
0.00801603206412782	-0.309899854870136\\
-0.00529063643326208	-0.296593186372745\\
-0.00801603206412826	-0.293867790741879\\
-0.020026031704629	-0.280561122244489\\
-0.0240480961923848	-0.276103666928489\\
-0.0335674421125736	-0.264529058116232\\
-0.0400801603206413	-0.256606147585127\\
-0.0462057588758648	-0.248496993987976\\
-0.0561122244488979	-0.235372559304863\\
-0.058143122187482	-0.232464929859719\\
-0.0692259403854342	-0.216432865731463\\
-0.0721442885771544	-0.212178227950378\\
-0.0796438295284154	-0.200400801603207\\
-0.0881763527054109	-0.186983928264193\\
-0.0897289331740245	-0.18436873747495\\
-0.0990687801991455	-0.168336673346694\\
-0.104208416833667	-0.159452267902992\\
-0.108078661342576	-0.152304609218437\\
-0.116632841383844	-0.13627254509018\\
-0.120240480961924	-0.129400209831092\\
-0.124752283040789	-0.120240480961924\\
-0.132520874517804	-0.104208416833667\\
-0.13627254509018	-0.096308850490239\\
-0.139905087192154	-0.0881763527054109\\
-0.146892501509108	-0.0721442885771544\\
-0.152304609218437	-0.0595014156465017\\
-0.15367215815938	-0.0561122244488979\\
-0.159885987743175	-0.0400801603206413\\
-0.165975112210302	-0.0240480961923848\\
-0.168336673346694	-0.0175711499752838\\
-0.171621426422991	-0.00801603206412826\\
-0.176904626791464	0.00801603206412782\\
-0.182007176338559	0.0240480961923843\\
-0.18436873747495	0.031889617937715\\
-0.186695011676916	0.0400801603206409\\
-0.190963287470593	0.0561122244488974\\
-0.194988693893878	0.0721442885771539\\
-0.198748967636206	0.0881763527054105\\
-0.200400801603207	0.0959392330681169\\
-0.202057402535111	0.104208416833667\\
-0.204912603682072	0.120240480961924\\
-0.207426878711987	0.13627254509018\\
-0.209566606502074	0.152304609218437\\
-0.211293229096941	0.168336673346693\\
-0.212562354681169	0.18436873747495\\
-0.213322657300812	0.200400801603206\\
-0.213514517539743	0.216432865731463\\
-0.213068330138315	0.232464929859719\\
-0.211902379345734	0.248496993987976\\
-0.209920147523395	0.264529058116232\\
-0.207006872475959	0.280561122244489\\
-0.203025096977946	0.296593186372745\\
-0.200400801603207	0.304862370138295\\
}--cycle;

\end{axis}
\end{tikzpicture}%
    \caption{$p(\vec{x} \given \vec{\mu}, \mat{\Sigma})$}
    \label{marginal_2d_pdf}
  \end{subfigure}
  \begin{subfigure}[t]{0.49\textwidth}
    % This file was created by matlab2tikz.
% Minimal pgfplots version: 1.3
%
\tikzsetnextfilename{2d_marginal_pdf}
\definecolor{mycolor1}{rgb}{0.12157,0.47059,0.70588}%
%
\begin{tikzpicture}

\begin{axis}[%
width=0.95092\smallfigurewidth,
height=\smallfigureheight,
at={(0\smallfigurewidth,0\smallfigureheight)},
scale only axis,
xmin=-4,
xmax=4,
xlabel={$x_1$},
ymin=0,
ymax=0.4,
axis x line*=bottom,
axis y line*=left,
legend style={legend cell align=left,align=left,fill=none,draw=none}
]
\addplot [color=mycolor1,solid]
  table[row sep=crcr]{%
-4	0.000133830225764885\\
-3.98396793587174	0.000142675349741047\\
-3.96793587174349	0.000152065976657072\\
-3.95190380761523	0.000162033024993012\\
-3.93587174348697	0.000172608985016727\\
-3.91983967935872	0.000183827987316515\\
-3.90380761523046	0.000195725873640051\\
-3.8877755511022	0.000208340270077416\\
-3.87174348697395	0.000221710662624091\\
-3.85571142284569	0.000235878475157708\\
-3.83967935871743	0.00025088714986005\\
-3.82364729458918	0.000266782230113281\\
-3.80761523046092	0.000283611445896645\\
-3.79158316633267	0.000301424801706908\\
-3.77555110220441	0.000320274667022632\\
-3.75951903807615	0.00034021586932888\\
-3.7434869739479	0.000361305789715299\\
-3.72745490981964	0.000383604461056511\\
-3.71142284569138	0.000407174668779565\\
-3.69539078156313	0.000432082054218673\\
-3.67935871743487	0.000458395220552653\\
-3.66332665330661	0.000486185841315487\\
-3.64729458917836	0.000515528771464996\\
-3.6312625250501	0.000546502160988996\\
-3.61523046092184	0.000579187571022355\\
-3.59919839679359	0.000613670092442086\\
-3.58316633266533	0.000650038466901082\\
-3.56713426853707	0.00068838521025417\\
-3.55110220440882	0.000728806738323016\\
-3.53507014028056	0.000771403494938878\\
-3.5190380761523	0.000816280082194388\\
-3.50300601202405	0.000863545392827412\\
-3.48697394789579	0.000913312744651563\\
-3.47094188376753	0.000965700016939215\\
-3.45490981963928	0.00102082978865373\\
-3.43887775551102	0.00107882947841831\\
-3.42284569138277	0.00113983148609902\\
-3.40681362725451	0.00120397333586992\\
-3.39078156312625	0.00127139782061737\\
-3.374749498998	0.00134225314753069\\
-3.35871743486974	0.00141669308471503\\
-3.34268537074148	0.00149487710865133\\
-3.32665330661323	0.00157697055231716\\
-3.31062124248497	0.00166314475377022\\
-3.29458917835671	0.00175357720498483\\
-3.27855711422846	0.0018484517007196\\
-3.2625250501002	0.0019479584871821\\
-3.24649298597194	0.00205229441024435\\
-3.23046092184369	0.00216166306294995\\
-3.21442885771543	0.00227627493204146\\
-3.19839679358717	0.00239634754322351\\
-3.18236472945892	0.00252210560486446\\
-3.16633266533066	0.00265378114982641\\
-3.1503006012024	0.00279161367510061\\
-3.13426853707415	0.002935850278912\\
-3.11823647294589	0.00308674579494412\\
-3.10220440881764	0.00324456292332248\\
-3.08617234468938	0.00340957235798186\\
-3.07014028056112	0.00358205291003031\\
-3.05410821643287	0.00376229162671013\\
-3.03807615230461	0.00395058390554404\\
-3.02204408817635	0.0041472336032425\\
-3.0060120240481	0.00435255313893657\\
-2.98997995991984	0.00456686359128916\\
-2.97394789579158	0.00479049478902667\\
-2.95791583166333	0.00502378539442237\\
-2.94188376753507	0.00526708297925248\\
-2.92585170340681	0.00552074409273662\\
-2.90981963927856	0.005785134320965\\
-2.8937875751503	0.00606062833730627\\
-2.87775551102204	0.00634760994328225\\
-2.86172344689379	0.00664647209938829\\
-2.84569138276553	0.00695761694533229\\
-2.82965931863727	0.00728145580915903\\
-2.81362725450902	0.00761840920472244\\
-2.79759519038076	0.00796890681696439\\
-2.7815631262525	0.00833338747445562\\
-2.76553106212425	0.00871229910865282\\
-2.74949899799599	0.0091060986993251\\
-2.73346693386774	0.00951525220560328\\
-2.71743486973948	0.00994023448210711\\
-2.70140280561122	0.0103815291796084\\
-2.68537074148297	0.0108396286296914\\
-2.66933867735471	0.0113150337128785\\
-2.65330661322645	0.0118082537096938\\
-2.6372745490982	0.0123198061341476\\
-2.62124248496994	0.0128502165491321\\
-2.60521042084168	0.0134000183632325\\
-2.58917835671343	0.013969752608466\\
-2.57314629258517	0.0145599676984796\\
-2.55711422845691	0.015171219166751\\
-2.54108216432866	0.0158040693843541\\
-2.5250501002004	0.016459087256871\\
-2.50901803607214	0.0171368479000506\\
-2.49298597194389	0.0178379322938401\\
-2.47695390781563	0.0185629269144352\\
-2.46092184368737	0.0193124233440244\\
-2.44488977955912	0.0200870178579282\\
-2.42885771543086	0.0208873109888638\\
-2.41282565130261	0.021713907068098\\
-2.39679358717435	0.0225674137432813\\
-2.38076152304609	0.0234484414727945\\
-2.36472945891784	0.0243576029964712\\
-2.34869739478958	0.0252955127826016\\
-2.33266533066132	0.0262627864511589\\
-2.31663326653307	0.0272600401732346\\
-2.30060120240481	0.0282878900467097\\
-2.28456913827655	0.0293469514482331\\
-2.2685370741483	0.0304378383616278\\
-2.25250501002004	0.031561162682888\\
-2.23647294589178	0.0327175335019856\\
-2.22044088176353	0.033907556361749\\
-2.20440881763527	0.0351318324941335\\
-2.18837675350701	0.0363909580342532\\
-2.17234468937876	0.0376855232125996\\
-2.1563126252505	0.0390161115259268\\
-2.14028056112224	0.0403832988873405\\
-2.12424849699399	0.0417876527561837\\
-2.10821643286573	0.0432297312483712\\
-2.09218436873747	0.0447100822278847\\
-2.07615230460922	0.0462292423801957\\
-2.06012024048096	0.0477877362684481\\
-2.04408817635271	0.0493860753732887\\
-2.02805611222445	0.0510247571172962\\
-2.01202404809619	0.052704263875018\\
-1.99599198396794	0.0544250619696859\\
-1.97995991983968	0.0561876006577405\\
-1.96392785571142	0.0579923111023547\\
-1.94789579158317	0.0598396053372053\\
-1.93186372745491	0.0617298752217999\\
-1.91583166332665	0.0636634913897248\\
-1.8997995991984	0.0656408021912354\\
-1.88376753507014	0.0676621326316657\\
-1.86773547094188	0.0697277833071892\\
-1.85170340681363	0.0718380293395149\\
-1.83567134268537	0.0739931193111512\\
-1.81963927855711	0.0761932742029252\\
-1.80360721442886	0.078438686335485\\
-1.7875751503006	0.0807295183165622\\
-1.77154308617234	0.0830659019958129\\
-1.75551102204409	0.0854479374290955\\
-1.73947895791583	0.0878756918540804\\
-1.72344689378758	0.0903491986791223\\
-1.70741482965932	0.0928684564873552\\
-1.69138276553106	0.0954334280580022\\
-1.67535070140281	0.098044039406911\\
-1.65931863727455	0.100700178848355\\
-1.64328657314629	0.103401696080149\\
-1.62725450901804	0.106148401294153\\
-1.61122244488978	0.108940064314234\\
-1.59519038076152	0.111776413763773\\
-1.57915831663327	0.114657136264811\\
-1.56312625250501	0.117581875670895\\
-1.54709418837675	0.120550232335723\\
-1.5310621242485	0.12356176241964\\
-1.51503006012024	0.126615977236029\\
-1.49899799599198	0.129712342639645\\
-1.48296593186373	0.132850278458862\\
-1.46693386773547	0.136029157973839\\
-1.45090180360721	0.139248307442514\\
-1.43486973947896	0.142507005676342\\
-1.4188376753507	0.145804483667619\\
-1.40280561122244	0.149139924270191\\
-1.38677354709419	0.152512461935312\\
-1.37074148296593	0.155921182504311\\
-1.35470941883768	0.159365123059726\\
-1.33867735470942	0.162843271836423\\
-1.32264529058116	0.166354568194204\\
-1.30661322645291	0.169897902653299\\
-1.29058116232465	0.173472116994052\\
-1.27454909819639	0.177076004422035\\
-1.25851703406814	0.180708309799727\\
-1.24248496993988	0.184367729945797\\
-1.22645290581162	0.188052914002922\\
-1.21042084168337	0.191762463874985\\
-1.19438877755511	0.195494934734372\\
-1.17835671342685	0.199248835599966\\
-1.1623246492986	0.203022629986358\\
-1.14629258517034	0.206814736624628\\
-1.13026052104208	0.210623530254964\\
-1.11422845691383	0.214447342491225\\
-1.09819639278557	0.218284462757478\\
-1.08216432865731	0.222133139296347\\
-1.06613226452906	0.225991580248921\\
-1.0501002004008	0.229857954805832\\
-1.03406813627255	0.23373039442894\\
-1.01803607214429	0.237606994142988\\
-1.00200400801603	0.241485813896379\\
-0.985971943887776	0.245364879990149\\
-0.969939879759519	0.249242186574038\\
-0.953907815631263	0.253115697208426\\
-0.937875751503006	0.256983346490764\\
-0.92184368737475	0.260843041745\\
-0.905811623246493	0.264692664772344\\
-0.889779559118236	0.268530073661598\\
-0.87374749498998	0.272353104657127\\
-0.857715430861723	0.276159574082418\\
-0.841683366733467	0.279947280317065\\
-0.82565130260521	0.283714005824839\\
-0.809619238476954	0.287457519230438\\
-0.793587174348697	0.291175577442345\\
-0.777555110220441	0.29486592781912\\
-0.761523046092184	0.29852631037635\\
-0.745490981963928	0.302154460031336\\
-0.729458917835671	0.305748108882535\\
-0.713426853707415	0.309304988520637\\
-0.697394789579158	0.31282283236809\\
-0.681362725450902	0.316299378043765\\
-0.665330661322646	0.319732369749404\\
-0.649298597194389	0.323119560674393\\
-0.633266533066132	0.326458715415327\\
-0.617234468937876	0.329747612406789\\
-0.601202404809619	0.332984046359693\\
-0.585170340681363	0.336165830703478\\
-0.569138276553106	0.339290800028423\\
-0.55310621242485	0.342356812524294\\
-0.537074148296593	0.345361752411504\\
-0.521042084168337	0.348303532360969\\
-0.50501002004008	0.351180095898786\\
-0.488977955911824	0.353989419791894\\
-0.472945891783567	0.356729516410858\\
-0.456913827655311	0.359398436065926\\
-0.440881763527054	0.361994269312532\\
-0.424849699398798	0.364515149222442\\
-0.408817635270541	0.366959253616761\\
-0.392785571142285	0.369324807257091\\
-0.376753507014028	0.371610083991139\\
-0.360721442885771	0.373813408849153\\
-0.344689378757515	0.375933160087644\\
-0.328657314629258	0.377967771176883\\
-0.312625250501002	0.379915732728785\\
-0.296593186372745	0.381775594361851\\
-0.280561122244489	0.38354596649995\\
-0.264529058116232	0.385225522101812\\
-0.248496993987976	0.386812998318237\\
-0.232464929859719	0.388307198074096\\
-0.216432865731463	0.389706991572382\\
-0.200400801603207	0.391011317717639\\
-0.18436873747495	0.392219185456281\\
-0.168336673346694	0.393329675031411\\
-0.152304609218437	0.394341939149931\\
-0.13627254509018	0.395255204059869\\
-0.120240480961924	0.396068770535991\\
-0.104208416833667	0.39678201477196\\
-0.0881763527054109	0.397394389177437\\
-0.0721442885771544	0.397905423078697\\
-0.0561122244488979	0.398314723321516\\
-0.0400801603206413	0.398621974775231\\
-0.0240480961923848	0.398826940737091\\
-0.00801603206412826	0.398929463236143\\
0.00801603206412782	0.398929463236143\\
0.0240480961923843	0.398826940737091\\
0.0400801603206409	0.398621974775231\\
0.0561122244488974	0.398314723321516\\
0.0721442885771539	0.397905423078697\\
0.0881763527054105	0.397394389177437\\
0.104208416833667	0.39678201477196\\
0.120240480961924	0.396068770535991\\
0.13627254509018	0.395255204059869\\
0.152304609218437	0.394341939149931\\
0.168336673346693	0.393329675031411\\
0.18436873747495	0.392219185456281\\
0.200400801603206	0.391011317717639\\
0.216432865731463	0.389706991572382\\
0.232464929859719	0.388307198074096\\
0.248496993987976	0.386812998318237\\
0.264529058116232	0.385225522101812\\
0.280561122244489	0.38354596649995\\
0.296593186372745	0.381775594361851\\
0.312625250501002	0.379915732728785\\
0.328657314629258	0.377967771176883\\
0.344689378757515	0.375933160087644\\
0.360721442885771	0.373813408849153\\
0.376753507014028	0.371610083991139\\
0.392785571142285	0.369324807257091\\
0.408817635270541	0.366959253616761\\
0.424849699398798	0.364515149222442\\
0.440881763527054	0.361994269312532\\
0.456913827655311	0.359398436065926\\
0.472945891783567	0.356729516410858\\
0.488977955911824	0.353989419791894\\
0.50501002004008	0.351180095898786\\
0.521042084168337	0.348303532360969\\
0.537074148296593	0.345361752411504\\
0.55310621242485	0.342356812524294\\
0.569138276553106	0.339290800028423\\
0.585170340681363	0.336165830703478\\
0.601202404809619	0.332984046359693\\
0.617234468937876	0.329747612406789\\
0.633266533066132	0.326458715415327\\
0.649298597194389	0.323119560674393\\
0.665330661322646	0.319732369749404\\
0.681362725450902	0.316299378043765\\
0.697394789579159	0.31282283236809\\
0.713426853707415	0.309304988520637\\
0.729458917835672	0.305748108882535\\
0.745490981963928	0.302154460031336\\
0.761523046092185	0.29852631037635\\
0.777555110220441	0.29486592781912\\
0.793587174348698	0.291175577442344\\
0.809619238476954	0.287457519230438\\
0.825651302605211	0.283714005824839\\
0.841683366733467	0.279947280317065\\
0.857715430861724	0.276159574082418\\
0.87374749498998	0.272353104657126\\
0.889779559118236	0.268530073661598\\
0.905811623246493	0.264692664772344\\
0.921843687374749	0.260843041745\\
0.937875751503006	0.256983346490764\\
0.953907815631262	0.253115697208426\\
0.969939879759519	0.249242186574039\\
0.985971943887775	0.245364879990149\\
1.00200400801603	0.241485813896379\\
1.01803607214429	0.237606994142988\\
1.03406813627254	0.23373039442894\\
1.0501002004008	0.229857954805832\\
1.06613226452906	0.225991580248922\\
1.08216432865731	0.222133139296347\\
1.09819639278557	0.218284462757478\\
1.11422845691383	0.214447342491225\\
1.13026052104208	0.210623530254964\\
1.14629258517034	0.206814736624628\\
1.1623246492986	0.203022629986358\\
1.17835671342685	0.199248835599966\\
1.19438877755511	0.195494934734372\\
1.21042084168337	0.191762463874985\\
1.22645290581162	0.188052914002922\\
1.24248496993988	0.184367729945797\\
1.25851703406814	0.180708309799727\\
1.27454909819639	0.177076004422035\\
1.29058116232465	0.173472116994052\\
1.30661322645291	0.169897902653299\\
1.32264529058116	0.166354568194204\\
1.33867735470942	0.162843271836423\\
1.35470941883768	0.159365123059726\\
1.37074148296593	0.155921182504311\\
1.38677354709419	0.152512461935312\\
1.40280561122244	0.149139924270191\\
1.4188376753507	0.145804483667619\\
1.43486973947896	0.142507005676342\\
1.45090180360721	0.139248307442514\\
1.46693386773547	0.136029157973839\\
1.48296593186373	0.132850278458862\\
1.49899799599198	0.129712342639645\\
1.51503006012024	0.126615977236029\\
1.5310621242485	0.12356176241964\\
1.54709418837675	0.120550232335723\\
1.56312625250501	0.117581875670895\\
1.57915831663327	0.114657136264811\\
1.59519038076152	0.111776413763773\\
1.61122244488978	0.108940064314234\\
1.62725450901804	0.106148401294153\\
1.64328657314629	0.103401696080149\\
1.65931863727455	0.100700178848355\\
1.67535070140281	0.098044039406911\\
1.69138276553106	0.0954334280580021\\
1.70741482965932	0.0928684564873551\\
1.72344689378758	0.0903491986791222\\
1.73947895791583	0.0878756918540804\\
1.75551102204409	0.0854479374290954\\
1.77154308617235	0.0830659019958128\\
1.7875751503006	0.0807295183165622\\
1.80360721442886	0.0784386863354851\\
1.81963927855711	0.0761932742029253\\
1.83567134268537	0.0739931193111512\\
1.85170340681363	0.0718380293395149\\
1.86773547094188	0.0697277833071893\\
1.88376753507014	0.0676621326316657\\
1.8997995991984	0.0656408021912355\\
1.91583166332665	0.0636634913897249\\
1.93186372745491	0.0617298752217999\\
1.94789579158317	0.0598396053372053\\
1.96392785571142	0.0579923111023548\\
1.97995991983968	0.0561876006577406\\
1.99599198396794	0.054425061969686\\
2.01202404809619	0.052704263875018\\
2.02805611222445	0.0510247571172962\\
2.04408817635271	0.0493860753732887\\
2.06012024048096	0.0477877362684481\\
2.07615230460922	0.0462292423801957\\
2.09218436873747	0.0447100822278847\\
2.10821643286573	0.0432297312483712\\
2.12424849699399	0.0417876527561837\\
2.14028056112224	0.0403832988873405\\
2.1563126252505	0.0390161115259268\\
2.17234468937876	0.0376855232125996\\
2.18837675350701	0.0363909580342532\\
2.20440881763527	0.0351318324941335\\
2.22044088176353	0.033907556361749\\
2.23647294589178	0.0327175335019856\\
2.25250501002004	0.031561162682888\\
2.2685370741483	0.0304378383616278\\
2.28456913827655	0.0293469514482331\\
2.30060120240481	0.0282878900467097\\
2.31663326653307	0.0272600401732346\\
2.33266533066132	0.0262627864511589\\
2.34869739478958	0.0252955127826016\\
2.36472945891784	0.0243576029964712\\
2.38076152304609	0.0234484414727945\\
2.39679358717435	0.0225674137432813\\
2.41282565130261	0.021713907068098\\
2.42885771543086	0.0208873109888638\\
2.44488977955912	0.0200870178579282\\
2.46092184368737	0.0193124233440244\\
2.47695390781563	0.0185629269144352\\
2.49298597194389	0.0178379322938401\\
2.50901803607214	0.0171368479000506\\
2.5250501002004	0.016459087256871\\
2.54108216432866	0.0158040693843541\\
2.55711422845691	0.015171219166751\\
2.57314629258517	0.0145599676984796\\
2.58917835671343	0.013969752608466\\
2.60521042084168	0.0134000183632325\\
2.62124248496994	0.0128502165491321\\
2.6372745490982	0.0123198061341475\\
2.65330661322645	0.0118082537096938\\
2.66933867735471	0.0113150337128785\\
2.68537074148297	0.0108396286296914\\
2.70140280561122	0.0103815291796084\\
2.71743486973948	0.00994023448210712\\
2.73346693386774	0.00951525220560329\\
2.74949899799599	0.00910609869932512\\
2.76553106212425	0.00871229910865283\\
2.7815631262525	0.00833338747445562\\
2.79759519038076	0.00796890681696439\\
2.81362725450902	0.00761840920472244\\
2.82965931863727	0.00728145580915903\\
2.84569138276553	0.00695761694533229\\
2.86172344689379	0.00664647209938829\\
2.87775551102204	0.00634760994328225\\
2.8937875751503	0.00606062833730627\\
2.90981963927856	0.005785134320965\\
2.92585170340681	0.00552074409273662\\
2.94188376753507	0.00526708297925248\\
2.95791583166333	0.00502378539442237\\
2.97394789579158	0.00479049478902667\\
2.98997995991984	0.00456686359128916\\
3.0060120240481	0.00435255313893657\\
3.02204408817635	0.0041472336032425\\
3.03807615230461	0.00395058390554404\\
3.05410821643287	0.00376229162671013\\
3.07014028056112	0.00358205291003031\\
3.08617234468938	0.00340957235798186\\
3.10220440881764	0.00324456292332248\\
3.11823647294589	0.00308674579494412\\
3.13426853707415	0.002935850278912\\
3.1503006012024	0.00279161367510061\\
3.16633266533066	0.00265378114982641\\
3.18236472945892	0.00252210560486446\\
3.19839679358717	0.00239634754322351\\
3.21442885771543	0.00227627493204146\\
3.23046092184369	0.00216166306294995\\
3.24649298597194	0.00205229441024435\\
3.2625250501002	0.0019479584871821\\
3.27855711422846	0.0018484517007196\\
3.29458917835671	0.00175357720498483\\
3.31062124248497	0.00166314475377022\\
3.32665330661323	0.00157697055231716\\
3.34268537074148	0.00149487710865133\\
3.35871743486974	0.00141669308471503\\
3.374749498998	0.00134225314753069\\
3.39078156312625	0.00127139782061736\\
3.40681362725451	0.00120397333586992\\
3.42284569138277	0.00113983148609902\\
3.43887775551102	0.00107882947841831\\
3.45490981963928	0.00102082978865373\\
3.47094188376754	0.000965700016939214\\
3.48697394789579	0.000913312744651562\\
3.50300601202405	0.00086354539282741\\
3.51903807615231	0.000816280082194387\\
3.53507014028056	0.000771403494938876\\
3.55110220440882	0.000728806738323014\\
3.56713426853707	0.000688385210254171\\
3.58316633266533	0.000650038466901083\\
3.59919839679359	0.000613670092442087\\
3.61523046092184	0.000579187571022356\\
3.6312625250501	0.000546502160988997\\
3.64729458917836	0.000515528771464997\\
3.66332665330661	0.000486185841315487\\
3.67935871743487	0.000458395220552653\\
3.69539078156313	0.000432082054218674\\
3.71142284569138	0.000407174668779566\\
3.72745490981964	0.000383604461056511\\
3.7434869739479	0.0003613057897153\\
3.75951903807615	0.000340215869328881\\
3.77555110220441	0.000320274667022632\\
3.79158316633267	0.000301424801706908\\
3.80761523046092	0.000283611445896645\\
3.82364729458918	0.000266782230113281\\
3.83967935871743	0.00025088714986005\\
3.85571142284569	0.000235878475157708\\
3.87174348697395	0.000221710662624091\\
3.8877755511022	0.000208340270077416\\
3.90380761523046	0.000195725873640051\\
3.91983967935872	0.000183827987316515\\
3.93587174348697	0.000172608985016727\\
3.95190380761523	0.000162033024993012\\
3.96793587174349	0.000152065976657072\\
3.98396793587174	0.000142675349741047\\
4	0.000133830225764885\\
};
\addlegendentry{$p(x_1)$};

\end{axis}
\end{tikzpicture}%
    \caption{$p(x_1 \given \mu_1, \Sigma_{11}) = \mc{N}(x_1; 0, 1)$}
    \label{marginal_pdf}
  \end{subfigure}
  \caption{Marginalization example.  (\subref{marginal_2d_pdf}) shows
    the joint density over $\vec{x} = [x_1, x_2]\trans$; this is the same
    density as in Figure \ref{2d_examples}(\subref{2d_example_3}).
    (\subref{marginal_pdf}) shows the marginal density of $x_1$.}
  \label{marginal_example}
\end{figure}

\subsection*{Conditioning}

Another common scenario will be when we have a set of variables
$\vec{x}$ with a joint multivariate Gaussian prior distribution, and
are then told the value of a subset of these variables.  We may then
condition our prior distribution on this observation, giving a
posterior distribution over the remaining variables.

Suppose again that $\vec{x}$ has a multivariate Gaussian distribution:
\begin{equation*}
  p(\vec{x} \given \vec{\mu}, \mat{\Sigma})
  =
  \mc{N}(\vec{x}, \vec{\mu}, \mat{\Sigma}),
\end{equation*}
and that we have partitioned as before: $\vec{x} = [\vec{x}_1,
  \vec{x}_2]\trans$.  Suppose now that we learn the exact value of the
subvector $\vec{x}_2$.  Remarkably, the posterior distribution
\begin{equation*}
  p(\vec{x}_1 \given \vec{x}_2, \vec{\mu}, \mat{\Sigma})
\end{equation*}
is a Gaussian distribution!  The formula is
\begin{equation*}
  p(\vec{x}_1 \given \vec{x}_2, \vec{\mu}, \mat{\Sigma})
  =
  \mc{N}(\vec{x}_1; \vec{\mu}_{1 \given 2}, \mat{\Sigma}_{11 \given 2}),
\end{equation*}
with
\begin{align*}
  \vec{\mu}_{1 \given 2}
  &=
  \vec{\mu}_1 + \mat{\Sigma}_{12}\mat{\Sigma}_{22}\inv (\vec{x}_2 - \vec{\mu}_2);
  \\
  \mat{\Sigma}_{11 \given 2}
  &=
  \mat{\Sigma}_{11} - \mat{\Sigma}_{12}\mat{\Sigma}_{22}\inv \mat{\Sigma}_{21}.
\end{align*}
So we adjust the mean by an amount dependent on: (1) the covariance
between $\vec{x}_1$ and $\vec{x}_2$, $\mat{\Sigma}_{12}$, (2) the
prior uncertainty in $\vec{x}_2$, $\mat{\Sigma}_{22}$, and (3) the
deviation of the observation from the prior mean, $(\vec{x}_2 -
\vec{\mu}_2)$.  Similarly, we reduce the uncertainty in $\vec{x}_1$,
$\mat{\Sigma}_{11}$, by an amount dependent on (1) and (2).  Notably,
the reduction of the covariance matrix does \emph{not} depend on the
values we observe.

Notice that if $\vec{x}_1$ and $\vec{x}_2$ are independent, then
$\mat{\Sigma}_{12} = \mat{0}$, and the conditioning operation does not
change the distribution of $\vec{x}_1$, as expected.

Figure \ref{conditional_example} illustrates the conditional
distribution of $x_1$ for the joint distribution shown in Figure
\ref{2d_examples}(\subref{2d_example_3}), after observing $x_2 = 2$.

\begin{figure}
  \centering
  \begin{subfigure}[t]{0.49\textwidth}
    % This file was created by matlab2tikz.
% Minimal pgfplots version: 1.3
%
\tikzsetnextfilename{2d_gaussian_pdf_conditional}
\definecolor{mycolor1}{rgb}{0.01430,0.01430,0.01430}%
\definecolor{mycolor2}{rgb}{0.15932,0.06827,0.17506}%
\definecolor{mycolor3}{rgb}{0.17345,0.11709,0.41691}%
\definecolor{mycolor4}{rgb}{0.10466,0.22842,0.49922}%
\definecolor{mycolor5}{rgb}{0.03136,0.34573,0.47968}%
\definecolor{mycolor6}{rgb}{0.00003,0.46181,0.36160}%
\definecolor{mycolor7}{rgb}{0.00000,0.57116,0.23204}%
\definecolor{mycolor8}{rgb}{0.09251,0.67012,0.06175}%
\definecolor{mycolor9}{rgb}{0.37724,0.75416,0.00000}%
\definecolor{mycolor10}{rgb}{0.74811,0.78629,0.07418}%
\definecolor{mycolor11}{rgb}{0.94890,0.82638,0.64748}%
\definecolor{mycolor12}{rgb}{0.96920,0.92730,0.89610}%
%
\begin{tikzpicture}

\begin{axis}[%
width=0.95092\squarefigurewidth,
height=\squarefigureheight,
at={(0\squarefigurewidth,0\squarefigureheight)},
scale only axis,
xmin=-4,
xmax=4,
xlabel={$x_1$},
ymin=-4,
ymax=4,
ylabel={$x_2$},
axis x line*=bottom,
axis y line*=left
]

\addplot[area legend,solid,fill=mycolor1,draw=black,forget plot]
table[row sep=crcr] {%
x	y\\
-4	4.00000000053783\\
-3.98396793587174	4.00000000057334\\
-3.96793587174349	4.00000000061096\\
-3.95190380761523	4.0000000006508\\
-3.93587174348697	4.00000000069296\\
-3.91983967935872	4.00000000073758\\
-3.90380761523046	4.00000000078476\\
-3.8877755511022	4.00000000083464\\
-3.87174348697395	4.00000000088735\\
-3.85571142284569	4.00000000094302\\
-3.83967935871743	4.0000000010018\\
-3.82364729458918	4.00000000106383\\
-3.80761523046092	4.00000000112927\\
-3.79158316633267	4.00000000119827\\
-3.77555110220441	4.000000001271\\
-3.75951903807615	4.00000000134762\\
-3.7434869739479	4.00000000142832\\
-3.72745490981964	4.00000000151325\\
-3.71142284569138	4.00000000160263\\
-3.69539078156313	4.00000000169662\\
-3.67935871743487	4.00000000179544\\
-3.66332665330661	4.00000000189928\\
-3.64729458917836	4.00000000200835\\
-3.6312625250501	4.00000000212287\\
-3.61523046092184	4.00000000224305\\
-3.59919839679359	4.00000000236912\\
-3.58316633266533	4.00000000250131\\
-3.56713426853707	4.00000000263986\\
-3.55110220440882	4.00000000278501\\
-3.53507014028056	4.00000000293701\\
-3.5190380761523	4.00000000309611\\
-3.50300601202405	4.00000000326257\\
-3.48697394789579	4.00000000343666\\
-3.47094188376753	4.00000000361864\\
-3.45490981963928	4.00000000380879\\
-3.43887775551102	4.00000000400738\\
-3.42284569138277	4.00000000421471\\
-3.40681362725451	4.00000000443105\\
-3.39078156312625	4.0000000046567\\
-3.374749498998	4.00000000489195\\
-3.35871743486974	4.00000000513711\\
-3.34268537074148	4.00000000539247\\
-3.32665330661323	4.00000000565835\\
-3.31062124248497	4.00000000593505\\
-3.29458917835671	4.00000000622288\\
-3.27855711422846	4.00000000652215\\
-3.2625250501002	4.00000000683318\\
-3.24649298597194	4.00000000715629\\
-3.23046092184369	4.00000000749178\\
-3.21442885771543	4.00000000783997\\
-3.19839679358717	4.00000000820119\\
-3.18236472945892	4.00000000857575\\
-3.16633266533066	4.00000000896395\\
-3.1503006012024	4.00000000936611\\
-3.13426853707415	4.00000000978255\\
-3.11823647294589	4.00000001021356\\
-3.10220440881764	4.00000001065945\\
-3.08617234468938	4.00000001112052\\
-3.07014028056112	4.00000001159706\\
-3.05410821643287	4.00000001208936\\
-3.03807615230461	4.0000000125977\\
-3.02204408817635	4.00000001312236\\
-3.0060120240481	4.00000001366359\\
-2.98997995991984	4.00000001422167\\
-2.97394789579158	4.00000001479684\\
-2.95791583166333	4.00000001538933\\
-2.94188376753507	4.00000001599937\\
-2.92585170340681	4.00000001662719\\
-2.90981963927856	4.00000001727298\\
-2.8937875751503	4.00000001793694\\
-2.87775551102204	4.00000001861924\\
-2.86172344689379	4.00000001932004\\
-2.84569138276553	4.00000002003949\\
-2.82965931863727	4.00000002077772\\
-2.81362725450902	4.00000002153485\\
-2.79759519038076	4.00000002231096\\
-2.7815631262525	4.00000002310613\\
-2.76553106212425	4.00000002392041\\
-2.74949899799599	4.00000002475385\\
-2.73346693386774	4.00000002560645\\
-2.71743486973948	4.0000000264782\\
-2.70140280561122	4.00000002736908\\
-2.68537074148297	4.00000002827904\\
-2.66933867735471	4.00000002920798\\
-2.65330661322645	4.0000000301558\\
-2.6372745490982	4.00000003112239\\
-2.62124248496994	4.00000003210757\\
-2.60521042084168	4.00000003311117\\
-2.58917835671343	4.00000003413298\\
-2.57314629258517	4.00000003517276\\
-2.55711422845691	4.00000003623025\\
-2.54108216432866	4.00000003730514\\
-2.5250501002004	4.00000003839712\\
-2.50901803607214	4.00000003950582\\
-2.49298597194389	4.00000004063087\\
-2.47695390781563	4.00000004177185\\
-2.46092184368737	4.00000004292832\\
-2.44488977955912	4.0000000440998\\
-2.42885771543086	4.00000004528579\\
-2.41282565130261	4.00000004648575\\
-2.39679358717435	4.0000000476991\\
-2.38076152304609	4.00000004892527\\
-2.36472945891784	4.0000000501636\\
-2.34869739478958	4.00000005141346\\
-2.33266533066132	4.00000005267415\\
-2.31663326653307	4.00000005394494\\
-2.30060120240481	4.0000000552251\\
-2.28456913827655	4.00000005651385\\
-2.2685370741483	4.00000005781037\\
-2.25250501002004	4.00000005911385\\
-2.23647294589178	4.00000006042342\\
-2.22044088176353	4.00000006173819\\
-2.20440881763527	4.00000006305725\\
-2.18837675350701	4.00000006437967\\
-2.17234468937876	4.00000006570449\\
-2.1563126252505	4.00000006703072\\
-2.14028056112224	4.00000006835736\\
-2.12424849699399	4.00000006968339\\
-2.10821643286573	4.00000007100775\\
-2.09218436873747	4.0000000723294\\
-2.07615230460922	4.00000007364725\\
-2.06012024048096	4.0000000749602\\
-2.04408817635271	4.00000007626715\\
-2.02805611222445	4.00000007756697\\
-2.01202404809619	4.00000007885854\\
-1.99599198396794	4.00000008014071\\
-1.97995991983968	4.00000008141234\\
-1.96392785571142	4.00000008267226\\
-1.94789579158317	4.00000008391932\\
-1.93186372745491	4.00000008515236\\
-1.91583166332665	4.0000000863702\\
-1.8997995991984	4.0000000875717\\
-1.88376753507014	4.00000008875568\\
-1.86773547094188	4.000000089921\\
-1.85170340681363	4.0000000910665\\
-1.83567134268537	4.00000009219105\\
-1.81963927855711	4.0000000932935\\
-1.80360721442886	4.00000009437275\\
-1.7875751503006	4.00000009542768\\
-1.77154308617234	4.00000009645721\\
-1.75551102204409	4.00000009746027\\
-1.73947895791583	4.0000000984358\\
-1.72344689378758	4.00000009938277\\
-1.70741482965932	4.00000010030017\\
-1.69138276553106	4.00000010118702\\
-1.67535070140281	4.00000010204237\\
-1.65931863727455	4.00000010286528\\
-1.64328657314629	4.00000010365485\\
-1.62725450901804	4.00000010441023\\
-1.61122244488978	4.00000010513056\\
-1.59519038076152	4.00000010581507\\
-1.57915831663327	4.00000010646298\\
-1.56312625250501	4.00000010707356\\
-1.54709418837675	4.00000010764614\\
-1.5310621242485	4.00000010818007\\
-1.51503006012024	4.00000010867473\\
-1.49899799599198	4.00000010912958\\
-1.48296593186373	4.00000010954409\\
-1.46693386773547	4.00000010991779\\
-1.45090180360721	4.00000011025024\\
-1.43486973947896	4.00000011054108\\
-1.4188376753507	4.00000011078997\\
-1.40280561122244	4.00000011099661\\
-1.38677354709419	4.00000011116077\\
-1.37074148296593	4.00000011128227\\
-1.35470941883768	4.00000011136095\\
-1.33867735470942	4.00000011139673\\
-1.32264529058116	4.00000011138958\\
-1.30661322645291	4.00000011133949\\
-1.29058116232465	4.00000011124652\\
-1.27454909819639	4.00000011111078\\
-1.25851703406814	4.00000011093244\\
-1.24248496993988	4.00000011071169\\
-1.22645290581162	4.00000011044878\\
-1.21042084168337	4.00000011014403\\
-1.19438877755511	4.00000010979778\\
-1.17835671342685	4.00000010941043\\
-1.1623246492986	4.00000010898242\\
-1.14629258517034	4.00000010851424\\
-1.13026052104208	4.00000010800642\\
-1.11422845691383	4.00000010745955\\
-1.09819639278557	4.00000010687423\\
-1.08216432865731	4.00000010625112\\
-1.06613226452906	4.00000010559093\\
-1.0501002004008	4.00000010489439\\
-1.03406813627255	4.00000010416229\\
-1.01803607214429	4.00000010339542\\
-1.00200400801603	4.00000010259463\\
-0.985971943887776	4.00000010176081\\
-0.969939879759519	4.00000010089486\\
-0.953907815631263	4.00000009999772\\
-0.937875751503006	4.00000009907035\\
-0.92184368737475	4.00000009811374\\
-0.905811623246493	4.00000009712892\\
-0.889779559118236	4.00000009611693\\
-0.87374749498998	4.00000009507881\\
-0.857715430861723	4.00000009401565\\
-0.841683366733467	4.00000009292854\\
-0.82565130260521	4.0000000918186\\
-0.809619238476954	4.00000009068694\\
-0.793587174348697	4.00000008953471\\
-0.777555110220441	4.00000008836304\\
-0.761523046092184	4.00000008717309\\
-0.745490981963928	4.00000008596601\\
-0.729458917835671	4.00000008474298\\
-0.713426853707415	4.00000008350514\\
-0.697394789579158	4.00000008225366\\
-0.681362725450902	4.00000008098971\\
-0.665330661322646	4.00000007971444\\
-0.649298597194389	4.00000007842901\\
-0.633266533066132	4.00000007713456\\
-0.617234468937876	4.00000007583223\\
-0.601202404809619	4.00000007452316\\
-0.585170340681363	4.00000007320845\\
-0.569138276553106	4.00000007188922\\
-0.55310621242485	4.00000007056655\\
-0.537074148296593	4.00000006924151\\
-0.521042084168337	4.00000006791516\\
-0.50501002004008	4.00000006658855\\
-0.488977955911824	4.00000006526268\\
-0.472945891783567	4.00000006393855\\
-0.456913827655311	4.00000006261714\\
-0.440881763527054	4.00000006129941\\
-0.424849699398798	4.00000005998627\\
-0.408817635270541	4.00000005867864\\
-0.392785571142285	4.00000005737738\\
-0.376753507014028	4.00000005608336\\
-0.360721442885771	4.00000005479739\\
-0.344689378757515	4.00000005352027\\
-0.328657314629258	4.00000005225276\\
-0.312625250501002	4.00000005099561\\
-0.296593186372745	4.00000004974951\\
-0.280561122244489	4.00000004851516\\
-0.264529058116232	4.0000000472932\\
-0.248496993987976	4.00000004608424\\
-0.232464929859719	4.00000004488888\\
-0.216432865731463	4.00000004370767\\
-0.200400801603207	4.00000004254114\\
-0.18436873747495	4.00000004138978\\
-0.168336673346694	4.00000004025406\\
-0.152304609218437	4.00000003913442\\
-0.13627254509018	4.00000003803125\\
-0.120240480961924	4.00000003694493\\
-0.104208416833667	4.0000000358758\\
-0.0881763527054109	4.00000003482419\\
-0.0721442885771544	4.00000003379037\\
-0.0561122244488979	4.00000003277461\\
-0.0400801603206413	4.00000003177712\\
-0.0240480961923848	4.00000003079812\\
-0.00801603206412826	4.00000002983777\\
0.00801603206412782	4.00000002889622\\
0.0240480961923843	4.0000000279736\\
0.0400801603206409	4.00000002707\\
0.0561122244488974	4.00000002618549\\
0.0721442885771539	4.00000002532012\\
0.0881763527054105	4.00000002447391\\
0.104208416833667	4.00000002364686\\
0.120240480961924	4.00000002283895\\
0.13627254509018	4.00000002205014\\
0.152304609218437	4.00000002128037\\
0.168336673346693	4.00000002052955\\
0.18436873747495	4.00000001979759\\
0.200400801603206	4.00000001908437\\
0.216432865731463	4.00000001838976\\
0.232464929859719	4.00000001771359\\
0.248496993987976	4.00000001705571\\
0.264529058116232	4.00000001641593\\
0.280561122244489	4.00000001579406\\
0.296593186372745	4.00000001518989\\
0.312625250501002	4.0000000146032\\
0.328657314629258	4.00000001403376\\
0.344689378757515	4.00000001348132\\
0.360721442885771	4.00000001294564\\
0.376753507014028	4.00000001242646\\
0.392785571142285	4.0000000119235\\
0.408817635270541	4.00000001143648\\
0.424849699398798	4.00000001096513\\
0.440881763527054	4.00000001050915\\
0.456913827655311	4.00000001006825\\
0.472945891783567	4.00000000964213\\
0.488977955911824	4.00000000923049\\
0.50501002004008	4.00000000883301\\
0.521042084168337	4.00000000844939\\
0.537074148296593	4.00000000807932\\
0.55310621242485	4.00000000772248\\
0.569138276553106	4.00000000737855\\
0.585170340681363	4.00000000704722\\
0.601202404809619	4.00000000672818\\
0.617234468937876	4.0000000064211\\
0.633266533066132	4.00000000612568\\
0.649298597194389	4.00000000584159\\
0.665330661322646	4.00000000556854\\
0.681362725450902	4.0000000053062\\
0.697394789579159	4.00000000505427\\
0.713426853707415	4.00000000481245\\
0.729458917835672	4.00000000458043\\
0.745490981963928	4.00000000435791\\
0.761523046092185	4.00000000414461\\
0.777555110220441	4.00000000394023\\
0.793587174348698	4.00000000374448\\
0.809619238476954	4.00000000355709\\
0.825651302605211	4.00000000337777\\
0.841683366733467	4.00000000320625\\
0.857715430861724	4.00000000304227\\
0.87374749498998	4.00000000288557\\
0.889779559118236	4.00000000273588\\
0.905811623246493	4.00000000259296\\
0.921843687374749	4.00000000245655\\
0.937875751503006	4.00000000232643\\
0.953907815631262	4.00000000220234\\
0.969939879759519	4.00000000208408\\
0.985971943887775	4.0000000019714\\
1.00200400801603	4.00000000186409\\
1.01803607214429	4.00000000176195\\
1.03406813627254	4.00000000166477\\
1.0501002004008	4.00000000157233\\
1.06613226452906	4.00000000148446\\
1.08216432865731	4.00000000140096\\
1.09819639278557	4.00000000132164\\
1.11422845691383	4.00000000124633\\
1.13026052104208	4.00000000117487\\
1.14629258517034	4.00000000110707\\
1.1623246492986	4.00000000104278\\
1.17835671342685	4.00000000098185\\
1.19438877755511	4.00000000092412\\
1.21042084168337	4.00000000086946\\
1.22645290581162	4.00000000081771\\
1.24248496993988	4.00000000076874\\
1.25851703406814	4.00000000072243\\
1.27454909819639	4.00000000067864\\
1.29058116232465	4.00000000063727\\
1.30661322645291	4.00000000059818\\
1.32264529058116	4.00000000056128\\
1.33867735470942	4.00000000052645\\
1.35470941883768	4.00000000049359\\
1.37074148296593	4.0000000004626\\
1.38677354709419	4.00000000043339\\
1.40280561122244	4.00000000040587\\
1.4188376753507	4.00000000037995\\
1.43486973947896	4.00000000035555\\
1.45090180360721	4.00000000033259\\
1.46693386773547	4.00000000031099\\
1.48296593186373	4.00000000029068\\
1.49899799599198	4.00000000027159\\
1.51503006012024	4.00000000025366\\
1.5310621242485	4.00000000023682\\
1.54709418837675	4.00000000022102\\
1.56312625250501	4.00000000020618\\
1.57915831663327	4.00000000019228\\
1.59519038076152	4.00000000017923\\
1.61122244488978	4.00000000016701\\
1.62725450901804	4.00000000015557\\
1.64328657314629	4.00000000014485\\
1.65931863727455	4.00000000013482\\
1.67535070140281	4.00000000012543\\
1.69138276553106	4.00000000011665\\
1.70741482965932	4.00000000010845\\
1.72344689378758	4.00000000010078\\
1.73947895791583	4.00000000009362\\
1.75551102204409	4.00000000008693\\
1.77154308617235	4.0000000000807\\
1.7875751503006	4.00000000007488\\
1.80360721442886	4.00000000006945\\
1.81963927855711	4.00000000006439\\
1.83567134268537	4.00000000005968\\
1.85170340681363	4.00000000005529\\
1.86773547094188	4.0000000000512\\
1.88376753507014	4.0000000000474\\
1.8997995991984	4.00000000004386\\
1.91583166332665	4.00000000004057\\
1.93186372745491	4.00000000003751\\
1.94789579158317	4.00000000003468\\
1.96392785571142	4.00000000003204\\
1.97995991983968	4.00000000002959\\
1.99599198396794	4.00000000002732\\
2.01202404809619	4.00000000002521\\
2.02805611222445	4.00000000002326\\
2.04408817635271	4.00000000002145\\
2.06012024048096	4.00000000001977\\
2.07615230460922	4.00000000001822\\
2.09218436873747	4.00000000001678\\
2.10821643286573	4.00000000001545\\
2.12424849699399	4.00000000001422\\
2.14028056112224	4.00000000001308\\
2.1563126252505	4.00000000001203\\
2.17234468937876	4.00000000001106\\
2.18837675350701	4.00000000001017\\
2.20440881763527	4.00000000000934\\
2.22044088176353	4.00000000000858\\
2.23647294589178	4.00000000000787\\
2.25250501002004	4.00000000000722\\
2.2685370741483	4.00000000000662\\
2.28456913827655	4.00000000000607\\
2.30060120240481	4.00000000000557\\
2.31663326653307	4.0000000000051\\
2.33266533066132	4.00000000000467\\
2.34869739478958	4.00000000000427\\
2.36472945891784	4.00000000000391\\
2.38076152304609	4.00000000000358\\
2.39679358717435	4.00000000000327\\
2.41282565130261	4.00000000000299\\
2.42885771543086	4.00000000000273\\
2.44488977955912	4.0000000000025\\
2.46092184368737	4.00000000000228\\
2.47695390781563	4.00000000000208\\
2.49298597194389	4.0000000000019\\
2.50901803607214	4.00000000000173\\
2.5250501002004	4.00000000000158\\
2.54108216432866	4.00000000000144\\
2.55711422845691	4.00000000000131\\
2.57314629258517	4.00000000000119\\
2.58917835671343	4.00000000000108\\
2.60521042084168	4.00000000000099\\
2.62124248496994	4.0000000000009\\
2.6372745490982	4.00000000000082\\
2.65330661322645	4.00000000000074\\
2.66933867735471	4.00000000000067\\
2.68537074148297	4.00000000000061\\
2.70140280561122	4.00000000000056\\
2.71743486973948	4.0000000000005\\
2.73346693386774	4.00000000000046\\
2.74949899799599	4.00000000000041\\
2.76553106212425	4.00000000000038\\
2.7815631262525	4.00000000000034\\
2.79759519038076	4.00000000000031\\
2.81362725450902	4.00000000000028\\
2.82965931863727	4.00000000000025\\
2.84569138276553	4.00000000000023\\
2.86172344689379	4.00000000000021\\
2.87775551102204	4.00000000000019\\
2.8937875751503	4.00000000000017\\
2.90981963927856	4.00000000000015\\
2.92585170340681	4.00000000000014\\
2.94188376753507	4.00000000000012\\
2.95791583166333	4.00000000000011\\
2.97394789579158	4.0000000000001\\
2.98997995991984	4.00000000000009\\
3.0060120240481	4.00000000000008\\
3.02204408817635	4.00000000000007\\
3.03807615230461	4.00000000000007\\
3.05410821643287	4.00000000000006\\
3.07014028056112	4.00000000000005\\
3.08617234468938	4.00000000000005\\
3.10220440881764	4.00000000000004\\
3.11823647294589	4.00000000000004\\
3.13426853707415	4.00000000000003\\
3.1503006012024	4.00000000000003\\
3.16633266533066	4.00000000000003\\
3.18236472945892	4.00000000000003\\
3.19839679358717	4.00000000000002\\
3.21442885771543	4.00000000000002\\
3.23046092184369	4.00000000000002\\
3.24649298597194	4.00000000000002\\
3.2625250501002	4.00000000000001\\
3.27855711422846	4.00000000000001\\
3.29458917835671	4.00000000000001\\
3.31062124248497	4.00000000000001\\
3.32665330661323	4.00000000000001\\
3.34268537074148	4.00000000000001\\
3.35871743486974	4.00000000000001\\
3.374749498998	4.00000000000001\\
3.39078156312625	4.00000000000001\\
3.40681362725451	4.00000000000001\\
3.42284569138277	4\\
3.43887775551102	4\\
3.45490981963928	4\\
3.47094188376754	4\\
3.48697394789579	4\\
3.50300601202405	4\\
3.51903807615231	4\\
3.53507014028056	4\\
3.55110220440882	4\\
3.56713426853707	4\\
3.58316633266533	4\\
3.59919839679359	4\\
3.61523046092184	4\\
3.6312625250501	4\\
3.64729458917836	4\\
3.66332665330661	4\\
3.67935871743487	4\\
3.69539078156313	4\\
3.71142284569138	4\\
3.72745490981964	4\\
3.7434869739479	4\\
3.75951903807615	4\\
3.77555110220441	4\\
3.79158316633267	4\\
3.80761523046092	4\\
3.82364729458918	4\\
3.83967935871743	4\\
3.85571142284569	4\\
3.87174348697395	4\\
3.8877755511022	4\\
3.90380761523046	4\\
3.91983967935872	4\\
3.93587174348697	4\\
3.95190380761523	4\\
3.96793587174349	4\\
3.98396793587174	4\\
4	4\\
4	3.98396793587174\\
4	3.96793587174349\\
4	3.95190380761523\\
4	3.93587174348697\\
4	3.91983967935872\\
4	3.90380761523046\\
4	3.8877755511022\\
4	3.87174348697395\\
4	3.85571142284569\\
4	3.83967935871743\\
4	3.82364729458918\\
4	3.80761523046092\\
4	3.79158316633267\\
4	3.77555110220441\\
4	3.75951903807615\\
4	3.7434869739479\\
4	3.72745490981964\\
4	3.71142284569138\\
4	3.69539078156313\\
4	3.67935871743487\\
4	3.66332665330661\\
4	3.64729458917836\\
4	3.6312625250501\\
4	3.61523046092184\\
4	3.59919839679359\\
4	3.58316633266533\\
4	3.56713426853707\\
4	3.55110220440882\\
4	3.53507014028056\\
4	3.51903807615231\\
4	3.50300601202405\\
4	3.48697394789579\\
4	3.47094188376754\\
4	3.45490981963928\\
4	3.43887775551102\\
4	3.42284569138277\\
4	3.40681362725451\\
4	3.39078156312625\\
4	3.374749498998\\
4	3.35871743486974\\
4	3.34268537074148\\
4	3.32665330661323\\
4	3.31062124248497\\
4	3.29458917835671\\
4	3.27855711422846\\
4	3.2625250501002\\
4	3.24649298597194\\
4	3.23046092184369\\
4	3.21442885771543\\
4	3.19839679358717\\
4	3.18236472945892\\
4	3.16633266533066\\
4	3.1503006012024\\
4	3.13426853707415\\
4	3.11823647294589\\
4	3.10220440881764\\
4	3.08617234468938\\
4	3.07014028056112\\
4	3.05410821643287\\
4	3.03807615230461\\
4	3.02204408817635\\
4	3.0060120240481\\
4	2.98997995991984\\
4	2.97394789579158\\
4	2.95791583166333\\
4	2.94188376753507\\
4	2.92585170340681\\
4	2.90981963927856\\
4	2.8937875751503\\
4	2.87775551102204\\
4	2.86172344689379\\
4	2.84569138276553\\
4	2.82965931863727\\
4	2.81362725450902\\
4.00000000000001	2.79759519038076\\
4.00000000000001	2.7815631262525\\
4.00000000000001	2.76553106212425\\
4.00000000000001	2.74949899799599\\
4.00000000000001	2.73346693386774\\
4.00000000000001	2.71743486973948\\
4.00000000000001	2.70140280561122\\
4.00000000000001	2.68537074148297\\
4.00000000000001	2.66933867735471\\
4.00000000000001	2.65330661322645\\
4.00000000000001	2.6372745490982\\
4.00000000000001	2.62124248496994\\
4.00000000000001	2.60521042084168\\
4.00000000000001	2.58917835671343\\
4.00000000000001	2.57314629258517\\
4.00000000000001	2.55711422845691\\
4.00000000000001	2.54108216432866\\
4.00000000000001	2.5250501002004\\
4.00000000000001	2.50901803607214\\
4.00000000000001	2.49298597194389\\
4.00000000000001	2.47695390781563\\
4.00000000000002	2.46092184368737\\
4.00000000000002	2.44488977955912\\
4.00000000000002	2.42885771543086\\
4.00000000000002	2.41282565130261\\
4.00000000000002	2.39679358717435\\
4.00000000000002	2.38076152304609\\
4.00000000000002	2.36472945891784\\
4.00000000000002	2.34869739478958\\
4.00000000000002	2.33266533066132\\
4.00000000000002	2.31663326653307\\
4.00000000000003	2.30060120240481\\
4.00000000000003	2.28456913827655\\
4.00000000000003	2.2685370741483\\
4.00000000000003	2.25250501002004\\
4.00000000000003	2.23647294589178\\
4.00000000000003	2.22044088176353\\
4.00000000000004	2.20440881763527\\
4.00000000000004	2.18837675350701\\
4.00000000000004	2.17234468937876\\
4.00000000000004	2.1563126252505\\
4.00000000000004	2.14028056112224\\
4.00000000000005	2.12424849699399\\
4.00000000000005	2.10821643286573\\
4.00000000000005	2.09218436873747\\
4.00000000000005	2.07615230460922\\
4.00000000000006	2.06012024048096\\
4.00000000000006	2.04408817635271\\
4.00000000000006	2.02805611222445\\
4.00000000000006	2.01202404809619\\
4.00000000000007	1.99599198396794\\
4.00000000000007	1.97995991983968\\
4.00000000000007	1.96392785571142\\
4.00000000000008	1.94789579158317\\
4.00000000000008	1.93186372745491\\
4.00000000000009	1.91583166332665\\
4.00000000000009	1.8997995991984\\
4.00000000000009	1.88376753507014\\
4.0000000000001	1.86773547094188\\
4.0000000000001	1.85170340681363\\
4.00000000000011	1.83567134268537\\
4.00000000000011	1.81963927855711\\
4.00000000000012	1.80360721442886\\
4.00000000000012	1.7875751503006\\
4.00000000000013	1.77154308617235\\
4.00000000000014	1.75551102204409\\
4.00000000000014	1.73947895791583\\
4.00000000000015	1.72344689378758\\
4.00000000000016	1.70741482965932\\
4.00000000000016	1.69138276553106\\
4.00000000000017	1.67535070140281\\
4.00000000000018	1.65931863727455\\
4.00000000000019	1.64328657314629\\
4.0000000000002	1.62725450901804\\
4.00000000000021	1.61122244488978\\
4.00000000000021	1.59519038076152\\
4.00000000000022	1.57915831663327\\
4.00000000000023	1.56312625250501\\
4.00000000000025	1.54709418837675\\
4.00000000000026	1.5310621242485\\
4.00000000000027	1.51503006012024\\
4.00000000000028	1.49899799599198\\
4.00000000000029	1.48296593186373\\
4.00000000000031	1.46693386773547\\
4.00000000000032	1.45090180360721\\
4.00000000000033	1.43486973947896\\
4.00000000000035	1.4188376753507\\
4.00000000000036	1.40280561122244\\
4.00000000000038	1.38677354709419\\
4.0000000000004	1.37074148296593\\
4.00000000000041	1.35470941883768\\
4.00000000000043	1.33867735470942\\
4.00000000000045	1.32264529058116\\
4.00000000000047	1.30661322645291\\
4.00000000000049	1.29058116232465\\
4.00000000000051	1.27454909819639\\
4.00000000000053	1.25851703406814\\
4.00000000000056	1.24248496993988\\
4.00000000000058	1.22645290581162\\
4.00000000000061	1.21042084168337\\
4.00000000000063	1.19438877755511\\
4.00000000000066	1.17835671342685\\
4.00000000000069	1.1623246492986\\
4.00000000000072	1.14629258517034\\
4.00000000000075	1.13026052104208\\
4.00000000000078	1.11422845691383\\
4.00000000000081	1.09819639278557\\
4.00000000000084	1.08216432865731\\
4.00000000000088	1.06613226452906\\
4.00000000000091	1.0501002004008\\
4.00000000000095	1.03406813627254\\
4.00000000000099	1.01803607214429\\
4.00000000000103	1.00200400801603\\
4.00000000000108	0.985971943887775\\
4.00000000000112	0.969939879759519\\
4.00000000000116	0.953907815631262\\
4.00000000000121	0.937875751503006\\
4.00000000000126	0.921843687374749\\
4.00000000000131	0.905811623246493\\
4.00000000000136	0.889779559118236\\
4.00000000000142	0.87374749498998\\
4.00000000000147	0.857715430861724\\
4.00000000000153	0.841683366733467\\
4.00000000000159	0.825651302605211\\
4.00000000000166	0.809619238476954\\
4.00000000000172	0.793587174348698\\
4.00000000000179	0.777555110220441\\
4.00000000000186	0.761523046092185\\
4.00000000000193	0.745490981963928\\
4.000000000002	0.729458917835672\\
4.00000000000208	0.713426853707415\\
4.00000000000216	0.697394789579159\\
4.00000000000225	0.681362725450902\\
4.00000000000233	0.665330661322646\\
4.00000000000242	0.649298597194389\\
4.00000000000251	0.633266533066132\\
4.00000000000261	0.617234468937876\\
4.0000000000027	0.601202404809619\\
4.00000000000281	0.585170340681363\\
4.00000000000291	0.569138276553106\\
4.00000000000302	0.55310621242485\\
4.00000000000313	0.537074148296593\\
4.00000000000325	0.521042084168337\\
4.00000000000337	0.50501002004008\\
4.00000000000349	0.488977955911824\\
4.00000000000362	0.472945891783567\\
4.00000000000375	0.456913827655311\\
4.00000000000388	0.440881763527054\\
4.00000000000403	0.424849699398798\\
4.00000000000417	0.408817635270541\\
4.00000000000432	0.392785571142285\\
4.00000000000448	0.376753507014028\\
4.00000000000464	0.360721442885771\\
4.0000000000048	0.344689378757515\\
4.00000000000497	0.328657314629258\\
4.00000000000514	0.312625250501002\\
4.00000000000532	0.296593186372745\\
4.00000000000551	0.280561122244489\\
4.0000000000057	0.264529058116232\\
4.0000000000059	0.248496993987976\\
4.00000000000611	0.232464929859719\\
4.00000000000632	0.216432865731463\\
4.00000000000653	0.200400801603206\\
4.00000000000676	0.18436873747495\\
4.00000000000699	0.168336673346693\\
4.00000000000722	0.152304609218437\\
4.00000000000747	0.13627254509018\\
4.00000000000772	0.120240480961924\\
4.00000000000798	0.104208416833667\\
4.00000000000824	0.0881763527054105\\
4.00000000000852	0.0721442885771539\\
4.0000000000088	0.0561122244488974\\
4.00000000000909	0.0400801603206409\\
4.00000000000939	0.0240480961923843\\
4.00000000000969	0.00801603206412782\\
4.00000000001001	-0.00801603206412826\\
4.00000000001033	-0.0240480961923848\\
4.00000000001067	-0.0400801603206413\\
4.00000000001101	-0.0561122244488979\\
4.00000000001137	-0.0721442885771544\\
4.00000000001173	-0.0881763527054109\\
4.0000000000121	-0.104208416833667\\
4.00000000001248	-0.120240480961924\\
4.00000000001288	-0.13627254509018\\
4.00000000001328	-0.152304609218437\\
4.0000000000137	-0.168336673346694\\
4.00000000001412	-0.18436873747495\\
4.00000000001456	-0.200400801603207\\
4.00000000001501	-0.216432865731463\\
4.00000000001547	-0.232464929859719\\
4.00000000001594	-0.248496993987976\\
4.00000000001643	-0.264529058116232\\
4.00000000001693	-0.280561122244489\\
4.00000000001744	-0.296593186372745\\
4.00000000001796	-0.312625250501002\\
4.0000000000185	-0.328657314629258\\
4.00000000001905	-0.344689378757515\\
4.00000000001962	-0.360721442885771\\
4.0000000000202	-0.376753507014028\\
4.00000000002079	-0.392785571142285\\
4.0000000000214	-0.408817635270541\\
4.00000000002202	-0.424849699398798\\
4.00000000002266	-0.440881763527054\\
4.00000000002332	-0.456913827655311\\
4.00000000002399	-0.472945891783567\\
4.00000000002467	-0.488977955911824\\
4.00000000002538	-0.50501002004008\\
4.0000000000261	-0.521042084168337\\
4.00000000002683	-0.537074148296593\\
4.00000000002759	-0.55310621242485\\
4.00000000002836	-0.569138276553106\\
4.00000000002914	-0.585170340681363\\
4.00000000002995	-0.601202404809619\\
4.00000000003078	-0.617234468937876\\
4.00000000003162	-0.633266533066132\\
4.00000000003248	-0.649298597194389\\
4.00000000003337	-0.665330661322646\\
4.00000000003427	-0.681362725450902\\
4.00000000003519	-0.697394789579158\\
4.00000000003613	-0.713426853707415\\
4.00000000003709	-0.729458917835671\\
4.00000000003808	-0.745490981963928\\
4.00000000003908	-0.761523046092184\\
4.00000000004011	-0.777555110220441\\
4.00000000004115	-0.793587174348697\\
4.00000000004222	-0.809619238476954\\
4.00000000004331	-0.82565130260521\\
4.00000000004442	-0.841683366733467\\
4.00000000004556	-0.857715430861723\\
4.00000000004672	-0.87374749498998\\
4.0000000000479	-0.889779559118236\\
4.00000000004911	-0.905811623246493\\
4.00000000005034	-0.92184368737475\\
4.00000000005159	-0.937875751503006\\
4.00000000005287	-0.953907815631263\\
4.00000000005418	-0.969939879759519\\
4.0000000000555	-0.985971943887776\\
4.00000000005686	-1.00200400801603\\
4.00000000005824	-1.01803607214429\\
4.00000000005964	-1.03406813627255\\
4.00000000006107	-1.0501002004008\\
4.00000000006253	-1.06613226452906\\
4.00000000006401	-1.08216432865731\\
4.00000000006552	-1.09819639278557\\
4.00000000006706	-1.11422845691383\\
4.00000000006863	-1.13026052104208\\
4.00000000007022	-1.14629258517034\\
4.00000000007184	-1.1623246492986\\
4.00000000007349	-1.17835671342685\\
4.00000000007516	-1.19438877755511\\
4.00000000007687	-1.21042084168337\\
4.0000000000786	-1.22645290581162\\
4.00000000008036	-1.24248496993988\\
4.00000000008216	-1.25851703406814\\
4.00000000008398	-1.27454909819639\\
4.00000000008583	-1.29058116232465\\
4.0000000000877	-1.30661322645291\\
4.00000000008961	-1.32264529058116\\
4.00000000009155	-1.33867735470942\\
4.00000000009352	-1.35470941883768\\
4.00000000009552	-1.37074148296593\\
4.00000000009755	-1.38677354709419\\
4.0000000000996	-1.40280561122244\\
4.00000000010169	-1.4188376753507\\
4.00000000010381	-1.43486973947896\\
4.00000000010596	-1.45090180360721\\
4.00000000010814	-1.46693386773547\\
4.00000000011035	-1.48296593186373\\
4.0000000001126	-1.49899799599198\\
4.00000000011487	-1.51503006012024\\
4.00000000011717	-1.5310621242485\\
4.00000000011951	-1.54709418837675\\
4.00000000012187	-1.56312625250501\\
4.00000000012427	-1.57915831663327\\
4.00000000012669	-1.59519038076152\\
4.00000000012915	-1.61122244488978\\
4.00000000013164	-1.62725450901804\\
4.00000000013416	-1.64328657314629\\
4.00000000013671	-1.65931863727455\\
4.00000000013929	-1.67535070140281\\
4.0000000001419	-1.69138276553106\\
4.00000000014454	-1.70741482965932\\
4.00000000014721	-1.72344689378758\\
4.00000000014991	-1.73947895791583\\
4.00000000015265	-1.75551102204409\\
4.00000000015541	-1.77154308617234\\
4.0000000001582	-1.7875751503006\\
4.00000000016102	-1.80360721442886\\
4.00000000016387	-1.81963927855711\\
4.00000000016675	-1.83567134268537\\
4.00000000016965	-1.85170340681363\\
4.00000000017259	-1.86773547094188\\
4.00000000017555	-1.88376753507014\\
4.00000000017854	-1.8997995991984\\
4.00000000018156	-1.91583166332665\\
4.00000000018461	-1.93186372745491\\
4.00000000018768	-1.94789579158317\\
4.00000000019079	-1.96392785571142\\
4.00000000019391	-1.97995991983968\\
4.00000000019707	-1.99599198396794\\
4.00000000020024	-2.01202404809619\\
4.00000000020345	-2.02805611222445\\
4.00000000020668	-2.04408817635271\\
4.00000000020993	-2.06012024048096\\
4.0000000002132	-2.07615230460922\\
4.0000000002165	-2.09218436873747\\
4.00000000021983	-2.10821643286573\\
4.00000000022317	-2.12424849699399\\
4.00000000022654	-2.14028056112224\\
4.00000000022992	-2.1563126252505\\
4.00000000023333	-2.17234468937876\\
4.00000000023676	-2.18837675350701\\
4.00000000024021	-2.20440881763527\\
4.00000000024368	-2.22044088176353\\
4.00000000024716	-2.23647294589178\\
4.00000000025066	-2.25250501002004\\
4.00000000025418	-2.2685370741483\\
4.00000000025772	-2.28456913827655\\
4.00000000026127	-2.30060120240481\\
4.00000000026484	-2.31663326653307\\
4.00000000026842	-2.33266533066132\\
4.00000000027201	-2.34869739478958\\
4.00000000027562	-2.36472945891784\\
4.00000000027924	-2.38076152304609\\
4.00000000028287	-2.39679358717435\\
4.00000000028651	-2.41282565130261\\
4.00000000029016	-2.42885771543086\\
4.00000000029382	-2.44488977955912\\
4.00000000029748	-2.46092184368737\\
4.00000000030116	-2.47695390781563\\
4.00000000030484	-2.49298597194389\\
4.00000000030852	-2.50901803607214\\
4.00000000031221	-2.5250501002004\\
4.0000000003159	-2.54108216432866\\
4.0000000003196	-2.55711422845691\\
4.0000000003233	-2.57314629258517\\
4.00000000032699	-2.58917835671343\\
4.00000000033069	-2.60521042084168\\
4.00000000033439	-2.62124248496994\\
4.00000000033808	-2.6372745490982\\
4.00000000034177	-2.65330661322645\\
4.00000000034546	-2.66933867735471\\
4.00000000034914	-2.68537074148297\\
4.00000000035282	-2.70140280561122\\
4.00000000035649	-2.71743486973948\\
4.00000000036015	-2.73346693386774\\
4.0000000003638	-2.74949899799599\\
4.00000000036744	-2.76553106212425\\
4.00000000037107	-2.7815631262525\\
4.00000000037469	-2.79759519038076\\
4.0000000003783	-2.81362725450902\\
4.00000000038189	-2.82965931863727\\
4.00000000038546	-2.84569138276553\\
4.00000000038902	-2.86172344689379\\
4.00000000039256	-2.87775551102204\\
4.00000000039608	-2.8937875751503\\
4.00000000039958	-2.90981963927856\\
4.00000000040307	-2.92585170340681\\
4.00000000040652	-2.94188376753507\\
4.00000000040996	-2.95791583166333\\
4.00000000041337	-2.97394789579158\\
4.00000000041676	-2.98997995991984\\
4.00000000042012	-3.0060120240481\\
4.00000000042346	-3.02204408817635\\
4.00000000042676	-3.03807615230461\\
4.00000000043004	-3.05410821643287\\
4.00000000043328	-3.07014028056112\\
4.0000000004365	-3.08617234468938\\
4.00000000043968	-3.10220440881764\\
4.00000000044282	-3.11823647294589\\
4.00000000044594	-3.13426853707415\\
4.00000000044901	-3.1503006012024\\
4.00000000045205	-3.16633266533066\\
4.00000000045505	-3.18236472945892\\
4.00000000045802	-3.19839679358717\\
4.00000000046094	-3.21442885771543\\
4.00000000046382	-3.23046092184369\\
4.00000000046666	-3.24649298597194\\
4.00000000046946	-3.2625250501002\\
4.00000000047221	-3.27855711422846\\
4.00000000047492	-3.29458917835671\\
4.00000000047758	-3.31062124248497\\
4.0000000004802	-3.32665330661323\\
4.00000000048277	-3.34268537074148\\
4.00000000048529	-3.35871743486974\\
4.00000000048776	-3.374749498998\\
4.00000000049018	-3.39078156312625\\
4.00000000049254	-3.40681362725451\\
4.00000000049486	-3.42284569138277\\
4.00000000049712	-3.43887775551102\\
4.00000000049933	-3.45490981963928\\
4.00000000050148	-3.47094188376753\\
4.00000000050358	-3.48697394789579\\
4.00000000050563	-3.50300601202405\\
4.00000000050761	-3.5190380761523\\
4.00000000050954	-3.53507014028056\\
4.00000000051141	-3.55110220440882\\
4.00000000051322	-3.56713426853707\\
4.00000000051497	-3.58316633266533\\
4.00000000051666	-3.59919839679359\\
4.00000000051829	-3.61523046092184\\
4.00000000051986	-3.6312625250501\\
4.00000000052136	-3.64729458917836\\
4.00000000052281	-3.66332665330661\\
4.00000000052419	-3.67935871743487\\
4.0000000005255	-3.69539078156313\\
4.00000000052675	-3.71142284569138\\
4.00000000052794	-3.72745490981964\\
4.00000000052906	-3.7434869739479\\
4.00000000053011	-3.75951903807615\\
4.0000000005311	-3.77555110220441\\
4.00000000053202	-3.79158316633267\\
4.00000000053288	-3.80761523046092\\
4.00000000053367	-3.82364729458918\\
4.00000000053439	-3.83967935871743\\
4.00000000053504	-3.85571142284569\\
4.00000000053563	-3.87174348697395\\
4.00000000053614	-3.8877755511022\\
4.00000000053659	-3.90380761523046\\
4.00000000053697	-3.91983967935872\\
4.00000000053728	-3.93587174348697\\
4.00000000053752	-3.95190380761523\\
4.00000000053769	-3.96793587174349\\
4.0000000005378	-3.98396793587174\\
4.00000000053783	-4\\
4	-4.00000000053783\\
3.98396793587174	-4.00000000057334\\
3.96793587174349	-4.00000000061096\\
3.95190380761523	-4.0000000006508\\
3.93587174348697	-4.00000000069296\\
3.91983967935872	-4.00000000073758\\
3.90380761523046	-4.00000000078476\\
3.8877755511022	-4.00000000083464\\
3.87174348697395	-4.00000000088735\\
3.85571142284569	-4.00000000094302\\
3.83967935871743	-4.0000000010018\\
3.82364729458918	-4.00000000106383\\
3.80761523046092	-4.00000000112927\\
3.79158316633267	-4.00000000119827\\
3.77555110220441	-4.000000001271\\
3.75951903807615	-4.00000000134762\\
3.7434869739479	-4.00000000142832\\
3.72745490981964	-4.00000000151325\\
3.71142284569138	-4.00000000160263\\
3.69539078156313	-4.00000000169662\\
3.67935871743487	-4.00000000179544\\
3.66332665330661	-4.00000000189928\\
3.64729458917836	-4.00000000200835\\
3.6312625250501	-4.00000000212287\\
3.61523046092184	-4.00000000224305\\
3.59919839679359	-4.00000000236912\\
3.58316633266533	-4.00000000250131\\
3.56713426853707	-4.00000000263986\\
3.55110220440882	-4.00000000278501\\
3.53507014028056	-4.00000000293701\\
3.51903807615231	-4.00000000309611\\
3.50300601202405	-4.00000000326257\\
3.48697394789579	-4.00000000343666\\
3.47094188376754	-4.00000000361864\\
3.45490981963928	-4.00000000380879\\
3.43887775551102	-4.00000000400738\\
3.42284569138277	-4.00000000421471\\
3.40681362725451	-4.00000000443105\\
3.39078156312625	-4.0000000046567\\
3.374749498998	-4.00000000489195\\
3.35871743486974	-4.00000000513711\\
3.34268537074148	-4.00000000539247\\
3.32665330661323	-4.00000000565835\\
3.31062124248497	-4.00000000593505\\
3.29458917835671	-4.00000000622288\\
3.27855711422846	-4.00000000652215\\
3.2625250501002	-4.00000000683318\\
3.24649298597194	-4.00000000715629\\
3.23046092184369	-4.00000000749178\\
3.21442885771543	-4.00000000783997\\
3.19839679358717	-4.00000000820119\\
3.18236472945892	-4.00000000857575\\
3.16633266533066	-4.00000000896395\\
3.1503006012024	-4.00000000936611\\
3.13426853707415	-4.00000000978255\\
3.11823647294589	-4.00000001021356\\
3.10220440881764	-4.00000001065945\\
3.08617234468938	-4.00000001112052\\
3.07014028056112	-4.00000001159706\\
3.05410821643287	-4.00000001208936\\
3.03807615230461	-4.0000000125977\\
3.02204408817635	-4.00000001312236\\
3.0060120240481	-4.0000000136636\\
2.98997995991984	-4.00000001422167\\
2.97394789579158	-4.00000001479684\\
2.95791583166333	-4.00000001538933\\
2.94188376753507	-4.00000001599937\\
2.92585170340681	-4.00000001662719\\
2.90981963927856	-4.00000001727298\\
2.8937875751503	-4.00000001793694\\
2.87775551102204	-4.00000001861924\\
2.86172344689379	-4.00000001932004\\
2.84569138276553	-4.00000002003949\\
2.82965931863727	-4.00000002077772\\
2.81362725450902	-4.00000002153485\\
2.79759519038076	-4.00000002231096\\
2.7815631262525	-4.00000002310613\\
2.76553106212425	-4.00000002392041\\
2.74949899799599	-4.00000002475385\\
2.73346693386774	-4.00000002560645\\
2.71743486973948	-4.0000000264782\\
2.70140280561122	-4.00000002736909\\
2.68537074148297	-4.00000002827904\\
2.66933867735471	-4.00000002920798\\
2.65330661322645	-4.0000000301558\\
2.6372745490982	-4.00000003112239\\
2.62124248496994	-4.00000003210757\\
2.60521042084168	-4.00000003311117\\
2.58917835671343	-4.00000003413298\\
2.57314629258517	-4.00000003517276\\
2.55711422845691	-4.00000003623025\\
2.54108216432866	-4.00000003730514\\
2.5250501002004	-4.00000003839712\\
2.50901803607214	-4.00000003950582\\
2.49298597194389	-4.00000004063087\\
2.47695390781563	-4.00000004177185\\
2.46092184368737	-4.00000004292832\\
2.44488977955912	-4.0000000440998\\
2.42885771543086	-4.00000004528579\\
2.41282565130261	-4.00000004648575\\
2.39679358717435	-4.0000000476991\\
2.38076152304609	-4.00000004892527\\
2.36472945891784	-4.0000000501636\\
2.34869739478958	-4.00000005141346\\
2.33266533066132	-4.00000005267415\\
2.31663326653307	-4.00000005394494\\
2.30060120240481	-4.0000000552251\\
2.28456913827655	-4.00000005651385\\
2.2685370741483	-4.00000005781037\\
2.25250501002004	-4.00000005911385\\
2.23647294589178	-4.00000006042342\\
2.22044088176353	-4.00000006173819\\
2.20440881763527	-4.00000006305725\\
2.18837675350701	-4.00000006437967\\
2.17234468937876	-4.00000006570449\\
2.1563126252505	-4.00000006703072\\
2.14028056112224	-4.00000006835736\\
2.12424849699399	-4.00000006968339\\
2.10821643286573	-4.00000007100775\\
2.09218436873747	-4.0000000723294\\
2.07615230460922	-4.00000007364725\\
2.06012024048096	-4.0000000749602\\
2.04408817635271	-4.00000007626715\\
2.02805611222445	-4.00000007756697\\
2.01202404809619	-4.00000007885854\\
1.99599198396794	-4.00000008014071\\
1.97995991983968	-4.00000008141234\\
1.96392785571142	-4.00000008267226\\
1.94789579158317	-4.00000008391932\\
1.93186372745491	-4.00000008515236\\
1.91583166332665	-4.0000000863702\\
1.8997995991984	-4.0000000875717\\
1.88376753507014	-4.00000008875568\\
1.86773547094188	-4.000000089921\\
1.85170340681363	-4.0000000910665\\
1.83567134268537	-4.00000009219105\\
1.81963927855711	-4.0000000932935\\
1.80360721442886	-4.00000009437275\\
1.7875751503006	-4.00000009542768\\
1.77154308617235	-4.00000009645721\\
1.75551102204409	-4.00000009746027\\
1.73947895791583	-4.0000000984358\\
1.72344689378758	-4.00000009938277\\
1.70741482965932	-4.00000010030017\\
1.69138276553106	-4.00000010118702\\
1.67535070140281	-4.00000010204237\\
1.65931863727455	-4.00000010286528\\
1.64328657314629	-4.00000010365485\\
1.62725450901804	-4.00000010441023\\
1.61122244488978	-4.00000010513056\\
1.59519038076152	-4.00000010581507\\
1.57915831663327	-4.00000010646298\\
1.56312625250501	-4.00000010707356\\
1.54709418837675	-4.00000010764614\\
1.5310621242485	-4.00000010818007\\
1.51503006012024	-4.00000010867473\\
1.49899799599198	-4.00000010912958\\
1.48296593186373	-4.00000010954409\\
1.46693386773547	-4.00000010991779\\
1.45090180360721	-4.00000011025025\\
1.43486973947896	-4.00000011054108\\
1.4188376753507	-4.00000011078997\\
1.40280561122244	-4.00000011099661\\
1.38677354709419	-4.00000011116077\\
1.37074148296593	-4.00000011128227\\
1.35470941883768	-4.00000011136095\\
1.33867735470942	-4.00000011139673\\
1.32264529058116	-4.00000011138958\\
1.30661322645291	-4.00000011133948\\
1.29058116232465	-4.00000011124652\\
1.27454909819639	-4.00000011111078\\
1.25851703406814	-4.00000011093244\\
1.24248496993988	-4.00000011071169\\
1.22645290581162	-4.00000011044878\\
1.21042084168337	-4.00000011014403\\
1.19438877755511	-4.00000010979778\\
1.17835671342685	-4.00000010941043\\
1.1623246492986	-4.00000010898242\\
1.14629258517034	-4.00000010851424\\
1.13026052104208	-4.00000010800642\\
1.11422845691383	-4.00000010745955\\
1.09819639278557	-4.00000010687422\\
1.08216432865731	-4.00000010625112\\
1.06613226452906	-4.00000010559093\\
1.0501002004008	-4.00000010489439\\
1.03406813627254	-4.00000010416229\\
1.01803607214429	-4.00000010339542\\
1.00200400801603	-4.00000010259463\\
0.985971943887775	-4.00000010176081\\
0.969939879759519	-4.00000010089486\\
0.953907815631262	-4.00000009999772\\
0.937875751503006	-4.00000009907035\\
0.921843687374749	-4.00000009811374\\
0.905811623246493	-4.00000009712892\\
0.889779559118236	-4.00000009611692\\
0.87374749498998	-4.00000009507881\\
0.857715430861724	-4.00000009401565\\
0.841683366733467	-4.00000009292854\\
0.825651302605211	-4.0000000918186\\
0.809619238476954	-4.00000009068694\\
0.793587174348698	-4.00000008953471\\
0.777555110220441	-4.00000008836304\\
0.761523046092185	-4.00000008717309\\
0.745490981963928	-4.00000008596601\\
0.729458917835672	-4.00000008474298\\
0.713426853707415	-4.00000008350514\\
0.697394789579159	-4.00000008225366\\
0.681362725450902	-4.00000008098971\\
0.665330661322646	-4.00000007971444\\
0.649298597194389	-4.00000007842901\\
0.633266533066132	-4.00000007713456\\
0.617234468937876	-4.00000007583223\\
0.601202404809619	-4.00000007452316\\
0.585170340681363	-4.00000007320845\\
0.569138276553106	-4.00000007188922\\
0.55310621242485	-4.00000007056655\\
0.537074148296593	-4.00000006924151\\
0.521042084168337	-4.00000006791516\\
0.50501002004008	-4.00000006658855\\
0.488977955911824	-4.00000006526268\\
0.472945891783567	-4.00000006393855\\
0.456913827655311	-4.00000006261714\\
0.440881763527054	-4.00000006129941\\
0.424849699398798	-4.00000005998627\\
0.408817635270541	-4.00000005867864\\
0.392785571142285	-4.00000005737738\\
0.376753507014028	-4.00000005608336\\
0.360721442885771	-4.00000005479739\\
0.344689378757515	-4.00000005352027\\
0.328657314629258	-4.00000005225276\\
0.312625250501002	-4.00000005099561\\
0.296593186372745	-4.00000004974951\\
0.280561122244489	-4.00000004851516\\
0.264529058116232	-4.0000000472932\\
0.248496993987976	-4.00000004608424\\
0.232464929859719	-4.00000004488888\\
0.216432865731463	-4.00000004370767\\
0.200400801603206	-4.00000004254114\\
0.18436873747495	-4.00000004138978\\
0.168336673346693	-4.00000004025406\\
0.152304609218437	-4.00000003913442\\
0.13627254509018	-4.00000003803125\\
0.120240480961924	-4.00000003694493\\
0.104208416833667	-4.0000000358758\\
0.0881763527054105	-4.00000003482419\\
0.0721442885771539	-4.00000003379037\\
0.0561122244488974	-4.00000003277461\\
0.0400801603206409	-4.00000003177712\\
0.0240480961923843	-4.00000003079812\\
0.00801603206412782	-4.00000002983777\\
-0.00801603206412826	-4.00000002889623\\
-0.0240480961923848	-4.0000000279736\\
-0.0400801603206413	-4.00000002707\\
-0.0561122244488979	-4.00000002618549\\
-0.0721442885771544	-4.00000002532012\\
-0.0881763527054109	-4.00000002447391\\
-0.104208416833667	-4.00000002364686\\
-0.120240480961924	-4.00000002283895\\
-0.13627254509018	-4.00000002205014\\
-0.152304609218437	-4.00000002128037\\
-0.168336673346694	-4.00000002052955\\
-0.18436873747495	-4.00000001979759\\
-0.200400801603207	-4.00000001908437\\
-0.216432865731463	-4.00000001838976\\
-0.232464929859719	-4.00000001771359\\
-0.248496993987976	-4.00000001705571\\
-0.264529058116232	-4.00000001641593\\
-0.280561122244489	-4.00000001579406\\
-0.296593186372745	-4.00000001518989\\
-0.312625250501002	-4.0000000146032\\
-0.328657314629258	-4.00000001403376\\
-0.344689378757515	-4.00000001348132\\
-0.360721442885771	-4.00000001294564\\
-0.376753507014028	-4.00000001242646\\
-0.392785571142285	-4.00000001192349\\
-0.408817635270541	-4.00000001143648\\
-0.424849699398798	-4.00000001096513\\
-0.440881763527054	-4.00000001050915\\
-0.456913827655311	-4.00000001006825\\
-0.472945891783567	-4.00000000964213\\
-0.488977955911824	-4.00000000923049\\
-0.50501002004008	-4.00000000883301\\
-0.521042084168337	-4.00000000844939\\
-0.537074148296593	-4.00000000807932\\
-0.55310621242485	-4.00000000772248\\
-0.569138276553106	-4.00000000737855\\
-0.585170340681363	-4.00000000704722\\
-0.601202404809619	-4.00000000672818\\
-0.617234468937876	-4.0000000064211\\
-0.633266533066132	-4.00000000612568\\
-0.649298597194389	-4.0000000058416\\
-0.665330661322646	-4.00000000556854\\
-0.681362725450902	-4.0000000053062\\
-0.697394789579158	-4.00000000505427\\
-0.713426853707415	-4.00000000481245\\
-0.729458917835671	-4.00000000458043\\
-0.745490981963928	-4.00000000435791\\
-0.761523046092184	-4.00000000414461\\
-0.777555110220441	-4.00000000394023\\
-0.793587174348697	-4.00000000374448\\
-0.809619238476954	-4.00000000355709\\
-0.82565130260521	-4.00000000337777\\
-0.841683366733467	-4.00000000320625\\
-0.857715430861723	-4.00000000304227\\
-0.87374749498998	-4.00000000288557\\
-0.889779559118236	-4.00000000273588\\
-0.905811623246493	-4.00000000259296\\
-0.92184368737475	-4.00000000245655\\
-0.937875751503006	-4.00000000232643\\
-0.953907815631263	-4.00000000220234\\
-0.969939879759519	-4.00000000208408\\
-0.985971943887776	-4.0000000019714\\
-1.00200400801603	-4.0000000018641\\
-1.01803607214429	-4.00000000176195\\
-1.03406813627255	-4.00000000166477\\
-1.0501002004008	-4.00000000157233\\
-1.06613226452906	-4.00000000148446\\
-1.08216432865731	-4.00000000140096\\
-1.09819639278557	-4.00000000132164\\
-1.11422845691383	-4.00000000124633\\
-1.13026052104208	-4.00000000117487\\
-1.14629258517034	-4.00000000110707\\
-1.1623246492986	-4.00000000104278\\
-1.17835671342685	-4.00000000098185\\
-1.19438877755511	-4.00000000092412\\
-1.21042084168337	-4.00000000086946\\
-1.22645290581162	-4.00000000081771\\
-1.24248496993988	-4.00000000076874\\
-1.25851703406814	-4.00000000072243\\
-1.27454909819639	-4.00000000067864\\
-1.29058116232465	-4.00000000063727\\
-1.30661322645291	-4.00000000059818\\
-1.32264529058116	-4.00000000056128\\
-1.33867735470942	-4.00000000052645\\
-1.35470941883768	-4.00000000049359\\
-1.37074148296593	-4.0000000004626\\
-1.38677354709419	-4.00000000043339\\
-1.40280561122244	-4.00000000040587\\
-1.4188376753507	-4.00000000037995\\
-1.43486973947896	-4.00000000035555\\
-1.45090180360721	-4.00000000033259\\
-1.46693386773547	-4.00000000031099\\
-1.48296593186373	-4.00000000029068\\
-1.49899799599198	-4.00000000027159\\
-1.51503006012024	-4.00000000025366\\
-1.5310621242485	-4.00000000023682\\
-1.54709418837675	-4.00000000022102\\
-1.56312625250501	-4.00000000020618\\
-1.57915831663327	-4.00000000019227\\
-1.59519038076152	-4.00000000017923\\
-1.61122244488978	-4.00000000016701\\
-1.62725450901804	-4.00000000015557\\
-1.64328657314629	-4.00000000014485\\
-1.65931863727455	-4.00000000013482\\
-1.67535070140281	-4.00000000012543\\
-1.69138276553106	-4.00000000011665\\
-1.70741482965932	-4.00000000010845\\
-1.72344689378758	-4.00000000010078\\
-1.73947895791583	-4.00000000009362\\
-1.75551102204409	-4.00000000008693\\
-1.77154308617234	-4.0000000000807\\
-1.7875751503006	-4.00000000007488\\
-1.80360721442886	-4.00000000006945\\
-1.81963927855711	-4.00000000006439\\
-1.83567134268537	-4.00000000005968\\
-1.85170340681363	-4.00000000005529\\
-1.86773547094188	-4.0000000000512\\
-1.88376753507014	-4.0000000000474\\
-1.8997995991984	-4.00000000004386\\
-1.91583166332665	-4.00000000004057\\
-1.93186372745491	-4.00000000003751\\
-1.94789579158317	-4.00000000003468\\
-1.96392785571142	-4.00000000003204\\
-1.97995991983968	-4.00000000002959\\
-1.99599198396794	-4.00000000002732\\
-2.01202404809619	-4.00000000002521\\
-2.02805611222445	-4.00000000002326\\
-2.04408817635271	-4.00000000002145\\
-2.06012024048096	-4.00000000001977\\
-2.07615230460922	-4.00000000001822\\
-2.09218436873747	-4.00000000001678\\
-2.10821643286573	-4.00000000001545\\
-2.12424849699399	-4.00000000001422\\
-2.14028056112224	-4.00000000001308\\
-2.1563126252505	-4.00000000001203\\
-2.17234468937876	-4.00000000001106\\
-2.18837675350701	-4.00000000001017\\
-2.20440881763527	-4.00000000000934\\
-2.22044088176353	-4.00000000000858\\
-2.23647294589178	-4.00000000000787\\
-2.25250501002004	-4.00000000000722\\
-2.2685370741483	-4.00000000000662\\
-2.28456913827655	-4.00000000000607\\
-2.30060120240481	-4.00000000000557\\
-2.31663326653307	-4.0000000000051\\
-2.33266533066132	-4.00000000000467\\
-2.34869739478958	-4.00000000000428\\
-2.36472945891784	-4.00000000000391\\
-2.38076152304609	-4.00000000000358\\
-2.39679358717435	-4.00000000000327\\
-2.41282565130261	-4.00000000000299\\
-2.42885771543086	-4.00000000000273\\
-2.44488977955912	-4.0000000000025\\
-2.46092184368737	-4.00000000000228\\
-2.47695390781563	-4.00000000000208\\
-2.49298597194389	-4.0000000000019\\
-2.50901803607214	-4.00000000000173\\
-2.5250501002004	-4.00000000000158\\
-2.54108216432866	-4.00000000000144\\
-2.55711422845691	-4.00000000000131\\
-2.57314629258517	-4.00000000000119\\
-2.58917835671343	-4.00000000000108\\
-2.60521042084168	-4.00000000000099\\
-2.62124248496994	-4.0000000000009\\
-2.6372745490982	-4.00000000000082\\
-2.65330661322645	-4.00000000000074\\
-2.66933867735471	-4.00000000000067\\
-2.68537074148297	-4.00000000000061\\
-2.70140280561122	-4.00000000000056\\
-2.71743486973948	-4.0000000000005\\
-2.73346693386774	-4.00000000000046\\
-2.74949899799599	-4.00000000000041\\
-2.76553106212425	-4.00000000000038\\
-2.7815631262525	-4.00000000000034\\
-2.79759519038076	-4.00000000000031\\
-2.81362725450902	-4.00000000000028\\
-2.82965931863727	-4.00000000000025\\
-2.84569138276553	-4.00000000000023\\
-2.86172344689379	-4.00000000000021\\
-2.87775551102204	-4.00000000000019\\
-2.8937875751503	-4.00000000000017\\
-2.90981963927856	-4.00000000000015\\
-2.92585170340681	-4.00000000000014\\
-2.94188376753507	-4.00000000000012\\
-2.95791583166333	-4.00000000000011\\
-2.97394789579158	-4.0000000000001\\
-2.98997995991984	-4.00000000000009\\
-3.0060120240481	-4.00000000000008\\
-3.02204408817635	-4.00000000000007\\
-3.03807615230461	-4.00000000000007\\
-3.05410821643287	-4.00000000000006\\
-3.07014028056112	-4.00000000000005\\
-3.08617234468938	-4.00000000000005\\
-3.10220440881764	-4.00000000000004\\
-3.11823647294589	-4.00000000000004\\
-3.13426853707415	-4.00000000000003\\
-3.1503006012024	-4.00000000000003\\
-3.16633266533066	-4.00000000000003\\
-3.18236472945892	-4.00000000000003\\
-3.19839679358717	-4.00000000000002\\
-3.21442885771543	-4.00000000000002\\
-3.23046092184369	-4.00000000000002\\
-3.24649298597194	-4.00000000000002\\
-3.2625250501002	-4.00000000000001\\
-3.27855711422846	-4.00000000000001\\
-3.29458917835671	-4.00000000000001\\
-3.31062124248497	-4.00000000000001\\
-3.32665330661323	-4.00000000000001\\
-3.34268537074148	-4.00000000000001\\
-3.35871743486974	-4.00000000000001\\
-3.374749498998	-4.00000000000001\\
-3.39078156312625	-4.00000000000001\\
-3.40681362725451	-4.00000000000001\\
-3.42284569138277	-4\\
-3.43887775551102	-4\\
-3.45490981963928	-4\\
-3.47094188376753	-4\\
-3.48697394789579	-4\\
-3.50300601202405	-4\\
-3.5190380761523	-4\\
-3.53507014028056	-4\\
-3.55110220440882	-4\\
-3.56713426853707	-4\\
-3.58316633266533	-4\\
-3.59919839679359	-4\\
-3.61523046092184	-4\\
-3.6312625250501	-4\\
-3.64729458917836	-4\\
-3.66332665330661	-4\\
-3.67935871743487	-4\\
-3.69539078156313	-4\\
-3.71142284569138	-4\\
-3.72745490981964	-4\\
-3.7434869739479	-4\\
-3.75951903807615	-4\\
-3.77555110220441	-4\\
-3.79158316633267	-4\\
-3.80761523046092	-4\\
-3.82364729458918	-4\\
-3.83967935871743	-4\\
-3.85571142284569	-4\\
-3.87174348697395	-4\\
-3.8877755511022	-4\\
-3.90380761523046	-4\\
-3.91983967935872	-4\\
-3.93587174348697	-4\\
-3.95190380761523	-4\\
-3.96793587174349	-4\\
-3.98396793587174	-4\\
-4	-4\\
-4	-3.98396793587174\\
-4	-3.96793587174349\\
-4	-3.95190380761523\\
-4	-3.93587174348697\\
-4	-3.91983967935872\\
-4	-3.90380761523046\\
-4	-3.8877755511022\\
-4	-3.87174348697395\\
-4	-3.85571142284569\\
-4	-3.83967935871743\\
-4	-3.82364729458918\\
-4	-3.80761523046092\\
-4	-3.79158316633267\\
-4	-3.77555110220441\\
-4	-3.75951903807615\\
-4	-3.7434869739479\\
-4	-3.72745490981964\\
-4	-3.71142284569138\\
-4	-3.69539078156313\\
-4	-3.67935871743487\\
-4	-3.66332665330661\\
-4	-3.64729458917836\\
-4	-3.6312625250501\\
-4	-3.61523046092184\\
-4	-3.59919839679359\\
-4	-3.58316633266533\\
-4	-3.56713426853707\\
-4	-3.55110220440882\\
-4	-3.53507014028056\\
-4	-3.5190380761523\\
-4	-3.50300601202405\\
-4	-3.48697394789579\\
-4	-3.47094188376753\\
-4	-3.45490981963928\\
-4	-3.43887775551102\\
-4	-3.42284569138277\\
-4	-3.40681362725451\\
-4	-3.39078156312625\\
-4	-3.374749498998\\
-4	-3.35871743486974\\
-4	-3.34268537074148\\
-4	-3.32665330661323\\
-4	-3.31062124248497\\
-4	-3.29458917835671\\
-4	-3.27855711422846\\
-4	-3.2625250501002\\
-4	-3.24649298597194\\
-4	-3.23046092184369\\
-4	-3.21442885771543\\
-4	-3.19839679358717\\
-4	-3.18236472945892\\
-4	-3.16633266533066\\
-4	-3.1503006012024\\
-4	-3.13426853707415\\
-4	-3.11823647294589\\
-4	-3.10220440881764\\
-4	-3.08617234468938\\
-4	-3.07014028056112\\
-4	-3.05410821643287\\
-4	-3.03807615230461\\
-4	-3.02204408817635\\
-4	-3.0060120240481\\
-4	-2.98997995991984\\
-4	-2.97394789579158\\
-4	-2.95791583166333\\
-4	-2.94188376753507\\
-4	-2.92585170340681\\
-4	-2.90981963927856\\
-4	-2.8937875751503\\
-4	-2.87775551102204\\
-4	-2.86172344689379\\
-4	-2.84569138276553\\
-4	-2.82965931863727\\
-4	-2.81362725450902\\
-4.00000000000001	-2.79759519038076\\
-4.00000000000001	-2.7815631262525\\
-4.00000000000001	-2.76553106212425\\
-4.00000000000001	-2.74949899799599\\
-4.00000000000001	-2.73346693386774\\
-4.00000000000001	-2.71743486973948\\
-4.00000000000001	-2.70140280561122\\
-4.00000000000001	-2.68537074148297\\
-4.00000000000001	-2.66933867735471\\
-4.00000000000001	-2.65330661322645\\
-4.00000000000001	-2.6372745490982\\
-4.00000000000001	-2.62124248496994\\
-4.00000000000001	-2.60521042084168\\
-4.00000000000001	-2.58917835671343\\
-4.00000000000001	-2.57314629258517\\
-4.00000000000001	-2.55711422845691\\
-4.00000000000001	-2.54108216432866\\
-4.00000000000001	-2.5250501002004\\
-4.00000000000001	-2.50901803607214\\
-4.00000000000001	-2.49298597194389\\
-4.00000000000002	-2.47695390781563\\
-4.00000000000002	-2.46092184368737\\
-4.00000000000002	-2.44488977955912\\
-4.00000000000002	-2.42885771543086\\
-4.00000000000002	-2.41282565130261\\
-4.00000000000002	-2.39679358717435\\
-4.00000000000002	-2.38076152304609\\
-4.00000000000002	-2.36472945891784\\
-4.00000000000002	-2.34869739478958\\
-4.00000000000002	-2.33266533066132\\
-4.00000000000002	-2.31663326653307\\
-4.00000000000003	-2.30060120240481\\
-4.00000000000003	-2.28456913827655\\
-4.00000000000003	-2.2685370741483\\
-4.00000000000003	-2.25250501002004\\
-4.00000000000003	-2.23647294589178\\
-4.00000000000003	-2.22044088176353\\
-4.00000000000004	-2.20440881763527\\
-4.00000000000004	-2.18837675350701\\
-4.00000000000004	-2.17234468937876\\
-4.00000000000004	-2.1563126252505\\
-4.00000000000004	-2.14028056112224\\
-4.00000000000005	-2.12424849699399\\
-4.00000000000005	-2.10821643286573\\
-4.00000000000005	-2.09218436873747\\
-4.00000000000005	-2.07615230460922\\
-4.00000000000006	-2.06012024048096\\
-4.00000000000006	-2.04408817635271\\
-4.00000000000006	-2.02805611222445\\
-4.00000000000006	-2.01202404809619\\
-4.00000000000007	-1.99599198396794\\
-4.00000000000007	-1.97995991983968\\
-4.00000000000007	-1.96392785571142\\
-4.00000000000008	-1.94789579158317\\
-4.00000000000008	-1.93186372745491\\
-4.00000000000009	-1.91583166332665\\
-4.00000000000009	-1.8997995991984\\
-4.00000000000009	-1.88376753507014\\
-4.0000000000001	-1.86773547094188\\
-4.0000000000001	-1.85170340681363\\
-4.00000000000011	-1.83567134268537\\
-4.00000000000011	-1.81963927855711\\
-4.00000000000012	-1.80360721442886\\
-4.00000000000012	-1.7875751503006\\
-4.00000000000013	-1.77154308617234\\
-4.00000000000014	-1.75551102204409\\
-4.00000000000014	-1.73947895791583\\
-4.00000000000015	-1.72344689378758\\
-4.00000000000016	-1.70741482965932\\
-4.00000000000016	-1.69138276553106\\
-4.00000000000017	-1.67535070140281\\
-4.00000000000018	-1.65931863727455\\
-4.00000000000019	-1.64328657314629\\
-4.0000000000002	-1.62725450901804\\
-4.00000000000021	-1.61122244488978\\
-4.00000000000021	-1.59519038076152\\
-4.00000000000022	-1.57915831663327\\
-4.00000000000023	-1.56312625250501\\
-4.00000000000025	-1.54709418837675\\
-4.00000000000026	-1.5310621242485\\
-4.00000000000027	-1.51503006012024\\
-4.00000000000028	-1.49899799599198\\
-4.00000000000029	-1.48296593186373\\
-4.00000000000031	-1.46693386773547\\
-4.00000000000032	-1.45090180360721\\
-4.00000000000033	-1.43486973947896\\
-4.00000000000035	-1.4188376753507\\
-4.00000000000036	-1.40280561122244\\
-4.00000000000038	-1.38677354709419\\
-4.0000000000004	-1.37074148296593\\
-4.00000000000041	-1.35470941883768\\
-4.00000000000043	-1.33867735470942\\
-4.00000000000045	-1.32264529058116\\
-4.00000000000047	-1.30661322645291\\
-4.00000000000049	-1.29058116232465\\
-4.00000000000051	-1.27454909819639\\
-4.00000000000053	-1.25851703406814\\
-4.00000000000056	-1.24248496993988\\
-4.00000000000058	-1.22645290581162\\
-4.00000000000061	-1.21042084168337\\
-4.00000000000063	-1.19438877755511\\
-4.00000000000066	-1.17835671342685\\
-4.00000000000069	-1.1623246492986\\
-4.00000000000072	-1.14629258517034\\
-4.00000000000075	-1.13026052104208\\
-4.00000000000078	-1.11422845691383\\
-4.00000000000081	-1.09819639278557\\
-4.00000000000084	-1.08216432865731\\
-4.00000000000088	-1.06613226452906\\
-4.00000000000092	-1.0501002004008\\
-4.00000000000095	-1.03406813627255\\
-4.00000000000099	-1.01803607214429\\
-4.00000000000103	-1.00200400801603\\
-4.00000000000108	-0.985971943887776\\
-4.00000000000112	-0.969939879759519\\
-4.00000000000116	-0.953907815631263\\
-4.00000000000121	-0.937875751503006\\
-4.00000000000126	-0.92184368737475\\
-4.00000000000131	-0.905811623246493\\
-4.00000000000136	-0.889779559118236\\
-4.00000000000142	-0.87374749498998\\
-4.00000000000147	-0.857715430861723\\
-4.00000000000153	-0.841683366733467\\
-4.00000000000159	-0.82565130260521\\
-4.00000000000166	-0.809619238476954\\
-4.00000000000172	-0.793587174348697\\
-4.00000000000179	-0.777555110220441\\
-4.00000000000186	-0.761523046092184\\
-4.00000000000193	-0.745490981963928\\
-4.000000000002	-0.729458917835671\\
-4.00000000000208	-0.713426853707415\\
-4.00000000000216	-0.697394789579158\\
-4.00000000000225	-0.681362725450902\\
-4.00000000000233	-0.665330661322646\\
-4.00000000000242	-0.649298597194389\\
-4.00000000000251	-0.633266533066132\\
-4.00000000000261	-0.617234468937876\\
-4.0000000000027	-0.601202404809619\\
-4.00000000000281	-0.585170340681363\\
-4.00000000000291	-0.569138276553106\\
-4.00000000000302	-0.55310621242485\\
-4.00000000000313	-0.537074148296593\\
-4.00000000000325	-0.521042084168337\\
-4.00000000000337	-0.50501002004008\\
-4.00000000000349	-0.488977955911824\\
-4.00000000000362	-0.472945891783567\\
-4.00000000000375	-0.456913827655311\\
-4.00000000000388	-0.440881763527054\\
-4.00000000000403	-0.424849699398798\\
-4.00000000000417	-0.408817635270541\\
-4.00000000000432	-0.392785571142285\\
-4.00000000000448	-0.376753507014028\\
-4.00000000000463	-0.360721442885771\\
-4.0000000000048	-0.344689378757515\\
-4.00000000000497	-0.328657314629258\\
-4.00000000000514	-0.312625250501002\\
-4.00000000000532	-0.296593186372745\\
-4.00000000000551	-0.280561122244489\\
-4.0000000000057	-0.264529058116232\\
-4.0000000000059	-0.248496993987976\\
-4.00000000000611	-0.232464929859719\\
-4.00000000000631	-0.216432865731463\\
-4.00000000000653	-0.200400801603207\\
-4.00000000000676	-0.18436873747495\\
-4.00000000000699	-0.168336673346694\\
-4.00000000000722	-0.152304609218437\\
-4.00000000000747	-0.13627254509018\\
-4.00000000000772	-0.120240480961924\\
-4.00000000000798	-0.104208416833667\\
-4.00000000000824	-0.0881763527054109\\
-4.00000000000852	-0.0721442885771544\\
-4.0000000000088	-0.0561122244488979\\
-4.00000000000909	-0.0400801603206413\\
-4.00000000000939	-0.0240480961923848\\
-4.00000000000969	-0.00801603206412826\\
-4.00000000001001	0.00801603206412782\\
-4.00000000001033	0.0240480961923843\\
-4.00000000001067	0.0400801603206409\\
-4.00000000001101	0.0561122244488974\\
-4.00000000001137	0.0721442885771539\\
-4.00000000001173	0.0881763527054105\\
-4.0000000000121	0.104208416833667\\
-4.00000000001248	0.120240480961924\\
-4.00000000001288	0.13627254509018\\
-4.00000000001328	0.152304609218437\\
-4.0000000000137	0.168336673346693\\
-4.00000000001412	0.18436873747495\\
-4.00000000001456	0.200400801603206\\
-4.00000000001501	0.216432865731463\\
-4.00000000001547	0.232464929859719\\
-4.00000000001594	0.248496993987976\\
-4.00000000001643	0.264529058116232\\
-4.00000000001693	0.280561122244489\\
-4.00000000001744	0.296593186372745\\
-4.00000000001796	0.312625250501002\\
-4.0000000000185	0.328657314629258\\
-4.00000000001905	0.344689378757515\\
-4.00000000001962	0.360721442885771\\
-4.0000000000202	0.376753507014028\\
-4.00000000002079	0.392785571142285\\
-4.0000000000214	0.408817635270541\\
-4.00000000002202	0.424849699398798\\
-4.00000000002266	0.440881763527054\\
-4.00000000002332	0.456913827655311\\
-4.00000000002399	0.472945891783567\\
-4.00000000002467	0.488977955911824\\
-4.00000000002538	0.50501002004008\\
-4.0000000000261	0.521042084168337\\
-4.00000000002683	0.537074148296593\\
-4.00000000002759	0.55310621242485\\
-4.00000000002836	0.569138276553106\\
-4.00000000002914	0.585170340681363\\
-4.00000000002995	0.601202404809619\\
-4.00000000003078	0.617234468937876\\
-4.00000000003162	0.633266533066132\\
-4.00000000003248	0.649298597194389\\
-4.00000000003337	0.665330661322646\\
-4.00000000003427	0.681362725450902\\
-4.00000000003519	0.697394789579159\\
-4.00000000003613	0.713426853707415\\
-4.00000000003709	0.729458917835672\\
-4.00000000003808	0.745490981963928\\
-4.00000000003908	0.761523046092185\\
-4.00000000004011	0.777555110220441\\
-4.00000000004115	0.793587174348698\\
-4.00000000004222	0.809619238476954\\
-4.00000000004331	0.825651302605211\\
-4.00000000004442	0.841683366733467\\
-4.00000000004556	0.857715430861724\\
-4.00000000004672	0.87374749498998\\
-4.0000000000479	0.889779559118236\\
-4.00000000004911	0.905811623246493\\
-4.00000000005034	0.921843687374749\\
-4.00000000005159	0.937875751503006\\
-4.00000000005287	0.953907815631262\\
-4.00000000005418	0.969939879759519\\
-4.0000000000555	0.985971943887775\\
-4.00000000005686	1.00200400801603\\
-4.00000000005824	1.01803607214429\\
-4.00000000005964	1.03406813627254\\
-4.00000000006107	1.0501002004008\\
-4.00000000006253	1.06613226452906\\
-4.00000000006401	1.08216432865731\\
-4.00000000006553	1.09819639278557\\
-4.00000000006706	1.11422845691383\\
-4.00000000006863	1.13026052104208\\
-4.00000000007022	1.14629258517034\\
-4.00000000007184	1.1623246492986\\
-4.00000000007349	1.17835671342685\\
-4.00000000007516	1.19438877755511\\
-4.00000000007687	1.21042084168337\\
-4.0000000000786	1.22645290581162\\
-4.00000000008037	1.24248496993988\\
-4.00000000008216	1.25851703406814\\
-4.00000000008398	1.27454909819639\\
-4.00000000008582	1.29058116232465\\
-4.0000000000877	1.30661322645291\\
-4.00000000008961	1.32264529058116\\
-4.00000000009155	1.33867735470942\\
-4.00000000009352	1.35470941883768\\
-4.00000000009552	1.37074148296593\\
-4.00000000009754	1.38677354709419\\
-4.0000000000996	1.40280561122244\\
-4.00000000010169	1.4188376753507\\
-4.00000000010381	1.43486973947896\\
-4.00000000010596	1.45090180360721\\
-4.00000000010814	1.46693386773547\\
-4.00000000011035	1.48296593186373\\
-4.00000000011259	1.49899799599198\\
-4.00000000011487	1.51503006012024\\
-4.00000000011717	1.5310621242485\\
-4.00000000011951	1.54709418837675\\
-4.00000000012187	1.56312625250501\\
-4.00000000012427	1.57915831663327\\
-4.00000000012669	1.59519038076152\\
-4.00000000012915	1.61122244488978\\
-4.00000000013164	1.62725450901804\\
-4.00000000013416	1.64328657314629\\
-4.00000000013671	1.65931863727455\\
-4.00000000013929	1.67535070140281\\
-4.0000000001419	1.69138276553106\\
-4.00000000014454	1.70741482965932\\
-4.00000000014721	1.72344689378758\\
-4.00000000014991	1.73947895791583\\
-4.00000000015265	1.75551102204409\\
-4.00000000015541	1.77154308617235\\
-4.0000000001582	1.7875751503006\\
-4.00000000016102	1.80360721442886\\
-4.00000000016387	1.81963927855711\\
-4.00000000016675	1.83567134268537\\
-4.00000000016965	1.85170340681363\\
-4.00000000017259	1.86773547094188\\
-4.00000000017555	1.88376753507014\\
-4.00000000017854	1.8997995991984\\
-4.00000000018156	1.91583166332665\\
-4.00000000018461	1.93186372745491\\
-4.00000000018769	1.94789579158317\\
-4.00000000019079	1.96392785571142\\
-4.00000000019391	1.97995991983968\\
-4.00000000019707	1.99599198396794\\
-4.00000000020024	2.01202404809619\\
-4.00000000020345	2.02805611222445\\
-4.00000000020668	2.04408817635271\\
-4.00000000020993	2.06012024048096\\
-4.0000000002132	2.07615230460922\\
-4.0000000002165	2.09218436873747\\
-4.00000000021983	2.10821643286573\\
-4.00000000022317	2.12424849699399\\
-4.00000000022654	2.14028056112224\\
-4.00000000022992	2.1563126252505\\
-4.00000000023333	2.17234468937876\\
-4.00000000023676	2.18837675350701\\
-4.00000000024021	2.20440881763527\\
-4.00000000024368	2.22044088176353\\
-4.00000000024716	2.23647294589178\\
-4.00000000025066	2.25250501002004\\
-4.00000000025418	2.2685370741483\\
-4.00000000025772	2.28456913827655\\
-4.00000000026127	2.30060120240481\\
-4.00000000026484	2.31663326653307\\
-4.00000000026842	2.33266533066132\\
-4.00000000027201	2.34869739478958\\
-4.00000000027562	2.36472945891784\\
-4.00000000027924	2.38076152304609\\
-4.00000000028287	2.39679358717435\\
-4.00000000028651	2.41282565130261\\
-4.00000000029016	2.42885771543086\\
-4.00000000029382	2.44488977955912\\
-4.00000000029748	2.46092184368737\\
-4.00000000030116	2.47695390781563\\
-4.00000000030484	2.49298597194389\\
-4.00000000030852	2.50901803607214\\
-4.00000000031221	2.5250501002004\\
-4.0000000003159	2.54108216432866\\
-4.0000000003196	2.55711422845691\\
-4.0000000003233	2.57314629258517\\
-4.00000000032699	2.58917835671343\\
-4.00000000033069	2.60521042084168\\
-4.00000000033439	2.62124248496994\\
-4.00000000033808	2.6372745490982\\
-4.00000000034177	2.65330661322645\\
-4.00000000034546	2.66933867735471\\
-4.00000000034914	2.68537074148297\\
-4.00000000035282	2.70140280561122\\
-4.00000000035649	2.71743486973948\\
-4.00000000036015	2.73346693386774\\
-4.0000000003638	2.74949899799599\\
-4.00000000036744	2.76553106212425\\
-4.00000000037107	2.7815631262525\\
-4.00000000037469	2.79759519038076\\
-4.0000000003783	2.81362725450902\\
-4.00000000038189	2.82965931863727\\
-4.00000000038546	2.84569138276553\\
-4.00000000038902	2.86172344689379\\
-4.00000000039256	2.87775551102204\\
-4.00000000039608	2.8937875751503\\
-4.00000000039958	2.90981963927856\\
-4.00000000040307	2.92585170340681\\
-4.00000000040653	2.94188376753507\\
-4.00000000040996	2.95791583166333\\
-4.00000000041337	2.97394789579158\\
-4.00000000041676	2.98997995991984\\
-4.00000000042012	3.0060120240481\\
-4.00000000042346	3.02204408817635\\
-4.00000000042676	3.03807615230461\\
-4.00000000043004	3.05410821643287\\
-4.00000000043328	3.07014028056112\\
-4.0000000004365	3.08617234468938\\
-4.00000000043968	3.10220440881764\\
-4.00000000044282	3.11823647294589\\
-4.00000000044594	3.13426853707415\\
-4.00000000044901	3.1503006012024\\
-4.00000000045205	3.16633266533066\\
-4.00000000045505	3.18236472945892\\
-4.00000000045802	3.19839679358717\\
-4.00000000046094	3.21442885771543\\
-4.00000000046382	3.23046092184369\\
-4.00000000046666	3.24649298597194\\
-4.00000000046946	3.2625250501002\\
-4.00000000047221	3.27855711422846\\
-4.00000000047492	3.29458917835671\\
-4.00000000047758	3.31062124248497\\
-4.0000000004802	3.32665330661323\\
-4.00000000048277	3.34268537074148\\
-4.00000000048529	3.35871743486974\\
-4.00000000048776	3.374749498998\\
-4.00000000049017	3.39078156312625\\
-4.00000000049254	3.40681362725451\\
-4.00000000049486	3.42284569138277\\
-4.00000000049712	3.43887775551102\\
-4.00000000049933	3.45490981963928\\
-4.00000000050148	3.47094188376754\\
-4.00000000050358	3.48697394789579\\
-4.00000000050563	3.50300601202405\\
-4.00000000050761	3.51903807615231\\
-4.00000000050954	3.53507014028056\\
-4.00000000051141	3.55110220440882\\
-4.00000000051322	3.56713426853707\\
-4.00000000051497	3.58316633266533\\
-4.00000000051666	3.59919839679359\\
-4.00000000051829	3.61523046092184\\
-4.00000000051986	3.6312625250501\\
-4.00000000052136	3.64729458917836\\
-4.00000000052281	3.66332665330661\\
-4.00000000052419	3.67935871743487\\
-4.0000000005255	3.69539078156313\\
-4.00000000052675	3.71142284569138\\
-4.00000000052794	3.72745490981964\\
-4.00000000052906	3.7434869739479\\
-4.00000000053011	3.75951903807615\\
-4.0000000005311	3.77555110220441\\
-4.00000000053202	3.79158316633267\\
-4.00000000053288	3.80761523046092\\
-4.00000000053367	3.82364729458918\\
-4.00000000053439	3.83967935871743\\
-4.00000000053504	3.85571142284569\\
-4.00000000053563	3.87174348697395\\
-4.00000000053614	3.8877755511022\\
-4.00000000053659	3.90380761523046\\
-4.00000000053697	3.91983967935872\\
-4.00000000053728	3.93587174348697\\
-4.00000000053752	3.95190380761523\\
-4.0000000005377	3.96793587174349\\
-4.0000000005378	3.98396793587174\\
-4.00000000053783	4\\
-4	4.00000000053783\\
}--cycle;


\addplot[area legend,solid,fill=mycolor2,draw=black,forget plot]
table[row sep=crcr] {%
x	y\\
-2.18837675350701	2.51218829191193\\
-2.18733967634176	2.5250501002004\\
-2.18597120113148	2.54108216432866\\
-2.18452376321569	2.55711422845691\\
-2.18299582363611	2.57314629258517\\
-2.18138579001696	2.58917835671343\\
-2.179692015072	2.60521042084168\\
-2.17791279505507	2.62124248496994\\
-2.17604636815184	2.6372745490982\\
-2.17409091281085	2.65330661322645\\
-2.17234468937876	2.66700321609212\\
-2.17205000108829	2.66933867735471\\
-2.169949223652	2.68537074148297\\
-2.16775450582966	2.70140280561122\\
-2.1654637519929	2.71743486973948\\
-2.1630747968375	2.73346693386774\\
-2.16058540325749	2.74949899799599\\
-2.15799326013909	2.76553106212425\\
-2.1563126252505	2.77555497956145\\
-2.15531263058558	2.7815631262525\\
-2.1525529883469	2.79759519038076\\
-2.14968393265089	2.81362725450902\\
-2.14670280909838	2.82965931863727\\
-2.1436068759187	2.84569138276553\\
-2.14039330115298	2.86172344689379\\
-2.14028056112224	2.86227095588746\\
-2.13710656700386	2.87775551102204\\
-2.13369847096792	2.8937875751503\\
-2.13016418146547	2.90981963927856\\
-2.12650044307887	2.92585170340681\\
-2.12424849699399	2.9354039450745\\
-2.12272457763207	2.94188376753507\\
-2.11884335835732	2.95791583166333\\
-2.11482256942741	2.97394789579158\\
-2.11065846187091	2.98997995991984\\
-2.10821643286573	2.99910755738531\\
-2.10636965934285	3.0060120240481\\
-2.10195962779741	3.02204408817635\\
-2.09739431105722	3.03807615230461\\
-2.09266938106967	3.05410821643287\\
-2.09218436873747	3.05571690098827\\
-2.08782724256108	3.07014028056112\\
-2.08282144178327	3.08617234468938\\
-2.07764187201208	3.10220440881764\\
-2.07615230460922	3.10670417198628\\
-2.0723199728081	3.11823647294589\\
-2.0668279322443	3.13426853707415\\
-2.06114596808467	3.1503006012024\\
-2.06012024048096	3.15313095886327\\
-2.05530808263616	3.16633266533066\\
-2.04927652476668	3.18236472945892\\
-2.04408817635271	3.19572743614558\\
-2.04304403780948	3.19839679358717\\
-2.03663353865448	3.21442885771543\\
-2.03000039851701	3.23046092184369\\
-2.02805611222445	3.23505272624421\\
-2.02316612665281	3.24649298597194\\
-2.01610504929561	3.2625250501002\\
-2.01202404809619	3.27154119214257\\
-2.00881293353938	3.27855711422846\\
-2.00128462031002	3.29458917835671\\
-1.99599198396794	3.30553466284577\\
-1.99350063178841	3.31062124248497\\
-1.98546057570419	3.32665330661323\\
-1.97995991983968	3.33730692442868\\
-1.97714210753077	3.34268537074148\\
-1.96853981202308	3.35871743486974\\
-1.96392785571142	3.36708567002893\\
-1.95963426252646	3.374749498998\\
-1.9504123103506	3.39078156312625\\
-1.94789579158317	3.39506013426401\\
-1.9408553430388	3.40681362725451\\
-1.93186372745491	3.42138894870863\\
-1.93094713526962	3.42284569138277\\
-1.92066168164344	3.43887775551102\\
-1.91583166332665	3.44621303528219\\
-1.90997946975807	3.45490981963928\\
-1.8997995991984	3.46964207855806\\
-1.89888029101296	3.47094188376754\\
-1.8873210583821	3.48697394789579\\
-1.88376753507014	3.4917892515538\\
-1.87527724755564	3.50300601202405\\
-1.86773547094188	3.51273640977956\\
-1.86271726148988	3.51903807615231\\
-1.85170340681363	3.53256061063652\\
-1.84959972635276	3.53507014028056\\
-1.8358740682194	3.55110220440882\\
-1.83567134268537	3.55133486431535\\
-1.82146947117734	3.56713426853707\\
-1.81963927855711	3.56913032217164\\
-1.80633755317204	3.58316633266533\\
-1.80360721442886	3.58599389569402\\
-1.79040271332929	3.59919839679359\\
-1.7875751503006	3.6019764910627\\
-1.77357601502004	3.61523046092184\\
-1.77154308617234	3.6171235514231\\
-1.75575261167513	3.6312625250501\\
-1.75551102204409	3.63147552060405\\
-1.73947895791583	3.64508353874812\\
-1.73675998401568	3.64729458917836\\
-1.72344689378758	3.65797252578098\\
-1.71645946836329	3.66332665330661\\
-1.70741482965932	3.67016942626541\\
-1.69466730704812	3.67935871743487\\
-1.69138276553106	3.68169890685915\\
-1.67535070140281	3.69260532634724\\
-1.67104927101087	3.69539078156313\\
-1.65931863727455	3.70291133356174\\
-1.64527066679244	3.71142284569138\\
-1.64328657314629	3.71261420294734\\
-1.62725450901804	3.72177920887112\\
-1.61670775334827	3.72745490981964\\
-1.61122244488978	3.7303851331788\\
-1.59519038076152	3.73847872555087\\
-1.5846136617279	3.7434869739479\\
-1.57915831663327	3.74605536787514\\
-1.56312625250501	3.75315746679068\\
-1.54767777357342	3.75951903807615\\
-1.54709418837675	3.75975836220358\\
-1.5310621242485	3.76593753192175\\
-1.51503006012024	3.77164679629925\\
-1.50316156487214	3.77555110220441\\
-1.49899799599198	3.77691813026863\\
-1.48296593186373	3.7817922824152\\
-1.46693386773547	3.78624079510059\\
-1.45090180360721	3.7902776572105\\
-1.44519057425423	3.79158316633267\\
-1.43486973947896	3.79394442068093\\
-1.4188376753507	3.79723728898389\\
-1.40280561122244	3.80015222065808\\
-1.38677354709419	3.80270011516812\\
-1.37074148296593	3.80489119658113\\
-1.35470941883768	3.80673504142009\\
-1.34536965782306	3.80761523046092\\
-1.33867735470942	3.80824897603755\\
-1.32264529058116	3.80944066381979\\
-1.30661322645291	3.81030619218637\\
-1.29058116232465	3.81085272362304\\
-1.27454909819639	3.81108688513842\\
-1.25851703406814	3.81101478617959\\
-1.24248496993988	3.81064203520423\\
-1.22645290581162	3.80997375496874\\
-1.21042084168337	3.80901459658709\\
-1.19438877755511	3.80776875241063\\
-1.19278466694236	3.80761523046092\\
-1.17835671342685	3.80626041352538\\
-1.1623246492986	3.80447940168065\\
-1.14629258517034	3.80242643223183\\
-1.13026052104208	3.80010406348015\\
-1.11422845691383	3.79751444724034\\
-1.09819639278557	3.79465933573617\\
-1.08238610784513	3.79158316633267\\
-1.08216432865731	3.79154076292006\\
-1.06613226452906	3.78821192551388\\
-1.0501002004008	3.7846257924548\\
-1.03406813627255	3.78078268667116\\
-1.01803607214429	3.77668255792351\\
-1.01387250326413	3.77555110220441\\
-1.00200400801603	3.7723776273203\\
-0.985971943887776	3.76783838447288\\
-0.969939879759519	3.76304552242583\\
-0.958735177018894	3.75951903807615\\
-0.953907815631263	3.75802296121048\\
-0.937875751503006	3.75280765424696\\
-0.92184368737475	3.74733952723095\\
-0.91104327932147	3.7434869739479\\
-0.905811623246493	3.74164795998862\\
-0.889779559118236	3.73576945779861\\
-0.87374749498998	3.72963642279804\\
-0.868262186531491	3.72745490981964\\
-0.857715430861723	3.7233182685757\\
-0.841683366733467	3.71678256208697\\
-0.829025370107675	3.71142284569138\\
-0.82565130260521	3.71001314770832\\
-0.809619238476954	3.70308011609804\\
-0.793587174348697	3.69588653505746\\
-0.792516323769336	3.69539078156313\\
-0.777555110220441	3.68855112833831\\
-0.761523046092184	3.68096111671956\\
-0.758238504575129	3.67935871743487\\
-0.745490981963928	3.6732143754226\\
-0.729458917835671	3.66522904433948\\
-0.725746813913297	3.66332665330661\\
-0.713426853707415	3.65708548600821\\
-0.697394789579158	3.64870434829676\\
-0.69476868281116	3.64729458917836\\
-0.681362725450902	3.64017704005774\\
-0.665330661322646	3.63139803025136\\
-0.665089071691607	3.6312625250501\\
-0.649298597194389	3.62249864928523\\
-0.636584308250442	3.61523046092184\\
-0.633266533066132	3.61335323719904\\
-0.617234468937876	3.60405701947759\\
-0.609073991017453	3.59919839679359\\
-0.601202404809619	3.59455732732911\\
-0.585170340681363	3.58485599522903\\
-0.582440001938185	3.58316633266533\\
-0.569138276553106	3.57501041351771\\
-0.556626181278272	3.56713426853707\\
-0.55310621242485	3.56493835321383\\
-0.537074148296593	3.55471227402176\\
-0.531542492394103	3.55110220440882\\
-0.521042084168337	3.5443073060698\\
-0.507113700500945	3.53507014028056\\
-0.50501002004008	3.53368642806086\\
-0.488977955911824	3.52292736966742\\
-0.483308498413444	3.51903807615231\\
-0.472945891783567	3.51198384860222\\
-0.46005621128414	3.50300601202405\\
-0.456913827655311	3.50083355292382\\
-0.440881763527054	3.48952899261935\\
-0.437328240215089	3.48697394789579\\
-0.424849699398798	3.47806371092255\\
-0.415086534647536	3.47094188376754\\
-0.408817635270541	3.46639943697746\\
-0.393288980010621	3.45490981963928\\
-0.392785571142285	3.454539734248\\
-0.376753507014028	3.44255838458529\\
-0.371923488697241	3.43887775551102\\
-0.360721442885771	3.43039175147638\\
-0.350951914584676	3.42284569138277\\
-0.344689378757515	3.41803573511502\\
-0.330353172608619	3.40681362725451\\
-0.328657314629258	3.40549328796713\\
-0.312625250501002	3.39280623394943\\
-0.310108731733566	3.39078156312625\\
-0.296593186372745	3.37996057446672\\
-0.29019815203592	3.374749498998\\
-0.280561122244489	3.36693288932542\\
-0.270605055257431	3.35871743486974\\
-0.264529058116232	3.35372555976707\\
-0.251314806296885	3.34268537074148\\
-0.248496993987976	3.34034081811575\\
-0.232464929859719	3.32678327605539\\
-0.232313243736369	3.32665330661323\\
-0.216432865731463	3.31309537364555\\
-0.213579442999436	3.31062124248497\\
-0.200400801603207	3.2992324911091\\
-0.195108165261117	3.29458917835671\\
-0.18436873747495	3.28519632550059\\
-0.17688838962525	3.27855711422846\\
-0.168336673346694	3.27098843500974\\
-0.158909757082702	3.2625250501002\\
-0.152304609218437	3.25661024098888\\
-0.141162530661818	3.24649298597194\\
-0.13627254509018	3.24206302979891\\
-0.12363755846904	3.23046092184369\\
-0.120240480961924	3.22734795447686\\
-0.10632623939084	3.21442885771543\\
-0.104208416833667	3.21246603622831\\
-0.0892204912486374	3.19839679358717\\
-0.0881763527054109	3.19741816574723\\
-0.0723127212168553	3.18236472945892\\
-0.0721442885771544	3.18220510436568\\
-0.0561122244488979	3.16683736834846\\
-0.0555916565288081	3.16633266533066\\
-0.0400801603206413	3.15130549133715\\
-0.0390544327169323	3.1503006012024\\
-0.0240480961923848	3.13560667108172\\
-0.0226962325176485	3.13426853707415\\
-0.00801603206412826	3.11974116466135\\
-0.00651134034866899	3.11823647294589\\
0.00801603206412782	3.10370910053309\\
0.00950559946698457	3.10220440881764\\
0.0240480961923843	3.08751047869695\\
0.0253596009134551	3.08617234468938\\
0.0400801603206409	3.07114517069587\\
0.0410553532443366	3.07014028056112\\
0.0561122244488974	3.05461291945066\\
0.0565972367810957	3.05410821643287\\
0.0719910800676468	3.03807615230461\\
0.0721442885771539	3.03791652721137\\
0.087246414967761	3.02204408817635\\
0.0881763527054105	3.0210654603364\\
0.102361643310788	3.0060120240481\\
0.104208416833667	3.00404920256098\\
0.117340166624626	2.98997995991984\\
0.120240480961924	2.98686699255302\\
0.1321851375395	2.97394789579158\\
0.13627254509018	2.96951793961855\\
0.146899470581772	2.95791583166333\\
0.152304609218437	2.95200102255201\\
0.161485852207839	2.94188376753507\\
0.168336673346693	2.93431508831635\\
0.175946750125508	2.92585170340681\\
0.18436873747495	2.91645885055069\\
0.190284421946428	2.90981963927856\\
0.200400801603206	2.89843088790269\\
0.204500923209734	2.8937875751503\\
0.216432865731463	2.88022964218263\\
0.218598114813846	2.87775551102204\\
0.232464929859719	2.86185341633595\\
0.232577669890459	2.86172344689379\\
0.246472484612783	2.84569138276553\\
0.248496993987976	2.84334683013979\\
0.260257551260827	2.82965931863727\\
0.264529058116232	2.8246674435346\\
0.273932429644883	2.81362725450902\\
0.280561122244489	2.80581064483644\\
0.28749819455161	2.79759519038076\\
0.296593186372745	2.78677420172124\\
0.300955760353324	2.7815631262525\\
0.312625250501002	2.76755573294743\\
0.314305885389588	2.76553106212425\\
0.327567928056745	2.74949899799599\\
0.328657314629258	2.74817865870861\\
0.340755276724305	2.73346693386774\\
0.344689378757515	2.72865697759999\\
0.353840505499915	2.71743486973948\\
0.360721442885771	2.70894886570484\\
0.366823878678944	2.70140280561122\\
0.376753507014028	2.68905137055723\\
0.37970552252767	2.68537074148297\\
0.39249088285182	2.66933867735471\\
0.392785571142285	2.66896859196343\\
0.405231158769934	2.65330661322645\\
0.408817635270541	2.64876416643638\\
0.417873568327678	2.6372745490982\\
0.424849699398798	2.62836431212495\\
0.430417805075108	2.62124248496994\\
0.440881763527054	2.60776546556524\\
0.442863436587081	2.60521042084168\\
0.45524297062471	2.58917835671343\\
0.456913827655311	2.5870058976132\\
0.467564961912667	2.57314629258517\\
0.472945891783567	2.56609206503508\\
0.479790945645323	2.55711422845691\\
0.488977955911824	2.54497145784377\\
0.491920100199112	2.54108216432866\\
0.503972942874826	2.5250501002004\\
0.50501002004008	2.5236663879807\\
0.515990373212707	2.50901803607214\\
0.521042084168337	2.50222313773313\\
0.527912695607077	2.49298597194389\\
0.537074148296593	2.48056397742858\\
0.539738724534669	2.47695390781563\\
0.551501787790471	2.46092184368737\\
0.55310621242485	2.45872592836413\\
0.563226279677954	2.44488977955912\\
0.569138276553106	2.43673386041149\\
0.574855395806361	2.42885771543086\\
0.585170340681363	2.4145153138663\\
0.586387604466831	2.41282565130261\\
0.597895556901213	2.39679358717435\\
0.601202404809619	2.39215251770987\\
0.609334863426597	2.38076152304609\\
0.617234468937876	2.36958808160184\\
0.620677433211101	2.36472945891784\\
0.631951634761235	2.34869739478958\\
0.633266533066132	2.34682017106677\\
0.643206984131002	2.33266533066132\\
0.649298597194389	2.32390145489646\\
0.654365277727699	2.31663326653307\\
0.665330661322646	2.30073670760607\\
0.665424433461496	2.30060120240481\\
0.676498350458976	2.28456913827655\\
0.681362725450902	2.27745158915594\\
0.687476579602111	2.2685370741483\\
0.697394789579159	2.25391476913844\\
0.698354673058552	2.25250501002004\\
0.709232506939536	2.23647294589178\\
0.713426853707415	2.23023177859338\\
0.720033795682041	2.22044088176353\\
0.729458917835672	2.20631120866813\\
0.730733503473829	2.20440881763527\\
0.7414301412799	2.18837675350701\\
0.745490981963928	2.18223241149474\\
0.752056579978297	2.17234468937876\\
0.761523046092185	2.15791502453519\\
0.762579557512944	2.1563126252505\\
0.773109252646421	2.14028056112224\\
0.777555110220441	2.13344090789742\\
0.783561943357505	2.12424849699399\\
0.793587174348698	2.10871218636007\\
0.793908866757872	2.10821643286573\\
0.804285298864114	2.09218436873747\\
0.809619238476954	2.08384163914413\\
0.814564392751905	2.07615230460922\\
0.824758302252478	2.06012024048096\\
0.825651302605211	2.0587105424979\\
0.834971324123861	2.04408817635271\\
0.841683366733467	2.03341582862003\\
0.845076051383763	2.02805611222445\\
0.855137738611259	2.01202404809619\\
0.857715430861724	2.00788740685225\\
0.865178068571839	1.99599198396794\\
0.87374749498998	1.98214143281808\\
0.875106761295386	1.97995991983968\\
0.885048557799838	1.96392785571142\\
0.889779559118236	1.95621033956214\\
0.89491406096065	1.94789579158317\\
0.904694649819828	1.93186372745491\\
0.905811623246493	1.93002471349564\\
0.914498021268976	1.91583166332665\\
0.921843687374749	1.90365215248145\\
0.924185695363177	1.8997995991984\\
0.933865262670698	1.88376753507014\\
0.937875751503006	1.87705615124094\\
0.943491311068582	1.86773547094188\\
0.953022059741463	1.85170340681363\\
0.953907815631262	1.85020732994795\\
0.962587153792168	1.83567134268537\\
0.969939879759519	1.8231657629068\\
0.972031583136088	1.81963927855711\\
0.981479033512833	1.80360721442886\\
0.985971943887775	1.79589449669732\\
0.990862346907323	1.7875751503006\\
1.00017244570538	1.77154308617235\\
1.00200400801603	1.76836961128823\\
1.00949495205705	1.75551102204409\\
1.01803607214429	1.74061041363493\\
1.01869091350944	1.73947895791583\\
1.02793446764556	1.72344689378758\\
1.03406813627254	1.71264641412607\\
1.03706881500154	1.70741482965932\\
1.04618567492923	1.69138276553106\\
1.0501002004008	1.6844253916532\\
1.05525837210245	1.67535070140281\\
1.06425307777468	1.65931863727455\\
1.06613226452906	1.65594739645577\\
1.07326397108547	1.64328657314629\\
1.08214091242481	1.62725450901804\\
1.08216432865731	1.62721210560544\\
1.09108973294102	1.61122244488978\\
1.09819639278557	1.59826655016503\\
1.09990251902498	1.59519038076152\\
1.10873952224162	1.57915831663327\\
1.11422845691383	1.56905753341269\\
1.11748836383966	1.56312625250501\\
1.12621695542465	1.54709418837675\\
1.13026052104208	1.53958302139598\\
1.13490125784064	1.5310621242485\\
1.14352540851714	1.51503006012024\\
1.14629258517034	1.50984126189114\\
1.15214446671673	1.49899799599198\\
1.16066802432549	1.48296593186373\\
1.1623246492986	1.47983010308346\\
1.16922102401272	1.46693386773547\\
1.17764771911111	1.45090180360721\\
1.17835671342685	1.44954698667167\\
1.18613373755529	1.43486973947896\\
1.19438877755511	1.41899119730041\\
1.19446961549923	1.4188376753507\\
1.2028851953898	1.40280561122244\\
1.21042084168337	1.38817291322035\\
1.21115094474724	1.38677354709419\\
1.21947777124482	1.37074148296593\\
1.22645290581162	1.35706794334549\\
1.22767214575648	1.35470941883768\\
1.23591362954052	1.33867735470942\\
1.24248496993988	1.32567209532447\\
1.24403527399133	1.32264529058116\\
1.25219472995507	1.30661322645291\\
1.25851703406814	1.29398071804332\\
1.26024217985847	1.29058116232465\\
1.26832283156246	1.27454909819639\\
1.27454909819639	1.26198868874564\\
1.27629451261218	1.25851703406814\\
1.28429949655346	1.24248496993988\\
1.29058116232465	1.22969039897374\\
1.29219372384407	1.22645290581162\\
1.30012609355036	1.21042084168337\\
1.30661322645291	1.19707973928056\\
1.30794107056714	1.19438877755511\\
1.31580380052522	1.17835671342685\\
1.32264529058116	1.16415008265747\\
1.32353761790295	1.1623246492986\\
1.33133360732971	1.14629258517034\\
1.33867735470942	1.13089426661871\\
1.33898424137981	1.13026052104208\\
1.34671631784397	1.11422845691383\\
1.35429512759832	1.09819639278557\\
1.35470941883768	1.09731620374474\\
1.36195255175066	1.08216432865731\\
1.36947185626311	1.06613226452906\\
1.37074148296593	1.06340823064926\\
1.37704274593921	1.0501002004008\\
1.38450516462339	1.03406813627254\\
1.38677354709419	1.02915302097975\\
1.39198715554453	1.01803607214429\\
1.39939524815508	1.00200400801603\\
1.40280561122244	0.994540998213194\\
1.40678585462307	0.985971943887775\\
1.41414212425145	0.969939879759519\\
1.4188376753507	0.959561938282485\\
1.42143873646836	0.953907815631262\\
1.42874563242771	0.937875751503006\\
1.43486973947896	0.924204941723015\\
1.43594551356707	0.921843687374749\\
1.44320543417176	0.905811623246493\\
1.45032497946464	0.889779559118236\\
1.45090180360721	0.888474049996076\\
1.4575210124409	0.87374749498998\\
1.46459710752866	0.857715430861724\\
1.46693386773547	0.852373059629644\\
1.47169167080326	0.841683366733467\\
1.4787265569285	0.825651302605211\\
1.48296593186373	0.815860418687749\\
1.48571653222177	0.809619238476954\\
1.49271240871521	0.793587174348698\\
1.49899799599198	0.778922138284666\\
1.49959453747732	0.777555110220441\\
1.50655356408688	0.761523046092185\\
1.51338033327571	0.745490981963928\\
1.51503006012024	0.741586676058771\\
1.52024874236831	0.729458917835672\\
1.52704167434769	0.713426853707415\\
1.5310621242485	0.703813283424752\\
1.53379647862441	0.697394789579159\\
1.540557604138	0.681362725450902\\
1.54709418837675	0.665569985450078\\
1.54719512090086	0.665330661322646\\
1.55392643861466	0.649298597194389\\
1.56053156729777	0.633266533066132\\
1.56312625250501	0.626904961780657\\
1.56714630555442	0.617234468937876\\
1.57372428420662	0.601202404809619\\
1.57915831663327	0.587738734608609\\
1.58021514095198	0.585170340681363\\
1.58676785760723	0.569138276553106\\
1.59319910861719	0.55310621242485\\
1.59519038076152	0.548097964027826\\
1.59966000209789	0.537074148296593\\
1.60606840850847	0.521042084168337\\
1.61122244488978	0.507940243399239\\
1.61239823384089	0.50501002004008\\
1.61878558460074	0.488977955911824\\
1.62505577178471	0.472945891783567\\
1.62725450901804	0.467270190835045\\
1.63134792809836	0.456913827655311\\
1.63759925671641	0.440881763527054\\
1.64328657314629	0.42604105665475\\
1.64375252317298	0.424849699398798\\
1.64998669320543	0.408817635270541\\
1.65610761012657	0.392785571142285\\
1.65931863727455	0.384274059012638\\
1.66221493444902	0.376753507014028\\
1.66832068705787	0.360721442885771\\
1.67431648597408	0.344689378757515\\
1.67535070140281	0.34190392354163\\
1.68037282410443	0.328657314629258\\
1.6863553242951	0.312625250501002\\
1.69138276553106	0.298933375797022\\
1.6922604173067	0.296593186372745\\
1.69823117961289	0.280561122244489\\
1.70409529753085	0.264529058116232\\
1.70741482965932	0.255339766946777\\
1.70994020395826	0.248496993987976\\
1.71579427084906	0.232464929859719\\
1.72154459568374	0.216432865731463\\
1.72344689378758	0.211078738205828\\
1.72732382048331	0.200400801603206\\
1.73306566673049	0.18436873747495\\
1.73870651950803	0.168336673346693\\
1.73947895791583	0.166125622916457\\
1.74441441207152	0.152304609218437\\
1.75004824908652	0.13627254509018\\
1.75551102204409	0.12045347651587\\
1.75558622659312	0.120240480961924\\
1.76121443676716	0.104208416833667\\
1.76674422382966	0.0881763527054105\\
1.77154308617235	0.0740373790784134\\
1.77220020625581	0.0721442885771539\\
1.77772567498475	0.0561122244488974\\
1.78315512949002	0.0400801603206409\\
1.7875751503006	0.0268261904615026\\
1.78852289179611	0.0240480961923843\\
1.7939492363902	0.00801603206412782\\
1.79928184315911	-0.00801603206412826\\
1.80360721442886	-0.0212205331636935\\
1.80455495592437	-0.0240480961923848\\
1.80988556418485	-0.0400801603206413\\
1.81512458387508	-0.0561122244488979\\
1.81963927855711	-0.0701482349425848\\
1.82029639864058	-0.0721442885771544\\
1.82553443681551	-0.0881763527054109\\
1.83068291346776	-0.104208416833667\\
1.83567134268537	-0.120007821055388\\
1.8357465472344	-0.120240480961924\\
1.84089496711966	-0.13627254509018\\
1.84595573486811	-0.152304609218437\\
1.85093096840582	-0.168336673346694\\
1.85170340681363	-0.17084620299073\\
1.85596559890135	-0.18436873747495\\
1.86094128787276	-0.200400801603207\\
1.86583317283805	-0.216432865731463\\
1.86773547094188	-0.222734532104207\\
1.87074410091915	-0.232464929859719\\
1.875637142341	-0.248496993987976\\
1.88044800294167	-0.264529058116232\\
1.88376753507014	-0.275745818586478\\
1.88522755825375	-0.280561122244489\\
1.89004018878716	-0.296593186372745\\
1.89477215796244	-0.312625250501002\\
1.89942534149261	-0.328657314629258\\
1.8997995991984	-0.329957119838733\\
1.90414662631756	-0.344689378757515\\
1.90880164898172	-0.360721442885771\\
1.9133792480818	-0.376753507014028\\
1.91583166332665	-0.385450291371113\\
1.91795194784638	-0.392785571142285\\
1.92253178338579	-0.408817635270541\\
1.92703546142494	-0.424849699398798\\
1.93146469803591	-0.440881763527054\\
1.93186372745491	-0.44233850620119\\
1.93595714653524	-0.456913827655311\\
1.94038838530739	-0.472945891783567\\
1.94474630364003	-0.488977955911824\\
1.94789579158317	-0.500731448902317\\
1.94907158053428	-0.50501002004008\\
1.95343168096556	-0.521042084168337\\
1.95771949174352	-0.537074148296593\\
1.96193658356709	-0.55310621242485\\
1.96392785571142	-0.560770041393917\\
1.96615824208962	-0.569138276553106\\
1.97037697660293	-0.585170340681363\\
1.97452588741304	-0.601202404809619\\
1.97860645615369	-0.617234468937876\\
1.97995991983968	-0.622612915250674\\
1.98271067567622	-0.633266533066132\\
1.98679217007759	-0.649298597194389\\
1.99080608711256	-0.665330661322646\\
1.99475382388421	-0.681362725450902\\
1.99599198396794	-0.686449305090096\\
1.99872633834385	-0.697394789579158\\
2.00267410246425	-0.713426853707415\\
2.0065563245599	-0.729458917835671\\
2.01037432125166	-0.745490981963928\\
2.01202404809619	-0.752506904049819\\
2.0142003569969	-0.761523046092184\\
2.01801729618351	-0.777555110220441\\
2.02177052494767	-0.793587174348697\\
2.02546128301537	-0.809619238476954\\
2.02805611222445	-0.821059498204689\\
2.02912549303033	-0.82565130260521\\
2.03281391529224	-0.841683366733467\\
2.03644026093137	-0.857715430861723\\
2.04000569592562	-0.87374749498998\\
2.04351135221013	-0.889779559118236\\
2.04408817635271	-0.892448916559835\\
2.04705370876431	-0.905811623246493\\
2.05055469160957	-0.92184368737475\\
2.05399613342972	-0.937875751503006\\
2.05737909650658	-0.953907815631263\\
2.06012024048096	-0.96710952209865\\
2.06072393619058	-0.969939879759519\\
2.06410048388159	-0.985971943887776\\
2.06741867270145	-1.00200400801603\\
2.07067949744408	-1.01803607214429\\
2.07388392213842	-1.03406813627255\\
2.07615230460922	-1.04560043723216\\
2.0770617153881	-1.0501002004008\\
2.08025679315444	-1.06613226452906\\
2.0833954375222	-1.08216432865731\\
2.08647854783053	-1.09819639278557\\
2.08950699428547	-1.11422845691383\\
2.09218436873747	-1.12865183648668\\
2.09249125540787	-1.13026052104208\\
2.09550706610889	-1.14629258517034\\
2.09846802604284	-1.1623246492986\\
2.10137494280979	-1.17835671342685\\
2.10422859624857	-1.19438877755511\\
2.10702973894416	-1.21042084168337\\
2.10821643286573	-1.21732530834615\\
2.10982899438516	-1.22645290581162\\
2.11261272985613	-1.24248496993988\\
2.11534357785558	-1.25851703406814\\
2.11802223036006	-1.27454909819639\\
2.12064935314585	-1.29058116232465\\
2.1232255862079	-1.30661322645291\\
2.12424849699399	-1.31309304891348\\
2.12579880104544	-1.32264529058116\\
2.12835227738831	-1.33867735470942\\
2.13085429246183	-1.35470941883768\\
2.13330542655544	-1.37074148296593\\
2.13570623492495	-1.38677354709419\\
2.13805724812759	-1.40280561122244\\
2.14028056112224	-1.41829016635703\\
2.14036139906636	-1.4188376753507\\
2.14268365816133	-1.43486973947896\\
2.14495534686681	-1.45090180360721\\
2.14717693583636	-1.46693386773547\\
2.14934887147192	-1.48296593186373\\
2.15147157617786	-1.49899799599198\\
2.1535454485973	-1.51503006012024\\
2.15557086383083	-1.5310621242485\\
2.1563126252505	-1.53707027093956\\
2.15758539475005	-1.54709418837675\\
2.15957253217634	-1.56312625250501\\
2.16151014528104	-1.57915831663327\\
2.1633985476971	-1.59519038076152\\
2.1652380295342	-1.61122244488978\\
2.16702885751804	-1.62725450901804\\
2.16877127511243	-1.64328657314629\\
2.17046550262438	-1.65931863727455\\
2.17211173729216	-1.67535070140281\\
2.17234468937876	-1.67768616266539\\
2.17374993028851	-1.69138276553106\\
2.17534536810776	-1.70741482965932\\
2.17689138091722	-1.72344689378758\\
2.17838810008491	-1.73947895791583\\
2.17983563341978	-1.75551102204409\\
2.18123406517753	-1.77154308617234\\
2.18258345604918	-1.7875751503006\\
2.18388384313207	-1.80360721442886\\
2.1851352398834	-1.81963927855711\\
2.18633763605613	-1.83567134268537\\
2.18749099761721	-1.85170340681363\\
2.18837675350701	-1.8645652151021\\
2.18860133989036	-1.86773547094188\\
2.18968554512439	-1.88376753507014\\
2.19071876149544	-1.8997995991984\\
2.19170087930555	-1.91583166332665\\
2.19263176387134	-1.93186372745491\\
2.19351125534943	-1.94789579158317\\
2.19433916854262	-1.96392785571142\\
2.19511529268665	-1.97995991983968\\
2.19583939121713	-1.99599198396794\\
2.19651120151655	-2.01202404809619\\
2.19713043464091	-2.02805611222445\\
2.19769677502566	-2.04408817635271\\
2.19820988017068	-2.06012024048096\\
2.19866938030378	-2.07615230460922\\
2.19907487802243	-2.09218436873747\\
2.19942594791327	-2.10821643286573\\
2.19972213614885	-2.12424849699399\\
2.19996296006125	-2.14028056112224\\
2.20014790769201	-2.1563126252505\\
2.20027643731779	-2.17234468937876\\
2.20034797695124	-2.18837675350701\\
2.20036192381657	-2.20440881763527\\
2.20031764379895	-2.22044088176353\\
2.20021447086739	-2.23647294589178\\
2.20005170647016	-2.25250501002004\\
2.19982861890213	-2.2685370741483\\
2.19954444264335	-2.28456913827655\\
2.19919837766783	-2.30060120240481\\
2.19878958872204	-2.31663326653307\\
2.19831720457188	-2.33266533066132\\
2.19778031721757	-2.34869739478958\\
2.19717798107516	-2.36472945891784\\
2.19650921212399	-2.38076152304609\\
2.19577298701883	-2.39679358717435\\
2.1949682421657	-2.41282565130261\\
2.19409387276027	-2.42885771543086\\
2.19314873178755	-2.44488977955912\\
2.19213162898178	-2.46092184368737\\
2.19104132974509	-2.47695390781563\\
2.1898765540237	-2.49298597194389\\
2.18863597514014	-2.50901803607214\\
2.18837675350701	-2.51218829191193\\
2.18733967634176	-2.5250501002004\\
2.18597120113148	-2.54108216432866\\
2.18452376321569	-2.55711422845691\\
2.18299582363611	-2.57314629258517\\
2.18138579001696	-2.58917835671343\\
2.179692015072	-2.60521042084168\\
2.17791279505507	-2.62124248496994\\
2.17604636815184	-2.6372745490982\\
2.17409091281085	-2.65330661322645\\
2.17234468937876	-2.66700321609212\\
2.17205000108829	-2.66933867735471\\
2.169949223652	-2.68537074148297\\
2.16775450582966	-2.70140280561122\\
2.1654637519929	-2.71743486973948\\
2.1630747968375	-2.73346693386774\\
2.16058540325749	-2.74949899799599\\
2.15799326013909	-2.76553106212425\\
2.1563126252505	-2.77555497956145\\
2.15531263058558	-2.7815631262525\\
2.1525529883469	-2.79759519038076\\
2.14968393265089	-2.81362725450902\\
2.14670280909838	-2.82965931863727\\
2.1436068759187	-2.84569138276553\\
2.14039330115298	-2.86172344689379\\
2.14028056112224	-2.86227095588746\\
2.13710656700386	-2.87775551102204\\
2.13369847096792	-2.8937875751503\\
2.13016418146547	-2.90981963927856\\
2.12650044307887	-2.92585170340681\\
2.12424849699399	-2.9354039450745\\
2.12272457763207	-2.94188376753507\\
2.11884335835732	-2.95791583166333\\
2.11482256942741	-2.97394789579158\\
2.11065846187091	-2.98997995991984\\
2.10821643286573	-2.99910755738531\\
2.10636965934285	-3.0060120240481\\
2.10195962779741	-3.02204408817635\\
2.09739431105722	-3.03807615230461\\
2.09266938106967	-3.05410821643287\\
2.09218436873747	-3.05571690098827\\
2.08782724256107	-3.07014028056112\\
2.08282144178327	-3.08617234468938\\
2.07764187201208	-3.10220440881764\\
2.07615230460922	-3.10670417198628\\
2.0723199728081	-3.11823647294589\\
2.0668279322443	-3.13426853707415\\
2.06114596808467	-3.1503006012024\\
2.06012024048096	-3.15313095886327\\
2.05530808263616	-3.16633266533066\\
2.04927652476668	-3.18236472945892\\
2.04408817635271	-3.19572743614558\\
2.04304403780948	-3.19839679358717\\
2.03663353865448	-3.21442885771543\\
2.03000039851701	-3.23046092184369\\
2.02805611222445	-3.23505272624421\\
2.02316612665281	-3.24649298597194\\
2.01610504929561	-3.2625250501002\\
2.01202404809619	-3.27154119214257\\
2.00881293353938	-3.27855711422846\\
2.00128462031003	-3.29458917835671\\
1.99599198396794	-3.30553466284577\\
1.99350063178841	-3.31062124248497\\
1.98546057570419	-3.32665330661323\\
1.97995991983968	-3.33730692442868\\
1.97714210753077	-3.34268537074148\\
1.96853981202308	-3.35871743486974\\
1.96392785571142	-3.36708567002893\\
1.95963426252646	-3.374749498998\\
1.9504123103506	-3.39078156312625\\
1.94789579158317	-3.39506013426401\\
1.94085534303881	-3.40681362725451\\
1.93186372745491	-3.42138894870863\\
1.93094713526962	-3.42284569138277\\
1.92066168164344	-3.43887775551102\\
1.91583166332665	-3.44621303528219\\
1.90997946975807	-3.45490981963928\\
1.8997995991984	-3.46964207855806\\
1.89888029101296	-3.47094188376753\\
1.88732105838211	-3.48697394789579\\
1.88376753507014	-3.4917892515538\\
1.87527724755564	-3.50300601202405\\
1.86773547094188	-3.51273640977956\\
1.86271726148988	-3.5190380761523\\
1.85170340681363	-3.53256061063652\\
1.84959972635276	-3.53507014028056\\
1.8358740682194	-3.55110220440882\\
1.83567134268537	-3.55133486431535\\
1.82146947117734	-3.56713426853707\\
1.81963927855711	-3.56913032217164\\
1.80633755317203	-3.58316633266533\\
1.80360721442886	-3.58599389569402\\
1.79040271332929	-3.59919839679359\\
1.7875751503006	-3.6019764910627\\
1.77357601502004	-3.61523046092184\\
1.77154308617235	-3.6171235514231\\
1.75575261167513	-3.6312625250501\\
1.75551102204409	-3.63147552060405\\
1.73947895791583	-3.64508353874812\\
1.73675998401568	-3.64729458917836\\
1.72344689378758	-3.65797252578098\\
1.71645946836328	-3.66332665330661\\
1.70741482965932	-3.67016942626541\\
1.69466730704812	-3.67935871743487\\
1.69138276553106	-3.68169890685915\\
1.67535070140281	-3.69260532634724\\
1.67104927101087	-3.69539078156313\\
1.65931863727455	-3.70291133356174\\
1.64527066679244	-3.71142284569138\\
1.64328657314629	-3.71261420294734\\
1.62725450901804	-3.72177920887112\\
1.61670775334827	-3.72745490981964\\
1.61122244488978	-3.7303851331788\\
1.59519038076152	-3.73847872555087\\
1.5846136617279	-3.7434869739479\\
1.57915831663327	-3.74605536787514\\
1.56312625250501	-3.75315746679068\\
1.54767777357342	-3.75951903807615\\
1.54709418837675	-3.75975836220358\\
1.5310621242485	-3.76593753192175\\
1.51503006012024	-3.77164679629925\\
1.50316156487214	-3.77555110220441\\
1.49899799599198	-3.77691813026864\\
1.48296593186373	-3.7817922824152\\
1.46693386773547	-3.78624079510059\\
1.45090180360721	-3.7902776572105\\
1.44519057425423	-3.79158316633267\\
1.43486973947896	-3.79394442068093\\
1.4188376753507	-3.79723728898389\\
1.40280561122244	-3.80015222065808\\
1.38677354709419	-3.80270011516812\\
1.37074148296593	-3.80489119658113\\
1.35470941883768	-3.80673504142009\\
1.34536965782306	-3.80761523046092\\
1.33867735470942	-3.80824897603755\\
1.32264529058116	-3.80944066381979\\
1.30661322645291	-3.81030619218637\\
1.29058116232465	-3.81085272362304\\
1.27454909819639	-3.81108688513842\\
1.25851703406814	-3.81101478617959\\
1.24248496993988	-3.81064203520423\\
1.22645290581162	-3.80997375496874\\
1.21042084168337	-3.80901459658709\\
1.19438877755511	-3.80776875241063\\
1.19278466694236	-3.80761523046092\\
1.17835671342685	-3.80626041352538\\
1.1623246492986	-3.80447940168065\\
1.14629258517034	-3.80242643223183\\
1.13026052104208	-3.80010406348015\\
1.11422845691383	-3.79751444724034\\
1.09819639278557	-3.79465933573617\\
1.08238610784513	-3.79158316633267\\
1.08216432865731	-3.79154076292006\\
1.06613226452906	-3.78821192551389\\
1.0501002004008	-3.7846257924548\\
1.03406813627254	-3.78078268667116\\
1.01803607214429	-3.77668255792351\\
1.01387250326413	-3.77555110220441\\
1.00200400801603	-3.7723776273203\\
0.985971943887775	-3.76783838447288\\
0.969939879759519	-3.76304552242583\\
0.958735177018894	-3.75951903807615\\
0.953907815631262	-3.75802296121048\\
0.937875751503006	-3.75280765424696\\
0.921843687374749	-3.74733952723095\\
0.911043279321471	-3.7434869739479\\
0.905811623246493	-3.74164795998862\\
0.889779559118236	-3.73576945779861\\
0.87374749498998	-3.72963642279804\\
0.868262186531492	-3.72745490981964\\
0.857715430861724	-3.7233182685757\\
0.841683366733467	-3.71678256208697\\
0.829025370107676	-3.71142284569138\\
0.825651302605211	-3.71001314770832\\
0.809619238476954	-3.70308011609804\\
0.793587174348698	-3.69588653505746\\
0.792516323769337	-3.69539078156313\\
0.777555110220441	-3.68855112833831\\
0.761523046092185	-3.68096111671956\\
0.75823850457513	-3.67935871743487\\
0.745490981963928	-3.6732143754226\\
0.729458917835672	-3.66522904433948\\
0.725746813913297	-3.66332665330661\\
0.713426853707415	-3.65708548600821\\
0.697394789579159	-3.64870434829676\\
0.694768682811162	-3.64729458917836\\
0.681362725450902	-3.64017704005774\\
0.665330661322646	-3.63139803025136\\
0.665089071691608	-3.6312625250501\\
0.649298597194389	-3.62249864928523\\
0.636584308250443	-3.61523046092184\\
0.633266533066132	-3.61335323719904\\
0.617234468937876	-3.60405701947759\\
0.609073991017455	-3.59919839679359\\
0.601202404809619	-3.59455732732911\\
0.585170340681363	-3.58485599522903\\
0.582440001938186	-3.58316633266533\\
0.569138276553106	-3.57501041351771\\
0.556626181278274	-3.56713426853707\\
0.55310621242485	-3.56493835321383\\
0.537074148296593	-3.55471227402176\\
0.531542492394102	-3.55110220440882\\
0.521042084168337	-3.5443073060698\\
0.507113700500945	-3.53507014028056\\
0.50501002004008	-3.53368642806086\\
0.488977955911824	-3.52292736966742\\
0.483308498413443	-3.5190380761523\\
0.472945891783567	-3.51198384860222\\
0.460056211284139	-3.50300601202405\\
0.456913827655311	-3.50083355292383\\
0.440881763527054	-3.48952899261935\\
0.437328240215089	-3.48697394789579\\
0.424849699398798	-3.47806371092255\\
0.415086534647535	-3.47094188376754\\
0.408817635270541	-3.46639943697746\\
0.393288980010621	-3.45490981963928\\
0.392785571142285	-3.454539734248\\
0.376753507014028	-3.44255838458529\\
0.371923488697241	-3.43887775551102\\
0.360721442885771	-3.43039175147638\\
0.350951914584676	-3.42284569138276\\
0.344689378757515	-3.41803573511502\\
0.330353172608618	-3.40681362725451\\
0.328657314629258	-3.40549328796713\\
0.312625250501002	-3.39280623394943\\
0.310108731733566	-3.39078156312625\\
0.296593186372745	-3.37996057446673\\
0.29019815203592	-3.374749498998\\
0.280561122244489	-3.36693288932542\\
0.270605055257431	-3.35871743486974\\
0.264529058116232	-3.35372555976707\\
0.251314806296885	-3.34268537074148\\
0.248496993987976	-3.34034081811575\\
0.232464929859719	-3.32678327605539\\
0.232313243736369	-3.32665330661323\\
0.216432865731463	-3.31309537364555\\
0.213579442999435	-3.31062124248497\\
0.200400801603206	-3.2992324911091\\
0.195108165261117	-3.29458917835671\\
0.18436873747495	-3.28519632550059\\
0.176888389625251	-3.27855711422846\\
0.168336673346693	-3.27098843500974\\
0.158909757082701	-3.2625250501002\\
0.152304609218437	-3.25661024098888\\
0.141162530661818	-3.24649298597194\\
0.13627254509018	-3.24206302979891\\
0.123637558469039	-3.23046092184369\\
0.120240480961924	-3.22734795447686\\
0.10632623939084	-3.21442885771543\\
0.104208416833667	-3.21246603622831\\
0.0892204912486378	-3.19839679358717\\
0.0881763527054105	-3.19741816574722\\
0.0723127212168553	-3.18236472945892\\
0.0721442885771539	-3.18220510436568\\
0.0561122244488974	-3.16683736834846\\
0.055591656528809	-3.16633266533066\\
0.0400801603206409	-3.15130549133715\\
0.0390544327169332	-3.1503006012024\\
0.0240480961923843	-3.13560667108172\\
0.0226962325176485	-3.13426853707415\\
0.00801603206412782	-3.11974116466135\\
0.00651134034866946	-3.11823647294589\\
-0.00801603206412826	-3.10370910053309\\
-0.00950559946698502	-3.10220440881764\\
-0.0240480961923848	-3.08751047869695\\
-0.0253596009134542	-3.08617234468938\\
-0.0400801603206413	-3.07114517069587\\
-0.0410553532443371	-3.07014028056112\\
-0.0561122244488979	-3.05461291945066\\
-0.0565972367810961	-3.05410821643287\\
-0.0719910800676459	-3.03807615230461\\
-0.0721442885771544	-3.03791652721137\\
-0.087246414967761	-3.02204408817635\\
-0.0881763527054109	-3.0210654603364\\
-0.102361643310788	-3.0060120240481\\
-0.104208416833667	-3.00404920256098\\
-0.117340166624625	-2.98997995991984\\
-0.120240480961924	-2.98686699255302\\
-0.1321851375395	-2.97394789579158\\
-0.13627254509018	-2.96951793961855\\
-0.146899470581772	-2.95791583166333\\
-0.152304609218437	-2.95200102255201\\
-0.161485852207838	-2.94188376753507\\
-0.168336673346694	-2.93431508831635\\
-0.175946750125508	-2.92585170340681\\
-0.18436873747495	-2.91645885055069\\
-0.190284421946428	-2.90981963927856\\
-0.200400801603207	-2.89843088790269\\
-0.204500923209734	-2.8937875751503\\
-0.216432865731463	-2.88022964218263\\
-0.218598114813846	-2.87775551102204\\
-0.232464929859719	-2.86185341633595\\
-0.232577669890459	-2.86172344689379\\
-0.246472484612783	-2.84569138276553\\
-0.248496993987976	-2.84334683013979\\
-0.260257551260827	-2.82965931863727\\
-0.264529058116232	-2.8246674435346\\
-0.273932429644883	-2.81362725450902\\
-0.280561122244489	-2.80581064483644\\
-0.28749819455161	-2.79759519038076\\
-0.296593186372745	-2.78677420172123\\
-0.300955760353324	-2.7815631262525\\
-0.312625250501002	-2.76755573294743\\
-0.314305885389588	-2.76553106212425\\
-0.327567928056745	-2.74949899799599\\
-0.328657314629258	-2.74817865870861\\
-0.340755276724304	-2.73346693386774\\
-0.344689378757515	-2.72865697759999\\
-0.353840505499914	-2.71743486973948\\
-0.360721442885771	-2.70894886570484\\
-0.366823878678944	-2.70140280561122\\
-0.376753507014028	-2.68905137055723\\
-0.379705522527669	-2.68537074148297\\
-0.39249088285182	-2.66933867735471\\
-0.392785571142285	-2.66896859196343\\
-0.405231158769934	-2.65330661322645\\
-0.408817635270541	-2.64876416643638\\
-0.417873568327679	-2.6372745490982\\
-0.424849699398798	-2.62836431212495\\
-0.430417805075109	-2.62124248496994\\
-0.440881763527054	-2.60776546556524\\
-0.442863436587081	-2.60521042084168\\
-0.455242970624711	-2.58917835671343\\
-0.456913827655311	-2.5870058976132\\
-0.467564961912668	-2.57314629258517\\
-0.472945891783567	-2.56609206503508\\
-0.479790945645323	-2.55711422845691\\
-0.488977955911824	-2.54497145784377\\
-0.491920100199112	-2.54108216432866\\
-0.503972942874826	-2.5250501002004\\
-0.50501002004008	-2.5236663879807\\
-0.515990373212706	-2.50901803607214\\
-0.521042084168337	-2.50222313773313\\
-0.527912695607077	-2.49298597194389\\
-0.537074148296593	-2.48056397742858\\
-0.539738724534669	-2.47695390781563\\
-0.551501787790471	-2.46092184368737\\
-0.55310621242485	-2.45872592836413\\
-0.563226279677955	-2.44488977955912\\
-0.569138276553106	-2.43673386041149\\
-0.574855395806361	-2.42885771543086\\
-0.585170340681363	-2.4145153138663\\
-0.586387604466832	-2.41282565130261\\
-0.597895556901214	-2.39679358717435\\
-0.601202404809619	-2.39215251770987\\
-0.609334863426596	-2.38076152304609\\
-0.617234468937876	-2.36958808160184\\
-0.620677433211101	-2.36472945891784\\
-0.631951634761235	-2.34869739478958\\
-0.633266533066132	-2.34682017106677\\
-0.643206984131002	-2.33266533066132\\
-0.649298597194389	-2.32390145489646\\
-0.654365277727698	-2.31663326653307\\
-0.665330661322646	-2.30073670760607\\
-0.665424433461496	-2.30060120240481\\
-0.676498350458976	-2.28456913827655\\
-0.681362725450902	-2.27745158915594\\
-0.687476579602111	-2.2685370741483\\
-0.697394789579158	-2.25391476913844\\
-0.698354673058551	-2.25250501002004\\
-0.709232506939537	-2.23647294589178\\
-0.713426853707415	-2.23023177859338\\
-0.720033795682041	-2.22044088176353\\
-0.729458917835671	-2.20631120866813\\
-0.730733503473829	-2.20440881763527\\
-0.741430141279901	-2.18837675350701\\
-0.745490981963928	-2.18223241149475\\
-0.752056579978297	-2.17234468937876\\
-0.761523046092184	-2.15791502453519\\
-0.762579557512944	-2.1563126252505\\
-0.773109252646421	-2.14028056112224\\
-0.777555110220441	-2.13344090789742\\
-0.783561943357505	-2.12424849699399\\
-0.793587174348697	-2.10871218636007\\
-0.793908866757872	-2.10821643286573\\
-0.804285298864114	-2.09218436873747\\
-0.809619238476954	-2.08384163914414\\
-0.814564392751905	-2.07615230460922\\
-0.824758302252477	-2.06012024048096\\
-0.82565130260521	-2.0587105424979\\
-0.834971324123861	-2.04408817635271\\
-0.841683366733467	-2.03341582862003\\
-0.845076051383762	-2.02805611222445\\
-0.855137738611258	-2.01202404809619\\
-0.857715430861723	-2.00788740685225\\
-0.865178068571839	-1.99599198396794\\
-0.87374749498998	-1.98214143281808\\
-0.875106761295386	-1.97995991983968\\
-0.885048557799838	-1.96392785571142\\
-0.889779559118236	-1.95621033956214\\
-0.89491406096065	-1.94789579158317\\
-0.904694649819828	-1.93186372745491\\
-0.905811623246493	-1.93002471349564\\
-0.914498021268976	-1.91583166332665\\
-0.92184368737475	-1.90365215248145\\
-0.924185695363177	-1.8997995991984\\
-0.933865262670698	-1.88376753507014\\
-0.937875751503006	-1.87705615124094\\
-0.943491311068581	-1.86773547094188\\
-0.953022059741462	-1.85170340681363\\
-0.953907815631263	-1.85020732994795\\
-0.962587153792167	-1.83567134268537\\
-0.969939879759519	-1.8231657629068\\
-0.972031583136088	-1.81963927855711\\
-0.981479033512832	-1.80360721442886\\
-0.985971943887776	-1.79589449669732\\
-0.990862346907323	-1.7875751503006\\
-1.00017244570538	-1.77154308617234\\
-1.00200400801603	-1.76836961128823\\
-1.00949495205705	-1.75551102204409\\
-1.01803607214429	-1.74061041363493\\
-1.01869091350944	-1.73947895791583\\
-1.02793446764556	-1.72344689378758\\
-1.03406813627255	-1.71264641412607\\
-1.03706881500154	-1.70741482965932\\
-1.04618567492923	-1.69138276553106\\
-1.0501002004008	-1.68442539165319\\
-1.05525837210245	-1.67535070140281\\
-1.06425307777468	-1.65931863727455\\
-1.06613226452906	-1.65594739645577\\
-1.07326397108547	-1.64328657314629\\
-1.08214091242481	-1.62725450901804\\
-1.08216432865731	-1.62721210560543\\
-1.09108973294102	-1.61122244488978\\
-1.09819639278557	-1.59826655016503\\
-1.09990251902498	-1.59519038076152\\
-1.10873952224162	-1.57915831663327\\
-1.11422845691383	-1.56905753341268\\
-1.11748836383966	-1.56312625250501\\
-1.12621695542465	-1.54709418837675\\
-1.13026052104208	-1.53958302139598\\
-1.13490125784064	-1.5310621242485\\
-1.14352540851714	-1.51503006012024\\
-1.14629258517034	-1.50984126189114\\
-1.15214446671673	-1.49899799599198\\
-1.16066802432549	-1.48296593186373\\
-1.1623246492986	-1.47983010308346\\
-1.16922102401271	-1.46693386773547\\
-1.17764771911111	-1.45090180360721\\
-1.17835671342685	-1.44954698667167\\
-1.18613373755529	-1.43486973947896\\
-1.19438877755511	-1.41899119730041\\
-1.19446961549923	-1.4188376753507\\
-1.2028851953898	-1.40280561122244\\
-1.21042084168337	-1.38817291322035\\
-1.21115094474724	-1.38677354709419\\
-1.21947777124482	-1.37074148296593\\
-1.22645290581162	-1.35706794334549\\
-1.22767214575648	-1.35470941883768\\
-1.23591362954052	-1.33867735470942\\
-1.24248496993988	-1.32567209532447\\
-1.24403527399133	-1.32264529058116\\
-1.25219472995507	-1.30661322645291\\
-1.25851703406814	-1.29398071804332\\
-1.26024217985847	-1.29058116232465\\
-1.26832283156246	-1.27454909819639\\
-1.27454909819639	-1.26198868874564\\
-1.27629451261218	-1.25851703406814\\
-1.28429949655346	-1.24248496993988\\
-1.29058116232465	-1.22969039897374\\
-1.29219372384407	-1.22645290581162\\
-1.30012609355036	-1.21042084168337\\
-1.30661322645291	-1.19707973928056\\
-1.30794107056714	-1.19438877755511\\
-1.31580380052522	-1.17835671342685\\
-1.32264529058116	-1.16415008265747\\
-1.32353761790295	-1.1623246492986\\
-1.33133360732971	-1.14629258517034\\
-1.33867735470942	-1.13089426661871\\
-1.33898424137981	-1.13026052104208\\
-1.34671631784397	-1.11422845691383\\
-1.35429512759832	-1.09819639278557\\
-1.35470941883768	-1.09731620374474\\
-1.36195255175066	-1.08216432865731\\
-1.36947185626311	-1.06613226452906\\
-1.37074148296593	-1.06340823064926\\
-1.37704274593921	-1.0501002004008\\
-1.38450516462339	-1.03406813627255\\
-1.38677354709419	-1.02915302097975\\
-1.39198715554453	-1.01803607214429\\
-1.39939524815508	-1.00200400801603\\
-1.40280561122244	-0.994540998213195\\
-1.40678585462307	-0.985971943887776\\
-1.41414212425145	-0.969939879759519\\
-1.4188376753507	-0.959561938282485\\
-1.42143873646836	-0.953907815631263\\
-1.42874563242771	-0.937875751503006\\
-1.43486973947896	-0.924204941723015\\
-1.43594551356707	-0.92184368737475\\
-1.44320543417176	-0.905811623246493\\
-1.45032497946464	-0.889779559118236\\
-1.45090180360721	-0.888474049996076\\
-1.4575210124409	-0.87374749498998\\
-1.46459710752866	-0.857715430861723\\
-1.46693386773547	-0.852373059629644\\
-1.47169167080326	-0.841683366733467\\
-1.4787265569285	-0.82565130260521\\
-1.48296593186373	-0.815860418687748\\
-1.48571653222177	-0.809619238476954\\
-1.49271240871521	-0.793587174348697\\
-1.49899799599198	-0.778922138284666\\
-1.49959453747732	-0.777555110220441\\
-1.50655356408688	-0.761523046092184\\
-1.51338033327571	-0.745490981963928\\
-1.51503006012024	-0.74158667605877\\
-1.52024874236831	-0.729458917835671\\
-1.52704167434769	-0.713426853707415\\
-1.5310621242485	-0.703813283424752\\
-1.53379647862441	-0.697394789579158\\
-1.54055760413801	-0.681362725450902\\
-1.54709418837675	-0.665569985450078\\
-1.54719512090086	-0.665330661322646\\
-1.55392643861466	-0.649298597194389\\
-1.56053156729777	-0.633266533066132\\
-1.56312625250501	-0.626904961780658\\
-1.56714630555442	-0.617234468937876\\
-1.57372428420663	-0.601202404809619\\
-1.57915831663327	-0.587738734608611\\
-1.58021514095198	-0.585170340681363\\
-1.58676785760723	-0.569138276553106\\
-1.59319910861719	-0.55310621242485\\
-1.59519038076152	-0.548097964027827\\
-1.59966000209789	-0.537074148296593\\
-1.60606840850847	-0.521042084168337\\
-1.61122244488978	-0.50794024339924\\
-1.61239823384089	-0.50501002004008\\
-1.61878558460074	-0.488977955911824\\
-1.62505577178471	-0.472945891783567\\
-1.62725450901804	-0.467270190835046\\
-1.63134792809836	-0.456913827655311\\
-1.63759925671641	-0.440881763527054\\
-1.64328657314629	-0.426041056654751\\
-1.64375252317298	-0.424849699398798\\
-1.64998669320543	-0.408817635270541\\
-1.65610761012657	-0.392785571142285\\
-1.65931863727455	-0.38427405901264\\
-1.66221493444902	-0.376753507014028\\
-1.66832068705787	-0.360721442885771\\
-1.67431648597408	-0.344689378757515\\
-1.67535070140281	-0.341903923541631\\
-1.68037282410443	-0.328657314629258\\
-1.6863553242951	-0.312625250501002\\
-1.69138276553106	-0.298933375797024\\
-1.69226041730671	-0.296593186372745\\
-1.69823117961289	-0.280561122244489\\
-1.70409529753085	-0.264529058116232\\
-1.70741482965932	-0.255339766946777\\
-1.70994020395826	-0.248496993987976\\
-1.71579427084906	-0.232464929859719\\
-1.72154459568374	-0.216432865731463\\
-1.72344689378758	-0.211078738205828\\
-1.72732382048331	-0.200400801603207\\
-1.73306566673049	-0.18436873747495\\
-1.73870651950802	-0.168336673346694\\
-1.73947895791583	-0.166125622916459\\
-1.74441441207152	-0.152304609218437\\
-1.75004824908651	-0.13627254509018\\
-1.75551102204409	-0.120453476515871\\
-1.75558622659312	-0.120240480961924\\
-1.76121443676716	-0.104208416833667\\
-1.76674422382965	-0.0881763527054109\\
-1.77154308617234	-0.0740373790784148\\
-1.77220020625581	-0.0721442885771544\\
-1.77772567498475	-0.0561122244488979\\
-1.78315512949002	-0.0400801603206413\\
-1.7875751503006	-0.0268261904615012\\
-1.78852289179611	-0.0240480961923848\\
-1.7939492363902	-0.00801603206412826\\
-1.79928184315911	0.00801603206412782\\
-1.80360721442886	0.0212205331636953\\
-1.80455495592437	0.0240480961923843\\
-1.80988556418485	0.0400801603206409\\
-1.81512458387508	0.0561122244488974\\
-1.81963927855711	0.0701482349425853\\
-1.82029639864058	0.0721442885771539\\
-1.82553443681551	0.0881763527054105\\
-1.83068291346776	0.104208416833667\\
-1.83567134268537	0.120007821055389\\
-1.8357465472344	0.120240480961924\\
-1.84089496711966	0.13627254509018\\
-1.84595573486811	0.152304609218437\\
-1.85093096840582	0.168336673346693\\
-1.85170340681363	0.170846202990731\\
-1.85596559890135	0.18436873747495\\
-1.86094128787276	0.200400801603206\\
-1.86583317283805	0.216432865731463\\
-1.86773547094188	0.222734532104208\\
-1.87074410091915	0.232464929859719\\
-1.875637142341	0.248496993987976\\
-1.88044800294167	0.264529058116232\\
-1.88376753507014	0.27574581858648\\
-1.88522755825375	0.280561122244489\\
-1.89004018878716	0.296593186372745\\
-1.89477215796244	0.312625250501002\\
-1.89942534149261	0.328657314629258\\
-1.8997995991984	0.329957119838733\\
-1.90414662631756	0.344689378757515\\
-1.90880164898172	0.360721442885771\\
-1.9133792480818	0.376753507014028\\
-1.91583166332665	0.385450291371114\\
-1.91795194784638	0.392785571142285\\
-1.92253178338579	0.408817635270541\\
-1.92703546142494	0.424849699398798\\
-1.93146469803591	0.440881763527054\\
-1.93186372745491	0.442338506201192\\
-1.93595714653524	0.456913827655311\\
-1.94038838530739	0.472945891783567\\
-1.94474630364003	0.488977955911824\\
-1.94789579158317	0.500731448902319\\
-1.94907158053428	0.50501002004008\\
-1.95343168096556	0.521042084168337\\
-1.95771949174352	0.537074148296593\\
-1.96193658356709	0.55310621242485\\
-1.96392785571142	0.560770041393919\\
-1.96615824208962	0.569138276553106\\
-1.97037697660293	0.585170340681363\\
-1.97452588741304	0.601202404809619\\
-1.97860645615369	0.617234468937876\\
-1.97995991983968	0.622612915250676\\
-1.98271067567622	0.633266533066132\\
-1.98679217007759	0.649298597194389\\
-1.99080608711256	0.665330661322646\\
-1.99475382388421	0.681362725450902\\
-1.99599198396794	0.686449305090098\\
-1.99872633834385	0.697394789579159\\
-2.00267410246425	0.713426853707415\\
-2.0065563245599	0.729458917835672\\
-2.01037432125166	0.745490981963928\\
-2.01202404809619	0.752506904049819\\
-2.0142003569969	0.761523046092185\\
-2.01801729618351	0.777555110220441\\
-2.02177052494767	0.793587174348698\\
-2.02546128301537	0.809619238476954\\
-2.02805611222445	0.821059498204689\\
-2.02912549303033	0.825651302605211\\
-2.03281391529224	0.841683366733467\\
-2.03644026093137	0.857715430861724\\
-2.04000569592562	0.87374749498998\\
-2.04351135221013	0.889779559118236\\
-2.04408817635271	0.892448916559834\\
-2.04705370876431	0.905811623246493\\
-2.05055469160957	0.921843687374749\\
-2.05399613342972	0.937875751503006\\
-2.05737909650658	0.953907815631262\\
-2.06012024048096	0.96710952209865\\
-2.06072393619058	0.969939879759519\\
-2.06410048388159	0.985971943887775\\
-2.06741867270145	1.00200400801603\\
-2.07067949744408	1.01803607214429\\
-2.07388392213842	1.03406813627254\\
-2.07615230460922	1.04560043723216\\
-2.0770617153881	1.0501002004008\\
-2.08025679315444	1.06613226452906\\
-2.0833954375222	1.08216432865731\\
-2.08647854783053	1.09819639278557\\
-2.08950699428547	1.11422845691383\\
-2.09218436873747	1.12865183648668\\
-2.09249125540787	1.13026052104208\\
-2.09550706610889	1.14629258517034\\
-2.09846802604284	1.1623246492986\\
-2.10137494280979	1.17835671342685\\
-2.10422859624857	1.19438877755511\\
-2.10702973894416	1.21042084168337\\
-2.10821643286573	1.21732530834615\\
-2.10982899438516	1.22645290581162\\
-2.11261272985613	1.24248496993988\\
-2.11534357785558	1.25851703406814\\
-2.11802223036006	1.27454909819639\\
-2.12064935314585	1.29058116232465\\
-2.1232255862079	1.30661322645291\\
-2.12424849699399	1.31309304891348\\
-2.12579880104544	1.32264529058116\\
-2.12835227738831	1.33867735470942\\
-2.13085429246183	1.35470941883768\\
-2.13330542655544	1.37074148296593\\
-2.13570623492495	1.38677354709419\\
-2.13805724812759	1.40280561122244\\
-2.14028056112224	1.41829016635702\\
-2.14036139906636	1.4188376753507\\
-2.14268365816133	1.43486973947896\\
-2.14495534686681	1.45090180360721\\
-2.14717693583636	1.46693386773547\\
-2.14934887147192	1.48296593186373\\
-2.15147157617786	1.49899799599198\\
-2.1535454485973	1.51503006012024\\
-2.15557086383083	1.5310621242485\\
-2.1563126252505	1.53707027093956\\
-2.15758539475005	1.54709418837675\\
-2.15957253217634	1.56312625250501\\
-2.16151014528104	1.57915831663327\\
-2.1633985476971	1.59519038076152\\
-2.1652380295342	1.61122244488978\\
-2.16702885751804	1.62725450901804\\
-2.16877127511243	1.64328657314629\\
-2.17046550262438	1.65931863727455\\
-2.17211173729216	1.67535070140281\\
-2.17234468937876	1.67768616266539\\
-2.17374993028851	1.69138276553106\\
-2.17534536810776	1.70741482965932\\
-2.17689138091722	1.72344689378758\\
-2.17838810008491	1.73947895791583\\
-2.17983563341978	1.75551102204409\\
-2.18123406517753	1.77154308617235\\
-2.18258345604918	1.7875751503006\\
-2.18388384313207	1.80360721442886\\
-2.1851352398834	1.81963927855711\\
-2.18633763605613	1.83567134268537\\
-2.18749099761721	1.85170340681363\\
-2.18837675350701	1.8645652151021\\
-2.18860133989036	1.86773547094188\\
-2.18968554512439	1.88376753507014\\
-2.19071876149544	1.8997995991984\\
-2.19170087930555	1.91583166332665\\
-2.19263176387134	1.93186372745491\\
-2.19351125534943	1.94789579158317\\
-2.19433916854262	1.96392785571142\\
-2.19511529268665	1.97995991983968\\
-2.19583939121713	1.99599198396794\\
-2.19651120151655	2.01202404809619\\
-2.19713043464091	2.02805611222445\\
-2.19769677502566	2.04408817635271\\
-2.19820988017068	2.06012024048096\\
-2.19866938030378	2.07615230460922\\
-2.19907487802243	2.09218436873747\\
-2.19942594791327	2.10821643286573\\
-2.19972213614885	2.12424849699399\\
-2.19996296006125	2.14028056112224\\
-2.20014790769201	2.1563126252505\\
-2.20027643731779	2.17234468937876\\
-2.20034797695124	2.18837675350701\\
-2.20036192381657	2.20440881763527\\
-2.20031764379895	2.22044088176353\\
-2.20021447086739	2.23647294589178\\
-2.20005170647016	2.25250501002004\\
-2.19982861890213	2.2685370741483\\
-2.19954444264334	2.28456913827655\\
-2.19919837766783	2.30060120240481\\
-2.19878958872204	2.31663326653307\\
-2.19831720457188	2.33266533066132\\
-2.19778031721757	2.34869739478958\\
-2.19717798107516	2.36472945891784\\
-2.19650921212399	2.38076152304609\\
-2.19577298701883	2.39679358717435\\
-2.1949682421657	2.41282565130261\\
-2.19409387276027	2.42885771543086\\
-2.19314873178755	2.44488977955912\\
-2.19213162898178	2.46092184368737\\
-2.19104132974509	2.47695390781563\\
-2.1898765540237	2.49298597194389\\
-2.18863597514014	2.50901803607214\\
-2.18837675350701	2.51218829191193\\
}--cycle;


\addplot[area legend,solid,fill=mycolor3,draw=black,forget plot]
table[row sep=crcr] {%
x	y\\
-1.85170340681363	2.08110478474631\\
-1.85096402425816	2.09218436873747\\
-1.84980986194504	2.10821643286573\\
-1.84856756467358	2.12424849699399\\
-1.84723538393176	2.14028056112224\\
-1.84581150654969	2.1563126252505\\
-1.84429405280818	2.17234468937876\\
-1.84268107447124	2.18837675350701\\
-1.8409705527399	2.20440881763527\\
-1.83916039612411	2.22044088176353\\
-1.83724843822968	2.23647294589178\\
-1.83567134268537	2.2490366856574\\
-1.83523737101778	2.25250501002004\\
-1.83313835568119	2.2685370741483\\
-1.83093089661092	2.28456913827655\\
-1.82861245985542	2.30060120240481\\
-1.82618042239087	2.31663326653307\\
-1.82363206922794	2.33266533066132\\
-1.82096459040213	2.34869739478958\\
-1.81963927855711	2.35635177958395\\
-1.81818857080282	2.36472945891784\\
-1.81529996453305	2.38076152304609\\
-1.8122831122404	2.39679358717435\\
-1.80913473079258	2.41282565130261\\
-1.80585142247804	2.42885771543086\\
-1.80360721442886	2.43941001692876\\
-1.8024385369892	2.44488977955912\\
-1.79890051672139	2.46092184368737\\
-1.79521614279353	2.47695390781563\\
-1.79138143889474	2.49298597194389\\
-1.7875751503006	2.50829095273659\\
-1.78739338660917	2.50901803607214\\
-1.78326946333912	2.5250501002004\\
-1.77898147439258	2.54108216432866\\
-1.7745247538334	2.55711422845691\\
-1.77154308617234	2.56748238129583\\
-1.76990175838651	2.57314629258517\\
-1.76511267720294	2.58917835671343\\
-1.7601383404456	2.60521042084168\\
-1.75551102204409	2.61959233253044\\
-1.75497487581311	2.62124248496994\\
-1.7496281134607	2.6372745490982\\
-1.74407694612616	2.65330661322645\\
-1.73947895791583	2.66613514873682\\
-1.73831692392196	2.66933867735471\\
-1.73234548685442	2.68537074148297\\
-1.72614739434424	2.70140280561122\\
-1.72344689378758	2.70818987746438\\
-1.71971566625049	2.71743486973948\\
-1.71304013232617	2.73346693386774\\
-1.70741482965932	2.74652047249504\\
-1.7061103722011	2.74949899799599\\
-1.69891012383408	2.76553106212425\\
-1.69143610370506	2.7815631262525\\
-1.69138276553106	2.78167509565841\\
-1.68365691788803	2.79759519038076\\
-1.67557984389159	2.81362725450902\\
-1.67535070140281	2.81407237793107\\
-1.66715968749157	2.82965931863727\\
-1.65931863727455	2.84405724240068\\
-1.65840866361447	2.84569138276553\\
-1.64927335199124	2.86172344689379\\
-1.64328657314629	2.87190089013585\\
-1.6397574132464	2.87775551102204\\
-1.62982373029581	2.8937875751503\\
-1.62725450901804	2.89783144559011\\
-1.61943156879626	2.90981963927856\\
-1.61122244488978	2.92202643431353\\
-1.60857531675665	2.92585170340681\\
-1.59719561432098	2.94188376753507\\
-1.59519038076152	2.94464271289486\\
-1.58523875260076	2.95791583166333\\
-1.57915831663327	2.9658107404219\\
-1.5726802746891	2.97394789579158\\
-1.56312625250501	2.98564600747063\\
-1.55945965112415	2.98997995991984\\
-1.54709418837675	3.00424549977693\\
-1.5455044615911	3.0060120240481\\
-1.5310621242485	3.02169538090901\\
-1.5307279572433	3.02204408817635\\
-1.51503006012024	3.03807216018363\\
-1.51502598115792	3.03807615230461\\
-1.49899799599198	3.05344370640559\\
-1.49827304395488	3.05410821643287\\
-1.48296593186373	3.06787012941409\\
-1.48031727468776	3.07014028056112\\
-1.46693386773547	3.08140454787408\\
-1.46097385579681	3.08617234468938\\
-1.45090180360721	3.09409375882733\\
-1.44001638721062	3.10220440881764\\
-1.43486973947896	3.10597882209824\\
-1.4188376753507	3.11710033903772\\
-1.41710903881226	3.11823647294589\\
-1.40280561122244	3.12750636962587\\
-1.39167734969043	3.13426853707415\\
-1.38677354709419	3.1372101514459\\
-1.37074148296593	3.14625626530284\\
-1.36307656718187	3.1503006012024\\
-1.35470941883768	3.15466661391826\\
-1.33867735470942	3.16246635164411\\
-1.33012237525181	3.16633266533066\\
-1.32264529058116	3.16968031772411\\
-1.30661322645291	3.17633463965954\\
-1.29075325444766	3.18236472945892\\
-1.29058116232465	3.18242966225713\\
-1.27454909819639	3.18802432785573\\
-1.25851703406814	3.19309930910985\\
-1.24248496993988	3.19767335714685\\
-1.23967836172005	3.19839679358717\\
-1.22645290581162	3.2017900554151\\
-1.21042084168337	3.20544337144136\\
-1.19438877755511	3.20864327520901\\
-1.17835671342685	3.21140421553854\\
-1.1623246492986	3.21373973178891\\
-1.15661502706466	3.21442885771543\\
-1.14629258517034	3.215673431823\\
-1.13026052104208	3.21720929975852\\
-1.11422845691383	3.21835227700143\\
-1.09819639278557	3.21911252681467\\
-1.08216432865731	3.21949948728803\\
-1.06613226452906	3.21952189989994\\
-1.0501002004008	3.21918783601784\\
-1.03406813627255	3.21850472144465\\
-1.01803607214429	3.21747935911114\\
-1.00200400801603	3.21611795000566\\
-0.985998095749708	3.21442885771543\\
-0.985971943887776	3.21442614155896\\
-0.969939879759519	3.21243069533663\\
-0.953907815631263	3.21011861233941\\
-0.937875751503006	3.2074939838205\\
-0.92184368737475	3.20456038185535\\
-0.905811623246493	3.20132087162849\\
-0.892586167338064	3.19839679358717\\
-0.889779559118236	3.19778513500252\\
-0.87374749498998	3.19398599487374\\
-0.857715430861723	3.18989208776193\\
-0.841683366733467	3.18550468284531\\
-0.830931707074993	3.18236472945892\\
-0.82565130260521	3.18084309491341\\
-0.809619238476954	3.17593153597101\\
-0.793587174348697	3.17073286629194\\
-0.780729021259947	3.16633266533066\\
-0.777555110220441	3.16526011168692\\
-0.761523046092184	3.15955813639492\\
-0.745490981963928	3.15357218820123\\
-0.737123833619732	3.1503006012024\\
-0.729458917835671	3.14733881557017\\
-0.713426853707415	3.14086475809335\\
-0.697777947004	3.13426853707415\\
-0.697394789579158	3.13410884736716\\
-0.681362725450902	3.12715612856513\\
-0.665330661322646	3.11991987857457\\
-0.661735583422272	3.11823647294589\\
-0.649298597194389	3.11247386624583\\
-0.633266533066132	3.10476552161437\\
-0.628119885334472	3.10220440881764\\
-0.617234468937876	3.09684155551826\\
-0.601202404809619	3.08866524528832\\
-0.596468517478442	3.08617234468938\\
-0.585170340681363	3.08027895349741\\
-0.569138276553106	3.07163682472855\\
-0.566442776085726	3.07014028056112\\
-0.55310621242485	3.06280212052261\\
-0.537799100333697	3.05410821643287\\
-0.537074148296593	3.05370005159222\\
-0.521042084168337	3.0444235273177\\
-0.510393744430537	3.03807615230461\\
-0.50501002004008	3.03489325179561\\
-0.488977955911824	3.02515213976752\\
-0.483995866906375	3.02204408817635\\
-0.472945891783567	3.01520356570113\\
-0.458503554440966	3.0060120240481\\
-0.456913827655311	3.00500778702454\\
-0.440881763527054	2.99463500693657\\
-0.433868346994613	2.98997995991984\\
-0.424849699398798	2.98403536906207\\
-0.409932488193913	2.97394789579158\\
-0.408817635270541	2.97319901951363\\
-0.392785571142285	2.96219094658839\\
-0.386705135174792	2.95791583166333\\
-0.376753507014028	2.95096193545089\\
-0.364065662450606	2.94188376753507\\
-0.360721442885771	2.93950507801007\\
-0.344689378757515	2.92785282855784\\
-0.341992196266275	2.92585170340681\\
-0.328657314629258	2.91601138957255\\
-0.32044819072278	2.90981963927856\\
-0.312625250501002	2.90394942268957\\
-0.299368989727407	2.8937875751503\\
-0.296593186372745	2.89167009450756\\
-0.280561122244489	2.87919735205685\\
-0.278741749184476	2.87775551102204\\
-0.264529058116232	2.86654115588278\\
-0.258542279271288	2.86172344689379\\
-0.248496993987976	2.85367293277856\\
-0.238727092594649	2.84569138276553\\
-0.232464929859719	2.84059517615725\\
-0.219276137482059	2.82965931863727\\
-0.216432865731463	2.82731020824735\\
-0.200400801603207	2.81382306215925\\
-0.200171659114423	2.81362725450902\\
-0.18436873747495	2.80016487245945\\
-0.181403709616206	2.79759519038076\\
-0.168336673346694	2.78630243725173\\
-0.162943743482435	2.7815631262525\\
-0.152304609218437	2.77223744705891\\
-0.14477725091542	2.76553106212425\\
-0.13627254509018	2.75797143184814\\
-0.126890685934725	2.74949899799599\\
-0.120240480961924	2.74350576301438\\
-0.109271399852711	2.73346693386774\\
-0.104208416833667	2.72884165515535\\
-0.0919075802425003	2.71743486973948\\
-0.0881763527054109	2.71398016764136\\
-0.0747881948941846	2.70140280561122\\
-0.0721442885771544	2.69892220598275\\
-0.0579029403073767	2.68537074148297\\
-0.0561122244488979	2.68366852299741\\
-0.0412421943145111	2.66933867735471\\
-0.0400801603206413	2.66821971978054\\
-0.0247969724703408	2.65330661322645\\
-0.0240480961923848	2.65257624647837\\
-0.00855888787954408	2.6372745490982\\
-0.00801603206412826	2.63673840286722\\
0.00747988583314737	2.62124248496994\\
0.00801603206412782	2.62070633873896\\
0.0233266486770152	2.60521042084168\\
0.0240480961923843	2.6044800540936\\
0.0389882058144757	2.58917835671343\\
0.0400801603206409	2.58805939913926\\
0.0544708966630596	2.57314629258517\\
0.0561122244488974	2.57144407409961\\
0.0697806217774118	2.55711422845691\\
0.0721442885771539	2.55463362882844\\
0.0849228676899447	2.54108216432866\\
0.0881763527054105	2.53762746223054\\
0.0999027298721873	2.5250501002004\\
0.104208416833667	2.52042482148801\\
0.114724933963928	2.50901803607214\\
0.120240480961924	2.50302480109053\\
0.129393855403837	2.49298597194389\\
0.13627254509018	2.48542634166778\\
0.143913537583111	2.47695390781563\\
0.152304609218437	2.46762822862203\\
0.158287708632811	2.46092184368737\\
0.168336673346693	2.44962909055835\\
0.172519796945656	2.44488977955912\\
0.18436873747495	2.43142739750955\\
0.186612945524132	2.42885771543086\\
0.200400801603206	2.41302145895284\\
0.200570025238683	2.41282565130261\\
0.214411163717496	2.39679358717435\\
0.216432865731463	2.39444447678443\\
0.22812561583566	2.38076152304609\\
0.232464929859719	2.37566531643781\\
0.241714242995323	2.36472945891784\\
0.248496993987976	2.35667894480261\\
0.255179057332447	2.34869739478958\\
0.264529058116232	2.33748303965031\\
0.268521848787055	2.33266533066132\\
0.280561122244489	2.31807510756787\\
0.281744193712368	2.31663326653307\\
0.294865693023195	2.30060120240481\\
0.296593186372745	2.29848372176207\\
0.307884804426556	2.28456913827655\\
0.312625250501002	2.27869892168757\\
0.320790737016945	2.2685370741483\\
0.328657314629258	2.25869676031403\\
0.333584388029848	2.25250501002004\\
0.344689378757515	2.23847407104281\\
0.346266474301825	2.23647294589178\\
0.358859841022653	2.22044088176353\\
0.360721442885771	2.21806219223852\\
0.37136467745965	2.20440881763527\\
0.376753507014028	2.19745492142283\\
0.383763238799898	2.18837675350701\\
0.392785571142285	2.17661980430382\\
0.396055667160588	2.17234468937876\\
0.408249324480346	2.1563126252505\\
0.408817635270541	2.15556374897255\\
0.420381676516927	2.14028056112224\\
0.424849699398798	2.13433597026447\\
0.432411864046547	2.12424849699399\\
0.440881763527054	2.11287147988246\\
0.444339516511447	2.10821643286573\\
0.456174445099843	2.09218436873747\\
0.456913827655311	2.09118013171392\\
0.467956638280627	2.07615230460922\\
0.472945891783567	2.069311782134\\
0.47963909184413	2.06012024048096\\
0.488977955911824	2.04719622794388\\
0.491220970321726	2.04408817635271\\
0.502735279239629	2.02805611222445\\
0.50501002004008	2.02487321171545\\
0.514184056143404	2.01202404809619\\
0.521042084168337	2.00233935898103\\
0.525533998792645	1.99599198396794\\
0.536788276952951	1.97995991983968\\
0.537074148296593	1.97955175499903\\
0.548013140606755	1.96392785571142\\
0.55310621242485	1.95658969567291\\
0.559140227809762	1.94789579158317\\
0.569138276553106	1.93336027162234\\
0.570167995371698	1.93186372745491\\
0.581160780256625	1.91583166332665\\
0.585170340681363	1.90993827213469\\
0.592072506252331	1.8997995991984\\
0.601202404809619	1.88626043566908\\
0.60288506512914	1.88376753507014\\
0.613657428555809	1.86773547094188\\
0.617234468937876	1.86237261764251\\
0.624359807395399	1.85170340681363\\
0.633266533066132	1.8382324554821\\
0.634962578770195	1.83567134268537\\
0.645529835186418	1.81963927855711\\
0.649298597194389	1.81387667185705\\
0.65602749226157	1.80360721442886\\
0.665330661322646	1.78925855592928\\
0.666424536732041	1.7875751503006\\
0.676801290439637	1.77154308617235\\
0.681362725450902	1.76443067766332\\
0.68709754087772	1.75551102204409\\
0.697293488050765	1.73947895791583\\
0.697394789579159	1.73931926820884\\
0.707491839219817	1.72344689378758\\
0.713426853707415	1.71401105067851\\
0.717588754260136	1.70741482965932\\
0.727616224980459	1.69138276553106\\
0.729458917835672	1.68842097989882\\
0.737618467272952	1.67535070140281\\
0.745490981963928	1.66259022427337\\
0.747516929577984	1.65931863727455\\
0.757388887898833	1.64328657314629\\
0.761523046092185	1.63651204421055\\
0.767195261899139	1.62725450901804\\
0.776907704806536	1.61122244488978\\
0.777555110220441	1.61014989124604\\
0.786623917378866	1.59519038076152\\
0.793587174348698	1.58355851759455\\
0.796233549095041	1.57915831663327\\
0.8058099752019	1.56312625250501\\
0.809619238476954	1.5566930590171\\
0.815331077458949	1.54709418837675\\
0.824760127386751	1.5310621242485\\
0.825651302605211	1.52954048970299\\
0.834194099609312	1.51503006012024\\
0.841683366733467	1.50213794937837\\
0.843517559081993	1.49899799599198\\
0.852828753689511	1.48296593186373\\
0.857715430861724	1.47446122603849\\
0.862065825221398	1.46693386773547\\
0.871240825915441	1.45090180360721\\
0.87374749498998	1.44649100489378\\
0.880392419679554	1.43486973947896\\
0.889435763559499	1.4188376753507\\
0.889779559118236	1.41822601676604\\
0.898502620912698	1.40280561122244\\
0.905811623246493	1.38969762513551\\
0.907453086973702	1.38677354709419\\
0.916401385107147	1.37074148296593\\
0.921843687374749	1.36087300710585\\
0.925266721059645	1.35470941883768\\
0.934093357167528	1.33867735470942\\
0.937875751503006	1.33174248081449\\
0.942873854060321	1.32264529058116\\
0.951582880981771	1.30661322645291\\
0.953907815631262	1.30230298107688\\
0.960278674444841	1.29058116232465\\
0.968874008993936	1.27454909819639\\
0.969939879759519	1.27255093581759\\
0.977485082284088	1.25851703406814\\
0.985970511113727	1.24248496993988\\
0.985971943887775	1.24248225378341\\
0.99449669779832	1.22645290581162\\
1.00200400801603	1.21210993397359\\
1.00289562534923	1.21042084168337\\
1.01131686927922	1.19438877755511\\
1.01803607214429	1.18140721482257\\
1.0196288895712	1.17835671342685\\
1.02794868041008	1.1623246492986\\
1.03406813627254	1.15036844889956\\
1.03617320976842	1.14629258517034\\
1.04439495700578	1.13026052104208\\
1.0501002004008	1.11898743521624\\
1.05253126766382	1.11422845691383\\
1.06065827319264	1.09819639278557\\
1.06613226452906	1.08725737084183\\
1.06870549438125	1.08216432865731\\
1.07674095704631	1.06613226452906\\
1.08216432865731	1.0551708299734\\
1.0846980757674	1.0501002004008\\
1.09264509570445	1.03406813627254\\
1.09819639278557	1.02271974124353\\
1.10051095719356	1.01803607214429\\
1.10837253996919	1.00200400801603\\
1.11422845691383	0.989895363173775\\
1.11614584785142	0.985971943887775\\
1.12392490841295	0.969939879759519\\
1.13026052104208	0.956688257674352\\
1.13160422455531	0.953907815631262\\
1.13930359099971	0.937875751503006\\
1.14629258517034	0.92308826148232\\
1.14688733506242	0.921843687374749\\
1.15450975223249	0.905811623246493\\
1.16200401996448	0.889779559118236\\
1.1623246492986	0.889090433191721\\
1.16954433383618	0.87374749498998\\
1.17696578019152	0.857715430861724\\
1.17835671342685	0.854690788684829\\
1.18440805698392	0.841683366733467\\
1.19175920173928	0.825651302605211\\
1.19438877755511	0.819865720098788\\
1.19910142407344	0.809619238476954\\
1.20638471539175	0.793587174348698\\
1.21042084168337	0.784601688074628\\
1.21362472005891	0.777555110220441\\
1.22084253737021	0.761523046092185\\
1.22645290581162	0.748884243791853\\
1.22797801334242	0.745490981963928\\
1.23513267009027	0.729458917835672\\
1.24216907752019	0.713426853707415\\
1.24248496993988	0.712703417267096\\
1.24925490248686	0.697394789579159\\
1.2562314894554	0.681362725450902\\
1.25851703406814	0.676065240973576\\
1.26320880990842	0.665330661322646\\
1.27012770666234	0.649298597194389\\
1.27454909819639	0.63892613146295\\
1.27699375358044	0.633266533066132\\
1.28385703316352	0.617234468937876\\
1.29058116232465	0.60126733760783\\
1.29060887963702	0.601202404809619\\
1.29741856006962	0.585170340681363\\
1.30411699672397	0.569138276553106\\
1.30661322645291	0.56310818675373\\
1.31081116406888	0.55310621242485\\
1.31745878918003	0.537074148296593\\
1.32264529058116	0.524389736561787\\
1.32403350548239	0.521042084168337\\
1.33063218027391	0.50501002004008\\
1.33712404097122	0.488977955911824\\
1.33867735470942	0.485111642225269\\
1.34363556339904	0.472945891783567\\
1.3500809692004	0.456913827655311\\
1.35470941883768	0.445247776242905\\
1.35646711206576	0.440881763527054\\
1.36286777974099	0.424849699398798\\
1.3691656409182	0.408817635270541\\
1.37074148296593	0.404773299370973\\
1.37548237873076	0.392785571142285\\
1.38173772597015	0.376753507014028\\
1.38677354709419	0.363663057257521\\
1.3879224466588	0.360721442885771\\
1.39413686273957	0.344689378757515\\
1.40025208659613	0.328657314629258\\
1.40280561122244	0.32189514718098\\
1.40636046095951	0.312625250501002\\
1.41243672444534	0.296593186372745\\
1.41841668542161	0.280561122244489\\
1.4188376753507	0.279424988336313\\
1.42444445727106	0.264529058116232\\
1.43038727181292	0.248496993987976\\
1.43486973947896	0.236239343140326\\
1.43627218400033	0.232464929859719\\
1.44217922799749	0.216432865731463\\
1.44799298112941	0.200400801603206\\
1.45090180360721	0.192290151612905\\
1.4537891721569	0.18436873747495\\
1.45956873876195	0.168336673346693\\
1.46525751441528	0.152304609218437\\
1.46693386773547	0.147536812403136\\
1.47096011403291	0.13627254509018\\
1.47661614380373	0.120240480961924\\
1.48218373363104	0.104208416833667\\
1.48296593186373	0.101938265686634\\
1.48778765267569	0.0881763527054105\\
1.49332379819697	0.0721442885771539\\
1.49877371175641	0.0561122244488974\\
1.49899799599198	0.0554477144216261\\
1.50427363149831	0.0400801603206409\\
1.50969326695727	0.0240480961923843\\
1.51502874080229	0.00801603206412782\\
1.51503006012024	0.00801203994315158\\
1.52041910140116	-0.00801603206412826\\
1.52572533108552	-0.0240480961923848\\
1.53094933679874	-0.0400801603206413\\
1.5310621242485	-0.0404288675879876\\
1.53622432482812	-0.0561122244488979\\
1.54141999058174	-0.0721442885771544\\
1.546535241786	-0.0881763527054109\\
1.54709418837675	-0.0899428769765749\\
1.551688776779	-0.104208416833667\\
1.55677646444501	-0.120240480961924\\
1.56178542281927	-0.13627254509018\\
1.56312625250501	-0.140606497539393\\
1.56681114278396	-0.152304609218437\\
1.57179318765974	-0.168336673346694\\
1.57669806798192	-0.18436873747495\\
1.57915831663327	-0.192505892844636\\
1.58158931382882	-0.200400801603207\\
1.5864678051518	-0.216432865731463\\
1.59127057938552	-0.232464929859719\\
1.59519038076152	-0.245738048628188\\
1.59602037820382	-0.248496993987976\\
1.60079716268189	-0.264529058116232\\
1.6054995631189	-0.280561122244489\\
1.61012914769989	-0.296593186372745\\
1.61122244488978	-0.300418455466027\\
1.61477729462685	-0.312625250501002\\
1.61938081609172	-0.328657314629258\\
1.62391268479772	-0.344689378757515\\
1.62725450901804	-0.356677572445968\\
1.62840340858265	-0.360721442885771\\
1.63290930970114	-0.376753507014028\\
1.63734461298571	-0.392785571142285\\
1.64171073109856	-0.408817635270541\\
1.64328657314629	-0.414672256156732\\
1.64607917023272	-0.424849699398798\\
1.65041882425813	-0.440881763527054\\
1.65469018763728	-0.456913827655311\\
1.6588945800742	-0.472945891783567\\
1.65931863727455	-0.474580032148421\\
1.66312834154604	-0.488977955911824\\
1.66730552696729	-0.50501002004008\\
1.67141648023444	-0.521042084168337\\
1.67535070140281	-0.536629024874543\\
1.67546529128451	-0.537074148296593\\
1.67954863901878	-0.55310621242485\\
1.68356639100834	-0.569138276553106\\
1.68751972819312	-0.585170340681363\\
1.69138276553106	-0.601090435403715\\
1.69141048284343	-0.601202404809619\\
1.69533503186013	-0.617234468937876\\
1.69919565132172	-0.633266533066132\\
1.70299343812526	-0.649298597194389\\
1.70672945483409	-0.665330661322646\\
1.70741482965932	-0.668309186823601\\
1.71047970488146	-0.681362725450902\\
1.7141847622063	-0.697394789579158\\
1.71782833817753	-0.713426853707415\\
1.7214114151401	-0.729458917835671\\
1.72344689378758	-0.738703910110774\\
1.72497200131837	-0.745490981963928\\
1.72852233483592	-0.761523046092184\\
1.73201231025909	-0.777555110220441\\
1.73544283162421	-0.793587174348697\\
1.7388147715124	-0.809619238476954\\
1.73947895791583	-0.812822767094841\\
1.74219405406295	-0.82565130260521\\
1.7455303014729	-0.841683366733467\\
1.74880791438502	-0.857715430861723\\
1.7520276893305	-0.87374749498998\\
1.75519039270997	-0.889779559118236\\
1.75551102204409	-0.891429711557734\\
1.75836421385233	-0.905811623246493\\
1.7614878128815	-0.92184368737475\\
1.76455409200172	-0.937875751503006\\
1.76756374294779	-0.953907815631263\\
1.77051742834487	-0.969939879759519\\
1.77154308617235	-0.975603791048864\\
1.77346047710993	-0.985971943887776\\
1.77637115270639	-1.00200400801603\\
1.77922541525254	-1.01803607214429\\
1.78202385321948	-1.03406813627255\\
1.78476702669091	-1.0501002004008\\
1.78745546766551	-1.06613226452906\\
1.7875751503006	-1.06685934786461\\
1.7901483801528	-1.08216432865731\\
1.79278818988801	-1.09819639278557\\
1.79537256765055	-1.11422845691383\\
1.79790197103384	-1.13026052104208\\
1.80037682989223	-1.14629258517034\\
1.80279754654409	-1.1623246492986\\
1.80360721442886	-1.16780441192895\\
1.80520003185577	-1.17835671342685\\
1.80756567487247	-1.19438877755511\\
1.80987621161925	-1.21042084168337\\
1.8121319683394	-1.22645290581162\\
1.81433324370427	-1.24248496993988\\
1.81648030891641	-1.25851703406814\\
1.81857340779153	-1.27454909819639\\
1.81963927855711	-1.28292677753027\\
1.82063428513926	-1.29058116232465\\
1.82266346550923	-1.30661322645291\\
1.82463738111443	-1.32264529058116\\
1.82655619601566	-1.33867735470942\\
1.82842004637171	-1.35470941883768\\
1.83022904041777	-1.37074148296593\\
1.83198325842196	-1.38677354709419\\
1.83368275261984	-1.40280561122244\\
1.83532754712663	-1.4188376753507\\
1.83567134268537	-1.42230599971333\\
1.83694387071105	-1.43486973947896\\
1.83851104478601	-1.45090180360721\\
1.84002173704505	-1.46693386773547\\
1.84147588385069	-1.48296593186373\\
1.84287339223335	-1.49899799599198\\
1.84421413968947	-1.51503006012024\\
1.84549797395561	-1.5310621242485\\
1.84672471275821	-1.54709418837675\\
1.84789414353857	-1.56312625250501\\
1.84900602315273	-1.57915831663327\\
1.8500600775457	-1.59519038076152\\
1.85105600139972	-1.61122244488978\\
1.85170340681363	-1.62230202888094\\
1.85199909229475	-1.62725450901804\\
1.85289456737126	-1.64328657314629\\
1.85372935442768	-1.65931863727455\\
1.85450304230896	-1.67535070140281\\
1.85521518624693	-1.69138276553106\\
1.85586530736635	-1.70741482965932\\
1.85645289216113	-1.72344689378758\\
1.85697739193992	-1.73947895791583\\
1.85743822224045	-1.75551102204409\\
1.85783476221172	-1.77154308617234\\
1.8581663539633	-1.7875751503006\\
1.85843230188081	-1.80360721442886\\
1.85863187190667	-1.81963927855711\\
1.85876429078523	-1.83567134268537\\
1.85882874527115	-1.85170340681363\\
1.85882438130004	-1.86773547094188\\
1.85875030312024	-1.88376753507014\\
1.8586055723846	-1.8997995991984\\
1.85838920720086	-1.91583166332665\\
1.85810018113967	-1.93186372745491\\
1.85773742219854	-1.94789579158317\\
1.85729981172059	-1.96392785571142\\
1.85678618326647	-1.97995991983968\\
1.85619532143793	-1.99599198396794\\
1.85552596065151	-2.01202404809619\\
1.85477678386049	-2.02805611222445\\
1.85394642122353	-2.04408817635271\\
1.85303344871797	-2.06012024048096\\
1.85203638669594	-2.07615230460922\\
1.85170340681363	-2.08110478474632\\
1.85096402425816	-2.09218436873747\\
1.84980986194504	-2.10821643286573\\
1.84856756467358	-2.12424849699399\\
1.84723538393176	-2.14028056112224\\
1.84581150654969	-2.1563126252505\\
1.84429405280818	-2.17234468937876\\
1.84268107447124	-2.18837675350701\\
1.8409705527399	-2.20440881763527\\
1.83916039612411	-2.22044088176353\\
1.83724843822968	-2.23647294589178\\
1.83567134268537	-2.24903668565741\\
1.83523737101778	-2.25250501002004\\
1.83313835568119	-2.2685370741483\\
1.83093089661092	-2.28456913827655\\
1.82861245985542	-2.30060120240481\\
1.82618042239087	-2.31663326653307\\
1.82363206922794	-2.33266533066132\\
1.82096459040213	-2.34869739478958\\
1.81963927855711	-2.35635177958396\\
1.81818857080282	-2.36472945891784\\
1.81529996453305	-2.38076152304609\\
1.8122831122404	-2.39679358717435\\
1.80913473079258	-2.41282565130261\\
1.80585142247804	-2.42885771543086\\
1.80360721442886	-2.43941001692876\\
1.8024385369892	-2.44488977955912\\
1.79890051672139	-2.46092184368737\\
1.79521614279353	-2.47695390781563\\
1.79138143889474	-2.49298597194389\\
1.7875751503006	-2.50829095273659\\
1.78739338660917	-2.50901803607214\\
1.78326946333912	-2.5250501002004\\
1.77898147439258	-2.54108216432866\\
1.7745247538334	-2.55711422845691\\
1.77154308617235	-2.56748238129583\\
1.76990175838651	-2.57314629258517\\
1.76511267720294	-2.58917835671343\\
1.7601383404456	-2.60521042084168\\
1.75551102204409	-2.61959233253044\\
1.75497487581311	-2.62124248496994\\
1.7496281134607	-2.6372745490982\\
1.74407694612616	-2.65330661322645\\
1.73947895791583	-2.66613514873682\\
1.73831692392196	-2.66933867735471\\
1.73234548685442	-2.68537074148297\\
1.72614739434424	-2.70140280561122\\
1.72344689378758	-2.70818987746438\\
1.71971566625049	-2.71743486973948\\
1.71304013232617	-2.73346693386774\\
1.70741482965932	-2.74652047249504\\
1.7061103722011	-2.74949899799599\\
1.69891012383408	-2.76553106212425\\
1.69143610370506	-2.7815631262525\\
1.69138276553106	-2.78167509565841\\
1.68365691788803	-2.79759519038076\\
1.67557984389159	-2.81362725450902\\
1.67535070140281	-2.81407237793107\\
1.66715968749157	-2.82965931863727\\
1.65931863727455	-2.84405724240068\\
1.65840866361447	-2.84569138276553\\
1.64927335199124	-2.86172344689379\\
1.64328657314629	-2.87190089013585\\
1.6397574132464	-2.87775551102204\\
1.62982373029581	-2.8937875751503\\
1.62725450901804	-2.8978314455901\\
1.61943156879626	-2.90981963927856\\
1.61122244488978	-2.92202643431353\\
1.60857531675665	-2.92585170340681\\
1.59719561432098	-2.94188376753507\\
1.59519038076152	-2.94464271289486\\
1.58523875260076	-2.95791583166333\\
1.57915831663327	-2.9658107404219\\
1.5726802746891	-2.97394789579158\\
1.56312625250501	-2.98564600747063\\
1.55945965112415	-2.98997995991984\\
1.54709418837675	-3.00424549977693\\
1.5455044615911	-3.0060120240481\\
1.5310621242485	-3.02169538090901\\
1.5307279572433	-3.02204408817635\\
1.51503006012024	-3.03807216018363\\
1.51502598115792	-3.03807615230461\\
1.49899799599198	-3.05344370640559\\
1.49827304395488	-3.05410821643287\\
1.48296593186373	-3.06787012941409\\
1.48031727468776	-3.07014028056112\\
1.46693386773547	-3.08140454787408\\
1.46097385579681	-3.08617234468938\\
1.45090180360721	-3.09409375882733\\
1.44001638721062	-3.10220440881764\\
1.43486973947896	-3.10597882209824\\
1.4188376753507	-3.11710033903772\\
1.41710903881226	-3.11823647294589\\
1.40280561122244	-3.12750636962587\\
1.39167734969043	-3.13426853707415\\
1.38677354709419	-3.1372101514459\\
1.37074148296593	-3.14625626530284\\
1.36307656718187	-3.1503006012024\\
1.35470941883768	-3.15466661391826\\
1.33867735470942	-3.16246635164411\\
1.33012237525181	-3.16633266533066\\
1.32264529058116	-3.16968031772411\\
1.30661322645291	-3.17633463965954\\
1.29075325444766	-3.18236472945892\\
1.29058116232465	-3.18242966225713\\
1.27454909819639	-3.18802432785573\\
1.25851703406814	-3.19309930910985\\
1.24248496993988	-3.19767335714685\\
1.23967836172005	-3.19839679358717\\
1.22645290581162	-3.2017900554151\\
1.21042084168337	-3.20544337144136\\
1.19438877755511	-3.20864327520901\\
1.17835671342685	-3.21140421553854\\
1.1623246492986	-3.21373973178892\\
1.15661502706466	-3.21442885771543\\
1.14629258517034	-3.215673431823\\
1.13026052104208	-3.21720929975852\\
1.11422845691383	-3.21835227700143\\
1.09819639278557	-3.21911252681467\\
1.08216432865731	-3.21949948728802\\
1.06613226452906	-3.21952189989994\\
1.0501002004008	-3.21918783601784\\
1.03406813627254	-3.21850472144465\\
1.01803607214429	-3.21747935911114\\
1.00200400801603	-3.21611795000566\\
0.985998095749708	-3.21442885771543\\
0.985971943887775	-3.21442614155896\\
0.969939879759519	-3.21243069533663\\
0.953907815631262	-3.21011861233941\\
0.937875751503006	-3.2074939838205\\
0.921843687374749	-3.20456038185535\\
0.905811623246493	-3.20132087162849\\
0.892586167338063	-3.19839679358717\\
0.889779559118236	-3.19778513500252\\
0.87374749498998	-3.19398599487374\\
0.857715430861724	-3.18989208776193\\
0.841683366733467	-3.18550468284531\\
0.830931707074993	-3.18236472945892\\
0.825651302605211	-3.18084309491341\\
0.809619238476954	-3.17593153597101\\
0.793587174348698	-3.17073286629194\\
0.780729021259948	-3.16633266533066\\
0.777555110220441	-3.16526011168692\\
0.761523046092185	-3.15955813639492\\
0.745490981963928	-3.15357218820123\\
0.737123833619732	-3.1503006012024\\
0.729458917835672	-3.14733881557017\\
0.713426853707415	-3.14086475809335\\
0.697777947004	-3.13426853707415\\
0.697394789579159	-3.13410884736716\\
0.681362725450902	-3.12715612856513\\
0.665330661322646	-3.11991987857457\\
0.661735583422272	-3.11823647294589\\
0.649298597194389	-3.11247386624583\\
0.633266533066132	-3.10476552161437\\
0.628119885334472	-3.10220440881764\\
0.617234468937876	-3.09684155551826\\
0.601202404809619	-3.08866524528832\\
0.596468517478442	-3.08617234468938\\
0.585170340681363	-3.08027895349741\\
0.569138276553106	-3.07163682472855\\
0.566442776085726	-3.07014028056112\\
0.55310621242485	-3.06280212052261\\
0.537799100333697	-3.05410821643287\\
0.537074148296593	-3.05370005159222\\
0.521042084168337	-3.0444235273177\\
0.510393744430537	-3.03807615230461\\
0.50501002004008	-3.03489325179561\\
0.488977955911824	-3.02515213976752\\
0.483995866906375	-3.02204408817635\\
0.472945891783567	-3.01520356570113\\
0.458503554440966	-3.0060120240481\\
0.456913827655311	-3.00500778702454\\
0.440881763527054	-2.99463500693657\\
0.433868346994613	-2.98997995991984\\
0.424849699398798	-2.98403536906207\\
0.409932488193913	-2.97394789579158\\
0.408817635270541	-2.97319901951363\\
0.392785571142285	-2.96219094658839\\
0.386705135174793	-2.95791583166333\\
0.376753507014028	-2.95096193545089\\
0.364065662450606	-2.94188376753507\\
0.360721442885771	-2.93950507801007\\
0.344689378757515	-2.92785282855784\\
0.341992196266275	-2.92585170340681\\
0.328657314629258	-2.91601138957255\\
0.32044819072278	-2.90981963927856\\
0.312625250501002	-2.90394942268957\\
0.299368989727407	-2.8937875751503\\
0.296593186372745	-2.89167009450756\\
0.280561122244489	-2.87919735205685\\
0.278741749184475	-2.87775551102204\\
0.264529058116232	-2.86654115588278\\
0.258542279271288	-2.86172344689379\\
0.248496993987976	-2.85367293277856\\
0.238727092594649	-2.84569138276553\\
0.232464929859719	-2.84059517615724\\
0.219276137482059	-2.82965931863727\\
0.216432865731463	-2.82731020824735\\
0.200400801603206	-2.81382306215925\\
0.200171659114422	-2.81362725450902\\
0.18436873747495	-2.80016487245945\\
0.181403709616206	-2.79759519038076\\
0.168336673346693	-2.78630243725173\\
0.162943743482434	-2.7815631262525\\
0.152304609218437	-2.77223744705891\\
0.144777250915421	-2.76553106212425\\
0.13627254509018	-2.75797143184814\\
0.126890685934725	-2.74949899799599\\
0.120240480961924	-2.74350576301438\\
0.109271399852711	-2.73346693386774\\
0.104208416833667	-2.72884165515535\\
0.0919075802425008	-2.71743486973948\\
0.0881763527054105	-2.71398016764136\\
0.0747881948941855	-2.70140280561122\\
0.0721442885771539	-2.69892220598275\\
0.0579029403073767	-2.68537074148297\\
0.0561122244488974	-2.68366852299741\\
0.0412421943145115	-2.66933867735471\\
0.0400801603206409	-2.66821971978054\\
0.0247969724703404	-2.65330661322645\\
0.0240480961923843	-2.65257624647837\\
0.00855888787954364	-2.6372745490982\\
0.00801603206412782	-2.63673840286722\\
-0.00747988583314781	-2.62124248496994\\
-0.00801603206412826	-2.62070633873896\\
-0.0233266486770157	-2.60521042084168\\
-0.0240480961923848	-2.6044800540936\\
-0.0389882058144761	-2.58917835671343\\
-0.0400801603206413	-2.58805939913926\\
-0.05447089666306	-2.57314629258517\\
-0.0561122244488979	-2.57144407409961\\
-0.0697806217774123	-2.55711422845691\\
-0.0721442885771544	-2.55463362882844\\
-0.0849228676899451	-2.54108216432866\\
-0.0881763527054109	-2.53762746223054\\
-0.0999027298721868	-2.5250501002004\\
-0.104208416833667	-2.52042482148801\\
-0.114724933963928	-2.50901803607214\\
-0.120240480961924	-2.50302480109053\\
-0.129393855403837	-2.49298597194389\\
-0.13627254509018	-2.48542634166778\\
-0.14391353758311	-2.47695390781563\\
-0.152304609218437	-2.46762822862203\\
-0.158287708632811	-2.46092184368737\\
-0.168336673346694	-2.44962909055835\\
-0.172519796945657	-2.44488977955912\\
-0.18436873747495	-2.43142739750955\\
-0.186612945524132	-2.42885771543086\\
-0.200400801603207	-2.41302145895284\\
-0.200570025238682	-2.41282565130261\\
-0.214411163717496	-2.39679358717435\\
-0.216432865731463	-2.39444447678443\\
-0.228125615835659	-2.38076152304609\\
-0.232464929859719	-2.37566531643781\\
-0.241714242995323	-2.36472945891784\\
-0.248496993987976	-2.35667894480261\\
-0.255179057332446	-2.34869739478958\\
-0.264529058116232	-2.33748303965031\\
-0.268521848787055	-2.33266533066132\\
-0.280561122244489	-2.31807510756787\\
-0.281744193712368	-2.31663326653307\\
-0.294865693023194	-2.30060120240481\\
-0.296593186372745	-2.29848372176207\\
-0.307884804426556	-2.28456913827655\\
-0.312625250501002	-2.27869892168757\\
-0.320790737016945	-2.2685370741483\\
-0.328657314629258	-2.25869676031403\\
-0.333584388029848	-2.25250501002004\\
-0.344689378757515	-2.23847407104281\\
-0.346266474301825	-2.23647294589178\\
-0.358859841022652	-2.22044088176353\\
-0.360721442885771	-2.21806219223852\\
-0.371364677459651	-2.20440881763527\\
-0.376753507014028	-2.19745492142283\\
-0.383763238799898	-2.18837675350701\\
-0.392785571142285	-2.17661980430382\\
-0.396055667160588	-2.17234468937876\\
-0.408249324480346	-2.1563126252505\\
-0.408817635270541	-2.15556374897255\\
-0.420381676516927	-2.14028056112224\\
-0.424849699398798	-2.13433597026447\\
-0.432411864046547	-2.12424849699399\\
-0.440881763527054	-2.11287147988246\\
-0.444339516511446	-2.10821643286573\\
-0.456174445099843	-2.09218436873747\\
-0.456913827655311	-2.09118013171392\\
-0.467956638280628	-2.07615230460922\\
-0.472945891783567	-2.069311782134\\
-0.47963909184413	-2.06012024048096\\
-0.488977955911824	-2.04719622794388\\
-0.491220970321727	-2.04408817635271\\
-0.502735279239629	-2.02805611222445\\
-0.50501002004008	-2.02487321171545\\
-0.514184056143404	-2.01202404809619\\
-0.521042084168337	-2.00233935898103\\
-0.525533998792644	-1.99599198396794\\
-0.53678827695295	-1.97995991983968\\
-0.537074148296593	-1.97955175499903\\
-0.548013140606755	-1.96392785571142\\
-0.55310621242485	-1.95658969567291\\
-0.559140227809761	-1.94789579158317\\
-0.569138276553106	-1.93336027162234\\
-0.570167995371698	-1.93186372745491\\
-0.581160780256625	-1.91583166332665\\
-0.585170340681363	-1.90993827213469\\
-0.592072506252331	-1.8997995991984\\
-0.601202404809619	-1.88626043566908\\
-0.60288506512914	-1.88376753507014\\
-0.613657428555808	-1.86773547094188\\
-0.617234468937876	-1.86237261764251\\
-0.624359807395399	-1.85170340681363\\
-0.633266533066132	-1.8382324554821\\
-0.634962578770196	-1.83567134268537\\
-0.645529835186418	-1.81963927855711\\
-0.649298597194389	-1.81387667185705\\
-0.656027492261569	-1.80360721442886\\
-0.665330661322646	-1.78925855592928\\
-0.666424536732041	-1.7875751503006\\
-0.676801290439637	-1.77154308617234\\
-0.681362725450902	-1.76443067766332\\
-0.687097540877721	-1.75551102204409\\
-0.697293488050765	-1.73947895791583\\
-0.697394789579158	-1.73931926820884\\
-0.707491839219817	-1.72344689378758\\
-0.713426853707415	-1.71401105067852\\
-0.717588754260136	-1.70741482965932\\
-0.727616224980459	-1.69138276553106\\
-0.729458917835671	-1.68842097989882\\
-0.737618467272952	-1.67535070140281\\
-0.745490981963928	-1.66259022427337\\
-0.747516929577984	-1.65931863727455\\
-0.757388887898833	-1.64328657314629\\
-0.761523046092184	-1.63651204421055\\
-0.767195261899139	-1.62725450901804\\
-0.776907704806536	-1.61122244488978\\
-0.777555110220441	-1.61014989124604\\
-0.786623917378867	-1.59519038076152\\
-0.793587174348697	-1.58355851759455\\
-0.796233549095041	-1.57915831663327\\
-0.8058099752019	-1.56312625250501\\
-0.809619238476954	-1.5566930590171\\
-0.815331077458949	-1.54709418837675\\
-0.824760127386751	-1.5310621242485\\
-0.82565130260521	-1.52954048970299\\
-0.834194099609312	-1.51503006012024\\
-0.841683366733467	-1.50213794937837\\
-0.843517559081993	-1.49899799599198\\
-0.852828753689511	-1.48296593186373\\
-0.857715430861723	-1.47446122603849\\
-0.862065825221398	-1.46693386773547\\
-0.871240825915441	-1.45090180360721\\
-0.87374749498998	-1.44649100489378\\
-0.880392419679554	-1.43486973947896\\
-0.889435763559499	-1.4188376753507\\
-0.889779559118236	-1.41822601676604\\
-0.898502620912698	-1.40280561122244\\
-0.905811623246493	-1.38969762513551\\
-0.907453086973702	-1.38677354709419\\
-0.916401385107147	-1.37074148296593\\
-0.92184368737475	-1.36087300710585\\
-0.925266721059645	-1.35470941883768\\
-0.934093357167528	-1.33867735470942\\
-0.937875751503006	-1.33174248081448\\
-0.942873854060321	-1.32264529058116\\
-0.951582880981771	-1.30661322645291\\
-0.953907815631263	-1.30230298107688\\
-0.96027867444484	-1.29058116232465\\
-0.968874008993936	-1.27454909819639\\
-0.969939879759519	-1.27255093581759\\
-0.977485082284089	-1.25851703406814\\
-0.985970511113727	-1.24248496993988\\
-0.985971943887776	-1.24248225378341\\
-0.99449669779832	-1.22645290581162\\
-1.00200400801603	-1.21210993397359\\
-1.00289562534923	-1.21042084168337\\
-1.01131686927922	-1.19438877755511\\
-1.01803607214429	-1.18140721482257\\
-1.0196288895712	-1.17835671342685\\
-1.02794868041008	-1.1623246492986\\
-1.03406813627255	-1.15036844889956\\
-1.03617320976842	-1.14629258517034\\
-1.04439495700578	-1.13026052104208\\
-1.0501002004008	-1.11898743521624\\
-1.05253126766383	-1.11422845691383\\
-1.06065827319264	-1.09819639278557\\
-1.06613226452906	-1.08725737084183\\
-1.06870549438125	-1.08216432865731\\
-1.07674095704631	-1.06613226452906\\
-1.08216432865731	-1.0551708299734\\
-1.0846980757674	-1.0501002004008\\
-1.09264509570445	-1.03406813627255\\
-1.09819639278557	-1.02271974124353\\
-1.10051095719356	-1.01803607214429\\
-1.10837253996919	-1.00200400801603\\
-1.11422845691383	-0.989895363173774\\
-1.11614584785142	-0.985971943887776\\
-1.12392490841295	-0.969939879759519\\
-1.13026052104208	-0.956688257674352\\
-1.13160422455531	-0.953907815631263\\
-1.13930359099971	-0.937875751503006\\
-1.14629258517034	-0.92308826148232\\
-1.14688733506242	-0.92184368737475\\
-1.15450975223249	-0.905811623246493\\
-1.16200401996448	-0.889779559118236\\
-1.1623246492986	-0.889090433191721\\
-1.16954433383618	-0.87374749498998\\
-1.17696578019152	-0.857715430861723\\
-1.17835671342685	-0.854690788684829\\
-1.18440805698392	-0.841683366733467\\
-1.19175920173928	-0.82565130260521\\
-1.19438877755511	-0.819865720098788\\
-1.19910142407344	-0.809619238476954\\
-1.20638471539175	-0.793587174348697\\
-1.21042084168337	-0.784601688074629\\
-1.21362472005891	-0.777555110220441\\
-1.22084253737021	-0.761523046092184\\
-1.22645290581162	-0.748884243791853\\
-1.22797801334242	-0.745490981963928\\
-1.23513267009027	-0.729458917835671\\
-1.24216907752019	-0.713426853707415\\
-1.24248496993988	-0.712703417267096\\
-1.24925490248686	-0.697394789579158\\
-1.2562314894554	-0.681362725450902\\
-1.25851703406814	-0.676065240973576\\
-1.26320880990842	-0.665330661322646\\
-1.27012770666234	-0.649298597194389\\
-1.27454909819639	-0.638926131462949\\
-1.27699375358044	-0.633266533066132\\
-1.28385703316352	-0.617234468937876\\
-1.29058116232465	-0.601267337607829\\
-1.29060887963702	-0.601202404809619\\
-1.29741856006962	-0.585170340681363\\
-1.30411699672397	-0.569138276553106\\
-1.30661322645291	-0.56310818675373\\
-1.31081116406888	-0.55310621242485\\
-1.31745878918003	-0.537074148296593\\
-1.32264529058116	-0.524389736561787\\
-1.32403350548239	-0.521042084168337\\
-1.33063218027391	-0.50501002004008\\
-1.33712404097122	-0.488977955911824\\
-1.33867735470942	-0.48511164222527\\
-1.34363556339904	-0.472945891783567\\
-1.3500809692004	-0.456913827655311\\
-1.35470941883768	-0.445247776242905\\
-1.35646711206576	-0.440881763527054\\
-1.36286777974099	-0.424849699398798\\
-1.3691656409182	-0.408817635270541\\
-1.37074148296593	-0.404773299370972\\
-1.37548237873076	-0.392785571142285\\
-1.38173772597015	-0.376753507014028\\
-1.38677354709419	-0.363663057257521\\
-1.3879224466588	-0.360721442885771\\
-1.39413686273957	-0.344689378757515\\
-1.40025208659613	-0.328657314629258\\
-1.40280561122244	-0.32189514718098\\
-1.40636046095951	-0.312625250501002\\
-1.41243672444534	-0.296593186372745\\
-1.41841668542161	-0.280561122244489\\
-1.4188376753507	-0.279424988336314\\
-1.42444445727107	-0.264529058116232\\
-1.43038727181292	-0.248496993987976\\
-1.43486973947896	-0.236239343140326\\
-1.43627218400033	-0.232464929859719\\
-1.44217922799749	-0.216432865731463\\
-1.44799298112941	-0.200400801603207\\
-1.45090180360721	-0.192290151612905\\
-1.4537891721569	-0.18436873747495\\
-1.45956873876195	-0.168336673346694\\
-1.46525751441528	-0.152304609218437\\
-1.46693386773547	-0.147536812403136\\
-1.47096011403291	-0.13627254509018\\
-1.47661614380372	-0.120240480961924\\
-1.48218373363104	-0.104208416833667\\
-1.48296593186373	-0.101938265686635\\
-1.48778765267569	-0.0881763527054109\\
-1.49332379819697	-0.0721442885771544\\
-1.49877371175641	-0.0561122244488979\\
-1.49899799599198	-0.0554477144216257\\
-1.50427363149831	-0.0400801603206413\\
-1.50969326695727	-0.0240480961923848\\
-1.51502874080229	-0.00801603206412826\\
-1.51503006012024	-0.00801203994315157\\
-1.52041910140116	0.00801603206412782\\
-1.52572533108552	0.0240480961923843\\
-1.53094933679874	0.0400801603206409\\
-1.5310621242485	0.0404288675879877\\
-1.53622432482812	0.0561122244488974\\
-1.54141999058174	0.0721442885771539\\
-1.546535241786	0.0881763527054105\\
-1.54709418837675	0.0899428769765749\\
-1.551688776779	0.104208416833667\\
-1.55677646444501	0.120240480961924\\
-1.56178542281927	0.13627254509018\\
-1.56312625250501	0.140606497539392\\
-1.56681114278396	0.152304609218437\\
-1.57179318765974	0.168336673346693\\
-1.57669806798192	0.18436873747495\\
-1.57915831663327	0.192505892844634\\
-1.58158931382882	0.200400801603206\\
-1.5864678051518	0.216432865731463\\
-1.59127057938552	0.232464929859719\\
-1.59519038076152	0.245738048628186\\
-1.59602037820382	0.248496993987976\\
-1.60079716268189	0.264529058116232\\
-1.6054995631189	0.280561122244489\\
-1.61012914769989	0.296593186372745\\
-1.61122244488978	0.300418455466026\\
-1.61477729462685	0.312625250501002\\
-1.61938081609172	0.328657314629258\\
-1.62391268479772	0.344689378757515\\
-1.62725450901804	0.356677572445966\\
-1.62840340858265	0.360721442885771\\
-1.63290930970114	0.376753507014028\\
-1.63734461298571	0.392785571142285\\
-1.64171073109856	0.408817635270541\\
-1.64328657314629	0.414672256156731\\
-1.64607917023272	0.424849699398798\\
-1.65041882425813	0.440881763527054\\
-1.65469018763728	0.456913827655311\\
-1.6588945800742	0.472945891783567\\
-1.65931863727455	0.474580032148419\\
-1.66312834154604	0.488977955911824\\
-1.66730552696729	0.50501002004008\\
-1.67141648023444	0.521042084168337\\
-1.67535070140281	0.536629024874541\\
-1.67546529128451	0.537074148296593\\
-1.67954863901878	0.55310621242485\\
-1.68356639100834	0.569138276553106\\
-1.68751972819312	0.585170340681363\\
-1.69138276553106	0.601090435403712\\
-1.69141048284343	0.601202404809619\\
-1.69533503186013	0.617234468937876\\
-1.69919565132171	0.633266533066132\\
-1.70299343812526	0.649298597194389\\
-1.70672945483409	0.665330661322646\\
-1.70741482965932	0.668309186823599\\
-1.71047970488146	0.681362725450902\\
-1.7141847622063	0.697394789579159\\
-1.71782833817753	0.713426853707415\\
-1.7214114151401	0.729458917835672\\
-1.72344689378758	0.738703910110772\\
-1.72497200131837	0.745490981963928\\
-1.72852233483592	0.761523046092185\\
-1.73201231025909	0.777555110220441\\
-1.73544283162421	0.793587174348698\\
-1.7388147715124	0.809619238476954\\
-1.73947895791583	0.812822767094839\\
-1.74219405406295	0.825651302605211\\
-1.7455303014729	0.841683366733467\\
-1.74880791438502	0.857715430861724\\
-1.7520276893305	0.87374749498998\\
-1.75519039270997	0.889779559118236\\
-1.75551102204409	0.891429711557732\\
-1.75836421385233	0.905811623246493\\
-1.7614878128815	0.921843687374749\\
-1.76455409200172	0.937875751503006\\
-1.76756374294779	0.953907815631262\\
-1.77051742834487	0.969939879759519\\
-1.77154308617234	0.975603791048862\\
-1.77346047710993	0.985971943887775\\
-1.77637115270639	1.00200400801603\\
-1.77922541525254	1.01803607214429\\
-1.78202385321948	1.03406813627254\\
-1.78476702669091	1.0501002004008\\
-1.78745546766551	1.06613226452906\\
-1.7875751503006	1.06685934786461\\
-1.7901483801528	1.08216432865731\\
-1.79278818988801	1.09819639278557\\
-1.79537256765055	1.11422845691383\\
-1.79790197103384	1.13026052104208\\
-1.80037682989223	1.14629258517034\\
-1.80279754654409	1.1623246492986\\
-1.80360721442886	1.16780441192895\\
-1.80520003185577	1.17835671342685\\
-1.80756567487247	1.19438877755511\\
-1.80987621161925	1.21042084168337\\
-1.8121319683394	1.22645290581162\\
-1.81433324370427	1.24248496993988\\
-1.81648030891641	1.25851703406814\\
-1.81857340779153	1.27454909819639\\
-1.81963927855711	1.28292677753027\\
-1.82063428513926	1.29058116232465\\
-1.82266346550923	1.30661322645291\\
-1.82463738111443	1.32264529058116\\
-1.82655619601566	1.33867735470942\\
-1.82842004637171	1.35470941883768\\
-1.83022904041777	1.37074148296593\\
-1.83198325842196	1.38677354709419\\
-1.83368275261984	1.40280561122244\\
-1.83532754712663	1.4188376753507\\
-1.83567134268537	1.42230599971334\\
-1.83694387071105	1.43486973947896\\
-1.83851104478601	1.45090180360721\\
-1.84002173704505	1.46693386773547\\
-1.84147588385069	1.48296593186373\\
-1.84287339223335	1.49899799599198\\
-1.84421413968947	1.51503006012024\\
-1.84549797395561	1.5310621242485\\
-1.84672471275821	1.54709418837675\\
-1.84789414353857	1.56312625250501\\
-1.84900602315273	1.57915831663327\\
-1.8500600775457	1.59519038076152\\
-1.85105600139972	1.61122244488978\\
-1.85170340681363	1.62230202888095\\
-1.85199909229475	1.62725450901804\\
-1.85289456737126	1.64328657314629\\
-1.85372935442768	1.65931863727455\\
-1.85450304230896	1.67535070140281\\
-1.85521518624693	1.69138276553106\\
-1.85586530736635	1.70741482965932\\
-1.85645289216113	1.72344689378758\\
-1.85697739193992	1.73947895791583\\
-1.85743822224045	1.75551102204409\\
-1.85783476221172	1.77154308617235\\
-1.8581663539633	1.7875751503006\\
-1.85843230188081	1.80360721442886\\
-1.85863187190667	1.81963927855711\\
-1.85876429078523	1.83567134268537\\
-1.85882874527115	1.85170340681363\\
-1.85882438130004	1.86773547094188\\
-1.85875030312025	1.88376753507014\\
-1.8586055723846	1.8997995991984\\
-1.85838920720086	1.91583166332665\\
-1.85810018113967	1.93186372745491\\
-1.85773742219854	1.94789579158317\\
-1.85729981172059	1.96392785571142\\
-1.85678618326647	1.97995991983968\\
-1.85619532143793	1.99599198396794\\
-1.85552596065151	2.01202404809619\\
-1.85477678386049	2.02805611222445\\
-1.85394642122353	2.04408817635271\\
-1.85303344871797	2.06012024048096\\
-1.85203638669594	2.07615230460922\\
-1.85170340681363	2.08110478474631\\
}--cycle;


\addplot[area legend,solid,fill=mycolor4,draw=black,forget plot]
table[row sep=crcr] {%
x	y\\
-1.61122244488978	1.92154062635485\\
-1.61011248878115	1.93186372745491\\
-1.60828067438538	1.94789579158317\\
-1.60633322202377	1.96392785571142\\
-1.60426738422376	1.97995991983968\\
-1.60208031121155	1.99599198396794\\
-1.59976904743487	2.01202404809619\\
-1.59733052793776	2.02805611222445\\
-1.59519038076152	2.04143268533249\\
-1.59476347033196	2.04408817635271\\
-1.59207201391084	2.06012024048096\\
-1.58924270019754	2.07615230460922\\
-1.58627189389273	2.09218436873747\\
-1.58315582550126	2.10821643286573\\
-1.57989058644274	2.12424849699399\\
-1.57915831663327	2.1277110612458\\
-1.5764782936904	2.14028056112224\\
-1.57290909694685	2.1563126252505\\
-1.56917665717801	2.17234468937876\\
-1.56527632738903	2.18837675350701\\
-1.56312625250501	2.19688521514885\\
-1.56120437078298	2.20440881763527\\
-1.55695444683608	2.22044088176353\\
-1.55251945803753	2.23647294589178\\
-1.5478938590338	2.25250501002004\\
-1.54709418837675	2.25519119398684\\
-1.54306685727617	2.2685370741483\\
-1.53803384825403	2.28456913827655\\
-1.53278921849702	2.30060120240481\\
-1.5310621242485	2.30571529601306\\
-1.52731554649612	2.31663326653307\\
-1.52160843687928	2.33266533066132\\
-1.5156649653976	2.34869739478958\\
-1.51503006012024	2.35036312387492\\
-1.50945191118913	2.36472945891784\\
-1.50297916329548	2.38076152304609\\
-1.49899799599198	2.39029006435056\\
-1.49622401985135	2.39679358717435\\
-1.48916536249221	2.41282565130261\\
-1.48296593186373	2.42638221231704\\
-1.48180767585431	2.42885771543086\\
-1.47409706992632	2.44488977955912\\
-1.46693386773547	2.4592173411896\\
-1.4660599246591	2.46092184368737\\
-1.45762008602662	2.47695390781563\\
-1.45090180360721	2.48925412846316\\
-1.4488063422536	2.49298597194389\\
-1.43954643592755	2.50901803607214\\
-1.43486973947896	2.51685864994032\\
-1.42983385031263	2.5250501002004\\
-1.41964657378708	2.54108216432866\\
-1.4188376753507	2.54232403453555\\
-1.40888141830834	2.55711422845691\\
-1.40280561122244	2.56585381624442\\
-1.39755434365011	2.57314629258517\\
-1.38677354709419	2.58766528351743\\
-1.38560707639808	2.58917835671343\\
-1.37292503730557	2.60521042084168\\
-1.37074148296593	2.6078980187191\\
-1.35944897462705	2.62124248496994\\
-1.35470941883768	2.62669324850733\\
-1.34510050105639	2.6372745490982\\
-1.33867735470942	2.64416791174315\\
-1.32975869254771	2.65330661322645\\
-1.32264529058116	2.660420015193\\
-1.31327547851173	2.66933867735471\\
-1.30661322645291	2.67553578751015\\
-1.29546845826235	2.68537074148297\\
-1.29058116232465	2.68959102694447\\
-1.27611129417885	2.70140280561122\\
-1.27454909819639	2.70265225874734\\
-1.25851703406814	2.71477960525793\\
-1.25478519058741	2.71743486973948\\
-1.24248496993988	2.7260271997387\\
-1.23108671830602	2.73346693386774\\
-1.22645290581162	2.73644003213387\\
-1.21042084168337	2.74606380126248\\
-1.20427225725145	2.74949899799599\\
-1.19438877755511	2.75493757065414\\
-1.17835671342685	2.76309270994689\\
-1.17317428761382	2.76553106212425\\
-1.1623246492986	2.7705684654317\\
-1.14629258517034	2.77738784786229\\
-1.13552081220309	2.7815631262525\\
-1.13026052104208	2.78357893720756\\
-1.11422845691383	2.78917463166277\\
-1.09819639278557	2.79418489914963\\
-1.08594047435841	2.79759519038076\\
-1.08216432865731	2.79863663408102\\
-1.06613226452906	2.80255991511815\\
-1.0501002004008	2.80595976525476\\
-1.03406813627255	2.80885428964474\\
-1.01803607214429	2.81126042517164\\
-1.00200400801603	2.81319399879284\\
-0.997331417501761	2.81362725450902\\
-0.985971943887776	2.814675928256\\
-0.969939879759519	2.81571419710024\\
-0.953907815631263	2.81631895101368\\
-0.937875751503006	2.81650189349093\\
-0.92184368737475	2.8162738607838\\
-0.905811623246493	2.81564485848873\\
-0.889779559118236	2.81462409541754\\
-0.878420085504249	2.81362725450902\\
-0.87374749498998	2.81322297105368\\
-0.857715430861723	2.81145662464762\\
-0.841683366733467	2.80932517220469\\
-0.82565130260521	2.80683508253215\\
-0.809619238476954	2.80399215955377\\
-0.793587174348697	2.80080156334239\\
-0.779046159184035	2.79759519038076\\
-0.777555110220441	2.79727052082353\\
-0.761523046092184	2.79343021580375\\
-0.745490981963928	2.78925838468812\\
-0.729458917835671	2.78475793913733\\
-0.718857090061259	2.7815631262525\\
-0.713426853707415	2.7799455667311\\
-0.697394789579158	2.77484091227589\\
-0.681362725450902	2.76941790865144\\
-0.670513087135676	2.76553106212425\\
-0.665330661322646	2.76369440978076\\
-0.649298597194389	2.75769375545286\\
-0.633266533066132	2.75138106085277\\
-0.628713705911684	2.74949899799599\\
-0.617234468937876	2.74480100256469\\
-0.601202404809619	2.73793042458416\\
-0.591235090961048	2.73346693386774\\
-0.585170340681363	2.7307767665998\\
-0.569138276553106	2.72336002335721\\
-0.556838055905579	2.71743486973948\\
-0.55310621242485	2.71565329829787\\
-0.537074148296593	2.70769959745392\\
-0.524849509067469	2.70140280561122\\
-0.521042084168337	2.69945820243091\\
-0.50501002004008	2.69097432997933\\
-0.494783771844951	2.68537074148297\\
-0.488977955911824	2.68221462676709\\
-0.472945891783567	2.67320501962568\\
-0.466283639724746	2.66933867735471\\
-0.456913827655311	2.66394142181914\\
-0.440881763527054	2.65440822496289\\
-0.439082402647768	2.65330661322645\\
-0.424849699398798	2.64465327283536\\
-0.413074526782555	2.6372745490982\\
-0.408817635270541	2.63462480283205\\
-0.392785571142285	2.6243608052396\\
-0.388046015352913	2.62124248496994\\
-0.376753507014028	2.6138584373754\\
-0.363888837953899	2.60521042084168\\
-0.360721442885771	2.60309375284415\\
-0.344689378757515	2.59210255044851\\
-0.340529533563966	2.58917835671343\\
-0.328657314629258	2.5808776181457\\
-0.317876518073337	2.57314629258517\\
-0.312625250501002	2.5693997148328\\
-0.296593186372745	2.55767866334882\\
-0.295837892673827	2.55711422845691\\
-0.280561122244489	2.54575051469963\\
-0.274429644828789	2.54108216432866\\
-0.264529058116232	2.53357697573616\\
-0.253532883154308	2.5250501002004\\
-0.248496993987976	2.52116109576389\\
-0.233116340787461	2.50901803607214\\
-0.232464929859719	2.50850572468606\\
-0.216432865731463	2.49564312778937\\
-0.213188106974364	2.49298597194389\\
-0.200400801603207	2.48254928829144\\
-0.193682519183802	2.47695390781563\\
-0.18436873747495	2.46922073142719\\
-0.174568524887323	2.46092184368737\\
-0.168336673346694	2.45565956781352\\
-0.155824596672748	2.44488977955912\\
-0.152304609218437	2.44186772271686\\
-0.137430801099601	2.42885771543086\\
-0.13627254509018	2.42784693855345\\
-0.120240480961924	2.41360762885972\\
-0.119374742175998	2.41282565130261\\
-0.104208416833667	2.39915131035514\\
-0.101637517500693	2.39679358717435\\
-0.0881763527054109	2.38446763068815\\
-0.0841951854019193	2.38076152304609\\
-0.0721442885771544	2.36955763291399\\
-0.0670336762908581	2.36472945891784\\
-0.0561122244488979	2.35442218420181\\
-0.0501399221213309	2.34869739478958\\
-0.0400801603206413	2.33906197695177\\
-0.0335017835616004	2.33266533066132\\
-0.0240480961923848	2.32347752968626\\
-0.0171079835705498	2.31663326653307\\
-0.00801603206412826	2.30766918771692\\
-0.000948046752016809	2.30060120240481\\
0.00801603206412782	2.29163712358867\\
0.0149877560696653	2.28456913827655\\
0.0240480961923843	2.27538133730149\\
0.0307084585490926	2.2685370741483\\
0.0400801603206409	2.25890165631049\\
0.046222445236839	2.25250501002004\\
0.0561122244488974	2.24219773530401\\
0.0615374941096744	2.23647294589178\\
0.0721442885771539	2.22526905575968\\
0.0766608155678959	2.22044088176353\\
0.0881763527054105	2.20811492527733\\
0.0915990882169153	2.20440881763527\\
0.104208416833667	2.19073447668781\\
0.106358491717682	2.18837675350701\\
0.120240480961924	2.17312666693587\\
0.120944736961574	2.17234468937876\\
0.135364739286021	2.1563126252505\\
0.13627254509018	2.15530184837309\\
0.149624586275568	2.14028056112224\\
0.152304609218437	2.13725850427999\\
0.163728008132806	2.12424849699399\\
0.168336673346693	2.11898622112013\\
0.177679279651716	2.10821643286573\\
0.18436873747495	2.10048325647729\\
0.191482314734416	2.09218436873747\\
0.200400801603206	2.08174768508503\\
0.205140685797516	2.07615230460922\\
0.216432865731463	2.06277739632644\\
0.218657641608958	2.06012024048096\\
0.232038019430153	2.04408817635271\\
0.232464929859719	2.04357586496662\\
0.245294482052848	2.02805611222445\\
0.248496993987976	2.02416710778794\\
0.258420727231885	2.01202404809619\\
0.264529058116232	2.0045188595037\\
0.271418988566259	1.99599198396794\\
0.280561122244489	1.98462827021065\\
0.284291241551083	1.97995991983968\\
0.296593186372745	1.96449229060333\\
0.297039214899656	1.96392785571142\\
0.309683479996603	1.94789579158317\\
0.312625250501002	1.94414921383079\\
0.322212717668509	1.93186372745491\\
0.328657314629258	1.92356298888719\\
0.334625405535997	1.91583166332665\\
0.344689378757515	1.90272379293348\\
0.346922486628156	1.8997995991984\\
0.359116872563305	1.88376753507014\\
0.360721442885771	1.88165086707261\\
0.371216990235094	1.86773547094188\\
0.376753507014028	1.86035142334735\\
0.383207437170299	1.85170340681363\\
0.392785571142285	1.83878966295503\\
0.395088476620054	1.83567134268537\\
0.406877006331786	1.81963927855711\\
0.408817635270541	1.81698953229097\\
0.418579485774504	1.80360721442886\\
0.424849699398798	1.79495387403776\\
0.430176934846032	1.7875751503006\\
0.440881763527054	1.77264469790878\\
0.441669019392435	1.77154308617235\\
0.453092393214548	1.75551102204409\\
0.456913827655311	1.75011376650852\\
0.464421380727239	1.73947895791583\\
0.472945891783567	1.72731323605855\\
0.475648009534167	1.72344689378758\\
0.486794006752962	1.70741482965932\\
0.488977955911824	1.70425871494344\\
0.497867416095764	1.69138276553106\\
0.50501002004008	1.68095428989917\\
0.508840595929194	1.67535070140281\\
0.519726753421353	1.65931863727455\\
0.521042084168337	1.65737403409424\\
0.530555404980253	1.64328657314629\\
0.537074148296593	1.63355130086073\\
0.54128514558374	1.62725450901804\\
0.551928074821154	1.61122244488978\\
0.55310621242485	1.60944087344817\\
0.562520884658426	1.59519038076152\\
0.569138276553106	1.585083470251\\
0.573015347633412	1.57915831663327\\
0.583430809687441	1.56312625250501\\
0.585170340681363	1.56043608523708\\
0.593794926995524	1.54709418837675\\
0.601202404809619	1.53552561496492\\
0.604060554477165	1.5310621242485\\
0.614263529626301	1.51503006012024\\
0.617234468937876	1.51033206468894\\
0.62440445512755	1.49899799599198\\
0.633266533066132	1.4848479947205\\
0.634446080277041	1.48296593186373\\
0.644450832711952	1.46693386773547\\
0.649298597194389	1.45909656106408\\
0.654372519869492	1.45090180360721\\
0.664208689869032	1.43486973947896\\
0.665330661322646	1.43303308713547\\
0.67401359899695	1.4188376753507\\
0.681362725450902	1.40669245774963\\
0.683718539634568	1.40280561122244\\
0.69337756591569	1.38677354709419\\
0.697394789579159	1.38005133311758\\
0.702969211433921	1.37074148296593\\
0.712472211428738	1.35470941883768\\
0.713426853707415	1.35309185931627\\
0.721953017014737	1.33867735470942\\
0.729458917835672	1.32584010346599\\
0.7313317452183	1.32264529058116\\
0.740677067007647	1.30661322645291\\
0.745490981963928	1.29827642076026\\
0.749947267981695	1.29058116232465\\
0.759148058525284	1.27454909819639\\
0.761523046092185	1.27038412361938\\
0.76831154460939	1.25851703406814\\
0.777372290686372	1.24248496993988\\
0.777555110220441	1.24216030038265\\
0.786430653651559	1.22645290581162\\
0.793587174348698	1.213627214645\\
0.795383112446186	1.21042084168337\\
0.804310296988349	1.19438877755511\\
0.809619238476954	1.18475368259986\\
0.813158487753533	1.17835671342685\\
0.821955812894223	1.1623246492986\\
0.825651302605211	1.15553247732173\\
0.830700749518797	1.14629258517034\\
0.839372188236657	1.13026052104208\\
0.841683366733467	1.12595843873776\\
0.848014685479281	1.11422845691383\\
0.856564069847169	1.09819639278557\\
0.857715430861724	1.09602576292418\\
0.865104748160702	1.08216432865731\\
0.873535775099905	1.06613226452906\\
0.87374749498998	1.06572798107372\\
0.881975064979866	1.0501002004008\\
0.889779559118236	1.03506497718107\\
0.890299918799153	1.03406813627254\\
0.898629447604153	1.01803607214429\\
0.905811623246493	1.00402161199574\\
0.906851749946105	1.00200400801603\\
0.915071400596516	0.985971943887775\\
0.921843687374749	0.972586486034307\\
0.923191066272462	0.969939879759519\\
0.931304129372398	0.953907815631262\\
0.937875751503006	0.94075039048492\\
0.939320894316716	0.937875751503006\\
0.947330547492817	0.921843687374749\\
0.953907815631262	0.908503319751152\\
0.955243971041258	0.905811623246493\\
0.96315328331582	0.889779559118236\\
0.969939879759519	0.8758344375812\\
0.970962750081582	0.87374749498998\\
0.978774686026535	0.857715430861724\\
0.985971943887775	0.842732040480446\\
0.986479407382706	0.841683366733467\\
0.994196831064096	0.825651302605211\\
1.00179955554845	0.809619238476954\\
1.00200400801603	0.809185982760772\\
1.00942152496197	0.793587174348698\\
1.01693390146419	0.777555110220441\\
1.01803607214429	0.775188280883066\\
1.02445030961641	0.761523046092185\\
1.03187494369613	0.745490981963928\\
1.03406813627254	0.740718017099647\\
1.03928446599602	0.729458917835672\\
1.04662387524361	0.713426853707415\\
1.0501002004008	0.705759364453157\\
1.05392501730387	0.697394789579159\\
1.0611816344267	0.681362725450902\\
1.06613226452906	0.670295386060036\\
1.06837273160182	0.665330661322646\\
1.07554890734466	0.649298597194389\\
1.08216432865731	0.634307976766395\\
1.0826281239054	0.633266533066132\\
1.08972612981108	0.617234468937876\\
1.09671960727013	0.601202404809619\\
1.09819639278557	0.597792113578492\\
1.10371348877188	0.585170340681363\\
1.11063195689279	0.569138276553106\\
1.11422845691383	0.560717717835118\\
1.11751092321085	0.55310621242485\\
1.12435644545861	0.537074148296593\\
1.13026052104208	0.523057895123392\\
1.13111812454601	0.521042084168337\\
1.13789269380047	0.50501002004008\\
1.14456803763354	0.488977955911824\\
1.14629258517034	0.484802677521606\\
1.15124007716546	0.472945891783567\\
1.1578471485081	0.456913827655311\\
1.1623246492986	0.445919166834502\\
1.16439772446391	0.440881763527054\\
1.17093832854351	0.424849699398798\\
1.17738359767137	0.408817635270541\\
1.17835671342685	0.406379283093181\\
1.18384039529476	0.392785571142285\\
1.19022155276666	0.376753507014028\\
1.19438877755511	0.366160015543915\\
1.19655191858787	0.360721442885771\\
1.20287059404896	0.344689378757515\\
1.20909740780311	0.328657314629258\\
1.21042084168337	0.325222117895742\\
1.21532898133727	0.312625250501002\\
1.22149536367499	0.296593186372745\\
1.22645290581162	0.283534220510624\\
1.2275947227263	0.280561122244489\\
1.2337021375955	0.264529058116232\\
1.23972077261114	0.248496993987976\\
1.24248496993988	0.241057259858942\\
1.24571542558249	0.232464929859719\\
1.25167685511604	0.216432865731463\\
1.25755193846325	0.200400801603206\\
1.25851703406814	0.197745537121656\\
1.2634381954357	0.18436873747495\\
1.26925765836088	0.168336673346693\\
1.27454909819639	0.153554062354551\\
1.27500191367548	0.152304609218437\\
1.28076696369929	0.13627254509018\\
1.28644814606728	0.120240480961924\\
1.29058116232465	0.108428702295168\\
1.29207665346788	0.104208416833667\\
1.29770473969571	0.0881763527054105\\
1.30325100518834	0.0721442885771539\\
1.30661322645291	0.0623093346043338\\
1.30875930199207	0.0561122244488974\\
1.31425362229937	0.0400801603206409\\
1.3196680212829	0.0240480961923843\\
1.32264529058116	0.0151294340306766\\
1.32505136586401	0.00801603206412782\\
1.3304148080451	-0.00801603206412826\\
1.33570008541116	-0.0240480961923848\\
1.33867735470942	-0.0331867976756923\\
1.34095344682717	-0.0400801603206413\\
1.34618859574639	-0.0561122244488979\\
1.35134719757311	-0.0721442885771544\\
1.35470941883768	-0.0827255891680218\\
1.35646524407426	-0.0881763527054109\\
1.36157438764822	-0.104208416833667\\
1.36660846670857	-0.120240480961924\\
1.37074148296593	-0.133584947212763\\
1.37158555308672	-0.13627254509018\\
1.3765706871207	-0.152304609218437\\
1.38148210725868	-0.168336673346694\\
1.38632129259891	-0.18436873747495\\
1.38677354709419	-0.185881810670942\\
1.39117509288114	-0.200400801603207\\
1.39596543227035	-0.216432865731463\\
1.40068469352238	-0.232464929859719\\
1.40280561122244	-0.23975740620047\\
1.40538428971324	-0.248496993987976\\
1.41005484300632	-0.264529058116232\\
1.41465534948129	-0.280561122244489\\
1.4188376753507	-0.295351316165849\\
1.4191940356332	-0.296593186372745\\
1.4237458150046	-0.312625250501002\\
1.4282284552075	-0.328657314629258\\
1.43264320670604	-0.344689378757515\\
1.43486973947896	-0.352880829017598\\
1.43703288051172	-0.360721442885771\\
1.44139826164899	-0.376753507014028\\
1.44569647964082	-0.392785571142285\\
1.44992868785174	-0.408817635270541\\
1.45090180360721	-0.412549478751272\\
1.45415805356282	-0.424849699398798\\
1.45833994169558	-0.440881763527054\\
1.46245636694497	-0.456913827655311\\
1.46650838862278	-0.472945891783567\\
1.46693386773547	-0.474650394281346\\
1.47056559973766	-0.488977955911824\\
1.47456604049386	-0.50501002004008\\
1.47850244808027	-0.521042084168337\\
1.48237579019866	-0.537074148296593\\
1.48296593186373	-0.539549651410414\\
1.48624839816075	-0.55310621242485\\
1.49006858026906	-0.569138276553106\\
1.49382589172678	-0.585170340681363\\
1.49752121047654	-0.601202404809619\\
1.49899799599198	-0.607705927633412\\
1.50119610909336	-0.617234468937876\\
1.50483636086096	-0.633266533066132\\
1.50841463880758	-0.649298597194389\\
1.51193173238232	-0.665330661322646\\
1.51503006012024	-0.679696996365562\\
1.5153950975408	-0.681362725450902\\
1.51885487702331	-0.697394789579158\\
1.52225331347104	-0.713426853707415\\
1.52559110852095	-0.729458917835671\\
1.52886893167208	-0.745490981963928\\
1.5310621242485	-0.756408952483932\\
1.53210621144258	-0.761523046092184\\
1.53532310962606	-0.777555110220441\\
1.53847964119444	-0.793587174348697\\
1.54157638824944	-0.809619238476954\\
1.54461390135166	-0.82565130260521\\
1.54709418837675	-0.838997182766671\\
1.54760165187168	-0.841683366733467\\
1.5505740365611	-0.857715430861723\\
1.55348655045289	-0.87374749498998\\
1.55633965606131	-0.889779559118236\\
1.55913378459536	-0.905811623246493\\
1.56186933611373	-0.92184368737475\\
1.56312625250501	-0.929367289861169\\
1.56457139531872	-0.937875751503006\\
1.56723569397651	-0.953907815631263\\
1.56984046882894	-0.969939879759519\\
1.57238602985503	-0.985971943887776\\
1.57487265565427	-1.00200400801603\\
1.57730059347975	-1.01803607214429\\
1.57915831663327	-1.03060557202073\\
1.57967867631418	-1.03406813627255\\
1.58202824084217	-1.0501002004008\\
1.58431784226701	-1.06613226452906\\
1.58654763393225	-1.08216432865731\\
1.58871773711821	-1.09819639278557\\
1.59082824094773	-1.11422845691383\\
1.59287920226471	-1.13026052104208\\
1.59487064548539	-1.14629258517034\\
1.59519038076152	-1.14894807619056\\
1.59682818575385	-1.1623246492986\\
1.5987296300381	-1.17835671342685\\
1.6005697919074	-1.19438877755511\\
1.60234859378189	-1.21042084168337\\
1.60406592419264	-1.22645290581162\\
1.60572163749197	-1.24248496993988\\
1.60731555353372	-1.25851703406814\\
1.60884745732288	-1.27454909819639\\
1.61031709863407	-1.29058116232465\\
1.61122244488978	-1.30090426342471\\
1.61173158890952	-1.30661322645291\\
1.61309527227241	-1.32264529058116\\
1.61439429698395	-1.33867735470942\\
1.61562829653605	-1.35470941883768\\
1.61679686674454	-1.37074148296593\\
1.61789956520109	-1.38677354709419\\
1.61893591068928	-1.40280561122244\\
1.61990538256408	-1.4188376753507\\
1.62080742009368	-1.43486973947896\\
1.62164142176267	-1.45090180360721\\
1.6224067445356	-1.46693386773547\\
1.62310270307978	-1.48296593186373\\
1.62372856894605	-1.49899799599198\\
1.62428356970646	-1.51503006012024\\
1.62476688804736	-1.5310621242485\\
1.62517766081672	-1.54709418837675\\
1.62551497802411	-1.56312625250501\\
1.62577788179188	-1.57915831663327\\
1.62596536525592	-1.59519038076152\\
1.62607637141434	-1.61122244488978\\
1.62610979192237	-1.62725450901804\\
1.62606446583149	-1.64328657314629\\
1.62593917827105	-1.65931863727455\\
1.62573265907016	-1.67535070140281\\
1.62544358131785	-1.69138276553106\\
1.62507055985917	-1.70741482965932\\
1.62461214972503	-1.72344689378758\\
1.62406684449307	-1.73947895791583\\
1.62343307457727	-1.75551102204409\\
1.62270920544335	-1.77154308617234\\
1.62189353574725	-1.7875751503006\\
1.62098429539374	-1.80360721442886\\
1.61997964351187	-1.81963927855711\\
1.61887766634416	-1.83567134268537\\
1.61767637504605	-1.85170340681363\\
1.61637370339194	-1.86773547094188\\
1.61496750538409	-1.88376753507014\\
1.61345555276042	-1.8997995991984\\
1.61183553239709	-1.91583166332665\\
1.61122244488978	-1.92154062635485\\
1.61011248878115	-1.93186372745491\\
1.60828067438538	-1.94789579158317\\
1.60633322202377	-1.96392785571142\\
1.60426738422377	-1.97995991983968\\
1.60208031121155	-1.99599198396794\\
1.59976904743487	-2.01202404809619\\
1.59733052793776	-2.02805611222445\\
1.59519038076152	-2.04143268533248\\
1.59476347033196	-2.04408817635271\\
1.59207201391084	-2.06012024048096\\
1.58924270019754	-2.07615230460922\\
1.58627189389273	-2.09218436873747\\
1.58315582550126	-2.10821643286573\\
1.57989058644274	-2.12424849699399\\
1.57915831663327	-2.1277110612458\\
1.5764782936904	-2.14028056112224\\
1.57290909694685	-2.1563126252505\\
1.56917665717801	-2.17234468937876\\
1.56527632738902	-2.18837675350701\\
1.56312625250501	-2.19688521514885\\
1.56120437078298	-2.20440881763527\\
1.55695444683608	-2.22044088176353\\
1.55251945803753	-2.23647294589178\\
1.5478938590338	-2.25250501002004\\
1.54709418837675	-2.25519119398684\\
1.54306685727617	-2.2685370741483\\
1.53803384825403	-2.28456913827655\\
1.53278921849702	-2.30060120240481\\
1.5310621242485	-2.30571529601306\\
1.52731554649612	-2.31663326653307\\
1.52160843687928	-2.33266533066132\\
1.5156649653976	-2.34869739478958\\
1.51503006012024	-2.35036312387492\\
1.50945191118913	-2.36472945891784\\
1.50297916329547	-2.38076152304609\\
1.49899799599198	-2.39029006435056\\
1.49622401985135	-2.39679358717435\\
1.48916536249221	-2.41282565130261\\
1.48296593186373	-2.42638221231704\\
1.48180767585431	-2.42885771543086\\
1.47409706992632	-2.44488977955912\\
1.46693386773547	-2.4592173411896\\
1.4660599246591	-2.46092184368737\\
1.45762008602662	-2.47695390781563\\
1.45090180360721	-2.48925412846316\\
1.4488063422536	-2.49298597194389\\
1.43954643592755	-2.50901803607214\\
1.43486973947896	-2.51685864994032\\
1.42983385031263	-2.5250501002004\\
1.41964657378708	-2.54108216432866\\
1.4188376753507	-2.54232403453555\\
1.40888141830834	-2.55711422845691\\
1.40280561122244	-2.56585381624442\\
1.39755434365011	-2.57314629258517\\
1.38677354709419	-2.58766528351743\\
1.38560707639808	-2.58917835671343\\
1.37292503730557	-2.60521042084168\\
1.37074148296593	-2.6078980187191\\
1.35944897462705	-2.62124248496994\\
1.35470941883768	-2.62669324850733\\
1.34510050105639	-2.6372745490982\\
1.33867735470942	-2.64416791174315\\
1.32975869254771	-2.65330661322645\\
1.32264529058116	-2.660420015193\\
1.31327547851173	-2.66933867735471\\
1.30661322645291	-2.67553578751015\\
1.29546845826235	-2.68537074148297\\
1.29058116232465	-2.68959102694447\\
1.27611129417885	-2.70140280561122\\
1.27454909819639	-2.70265225874734\\
1.25851703406814	-2.71477960525793\\
1.2547851905874	-2.71743486973948\\
1.24248496993988	-2.7260271997387\\
1.23108671830602	-2.73346693386774\\
1.22645290581162	-2.73644003213387\\
1.21042084168337	-2.74606380126247\\
1.20427225725145	-2.74949899799599\\
1.19438877755511	-2.75493757065414\\
1.17835671342685	-2.76309270994689\\
1.17317428761382	-2.76553106212425\\
1.1623246492986	-2.7705684654317\\
1.14629258517034	-2.77738784786229\\
1.13552081220309	-2.7815631262525\\
1.13026052104208	-2.78357893720756\\
1.11422845691383	-2.78917463166277\\
1.09819639278557	-2.79418489914963\\
1.08594047435841	-2.79759519038076\\
1.08216432865731	-2.79863663408102\\
1.06613226452906	-2.80255991511815\\
1.0501002004008	-2.80595976525476\\
1.03406813627254	-2.80885428964474\\
1.01803607214429	-2.81126042517164\\
1.00200400801603	-2.81319399879284\\
0.997331417501759	-2.81362725450902\\
0.985971943887775	-2.814675928256\\
0.969939879759519	-2.81571419710024\\
0.953907815631262	-2.81631895101368\\
0.937875751503006	-2.81650189349093\\
0.921843687374749	-2.8162738607838\\
0.905811623246493	-2.81564485848873\\
0.889779559118236	-2.81462409541754\\
0.878420085504253	-2.81362725450902\\
0.87374749498998	-2.81322297105368\\
0.857715430861724	-2.81145662464762\\
0.841683366733467	-2.80932517220469\\
0.825651302605211	-2.80683508253215\\
0.809619238476954	-2.80399215955377\\
0.793587174348698	-2.80080156334239\\
0.779046159184034	-2.79759519038076\\
0.777555110220441	-2.79727052082353\\
0.761523046092185	-2.79343021580375\\
0.745490981963928	-2.78925838468812\\
0.729458917835672	-2.78475793913733\\
0.718857090061259	-2.7815631262525\\
0.713426853707415	-2.7799455667311\\
0.697394789579159	-2.77484091227589\\
0.681362725450902	-2.76941790865144\\
0.670513087135677	-2.76553106212425\\
0.665330661322646	-2.76369440978076\\
0.649298597194389	-2.75769375545286\\
0.633266533066132	-2.75138106085277\\
0.628713705911686	-2.74949899799599\\
0.617234468937876	-2.74480100256469\\
0.601202404809619	-2.73793042458416\\
0.591235090961049	-2.73346693386774\\
0.585170340681363	-2.7307767665998\\
0.569138276553106	-2.72336002335721\\
0.556838055905581	-2.71743486973948\\
0.55310621242485	-2.71565329829787\\
0.537074148296593	-2.70769959745392\\
0.524849509067469	-2.70140280561122\\
0.521042084168337	-2.69945820243091\\
0.50501002004008	-2.69097432997933\\
0.494783771844953	-2.68537074148297\\
0.488977955911824	-2.68221462676709\\
0.472945891783567	-2.67320501962568\\
0.466283639724747	-2.66933867735471\\
0.456913827655311	-2.66394142181914\\
0.440881763527054	-2.65440822496289\\
0.439082402647767	-2.65330661322645\\
0.424849699398798	-2.64465327283535\\
0.413074526782554	-2.6372745490982\\
0.408817635270541	-2.63462480283205\\
0.392785571142285	-2.6243608052396\\
0.388046015352912	-2.62124248496994\\
0.376753507014028	-2.6138584373754\\
0.363888837953899	-2.60521042084168\\
0.360721442885771	-2.60309375284415\\
0.344689378757515	-2.59210255044851\\
0.340529533563966	-2.58917835671343\\
0.328657314629258	-2.5808776181457\\
0.317876518073336	-2.57314629258517\\
0.312625250501002	-2.5693997148328\\
0.296593186372745	-2.55767866334882\\
0.295837892673827	-2.55711422845691\\
0.280561122244489	-2.54575051469963\\
0.274429644828789	-2.54108216432866\\
0.264529058116232	-2.53357697573616\\
0.253532883154308	-2.5250501002004\\
0.248496993987976	-2.52116109576389\\
0.233116340787461	-2.50901803607214\\
0.232464929859719	-2.50850572468606\\
0.216432865731463	-2.49564312778937\\
0.213188106974364	-2.49298597194389\\
0.200400801603206	-2.48254928829144\\
0.193682519183801	-2.47695390781563\\
0.18436873747495	-2.46922073142719\\
0.174568524887323	-2.46092184368737\\
0.168336673346693	-2.45565956781352\\
0.155824596672748	-2.44488977955912\\
0.152304609218437	-2.44186772271686\\
0.137430801099602	-2.42885771543086\\
0.13627254509018	-2.42784693855345\\
0.120240480961924	-2.41360762885972\\
0.119374742175997	-2.41282565130261\\
0.104208416833667	-2.39915131035514\\
0.101637517500693	-2.39679358717435\\
0.0881763527054105	-2.38446763068815\\
0.0841951854019198	-2.38076152304609\\
0.0721442885771539	-2.36955763291399\\
0.0670336762908576	-2.36472945891784\\
0.0561122244488974	-2.35442218420181\\
0.0501399221213309	-2.34869739478958\\
0.0400801603206409	-2.33906197695177\\
0.0335017835616004	-2.33266533066132\\
0.0240480961923843	-2.32347752968626\\
0.0171079835705493	-2.31663326653307\\
0.00801603206412782	-2.30766918771692\\
0.00094804675201682	-2.30060120240481\\
-0.00801603206412826	-2.29163712358866\\
-0.0149877560696652	-2.28456913827655\\
-0.0240480961923848	-2.27538133730149\\
-0.0307084585490926	-2.2685370741483\\
-0.0400801603206413	-2.25890165631049\\
-0.046222445236839	-2.25250501002004\\
-0.0561122244488979	-2.24219773530401\\
-0.0615374941096744	-2.23647294589178\\
-0.0721442885771544	-2.22526905575968\\
-0.0766608155678955	-2.22044088176353\\
-0.0881763527054109	-2.20811492527733\\
-0.0915990882169157	-2.20440881763527\\
-0.104208416833667	-2.19073447668781\\
-0.106358491717682	-2.18837675350701\\
-0.120240480961924	-2.17312666693587\\
-0.120944736961575	-2.17234468937876\\
-0.13536473928602	-2.1563126252505\\
-0.13627254509018	-2.15530184837309\\
-0.149624586275567	-2.14028056112224\\
-0.152304609218437	-2.13725850427999\\
-0.163728008132807	-2.12424849699399\\
-0.168336673346694	-2.11898622112013\\
-0.177679279651716	-2.10821643286573\\
-0.18436873747495	-2.10048325647729\\
-0.191482314734416	-2.09218436873747\\
-0.200400801603207	-2.08174768508503\\
-0.205140685797516	-2.07615230460922\\
-0.216432865731463	-2.06277739632644\\
-0.218657641608958	-2.06012024048096\\
-0.232038019430153	-2.04408817635271\\
-0.232464929859719	-2.04357586496662\\
-0.245294482052848	-2.02805611222445\\
-0.248496993987976	-2.02416710778794\\
-0.258420727231884	-2.01202404809619\\
-0.264529058116232	-2.0045188595037\\
-0.271418988566259	-1.99599198396794\\
-0.280561122244489	-1.98462827021066\\
-0.284291241551083	-1.97995991983968\\
-0.296593186372745	-1.96449229060333\\
-0.297039214899656	-1.96392785571142\\
-0.309683479996603	-1.94789579158317\\
-0.312625250501002	-1.94414921383079\\
-0.322212717668508	-1.93186372745491\\
-0.328657314629258	-1.92356298888719\\
-0.334625405535997	-1.91583166332665\\
-0.344689378757515	-1.90272379293348\\
-0.346922486628155	-1.8997995991984\\
-0.359116872563304	-1.88376753507014\\
-0.360721442885771	-1.88165086707261\\
-0.371216990235094	-1.86773547094188\\
-0.376753507014028	-1.86035142334735\\
-0.383207437170299	-1.85170340681363\\
-0.392785571142285	-1.83878966295503\\
-0.395088476620055	-1.83567134268537\\
-0.406877006331786	-1.81963927855711\\
-0.408817635270541	-1.81698953229097\\
-0.418579485774505	-1.80360721442886\\
-0.424849699398798	-1.79495387403776\\
-0.430176934846032	-1.7875751503006\\
-0.440881763527054	-1.77264469790878\\
-0.441669019392435	-1.77154308617234\\
-0.453092393214548	-1.75551102204409\\
-0.456913827655311	-1.75011376650852\\
-0.46442138072724	-1.73947895791583\\
-0.472945891783567	-1.72731323605855\\
-0.475648009534168	-1.72344689378758\\
-0.486794006752962	-1.70741482965932\\
-0.488977955911824	-1.70425871494344\\
-0.497867416095764	-1.69138276553106\\
-0.50501002004008	-1.68095428989917\\
-0.508840595929195	-1.67535070140281\\
-0.519726753421353	-1.65931863727455\\
-0.521042084168337	-1.65737403409424\\
-0.530555404980253	-1.64328657314629\\
-0.537074148296593	-1.63355130086073\\
-0.54128514558374	-1.62725450901804\\
-0.551928074821154	-1.61122244488978\\
-0.55310621242485	-1.60944087344817\\
-0.562520884658426	-1.59519038076152\\
-0.569138276553106	-1.585083470251\\
-0.573015347633412	-1.57915831663327\\
-0.583430809687441	-1.56312625250501\\
-0.585170340681363	-1.56043608523708\\
-0.593794926995524	-1.54709418837675\\
-0.601202404809619	-1.53552561496492\\
-0.604060554477165	-1.5310621242485\\
-0.614263529626301	-1.51503006012024\\
-0.617234468937876	-1.51033206468894\\
-0.62440445512755	-1.49899799599198\\
-0.633266533066132	-1.4848479947205\\
-0.634446080277042	-1.48296593186373\\
-0.644450832711952	-1.46693386773547\\
-0.649298597194389	-1.45909656106408\\
-0.654372519869492	-1.45090180360721\\
-0.664208689869032	-1.43486973947896\\
-0.665330661322646	-1.43303308713547\\
-0.67401359899695	-1.4188376753507\\
-0.681362725450902	-1.40669245774963\\
-0.683718539634568	-1.40280561122244\\
-0.693377565915689	-1.38677354709419\\
-0.697394789579158	-1.38005133311758\\
-0.702969211433921	-1.37074148296593\\
-0.712472211428738	-1.35470941883768\\
-0.713426853707415	-1.35309185931627\\
-0.721953017014737	-1.33867735470942\\
-0.729458917835671	-1.32584010346599\\
-0.7313317452183	-1.32264529058116\\
-0.740677067007647	-1.30661322645291\\
-0.745490981963928	-1.29827642076026\\
-0.749947267981696	-1.29058116232465\\
-0.759148058525283	-1.27454909819639\\
-0.761523046092184	-1.27038412361938\\
-0.76831154460939	-1.25851703406814\\
-0.777372290686371	-1.24248496993988\\
-0.777555110220441	-1.24216030038265\\
-0.786430653651558	-1.22645290581162\\
-0.793587174348697	-1.213627214645\\
-0.795383112446185	-1.21042084168337\\
-0.804310296988349	-1.19438877755511\\
-0.809619238476954	-1.18475368259986\\
-0.813158487753533	-1.17835671342685\\
-0.821955812894222	-1.1623246492986\\
-0.82565130260521	-1.15553247732173\\
-0.830700749518797	-1.14629258517034\\
-0.839372188236656	-1.13026052104208\\
-0.841683366733467	-1.12595843873776\\
-0.848014685479281	-1.11422845691383\\
-0.856564069847169	-1.09819639278557\\
-0.857715430861723	-1.09602576292418\\
-0.865104748160702	-1.08216432865731\\
-0.873535775099906	-1.06613226452906\\
-0.87374749498998	-1.06572798107372\\
-0.881975064979865	-1.0501002004008\\
-0.889779559118236	-1.03506497718107\\
-0.890299918799153	-1.03406813627255\\
-0.898629447604153	-1.01803607214429\\
-0.905811623246493	-1.00402161199574\\
-0.906851749946105	-1.00200400801603\\
-0.915071400596516	-0.985971943887776\\
-0.92184368737475	-0.972586486034305\\
-0.923191066272462	-0.969939879759519\\
-0.931304129372398	-0.953907815631263\\
-0.937875751503006	-0.94075039048492\\
-0.939320894316716	-0.937875751503006\\
-0.947330547492816	-0.92184368737475\\
-0.953907815631263	-0.908503319751151\\
-0.955243971041259	-0.905811623246493\\
-0.96315328331582	-0.889779559118236\\
-0.969939879759519	-0.875834437581199\\
-0.970962750081582	-0.87374749498998\\
-0.978774686026536	-0.857715430861723\\
-0.985971943887776	-0.842732040480445\\
-0.986479407382707	-0.841683366733467\\
-0.994196831064096	-0.82565130260521\\
-1.00179955554845	-0.809619238476954\\
-1.00200400801603	-0.809185982760772\\
-1.00942152496197	-0.793587174348697\\
-1.01693390146419	-0.777555110220441\\
-1.01803607214429	-0.775188280883064\\
-1.02445030961641	-0.761523046092184\\
-1.03187494369613	-0.745490981963928\\
-1.03406813627255	-0.740718017099646\\
-1.03928446599602	-0.729458917835671\\
-1.04662387524361	-0.713426853707415\\
-1.0501002004008	-0.705759364453155\\
-1.05392501730387	-0.697394789579158\\
-1.0611816344267	-0.681362725450902\\
-1.06613226452906	-0.670295386060035\\
-1.06837273160182	-0.665330661322646\\
-1.07554890734466	-0.649298597194389\\
-1.08216432865731	-0.634307976766393\\
-1.0826281239054	-0.633266533066132\\
-1.08972612981108	-0.617234468937876\\
-1.09671960727013	-0.601202404809619\\
-1.09819639278557	-0.59779211357849\\
-1.10371348877188	-0.585170340681363\\
-1.11063195689279	-0.569138276553106\\
-1.11422845691383	-0.560717717835118\\
-1.11751092321085	-0.55310621242485\\
-1.12435644545861	-0.537074148296593\\
-1.13026052104208	-0.523057895123392\\
-1.13111812454601	-0.521042084168337\\
-1.13789269380047	-0.50501002004008\\
-1.14456803763354	-0.488977955911824\\
-1.14629258517034	-0.484802677521605\\
-1.15124007716546	-0.472945891783567\\
-1.1578471485081	-0.456913827655311\\
-1.1623246492986	-0.445919166834502\\
-1.16439772446391	-0.440881763527054\\
-1.17093832854351	-0.424849699398798\\
-1.17738359767137	-0.408817635270541\\
-1.17835671342685	-0.406379283093181\\
-1.18384039529476	-0.392785571142285\\
-1.19022155276666	-0.376753507014028\\
-1.19438877755511	-0.366160015543915\\
-1.19655191858787	-0.360721442885771\\
-1.20287059404896	-0.344689378757515\\
-1.20909740780311	-0.328657314629258\\
-1.21042084168337	-0.325222117895742\\
-1.21532898133727	-0.312625250501002\\
-1.22149536367499	-0.296593186372745\\
-1.22645290581162	-0.283534220510624\\
-1.2275947227263	-0.280561122244489\\
-1.2337021375955	-0.264529058116232\\
-1.23972077261114	-0.248496993987976\\
-1.24248496993988	-0.241057259858942\\
-1.24571542558249	-0.232464929859719\\
-1.25167685511604	-0.216432865731463\\
-1.25755193846325	-0.200400801603207\\
-1.25851703406814	-0.197745537121656\\
-1.2634381954357	-0.18436873747495\\
-1.26925765836088	-0.168336673346694\\
-1.27454909819639	-0.153554062354551\\
-1.27500191367548	-0.152304609218437\\
-1.28076696369929	-0.13627254509018\\
-1.28644814606728	-0.120240480961924\\
-1.29058116232465	-0.108428702295168\\
-1.29207665346788	-0.104208416833667\\
-1.29770473969571	-0.0881763527054109\\
-1.30325100518834	-0.0721442885771544\\
-1.30661322645291	-0.0623093346043342\\
-1.30875930199207	-0.0561122244488979\\
-1.31425362229937	-0.0400801603206413\\
-1.3196680212829	-0.0240480961923848\\
-1.32264529058116	-0.0151294340306766\\
-1.32505136586401	-0.00801603206412826\\
-1.3304148080451	0.00801603206412782\\
-1.33570008541116	0.0240480961923843\\
-1.33867735470942	0.0331867976756918\\
-1.34095344682717	0.0400801603206409\\
-1.34618859574639	0.0561122244488974\\
-1.35134719757311	0.0721442885771539\\
-1.35470941883768	0.0827255891680218\\
-1.35646524407426	0.0881763527054105\\
-1.36157438764822	0.104208416833667\\
-1.36660846670857	0.120240480961924\\
-1.37074148296593	0.133584947212763\\
-1.37158555308672	0.13627254509018\\
-1.3765706871207	0.152304609218437\\
-1.38148210725868	0.168336673346693\\
-1.38632129259891	0.18436873747495\\
-1.38677354709419	0.185881810670943\\
-1.39117509288114	0.200400801603206\\
-1.39596543227035	0.216432865731463\\
-1.40068469352238	0.232464929859719\\
-1.40280561122244	0.23975740620047\\
-1.40538428971324	0.248496993987976\\
-1.41005484300632	0.264529058116232\\
-1.41465534948129	0.280561122244489\\
-1.4188376753507	0.29535131616585\\
-1.4191940356332	0.296593186372745\\
-1.4237458150046	0.312625250501002\\
-1.4282284552075	0.328657314629258\\
-1.43264320670604	0.344689378757515\\
-1.43486973947896	0.352880829017598\\
-1.43703288051172	0.360721442885771\\
-1.44139826164899	0.376753507014028\\
-1.44569647964082	0.392785571142285\\
-1.44992868785174	0.408817635270541\\
-1.45090180360721	0.412549478751272\\
-1.45415805356282	0.424849699398798\\
-1.45833994169558	0.440881763527054\\
-1.46245636694497	0.456913827655311\\
-1.46650838862278	0.472945891783567\\
-1.46693386773547	0.474650394281346\\
-1.47056559973766	0.488977955911824\\
-1.47456604049386	0.50501002004008\\
-1.47850244808027	0.521042084168337\\
-1.48237579019867	0.537074148296593\\
-1.48296593186373	0.539549651410414\\
-1.48624839816075	0.55310621242485\\
-1.49006858026906	0.569138276553106\\
-1.49382589172678	0.585170340681363\\
-1.49752121047654	0.601202404809619\\
-1.49899799599198	0.607705927633412\\
-1.50119610909336	0.617234468937876\\
-1.50483636086096	0.633266533066132\\
-1.50841463880758	0.649298597194389\\
-1.51193173238232	0.665330661322646\\
-1.51503006012024	0.679696996365563\\
-1.5153950975408	0.681362725450902\\
-1.51885487702331	0.697394789579159\\
-1.52225331347104	0.713426853707415\\
-1.52559110852095	0.729458917835672\\
-1.52886893167208	0.745490981963928\\
-1.5310621242485	0.756408952483931\\
-1.53210621144258	0.761523046092185\\
-1.53532310962606	0.777555110220441\\
-1.53847964119444	0.793587174348698\\
-1.54157638824944	0.809619238476954\\
-1.54461390135166	0.825651302605211\\
-1.54709418837675	0.838997182766671\\
-1.54760165187168	0.841683366733467\\
-1.5505740365611	0.857715430861724\\
-1.55348655045289	0.87374749498998\\
-1.55633965606131	0.889779559118236\\
-1.55913378459536	0.905811623246493\\
-1.56186933611373	0.921843687374749\\
-1.56312625250501	0.929367289861167\\
-1.56457139531872	0.937875751503006\\
-1.56723569397651	0.953907815631262\\
-1.56984046882894	0.969939879759519\\
-1.57238602985503	0.985971943887775\\
-1.57487265565427	1.00200400801603\\
-1.57730059347975	1.01803607214429\\
-1.57915831663327	1.03060557202073\\
-1.57967867631418	1.03406813627254\\
-1.58202824084217	1.0501002004008\\
-1.58431784226701	1.06613226452906\\
-1.58654763393225	1.08216432865731\\
-1.58871773711821	1.09819639278557\\
-1.59082824094773	1.11422845691383\\
-1.59287920226471	1.13026052104208\\
-1.59487064548539	1.14629258517034\\
-1.59519038076152	1.14894807619056\\
-1.59682818575385	1.1623246492986\\
-1.5987296300381	1.17835671342685\\
-1.6005697919074	1.19438877755511\\
-1.60234859378189	1.21042084168337\\
-1.60406592419264	1.22645290581162\\
-1.60572163749197	1.24248496993988\\
-1.60731555353372	1.25851703406814\\
-1.60884745732288	1.27454909819639\\
-1.61031709863407	1.29058116232465\\
-1.61122244488978	1.30090426342471\\
-1.61173158890952	1.30661322645291\\
-1.61309527227241	1.32264529058116\\
-1.61439429698395	1.33867735470942\\
-1.61562829653605	1.35470941883768\\
-1.61679686674454	1.37074148296593\\
-1.61789956520109	1.38677354709419\\
-1.61893591068928	1.40280561122244\\
-1.61990538256409	1.4188376753507\\
-1.62080742009368	1.43486973947896\\
-1.62164142176267	1.45090180360721\\
-1.6224067445356	1.46693386773547\\
-1.62310270307978	1.48296593186373\\
-1.62372856894605	1.49899799599198\\
-1.62428356970646	1.51503006012024\\
-1.62476688804736	1.5310621242485\\
-1.62517766081672	1.54709418837675\\
-1.62551497802411	1.56312625250501\\
-1.62577788179189	1.57915831663327\\
-1.62596536525592	1.59519038076152\\
-1.62607637141434	1.61122244488978\\
-1.62610979192237	1.62725450901804\\
-1.62606446583149	1.64328657314629\\
-1.62593917827105	1.65931863727455\\
-1.62573265907016	1.67535070140281\\
-1.62544358131785	1.69138276553106\\
-1.62507055985917	1.70741482965932\\
-1.62461214972503	1.72344689378758\\
-1.62406684449307	1.73947895791583\\
-1.62343307457727	1.75551102204409\\
-1.62270920544335	1.77154308617235\\
-1.62189353574725	1.7875751503006\\
-1.62098429539374	1.80360721442886\\
-1.61997964351187	1.81963927855711\\
-1.61887766634416	1.83567134268537\\
-1.61767637504605	1.85170340681363\\
-1.61637370339194	1.86773547094188\\
-1.61496750538409	1.88376753507014\\
-1.61345555276042	1.8997995991984\\
-1.61183553239709	1.91583166332665\\
-1.61122244488978	1.92154062635485\\
}--cycle;


\addplot[area legend,solid,fill=mycolor5,draw=black,forget plot]
table[row sep=crcr] {%
x	y\\
-1.43486973947896	1.57614782301339\\
-1.43471943107953	1.57915831663327\\
-1.43381826014665	1.59519038076152\\
-1.43280952025882	1.61122244488978\\
-1.43169085204059	1.62725450901804\\
-1.4304598001939	1.64328657314629\\
-1.42911381027157	1.65931863727455\\
-1.4276502253059	1.67535070140281\\
-1.42606628228583	1.69138276553106\\
-1.42435910847566	1.70741482965932\\
-1.4225257175681	1.72344689378758\\
-1.42056300566399	1.73947895791583\\
-1.4188376753507	1.75270116524991\\
-1.41846804884804	1.75551102204409\\
-1.41623805524873	1.77154308617235\\
-1.41386758907841	1.7875751503006\\
-1.41135291389638	1.80360721442886\\
-1.40869014715433	1.81963927855711\\
-1.40587525464174	1.83567134268537\\
-1.40290404467388	1.85170340681363\\
-1.40280561122244	1.85221236475073\\
-1.39976673853409	1.86773547094188\\
-1.39646204917176	1.88376753507014\\
-1.39298525898563	1.8997995991984\\
-1.38933130901527	1.91583166332665\\
-1.38677354709419	1.92656076872958\\
-1.38549017444567	1.93186372745491\\
-1.38144940352448	1.94789579158317\\
-1.37721190834076	1.96392785571142\\
-1.37277155563299	1.97995991983968\\
-1.37074148296593	1.98700845858851\\
-1.36810682049629	1.99599198396794\\
-1.36321095195326	2.01202404809619\\
-1.35808788971171	2.02805611222445\\
-1.35470941883768	2.03821482835606\\
-1.35271501593282	2.04408817635271\\
-1.34706954954247	2.06012024048096\\
-1.34116778578055	2.07615230460922\\
-1.33867735470942	2.08268121363305\\
-1.33496551159711	2.09218436873747\\
-1.3284588477304	2.10821643286573\\
-1.32264529058116	2.12195623172791\\
-1.32164969012634	2.12424849699399\\
-1.31446628330718	2.14028056112224\\
-1.30696119426195	2.1563126252505\\
-1.30661322645291	2.1570351362201\\
-1.29902049060337	2.17234468937876\\
-1.29071804569593	2.18837675350701\\
-1.29058116232465	2.18863394640411\\
-1.28191145320043	2.20440881763527\\
-1.27454909819639	2.2172723129112\\
-1.27267209655242	2.22044088176353\\
-1.26287839719742	2.23647294589178\\
-1.25851703406814	2.2433713531395\\
-1.25252169552617	2.25250501002004\\
-1.24248496993988	2.26725508244213\\
-1.24157692121844	2.2685370741483\\
-1.2298926066466	2.28456913827655\\
-1.22645290581162	2.28914032936524\\
-1.2174438451519	2.30060120240481\\
-1.21042084168337	2.3092557276721\\
-1.20414790670024	2.31663326653307\\
-1.19438877755511	2.32776952279695\\
-1.18987869651441	2.33266533066132\\
-1.17835671342685	2.34481988990147\\
-1.17447920853874	2.34869739478958\\
-1.1623246492986	2.36052717493355\\
-1.15775251694412	2.36472945891784\\
-1.14629258517034	2.37499623458497\\
-1.13944940232465	2.38076152304609\\
-1.13026052104208	2.38831840762883\\
-1.11925105511328	2.39679358717435\\
-1.11422845691383	2.40057318173519\\
-1.09819639278557	2.41182795528761\\
-1.09667379717982	2.41282565130261\\
-1.08216432865731	2.42214004892704\\
-1.07080768800794	2.42885771543086\\
-1.06613226452906	2.4315707992682\\
-1.0501002004008	2.4401641685971\\
-1.0404549230817	2.44488977955912\\
-1.03406813627255	2.4479659469966\\
-1.01803607214429	2.45501471015489\\
-1.00309160510989	2.46092184368737\\
-1.00200400801603	2.46134532663581\\
-0.985971943887776	2.46699364640741\\
-0.969939879759519	2.47198542314756\\
-0.953907815631263	2.47634725722471\\
-0.951348053022255	2.47695390781563\\
-0.937875751503006	2.48011007091374\\
-0.92184368737475	2.48329258491622\\
-0.905811623246493	2.48591508836163\\
-0.889779559118236	2.48799746033493\\
-0.87374749498998	2.4895582254375\\
-0.857715430861723	2.49061462510628\\
-0.841683366733467	2.49118268369602\\
-0.82565130260521	2.49127726968805\\
-0.809619238476954	2.49091215235856\\
-0.793587174348697	2.49010005421127\\
-0.777555110220441	2.48885269945374\\
-0.761523046092184	2.48718085877326\\
-0.745490981963928	2.48509439064656\\
-0.729458917835671	2.48260227939784\\
-0.713426853707415	2.47971267020162\\
-0.699954552188166	2.47695390781563\\
-0.697394789579158	2.47643563232263\\
-0.681362725450902	2.47279233708659\\
-0.665330661322646	2.46877490054165\\
-0.649298597194389	2.46438859305215\\
-0.637613172394768	2.46092184368737\\
-0.633266533066132	2.45964550325484\\
-0.617234468937876	2.45456571746513\\
-0.601202404809619	2.44913301662021\\
-0.589448840589472	2.44488977955912\\
-0.585170340681363	2.44335978778237\\
-0.569138276553106	2.43726879886481\\
-0.55310621242485	2.43083597158032\\
-0.548430788945969	2.42885771543086\\
-0.537074148296593	2.42409432489073\\
-0.521042084168337	2.41703015422659\\
-0.511937632205333	2.41282565130261\\
-0.50501002004008	2.40965263427974\\
-0.488977955911824	2.40197261032635\\
-0.47861633688666	2.39679358717435\\
-0.472945891783567	2.39398111250709\\
-0.456913827655311	2.38569774828604\\
-0.447724946372745	2.38076152304609\\
-0.440881763527054	2.37711175756346\\
-0.424849699398798	2.36823470705649\\
-0.418742219050504	2.36472945891784\\
-0.408817635270541	2.35907140958386\\
-0.392785571142285	2.34960756594041\\
-0.391293580145418	2.34869739478958\\
-0.376753507014028	2.33988194566119\\
-0.365231523926469	2.33266533066132\\
-0.360721442885771	2.3298571615125\\
-0.344689378757515	2.31956044169135\\
-0.340265649230689	2.31663326653307\\
-0.328657314629258	2.30899339723786\\
-0.316276698443884	2.30060120240481\\
-0.312625250501002	2.298138761015\\
-0.296593186372745	2.28701987652264\\
-0.293153485537765	2.28456913827655\\
-0.280561122244489	2.27563866976863\\
-0.270812795449883	2.2685370741483\\
-0.264529058116232	2.26397936875109\\
-0.249116841161576	2.25250501002004\\
-0.248496993987976	2.25204542916615\\
-0.232464929859719	2.23986734300685\\
-0.228103566730434	2.23647294589178\\
-0.216432865731463	2.22742240388654\\
-0.20764169030169	2.22044088176353\\
-0.200400801603207	2.21470970355198\\
-0.187683817264239	2.20440881763527\\
-0.18436873747495	2.20173183890034\\
-0.168336673346694	2.18849214306497\\
-0.16819978997541	2.18837675350701\\
-0.152304609218437	2.17501187444054\\
-0.149198527430876	2.17234468937876\\
-0.13627254509018	2.16127082520807\\
-0.13060223360748	2.1563126252505\\
-0.120240480961924	2.14727077441581\\
-0.112387424107648	2.14028056112224\\
-0.104208416833667	2.13301329770614\\
-0.0945324789276519	2.12424849699399\\
-0.0881763527054109	2.11849976935194\\
-0.0770174803552489	2.10821643286573\\
-0.0721442885771544	2.10373136403057\\
-0.0598240675612109	2.09218436873747\\
-0.0561122244488979	2.08870905833842\\
-0.042935305942095	2.07615230460922\\
-0.0400801603206413	2.07343363204895\\
-0.0263355695258069	2.06012024048096\\
-0.0240480961923848	2.05790566911636\\
-0.0100104349689827	2.04408817635271\\
-0.00801603206412826	2.04212555842685\\
0.00605341413827421	2.02805611222445\\
0.00801603206412782	2.02609349429859\\
0.0218682737844108	2.01202404809619\\
0.0240480961923843	2.0098094767316\\
0.0374454978510013	1.99599198396794\\
0.0400801603206409	1.99327331140766\\
0.0527955690962666	1.97995991983968\\
0.0561122244488974	1.97648460944062\\
0.0679281635703945	1.96392785571142\\
0.0721442885771539	1.95944278687626\\
0.082852209135698	1.94789579158317\\
0.0881763527054105	1.94214706394112\\
0.0975759386864971	1.93186372745491\\
0.104208416833667	1.9245964640388\\
0.112106938595429	1.91583166332665\\
0.120240480961924	1.90678981249196\\
0.126452192853365	1.8997995991984\\
0.13627254509018	1.88872573502771\\
0.14061812331798	1.88376753507014\\
0.152304609218437	1.87040265600366\\
0.154610626440267	1.86773547094188\\
0.168336673346693	1.85181879637159\\
0.16843510679813	1.85170340681363\\
0.182094263918885	1.83567134268537\\
0.18436873747495	1.83299436395044\\
0.195597114389196	1.81963927855711\\
0.200400801603206	1.81390810034557\\
0.208948104277145	1.80360721442886\\
0.216432865731463	1.79455667242361\\
0.222151133760788	1.7875751503006\\
0.232464929859719	1.77493748328741\\
0.235209729534101	1.77154308617235\\
0.248127367485314	1.75551102204409\\
0.248496993987976	1.7550514411902\\
0.260910749817074	1.73947895791583\\
0.264529058116232	1.73492125251862\\
0.273561692861503	1.72344689378758\\
0.280561122244489	1.71451642527965\\
0.286082555369444	1.70741482965932\\
0.296593186372745	1.69383350377715\\
0.298475405129103	1.69138276553106\\
0.31074681341069	1.67535070140281\\
0.312625250501002	1.672888260013\\
0.322901385421872	1.65931863727455\\
0.328657314629258	1.65167876797935\\
0.334936639563524	1.64328657314629\\
0.344689378757515	1.63018168417632\\
0.346853715889065	1.62725450901804\\
0.358661223665631	1.61122244488978\\
0.360721442885771	1.60841427574096\\
0.370363623606131	1.59519038076152\\
0.376753507014028	1.58637493163314\\
0.381954431452667	1.57915831663327\\
0.392785571142285	1.56403642365584\\
0.393433997955205	1.56312625250501\\
0.404822061190901	1.54709418837675\\
0.408817635270541	1.54143613904278\\
0.416106673164823	1.5310621242485\\
0.424849699398798	1.51853530825889\\
0.427284817757179	1.51503006012024\\
0.438370195933025	1.49899799599198\\
0.440881763527054	1.49534823050935\\
0.44936577793113	1.48296593186373\\
0.456913827655311	1.47187009297541\\
0.460258527737727	1.46693386773547\\
0.47105948890644	1.45090180360721\\
0.472945891783567	1.44808932893996\\
0.481780563993476	1.43486973947896\\
0.488977955911824	1.42401669850271\\
0.492401403989198	1.4188376753507\\
0.502935212529043	1.40280561122244\\
0.50501002004008	1.39963259419958\\
0.513393889713681	1.38677354709419\\
0.521042084168337	1.37494598588992\\
0.523753976742535	1.37074148296593\\
0.534036970397823	1.35470941883768\\
0.537074148296593	1.34994602829754\\
0.54424314245709	1.33867735470942\\
0.55310621242485	1.32462354673062\\
0.554351490046198	1.32264529058116\\
0.564399151523312	1.30661322645291\\
0.569138276553106	1.29899224575859\\
0.574360665418839	1.29058116232465\\
0.584232230108001	1.27454909819639\\
0.585170340681363	1.27301910641964\\
0.594051326219487	1.25851703406814\\
0.601202404809619	1.24672820700097\\
0.603774129129969	1.24248496993988\\
0.613432791996306	1.22645290581162\\
0.617234468937876	1.22009677958938\\
0.62301858987926	1.21042084168337\\
0.632513854057083	1.19438877755511\\
0.633266533066132	1.19311243712257\\
0.641966158922827	1.17835671342685\\
0.649298597194389	1.16579139866338\\
0.651321859979967	1.1623246492986\\
0.660624815805226	1.14629258517034\\
0.665330661322646	1.13811357789636\\
0.669850938713261	1.13026052104208\\
0.679002064889276	1.11422845691383\\
0.681362725450902	1.11006688618478\\
0.688101307851878	1.09819639278557\\
0.69710495342446	1.08216432865731\\
0.697394789579159	1.08164605316432\\
0.706079738533738	1.06613226452906\\
0.713426853707415	1.05285896278679\\
0.714956057633846	1.0501002004008\\
0.723792570976108	1.03406813627254\\
0.729458917835672	1.0236844437265\\
0.732546518759912	1.01803607214429\\
0.741245729576007	1.00200400801603\\
0.745490981963928	0.994112426718699\\
0.749879009420214	0.985971943887775\\
0.75844473699409	0.969939879759519\\
0.761523046092185	0.964134766588891\\
0.766958806572806	0.953907815631262\\
0.775394727267325	0.937875751503006\\
0.777555110220441	0.933742479012859\\
0.783790805290208	0.921843687374749\\
0.79210045799249	0.905811623246493\\
0.793587174348698	0.902925705513879\\
0.800379530336358	0.889779559118236\\
0.808566321619408	0.87374749498998\\
0.809619238476954	0.871673675404656\\
0.816729146876445	0.857715430861724\\
0.824796355889865	0.841683366733467\\
0.825651302605211	0.839974664477626\\
0.832843470353536	0.825651302605211\\
0.840794253455332	0.809619238476954\\
0.841683366733467	0.807815950229083\\
0.848725975563253	0.793587174348698\\
0.856563370704004	0.777555110220441\\
0.857715430861724	0.775183763382832\\
0.864379804955127	0.761523046092185\\
0.872106735825037	0.745490981963928\\
0.87374749498998	0.742063235457539\\
0.879807776186825	0.729458917835672\\
0.887427056135494	0.713426853707415\\
0.889779559118236	0.708438342098458\\
0.895012388955042	0.697394789579159\\
0.902526724693145	0.681362725450902\\
0.905811623246493	0.674291841868642\\
0.909995831124626	0.665330661322646\\
0.917407826216055	0.649298597194389\\
0.921843687374749	0.639605210166722\\
0.924759984175293	0.633266533066132\\
0.932072142327726	0.617234468937876\\
0.937875751503006	0.604358567907728\\
0.939306427983218	0.601202404809619\\
0.946521156144502	0.585170340681363\\
0.953640079044779	0.569138276553106\\
0.953907815631262	0.568531625962186\\
0.960756056219921	0.55310621242485\\
0.967780754132099	0.537074148296593\\
0.969939879759519	0.532105663628523\\
0.974777739858769	0.521042084168337\\
0.981710376076964	0.50501002004008\\
0.985971943887775	0.495049758631863\\
0.988586815811709	0.488977955911824\\
0.99542946636084	0.472945891783567\\
1.00200400801603	0.45733731060375\\
1.00218360635947	0.456913827655311\\
1.00893826228546	0.440881763527054\\
1.01560260849886	0.424849699398798\\
1.01803607214429	0.418942565866309\\
1.02223671829137	0.408817635270541\\
1.02881559873643	0.392785571142285\\
1.03406813627254	0.379829674451507\\
1.03532450668279	0.376753507014028\\
1.04181969582964	0.360721442885771\\
1.04822789550428	0.344689378757515\\
1.0501002004008	0.339963767795497\\
1.05461421190746	0.328657314629258\\
1.06094089149501	0.312625250501002\\
1.06613226452906	0.299306270210088\\
1.06719817822434	0.296593186372745\\
1.07344490733363	0.280561122244489\\
1.07960754402498	0.264529058116232\\
1.08216432865731	0.257811391612406\\
1.08573861773676	0.248496993987976\\
1.09182313616956	0.232464929859719\\
1.09782580665023	0.216432865731463\\
1.09819639278557	0.215435169716469\\
1.10382818894802	0.200400801603206\\
1.10975436746823	0.18436873747495\\
1.11422845691383	0.172116267907533\\
1.11562073759942	0.168336673346693\\
1.12147167987081	0.152304609218437\\
1.12724301738029	0.13627254509018\\
1.13026052104208	0.127797365544665\\
1.13297542122262	0.120240480961924\\
1.13867283245662	0.104208416833667\\
1.14429246627308	0.0881763527054105\\
1.14629258517034	0.0824110642442843\\
1.14988753346395	0.0721442885771539\\
1.15543435708166	0.0561122244488974\\
1.16090507180902	0.0400801603206409\\
1.1623246492986	0.0358778763363548\\
1.16635911296884	0.0240480961923843\\
1.17175794302227	0.00801603206412782\\
1.17708217805388	-0.00801603206412826\\
1.17835671342685	-0.0118935369522394\\
1.1823911770971	-0.0240480961923848\\
1.1876442649766	-0.0400801603206413\\
1.19282412066439	-0.0561122244488979\\
1.19438877755511	-0.0610080323132715\\
1.19798372584872	-0.0721442885771544\\
1.20309298566784	-0.0881763527054109\\
1.20813022818325	-0.104208416833667\\
1.21042084168337	-0.111585955694638\\
1.21313574186391	-0.120240480961924\\
1.21810275454298	-0.13627254509018\\
1.22299881944837	-0.152304609218437\\
1.22645290581162	-0.163765482258007\\
1.22784518649722	-0.168336673346694\\
1.23267120256016	-0.18436873747495\\
1.23742719710257	-0.200400801603207\\
1.24211438380454	-0.216432865731463\\
1.24248496993988	-0.217714857437629\\
1.24679493303642	-0.232464929859719\\
1.25141163716841	-0.248496993987976\\
1.2559602494358	-0.264529058116232\\
1.25851703406814	-0.273662714996768\\
1.26046950850622	-0.280561122244489\\
1.26494737463194	-0.296593186372745\\
1.26935772516235	-0.312625250501002\\
1.27370159972485	-0.328657314629258\\
1.27454909819639	-0.331825883481582\\
1.2780285849571	-0.344689378757515\\
1.28230065775349	-0.360721442885771\\
1.28650662502303	-0.376753507014028\\
1.29058116232465	-0.392528378245189\\
1.29064836143032	-0.392785571142285\\
1.29478180847173	-0.408817635270541\\
1.29884938139001	-0.424849699398798\\
1.30285195075063	-0.440881763527054\\
1.30661322645291	-0.456191316685715\\
1.30679282479634	-0.456913827655311\\
1.31072117881957	-0.472945891783567\\
1.31458455512255	-0.488977955911824\\
1.31838372277035	-0.50501002004008\\
1.32211941541806	-0.521042084168337\\
1.32264529058116	-0.523334349434413\\
1.32583504765005	-0.537074148296593\\
1.32949353116982	-0.55310621242485\\
1.33308829272707	-0.569138276553106\\
1.33661996515266	-0.585170340681363\\
1.33867735470942	-0.594673495785783\\
1.34010803118963	-0.601202404809619\\
1.34355990970964	-0.617234468937876\\
1.34694824139766	-0.633266533066132\\
1.35027355767898	-0.649298597194389\\
1.35353635514788	-0.665330661322646\\
1.35470941883768	-0.671204009319294\\
1.35676341650596	-0.681362725450902\\
1.35994224867448	-0.697394789579158\\
1.36305782068158	-0.713426853707415\\
1.36611052617763	-0.729458917835671\\
1.36910072380099	-0.745490981963928\\
1.37074148296593	-0.754474507343357\\
1.37204493428342	-0.761523046092184\\
1.37494628488202	-0.777555110220441\\
1.37778409179572	-0.793587174348697\\
1.380558607853	-0.809619238476954\\
1.38327005023437	-0.82565130260521\\
1.38591860037884	-0.841683366733467\\
1.38677354709419	-0.846986325458802\\
1.38852514086549	-0.857715430861723\\
1.39107767773152	-0.87374749498998\\
1.39356590308185	-0.889779559118236\\
1.39598988725695	-0.905811623246493\\
1.39834966361338	-0.92184368737475\\
1.40064522826933	-0.937875751503006\\
1.40280561122244	-0.953398857694159\\
1.40287734648678	-0.953907815631263\\
1.40506851430965	-0.969939879759519\\
1.40719363867873	-0.985971943887776\\
1.40925259695101	-1.00200400801603\\
1.41124522734124	-1.01803607214429\\
1.41317132849114	-1.03406813627255\\
1.41503065900035	-1.0501002004008\\
1.4168229369183	-1.06613226452906\\
1.418547839196	-1.08216432865731\\
1.4188376753507	-1.0849741854515\\
1.42021886398182	-1.09819639278557\\
1.42182337322298	-1.11422845691383\\
1.42335795274132	-1.13026052104208\\
1.42482214355233	-1.14629258517034\\
1.42621544244241	-1.1623246492986\\
1.4275373012074	-1.17835671342685\\
1.42878712584424	-1.19438877755511\\
1.4299642756944	-1.21042084168337\\
1.43106806253739	-1.22645290581162\\
1.43209774963311	-1.24248496993988\\
1.43305255071117	-1.25851703406814\\
1.4339316289056	-1.27454909819639\\
1.43473409563297	-1.29058116232465\\
1.43486973947896	-1.29359165594453\\
1.435463735431	-1.30661322645291\\
1.43611501710031	-1.32264529058116\\
1.43668581376201	-1.33867735470942\\
1.43717505973574	-1.35470941883768\\
1.43758163205316	-1.37074148296593\\
1.43790434902837	-1.38677354709419\\
1.43814196875914	-1.40280561122244\\
1.43829318755633	-1.4188376753507\\
1.43835663829887	-1.43486973947896\\
1.43833088871135	-1.45090180360721\\
1.43821443956137	-1.46693386773547\\
1.43800572277337	-1.48296593186373\\
1.43770309945568	-1.49899799599198\\
1.43730485783734	-1.51503006012024\\
1.43680921111101	-1.5310621242485\\
1.43621429517806	-1.54709418837675\\
1.43551816629188	-1.56312625250501\\
1.43486973947896	-1.57614782301339\\
1.43471943107953	-1.57915831663327\\
1.43381826014665	-1.59519038076152\\
1.43280952025882	-1.61122244488978\\
1.43169085204059	-1.62725450901804\\
1.4304598001939	-1.64328657314629\\
1.42911381027157	-1.65931863727455\\
1.4276502253059	-1.67535070140281\\
1.42606628228583	-1.69138276553106\\
1.42435910847566	-1.70741482965932\\
1.4225257175681	-1.72344689378758\\
1.42056300566399	-1.73947895791583\\
1.4188376753507	-1.75270116524991\\
1.41846804884804	-1.75551102204409\\
1.41623805524873	-1.77154308617234\\
1.41386758907841	-1.7875751503006\\
1.41135291389638	-1.80360721442886\\
1.40869014715433	-1.81963927855711\\
1.40587525464174	-1.83567134268537\\
1.40290404467388	-1.85170340681363\\
1.40280561122244	-1.85221236475073\\
1.39976673853409	-1.86773547094188\\
1.39646204917176	-1.88376753507014\\
1.39298525898563	-1.8997995991984\\
1.38933130901527	-1.91583166332665\\
1.38677354709419	-1.92656076872958\\
1.38549017444567	-1.93186372745491\\
1.38144940352448	-1.94789579158317\\
1.37721190834076	-1.96392785571142\\
1.37277155563299	-1.97995991983968\\
1.37074148296593	-1.98700845858851\\
1.36810682049629	-1.99599198396794\\
1.36321095195326	-2.01202404809619\\
1.35808788971171	-2.02805611222445\\
1.35470941883768	-2.03821482835606\\
1.35271501593282	-2.04408817635271\\
1.34706954954247	-2.06012024048096\\
1.34116778578055	-2.07615230460922\\
1.33867735470942	-2.08268121363305\\
1.33496551159711	-2.09218436873747\\
1.3284588477304	-2.10821643286573\\
1.32264529058116	-2.12195623172791\\
1.32164969012635	-2.12424849699399\\
1.31446628330718	-2.14028056112224\\
1.30696119426195	-2.1563126252505\\
1.30661322645291	-2.1570351362201\\
1.29902049060337	-2.17234468937876\\
1.29071804569593	-2.18837675350701\\
1.29058116232465	-2.18863394640411\\
1.28191145320043	-2.20440881763527\\
1.27454909819639	-2.2172723129112\\
1.27267209655242	-2.22044088176353\\
1.26287839719742	-2.23647294589178\\
1.25851703406814	-2.2433713531395\\
1.25252169552617	-2.25250501002004\\
1.24248496993988	-2.26725508244213\\
1.24157692121844	-2.2685370741483\\
1.2298926066466	-2.28456913827655\\
1.22645290581162	-2.28914032936524\\
1.2174438451519	-2.30060120240481\\
1.21042084168337	-2.3092557276721\\
1.20414790670024	-2.31663326653307\\
1.19438877755511	-2.32776952279695\\
1.18987869651441	-2.33266533066132\\
1.17835671342685	-2.34481988990147\\
1.17447920853874	-2.34869739478958\\
1.1623246492986	-2.36052717493355\\
1.15775251694412	-2.36472945891784\\
1.14629258517034	-2.37499623458497\\
1.13944940232465	-2.38076152304609\\
1.13026052104208	-2.38831840762883\\
1.11925105511328	-2.39679358717435\\
1.11422845691383	-2.40057318173519\\
1.09819639278557	-2.41182795528761\\
1.09667379717982	-2.41282565130261\\
1.08216432865731	-2.42214004892704\\
1.07080768800794	-2.42885771543086\\
1.06613226452906	-2.4315707992682\\
1.0501002004008	-2.4401641685971\\
1.0404549230817	-2.44488977955912\\
1.03406813627254	-2.4479659469966\\
1.01803607214429	-2.45501471015489\\
1.00309160510989	-2.46092184368737\\
1.00200400801603	-2.46134532663581\\
0.985971943887775	-2.46699364640741\\
0.969939879759519	-2.47198542314756\\
0.953907815631262	-2.47634725722471\\
0.951348053022254	-2.47695390781563\\
0.937875751503006	-2.48011007091374\\
0.921843687374749	-2.48329258491622\\
0.905811623246493	-2.48591508836163\\
0.889779559118236	-2.48799746033493\\
0.87374749498998	-2.4895582254375\\
0.857715430861724	-2.49061462510628\\
0.841683366733467	-2.49118268369602\\
0.825651302605211	-2.49127726968805\\
0.809619238476954	-2.49091215235856\\
0.793587174348698	-2.49010005421127\\
0.777555110220441	-2.48885269945374\\
0.761523046092185	-2.48718085877326\\
0.745490981963928	-2.48509439064656\\
0.729458917835672	-2.48260227939784\\
0.713426853707415	-2.47971267020162\\
0.699954552188167	-2.47695390781563\\
0.697394789579159	-2.47643563232263\\
0.681362725450902	-2.47279233708659\\
0.665330661322646	-2.46877490054165\\
0.649298597194389	-2.46438859305215\\
0.637613172394769	-2.46092184368737\\
0.633266533066132	-2.45964550325484\\
0.617234468937876	-2.45456571746513\\
0.601202404809619	-2.44913301662021\\
0.589448840589471	-2.44488977955912\\
0.585170340681363	-2.44335978778237\\
0.569138276553106	-2.43726879886481\\
0.55310621242485	-2.43083597158032\\
0.548430788945969	-2.42885771543086\\
0.537074148296593	-2.42409432489073\\
0.521042084168337	-2.41703015422659\\
0.511937632205333	-2.41282565130261\\
0.50501002004008	-2.40965263427974\\
0.488977955911824	-2.40197261032635\\
0.47861633688666	-2.39679358717435\\
0.472945891783567	-2.39398111250709\\
0.456913827655311	-2.38569774828604\\
0.447724946372745	-2.38076152304609\\
0.440881763527054	-2.37711175756346\\
0.424849699398798	-2.36823470705649\\
0.418742219050504	-2.36472945891784\\
0.408817635270541	-2.35907140958386\\
0.392785571142285	-2.34960756594041\\
0.391293580145418	-2.34869739478958\\
0.376753507014028	-2.33988194566119\\
0.365231523926469	-2.33266533066132\\
0.360721442885771	-2.3298571615125\\
0.344689378757515	-2.31956044169135\\
0.340265649230688	-2.31663326653307\\
0.328657314629258	-2.30899339723786\\
0.316276698443884	-2.30060120240481\\
0.312625250501002	-2.298138761015\\
0.296593186372745	-2.28701987652264\\
0.293153485537765	-2.28456913827655\\
0.280561122244489	-2.27563866976863\\
0.270812795449883	-2.2685370741483\\
0.264529058116232	-2.26397936875109\\
0.249116841161576	-2.25250501002004\\
0.248496993987976	-2.25204542916615\\
0.232464929859719	-2.23986734300685\\
0.228103566730433	-2.23647294589178\\
0.216432865731463	-2.22742240388654\\
0.20764169030169	-2.22044088176353\\
0.200400801603206	-2.21470970355198\\
0.187683817264239	-2.20440881763527\\
0.18436873747495	-2.20173183890034\\
0.168336673346693	-2.18849214306497\\
0.168199789975409	-2.18837675350701\\
0.152304609218437	-2.17501187444054\\
0.149198527430876	-2.17234468937876\\
0.13627254509018	-2.16127082520807\\
0.13060223360748	-2.1563126252505\\
0.120240480961924	-2.14727077441581\\
0.112387424107648	-2.14028056112224\\
0.104208416833667	-2.13301329770613\\
0.0945324789276524	-2.12424849699399\\
0.0881763527054105	-2.11849976935194\\
0.0770174803552489	-2.10821643286573\\
0.0721442885771539	-2.10373136403057\\
0.0598240675612114	-2.09218436873747\\
0.0561122244488974	-2.08870905833842\\
0.0429353059420955	-2.07615230460922\\
0.0400801603206409	-2.07343363204895\\
0.0263355695258069	-2.06012024048096\\
0.0240480961923843	-2.05790566911636\\
0.0100104349689827	-2.04408817635271\\
0.00801603206412782	-2.04212555842685\\
-0.0060534141382742	-2.02805611222445\\
-0.00801603206412826	-2.02609349429859\\
-0.0218682737844103	-2.01202404809619\\
-0.0240480961923848	-2.00980947673159\\
-0.0374454978510004	-1.99599198396794\\
-0.0400801603206413	-1.99327331140766\\
-0.0527955690962666	-1.97995991983968\\
-0.0561122244488979	-1.97648460944062\\
-0.067928163570394	-1.96392785571142\\
-0.0721442885771544	-1.95944278687626\\
-0.082852209135697	-1.94789579158317\\
-0.0881763527054109	-1.94214706394111\\
-0.0975759386864961	-1.93186372745491\\
-0.104208416833667	-1.9245964640388\\
-0.112106938595428	-1.91583166332665\\
-0.120240480961924	-1.90678981249196\\
-0.126452192853364	-1.8997995991984\\
-0.13627254509018	-1.8887257350277\\
-0.140618123317979	-1.88376753507014\\
-0.152304609218437	-1.87040265600366\\
-0.154610626440267	-1.86773547094188\\
-0.168336673346694	-1.85181879637159\\
-0.168435106798128	-1.85170340681363\\
-0.182094263918885	-1.83567134268537\\
-0.18436873747495	-1.83299436395044\\
-0.195597114389196	-1.81963927855711\\
-0.200400801603207	-1.81390810034557\\
-0.208948104277145	-1.80360721442886\\
-0.216432865731463	-1.79455667242361\\
-0.222151133760787	-1.7875751503006\\
-0.232464929859719	-1.77493748328741\\
-0.235209729534102	-1.77154308617234\\
-0.248127367485314	-1.75551102204409\\
-0.248496993987976	-1.7550514411902\\
-0.260910749817074	-1.73947895791583\\
-0.264529058116232	-1.73492125251862\\
-0.273561692861503	-1.72344689378758\\
-0.280561122244489	-1.71451642527965\\
-0.286082555369444	-1.70741482965932\\
-0.296593186372745	-1.69383350377715\\
-0.298475405129103	-1.69138276553106\\
-0.31074681341069	-1.67535070140281\\
-0.312625250501002	-1.672888260013\\
-0.322901385421872	-1.65931863727455\\
-0.328657314629258	-1.65167876797935\\
-0.334936639563524	-1.64328657314629\\
-0.344689378757515	-1.63018168417632\\
-0.346853715889066	-1.62725450901804\\
-0.358661223665631	-1.61122244488978\\
-0.360721442885771	-1.60841427574096\\
-0.370363623606131	-1.59519038076152\\
-0.376753507014028	-1.58637493163314\\
-0.381954431452668	-1.57915831663327\\
-0.392785571142285	-1.56403642365584\\
-0.393433997955205	-1.56312625250501\\
-0.404822061190901	-1.54709418837675\\
-0.408817635270541	-1.54143613904278\\
-0.416106673164823	-1.5310621242485\\
-0.424849699398798	-1.51853530825889\\
-0.427284817757179	-1.51503006012024\\
-0.438370195933025	-1.49899799599198\\
-0.440881763527054	-1.49534823050935\\
-0.44936577793113	-1.48296593186373\\
-0.456913827655311	-1.47187009297541\\
-0.460258527737727	-1.46693386773547\\
-0.47105948890644	-1.45090180360721\\
-0.472945891783567	-1.44808932893996\\
-0.481780563993476	-1.43486973947896\\
-0.488977955911824	-1.42401669850271\\
-0.492401403989197	-1.4188376753507\\
-0.502935212529043	-1.40280561122244\\
-0.50501002004008	-1.39963259419958\\
-0.513393889713682	-1.38677354709419\\
-0.521042084168337	-1.37494598588992\\
-0.523753976742535	-1.37074148296593\\
-0.534036970397823	-1.35470941883768\\
-0.537074148296593	-1.34994602829754\\
-0.54424314245709	-1.33867735470942\\
-0.55310621242485	-1.32462354673062\\
-0.554351490046198	-1.32264529058116\\
-0.564399151523312	-1.30661322645291\\
-0.569138276553106	-1.29899224575859\\
-0.57436066541884	-1.29058116232465\\
-0.584232230108001	-1.27454909819639\\
-0.585170340681363	-1.27301910641964\\
-0.594051326219487	-1.25851703406814\\
-0.601202404809619	-1.24672820700097\\
-0.603774129129969	-1.24248496993988\\
-0.613432791996307	-1.22645290581162\\
-0.617234468937876	-1.22009677958938\\
-0.62301858987926	-1.21042084168337\\
-0.632513854057083	-1.19438877755511\\
-0.633266533066132	-1.19311243712257\\
-0.641966158922827	-1.17835671342685\\
-0.649298597194389	-1.16579139866338\\
-0.651321859979967	-1.1623246492986\\
-0.660624815805227	-1.14629258517034\\
-0.665330661322646	-1.13811357789636\\
-0.669850938713261	-1.13026052104208\\
-0.679002064889275	-1.11422845691383\\
-0.681362725450902	-1.11006688618478\\
-0.688101307851878	-1.09819639278557\\
-0.69710495342446	-1.08216432865731\\
-0.697394789579158	-1.08164605316432\\
-0.706079738533738	-1.06613226452906\\
-0.713426853707415	-1.05285896278679\\
-0.714956057633846	-1.0501002004008\\
-0.723792570976108	-1.03406813627255\\
-0.729458917835671	-1.0236844437265\\
-0.732546518759912	-1.01803607214429\\
-0.741245729576007	-1.00200400801603\\
-0.745490981963928	-0.9941124267187\\
-0.749879009420214	-0.985971943887776\\
-0.75844473699409	-0.969939879759519\\
-0.761523046092184	-0.964134766588893\\
-0.766958806572806	-0.953907815631263\\
-0.775394727267324	-0.937875751503006\\
-0.777555110220441	-0.93374247901286\\
-0.783790805290207	-0.92184368737475\\
-0.792100457992489	-0.905811623246493\\
-0.793587174348697	-0.90292570551388\\
-0.800379530336358	-0.889779559118236\\
-0.808566321619409	-0.87374749498998\\
-0.809619238476954	-0.871673675404657\\
-0.816729146876445	-0.857715430861723\\
-0.824796355889865	-0.841683366733467\\
-0.82565130260521	-0.839974664477627\\
-0.832843470353537	-0.82565130260521\\
-0.840794253455332	-0.809619238476954\\
-0.841683366733467	-0.807815950229083\\
-0.848725975563253	-0.793587174348697\\
-0.856563370704004	-0.777555110220441\\
-0.857715430861723	-0.775183763382833\\
-0.864379804955127	-0.761523046092184\\
-0.872106735825038	-0.745490981963928\\
-0.87374749498998	-0.74206323545754\\
-0.879807776186825	-0.729458917835671\\
-0.887427056135495	-0.713426853707415\\
-0.889779559118236	-0.708438342098458\\
-0.895012388955043	-0.697394789579158\\
-0.902526724693145	-0.681362725450902\\
-0.905811623246493	-0.67429184186864\\
-0.909995831124627	-0.665330661322646\\
-0.917407826216054	-0.649298597194389\\
-0.92184368737475	-0.639605210166721\\
-0.924759984175294	-0.633266533066132\\
-0.932072142327726	-0.617234468937876\\
-0.937875751503006	-0.604358567907728\\
-0.939306427983218	-0.601202404809619\\
-0.946521156144503	-0.585170340681363\\
-0.953640079044779	-0.569138276553106\\
-0.953907815631263	-0.568531625962185\\
-0.960756056219921	-0.55310621242485\\
-0.967780754132098	-0.537074148296593\\
-0.969939879759519	-0.532105663628523\\
-0.97477773985877	-0.521042084168337\\
-0.981710376076964	-0.50501002004008\\
-0.985971943887776	-0.495049758631863\\
-0.988586815811709	-0.488977955911824\\
-0.99542946636084	-0.472945891783567\\
-1.00200400801603	-0.45733731060375\\
-1.00218360635947	-0.456913827655311\\
-1.00893826228546	-0.440881763527054\\
-1.01560260849885	-0.424849699398798\\
-1.01803607214429	-0.418942565866308\\
-1.02223671829137	-0.408817635270541\\
-1.02881559873643	-0.392785571142285\\
-1.03406813627255	-0.379829674451506\\
-1.03532450668279	-0.376753507014028\\
-1.04181969582964	-0.360721442885771\\
-1.04822789550428	-0.344689378757515\\
-1.0501002004008	-0.339963767795496\\
-1.05461421190746	-0.328657314629258\\
-1.06094089149501	-0.312625250501002\\
-1.06613226452906	-0.299306270210088\\
-1.06719817822434	-0.296593186372745\\
-1.07344490733363	-0.280561122244489\\
-1.07960754402498	-0.264529058116232\\
-1.08216432865731	-0.257811391612405\\
-1.08573861773676	-0.248496993987976\\
-1.09182313616956	-0.232464929859719\\
-1.09782580665023	-0.216432865731463\\
-1.09819639278557	-0.215435169716468\\
-1.10382818894802	-0.200400801603207\\
-1.10975436746823	-0.18436873747495\\
-1.11422845691383	-0.172116267907533\\
-1.11562073759942	-0.168336673346694\\
-1.12147167987081	-0.152304609218437\\
-1.12724301738029	-0.13627254509018\\
-1.13026052104208	-0.127797365544665\\
-1.13297542122262	-0.120240480961924\\
-1.13867283245662	-0.104208416833667\\
-1.14429246627308	-0.0881763527054109\\
-1.14629258517034	-0.0824110642442843\\
-1.14988753346395	-0.0721442885771544\\
-1.15543435708166	-0.0561122244488979\\
-1.16090507180902	-0.0400801603206413\\
-1.1623246492986	-0.0358778763363548\\
-1.16635911296884	-0.0240480961923848\\
-1.17175794302227	-0.00801603206412826\\
-1.17708217805388	0.00801603206412782\\
-1.17835671342685	0.0118935369522394\\
-1.1823911770971	0.0240480961923843\\
-1.1876442649766	0.0400801603206409\\
-1.19282412066439	0.0561122244488974\\
-1.19438877755511	0.0610080323132719\\
-1.19798372584872	0.0721442885771539\\
-1.20309298566784	0.0881763527054105\\
-1.20813022818325	0.104208416833667\\
-1.21042084168337	0.111585955694638\\
-1.21313574186391	0.120240480961924\\
-1.21810275454298	0.13627254509018\\
-1.22299881944837	0.152304609218437\\
-1.22645290581162	0.163765482258007\\
-1.22784518649722	0.168336673346693\\
-1.23267120256016	0.18436873747495\\
-1.23742719710257	0.200400801603206\\
-1.24211438380454	0.216432865731463\\
-1.24248496993988	0.217714857437629\\
-1.24679493303642	0.232464929859719\\
-1.25141163716841	0.248496993987976\\
-1.2559602494358	0.264529058116232\\
-1.25851703406814	0.273662714996768\\
-1.26046950850622	0.280561122244489\\
-1.26494737463194	0.296593186372745\\
-1.26935772516235	0.312625250501002\\
-1.27370159972485	0.328657314629258\\
-1.27454909819639	0.331825883481581\\
-1.2780285849571	0.344689378757515\\
-1.28230065775349	0.360721442885771\\
-1.28650662502303	0.376753507014028\\
-1.29058116232465	0.392528378245189\\
-1.29064836143032	0.392785571142285\\
-1.29478180847173	0.408817635270541\\
-1.29884938139001	0.424849699398798\\
-1.30285195075063	0.440881763527054\\
-1.30661322645291	0.456191316685715\\
-1.30679282479634	0.456913827655311\\
-1.31072117881957	0.472945891783567\\
-1.31458455512255	0.488977955911824\\
-1.31838372277035	0.50501002004008\\
-1.32211941541806	0.521042084168337\\
-1.32264529058116	0.523334349434413\\
-1.32583504765005	0.537074148296593\\
-1.32949353116982	0.55310621242485\\
-1.33308829272707	0.569138276553106\\
-1.33661996515266	0.585170340681363\\
-1.33867735470942	0.594673495785783\\
-1.34010803118963	0.601202404809619\\
-1.34355990970964	0.617234468937876\\
-1.34694824139766	0.633266533066132\\
-1.35027355767898	0.649298597194389\\
-1.35353635514788	0.665330661322646\\
-1.35470941883768	0.671204009319294\\
-1.35676341650596	0.681362725450902\\
-1.35994224867448	0.697394789579159\\
-1.36305782068158	0.713426853707415\\
-1.36611052617763	0.729458917835672\\
-1.36910072380099	0.745490981963928\\
-1.37074148296593	0.754474507343357\\
-1.37204493428342	0.761523046092185\\
-1.37494628488202	0.777555110220441\\
-1.37778409179572	0.793587174348698\\
-1.380558607853	0.809619238476954\\
-1.38327005023437	0.825651302605211\\
-1.38591860037884	0.841683366733467\\
-1.38677354709419	0.846986325458802\\
-1.38852514086549	0.857715430861724\\
-1.39107767773152	0.87374749498998\\
-1.39356590308185	0.889779559118236\\
-1.39598988725695	0.905811623246493\\
-1.39834966361338	0.921843687374749\\
-1.40064522826933	0.937875751503006\\
-1.40280561122244	0.953398857694158\\
-1.40287734648678	0.953907815631262\\
-1.40506851430965	0.969939879759519\\
-1.40719363867873	0.985971943887775\\
-1.40925259695101	1.00200400801603\\
-1.41124522734124	1.01803607214429\\
-1.41317132849114	1.03406813627254\\
-1.41503065900035	1.0501002004008\\
-1.4168229369183	1.06613226452906\\
-1.418547839196	1.08216432865731\\
-1.4188376753507	1.0849741854515\\
-1.42021886398182	1.09819639278557\\
-1.42182337322298	1.11422845691383\\
-1.42335795274132	1.13026052104208\\
-1.42482214355234	1.14629258517034\\
-1.42621544244241	1.1623246492986\\
-1.4275373012074	1.17835671342685\\
-1.42878712584424	1.19438877755511\\
-1.4299642756944	1.21042084168337\\
-1.43106806253739	1.22645290581162\\
-1.43209774963311	1.24248496993988\\
-1.43305255071117	1.25851703406814\\
-1.4339316289056	1.27454909819639\\
-1.43473409563298	1.29058116232465\\
-1.43486973947896	1.29359165594453\\
-1.435463735431	1.30661322645291\\
-1.43611501710031	1.32264529058116\\
-1.43668581376201	1.33867735470942\\
-1.43717505973574	1.35470941883768\\
-1.43758163205316	1.37074148296593\\
-1.43790434902837	1.38677354709419\\
-1.43814196875914	1.40280561122244\\
-1.43829318755633	1.4188376753507\\
-1.43835663829887	1.43486973947896\\
-1.43833088871135	1.45090180360721\\
-1.43821443956137	1.46693386773547\\
-1.43800572277337	1.48296593186373\\
-1.43770309945568	1.49899799599198\\
-1.43730485783734	1.51503006012024\\
-1.43680921111101	1.5310621242485\\
-1.43621429517806	1.54709418837675\\
-1.43551816629188	1.56312625250501\\
-1.43486973947896	1.57614782301339\\
}--cycle;


\addplot[area legend,solid,fill=mycolor6,draw=black,forget plot]
table[row sep=crcr] {%
x	y\\
-1.25851703406814	1.5366182780483\\
-1.2571939945978	1.54709418837675\\
-1.25502463969909	1.56312625250501\\
-1.25269977496859	1.57915831663327\\
-1.25021500021237	1.59519038076152\\
-1.24756573216555	1.61122244488978\\
-1.24474719693108	1.62725450901804\\
-1.24248496993988	1.63940453353308\\
-1.24174975232708	1.64328657314629\\
-1.23855607385912	1.65931863727455\\
-1.23517430122925	1.67535070140281\\
-1.23159857003181	1.69138276553106\\
-1.22782276802783	1.70741482965932\\
-1.22645290581162	1.71297294500933\\
-1.22381713524383	1.72344689378758\\
-1.21958196768543	1.73947895791583\\
-1.21512182688343	1.75551102204409\\
-1.21042914983683	1.77154308617235\\
-1.21042084168337	1.77157044071457\\
-1.20544003002271	1.7875751503006\\
-1.2001956701131	1.80360721442886\\
-1.19468712586423	1.81963927855711\\
-1.19438877755511	1.82047757060002\\
-1.18883032863662	1.83567134268537\\
-1.18267763612223	1.85170340681363\\
-1.17835671342685	1.86248676464928\\
-1.17618880789234	1.86773547094188\\
-1.16930974881892	1.88376753507014\\
-1.1623246492986	1.89929379840314\\
-1.16208939750908	1.8997995991984\\
-1.15438504264813	1.91583166332665\\
-1.14630279283667	1.93186372745491\\
-1.14629258517034	1.93188336299896\\
-1.13765350560881	1.94789579158317\\
-1.13026052104208	1.96099723421974\\
-1.12853963234662	1.96392785571142\\
-1.11879866336283	1.97995991983968\\
-1.11422845691383	1.98719867302287\\
-1.10842999965929	1.99599198396794\\
-1.09819639278557	2.01090995449779\\
-1.09739566362948	2.01202404809619\\
-1.08549980486119	2.02805611222445\\
-1.08216432865731	2.03239562534614\\
-1.07270745353508	2.04408817635271\\
-1.06613226452906	2.05194028088218\\
-1.05889760819504	2.06012024048096\\
-1.0501002004008	2.06974519316727\\
-1.04388952525939	2.07615230460922\\
-1.03406813627255	2.08597369359606\\
-1.02745181211216	2.09218436873747\\
-1.01803607214429	2.10076645757832\\
-1.00928481999218	2.10821643286573\\
-1.00200400801603	2.11424476645324\\
-0.988995122323532	2.12424849699399\\
-0.985971943887776	2.12651320994356\\
-0.969939879759519	2.1376506054179\\
-0.965832546162879	2.14028056112224\\
-0.953907815631263	2.14773689968831\\
-0.93884409198786	2.1563126252505\\
-0.937875751503006	2.15685176728282\\
-0.92184368737475	2.16503328233086\\
-0.905838977788721	2.17234468937876\\
-0.905811623246493	2.17235693884157\\
-0.889779559118236	2.1788492796017\\
-0.87374749498998	2.18456798436975\\
-0.861530437300388	2.18837675350701\\
-0.857715430861723	2.18954616042773\\
-0.841683366733467	2.19381497273679\\
-0.82565130260521	2.19741021755645\\
-0.809619238476954	2.20035935280452\\
-0.793587174348697	2.20268792893981\\
-0.777657159274506	2.20440881763527\\
-0.777555110220441	2.20441970542425\\
-0.761523046092184	2.20557727650906\\
-0.745490981963928	2.20618078314376\\
-0.729458917835671	2.20624879964812\\
-0.713426853707415	2.20579850771998\\
-0.697394789579158	2.20484577011933\\
-0.692563647657071	2.20440881763527\\
-0.681362725450902	2.20340688550526\\
-0.665330661322646	2.20149581598421\\
-0.649298597194389	2.19912452309507\\
-0.633266533066132	2.19630450520987\\
-0.617234468937876	2.1930462183839\\
-0.601202404809619	2.18935912257919\\
-0.597387398370955	2.18837675350701\\
-0.585170340681363	2.18526062119864\\
-0.569138276553106	2.18075511445476\\
-0.55310621242485	2.17584697356038\\
-0.542541012275099	2.17234468937876\\
-0.537074148296593	2.17054823307935\\
-0.521042084168337	2.16487272763267\\
-0.50501002004008	2.15881437208123\\
-0.498795492688808	2.1563126252505\\
-0.488977955911824	2.15239180075642\\
-0.472945891783567	2.14560656143752\\
-0.461021161251951	2.14028056112224\\
-0.456913827655311	2.13845970574222\\
-0.440881763527054	2.1309704822989\\
-0.427159101112417	2.12424849699399\\
-0.424849699398798	2.12312507020458\\
-0.408817635270541	2.11495108169751\\
-0.396163769225858	2.10821643286573\\
-0.392785571142285	2.10643000206709\\
-0.376753507014028	2.09758696454167\\
-0.367337767046153	2.09218436873747\\
-0.360721442885771	2.0884104077147\\
-0.344689378757515	2.07891068588303\\
-0.340200968789764	2.07615230460922\\
-0.328657314629258	2.06909626596799\\
-0.314476241216867	2.06012024048096\\
-0.312625250501002	2.05895464809079\\
-0.296593186372745	2.04851186060859\\
-0.290017997366719	2.04408817635271\\
-0.280561122244489	2.03775510652911\\
-0.266525006674562	2.02805611222445\\
-0.264529058116232	2.02668291954234\\
-0.248496993987976	2.01531578552215\\
-0.24398720621718	2.01202404809619\\
-0.232464929859719	2.00364619026176\\
-0.222231322986003	1.99599198396794\\
-0.216432865731463	1.99167061710524\\
-0.201144387012885	1.97995991983968\\
-0.200400801603207	1.97939224687704\\
-0.18436873747495	1.96682922522964\\
-0.180759277272084	1.96392785571142\\
-0.168336673346694	1.95397046507693\\
-0.160943688779969	1.94789579158317\\
-0.152304609218437	1.94081544756269\\
-0.141641876092624	1.93186372745491\\
-0.13627254509018	1.92736641042022\\
-0.122819082958197	1.91583166332665\\
-0.120240480961924	1.91362535945474\\
-0.104443668623181	1.8997995991984\\
-0.104208416833667	1.89959407113954\\
-0.0881763527054109	1.88528235039355\\
-0.0865148160541831	1.88376753507014\\
-0.0721442885771544	1.8706828177036\\
-0.0689729344898847	1.86773547094188\\
-0.0561122244488979	1.85579545394615\\
-0.0517913017535205	1.85170340681363\\
-0.0400801603206413	1.8406211431019\\
-0.0349491946272899	1.83567134268537\\
-0.0240480961923848	1.82516054716343\\
-0.0184276157516171	1.81963927855711\\
-0.00801603206412826	1.80941410698685\\
-0.00220913950613561	1.80360721442886\\
0.00801603206412782	1.79338204285859\\
0.0137222256779407	1.7875751503006\\
0.0240480961923843	1.77706435477866\\
0.0293811605494023	1.77154308617235\\
0.0400801603206409	1.76046082246061\\
0.044781145520701	1.75551102204409\\
0.0561122244488974	1.74357100504835\\
0.0599345598845288	1.73947895791583\\
0.0721442885771539	1.72639424054929\\
0.0748527710502992	1.72344689378758\\
0.0881763527054105	1.70892964498273\\
0.0895462149216145	1.70741482965932\\
0.104023002497253	1.69138276553106\\
0.104208416833667	1.6911772374722\\
0.118282065685979	1.67535070140281\\
0.120240480961924	1.67314439753089\\
0.132343649009419	1.65931863727455\\
0.13627254509018	1.65482132023986\\
0.146215470783112	1.64328657314629\\
0.152304609218437	1.63620622912581\\
0.159904586324255	1.62725450901804\\
0.168336673346693	1.61729711838354\\
0.173417435572365	1.61122244488978\\
0.18436873747495	1.59809175027974\\
0.186759886243764	1.59519038076152\\
0.199935144005117	1.57915831663327\\
0.200400801603206	1.57859064367063\\
0.212940471362413	1.56312625250501\\
0.216432865731463	1.55880488564231\\
0.225792837809639	1.54709418837675\\
0.232464929859719	1.53871633054232\\
0.238496370489154	1.5310621242485\\
0.248496993987976	1.5183217975462\\
0.251054773962973	1.51503006012024\\
0.263468599425322	1.49899799599198\\
0.264529058116232	1.49762480330988\\
0.275739145546932	1.48296593186373\\
0.280561122244489	1.47663286204013\\
0.287877223636193	1.46693386773547\\
0.296593186372745	1.4553254878631\\
0.299885216061356	1.45090180360721\\
0.311764107097419	1.43486973947896\\
0.312625250501002	1.43370414708879\\
0.323515162650359	1.4188376753507\\
0.328657314629258	1.41178163670947\\
0.335145841858041	1.40280561122244\\
0.344689378757515	1.389531928368\\
0.346657440132767	1.38677354709419\\
0.358051131200599	1.37074148296593\\
0.360721442885771	1.36696752194316\\
0.369331448980606	1.35470941883768\\
0.376753507014028	1.34407995051361\\
0.380499925436401	1.33867735470942\\
0.391557950811915	1.32264529058116\\
0.392785571142285	1.32085885978252\\
0.402510992260561	1.30661322645291\\
0.408817635270541	1.29731581115643\\
0.41335791184124	1.29058116232465\\
0.424099699060212	1.27454909819639\\
0.424849699398798	1.27342567140698\\
0.434745077414153	1.25851703406814\\
0.440881763527054	1.2492069552448\\
0.445288689743263	1.24248496993988\\
0.455732556726431	1.22645290581162\\
0.456913827655311	1.2246320504316\\
0.46608675354468	1.21042084168337\\
0.472945891783567	1.19971477787038\\
0.476342309784794	1.19438877755511\\
0.486505544007245	1.17835671342685\\
0.488977955911824	1.17443588893277\\
0.496582201926817	1.1623246492986\\
0.50501002004008	1.14879433200106\\
0.506562227335664	1.14629258517034\\
0.516461159868609	1.13026052104208\\
0.521042084168337	1.12278855929599\\
0.526271334030238	1.11422845691383\\
0.535990169562516	1.09819639278557\\
0.537074148296593	1.09639993648616\\
0.545635936496428	1.08216432865731\\
0.55310621242485	1.06963454871068\\
0.555188309789687	1.06613226452906\\
0.564665976417858	1.0501002004008\\
0.569138276553106	1.04247856134855\\
0.574060918999784	1.03406813627254\\
0.583370807925782	1.01803607214429\\
0.585170340681363	1.01491993983592\\
0.592612661348774	1.00200400801603\\
0.601202404809619	0.986954312959955\\
0.601762078998257	0.985971943887775\\
0.610851960394462	0.969939879759519\\
0.617234468937876	0.95857728050815\\
0.619853429263515	0.953907815631262\\
0.628786712597152	0.937875751503006\\
0.633266533066132	0.929771439077608\\
0.637643590364386	0.921843687374749\\
0.646424307191761	0.905811623246493\\
0.649298597194389	0.900527328706294\\
0.655139621623025	0.889779559118236\\
0.663771644745461	0.87374749498998\\
0.665330661322646	0.870834493338914\\
0.672348105225822	0.857715430861724\\
0.680835154463941	0.841683366733467\\
0.681362725450902	0.840681434603457\\
0.689275162647607	0.825651302605211\\
0.697394789579159	0.810056190961008\\
0.697622326369228	0.809619238476954\\
0.705926469923734	0.793587174348698\\
0.713426853707415	0.778944800305149\\
0.714139046579946	0.777555110220441\\
0.722307271821182	0.761523046092185\\
0.729458917835672	0.747330963976777\\
0.730386855299095	0.745490981963928\\
0.738422394955954	0.729458917835672\\
0.745490981963928	0.715198819215905\\
0.746370303804596	0.713426853707415\\
0.754276259900431	0.697394789579159\\
0.761523046092185	0.682531184324693\\
0.762093544238379	0.681362725450902\\
0.76987289232088	0.665330661322646\\
0.777555110220441	0.649309484983373\\
0.77756034005763	0.649298597194389\\
0.785215933182017	0.633266533066132\\
0.792780544847256	0.617234468937876\\
0.793587174348698	0.615513580242413\\
0.800308648052462	0.601202404809619\\
0.807753133342204	0.585170340681363\\
0.809619238476954	0.581120875850607\\
0.815153935541872	0.569138276553106\\
0.822480850118346	0.55310621242485\\
0.825651302605211	0.546107612346024\\
0.829754334897784	0.537074148296593\\
0.836966124067748	0.521042084168337\\
0.841683366733467	0.510448239269849\\
0.844112032787387	0.50501002004008\\
0.851211034475311	0.488977955911824\\
0.857715430861724	0.474115298704278\\
0.858228869286926	0.472945891783567\\
0.865217316805395	0.456913827655311\\
0.872120783966104	0.440881763527054\\
0.87374749498998	0.437072994389789\\
0.878986367722158	0.424849699398798\\
0.885781952461454	0.408817635270541\\
0.889779559118236	0.399290161365227\\
0.892519249360777	0.392785571142285\\
0.899208927971631	0.376753507014028\\
0.905811623246493	0.36073369234858\\
0.905816692864383	0.360721442885771\\
0.912402345178452	0.344689378757515\\
0.918906420569259	0.328657314629258\\
0.921843687374749	0.321345907581362\\
0.925362512507619	0.312625250501002\\
0.931764677049471	0.296593186372745\\
0.937875751503006	0.281100264276807\\
0.93808941405656	0.280561122244489\\
0.944391308699561	0.264529058116232\\
0.950614039736601	0.248496993987976\\
0.953907815631262	0.239921268425783\\
0.956785885963635	0.232464929859719\\
0.962910080477966	0.216432865731463\\
0.96895700668126	0.200400801603206\\
0.969939879759519	0.197770845898868\\
0.974974727817517	0.18436873747495\\
0.980924619413404	0.168336673346693\\
0.985971943887775	0.154569322168009\\
0.986806821734876	0.152304609218437\\
0.992660954792984	0.13627254509018\\
0.998439532095766	0.120240480961924\\
1.00200400801603	0.110236750421178\\
1.00416444469553	0.104208416833667\\
1.00984840578382	0.0881763527054105\\
1.01545825797897	0.0721442885771539\\
1.01803607214429	0.0646943132897401\\
1.02102360004871	0.0561122244488974\\
1.02653975825509	0.0400801603206409\\
1.03198308268065	0.0240480961923843\\
1.03406813627254	0.0178374210509739\\
1.03738620316287	0.00801603206412782\\
1.04273653858697	-0.00801603206412826\\
1.04801514680891	-0.0240480961923848\\
1.0501002004008	-0.0304552076343362\\
1.05325301903754	-0.0400801603206413\\
1.05843912796602	-0.0561122244488979\\
1.06355445036374	-0.0721442885771544\\
1.06613226452906	-0.0803242481759373\\
1.06862366525489	-0.0881763527054109\\
1.07364676385577	-0.104208416833667\\
1.07859985273705	-0.120240480961924\\
1.08216432865731	-0.131933031968492\\
1.08349661050439	-0.13627254509018\\
1.08835753705469	-0.152304609218437\\
1.0931490683112	-0.168336673346694\\
1.09787225175611	-0.18436873747495\\
1.09819639278557	-0.18548283107335\\
1.10256838432536	-0.200400801603207\\
1.10719865763227	-0.216432865731463\\
1.1117609582029	-0.232464929859719\\
1.11422845691383	-0.241258240804784\\
1.11627507655532	-0.248496993987976\\
1.12074401411038	-0.264529058116232\\
1.12514519320558	-0.280561122244489\\
1.12947946693003	-0.296593186372745\\
1.13026052104208	-0.29952380786443\\
1.13377934617495	-0.312625250501002\\
1.13801878451301	-0.328657314629258\\
1.14219129312577	-0.344689378757515\\
1.14629258517034	-0.360701807341723\\
1.14629765478823	-0.360721442885771\\
1.15037434760639	-0.376753507014028\\
1.15438392376633	-0.392785571142285\\
1.15832704264181	-0.408817635270541\\
1.162204325077	-0.424849699398798\\
1.1623246492986	-0.425355500194049\\
1.16604833959957	-0.440881763527054\\
1.16982653524227	-0.456913827655311\\
1.17353838602857	-0.472945891783567\\
1.17718439541235	-0.488977955911824\\
1.17835671342685	-0.494226662204424\\
1.18078537948077	-0.50501002004008\\
1.18432949877743	-0.521042084168337\\
1.1878070192803	-0.537074148296593\\
1.19121832506825	-0.55310621242485\\
1.19438877755511	-0.568299984510198\\
1.19456520206247	-0.569138276553106\\
1.19787134074402	-0.585170340681363\\
1.20111025125887	-0.601202404809619\\
1.20428219543686	-0.617234468937876\\
1.20738739493193	-0.633266533066132\\
1.21042084168337	-0.649271242652159\\
1.21042607152056	-0.649298597194389\\
1.21342083185755	-0.665330661322646\\
1.21634748309878	-0.681362725450902\\
1.21920611961987	-0.697394789579158\\
1.22199679407919	-0.713426853707415\\
1.2247195170619	-0.729458917835671\\
1.22645290581162	-0.739932866613912\\
1.22738084327505	-0.745490981963928\\
1.22998524429604	-0.761523046092184\\
1.23251986702919	-0.777555110220441\\
1.2349845861562	-0.793587174348697\\
1.23737923194097	-0.809619238476954\\
1.23970358965474	-0.82565130260521\\
1.24195739895292	-0.841683366733467\\
1.24248496993988	-0.845565406346682\\
1.24415072011282	-0.857715430861723\\
1.24627484657362	-0.87374749498998\\
1.24832599436852	-0.889779559118236\\
1.25030375289917	-0.905811623246493\\
1.25220766250782	-0.92184368737475\\
1.25403721359916	-0.937875751503006\\
1.25579184570513	-0.953907815631263\\
1.25747094649076	-0.969939879759519\\
1.25851703406814	-0.980415790087977\\
1.25907670825677	-0.985971943887776\\
1.26061005025889	-1.00200400801603\\
1.26206463497995	-1.01803607214429\\
1.26343967651481	-1.03406813627255\\
1.2647343321079	-1.0501002004008\\
1.26594770084412	-1.06613226452906\\
1.26707882226797	-1.08216432865731\\
1.26812667492813	-1.09819639278557\\
1.2690901748448	-1.11422845691383\\
1.26996817389666	-1.13026052104208\\
1.2707594581245	-1.14629258517034\\
1.2714627459479	-1.1623246492986\\
1.27207668629181	-1.17835671342685\\
1.27259985661893	-1.19438877755511\\
1.27303076086424	-1.21042084168337\\
1.27336782726751	-1.22645290581162\\
1.27360940609925	-1.24248496993988\\
1.27375376727568	-1.25851703406814\\
1.27379909785781	-1.27454909819639\\
1.27374349942935	-1.29058116232465\\
1.27358498534827	-1.30661322645291\\
1.27332147786602	-1.32264529058116\\
1.27295080510859	-1.33867735470942\\
1.27247069791279	-1.35470941883768\\
1.27187878651122	-1.37074148296593\\
1.27117259705861	-1.38677354709419\\
1.27034954799212	-1.40280561122244\\
1.26940694621749	-1.4188376753507\\
1.2683419831128	-1.43486973947896\\
1.2671517303407	-1.45090180360721\\
1.26583313545984	-1.46693386773547\\
1.26438301732549	-1.48296593186373\\
1.26279806126875	-1.49899799599198\\
1.26107481404313	-1.51503006012024\\
1.25920967852684	-1.5310621242485\\
1.25851703406814	-1.5366182780483\\
1.2571939945978	-1.54709418837675\\
1.25502463969909	-1.56312625250501\\
1.25269977496859	-1.57915831663327\\
1.25021500021237	-1.59519038076152\\
1.24756573216555	-1.61122244488978\\
1.24474719693108	-1.62725450901804\\
1.24248496993988	-1.63940453353308\\
1.24174975232708	-1.64328657314629\\
1.23855607385912	-1.65931863727455\\
1.23517430122925	-1.67535070140281\\
1.23159857003181	-1.69138276553106\\
1.22782276802783	-1.70741482965932\\
1.22645290581162	-1.71297294500933\\
1.22381713524383	-1.72344689378758\\
1.21958196768543	-1.73947895791583\\
1.21512182688343	-1.75551102204409\\
1.21042914983683	-1.77154308617234\\
1.21042084168337	-1.77157044071457\\
1.20544003002271	-1.7875751503006\\
1.2001956701131	-1.80360721442886\\
1.19468712586423	-1.81963927855711\\
1.19438877755511	-1.82047757060002\\
1.18883032863661	-1.83567134268537\\
1.18267763612223	-1.85170340681363\\
1.17835671342685	-1.86248676464928\\
1.17618880789234	-1.86773547094188\\
1.16930974881892	-1.88376753507014\\
1.1623246492986	-1.89929379840315\\
1.16208939750908	-1.8997995991984\\
1.15438504264813	-1.91583166332665\\
1.14630279283667	-1.93186372745491\\
1.14629258517034	-1.93188336299896\\
1.13765350560881	-1.94789579158317\\
1.13026052104208	-1.96099723421974\\
1.12853963234662	-1.96392785571142\\
1.11879866336283	-1.97995991983968\\
1.11422845691383	-1.98719867302287\\
1.10842999965929	-1.99599198396794\\
1.09819639278557	-2.01090995449779\\
1.09739566362948	-2.01202404809619\\
1.08549980486119	-2.02805611222445\\
1.08216432865731	-2.03239562534614\\
1.07270745353508	-2.04408817635271\\
1.06613226452906	-2.05194028088218\\
1.05889760819504	-2.06012024048096\\
1.0501002004008	-2.06974519316727\\
1.04388952525939	-2.07615230460922\\
1.03406813627254	-2.08597369359606\\
1.02745181211216	-2.09218436873747\\
1.01803607214429	-2.10076645757832\\
1.00928481999218	-2.10821643286573\\
1.00200400801603	-2.11424476645324\\
0.988995122323532	-2.12424849699399\\
0.985971943887775	-2.12651320994356\\
0.969939879759519	-2.1376506054179\\
0.965832546162879	-2.14028056112224\\
0.953907815631262	-2.14773689968831\\
0.938844091987861	-2.1563126252505\\
0.937875751503006	-2.15685176728282\\
0.921843687374749	-2.16503328233086\\
0.905838977788721	-2.17234468937876\\
0.905811623246493	-2.17235693884156\\
0.889779559118236	-2.1788492796017\\
0.87374749498998	-2.18456798436975\\
0.861530437300388	-2.18837675350701\\
0.857715430861724	-2.18954616042773\\
0.841683366733467	-2.19381497273678\\
0.825651302605211	-2.19741021755645\\
0.809619238476954	-2.20035935280452\\
0.793587174348698	-2.20268792893981\\
0.777657159274506	-2.20440881763527\\
0.777555110220441	-2.20441970542426\\
0.761523046092185	-2.20557727650906\\
0.745490981963928	-2.20618078314376\\
0.729458917835672	-2.20624879964812\\
0.713426853707415	-2.20579850771998\\
0.697394789579159	-2.20484577011933\\
0.692563647657066	-2.20440881763527\\
0.681362725450902	-2.20340688550526\\
0.665330661322646	-2.20149581598421\\
0.649298597194389	-2.19912452309507\\
0.633266533066132	-2.19630450520987\\
0.617234468937876	-2.1930462183839\\
0.601202404809619	-2.18935912257919\\
0.597387398370955	-2.18837675350701\\
0.585170340681363	-2.18526062119864\\
0.569138276553106	-2.18075511445476\\
0.55310621242485	-2.17584697356038\\
0.542541012275099	-2.17234468937876\\
0.537074148296593	-2.17054823307935\\
0.521042084168337	-2.16487272763267\\
0.50501002004008	-2.15881437208122\\
0.498795492688808	-2.1563126252505\\
0.488977955911824	-2.15239180075642\\
0.472945891783567	-2.14560656143752\\
0.46102116125195	-2.14028056112224\\
0.456913827655311	-2.13845970574222\\
0.440881763527054	-2.1309704822989\\
0.427159101112417	-2.12424849699399\\
0.424849699398798	-2.12312507020458\\
0.408817635270541	-2.11495108169751\\
0.396163769225858	-2.10821643286573\\
0.392785571142285	-2.10643000206709\\
0.376753507014028	-2.09758696454167\\
0.367337767046153	-2.09218436873747\\
0.360721442885771	-2.0884104077147\\
0.344689378757515	-2.07891068588303\\
0.340200968789764	-2.07615230460922\\
0.328657314629258	-2.06909626596799\\
0.314476241216868	-2.06012024048096\\
0.312625250501002	-2.05895464809079\\
0.296593186372745	-2.04851186060859\\
0.29001799736672	-2.04408817635271\\
0.280561122244489	-2.03775510652911\\
0.266525006674562	-2.02805611222445\\
0.264529058116232	-2.02668291954234\\
0.248496993987976	-2.01531578552215\\
0.243987206217179	-2.01202404809619\\
0.232464929859719	-2.00364619026176\\
0.222231322986004	-1.99599198396794\\
0.216432865731463	-1.99167061710524\\
0.201144387012887	-1.97995991983968\\
0.200400801603206	-1.97939224687704\\
0.18436873747495	-1.96682922522964\\
0.180759277272084	-1.96392785571142\\
0.168336673346693	-1.95397046507693\\
0.160943688779969	-1.94789579158317\\
0.152304609218437	-1.94081544756269\\
0.141641876092625	-1.93186372745491\\
0.13627254509018	-1.92736641042022\\
0.122819082958197	-1.91583166332665\\
0.120240480961924	-1.91362535945474\\
0.104443668623181	-1.8997995991984\\
0.104208416833667	-1.89959407113954\\
0.0881763527054105	-1.88528235039355\\
0.0865148160541835	-1.88376753507014\\
0.0721442885771539	-1.8706828177036\\
0.0689729344898856	-1.86773547094188\\
0.0561122244488974	-1.85579545394615\\
0.051791301753521	-1.85170340681363\\
0.0400801603206409	-1.8406211431019\\
0.0349491946272904	-1.83567134268537\\
0.0240480961923843	-1.82516054716343\\
0.0184276157516176	-1.81963927855711\\
0.00801603206412782	-1.80941410698685\\
0.00220913950613607	-1.80360721442886\\
-0.00801603206412826	-1.79338204285859\\
-0.0137222256779403	-1.7875751503006\\
-0.0240480961923848	-1.77706435477866\\
-0.0293811605494027	-1.77154308617234\\
-0.0400801603206413	-1.76046082246061\\
-0.0447811455207014	-1.75551102204409\\
-0.0561122244488979	-1.74357100504835\\
-0.0599345598845293	-1.73947895791583\\
-0.0721442885771544	-1.72639424054929\\
-0.0748527710502996	-1.72344689378758\\
-0.0881763527054109	-1.70892964498273\\
-0.089546214921614	-1.70741482965932\\
-0.104023002497253	-1.69138276553106\\
-0.104208416833667	-1.6911772374722\\
-0.11828206568598	-1.67535070140281\\
-0.120240480961924	-1.67314439753089\\
-0.13234364900942	-1.65931863727455\\
-0.13627254509018	-1.65482132023986\\
-0.146215470783112	-1.64328657314629\\
-0.152304609218437	-1.63620622912581\\
-0.159904586324256	-1.62725450901804\\
-0.168336673346694	-1.61729711838354\\
-0.173417435572365	-1.61122244488978\\
-0.18436873747495	-1.59809175027974\\
-0.186759886243765	-1.59519038076152\\
-0.199935144005117	-1.57915831663327\\
-0.200400801603207	-1.57859064367062\\
-0.212940471362413	-1.56312625250501\\
-0.216432865731463	-1.55880488564231\\
-0.225792837809638	-1.54709418837675\\
-0.232464929859719	-1.53871633054232\\
-0.238496370489154	-1.5310621242485\\
-0.248496993987976	-1.5183217975462\\
-0.251054773962973	-1.51503006012024\\
-0.263468599425322	-1.49899799599198\\
-0.264529058116232	-1.49762480330988\\
-0.275739145546932	-1.48296593186373\\
-0.280561122244489	-1.47663286204013\\
-0.287877223636193	-1.46693386773547\\
-0.296593186372745	-1.4553254878631\\
-0.299885216061356	-1.45090180360721\\
-0.311764107097419	-1.43486973947896\\
-0.312625250501002	-1.43370414708879\\
-0.323515162650359	-1.4188376753507\\
-0.328657314629258	-1.41178163670947\\
-0.335145841858041	-1.40280561122244\\
-0.344689378757515	-1.389531928368\\
-0.346657440132767	-1.38677354709419\\
-0.358051131200599	-1.37074148296593\\
-0.360721442885771	-1.36696752194316\\
-0.369331448980606	-1.35470941883768\\
-0.376753507014028	-1.34407995051361\\
-0.380499925436401	-1.33867735470942\\
-0.391557950811915	-1.32264529058116\\
-0.392785571142285	-1.32085885978252\\
-0.402510992260561	-1.30661322645291\\
-0.408817635270541	-1.29731581115643\\
-0.41335791184124	-1.29058116232465\\
-0.424099699060212	-1.27454909819639\\
-0.424849699398798	-1.27342567140698\\
-0.434745077414153	-1.25851703406814\\
-0.440881763527054	-1.2492069552448\\
-0.445288689743263	-1.24248496993988\\
-0.455732556726431	-1.22645290581162\\
-0.456913827655311	-1.2246320504316\\
-0.466086753544681	-1.21042084168337\\
-0.472945891783567	-1.19971477787038\\
-0.476342309784793	-1.19438877755511\\
-0.486505544007245	-1.17835671342685\\
-0.488977955911824	-1.17443588893277\\
-0.496582201926817	-1.1623246492986\\
-0.50501002004008	-1.14879433200106\\
-0.506562227335664	-1.14629258517034\\
-0.516461159868609	-1.13026052104208\\
-0.521042084168337	-1.12278855929599\\
-0.526271334030238	-1.11422845691383\\
-0.535990169562516	-1.09819639278557\\
-0.537074148296593	-1.09639993648616\\
-0.545635936496428	-1.08216432865731\\
-0.55310621242485	-1.06963454871068\\
-0.555188309789687	-1.06613226452906\\
-0.564665976417857	-1.0501002004008\\
-0.569138276553106	-1.04247856134855\\
-0.574060918999784	-1.03406813627255\\
-0.583370807925782	-1.01803607214429\\
-0.585170340681363	-1.01491993983592\\
-0.592612661348773	-1.00200400801603\\
-0.601202404809619	-0.986954312959955\\
-0.601762078998257	-0.985971943887776\\
-0.610851960394462	-0.969939879759519\\
-0.617234468937876	-0.95857728050815\\
-0.619853429263515	-0.953907815631263\\
-0.628786712597152	-0.937875751503006\\
-0.633266533066132	-0.929771439077608\\
-0.637643590364386	-0.92184368737475\\
-0.646424307191761	-0.905811623246493\\
-0.649298597194389	-0.900527328706294\\
-0.655139621623025	-0.889779559118236\\
-0.663771644745462	-0.87374749498998\\
-0.665330661322646	-0.870834493338914\\
-0.672348105225823	-0.857715430861723\\
-0.680835154463942	-0.841683366733467\\
-0.681362725450902	-0.840681434603458\\
-0.689275162647607	-0.82565130260521\\
-0.697394789579158	-0.81005619096101\\
-0.697622326369228	-0.809619238476954\\
-0.705926469923734	-0.793587174348697\\
-0.713426853707415	-0.77894480030515\\
-0.714139046579946	-0.777555110220441\\
-0.722307271821182	-0.761523046092184\\
-0.729458917835671	-0.747330963976778\\
-0.730386855299095	-0.745490981963928\\
-0.738422394955954	-0.729458917835671\\
-0.745490981963928	-0.715198819215906\\
-0.746370303804596	-0.713426853707415\\
-0.754276259900432	-0.697394789579158\\
-0.761523046092184	-0.682531184324695\\
-0.76209354423838	-0.681362725450902\\
-0.76987289232088	-0.665330661322646\\
-0.777555110220441	-0.649309484983374\\
-0.77756034005763	-0.649298597194389\\
-0.785215933182017	-0.633266533066132\\
-0.792780544847256	-0.617234468937876\\
-0.793587174348697	-0.615513580242414\\
-0.800308648052461	-0.601202404809619\\
-0.807753133342203	-0.585170340681363\\
-0.809619238476954	-0.581120875850608\\
-0.815153935541872	-0.569138276553106\\
-0.822480850118345	-0.55310621242485\\
-0.82565130260521	-0.546107612346026\\
-0.829754334897784	-0.537074148296593\\
-0.836966124067748	-0.521042084168337\\
-0.841683366733467	-0.510448239269851\\
-0.844112032787387	-0.50501002004008\\
-0.851211034475311	-0.488977955911824\\
-0.857715430861723	-0.474115298704279\\
-0.858228869286925	-0.472945891783567\\
-0.865217316805395	-0.456913827655311\\
-0.872120783966104	-0.440881763527054\\
-0.87374749498998	-0.437072994389791\\
-0.878986367722158	-0.424849699398798\\
-0.885781952461454	-0.408817635270541\\
-0.889779559118236	-0.399290161365226\\
-0.892519249360777	-0.392785571142285\\
-0.899208927971631	-0.376753507014028\\
-0.905811623246493	-0.360733692348578\\
-0.905816692864383	-0.360721442885771\\
-0.912402345178453	-0.344689378757515\\
-0.918906420569258	-0.328657314629258\\
-0.92184368737475	-0.321345907581361\\
-0.925362512507619	-0.312625250501002\\
-0.93176467704947	-0.296593186372745\\
-0.937875751503006	-0.281100264276806\\
-0.938089414056559	-0.280561122244489\\
-0.944391308699561	-0.264529058116232\\
-0.950614039736601	-0.248496993987976\\
-0.953907815631263	-0.239921268425782\\
-0.956785885963635	-0.232464929859719\\
-0.962910080477966	-0.216432865731463\\
-0.968957006681261	-0.200400801603207\\
-0.969939879759519	-0.197770845898866\\
-0.974974727817517	-0.18436873747495\\
-0.980924619413404	-0.168336673346694\\
-0.985971943887776	-0.154569322168008\\
-0.986806821734876	-0.152304609218437\\
-0.992660954792983	-0.13627254509018\\
-0.998439532095766	-0.120240480961924\\
-1.00200400801603	-0.110236750421177\\
-1.00416444469553	-0.104208416833667\\
-1.00984840578382	-0.0881763527054109\\
-1.01545825797897	-0.0721442885771544\\
-1.01803607214429	-0.0646943132897392\\
-1.02102360004871	-0.0561122244488979\\
-1.02653975825509	-0.0400801603206413\\
-1.03198308268065	-0.0240480961923848\\
-1.03406813627255	-0.0178374210509725\\
-1.03738620316287	-0.00801603206412826\\
-1.04273653858697	0.00801603206412782\\
-1.04801514680891	0.0240480961923843\\
-1.0501002004008	0.0304552076343385\\
-1.05325301903754	0.0400801603206409\\
-1.05843912796601	0.0561122244488974\\
-1.06355445036374	0.0721442885771539\\
-1.06613226452906	0.0803242481759391\\
-1.06862366525489	0.0881763527054105\\
-1.07364676385577	0.104208416833667\\
-1.07859985273705	0.120240480961924\\
-1.08216432865731	0.131933031968493\\
-1.08349661050439	0.13627254509018\\
-1.08835753705469	0.152304609218437\\
-1.0931490683112	0.168336673346693\\
-1.09787225175611	0.18436873747495\\
-1.09819639278557	0.185482831073351\\
-1.10256838432536	0.200400801603206\\
-1.10719865763227	0.216432865731463\\
-1.1117609582029	0.232464929859719\\
-1.11422845691383	0.241258240804785\\
-1.11627507655532	0.248496993987976\\
-1.12074401411038	0.264529058116232\\
-1.12514519320558	0.280561122244489\\
-1.12947946693003	0.296593186372745\\
-1.13026052104208	0.299523807864431\\
-1.13377934617495	0.312625250501002\\
-1.13801878451301	0.328657314629258\\
-1.14219129312577	0.344689378757515\\
-1.14629258517034	0.360701807341723\\
-1.14629765478823	0.360721442885771\\
-1.15037434760639	0.376753507014028\\
-1.15438392376633	0.392785571142285\\
-1.15832704264181	0.408817635270541\\
-1.162204325077	0.424849699398798\\
-1.1623246492986	0.425355500194049\\
-1.16604833959957	0.440881763527054\\
-1.16982653524227	0.456913827655311\\
-1.17353838602857	0.472945891783567\\
-1.17718439541235	0.488977955911824\\
-1.17835671342685	0.494226662204425\\
-1.18078537948077	0.50501002004008\\
-1.18432949877743	0.521042084168337\\
-1.1878070192803	0.537074148296593\\
-1.19121832506825	0.55310621242485\\
-1.19438877755511	0.568299984510198\\
-1.19456520206247	0.569138276553106\\
-1.19787134074402	0.585170340681363\\
-1.20111025125887	0.601202404809619\\
-1.20428219543685	0.617234468937876\\
-1.20738739493193	0.633266533066132\\
-1.21042084168337	0.649271242652159\\
-1.21042607152056	0.649298597194389\\
-1.21342083185755	0.665330661322646\\
-1.21634748309878	0.681362725450902\\
-1.21920611961987	0.697394789579159\\
-1.22199679407919	0.713426853707415\\
-1.2247195170619	0.729458917835672\\
-1.22645290581162	0.739932866613911\\
-1.22738084327505	0.745490981963928\\
-1.22998524429604	0.761523046092185\\
-1.23251986702919	0.777555110220441\\
-1.2349845861562	0.793587174348698\\
-1.23737923194097	0.809619238476954\\
-1.23970358965474	0.825651302605211\\
-1.24195739895292	0.841683366733467\\
-1.24248496993988	0.845565406346681\\
-1.24415072011282	0.857715430861724\\
-1.24627484657362	0.87374749498998\\
-1.24832599436852	0.889779559118236\\
-1.25030375289917	0.905811623246493\\
-1.25220766250782	0.921843687374749\\
-1.25403721359916	0.937875751503006\\
-1.25579184570513	0.953907815631262\\
-1.25747094649076	0.969939879759519\\
-1.25851703406814	0.980415790087977\\
-1.25907670825677	0.985971943887775\\
-1.26061005025888	1.00200400801603\\
-1.26206463497995	1.01803607214429\\
-1.26343967651481	1.03406813627254\\
-1.2647343321079	1.0501002004008\\
-1.26594770084412	1.06613226452906\\
-1.26707882226797	1.08216432865731\\
-1.26812667492813	1.09819639278557\\
-1.2690901748448	1.11422845691383\\
-1.26996817389666	1.13026052104208\\
-1.2707594581245	1.14629258517034\\
-1.2714627459479	1.1623246492986\\
-1.27207668629181	1.17835671342685\\
-1.27259985661893	1.19438877755511\\
-1.27303076086424	1.21042084168337\\
-1.27336782726751	1.22645290581162\\
-1.27360940609925	1.24248496993988\\
-1.27375376727568	1.25851703406814\\
-1.27379909785781	1.27454909819639\\
-1.27374349942935	1.29058116232465\\
-1.27358498534827	1.30661322645291\\
-1.27332147786602	1.32264529058116\\
-1.27295080510859	1.33867735470942\\
-1.27247069791279	1.35470941883768\\
-1.27187878651122	1.37074148296593\\
-1.27117259705861	1.38677354709419\\
-1.27034954799212	1.40280561122244\\
-1.26940694621749	1.4188376753507\\
-1.2683419831128	1.43486973947896\\
-1.2671517303407	1.45090180360721\\
-1.26583313545984	1.46693386773547\\
-1.26438301732549	1.48296593186373\\
-1.26279806126875	1.49899799599198\\
-1.26107481404313	1.51503006012024\\
-1.25920967852684	1.5310621242485\\
-1.25851703406814	1.5366182780483\\
}--cycle;


\addplot[area legend,solid,fill=mycolor7,draw=black,forget plot]
table[row sep=crcr] {%
x	y\\
-1.11422845691383	1.29527490685972\\
-1.11319708140095	1.30661322645291\\
-1.11158897321772	1.32264529058116\\
-1.1098226773115	1.33867735470942\\
-1.10789361042715	1.35470941883768\\
-1.10579698497173	1.37074148296593\\
-1.10352780003294	1.38677354709419\\
-1.1010808319104	1.40280561122244\\
-1.0984506241303	1.4188376753507\\
-1.09819639278557	1.42029804511158\\
-1.09560322010549	1.43486973947896\\
-1.09255398071092	1.45090180360721\\
-1.08929907735243	1.46693386773547\\
-1.08583167329455	1.48296593186373\\
-1.08216432865731	1.49891333546637\\
-1.0821443520793	1.49899799599198\\
-1.07817537844418	1.51503006012024\\
-1.07396500034531	1.5310621242485\\
-1.06950457834855	1.54709418837675\\
-1.06613226452906	1.55859377815824\\
-1.0647629193205	1.56312625250501\\
-1.05969013932373	1.57915831663327\\
-1.05432972957213	1.59519038076152\\
-1.0501002004008	1.60721650529931\\
-1.04864334990002	1.61122244488978\\
-1.04255657480825	1.62725450901804\\
-1.0361352655747	1.64328657314629\\
-1.03406813627255	1.64823587357896\\
-1.0292623072434	1.65931863727455\\
-1.02196731801052	1.67535070140281\\
-1.01803607214429	1.68361504268286\\
-1.01418588617855	1.69138276553106\\
-1.00587990953774	1.70741482965932\\
-1.00200400801603	1.71458298935537\\
-0.996990227098053	1.72344689378758\\
-0.987502407962091	1.73947895791583\\
-0.985971943887776	1.74197122833202\\
-0.977236362534941	1.75551102204409\\
-0.969939879759519	1.7663285202369\\
-0.966230684303345	1.77154308617235\\
-0.954346209377059	1.7875751503006\\
-0.953907815631263	1.78814656274547\\
-0.94133327615024	1.80360721442886\\
-0.937875751503006	1.8076965221177\\
-0.927121552749067	1.81963927855711\\
-0.92184368737475	1.82528934752995\\
-0.911461692219334	1.83567134268537\\
-0.905811623246493	1.84112894737434\\
-0.89402410506561	1.85170340681363\\
-0.889779559118236	1.85538858208849\\
-0.874365875445441	1.86773547094188\\
-0.87374749498998	1.86821576553839\\
-0.857715430861723	1.87970629675285\\
-0.851492446151481	1.88376753507014\\
-0.841683366733467	1.8899920287837\\
-0.82565130260521	1.89916697663983\\
-0.824429387048633	1.8997995991984\\
-0.809619238476954	1.90727502109806\\
-0.793587174348697	1.91442224207739\\
-0.790015546419613	1.91583166332665\\
-0.777555110220441	1.92063769330967\\
-0.761523046092184	1.92599125684182\\
-0.745490981963928	1.93053135242795\\
-0.739874814442953	1.93186372745491\\
-0.729458917835671	1.93428701368306\\
-0.713426853707415	1.9373019131192\\
-0.697394789579158	1.9396132115098\\
-0.681362725450902	1.94125040392693\\
-0.665330661322646	1.94224079591663\\
-0.649298597194389	1.94260964305413\\
-0.633266533066132	1.94238027890235\\
-0.617234468937876	1.94157423236356\\
-0.601202404809619	1.94021133531567\\
-0.585170340681363	1.93830982133684\\
-0.569138276553106	1.93588641624351\\
-0.55310621242485	1.93295642109648\\
-0.54801938157678	1.93186372745491\\
-0.537074148296593	1.92953305696762\\
-0.521042084168337	1.92563085023898\\
-0.50501002004008	1.92126222414596\\
-0.488977955911824	1.91643779725655\\
-0.487146074453037	1.91583166332665\\
-0.472945891783567	1.91116918961357\\
-0.456913827655311	1.9054657821794\\
-0.442103679083631	1.8997995991984\\
-0.440881763527054	1.89933546148151\\
-0.424849699398798	1.89279061863763\\
-0.408817635270541	1.8858346643723\\
-0.404332999824192	1.88376753507014\\
-0.392785571142285	1.8784790028984\\
-0.376753507014028	1.87072758265841\\
-0.370881128353451	1.86773547094188\\
-0.360721442885771	1.86258938978954\\
-0.344689378757515	1.85406785039725\\
-0.340444832810142	1.85170340681363\\
-0.328657314629258	1.84517246283947\\
-0.312625250501002	1.83590190159421\\
-0.312243254575273	1.83567134268537\\
-0.296593186372745	1.82627154452174\\
-0.285972676222855	1.81963927855711\\
-0.280561122244489	1.81627548532647\\
-0.264529058116232	1.80592295097609\\
-0.261071533468998	1.80360721442886\\
-0.248496993987976	1.795219764379\\
-0.237425008585017	1.7875751503006\\
-0.232464929859719	1.78416365914826\\
-0.216432865731463	1.77276117036572\\
-0.21477460953327	1.77154308617235\\
-0.200400801603207	1.76101992105668\\
-0.193104318827785	1.75551102204409\\
-0.18436873747495	1.7489359039608\\
-0.172173345241227	1.73947895791583\\
-0.168336673346694	1.73651220130204\\
-0.152304609218437	1.72375235381353\\
-0.151931278795002	1.72344689378758\\
-0.13627254509018	1.71066447591853\\
-0.132396643568474	1.70741482965932\\
-0.120240480961924	1.69724375478096\\
-0.113414571508778	1.69138276553106\\
-0.104208416833667	1.68349221526087\\
-0.0949445761708533	1.67535070140281\\
-0.0881763527054109	1.66941162343815\\
-0.0769501176063007	1.65931863727455\\
-0.0721442885771544	1.6550034888922\\
-0.0593981639867885	1.64328657314629\\
-0.0561122244488979	1.64026906664833\\
-0.0422588104508495	1.62725450901804\\
-0.0400801603206413	1.62520935879441\\
-0.025504946693164	1.61122244488978\\
-0.0240480961923848	1.60982511577077\\
-0.00911196622221128	1.59519038076152\\
-0.00801603206412826	1.59411683733554\\
0.00694248863814595	1.57915831663327\\
0.00801603206412782	1.57808477320728\\
0.0226787509838308	1.56312625250501\\
0.0240480961923843	1.561728923386\\
0.0381153270050993	1.54709418837675\\
0.0400801603206409	1.54504903815313\\
0.0532690706591171	1.5310621242485\\
0.0561122244488974	1.52804461775053\\
0.0681553383640204	1.51503006012024\\
0.0721442885771539	1.5107149117379\\
0.082788126291551	1.49899799599198\\
0.0881763527054105	1.49305891802733\\
0.0971801924616865	1.48296593186373\\
0.104208416833667	1.47507538159354\\
0.111343165528784	1.46693386773547\\
0.120240480961924	1.45676279285712\\
0.125287641884237	1.45090180360721\\
0.13627254509018	1.43811938573817\\
0.139023272476939	1.43486973947896\\
0.152304609218437	1.41914313537665\\
0.152558840563168	1.4188376753507\\
0.165878065541669	1.40280561122244\\
0.168336673346693	1.39983885460865\\
0.179011697335403	1.38677354709419\\
0.18436873747495	1.3801984290109\\
0.191969329661104	1.37074148296593\\
0.200400801603206	1.36021831785027\\
0.20475718723836	1.35470941883768\\
0.216432865731463	1.33989543890279\\
0.21738088618959	1.33867735470942\\
0.229825446163612	1.32264529058116\\
0.232464929859719	1.31923379942882\\
0.242111014129311	1.30661322645291\\
0.248496993987976	1.29822577640305\\
0.254249679111395	1.29058116232465\\
0.264529058116232	1.27686483474363\\
0.266245197116871	1.27454909819639\\
0.278085913715293	1.25851703406814\\
0.280561122244489	1.25515324083749\\
0.289782528321357	1.24248496993988\\
0.296593186372745	1.23308517177625\\
0.301349186753381	1.22645290581162\\
0.312625250501002	1.2106514005922\\
0.312788164902552	1.21042084168337\\
0.324080466620025	1.19438877755511\\
0.328657314629258	1.18785783358095\\
0.335252744928811	1.17835671342685\\
0.344689378757515	1.16468909288222\\
0.346307219645143	1.1623246492986\\
0.357232011798649	1.14629258517034\\
0.360721442885771	1.141146504018\\
0.368039327265191	1.13026052104208\\
0.376753507014028	1.11722056863035\\
0.378736718869431	1.11422845691383\\
0.389314546384562	1.09819639278557\\
0.392785571142285	1.09290786061383\\
0.399781828147799	1.08216432865731\\
0.408817635270541	1.06819939383122\\
0.410145301707386	1.06613226452906\\
0.420395503992964	1.0501002004008\\
0.424849699398798	1.04309121984003\\
0.430543682829265	1.03406813627254\\
0.440592084630359	1.01803607214429\\
0.440881763527054	1.0175719344274\\
0.450533390184912	1.00200400801603\\
0.456913827655311	0.991638126868775\\
0.460379816391221	0.985971943887775\\
0.470127348333343	0.969939879759519\\
0.472945891783567	0.965277406046434\\
0.479778661095603	0.953907815631262\\
0.488977955911824	0.938481885432903\\
0.489337460739446	0.937875751503006\\
0.498800763532824	0.921843687374749\\
0.50501002004008	0.911242184065805\\
0.508174479282237	0.905811623246493\\
0.517457049057759	0.889779559118236\\
0.521042084168337	0.883546681902311\\
0.526651618119988	0.87374749498998\\
0.535757770323694	0.857715430861724\\
0.537074148296593	0.855384760374438\\
0.544778659699726	0.841683366733467\\
0.55310621242485	0.826743996246779\\
0.553712876870233	0.825651302605211\\
0.562564761636584	0.809619238476954\\
0.569138276553106	0.797609863137293\\
0.571332121958972	0.793587174348698\\
0.5800184813075	0.777555110220441\\
0.585170340681363	0.767969139974112\\
0.588623238808741	0.761523046092185\\
0.597147798796265	0.745490981963928\\
0.601202404809619	0.737806525696426\\
0.605593806770739	0.729458917835672\\
0.613960138273222	0.713426853707415\\
0.617234468937876	0.707105294487803\\
0.62225086427468	0.697394789579159\\
0.630462387886829	0.681362725450902\\
0.633266533066132	0.675847212770085\\
0.638600927584093	0.665330661322646\\
0.64666091823862	0.649298597194389\\
0.649298597194389	0.644012448665355\\
0.654650008190866	0.633266533066132\\
0.662561599507746	0.617234468937876\\
0.665330661322646	0.611579473271343\\
0.670403628910887	0.601202404809619\\
0.678169817286189	0.585170340681363\\
0.681362725450902	0.578524953025124\\
0.685866838737787	0.569138276553106\\
0.69349048718108	0.55310621242485\\
0.697394789579159	0.544823632351482\\
0.701044226507351	0.537074148296593\\
0.708528068235948	0.521042084168337\\
0.713426853707415	0.510448205704373\\
0.715939933420816	0.50501002004008\\
0.723286575218568	0.488977955911824\\
0.729458917835672	0.475369178011716\\
0.7305576644713	0.472945891783567\\
0.737769589818998	0.456913827655311\\
0.744902759144508	0.440881763527054\\
0.745490981963928	0.439549388500092\\
0.751980270797614	0.424849699398798\\
0.758981402159146	0.408817635270541\\
0.761523046092185	0.402945164657454\\
0.765921363119241	0.392785571142285\\
0.772792646047509	0.376753507014028\\
0.777555110220441	0.365527472868786\\
0.779595206106097	0.360721442885771\\
0.78633871591627	0.344689378757515\\
0.79300605003884	0.328657314629258\\
0.793587174348698	0.327247893379989\\
0.799621440602657	0.312625250501002\\
0.806163058730893	0.296593186372745\\
0.809619238476954	0.288036544144152\\
0.81264225875116	0.280561122244489\\
0.819059962112337	0.264529058116232\\
0.825403267700109	0.248496993987976\\
0.825651302605211	0.247864371429403\\
0.831697706153768	0.232464929859719\\
0.837918726609621	0.216432865731463\\
0.841683366733467	0.206625295316764\\
0.84407685759974	0.200400801603206\\
0.850177130537457	0.18436873747495\\
0.856204317512007	0.168336673346693\\
0.857715430861724	0.164275435029403\\
0.862178565765762	0.152304609218437\\
0.868086222771197	0.13627254509018\\
0.87374749498998	0.120720775558425\\
0.873922746025479	0.120240480961924\\
0.879712154483288	0.104208416833667\\
0.88542933896269	0.0881763527054105\\
0.889779559118236	0.075829463852014\\
0.891081352733647	0.0721442885771539\\
0.896681104837513	0.0561122244488974\\
0.902209517311712	0.0400801603206409\\
0.905811623246493	0.0295057008813569\\
0.907676036994777	0.0240480961923843\\
0.913087586458924	0.00801603206412782\\
0.918428489847537	-0.00801603206412826\\
0.921843687374749	-0.0183980272195427\\
0.923708101123034	-0.0240480961923848\\
0.928932465899462	-0.0400801603206413\\
0.934086689113959	-0.0561122244488979\\
0.937875751503006	-0.0680549808883071\\
0.939177545118417	-0.0721442885771544\\
0.944215309975964	-0.0881763527054109\\
0.949183249838047	-0.104208416833667\\
0.953907815631262	-0.119669068517051\\
0.954083066666762	-0.120240480961924\\
0.958934384098946	-0.13627254509018\\
0.963716005591637	-0.152304609218437\\
0.968428766409803	-0.168336673346694\\
0.969939879759519	-0.173551239282138\\
0.973086645577692	-0.18436873747495\\
0.977681480427534	-0.200400801603207\\
0.982207303763929	-0.216432865731463\\
0.985971943887775	-0.229972659443533\\
0.986667731865482	-0.232464929859719\\
0.991074875445203	-0.248496993987976\\
0.995412667523159	-0.264529058116232\\
0.99968171568218	-0.280561122244489\\
1.00200400801603	-0.289425026676692\\
1.00389005021323	-0.296593186372745\\
1.00803827426999	-0.312625250501002\\
1.01211712930511	-0.328657314629258\\
1.01612708530091	-0.344689378757515\\
1.01803607214429	-0.352457101605711\\
1.02007616802994	-0.360721442885771\\
1.0239626482088	-0.376753507014028\\
1.02777931323051	-0.392785571142285\\
1.03152649233951	-0.408817635270541\\
1.03406813627254	-0.419900398966124\\
1.03520845019296	-0.424849699398798\\
1.03882897978193	-0.440881763527054\\
1.04237880825587	-0.456913827655311\\
1.04585811988899	-0.472945891783567\\
1.04926705264233	-0.488977955911824\\
1.0501002004008	-0.492983895502286\\
1.0526132801142	-0.50501002004008\\
1.05589018644029	-0.521042084168337\\
1.0590950852347	-0.537074148296593\\
1.06222796213098	-0.55310621242485\\
1.06528875386917	-0.569138276553106\\
1.06613226452906	-0.573670750899876\\
1.06828280652761	-0.585170340681363\\
1.0712052321173	-0.601202404809619\\
1.07405350670123	-0.617234468937876\\
1.07682740507046	-0.633266533066132\\
1.07952664970155	-0.649298597194389\\
1.08215090986002	-0.665330661322646\\
1.08216432865731	-0.66541532184826\\
1.08470439642679	-0.681362725450902\\
1.08718072399412	-0.697394789579158\\
1.08957941337184	-0.713426853707415\\
1.09189995632039	-0.729458917835671\\
1.09414178677222	-0.745490981963928\\
1.09630427959414	-0.761523046092184\\
1.09819639278557	-0.776094740459558\\
1.09838695251236	-0.777555110220441\\
1.10039023819144	-0.793587174348697\\
1.10231088468654	-0.809619238476954\\
1.10414806928106	-0.82565130260521\\
1.1059009041887	-0.841683366733467\\
1.10756843491448	-0.857715430861723\\
1.10914963852203	-0.87374749498998\\
1.11064342180325	-0.889779559118236\\
1.11204861934615	-0.905811623246493\\
1.11336399149656	-0.92184368737475\\
1.11422845691383	-0.933182006967936\\
1.11458796174145	-0.937875751503006\\
1.1157184454276	-0.953907815631263\\
1.11675443604705	-0.969939879759519\\
1.11769444564974	-0.985971943887776\\
1.11853690276846	-1.00200400801603\\
1.11928014981559	-1.01803607214429\\
1.11992244034429	-1.03406813627255\\
1.12046193616776	-1.0501002004008\\
1.12089670432974	-1.06613226452906\\
1.12122471391934	-1.08216432865731\\
1.12144383272227	-1.09819639278557\\
1.12155182370072	-1.11422845691383\\
1.12154634129325	-1.13026052104208\\
1.12142492752561	-1.14629258517034\\
1.12118500792299	-1.1623246492986\\
1.12082388721338	-1.17835671342685\\
1.12033874481139	-1.19438877755511\\
1.1197266300709	-1.21042084168337\\
1.11898445729446	-1.22645290581162\\
1.11810900048634	-1.24248496993988\\
1.11709688783561	-1.25851703406814\\
1.11594459591447	-1.27454909819639\\
1.11464844357632	-1.29058116232465\\
1.11422845691383	-1.29527490685972\\
1.11319708140095	-1.30661322645291\\
1.11158897321772	-1.32264529058116\\
1.1098226773115	-1.33867735470942\\
1.10789361042715	-1.35470941883768\\
1.10579698497173	-1.37074148296593\\
1.10352780003294	-1.38677354709419\\
1.1010808319104	-1.40280561122244\\
1.0984506241303	-1.4188376753507\\
1.09819639278557	-1.42029804511158\\
1.09560322010549	-1.43486973947896\\
1.09255398071092	-1.45090180360721\\
1.08929907735243	-1.46693386773547\\
1.08583167329455	-1.48296593186373\\
1.08216432865731	-1.49891333546637\\
1.0821443520793	-1.49899799599198\\
1.07817537844418	-1.51503006012024\\
1.07396500034531	-1.5310621242485\\
1.06950457834855	-1.54709418837675\\
1.06613226452906	-1.55859377815824\\
1.0647629193205	-1.56312625250501\\
1.05969013932373	-1.57915831663327\\
1.05432972957213	-1.59519038076152\\
1.0501002004008	-1.60721650529932\\
1.04864334990002	-1.61122244488978\\
1.04255657480825	-1.62725450901804\\
1.0361352655747	-1.64328657314629\\
1.03406813627254	-1.64823587357897\\
1.0292623072434	-1.65931863727455\\
1.02196731801052	-1.67535070140281\\
1.01803607214429	-1.68361504268287\\
1.01418588617855	-1.69138276553106\\
1.00587990953774	-1.70741482965932\\
1.00200400801603	-1.71458298935537\\
0.996990227098052	-1.72344689378758\\
0.987502407962091	-1.73947895791583\\
0.985971943887775	-1.74197122833202\\
0.977236362534941	-1.75551102204409\\
0.969939879759519	-1.7663285202369\\
0.966230684303344	-1.77154308617234\\
0.954346209377058	-1.7875751503006\\
0.953907815631262	-1.78814656274547\\
0.94133327615024	-1.80360721442886\\
0.937875751503006	-1.8076965221177\\
0.927121552749066	-1.81963927855711\\
0.921843687374749	-1.82528934752995\\
0.911461692219333	-1.83567134268537\\
0.905811623246493	-1.84112894737434\\
0.894024105065609	-1.85170340681363\\
0.889779559118236	-1.85538858208849\\
0.874365875445442	-1.86773547094188\\
0.87374749498998	-1.86821576553839\\
0.857715430861724	-1.87970629675285\\
0.851492446151481	-1.88376753507014\\
0.841683366733467	-1.8899920287837\\
0.825651302605211	-1.89916697663982\\
0.824429387048632	-1.8997995991984\\
0.809619238476954	-1.90727502109806\\
0.793587174348698	-1.91442224207739\\
0.790015546419612	-1.91583166332665\\
0.777555110220441	-1.92063769330967\\
0.761523046092185	-1.92599125684182\\
0.745490981963928	-1.93053135242795\\
0.739874814442951	-1.93186372745491\\
0.729458917835672	-1.93428701368306\\
0.713426853707415	-1.9373019131192\\
0.697394789579159	-1.9396132115098\\
0.681362725450902	-1.94125040392693\\
0.665330661322646	-1.94224079591663\\
0.649298597194389	-1.94260964305413\\
0.633266533066132	-1.94238027890235\\
0.617234468937876	-1.94157423236356\\
0.601202404809619	-1.94021133531566\\
0.585170340681363	-1.93830982133684\\
0.569138276553106	-1.93588641624351\\
0.55310621242485	-1.93295642109648\\
0.548019381576782	-1.93186372745491\\
0.537074148296593	-1.92953305696762\\
0.521042084168337	-1.92563085023898\\
0.50501002004008	-1.92126222414596\\
0.488977955911824	-1.91643779725655\\
0.487146074453037	-1.91583166332665\\
0.472945891783567	-1.91116918961357\\
0.456913827655311	-1.9054657821794\\
0.442103679083632	-1.8997995991984\\
0.440881763527054	-1.89933546148151\\
0.424849699398798	-1.89279061863763\\
0.408817635270541	-1.8858346643723\\
0.404332999824194	-1.88376753507014\\
0.392785571142285	-1.8784790028984\\
0.376753507014028	-1.87072758265841\\
0.370881128353451	-1.86773547094188\\
0.360721442885771	-1.86258938978954\\
0.344689378757515	-1.85406785039725\\
0.340444832810142	-1.85170340681363\\
0.328657314629258	-1.84517246283947\\
0.312625250501002	-1.83590190159421\\
0.312243254575274	-1.83567134268537\\
0.296593186372745	-1.82627154452174\\
0.285972676222856	-1.81963927855711\\
0.280561122244489	-1.81627548532647\\
0.264529058116232	-1.8059229509761\\
0.261071533468998	-1.80360721442886\\
0.248496993987976	-1.795219764379\\
0.237425008585017	-1.7875751503006\\
0.232464929859719	-1.78416365914826\\
0.216432865731463	-1.77276117036572\\
0.21477460953327	-1.77154308617234\\
0.200400801603206	-1.76101992105668\\
0.193104318827784	-1.75551102204409\\
0.18436873747495	-1.7489359039608\\
0.172173345241226	-1.73947895791583\\
0.168336673346693	-1.73651220130204\\
0.152304609218437	-1.72375235381353\\
0.151931278795001	-1.72344689378758\\
0.13627254509018	-1.71066447591853\\
0.132396643568474	-1.70741482965932\\
0.120240480961924	-1.69724375478096\\
0.113414571508777	-1.69138276553106\\
0.104208416833667	-1.68349221526087\\
0.0949445761708529	-1.67535070140281\\
0.0881763527054105	-1.66941162343815\\
0.0769501176063007	-1.65931863727455\\
0.0721442885771539	-1.6550034888922\\
0.0593981639867881	-1.64328657314629\\
0.0561122244488974	-1.64026906664833\\
0.042258810450849	-1.62725450901804\\
0.0400801603206409	-1.62520935879441\\
0.0255049466931636	-1.61122244488978\\
0.0240480961923843	-1.60982511577077\\
0.00911196622221084	-1.59519038076152\\
0.00801603206412782	-1.59411683733554\\
-0.0069424886381464	-1.57915831663327\\
-0.00801603206412826	-1.57808477320728\\
-0.0226787509838312	-1.56312625250501\\
-0.0240480961923848	-1.561728923386\\
-0.0381153270050997	-1.54709418837675\\
-0.0400801603206413	-1.54504903815313\\
-0.0532690706591176	-1.5310621242485\\
-0.0561122244488979	-1.52804461775053\\
-0.0681553383640199	-1.51503006012024\\
-0.0721442885771544	-1.5107149117379\\
-0.082788126291551	-1.49899799599198\\
-0.0881763527054109	-1.49305891802733\\
-0.0971801924616865	-1.48296593186373\\
-0.104208416833667	-1.47507538159353\\
-0.111343165528784	-1.46693386773547\\
-0.120240480961924	-1.45676279285711\\
-0.125287641884237	-1.45090180360721\\
-0.13627254509018	-1.43811938573817\\
-0.13902327247694	-1.43486973947896\\
-0.152304609218437	-1.41914313537665\\
-0.152558840563168	-1.4188376753507\\
-0.165878065541669	-1.40280561122244\\
-0.168336673346694	-1.39983885460865\\
-0.179011697335402	-1.38677354709419\\
-0.18436873747495	-1.3801984290109\\
-0.191969329661105	-1.37074148296593\\
-0.200400801603207	-1.36021831785027\\
-0.204757187238359	-1.35470941883768\\
-0.216432865731463	-1.33989543890279\\
-0.21738088618959	-1.33867735470942\\
-0.229825446163611	-1.32264529058116\\
-0.232464929859719	-1.31923379942883\\
-0.242111014129311	-1.30661322645291\\
-0.248496993987976	-1.29822577640305\\
-0.254249679111395	-1.29058116232465\\
-0.264529058116232	-1.27686483474363\\
-0.266245197116871	-1.27454909819639\\
-0.278085913715293	-1.25851703406814\\
-0.280561122244489	-1.25515324083749\\
-0.289782528321357	-1.24248496993988\\
-0.296593186372745	-1.23308517177625\\
-0.301349186753381	-1.22645290581162\\
-0.312625250501002	-1.2106514005922\\
-0.312788164902552	-1.21042084168337\\
-0.324080466620025	-1.19438877755511\\
-0.328657314629258	-1.18785783358095\\
-0.33525274492881	-1.17835671342685\\
-0.344689378757515	-1.16468909288222\\
-0.346307219645143	-1.1623246492986\\
-0.357232011798649	-1.14629258517034\\
-0.360721442885771	-1.141146504018\\
-0.368039327265191	-1.13026052104208\\
-0.376753507014028	-1.11722056863035\\
-0.37873671886943	-1.11422845691383\\
-0.389314546384562	-1.09819639278557\\
-0.392785571142285	-1.09290786061383\\
-0.399781828147799	-1.08216432865731\\
-0.408817635270541	-1.06819939383122\\
-0.410145301707386	-1.06613226452906\\
-0.420395503992964	-1.0501002004008\\
-0.424849699398798	-1.04309121984003\\
-0.430543682829265	-1.03406813627255\\
-0.440592084630359	-1.01803607214429\\
-0.440881763527054	-1.0175719344274\\
-0.450533390184912	-1.00200400801603\\
-0.456913827655311	-0.991638126868775\\
-0.46037981639122	-0.985971943887776\\
-0.470127348333342	-0.969939879759519\\
-0.472945891783567	-0.965277406046435\\
-0.479778661095603	-0.953907815631263\\
-0.488977955911824	-0.938481885432904\\
-0.489337460739446	-0.937875751503006\\
-0.498800763532823	-0.92184368737475\\
-0.50501002004008	-0.911242184065805\\
-0.508174479282237	-0.905811623246493\\
-0.517457049057759	-0.889779559118236\\
-0.521042084168337	-0.883546681902311\\
-0.526651618119988	-0.87374749498998\\
-0.535757770323694	-0.857715430861723\\
-0.537074148296593	-0.855384760374438\\
-0.544778659699726	-0.841683366733467\\
-0.55310621242485	-0.826743996246779\\
-0.553712876870233	-0.82565130260521\\
-0.562564761636584	-0.809619238476954\\
-0.569138276553106	-0.797609863137293\\
-0.571332121958972	-0.793587174348697\\
-0.5800184813075	-0.777555110220441\\
-0.585170340681363	-0.767969139974112\\
-0.588623238808741	-0.761523046092184\\
-0.597147798796266	-0.745490981963928\\
-0.601202404809619	-0.737806525696426\\
-0.60559380677074	-0.729458917835671\\
-0.613960138273222	-0.713426853707415\\
-0.617234468937876	-0.707105294487803\\
-0.622250864274681	-0.697394789579158\\
-0.630462387886829	-0.681362725450902\\
-0.633266533066132	-0.675847212770085\\
-0.638600927584093	-0.665330661322646\\
-0.646660918238621	-0.649298597194389\\
-0.649298597194389	-0.644012448665355\\
-0.654650008190866	-0.633266533066132\\
-0.662561599507746	-0.617234468937876\\
-0.665330661322646	-0.611579473271343\\
-0.670403628910887	-0.601202404809619\\
-0.678169817286189	-0.585170340681363\\
-0.681362725450902	-0.578524953025125\\
-0.685866838737788	-0.569138276553106\\
-0.693490487181081	-0.55310621242485\\
-0.697394789579158	-0.544823632351483\\
-0.701044226507351	-0.537074148296593\\
-0.708528068235948	-0.521042084168337\\
-0.713426853707415	-0.510448205704374\\
-0.715939933420816	-0.50501002004008\\
-0.723286575218568	-0.488977955911824\\
-0.729458917835671	-0.475369178011717\\
-0.730557664471301	-0.472945891783567\\
-0.737769589818998	-0.456913827655311\\
-0.744902759144508	-0.440881763527054\\
-0.745490981963928	-0.439549388500093\\
-0.751980270797614	-0.424849699398798\\
-0.758981402159147	-0.408817635270541\\
-0.761523046092184	-0.402945164657456\\
-0.765921363119241	-0.392785571142285\\
-0.772792646047509	-0.376753507014028\\
-0.777555110220441	-0.365527472868787\\
-0.779595206106096	-0.360721442885771\\
-0.78633871591627	-0.344689378757515\\
-0.79300605003884	-0.328657314629258\\
-0.793587174348697	-0.327247893379991\\
-0.799621440602657	-0.312625250501002\\
-0.806163058730893	-0.296593186372745\\
-0.809619238476954	-0.288036544144153\\
-0.81264225875116	-0.280561122244489\\
-0.819059962112338	-0.264529058116232\\
-0.825403267700109	-0.248496993987976\\
-0.82565130260521	-0.247864371429404\\
-0.831697706153768	-0.232464929859719\\
-0.83791872660962	-0.216432865731463\\
-0.841683366733467	-0.206625295316765\\
-0.84407685759974	-0.200400801603207\\
-0.850177130537458	-0.18436873747495\\
-0.856204317512007	-0.168336673346694\\
-0.857715430861723	-0.164275435029404\\
-0.862178565765761	-0.152304609218437\\
-0.868086222771196	-0.13627254509018\\
-0.87374749498998	-0.120720775558426\\
-0.873922746025479	-0.120240480961924\\
-0.879712154483288	-0.104208416833667\\
-0.885429338962689	-0.0881763527054109\\
-0.889779559118236	-0.075829463852013\\
-0.891081352733647	-0.0721442885771544\\
-0.896681104837513	-0.0561122244488979\\
-0.902209517311712	-0.0400801603206413\\
-0.905811623246493	-0.0295057008813555\\
-0.907676036994777	-0.0240480961923848\\
-0.913087586458924	-0.00801603206412826\\
-0.918428489847537	0.00801603206412782\\
-0.92184368737475	0.0183980272195441\\
-0.923708101123034	0.0240480961923843\\
-0.928932465899462	0.0400801603206409\\
-0.934086689113959	0.0561122244488974\\
-0.937875751503006	0.0680549808883089\\
-0.939177545118416	0.0721442885771539\\
-0.944215309975965	0.0881763527054105\\
-0.949183249838047	0.104208416833667\\
-0.953907815631263	0.119669068517051\\
-0.954083066666762	0.120240480961924\\
-0.958934384098946	0.13627254509018\\
-0.963716005591637	0.152304609218437\\
-0.968428766409803	0.168336673346693\\
-0.969939879759519	0.173551239282139\\
-0.973086645577692	0.18436873747495\\
-0.977681480427534	0.200400801603206\\
-0.982207303763929	0.216432865731463\\
-0.985971943887776	0.229972659443534\\
-0.986667731865482	0.232464929859719\\
-0.991074875445203	0.248496993987976\\
-0.995412667523159	0.264529058116232\\
-0.99968171568218	0.280561122244489\\
-1.00200400801603	0.289425026676694\\
-1.00389005021323	0.296593186372745\\
-1.00803827426999	0.312625250501002\\
-1.01211712930511	0.328657314629258\\
-1.01612708530091	0.344689378757515\\
-1.01803607214429	0.352457101605714\\
-1.02007616802994	0.360721442885771\\
-1.0239626482088	0.376753507014028\\
-1.02777931323051	0.392785571142285\\
-1.03152649233951	0.408817635270541\\
-1.03406813627255	0.419900398966127\\
-1.03520845019296	0.424849699398798\\
-1.03882897978193	0.440881763527054\\
-1.04237880825587	0.456913827655311\\
-1.04585811988899	0.472945891783567\\
-1.04926705264232	0.488977955911824\\
-1.0501002004008	0.492983895502289\\
-1.0526132801142	0.50501002004008\\
-1.05589018644029	0.521042084168337\\
-1.0590950852347	0.537074148296593\\
-1.06222796213098	0.55310621242485\\
-1.06528875386917	0.569138276553106\\
-1.06613226452906	0.57367075089988\\
-1.06828280652761	0.585170340681363\\
-1.0712052321173	0.601202404809619\\
-1.07405350670123	0.617234468937876\\
-1.07682740507046	0.633266533066132\\
-1.07952664970155	0.649298597194389\\
-1.08215090986002	0.665330661322646\\
-1.08216432865731	0.665415321848262\\
-1.08470439642679	0.681362725450902\\
-1.08718072399412	0.697394789579159\\
-1.08957941337184	0.713426853707415\\
-1.09189995632039	0.729458917835672\\
-1.09414178677222	0.745490981963928\\
-1.09630427959414	0.761523046092185\\
-1.09819639278557	0.776094740459559\\
-1.09838695251236	0.777555110220441\\
-1.10039023819144	0.793587174348698\\
-1.10231088468654	0.809619238476954\\
-1.10414806928106	0.825651302605211\\
-1.1059009041887	0.841683366733467\\
-1.10756843491448	0.857715430861724\\
-1.10914963852203	0.87374749498998\\
-1.11064342180325	0.889779559118236\\
-1.11204861934615	0.905811623246493\\
-1.11336399149656	0.921843687374749\\
-1.11422845691383	0.933182006967935\\
-1.11458796174145	0.937875751503006\\
-1.1157184454276	0.953907815631262\\
-1.11675443604705	0.969939879759519\\
-1.11769444564974	0.985971943887775\\
-1.11853690276846	1.00200400801603\\
-1.11928014981559	1.01803607214429\\
-1.11992244034429	1.03406813627254\\
-1.12046193616776	1.0501002004008\\
-1.12089670432974	1.06613226452906\\
-1.12122471391934	1.08216432865731\\
-1.12144383272227	1.09819639278557\\
-1.12155182370072	1.11422845691383\\
-1.12154634129325	1.13026052104208\\
-1.12142492752561	1.14629258517034\\
-1.12118500792299	1.1623246492986\\
-1.12082388721338	1.17835671342685\\
-1.12033874481139	1.19438877755511\\
-1.1197266300709	1.21042084168337\\
-1.11898445729446	1.22645290581162\\
-1.11810900048634	1.24248496993988\\
-1.11709688783561	1.25851703406814\\
-1.11594459591447	1.27454909819639\\
-1.11464844357632	1.29058116232465\\
-1.11422845691383	1.29527490685972\\
}--cycle;


\addplot[area legend,solid,fill=mycolor8,draw=black,forget plot]
table[row sep=crcr] {%
x	y\\
-0.969939879759519	1.10267832172919\\
-0.969057219367736	1.11422845691383\\
-0.967666023500817	1.13026052104208\\
-0.96610008118631	1.14629258517034\\
-0.964353882669638	1.1623246492986\\
-0.962421651452387	1.17835671342685\\
-0.960297331194264	1.19438877755511\\
-0.957974571822777	1.21042084168337\\
-0.955446714796597	1.22645290581162\\
-0.953907815631263	1.23550430311173\\
-0.95268652032031	1.24248496993988\\
-0.949675465068525	1.25851703406814\\
-0.946430689407037	1.27454909819639\\
-0.942943626489223	1.29058116232465\\
-0.939205278892995	1.30661322645291\\
-0.937875751503006	1.31199129212065\\
-0.935151423287963	1.32264529058116\\
-0.930790674073424	1.33867735470942\\
-0.926138506431433	1.35470941883768\\
-0.92184368737475	1.36863159698073\\
-0.921167596555201	1.37074148296593\\
-0.91576922275681	1.38677354709419\\
-0.910028348601568	1.40280561122244\\
-0.905811623246493	1.4139460396668\\
-0.903879403874429	1.4188376753507\\
-0.89722860759457	1.43486973947896\\
-0.890169387584158	1.45090180360721\\
-0.889779559118236	1.45175054669336\\
-0.882464023832977	1.46693386773547\\
-0.874275126038673	1.48296593186373\\
-0.87374749498998	1.4839573570158\\
-0.865310295946134	1.49899799599198\\
-0.857715430861723	1.51182056797403\\
-0.85570327730774	1.51503006012024\\
-0.845197697745205	1.5310621242485\\
-0.841683366733467	1.5361856585203\\
-0.83371642719017	1.54709418837675\\
-0.82565130260521	1.55763141692518\\
-0.821149424312379	1.56312625250501\\
-0.809619238476954	1.57659144839573\\
-0.807252347724585	1.57915831663327\\
-0.793587174348697	1.59337379320406\\
-0.791693855339852	1.59519038076152\\
-0.777555110220441	1.60823431958121\\
-0.774015623754877	1.61122244488978\\
-0.761523046092184	1.62138648351191\\
-0.753568053904924	1.62725450901804\\
-0.745490981963928	1.6330089628635\\
-0.729458917835671	1.64325132499136\\
-0.729397545610944	1.64328657314629\\
-0.713426853707415	1.65217400092951\\
-0.698733223512349	1.65931863727455\\
-0.697394789579158	1.65995043575415\\
-0.681362725450902	1.66660099071443\\
-0.665330661322646	1.67225062416767\\
-0.654827748809892	1.67535070140281\\
-0.649298597194389	1.67694127241858\\
-0.633266533066132	1.68072001757187\\
-0.617234468937876	1.6836551377129\\
-0.601202404809619	1.68578743348757\\
-0.585170340681363	1.68715450310275\\
-0.569138276553106	1.6877909773566\\
-0.55310621242485	1.68772873313515\\
-0.537074148296593	1.68699708749149\\
-0.521042084168337	1.68562297418631\\
-0.50501002004008	1.68363110436025\\
-0.488977955911824	1.681044112825\\
-0.472945891783567	1.67788269129889\\
-0.462062759340567	1.67535070140281\\
-0.456913827655311	1.67416168445254\\
-0.440881763527054	1.66989348069114\\
-0.424849699398798	1.66510347263781\\
-0.408817635270541	1.65980612744266\\
-0.40747920133735	1.65931863727455\\
-0.392785571142285	1.65400113858226\\
-0.376753507014028	1.64771432457223\\
-0.366276833404529	1.64328657314629\\
-0.360721442885771	1.64095253996619\\
-0.344689378757515	1.63372219754025\\
-0.331202431040213	1.62725450901804\\
-0.328657314629258	1.62604056421676\\
-0.312625250501002	1.61790654165211\\
-0.300132672838309	1.61122244488978\\
-0.296593186372745	1.60933788456875\\
-0.280561122244489	1.60033434295526\\
-0.271830152307451	1.59519038076152\\
-0.264529058116232	1.59090760245724\\
-0.248496993987976	1.5810633694857\\
-0.245529404938141	1.57915831663327\\
-0.232464929859719	1.57080380915392\\
-0.220934744024294	1.56312625250501\\
-0.216432865731463	1.56013937290885\\
-0.200400801603207	1.54907207324439\\
-0.19764255158661	1.54709418837675\\
-0.18436873747495	1.53760511938098\\
-0.175535084883226	1.5310621242485\\
-0.168336673346694	1.52574538852246\\
-0.15431676277242	1.51503006012024\\
-0.152304609218437	1.51349613165084\\
-0.13627254509018	1.50085872488534\\
-0.133985795284029	1.49899799599198\\
-0.120240480961924	1.48783640452733\\
-0.11442157751627	1.48296593186373\\
-0.104208416833667	1.47443279432367\\
-0.0954918879906701	1.46693386773547\\
-0.0881763527054109	1.46064992979705\\
-0.0771441629499948	1.45090180360721\\
-0.0721442885771544	1.44648955083659\\
-0.0593315894323775	1.43486973947896\\
-0.0561122244488979	1.4319531039411\\
-0.0420123796927054	1.4188376753507\\
-0.0400801603206413	1.41704174408248\\
-0.0251491528545071	1.40280561122244\\
-0.0240480961923848	1.40175633619233\\
-0.00870840581046833	1.38677354709419\\
-0.00801603206412826	1.38609745627464\\
0.00733994124458037	1.37074148296593\\
0.00801603206412782	1.37006539214638\\
0.0230229365910116	1.35470941883768\\
0.0240480961923843	1.35366014380756\\
0.0383649358792694	1.33867735470942\\
0.0400801603206409	1.33688142344119\\
0.0533878962338543	1.32264529058116\\
0.0561122244488974	1.3197286550433\\
0.0681116322727954	1.30661322645291\\
0.0721442885771539	1.30220097368228\\
0.0825540396247736	1.29058116232465\\
0.0881763527054105	1.28429722438623\\
0.0967312906094415	1.27454909819639\\
0.104208416833667	1.26601596065633\\
0.110658005995674	1.25851703406814\\
0.120240480961924	1.24735544260349\\
0.124347406134783	1.24248496993988\\
0.13627254509018	1.22831363470498\\
0.137811444255516	1.22645290581162\\
0.151041518246361	1.21042084168337\\
0.152304609218437	1.20888691321397\\
0.164043027344138	1.19438877755511\\
0.168336673346693	1.18907204182907\\
0.176850509167818	1.17835671342685\\
0.18436873747495	1.16886764443108\\
0.189472282332026	1.1623246492986\\
0.200400801603206	1.14827047003798\\
0.201915825520064	1.14629258517034\\
0.21415900947276	1.13026052104208\\
0.216432865731463	1.12727364144593\\
0.226219685290747	1.11422845691383\\
0.232464929859719	1.10587394943448\\
0.238124052009175	1.09819639278557\\
0.248496993987976	1.08406938150975\\
0.249877068771455	1.08216432865731\\
0.261449565491779	1.06613226452906\\
0.264529058116232	1.06184948622477\\
0.272867526498327	1.0501002004008\\
0.280561122244489	1.03921209846628\\
0.284150637741463	1.03406813627254\\
0.295289049977966	1.01803607214429\\
0.296593186372745	1.01615151182326\\
0.306265440129244	1.00200400801603\\
0.312625250501002	0.992656040650101\\
0.317120335276949	0.985971943887775\\
0.327848425282381	0.969939879759519\\
0.328657314629258	0.968725934958247\\
0.338421193847936	0.953907815631262\\
0.344689378757515	0.944343440025218\\
0.348883112072032	0.937875751503006\\
0.359223219238986	0.921843687374749\\
0.360721442885771	0.919509654194643\\
0.369424311362763	0.905811623246493\\
0.376753507014028	0.894207310544178\\
0.37952284977305	0.889779559118236\\
0.389495963689498	0.87374749498998\\
0.392785571142285	0.86842999629769\\
0.399352061568488	0.857715430861724\\
0.408817635270541	0.842170856901581\\
0.409111855743252	0.841683366733467\\
0.418737689933317	0.825651302605211\\
0.424849699398798	0.81540407384021\\
0.428270836853321	0.809619238476954\\
0.437693808705884	0.793587174348698\\
0.440881763527054	0.788129953637036\\
0.447009199199949	0.777555110220441\\
0.456233617637009	0.761523046092185\\
0.456913827655311	0.760334029141922\\
0.465339483189213	0.745490981963928\\
0.472945891783567	0.731990907731753\\
0.474362142761551	0.729458917835672\\
0.48327342673314	0.713426853707415\\
0.488977955911824	0.703088201001349\\
0.492097725170469	0.697394789579159\\
0.500821999933126	0.681362725450902\\
0.50501002004008	0.673611064280093\\
0.509454470646171	0.665330661322646\\
0.51799543742193	0.649298597194389\\
0.521042084168337	0.643538805849636\\
0.526442067692356	0.633266533066132\\
0.534803268494776	0.617234468937876\\
0.537074148296593	0.612848790898299\\
0.543069523425128	0.601202404809619\\
0.551254345150408	0.585170340681363\\
0.55310621242485	0.581516308285451\\
0.559345189869164	0.569138276553106\\
0.567356868153898	0.55310621242485\\
0.569138276553106	0.549514424250385\\
0.575276788374131	0.537074148296593\\
0.583118411224556	0.521042084168337\\
0.585170340681363	0.516813821740019\\
0.590871432247094	0.50501002004008\\
0.598545943444414	0.488977955911824\\
0.601202404809619	0.483382623868328\\
0.606135647689298	0.472945891783567\\
0.613645849975363	0.456913827655311\\
0.617234468937876	0.44918619983715\\
0.621075393118784	0.440881763527054\\
0.62842395116611	0.424849699398798\\
0.633266533066132	0.414186951439601\\
0.635696076953843	0.408817635270541\\
0.642885520123645	0.392785571142285\\
0.649298597194389	0.378344078029799\\
0.650002573926212	0.376753507014028\\
0.657035298817787	0.360721442885771\\
0.663996859551543	0.344689378757515\\
0.665330661322646	0.341589301522375\\
0.670877512781642	0.328657314629258\\
0.677684346364305	0.312625250501002\\
0.681362725450902	0.303875539812623\\
0.684415884465412	0.296593186372745\\
0.691070078445277	0.280561122244489\\
0.697394789579159	0.265160856595837\\
0.697653645295825	0.264529058116232\\
0.704157161442419	0.248496993987976\\
0.71058999818202	0.232464929859719\\
0.713426853707415	0.225320293514676\\
0.716948222425681	0.216432865731463\\
0.723231725412169	0.200400801603206\\
0.729445402395971	0.18436873747495\\
0.729458917835672	0.184333489320014\\
0.7355812556841	0.168336673346693\\
0.741646656558198	0.152304609218437\\
0.745490981963928	0.142026998935648\\
0.747640164685551	0.13627254509018\\
0.753558788829143	0.120240480961924\\
0.759407488220804	0.104208416833667\\
0.761523046092185	0.0983403913275378\\
0.765182801678966	0.0881763527054105\\
0.77088533435874	0.0721442885771539\\
0.776518228317745	0.0561122244488974\\
0.777555110220441	0.053124099140325\\
0.782076830845543	0.0400801603206409\\
0.787563890959639	0.0240480961923843\\
0.792981374834583	0.00801603206412782\\
0.793587174348698	0.00619944450666274\\
0.798324247801242	-0.00801603206412826\\
0.803595955087895	-0.0240480961923848\\
0.80879792546851	-0.0400801603206413\\
0.809619238476954	-0.0426470285581829\\
0.81392555091599	-0.0561122244488979\\
0.818981526743509	-0.0721442885771544\\
0.823967381477785	-0.0881763527054109\\
0.825651302605211	-0.0936711882852378\\
0.828879743236386	-0.104208416833667\\
0.833719109470425	-0.120240480961924\\
0.838487745757591	-0.13627254509018\\
0.841683366733467	-0.14718107494664\\
0.843184328639133	-0.152304609218437\\
0.847805704581895	-0.168336673346694\\
0.852355515131915	-0.18436873747495\\
0.856834199960817	-0.200400801603207\\
0.857715430861724	-0.203610293749422\\
0.86123679957999	-0.216432865731463\\
0.865565666819784	-0.232464929859719\\
0.869822225003994	-0.248496993987976\\
0.87374749498998	-0.263537632964165\\
0.874006350706646	-0.264529058116232\\
0.878111639002643	-0.280561122244489\\
0.8821431997404	-0.296593186372745\\
0.88610118003164	-0.312625250501002\\
0.889779559118236	-0.32780857154311\\
0.889985305387594	-0.328657314629258\\
0.893788465297644	-0.344689378757515\\
0.897516260741634	-0.360721442885771\\
0.901168657328444	-0.376753507014028\\
0.904745563887746	-0.392785571142285\\
0.905811623246493	-0.397677206826184\\
0.908241167134204	-0.408817635270541\\
0.911657017285245	-0.424849699398798\\
0.914995079503241	-0.440881763527054\\
0.918255068412237	-0.456913827655311\\
0.921436637000515	-0.472945891783567\\
0.921843687374749	-0.47505577776877\\
0.924531625014911	-0.488977955911824\\
0.92754477894048	-0.50501002004008\\
0.93047666597542	-0.521042084168337\\
0.933326728992571	-0.537074148296593\\
0.936094343103798	-0.55310621242485\\
0.937875751503006	-0.563760210885365\\
0.938775714503375	-0.569138276553106\\
0.941366364051953	-0.585170340681363\\
0.943871126631541	-0.601202404809619\\
0.946289147749201	-0.617234468937876\\
0.94861949758841	-0.633266533066132\\
0.950861168884856	-0.649298597194389\\
0.953013074674022	-0.665330661322646\\
0.953907815631262	-0.672311328150791\\
0.955068713952377	-0.681362725450902\\
0.957027584889907	-0.697394789579158\\
0.958892252810273	-0.713426853707415\\
0.960661363087028	-0.729458917835671\\
0.962333471165166	-0.745490981963928\\
0.963907039627009	-0.761523046092184\\
0.965380435091499	-0.777555110220441\\
0.966751924938348	-0.793587174348697\\
0.968019673847939	-0.809619238476954\\
0.969181740147287	-0.82565130260521\\
0.969939879759519	-0.837201437789848\\
0.97023410023223	-0.841683366733467\\
0.971171862150369	-0.857715430861723\\
0.971997666920664	-0.87374749498998\\
0.972709222518539	-0.889779559118236\\
0.973304114743161	-0.905811623246493\\
0.973779802541407	-0.92184368737475\\
0.974133613074036	-0.937875751503006\\
0.974362736509592	-0.953907815631263\\
0.974464220530582	-0.969939879759519\\
0.974434964535466	-0.985971943887776\\
0.974271713518829	-1.00200400801603\\
0.973971051610933	-1.01803607214429\\
0.973529395256493	-1.03406813627255\\
0.972942986011204	-1.0501002004008\\
0.97220788293294	-1.06613226452906\\
0.971319954542998	-1.08216432865731\\
0.970274870330963	-1.09819639278557\\
0.969939879759519	-1.10267832172919\\
0.969057219367736	-1.11422845691383\\
0.967666023500816	-1.13026052104208\\
0.96610008118631	-1.14629258517034\\
0.964353882669638	-1.1623246492986\\
0.962421651452387	-1.17835671342685\\
0.960297331194265	-1.19438877755511\\
0.957974571822777	-1.21042084168337\\
0.955446714796598	-1.22645290581162\\
0.953907815631262	-1.23550430311173\\
0.95268652032031	-1.24248496993988\\
0.949675465068525	-1.25851703406814\\
0.946430689407037	-1.27454909819639\\
0.942943626489224	-1.29058116232465\\
0.939205278892995	-1.30661322645291\\
0.937875751503006	-1.31199129212065\\
0.935151423287963	-1.32264529058116\\
0.930790674073424	-1.33867735470942\\
0.926138506431433	-1.35470941883768\\
0.921843687374749	-1.36863159698073\\
0.921167596555201	-1.37074148296593\\
0.91576922275681	-1.38677354709419\\
0.910028348601568	-1.40280561122244\\
0.905811623246493	-1.4139460396668\\
0.903879403874429	-1.4188376753507\\
0.897228607594571	-1.43486973947896\\
0.890169387584157	-1.45090180360721\\
0.889779559118236	-1.45175054669336\\
0.882464023832978	-1.46693386773547\\
0.874275126038673	-1.48296593186373\\
0.87374749498998	-1.4839573570158\\
0.865310295946134	-1.49899799599198\\
0.857715430861724	-1.51182056797403\\
0.85570327730774	-1.51503006012024\\
0.845197697745205	-1.5310621242485\\
0.841683366733467	-1.53618565852029\\
0.83371642719017	-1.54709418837675\\
0.825651302605211	-1.55763141692518\\
0.821149424312379	-1.56312625250501\\
0.809619238476954	-1.57659144839572\\
0.807252347724586	-1.57915831663327\\
0.793587174348698	-1.59337379320406\\
0.791693855339852	-1.59519038076152\\
0.777555110220441	-1.60823431958121\\
0.774015623754877	-1.61122244488978\\
0.761523046092185	-1.62138648351191\\
0.753568053904925	-1.62725450901804\\
0.745490981963928	-1.6330089628635\\
0.729458917835672	-1.64325132499136\\
0.729397545610944	-1.64328657314629\\
0.713426853707415	-1.65217400092951\\
0.698733223512349	-1.65931863727455\\
0.697394789579159	-1.65995043575415\\
0.681362725450902	-1.66660099071443\\
0.665330661322646	-1.67225062416767\\
0.654827748809895	-1.67535070140281\\
0.649298597194389	-1.67694127241858\\
0.633266533066132	-1.68072001757187\\
0.617234468937876	-1.6836551377129\\
0.601202404809619	-1.68578743348757\\
0.585170340681363	-1.68715450310274\\
0.569138276553106	-1.6877909773566\\
0.55310621242485	-1.68772873313515\\
0.537074148296593	-1.68699708749149\\
0.521042084168337	-1.68562297418631\\
0.50501002004008	-1.68363110436025\\
0.488977955911824	-1.681044112825\\
0.472945891783567	-1.67788269129889\\
0.462062759340563	-1.67535070140281\\
0.456913827655311	-1.67416168445254\\
0.440881763527054	-1.66989348069114\\
0.424849699398798	-1.66510347263781\\
0.408817635270541	-1.65980612744266\\
0.407479201337349	-1.65931863727455\\
0.392785571142285	-1.65400113858226\\
0.376753507014028	-1.64771432457223\\
0.366276833404527	-1.64328657314629\\
0.360721442885771	-1.64095253996619\\
0.344689378757515	-1.63372219754025\\
0.331202431040212	-1.62725450901804\\
0.328657314629258	-1.62604056421676\\
0.312625250501002	-1.61790654165211\\
0.300132672838309	-1.61122244488978\\
0.296593186372745	-1.60933788456875\\
0.280561122244489	-1.60033434295526\\
0.27183015230745	-1.59519038076152\\
0.264529058116232	-1.59090760245724\\
0.248496993987976	-1.5810633694857\\
0.245529404938141	-1.57915831663327\\
0.232464929859719	-1.57080380915392\\
0.220934744024294	-1.56312625250501\\
0.216432865731463	-1.56013937290885\\
0.200400801603206	-1.54907207324439\\
0.19764255158661	-1.54709418837675\\
0.18436873747495	-1.53760511938098\\
0.175535084883226	-1.5310621242485\\
0.168336673346693	-1.52574538852245\\
0.15431676277242	-1.51503006012024\\
0.152304609218437	-1.51349613165084\\
0.13627254509018	-1.50085872488534\\
0.133985795284029	-1.49899799599198\\
0.120240480961924	-1.48783640452733\\
0.11442157751627	-1.48296593186373\\
0.104208416833667	-1.47443279432367\\
0.0954918879906697	-1.46693386773547\\
0.0881763527054105	-1.46064992979705\\
0.0771441629499948	-1.45090180360721\\
0.0721442885771539	-1.44648955083659\\
0.0593315894323771	-1.43486973947896\\
0.0561122244488974	-1.4319531039411\\
0.042012379692705	-1.4188376753507\\
0.0400801603206409	-1.41704174408247\\
0.025149152854508	-1.40280561122244\\
0.0240480961923843	-1.40175633619233\\
0.0087084058104688	-1.38677354709419\\
0.00801603206412782	-1.38609745627464\\
-0.0073399412445799	-1.37074148296593\\
-0.00801603206412826	-1.37006539214638\\
-0.023022936591012	-1.35470941883768\\
-0.0240480961923848	-1.35366014380756\\
-0.0383649358792694	-1.33867735470942\\
-0.0400801603206413	-1.33688142344119\\
-0.0533878962338543	-1.32264529058116\\
-0.0561122244488979	-1.3197286550433\\
-0.0681116322727958	-1.30661322645291\\
-0.0721442885771544	-1.30220097368228\\
-0.0825540396247731	-1.29058116232465\\
-0.0881763527054109	-1.28429722438623\\
-0.0967312906094415	-1.27454909819639\\
-0.104208416833667	-1.26601596065633\\
-0.110658005995674	-1.25851703406814\\
-0.120240480961924	-1.24735544260349\\
-0.124347406134783	-1.24248496993988\\
-0.13627254509018	-1.22831363470498\\
-0.137811444255515	-1.22645290581162\\
-0.151041518246361	-1.21042084168337\\
-0.152304609218437	-1.20888691321397\\
-0.164043027344137	-1.19438877755511\\
-0.168336673346694	-1.18907204182907\\
-0.176850509167818	-1.17835671342685\\
-0.18436873747495	-1.16886764443108\\
-0.189472282332025	-1.1623246492986\\
-0.200400801603207	-1.14827047003798\\
-0.201915825520063	-1.14629258517034\\
-0.214159009472761	-1.13026052104208\\
-0.216432865731463	-1.12727364144593\\
-0.226219685290747	-1.11422845691383\\
-0.232464929859719	-1.10587394943448\\
-0.238124052009175	-1.09819639278557\\
-0.248496993987976	-1.08406938150975\\
-0.249877068771455	-1.08216432865731\\
-0.261449565491779	-1.06613226452906\\
-0.264529058116232	-1.06184948622477\\
-0.272867526498328	-1.0501002004008\\
-0.280561122244489	-1.03921209846628\\
-0.284150637741463	-1.03406813627255\\
-0.295289049977966	-1.01803607214429\\
-0.296593186372745	-1.01615151182326\\
-0.306265440129243	-1.00200400801603\\
-0.312625250501002	-0.992656040650101\\
-0.317120335276949	-0.985971943887776\\
-0.327848425282381	-0.969939879759519\\
-0.328657314629258	-0.968725934958247\\
-0.338421193847936	-0.953907815631263\\
-0.344689378757515	-0.944343440025219\\
-0.348883112072033	-0.937875751503006\\
-0.359223219238986	-0.92184368737475\\
-0.360721442885771	-0.919509654194644\\
-0.369424311362763	-0.905811623246493\\
-0.376753507014028	-0.894207310544178\\
-0.379522849773049	-0.889779559118236\\
-0.389495963689499	-0.87374749498998\\
-0.392785571142285	-0.86842999629769\\
-0.399352061568489	-0.857715430861723\\
-0.408817635270541	-0.84217085690158\\
-0.409111855743253	-0.841683366733467\\
-0.418737689933318	-0.82565130260521\\
-0.424849699398798	-0.81540407384021\\
-0.428270836853321	-0.809619238476954\\
-0.437693808705884	-0.793587174348697\\
-0.440881763527054	-0.788129953637036\\
-0.447009199199949	-0.777555110220441\\
-0.45623361763701	-0.761523046092184\\
-0.456913827655311	-0.760334029141922\\
-0.465339483189214	-0.745490981963928\\
-0.472945891783567	-0.731990907731753\\
-0.474362142761551	-0.729458917835671\\
-0.48327342673314	-0.713426853707415\\
-0.488977955911824	-0.703088201001348\\
-0.49209772517047	-0.697394789579158\\
-0.500821999933126	-0.681362725450902\\
-0.50501002004008	-0.673611064280093\\
-0.509454470646172	-0.665330661322646\\
-0.51799543742193	-0.649298597194389\\
-0.521042084168337	-0.643538805849636\\
-0.526442067692356	-0.633266533066132\\
-0.534803268494776	-0.617234468937876\\
-0.537074148296593	-0.612848790898299\\
-0.543069523425128	-0.601202404809619\\
-0.551254345150408	-0.585170340681363\\
-0.55310621242485	-0.581516308285452\\
-0.559345189869164	-0.569138276553106\\
-0.567356868153898	-0.55310621242485\\
-0.569138276553106	-0.549514424250385\\
-0.57527678837413	-0.537074148296593\\
-0.583118411224556	-0.521042084168337\\
-0.585170340681363	-0.516813821740019\\
-0.590871432247094	-0.50501002004008\\
-0.598545943444414	-0.488977955911824\\
-0.601202404809619	-0.483382623868328\\
-0.606135647689298	-0.472945891783567\\
-0.613645849975363	-0.456913827655311\\
-0.617234468937876	-0.449186199837151\\
-0.621075393118784	-0.440881763527054\\
-0.62842395116611	-0.424849699398798\\
-0.633266533066132	-0.414186951439601\\
-0.635696076953843	-0.408817635270541\\
-0.642885520123645	-0.392785571142285\\
-0.649298597194389	-0.378344078029798\\
-0.650002573926212	-0.376753507014028\\
-0.657035298817786	-0.360721442885771\\
-0.663996859551543	-0.344689378757515\\
-0.665330661322646	-0.341589301522375\\
-0.670877512781641	-0.328657314629258\\
-0.677684346364305	-0.312625250501002\\
-0.681362725450902	-0.303875539812625\\
-0.684415884465412	-0.296593186372745\\
-0.691070078445277	-0.280561122244489\\
-0.697394789579158	-0.265160856595838\\
-0.697653645295824	-0.264529058116232\\
-0.704157161442419	-0.248496993987976\\
-0.710589998182019	-0.232464929859719\\
-0.713426853707415	-0.225320293514677\\
-0.716948222425681	-0.216432865731463\\
-0.723231725412169	-0.200400801603207\\
-0.729445402395971	-0.18436873747495\\
-0.729458917835671	-0.184333489320016\\
-0.735581255684099	-0.168336673346694\\
-0.741646656558198	-0.152304609218437\\
-0.745490981963928	-0.142026998935649\\
-0.747640164685551	-0.13627254509018\\
-0.753558788829143	-0.120240480961924\\
-0.759407488220804	-0.104208416833667\\
-0.761523046092184	-0.0983403913275397\\
-0.765182801678965	-0.0881763527054109\\
-0.770885334358739	-0.0721442885771544\\
-0.776518228317744	-0.0561122244488979\\
-0.777555110220441	-0.0531240991403255\\
-0.782076830845542	-0.0400801603206413\\
-0.787563890959638	-0.0240480961923848\\
-0.792981374834583	-0.00801603206412826\\
-0.793587174348697	-0.00619944450666455\\
-0.798324247801242	0.00801603206412782\\
-0.803595955087895	0.0240480961923843\\
-0.80879792546851	0.0400801603206409\\
-0.809619238476954	0.0426470285581824\\
-0.81392555091599	0.0561122244488974\\
-0.818981526743509	0.0721442885771539\\
-0.823967381477785	0.0881763527054105\\
-0.82565130260521	0.0936711882852369\\
-0.828879743236386	0.104208416833667\\
-0.833719109470425	0.120240480961924\\
-0.838487745757591	0.13627254509018\\
-0.841683366733467	0.147181074946638\\
-0.843184328639133	0.152304609218437\\
-0.847805704581895	0.168336673346693\\
-0.852355515131915	0.18436873747495\\
-0.856834199960817	0.200400801603206\\
-0.857715430861723	0.20361029374942\\
-0.861236799579989	0.216432865731463\\
-0.865565666819784	0.232464929859719\\
-0.869822225003994	0.248496993987976\\
-0.87374749498998	0.263537632964162\\
-0.874006350706646	0.264529058116232\\
-0.878111639002643	0.280561122244489\\
-0.8821431997404	0.296593186372745\\
-0.88610118003164	0.312625250501002\\
-0.889779559118236	0.327808571543111\\
-0.889985305387594	0.328657314629258\\
-0.893788465297644	0.344689378757515\\
-0.897516260741634	0.360721442885771\\
-0.901168657328444	0.376753507014028\\
-0.904745563887746	0.392785571142285\\
-0.905811623246493	0.397677206826187\\
-0.908241167134204	0.408817635270541\\
-0.911657017285245	0.424849699398798\\
-0.914995079503241	0.440881763527054\\
-0.918255068412237	0.456913827655311\\
-0.921436637000515	0.472945891783567\\
-0.92184368737475	0.475055777768773\\
-0.924531625014912	0.488977955911824\\
-0.927544778940481	0.50501002004008\\
-0.930476665975421	0.521042084168337\\
-0.933326728992572	0.537074148296593\\
-0.936094343103798	0.55310621242485\\
-0.937875751503006	0.563760210885367\\
-0.938775714503375	0.569138276553106\\
-0.941366364051952	0.585170340681363\\
-0.943871126631541	0.601202404809619\\
-0.9462891477492	0.617234468937876\\
-0.94861949758841	0.633266533066132\\
-0.950861168884856	0.649298597194389\\
-0.953013074674022	0.665330661322646\\
-0.953907815631263	0.672311328150798\\
-0.955068713952377	0.681362725450902\\
-0.957027584889908	0.697394789579159\\
-0.958892252810273	0.713426853707415\\
-0.960661363087028	0.729458917835672\\
-0.962333471165166	0.745490981963928\\
-0.963907039627009	0.761523046092185\\
-0.965380435091499	0.777555110220441\\
-0.966751924938348	0.793587174348698\\
-0.968019673847939	0.809619238476954\\
-0.969181740147288	0.825651302605211\\
-0.969939879759519	0.837201437789848\\
-0.970234100232231	0.841683366733467\\
-0.971171862150368	0.857715430861724\\
-0.971997666920663	0.87374749498998\\
-0.97270922251854	0.889779559118236\\
-0.973304114743161	0.905811623246493\\
-0.973779802541406	0.921843687374749\\
-0.974133613074036	0.937875751503006\\
-0.974362736509593	0.953907815631262\\
-0.974464220530582	0.969939879759519\\
-0.974434964535466	0.985971943887775\\
-0.97427171351883	1.00200400801603\\
-0.973971051610933	1.01803607214429\\
-0.973529395256494	1.03406813627254\\
-0.972942986011204	1.0501002004008\\
-0.97220788293294	1.06613226452906\\
-0.971319954542998	1.08216432865731\\
-0.970274870330964	1.09819639278557\\
-0.969939879759519	1.10267832172919\\
}--cycle;


\addplot[area legend,solid,fill=mycolor9,draw=black,forget plot]
table[row sep=crcr] {%
x	y\\
-0.82565130260521	0.866245169292337\\
-0.825462060494201	0.87374749498998\\
-0.824887980644687	0.889779559118236\\
-0.824136376831622	0.905811623246493\\
-0.823201350070427	0.921843687374749\\
-0.822076686092449	0.937875751503006\\
-0.820755838230285	0.953907815631262\\
-0.819231909149522	0.969939879759519\\
-0.817497631338341	0.985971943887775\\
-0.815545346258589	1.00200400801603\\
-0.813366982053292	1.01803607214429\\
-0.810954029695988	1.03406813627254\\
-0.809619238476954	1.042176361739\\
-0.80826563518729	1.0501002004008\\
-0.805283581699165	1.06613226452906\\
-0.802028601208808	1.08216432865731\\
-0.79848934370967	1.09819639278557\\
-0.794653812367824	1.11422845691383\\
-0.793587174348697	1.11840086240547\\
-0.79042052310809	1.13026052104208\\
-0.785819903998319	1.14629258517034\\
-0.780866613022514	1.1623246492986\\
-0.777555110220441	1.17236205496192\\
-0.775478014964858	1.17835671342685\\
-0.769575432731745	1.19438877755511\\
-0.763245346289689	1.21042084168337\\
-0.761523046092184	1.21454979020275\\
-0.75627569446411	1.22645290581162\\
-0.748740773012284	1.24248496993988\\
-0.745490981963928	1.24902269292138\\
-0.7404736102047	1.25851703406814\\
-0.731474334627966	1.27454909819639\\
-0.729458917835671	1.27796128310748\\
-0.721480456734734	1.29058116232465\\
-0.713426853707415	1.30259251812186\\
-0.710523412722237	1.30661322645291\\
-0.698314123476152	1.32264529058116\\
-0.697394789579158	1.32379662789258\\
-0.684481340561341	1.33867735470942\\
-0.681362725450902	1.34209568133149\\
-0.66874898794471	1.35470941883768\\
-0.665330661322646	1.35797099976621\\
-0.650511183447168	1.37074148296593\\
-0.649298597194389	1.37174138065302\\
-0.633266533066132	1.38361953634077\\
-0.62845300893667	1.38677354709419\\
-0.617234468937876	1.39383666296571\\
-0.601202404809619	1.4025892937035\\
-0.600745994797927	1.40280561122244\\
-0.585170340681363	1.40992573720582\\
-0.569138276553106	1.4160452009309\\
-0.560232538149745	1.4188376753507\\
-0.55310621242485	1.42100045734942\\
-0.537074148296593	1.42486816501932\\
-0.521042084168337	1.42774204570758\\
-0.50501002004008	1.42967718233409\\
-0.488977955911824	1.43072393472941\\
-0.472945891783567	1.43092834074326\\
-0.456913827655311	1.43033247579527\\
-0.440881763527054	1.42897477560459\\
-0.424849699398798	1.42689032622384\\
-0.408817635270541	1.42411112497851\\
-0.392785571142285	1.42066631546198\\
-0.38565924541739	1.4188376753507\\
-0.376753507014028	1.41656648519904\\
-0.360721442885771	1.41183688755087\\
-0.344689378757515	1.40651358659104\\
-0.334651973094187	1.40280561122244\\
-0.328657314629258	1.40060301102082\\
-0.312625250501002	1.39411330148802\\
-0.296593186372745	1.38708617343689\\
-0.295935787071416	1.38677354709419\\
-0.280561122244489	1.37949587086625\\
-0.264529058116232	1.37139710129056\\
-0.263316471863453	1.37074148296593\\
-0.248496993987976	1.36276180547724\\
-0.234356124466997	1.35470941883768\\
-0.232464929859719	1.35363663963936\\
-0.216432865731463	1.3439988072933\\
-0.208016305591172	1.33867735470942\\
-0.200400801603207	1.33387843940809\\
-0.18436873747495	1.32328265171227\\
-0.183449403577956	1.32264529058116\\
-0.168336673346694	1.31219743040739\\
-0.160593790422456	1.30661322645291\\
-0.152304609218437	1.30065037543827\\
-0.138863158302592	1.29058116232465\\
-0.13627254509018	1.28864497519273\\
-0.120240480961924	1.27617679261093\\
-0.118225064169629	1.27454909819639\\
-0.104208416833667	1.26324929788337\\
-0.0985418869905856	1.25851703406814\\
-0.0881763527054109	1.24987385108974\\
-0.0796107483583451	1.24248496993988\\
-0.0721442885771544	1.23605250728087\\
-0.0613595760769719	1.22645290581162\\
-0.0561122244488979	1.22178697497293\\
-0.043725344827961	1.21042084168337\\
-0.0400801603206413	1.20707861834823\\
-0.0266527533644603	1.19438877755511\\
-0.0240480961923848	1.19192845901123\\
-0.0100931273197109	1.17835671342685\\
-0.00801603206412826	1.17633717730267\\
0.00599649593994431	1.1623246492986\\
0.00801603206412782	1.16030511317441\\
0.0216541346291227	1.14629258517034\\
0.0240480961923843	1.14383226662646\\
0.0369135090800333	1.13026052104208\\
0.0400801603206409	1.12691829770695\\
0.0518045890419841	1.11422845691383\\
0.0561122244488974	1.10956252607513\\
0.0663540590020255	1.09819639278557\\
0.0721442885771539	1.09176393012657\\
0.0805857154372647	1.08216432865731\\
0.0881763527054105	1.07352114567892\\
0.0945208072747595	1.06613226452906\\
0.104208416833667	1.05483246421604\\
0.108178328748497	1.0501002004008\\
0.120240480961924	1.03569583068708\\
0.121575272180958	1.03406813627254\\
0.134692032149139	1.01803607214429\\
0.13627254509018	1.01609988501237\\
0.147546506819147	1.00200400801603\\
0.152304609218437	0.9960411570014\\
0.160183002079825	0.985971943887775\\
0.168336673346693	0.975524083714005\\
0.172612977213375	0.969939879759519\\
0.18436873747495	0.954545176762373\\
0.184846659229807	0.953907815631262\\
0.196826185090445	0.937875751503006\\
0.200400801603206	0.933076836201674\\
0.208621100028441	0.921843687374749\\
0.216432865731463	0.91113307583037\\
0.220248467550492	0.905811623246493\\
0.231701607899197	0.889779559118236\\
0.232464929859719	0.888706779919921\\
0.242934002994659	0.87374749498998\\
0.248496993987976	0.865767817501285\\
0.254022025382232	0.857715430861724\\
0.264529058116232	0.842338985058091\\
0.264970265322122	0.841683366733467\\
0.275708137982974	0.825651302605211\\
0.280561122244489	0.818373626377271\\
0.286312909771483	0.809619238476954\\
0.296593186372745	0.793899800691397\\
0.296794755951398	0.793587174348698\\
0.307077408297698	0.777555110220441\\
0.312625250501002	0.768862800486012\\
0.317245772034545	0.761523046092185\\
0.327285519079521	0.745490981963928\\
0.328657314629258	0.743288381762303\\
0.337156300817286	0.729458917835672\\
0.344689378757515	0.717134829076012\\
0.346927181285696	0.713426853707415\\
0.356545602273577	0.697394789579159\\
0.360721442885771	0.690394001779324\\
0.366042486545644	0.681362725450902\\
0.375431709792827	0.665330661322646\\
0.376753507014028	0.663059471170983\\
0.384667400047787	0.649298597194389\\
0.392785571142285	0.635095173177409\\
0.393818986979759	0.633266533066132\\
0.402817842545186	0.617234468937876\\
0.408817635270541	0.60647585443743\\
0.411726576130805	0.601202404809619\\
0.42050868119499	0.585170340681363\\
0.424849699398798	0.577190927426249\\
0.429184778407611	0.569138276553106\\
0.437753779839534	0.55310621242485\\
0.440881763527054	0.547211248550484\\
0.446206654881929	0.537074148296593\\
0.454566045756848	0.521042084168337\\
0.456913827655311	0.516504820484648\\
0.462804351611308	0.50501002004008\\
0.470957473100776	0.488977955911824\\
0.472945891783567	0.48503655717613\\
0.478989140028764	0.472945891783567\\
0.486939183233613	0.456913827655311\\
0.488977955911824	0.452768022905766\\
0.494771454425069	0.440881763527054\\
0.502521462138104	0.424849699398798\\
0.50501002004008	0.419657142253924\\
0.510160926694367	0.408817635270541\\
0.517713795080866	0.392785571142285\\
0.521042084168337	0.385657877370906\\
0.525166418500278	0.376753507014028\\
0.532524898685601	0.360721442885771\\
0.537074148296593	0.350719868426134\\
0.539796051007148	0.344689378757515\\
0.546962750561728	0.328657314629258\\
0.55310621242485	0.314788032499723\\
0.554057232309417	0.312625250501002\\
0.561034616622361	0.296593186372745\\
0.567946882818409	0.280561122244489\\
0.569138276553106	0.277768647824683\\
0.574747076214539	0.264529058116232\\
0.581471212238471	0.248496993987976\\
0.585170340681363	0.23958505584309\\
0.588106045174488	0.232464929859719\\
0.594644196129463	0.216432865731463\\
0.601116114696609	0.200400801603206\\
0.601202404809619	0.200184484084263\\
0.607470969037548	0.18436873747495\\
0.613757593813339	0.168336673346693\\
0.617234468937876	0.159367725089954\\
0.619956043021835	0.152304609218437\\
0.626059204199459	0.13627254509018\\
0.632094429226009	0.120240480961924\\
0.633266533066132	0.117086470208509\\
0.63802471474842	0.104208416833667\\
0.643876682886971	0.0881763527054105\\
0.649298597194389	0.0731441862642451\\
0.649657305025131	0.0721442885771539\\
0.655327533694953	0.0561122244488974\\
0.660927542575827	0.0400801603206409\\
0.665330661322646	0.0273096771209221\\
0.666449447450091	0.0240480961923843\\
0.671867442661018	0.00801603206412782\\
0.677214141975015	-0.00801603206412826\\
0.681362725450902	-0.0206297695703205\\
0.682481511578348	-0.0240480961923848\\
0.687645617002839	-0.0400801603206413\\
0.692737070130544	-0.0561122244488979\\
0.697394789579159	-0.0709929512657397\\
0.6977534974099	-0.0721442885771544\\
0.702661469168185	-0.0881763527054109\\
0.707495148697988	-0.104208416833667\\
0.712254749867292	-0.120240480961924\\
0.713426853707415	-0.124261189292971\\
0.7169126464296	-0.13627254509018\\
0.721485427396051	-0.152304609218437\\
0.725982042711135	-0.168336673346694\\
0.729458917835672	-0.180956552563859\\
0.730395021803703	-0.18436873747495\\
0.734703172616581	-0.200400801603207\\
0.738932773283772	-0.216432865731463\\
0.743083645568415	-0.232464929859719\\
0.745490981963928	-0.241959271006479\\
0.747141849129499	-0.248496993987976\\
0.751099781625361	-0.264529058116232\\
0.754976124731029	-0.280561122244489\\
0.758770453206528	-0.296593186372745\\
0.761523046092185	-0.308496301981619\\
0.762474065976751	-0.312625250501002\\
0.766070269536165	-0.328657314629258\\
0.769581097463983	-0.344689378757515\\
0.773005860609449	-0.360721442885771\\
0.776343786249278	-0.376753507014028\\
0.777555110220441	-0.382748165478959\\
0.779575252649355	-0.392785571142285\\
0.782706016874727	-0.408817635270541\\
0.785745855233877	-0.424849699398798\\
0.788693703292883	-0.440881763527054\\
0.791548401670487	-0.456913827655311\\
0.793587174348698	-0.468773486291924\\
0.794301571550162	-0.472945891783567\\
0.796938650811452	-0.488977955911824\\
0.799477698304696	-0.50501002004008\\
0.801917229327064	-0.521042084168337\\
0.804255650319159	-0.537074148296593\\
0.806491254789434	-0.55310621242485\\
0.808622218986226	-0.569138276553106\\
0.809619238476954	-0.577062115214904\\
0.810635043483407	-0.585170340681363\\
0.812528179337217	-0.601202404809619\\
0.814310405770573	-0.617234468937876\\
0.815979507096244	-0.633266533066132\\
0.817533131510713	-0.649298597194389\\
0.818968785424053	-0.665330661322646\\
0.820283827444078	-0.681362725450902\\
0.821475461993017	-0.697394789579158\\
0.822540732533354	-0.713426853707415\\
0.82347651437768	-0.729458917835671\\
0.824279507055472	-0.745490981963928\\
0.824946226207663	-0.761523046092184\\
0.825472994977549	-0.777555110220441\\
0.825651302605211	-0.785057435918098\\
0.825852872183863	-0.793587174348697\\
0.826084201198702	-0.809619238476954\\
0.826165509867008	-0.82565130260521\\
0.826092509811099	-0.841683366733467\\
0.825860684610487	-0.857715430861723\\
0.825651302605211	-0.866245169292336\\
0.825462060494201	-0.87374749498998\\
0.824887980644688	-0.889779559118236\\
0.824136376831622	-0.905811623246493\\
0.823201350070428	-0.92184368737475\\
0.822076686092449	-0.937875751503006\\
0.820755838230285	-0.953907815631263\\
0.819231909149522	-0.969939879759519\\
0.817497631338341	-0.985971943887776\\
0.815545346258589	-1.00200400801603\\
0.813366982053292	-1.01803607214429\\
0.810954029695989	-1.03406813627255\\
0.809619238476954	-1.042176361739\\
0.80826563518729	-1.0501002004008\\
0.805283581699165	-1.06613226452906\\
0.802028601208808	-1.08216432865731\\
0.79848934370967	-1.09819639278557\\
0.794653812367823	-1.11422845691383\\
0.793587174348698	-1.11840086240547\\
0.790420523108089	-1.13026052104208\\
0.785819903998319	-1.14629258517034\\
0.780866613022514	-1.1623246492986\\
0.777555110220441	-1.17236205496192\\
0.775478014964857	-1.17835671342685\\
0.769575432731745	-1.19438877755511\\
0.763245346289689	-1.21042084168337\\
0.761523046092185	-1.21454979020275\\
0.75627569446411	-1.22645290581162\\
0.748740773012284	-1.24248496993988\\
0.745490981963928	-1.24902269292138\\
0.7404736102047	-1.25851703406814\\
0.731474334627967	-1.27454909819639\\
0.729458917835672	-1.27796128310748\\
0.721480456734734	-1.29058116232465\\
0.713426853707415	-1.30259251812186\\
0.710523412722236	-1.30661322645291\\
0.698314123476151	-1.32264529058116\\
0.697394789579159	-1.32379662789258\\
0.68448134056134	-1.33867735470942\\
0.681362725450902	-1.34209568133148\\
0.66874898794471	-1.35470941883768\\
0.665330661322646	-1.35797099976621\\
0.650511183447168	-1.37074148296593\\
0.649298597194389	-1.37174138065302\\
0.633266533066132	-1.38361953634077\\
0.62845300893667	-1.38677354709419\\
0.617234468937876	-1.39383666296571\\
0.601202404809619	-1.4025892937035\\
0.600745994797927	-1.40280561122244\\
0.585170340681363	-1.40992573720582\\
0.569138276553106	-1.4160452009309\\
0.560232538149745	-1.4188376753507\\
0.55310621242485	-1.42100045734942\\
0.537074148296593	-1.42486816501932\\
0.521042084168337	-1.42774204570758\\
0.50501002004008	-1.42967718233408\\
0.488977955911824	-1.43072393472941\\
0.472945891783567	-1.43092834074326\\
0.456913827655311	-1.43033247579527\\
0.440881763527054	-1.42897477560459\\
0.424849699398798	-1.42689032622384\\
0.408817635270541	-1.42411112497851\\
0.392785571142285	-1.42066631546198\\
0.385659245417389	-1.4188376753507\\
0.376753507014028	-1.41656648519904\\
0.360721442885771	-1.41183688755087\\
0.344689378757515	-1.40651358659104\\
0.334651973094188	-1.40280561122244\\
0.328657314629258	-1.40060301102082\\
0.312625250501002	-1.39411330148802\\
0.296593186372745	-1.38708617343689\\
0.295935787071416	-1.38677354709419\\
0.280561122244489	-1.37949587086625\\
0.264529058116232	-1.37139710129056\\
0.263316471863453	-1.37074148296593\\
0.248496993987976	-1.36276180547724\\
0.234356124466996	-1.35470941883768\\
0.232464929859719	-1.35363663963936\\
0.216432865731463	-1.3439988072933\\
0.208016305591172	-1.33867735470942\\
0.200400801603206	-1.33387843940809\\
0.18436873747495	-1.32328265171227\\
0.183449403577957	-1.32264529058116\\
0.168336673346693	-1.31219743040739\\
0.160593790422456	-1.30661322645291\\
0.152304609218437	-1.30065037543827\\
0.138863158302593	-1.29058116232465\\
0.13627254509018	-1.28864497519273\\
0.120240480961924	-1.27617679261093\\
0.118225064169628	-1.27454909819639\\
0.104208416833667	-1.26324929788337\\
0.098541886990586	-1.25851703406814\\
0.0881763527054105	-1.24987385108974\\
0.0796107483583451	-1.24248496993988\\
0.0721442885771539	-1.23605250728087\\
0.0613595760769714	-1.22645290581162\\
0.0561122244488974	-1.22178697497293\\
0.043725344827961	-1.21042084168337\\
0.0400801603206409	-1.20707861834823\\
0.0266527533644607	-1.19438877755511\\
0.0240480961923843	-1.19192845901123\\
0.0100931273197114	-1.17835671342685\\
0.00801603206412782	-1.17633717730267\\
-0.00599649593994475	-1.1623246492986\\
-0.00801603206412826	-1.16030511317441\\
-0.0216541346291227	-1.14629258517034\\
-0.0240480961923848	-1.14383226662646\\
-0.0369135090800337	-1.13026052104208\\
-0.0400801603206413	-1.12691829770695\\
-0.0518045890419846	-1.11422845691383\\
-0.0561122244488979	-1.10956252607513\\
-0.0663540590020255	-1.09819639278557\\
-0.0721442885771544	-1.09176393012657\\
-0.0805857154372647	-1.08216432865731\\
-0.0881763527054109	-1.07352114567892\\
-0.0945208072747585	-1.06613226452906\\
-0.104208416833667	-1.05483246421604\\
-0.108178328748496	-1.0501002004008\\
-0.120240480961924	-1.03569583068708\\
-0.121575272180958	-1.03406813627255\\
-0.134692032149139	-1.01803607214429\\
-0.13627254509018	-1.01609988501237\\
-0.147546506819146	-1.00200400801603\\
-0.152304609218437	-0.996041157001399\\
-0.160183002079824	-0.985971943887776\\
-0.168336673346694	-0.975524083714005\\
-0.172612977213376	-0.969939879759519\\
-0.18436873747495	-0.954545176762373\\
-0.184846659229807	-0.953907815631263\\
-0.196826185090445	-0.937875751503006\\
-0.200400801603207	-0.933076836201674\\
-0.208621100028441	-0.92184368737475\\
-0.216432865731463	-0.91113307583037\\
-0.220248467550491	-0.905811623246493\\
-0.231701607899196	-0.889779559118236\\
-0.232464929859719	-0.888706779919921\\
-0.242934002994659	-0.87374749498998\\
-0.248496993987976	-0.865767817501284\\
-0.254022025382232	-0.857715430861723\\
-0.264529058116232	-0.842338985058091\\
-0.264970265322122	-0.841683366733467\\
-0.275708137982974	-0.82565130260521\\
-0.280561122244489	-0.818373626377271\\
-0.286312909771483	-0.809619238476954\\
-0.296593186372745	-0.793899800691397\\
-0.296794755951398	-0.793587174348697\\
-0.307077408297698	-0.777555110220441\\
-0.312625250501002	-0.768862800486012\\
-0.317245772034545	-0.761523046092184\\
-0.327285519079521	-0.745490981963928\\
-0.328657314629258	-0.743288381762303\\
-0.337156300817286	-0.729458917835671\\
-0.344689378757515	-0.717134829076012\\
-0.346927181285696	-0.713426853707415\\
-0.356545602273577	-0.697394789579158\\
-0.360721442885771	-0.690394001779324\\
-0.366042486545644	-0.681362725450902\\
-0.375431709792827	-0.665330661322646\\
-0.376753507014028	-0.663059471170983\\
-0.384667400047787	-0.649298597194389\\
-0.392785571142285	-0.635095173177409\\
-0.393818986979759	-0.633266533066132\\
-0.402817842545187	-0.617234468937876\\
-0.408817635270541	-0.60647585443743\\
-0.411726576130805	-0.601202404809619\\
-0.42050868119499	-0.585170340681363\\
-0.424849699398798	-0.577190927426249\\
-0.429184778407611	-0.569138276553106\\
-0.437753779839535	-0.55310621242485\\
-0.440881763527054	-0.547211248550484\\
-0.446206654881929	-0.537074148296593\\
-0.454566045756848	-0.521042084168337\\
-0.456913827655311	-0.516504820484648\\
-0.462804351611308	-0.50501002004008\\
-0.470957473100776	-0.488977955911824\\
-0.472945891783567	-0.485036557176129\\
-0.478989140028764	-0.472945891783567\\
-0.486939183233613	-0.456913827655311\\
-0.488977955911824	-0.452768022905766\\
-0.494771454425069	-0.440881763527054\\
-0.502521462138104	-0.424849699398798\\
-0.50501002004008	-0.419657142253924\\
-0.510160926694366	-0.408817635270541\\
-0.517713795080866	-0.392785571142285\\
-0.521042084168337	-0.385657877370906\\
-0.525166418500279	-0.376753507014028\\
-0.532524898685601	-0.360721442885771\\
-0.537074148296593	-0.350719868426134\\
-0.539796051007148	-0.344689378757515\\
-0.546962750561728	-0.328657314629258\\
-0.55310621242485	-0.314788032499723\\
-0.554057232309416	-0.312625250501002\\
-0.561034616622361	-0.296593186372745\\
-0.56794688281841	-0.280561122244489\\
-0.569138276553106	-0.277768647824683\\
-0.574747076214539	-0.264529058116232\\
-0.581471212238471	-0.248496993987976\\
-0.585170340681363	-0.23958505584309\\
-0.588106045174488	-0.232464929859719\\
-0.594644196129463	-0.216432865731463\\
-0.601116114696608	-0.200400801603207\\
-0.601202404809619	-0.200184484084262\\
-0.607470969037547	-0.18436873747495\\
-0.613757593813339	-0.168336673346694\\
-0.617234468937876	-0.159367725089954\\
-0.619956043021835	-0.152304609218437\\
-0.626059204199459	-0.13627254509018\\
-0.632094429226009	-0.120240480961924\\
-0.633266533066132	-0.117086470208509\\
-0.638024714748419	-0.104208416833667\\
-0.643876682886971	-0.0881763527054109\\
-0.649298597194389	-0.0731441862642456\\
-0.649657305025131	-0.0721442885771544\\
-0.655327533694952	-0.0561122244488979\\
-0.660927542575827	-0.0400801603206413\\
-0.665330661322646	-0.0273096771209217\\
-0.666449447450091	-0.0240480961923848\\
-0.671867442661018	-0.00801603206412826\\
-0.677214141975015	0.00801603206412782\\
-0.681362725450902	0.0206297695703186\\
-0.682481511578348	0.0240480961923843\\
-0.687645617002839	0.0400801603206409\\
-0.692737070130543	0.0561122244488974\\
-0.697394789579158	0.0709929512657383\\
-0.6977534974099	0.0721442885771539\\
-0.702661469168184	0.0881763527054105\\
-0.707495148697989	0.104208416833667\\
-0.712254749867292	0.120240480961924\\
-0.713426853707415	0.124261189292968\\
-0.7169126464296	0.13627254509018\\
-0.72148542739605	0.152304609218437\\
-0.725982042711135	0.168336673346693\\
-0.729458917835671	0.180956552563858\\
-0.730395021803702	0.18436873747495\\
-0.734703172616582	0.200400801603206\\
-0.738932773283771	0.216432865731463\\
-0.743083645568415	0.232464929859719\\
-0.745490981963928	0.241959271006478\\
-0.747141849129499	0.248496993987976\\
-0.75109978162536	0.264529058116232\\
-0.754976124731029	0.280561122244489\\
-0.758770453206528	0.296593186372745\\
-0.761523046092184	0.308496301981619\\
-0.762474065976751	0.312625250501002\\
-0.766070269536166	0.328657314629258\\
-0.769581097463983	0.344689378757515\\
-0.773005860609449	0.360721442885771\\
-0.776343786249279	0.376753507014028\\
-0.777555110220441	0.382748165478957\\
-0.779575252649355	0.392785571142285\\
-0.782706016874727	0.408817635270541\\
-0.785745855233877	0.424849699398798\\
-0.788693703292883	0.440881763527054\\
-0.791548401670487	0.456913827655311\\
-0.793587174348697	0.468773486291922\\
-0.794301571550162	0.472945891783567\\
-0.796938650811452	0.488977955911824\\
-0.799477698304695	0.50501002004008\\
-0.801917229327064	0.521042084168337\\
-0.804255650319159	0.537074148296593\\
-0.806491254789434	0.55310621242485\\
-0.808622218986225	0.569138276553106\\
-0.809619238476954	0.577062115214904\\
-0.810635043483407	0.585170340681363\\
-0.812528179337218	0.601202404809619\\
-0.814310405770573	0.617234468937876\\
-0.815979507096244	0.633266533066132\\
-0.817533131510713	0.649298597194389\\
-0.818968785424053	0.665330661322646\\
-0.820283827444077	0.681362725450902\\
-0.821475461993016	0.697394789579159\\
-0.822540732533354	0.713426853707415\\
-0.82347651437768	0.729458917835672\\
-0.824279507055473	0.745490981963928\\
-0.824946226207662	0.761523046092185\\
-0.825472994977549	0.777555110220441\\
-0.82565130260521	0.785057435918085\\
-0.825852872183862	0.793587174348698\\
-0.826084201198703	0.809619238476954\\
-0.826165509867008	0.825651302605211\\
-0.8260925098111	0.841683366733467\\
-0.825860684610486	0.857715430861724\\
-0.82565130260521	0.866245169292337\\
}--cycle;


\addplot[area legend,solid,fill=mycolor10,draw=black,forget plot]
table[row sep=crcr] {%
x	y\\
-0.665330661322646	0.758105025403793\\
-0.665082784862698	0.761523046092185\\
-0.663686564675468	0.777555110220441\\
-0.662033485917416	0.793587174348698\\
-0.660112238187751	0.809619238476954\\
-0.657910790516671	0.825651302605211\\
-0.655416339751623	0.841683366733467\\
-0.652615254428646	0.857715430861724\\
-0.649493013666892	0.87374749498998\\
-0.649298597194389	0.87466276666894\\
-0.645911158536872	0.889779559118236\\
-0.641949453571151	0.905811623246493\\
-0.637595573522108	0.921843687374749\\
-0.633266533066132	0.936427568599341\\
-0.632809737264636	0.937875751503006\\
-0.62737690649475	0.953907815631262\\
-0.621452974746538	0.969939879759519\\
-0.617234468937876	0.980509918222073\\
-0.614898112565123	0.985971943887775\\
-0.607542947700778	1.00200400801603\\
-0.601202404809619	1.0147880599505\\
-0.599461200225299	1.01803607214429\\
-0.590292763112442	1.03406813627254\\
-0.585170340681363	1.04241611958894\\
-0.58002212340773	1.0501002004008\\
-0.569138276553106	1.06525453592202\\
-0.568443290727499	1.06613226452906\\
-0.55496007841122	1.08216432865731\\
-0.55310621242485	1.08423901886186\\
-0.539148838501137	1.09819639278557\\
-0.537074148296593	1.10015445046976\\
-0.521042084168337	1.11349529538232\\
-0.520031608379621	1.11422845691383\\
-0.50501002004008	1.1245759618867\\
-0.495273095705868	1.13026052104208\\
-0.488977955911824	1.13376186558903\\
-0.472945891783567	1.14121295074715\\
-0.459322825664768	1.14629258517034\\
-0.456913827655311	1.1471525085698\\
-0.440881763527054	1.15164454166159\\
-0.424849699398798	1.1548759669644\\
-0.408817635270541	1.15693170935783\\
-0.392785571142285	1.15788836523612\\
-0.376753507014028	1.15781506554153\\
-0.360721442885771	1.15677423201626\\
-0.344689378757515	1.15482224148008\\
-0.328657314629258	1.15201001063494\\
-0.312625250501002	1.14838351198749\\
-0.305073103182926	1.14629258517034\\
-0.296593186372745	1.14395622879759\\
-0.280561122244489	1.13876287781614\\
-0.264529058116232	1.13286477697221\\
-0.258233918322189	1.13026052104208\\
-0.248496993987976	1.12624895384855\\
-0.232464929859719	1.11895667271266\\
-0.222939668618048	1.11422845691383\\
-0.216432865731463	1.11101016607699\\
-0.200400801603207	1.10241489859423\\
-0.193078381798149	1.09819639278557\\
-0.18436873747495	1.09319414797708\\
-0.168336673346694	1.0833749248919\\
-0.166482807360323	1.08216432865731\\
-0.152304609218437	1.07292956533563\\
-0.142421230303577	1.06613226452906\\
-0.13627254509018	1.06191327279785\\
-0.120240480961924	1.05032770442359\\
-0.119941671271634	1.0501002004008\\
-0.104208416833667	1.03814294264968\\
-0.0990859944025878	1.03406813627254\\
-0.0881763527054109	1.02540306136772\\
-0.0792960980873046	1.01803607214429\\
-0.0721442885771544	1.0121106072761\\
-0.0604426536847802	1.00200400801603\\
-0.0561122244488979	0.998267697634059\\
-0.0424165166933938	0.985971943887775\\
-0.0400801603206413	0.983876022989052\\
-0.0251250206077705	0.969939879759519\\
-0.0240480961923848	0.968936849352776\\
-0.00848905848976323	0.953907815631262\\
-0.00801603206412826	0.953451019829765\\
0.007559236262631	0.937875751503006\\
0.00801603206412782	0.937418955701508\\
0.023078348608166	0.921843687374749\\
0.0240480961923843	0.920840656968006\\
0.038118883178968	0.905811623246493\\
0.0400801603206409	0.903715702347771\\
0.052724785791382	0.889779559118236\\
0.0561122244488974	0.886043248736265\\
0.0669343440570812	0.87374749498998\\
0.0721442885771539	0.867822030121793\\
0.0807810122405978	0.857715430861724\\
0.0881763527054105	0.849050355956901\\
0.0942940952626445	0.841683366733467\\
0.104208416833667	0.829726108982349\\
0.107499319045926	0.825651302605211\\
0.120240480961924	0.809846742499742\\
0.12041930855841	0.809619238476954\\
0.132975369684951	0.793587174348698\\
0.13627254509018	0.789368182617485\\
0.145279652829894	0.777555110220441\\
0.152304609218437	0.768320346898754\\
0.157354243807146	0.761523046092185\\
0.168336673346693	0.746701578198513\\
0.169213498649643	0.745490981963928\\
0.180773489041761	0.729458917835672\\
0.18436873747495	0.724456673027176\\
0.192122093660993	0.713426853707415\\
0.200400801603206	0.70161329538782\\
0.203294662397068	0.697394789579159\\
0.214244932658426	0.681362725450902\\
0.216432865731463	0.67814443461406\\
0.224965119394612	0.665330661322646\\
0.232464929859719	0.654026812993221\\
0.235540035818519	0.649298597194389\\
0.245914642067158	0.633266533066132\\
0.248496993987976	0.629254965872603\\
0.256086167319485	0.617234468937876\\
0.264529058116232	0.603806660739741\\
0.266136311158703	0.601202404809619\\
0.275967306441132	0.585170340681363\\
0.280561122244489	0.577640633327164\\
0.285655289054162	0.569138276553106\\
0.295210419473976	0.55310621242485\\
0.296593186372745	0.550769856052097\\
0.30455760395013	0.537074148296593\\
0.312625250501002	0.523133010985484\\
0.313815042172045	0.521042084168337\\
0.322866732980162	0.50501002004008\\
0.328657314629258	0.494695381376427\\
0.331814981981456	0.488977955911824\\
0.340604525579819	0.472945891783567\\
0.344689378757515	0.465443483965051\\
0.349260440368795	0.456913827655311\\
0.357791295522285	0.440881763527054\\
0.360721442885771	0.435331346244714\\
0.366170395731711	0.424849699398798\\
0.374445905957461	0.408817635270541\\
0.376753507014028	0.404308051513471\\
0.382562459543581	0.392785571142285\\
0.39058584797749	0.376753507014028\\
0.392785571142285	0.372317222951554\\
0.398452948292133	0.360721442885771\\
0.406227313185292	0.344689378757515\\
0.408817635270541	0.339296438816749\\
0.413856949942716	0.328657314629258\\
0.421385260700624	0.312625250501002\\
0.424849699398798	0.305176568166802\\
0.428788385313387	0.296593186372745\\
0.436073478999931	0.280561122244489\\
0.440881763527054	0.269881014607484\\
0.443260064715237	0.264529058116232\\
0.450304642951515	0.248496993987976\\
0.456913827655311	0.233324853259175\\
0.457283740180685	0.232464929859719\\
0.464090366375889	0.216432865731463\\
0.470835143423245	0.200400801603206\\
0.472945891783567	0.195321167180018\\
0.477441250432316	0.18436873747495\\
0.483947723367293	0.168336673346693\\
0.488977955911824	0.155805953765381\\
0.490366928106163	0.152304609218437\\
0.496637367872314	0.13627254509018\\
0.502841024354953	0.120240480961924\\
0.50501002004008	0.114555921806544\\
0.508912636015967	0.104208416833667\\
0.514879435468517	0.0881763527054105\\
0.520776905952092	0.0721442885771539\\
0.521042084168337	0.0714111270456462\\
0.526513104852846	0.0561122244488974\\
0.532172574628555	0.0400801603206409\\
0.537074148296593	0.0260061538765745\\
0.537748689204562	0.0240480961923843\\
0.54317196671062	0.00801603206412782\\
0.5485197049853	-0.00801603206412826\\
0.55310621242485	-0.0219734059878415\\
0.553780753332819	-0.0240480961923848\\
0.558890672250114	-0.0400801603206413\\
0.56392155566562	-0.0561122244488979\\
0.568873098336861	-0.0721442885771544\\
0.569138276553106	-0.0730220171841901\\
0.573668496296285	-0.0881763527054109\\
0.57837666810335	-0.104208416833667\\
0.583001344996236	-0.120240480961924\\
0.585170340681363	-0.127924561773787\\
0.587502535040489	-0.13627254509018\\
0.591881390411667	-0.152304609218437\\
0.596172172265089	-0.168336673346694\\
0.60037395363977	-0.18436873747495\\
0.601202404809619	-0.187616749668742\\
0.60442986328447	-0.200400801603207\\
0.608378943530198	-0.216432865731463\\
0.612233727171058	-0.232464929859719\\
0.615992884122811	-0.248496993987976\\
0.617234468937876	-0.25395901965368\\
0.619612770126059	-0.264529058116232\\
0.623111489402566	-0.280561122244489\\
0.626508521446038	-0.296593186372745\\
0.629802094367958	-0.312625250501002\\
0.632990291864623	-0.328657314629258\\
0.633266533066132	-0.330105497532926\\
0.6360197366617	-0.344689378757515\\
0.638933910215981	-0.360721442885771\\
0.641735488262777	-0.376753507014028\\
0.644422048509105	-0.392785571142285\\
0.646990996137822	-0.408817635270541\\
0.649298597194389	-0.423934427719837\\
0.649436850751654	-0.424849699398798\\
0.651716318887212	-0.440881763527054\\
0.653869658805669	-0.456913827655311\\
0.655893692198952	-0.472945891783567\\
0.657785033616345	-0.488977955911824\\
0.659540079673549	-0.50501002004008\\
0.661154997500123	-0.521042084168337\\
0.662625712366185	-0.537074148296593\\
0.663947894423875	-0.55310621242485\\
0.665116944493114	-0.569138276553106\\
0.665330661322646	-0.572556297241498\\
0.666110017040159	-0.585170340681363\\
0.666937914365116	-0.601202404809619\\
0.667600197623856	-0.617234468937876\\
0.668091410176274	-0.633266533066132\\
0.668405767281445	-0.649298597194389\\
0.668537136825009	-0.665330661322646\\
0.66847901861384	-0.681362725450902\\
0.668224522116507	-0.697394789579158\\
0.6677663425161	-0.713426853707415\\
0.66709673492865	-0.729458917835671\\
0.666207486625594	-0.745490981963928\\
0.665330661322646	-0.758105025403793\\
0.665082784862698	-0.761523046092184\\
0.663686564675469	-0.777555110220441\\
0.662033485917416	-0.793587174348697\\
0.660112238187751	-0.809619238476954\\
0.657910790516671	-0.82565130260521\\
0.655416339751623	-0.841683366733467\\
0.652615254428647	-0.857715430861723\\
0.649493013666891	-0.87374749498998\\
0.649298597194389	-0.87466276666894\\
0.645911158536872	-0.889779559118236\\
0.641949453571151	-0.905811623246493\\
0.637595573522108	-0.92184368737475\\
0.633266533066132	-0.936427568599342\\
0.632809737264635	-0.937875751503006\\
0.627376906494749	-0.953907815631263\\
0.621452974746537	-0.969939879759519\\
0.617234468937876	-0.980509918222073\\
0.614898112565123	-0.985971943887776\\
0.607542947700777	-1.00200400801603\\
0.601202404809619	-1.0147880599505\\
0.599461200225299	-1.01803607214429\\
0.590292763112442	-1.03406813627255\\
0.585170340681363	-1.04241611958894\\
0.580022123407729	-1.0501002004008\\
0.569138276553106	-1.06525453592202\\
0.5684432907275	-1.06613226452906\\
0.55496007841122	-1.08216432865731\\
0.55310621242485	-1.08423901886186\\
0.539148838501136	-1.09819639278557\\
0.537074148296593	-1.10015445046976\\
0.521042084168337	-1.11349529538232\\
0.520031608379621	-1.11422845691383\\
0.50501002004008	-1.12457596188671\\
0.495273095705867	-1.13026052104208\\
0.488977955911824	-1.13376186558903\\
0.472945891783567	-1.14121295074715\\
0.459322825664769	-1.14629258517034\\
0.456913827655311	-1.1471525085698\\
0.440881763527054	-1.15164454166159\\
0.424849699398798	-1.1548759669644\\
0.408817635270541	-1.15693170935783\\
0.392785571142285	-1.15788836523612\\
0.376753507014028	-1.15781506554153\\
0.360721442885771	-1.15677423201626\\
0.344689378757515	-1.15482224148008\\
0.328657314629258	-1.15201001063494\\
0.312625250501002	-1.14838351198749\\
0.305073103182926	-1.14629258517034\\
0.296593186372745	-1.14395622879759\\
0.280561122244489	-1.13876287781614\\
0.264529058116232	-1.13286477697221\\
0.258233918322189	-1.13026052104208\\
0.248496993987976	-1.12624895384855\\
0.232464929859719	-1.11895667271266\\
0.222939668618048	-1.11422845691383\\
0.216432865731463	-1.11101016607699\\
0.200400801603206	-1.10241489859423\\
0.19307838179815	-1.09819639278557\\
0.18436873747495	-1.09319414797708\\
0.168336673346693	-1.0833749248919\\
0.166482807360324	-1.08216432865731\\
0.152304609218437	-1.07292956533563\\
0.142421230303577	-1.06613226452906\\
0.13627254509018	-1.06191327279785\\
0.120240480961924	-1.05032770442359\\
0.119941671271634	-1.0501002004008\\
0.104208416833667	-1.03814294264968\\
0.0990859944025882	-1.03406813627255\\
0.0881763527054105	-1.02540306136772\\
0.079296098087305	-1.01803607214429\\
0.0721442885771539	-1.0121106072761\\
0.0604426536847797	-1.00200400801603\\
0.0561122244488974	-0.99826769763406\\
0.0424165166933943	-0.985971943887776\\
0.0400801603206409	-0.983876022989052\\
0.0251250206077718	-0.969939879759519\\
0.0240480961923843	-0.968936849352774\\
0.00848905848976415	-0.953907815631263\\
0.00801603206412782	-0.953451019829765\\
-0.00755923626263144	-0.937875751503006\\
-0.00801603206412826	-0.937418955701509\\
-0.0230783486081656	-0.92184368737475\\
-0.0240480961923848	-0.920840656968006\\
-0.0381188831789675	-0.905811623246493\\
-0.0400801603206413	-0.90371570234777\\
-0.0527247857913811	-0.889779559118236\\
-0.0561122244488979	-0.886043248736264\\
-0.0669343440570817	-0.87374749498998\\
-0.0721442885771544	-0.867822030121793\\
-0.0807810122405983	-0.857715430861723\\
-0.0881763527054109	-0.8490503559569\\
-0.0942940952626444	-0.841683366733467\\
-0.104208416833667	-0.829726108982349\\
-0.107499319045926	-0.82565130260521\\
-0.120240480961924	-0.809846742499741\\
-0.120419308558411	-0.809619238476954\\
-0.132975369684952	-0.793587174348697\\
-0.13627254509018	-0.789368182617485\\
-0.145279652829894	-0.777555110220441\\
-0.152304609218437	-0.768320346898754\\
-0.157354243807146	-0.761523046092184\\
-0.168336673346694	-0.746701578198513\\
-0.169213498649643	-0.745490981963928\\
-0.180773489041761	-0.729458917835671\\
-0.18436873747495	-0.724456673027176\\
-0.192122093660993	-0.713426853707415\\
-0.200400801603207	-0.70161329538782\\
-0.203294662397068	-0.697394789579158\\
-0.214244932658426	-0.681362725450902\\
-0.216432865731463	-0.678144434614059\\
-0.224965119394612	-0.665330661322646\\
-0.232464929859719	-0.654026812993221\\
-0.235540035818519	-0.649298597194389\\
-0.245914642067158	-0.633266533066132\\
-0.248496993987976	-0.629254965872603\\
-0.256086167319485	-0.617234468937876\\
-0.264529058116232	-0.603806660739742\\
-0.266136311158703	-0.601202404809619\\
-0.275967306441132	-0.585170340681363\\
-0.280561122244489	-0.577640633327163\\
-0.285655289054161	-0.569138276553106\\
-0.295210419473976	-0.55310621242485\\
-0.296593186372745	-0.550769856052097\\
-0.30455760395013	-0.537074148296593\\
-0.312625250501002	-0.523133010985485\\
-0.313815042172045	-0.521042084168337\\
-0.322866732980163	-0.50501002004008\\
-0.328657314629258	-0.494695381376427\\
-0.331814981981456	-0.488977955911824\\
-0.34060452557982	-0.472945891783567\\
-0.344689378757515	-0.465443483965052\\
-0.349260440368795	-0.456913827655311\\
-0.357791295522285	-0.440881763527054\\
-0.360721442885771	-0.435331346244714\\
-0.366170395731711	-0.424849699398798\\
-0.374445905957461	-0.408817635270541\\
-0.376753507014028	-0.404308051513471\\
-0.382562459543581	-0.392785571142285\\
-0.390585847977489	-0.376753507014028\\
-0.392785571142285	-0.372317222951554\\
-0.398452948292133	-0.360721442885771\\
-0.406227313185292	-0.344689378757515\\
-0.408817635270541	-0.339296438816749\\
-0.413856949942716	-0.328657314629258\\
-0.421385260700624	-0.312625250501002\\
-0.424849699398798	-0.305176568166802\\
-0.428788385313387	-0.296593186372745\\
-0.436073478999931	-0.280561122244489\\
-0.440881763527054	-0.269881014607484\\
-0.443260064715237	-0.264529058116232\\
-0.450304642951515	-0.248496993987976\\
-0.456913827655311	-0.233324853259175\\
-0.457283740180684	-0.232464929859719\\
-0.46409036637589	-0.216432865731463\\
-0.470835143423245	-0.200400801603207\\
-0.472945891783567	-0.195321167180017\\
-0.477441250432316	-0.18436873747495\\
-0.483947723367294	-0.168336673346694\\
-0.488977955911824	-0.155805953765381\\
-0.490366928106163	-0.152304609218437\\
-0.496637367872314	-0.13627254509018\\
-0.502841024354953	-0.120240480961924\\
-0.50501002004008	-0.114555921806545\\
-0.508912636015967	-0.104208416833667\\
-0.514879435468517	-0.0881763527054109\\
-0.520776905952091	-0.0721442885771544\\
-0.521042084168337	-0.0714111270456461\\
-0.526513104852846	-0.0561122244488979\\
-0.532172574628555	-0.0400801603206413\\
-0.537074148296593	-0.026006153876575\\
-0.537748689204563	-0.0240480961923848\\
-0.54317196671062	-0.00801603206412826\\
-0.548519704985299	0.00801603206412782\\
-0.55310621242485	0.0219734059878419\\
-0.553780753332819	0.0240480961923843\\
-0.558890672250114	0.0400801603206409\\
-0.563921555665619	0.0561122244488974\\
-0.56887309833686	0.0721442885771539\\
-0.569138276553106	0.073022017184192\\
-0.573668496296284	0.0881763527054105\\
-0.57837666810335	0.104208416833667\\
-0.583001344996236	0.120240480961924\\
-0.585170340681363	0.127924561773787\\
-0.58750253504049	0.13627254509018\\
-0.591881390411667	0.152304609218437\\
-0.596172172265089	0.168336673346693\\
-0.60037395363977	0.18436873747495\\
-0.601202404809619	0.187616749668742\\
-0.60442986328447	0.200400801603206\\
-0.608378943530199	0.216432865731463\\
-0.612233727171058	0.232464929859719\\
-0.615992884122811	0.248496993987976\\
-0.617234468937876	0.25395901965368\\
-0.619612770126059	0.264529058116232\\
-0.623111489402566	0.280561122244489\\
-0.626508521446038	0.296593186372745\\
-0.629802094367959	0.312625250501002\\
-0.632990291864623	0.328657314629258\\
-0.633266533066132	0.330105497532926\\
-0.6360197366617	0.344689378757515\\
-0.638933910215981	0.360721442885771\\
-0.641735488262777	0.376753507014028\\
-0.644422048509105	0.392785571142285\\
-0.646990996137822	0.408817635270541\\
-0.649298597194389	0.423934427719838\\
-0.649436850751654	0.424849699398798\\
-0.651716318887212	0.440881763527054\\
-0.653869658805669	0.456913827655311\\
-0.655893692198952	0.472945891783567\\
-0.657785033616345	0.488977955911824\\
-0.659540079673549	0.50501002004008\\
-0.661154997500123	0.521042084168337\\
-0.662625712366185	0.537074148296593\\
-0.663947894423875	0.55310621242485\\
-0.665116944493114	0.569138276553106\\
-0.665330661322646	0.572556297241498\\
-0.666110017040159	0.585170340681363\\
-0.666937914365116	0.601202404809619\\
-0.667600197623856	0.617234468937876\\
-0.668091410176274	0.633266533066132\\
-0.668405767281445	0.649298597194389\\
-0.668537136825008	0.665330661322646\\
-0.668479018613839	0.681362725450902\\
-0.668224522116507	0.697394789579159\\
-0.6677663425161	0.713426853707415\\
-0.66709673492865	0.729458917835672\\
-0.666207486625595	0.745490981963928\\
-0.665330661322646	0.758105025403793\\
}--cycle;


\addplot[area legend,solid,fill=mycolor11,draw=black,forget plot]
table[row sep=crcr] {%
x	y\\
-0.472945891783567	0.631558403398626\\
-0.472605099508047	0.633266533066132\\
-0.468985504333452	0.649298597194389\\
-0.464849004208713	0.665330661322646\\
-0.460159075615718	0.681362725450902\\
-0.456913827655311	0.691303111073879\\
-0.454736499135691	0.697394789579159\\
-0.448395418577774	0.713426853707415\\
-0.441293873286597	0.729458917835672\\
-0.440881763527054	0.730315450863864\\
-0.432752526676281	0.745490981963928\\
-0.424849699398798	0.758778909934585\\
-0.423006465444712	0.761523046092185\\
-0.411260209628666	0.777555110220441\\
-0.408817635270541	0.780620139334836\\
-0.396869159056258	0.793587174348698\\
-0.392785571142285	0.797670762262671\\
-0.378610477861005	0.809619238476954\\
-0.376753507014028	0.811071611595405\\
-0.360721442885771	0.821430877328627\\
-0.352418655016157	0.825651302605211\\
-0.344689378757515	0.829329654544471\\
-0.328657314629258	0.835068349076468\\
-0.312625250501002	0.838968684198747\\
-0.296593186372745	0.841199718536656\\
-0.285769468609448	0.841683366733467\\
-0.280561122244489	0.841903918836987\\
-0.275352775879529	0.841683366733467\\
-0.264529058116232	0.841227025015592\\
-0.248496993987976	0.83926680531526\\
-0.232464929859719	0.836128895371847\\
-0.216432865731463	0.831901692994961\\
-0.200400801603207	0.82666322786356\\
-0.197825995821692	0.825651302605211\\
-0.18436873747495	0.820378748496846\\
-0.168336673346694	0.813184089207794\\
-0.161302958637682	0.809619238476954\\
-0.152304609218437	0.805070371891943\\
-0.13627254509018	0.796098946117739\\
-0.132188957176207	0.793587174348698\\
-0.120240480961924	0.786252889299523\\
-0.107126535443859	0.777555110220441\\
-0.104208416833667	0.77562315720775\\
-0.0881763527054109	0.764164839203355\\
-0.084723917528645	0.761523046092185\\
-0.0721442885771544	0.751909514130287\\
-0.0642414612996724	0.745490981963928\\
-0.0561122244488979	0.738895348584444\\
-0.045128831381008	0.729458917835672\\
-0.0400801603206413	0.725124673100652\\
-0.0271702987996977	0.713426853707415\\
-0.0240480961923848	0.710599232962271\\
-0.0101933605837475	0.697394789579159\\
-0.00801603206412826	0.695320190446341\\
0.00594143293131114	0.681362725450902\\
0.00801603206412782	0.679288126318085\\
0.0213479068822207	0.665330661322646\\
0.0240480961923843	0.662503040577503\\
0.0361197728705266	0.649298597194389\\
0.0400801603206409	0.64496435245937\\
0.0503347757565049	0.633266533066132\\
0.0561122244488974	0.626670899686649\\
0.0640578544475958	0.617234468937876\\
0.0721442885771539	0.607620936975979\\
0.0773435796145164	0.601202404809619\\
0.0881763527054105	0.587812133792533\\
0.0902380540757649	0.585170340681363\\
0.102710955361553	0.569138276553106\\
0.104208416833667	0.567206323540415\\
0.114761222290392	0.55310621242485\\
0.120240480961924	0.545771927375676\\
0.126524458918498	0.537074148296593\\
0.13627254509018	0.523553855937378\\
0.138026478403903	0.521042084168337\\
0.149159038093379	0.50501002004008\\
0.152304609218437	0.50046115345507\\
0.160001583014829	0.488977955911824\\
0.168336673346693	0.476510742514408\\
0.170649960884252	0.472945891783567\\
0.180987089305425	0.456913827655311\\
0.18436873747495	0.451641273546946\\
0.191071979166073	0.440881763527054\\
0.200400801603206	0.425861624657147\\
0.201012065831786	0.424849699398798\\
0.210603949389608	0.408817635270541\\
0.216432865731463	0.399035961532035\\
0.220058021957355	0.392785571142285\\
0.229290611656081	0.376753507014028\\
0.232464929859719	0.371199035652408\\
0.238297033847969	0.360721442885771\\
0.247173167878561	0.344689378757515\\
0.248496993987976	0.342272817339308\\
0.255766886224913	0.328657314629258\\
0.264289549707929	0.312625250501002\\
0.264529058116232	0.312168908783128\\
0.272502383661635	0.296593186372745\\
0.280561122244489	0.28078167434801\\
0.280670880448816	0.280561122244489\\
0.288535564991394	0.264529058116232\\
0.296360600228989	0.248496993987976\\
0.296593186372745	0.248013345791166\\
0.303895898217337	0.232464929859719\\
0.311376856476042	0.216432865731463\\
0.312625250501002	0.213718183196744\\
0.318610457811489	0.200400801603206\\
0.325750495336266	0.18436873747495\\
0.328657314629258	0.177753719817951\\
0.332704086079932	0.168336673346693\\
0.339506322152324	0.152304609218437\\
0.344689378757515	0.139950897029441\\
0.346199540087413	0.13627254509018\\
0.352667032756419	0.120240480961924\\
0.359067068258363	0.104208416833667\\
0.360721442885771	0.0999879915570851\\
0.36525335024532	0.0881763527054105\\
0.371314638075837	0.0721442885771539\\
0.376753507014028	0.0575645975673485\\
0.377284149442735	0.0561122244488974\\
0.383009686993799	0.0400801603206409\\
0.388652960980946	0.0240480961923843\\
0.392785571142285	0.0120996199781027\\
0.394169229744255	0.00801603206412782\\
0.399472542357823	-0.00801603206412826\\
0.404685025109202	-0.0240480961923848\\
0.408817635270541	-0.0370151312062448\\
0.409774840597944	-0.0400801603206413\\
0.414642913275298	-0.0561122244488979\\
0.419410830460606	-0.0721442885771544\\
0.424076362139965	-0.0881763527054109\\
0.424849699398798	-0.0909204888630107\\
0.428518805443235	-0.104208416833667\\
0.432827353397701	-0.120240480961924\\
0.437022813017709	-0.13627254509018\\
0.440881763527054	-0.151448076190245\\
0.441095225623083	-0.152304609218437\\
0.444928534977728	-0.168336673346694\\
0.448637216736993	-0.18436873747495\\
0.452217531449852	-0.200400801603207\\
0.45566543363035	-0.216432865731463\\
0.456913827655311	-0.222524544236743\\
0.458909321024828	-0.232464929859719\\
0.46197191826691	-0.248496993987976\\
0.464888270402216	-0.264529058116232\\
0.467653230225484	-0.280561122244489\\
0.470261256084326	-0.296593186372745\\
0.472706383375263	-0.312625250501002\\
0.472945891783567	-0.314333380168509\\
0.474911396503024	-0.328657314629258\\
0.476934843536039	-0.344689378757515\\
0.478777995771817	-0.360721442885771\\
0.48043346647087	-0.376753507014028\\
0.481893321940788	-0.392785571142285\\
0.483149039569969	-0.408817635270541\\
0.484191461832447	-0.424849699398798\\
0.485010745812909	-0.440881763527054\\
0.485596307742298	-0.456913827655311\\
0.48593676196695	-0.472945891783567\\
0.486019853696573	-0.488977955911824\\
0.485832384786766	-0.50501002004008\\
0.485360131708188	-0.521042084168337\\
0.484587754734332	-0.537074148296593\\
0.483498697240291	-0.55310621242485\\
0.482075073842231	-0.569138276553106\\
0.480297545917277	-0.585170340681363\\
0.47814518282093	-0.601202404809619\\
0.475595306857631	-0.617234468937876\\
0.472945891783567	-0.631558403398626\\
0.472605099508047	-0.633266533066132\\
0.468985504333452	-0.649298597194389\\
0.464849004208713	-0.665330661322646\\
0.46015907561572	-0.681362725450902\\
0.456913827655311	-0.69130311107388\\
0.454736499135691	-0.697394789579158\\
0.448395418577773	-0.713426853707415\\
0.441293873286598	-0.729458917835671\\
0.440881763527054	-0.730315450863864\\
0.432752526676281	-0.745490981963928\\
0.424849699398798	-0.758778909934585\\
0.423006465444711	-0.761523046092184\\
0.411260209628669	-0.777555110220441\\
0.408817635270541	-0.780620139334837\\
0.396869159056258	-0.793587174348697\\
0.392785571142285	-0.797670762262671\\
0.378610477861005	-0.809619238476954\\
0.376753507014028	-0.811071611595405\\
0.360721442885771	-0.821430877328629\\
0.352418655016159	-0.82565130260521\\
0.344689378757515	-0.829329654544471\\
0.328657314629258	-0.835068349076468\\
0.312625250501002	-0.838968684198748\\
0.296593186372745	-0.841199718536656\\
0.285769468609449	-0.841683366733467\\
0.280561122244489	-0.841903918836987\\
0.275352775879529	-0.841683366733467\\
0.264529058116232	-0.841227025015592\\
0.248496993987976	-0.839266805315259\\
0.232464929859719	-0.836128895371846\\
0.216432865731463	-0.83190169299496\\
0.200400801603206	-0.826663227863559\\
0.197825995821691	-0.82565130260521\\
0.18436873747495	-0.820378748496845\\
0.168336673346693	-0.813184089207795\\
0.161302958637679	-0.809619238476954\\
0.152304609218437	-0.805070371891943\\
0.13627254509018	-0.796098946117737\\
0.132188957176207	-0.793587174348697\\
0.120240480961924	-0.786252889299523\\
0.107126535443859	-0.777555110220441\\
0.104208416833667	-0.77562315720775\\
0.0881763527054105	-0.764164839203354\\
0.0847239175286446	-0.761523046092184\\
0.0721442885771539	-0.751909514130287\\
0.064241461299671	-0.745490981963928\\
0.0561122244488974	-0.738895348584444\\
0.0451288313810076	-0.729458917835671\\
0.0400801603206409	-0.725124673100651\\
0.0271702987996958	-0.713426853707415\\
0.0240480961923843	-0.710599232962272\\
0.010193360583747	-0.697394789579158\\
0.00801603206412782	-0.69532019044634\\
-0.00594143293131158	-0.681362725450902\\
-0.00801603206412826	-0.679288126318085\\
-0.0213479068822197	-0.665330661322646\\
-0.0240480961923848	-0.662503040577502\\
-0.0361197728705266	-0.649298597194389\\
-0.0400801603206413	-0.644964352459369\\
-0.050334775756504	-0.633266533066132\\
-0.0561122244488979	-0.626670899686648\\
-0.0640578544475958	-0.617234468937876\\
-0.0721442885771544	-0.607620936975978\\
-0.0773435796145168	-0.601202404809619\\
-0.0881763527054109	-0.587812133792533\\
-0.0902380540757654	-0.585170340681363\\
-0.102710955361554	-0.569138276553106\\
-0.104208416833667	-0.567206323540416\\
-0.114761222290391	-0.55310621242485\\
-0.120240480961924	-0.545771927375675\\
-0.126524458918498	-0.537074148296593\\
-0.13627254509018	-0.523553855937377\\
-0.138026478403903	-0.521042084168337\\
-0.149159038093379	-0.50501002004008\\
-0.152304609218437	-0.500461153455069\\
-0.160001583014829	-0.488977955911824\\
-0.168336673346694	-0.476510742514407\\
-0.170649960884251	-0.472945891783567\\
-0.180987089305424	-0.456913827655311\\
-0.18436873747495	-0.451641273546945\\
-0.191071979166074	-0.440881763527054\\
-0.200400801603207	-0.425861624657147\\
-0.201012065831786	-0.424849699398798\\
-0.210603949389609	-0.408817635270541\\
-0.216432865731463	-0.399035961532034\\
-0.220058021957355	-0.392785571142285\\
-0.229290611656081	-0.376753507014028\\
-0.232464929859719	-0.371199035652408\\
-0.238297033847968	-0.360721442885771\\
-0.247173167878561	-0.344689378757515\\
-0.248496993987976	-0.342272817339309\\
-0.255766886224913	-0.328657314629258\\
-0.264289549707929	-0.312625250501002\\
-0.264529058116232	-0.312168908783128\\
-0.272502383661634	-0.296593186372745\\
-0.280561122244489	-0.28078167434801\\
-0.280670880448816	-0.280561122244489\\
-0.288535564991394	-0.264529058116232\\
-0.296360600228989	-0.248496993987976\\
-0.296593186372745	-0.248013345791166\\
-0.303895898217337	-0.232464929859719\\
-0.311376856476042	-0.216432865731463\\
-0.312625250501002	-0.213718183196745\\
-0.318610457811488	-0.200400801603207\\
-0.325750495336266	-0.18436873747495\\
-0.328657314629258	-0.177753719817951\\
-0.332704086079932	-0.168336673346694\\
-0.339506322152324	-0.152304609218437\\
-0.344689378757515	-0.139950897029441\\
-0.346199540087413	-0.13627254509018\\
-0.352667032756418	-0.120240480961924\\
-0.359067068258364	-0.104208416833667\\
-0.360721442885771	-0.0999879915570855\\
-0.365253350245319	-0.0881763527054109\\
-0.371314638075837	-0.0721442885771544\\
-0.376753507014028	-0.0575645975673489\\
-0.377284149442735	-0.0561122244488979\\
-0.383009686993798	-0.0400801603206413\\
-0.388652960980946	-0.0240480961923848\\
-0.392785571142285	-0.0120996199781018\\
-0.394169229744254	-0.00801603206412826\\
-0.399472542357823	0.00801603206412782\\
-0.404685025109202	0.0240480961923843\\
-0.408817635270541	0.0370151312062466\\
-0.409774840597943	0.0400801603206409\\
-0.414642913275298	0.0561122244488974\\
-0.419410830460605	0.0721442885771539\\
-0.424076362139966	0.0881763527054105\\
-0.424849699398798	0.0909204888630098\\
-0.428518805443234	0.104208416833667\\
-0.432827353397701	0.120240480961924\\
-0.437022813017709	0.13627254509018\\
-0.440881763527054	0.151448076190245\\
-0.441095225623083	0.152304609218437\\
-0.444928534977728	0.168336673346693\\
-0.448637216736993	0.18436873747495\\
-0.452217531449852	0.200400801603206\\
-0.455665433630351	0.216432865731463\\
-0.456913827655311	0.222524544236743\\
-0.458909321024828	0.232464929859719\\
-0.461971918266909	0.248496993987976\\
-0.464888270402216	0.264529058116232\\
-0.467653230225484	0.280561122244489\\
-0.470261256084326	0.296593186372745\\
-0.472706383375263	0.312625250501002\\
-0.472945891783567	0.314333380168508\\
-0.474911396503024	0.328657314629258\\
-0.476934843536038	0.344689378757515\\
-0.478777995771816	0.360721442885771\\
-0.48043346647087	0.376753507014028\\
-0.481893321940788	0.392785571142285\\
-0.483149039569969	0.408817635270541\\
-0.484191461832447	0.424849699398798\\
-0.485010745812909	0.440881763527054\\
-0.485596307742298	0.456913827655311\\
-0.48593676196695	0.472945891783567\\
-0.486019853696573	0.488977955911824\\
-0.485832384786766	0.50501002004008\\
-0.485360131708188	0.521042084168337\\
-0.484587754734332	0.537074148296593\\
-0.483498697240291	0.55310621242485\\
-0.482075073842231	0.569138276553106\\
-0.480297545917278	0.585170340681363\\
-0.47814518282093	0.601202404809619\\
-0.475595306857631	0.617234468937876\\
-0.472945891783567	0.631558403398626\\
}--cycle;


\addplot[area legend,solid,fill=mycolor12,draw=black,forget plot]
table[row sep=crcr] {%
x	y\\
-0.200400801603207	0.304862370138295\\
-0.197344662109513	0.312625250501002\\
-0.189390682595018	0.328657314629258\\
-0.18436873747495	0.336847857012185\\
-0.177891791257849	0.344689378757515\\
-0.168336673346694	0.35424449666867\\
-0.158402802348644	0.360721442885771\\
-0.152304609218437	0.364110634083375\\
-0.13627254509018	0.368853940670599\\
-0.120240480961924	0.369881171754941\\
-0.104208416833667	0.367869101570326\\
-0.0881763527054109	0.363336633675013\\
-0.082078159575204	0.360721442885771\\
-0.0721442885771544	0.356466805104686\\
-0.0561122244488979	0.34759700820266\\
-0.0518105910934108	0.344689378757515\\
-0.0400801603206413	0.336766468226409\\
-0.0297366567970107	0.328657314629258\\
-0.0240480961923848	0.324199859313259\\
-0.0110721715578217	0.312625250501002\\
-0.00801603206412826	0.309899854870135\\
0.00529063643326164	0.296593186372745\\
0.00801603206412782	0.293867790741879\\
0.0200260317046286	0.280561122244489\\
0.0240480961923843	0.276103666928489\\
0.0335674421125732	0.264529058116232\\
0.0400801603206409	0.256606147585127\\
0.0462057588758648	0.248496993987976\\
0.0561122244488974	0.235372559304864\\
0.0581431221874816	0.232464929859719\\
0.0692259403854356	0.216432865731463\\
0.0721442885771539	0.212178227950381\\
0.0796438295284159	0.200400801603206\\
0.0881763527054105	0.186983928264192\\
0.0897289331740241	0.18436873747495\\
0.099068780199145	0.168336673346693\\
0.104208416833667	0.159452267902991\\
0.108078661342575	0.152304609218437\\
0.116632841383845	0.13627254509018\\
0.120240480961924	0.129400209831093\\
0.124752283040789	0.120240480961924\\
0.132520874517804	0.104208416833667\\
0.13627254509018	0.0963088504902381\\
0.139905087192154	0.0881763527054105\\
0.146892501509108	0.0721442885771539\\
0.152304609218437	0.0595014156465018\\
0.153672158159379	0.0561122244488974\\
0.159885987743175	0.0400801603206409\\
0.165975112210303	0.0240480961923843\\
0.168336673346693	0.0175711499752883\\
0.171621426422992	0.00801603206412782\\
0.176904626791464	-0.00801603206412826\\
0.182007176338558	-0.0240480961923848\\
0.18436873747495	-0.0318896179377154\\
0.186695011676916	-0.0400801603206413\\
0.190963287470592	-0.0561122244488979\\
0.194988693893878	-0.0721442885771544\\
0.198748967636206	-0.0881763527054109\\
0.200400801603206	-0.0959392330681169\\
0.202057402535111	-0.104208416833667\\
0.204912603682072	-0.120240480961924\\
0.207426878711987	-0.13627254509018\\
0.209566606502076	-0.152304609218437\\
0.211293229096941	-0.168336673346694\\
0.212562354681171	-0.18436873747495\\
0.213322657300812	-0.200400801603207\\
0.213514517539743	-0.216432865731463\\
0.213068330138315	-0.232464929859719\\
0.211902379345737	-0.248496993987976\\
0.209920147523395	-0.264529058116232\\
0.207006872475959	-0.280561122244489\\
0.203025096977945	-0.296593186372745\\
0.200400801603206	-0.304862370138296\\
0.197344662109512	-0.312625250501002\\
0.189390682595018	-0.328657314629258\\
0.18436873747495	-0.336847857012184\\
0.17789179125785	-0.344689378757515\\
0.168336673346693	-0.354244496668672\\
0.158402802348643	-0.360721442885771\\
0.152304609218437	-0.364110634083375\\
0.13627254509018	-0.3688539406706\\
0.120240480961924	-0.369881171754941\\
0.104208416833667	-0.367869101570326\\
0.0881763527054105	-0.363336633675014\\
0.0820781595752036	-0.360721442885771\\
0.0721442885771539	-0.356466805104686\\
0.0561122244488974	-0.347597008202659\\
0.0518105910934104	-0.344689378757515\\
0.0400801603206409	-0.336766468226409\\
0.0297366567970103	-0.328657314629258\\
0.0240480961923843	-0.324199859313259\\
0.0110721715578212	-0.312625250501002\\
0.00801603206412782	-0.309899854870136\\
-0.00529063643326208	-0.296593186372745\\
-0.00801603206412826	-0.293867790741879\\
-0.020026031704629	-0.280561122244489\\
-0.0240480961923848	-0.276103666928489\\
-0.0335674421125736	-0.264529058116232\\
-0.0400801603206413	-0.256606147585127\\
-0.0462057588758648	-0.248496993987976\\
-0.0561122244488979	-0.235372559304863\\
-0.058143122187482	-0.232464929859719\\
-0.0692259403854342	-0.216432865731463\\
-0.0721442885771544	-0.212178227950378\\
-0.0796438295284154	-0.200400801603207\\
-0.0881763527054109	-0.186983928264193\\
-0.0897289331740245	-0.18436873747495\\
-0.0990687801991455	-0.168336673346694\\
-0.104208416833667	-0.159452267902992\\
-0.108078661342576	-0.152304609218437\\
-0.116632841383844	-0.13627254509018\\
-0.120240480961924	-0.129400209831092\\
-0.124752283040789	-0.120240480961924\\
-0.132520874517804	-0.104208416833667\\
-0.13627254509018	-0.096308850490239\\
-0.139905087192154	-0.0881763527054109\\
-0.146892501509108	-0.0721442885771544\\
-0.152304609218437	-0.0595014156465017\\
-0.15367215815938	-0.0561122244488979\\
-0.159885987743175	-0.0400801603206413\\
-0.165975112210302	-0.0240480961923848\\
-0.168336673346694	-0.0175711499752838\\
-0.171621426422991	-0.00801603206412826\\
-0.176904626791464	0.00801603206412782\\
-0.182007176338559	0.0240480961923843\\
-0.18436873747495	0.031889617937715\\
-0.186695011676916	0.0400801603206409\\
-0.190963287470593	0.0561122244488974\\
-0.194988693893878	0.0721442885771539\\
-0.198748967636206	0.0881763527054105\\
-0.200400801603207	0.0959392330681169\\
-0.202057402535111	0.104208416833667\\
-0.204912603682072	0.120240480961924\\
-0.207426878711987	0.13627254509018\\
-0.209566606502074	0.152304609218437\\
-0.211293229096941	0.168336673346693\\
-0.212562354681169	0.18436873747495\\
-0.213322657300812	0.200400801603206\\
-0.213514517539743	0.216432865731463\\
-0.213068330138315	0.232464929859719\\
-0.211902379345734	0.248496993987976\\
-0.209920147523395	0.264529058116232\\
-0.207006872475959	0.280561122244489\\
-0.203025096977946	0.296593186372745\\
-0.200400801603207	0.304862370138295\\
}--cycle;

\addplot [color=white,dotted,forget plot]
  table[row sep=crcr]{%
-4	2\\
-3.98396793587174	2\\
-3.96793587174349	2\\
-3.95190380761523	2\\
-3.93587174348697	2\\
-3.91983967935872	2\\
-3.90380761523046	2\\
-3.8877755511022	2\\
-3.87174348697395	2\\
-3.85571142284569	2\\
-3.83967935871743	2\\
-3.82364729458918	2\\
-3.80761523046092	2\\
-3.79158316633267	2\\
-3.77555110220441	2\\
-3.75951903807615	2\\
-3.7434869739479	2\\
-3.72745490981964	2\\
-3.71142284569138	2\\
-3.69539078156313	2\\
-3.67935871743487	2\\
-3.66332665330661	2\\
-3.64729458917836	2\\
-3.6312625250501	2\\
-3.61523046092184	2\\
-3.59919839679359	2\\
-3.58316633266533	2\\
-3.56713426853707	2\\
-3.55110220440882	2\\
-3.53507014028056	2\\
-3.5190380761523	2\\
-3.50300601202405	2\\
-3.48697394789579	2\\
-3.47094188376753	2\\
-3.45490981963928	2\\
-3.43887775551102	2\\
-3.42284569138277	2\\
-3.40681362725451	2\\
-3.39078156312625	2\\
-3.374749498998	2\\
-3.35871743486974	2\\
-3.34268537074148	2\\
-3.32665330661323	2\\
-3.31062124248497	2\\
-3.29458917835671	2\\
-3.27855711422846	2\\
-3.2625250501002	2\\
-3.24649298597194	2\\
-3.23046092184369	2\\
-3.21442885771543	2\\
-3.19839679358717	2\\
-3.18236472945892	2\\
-3.16633266533066	2\\
-3.1503006012024	2\\
-3.13426853707415	2\\
-3.11823647294589	2\\
-3.10220440881764	2\\
-3.08617234468938	2\\
-3.07014028056112	2\\
-3.05410821643287	2\\
-3.03807615230461	2\\
-3.02204408817635	2\\
-3.0060120240481	2\\
-2.98997995991984	2\\
-2.97394789579158	2\\
-2.95791583166333	2\\
-2.94188376753507	2\\
-2.92585170340681	2\\
-2.90981963927856	2\\
-2.8937875751503	2\\
-2.87775551102204	2\\
-2.86172344689379	2\\
-2.84569138276553	2\\
-2.82965931863727	2\\
-2.81362725450902	2\\
-2.79759519038076	2\\
-2.7815631262525	2\\
-2.76553106212425	2\\
-2.74949899799599	2\\
-2.73346693386774	2\\
-2.71743486973948	2\\
-2.70140280561122	2\\
-2.68537074148297	2\\
-2.66933867735471	2\\
-2.65330661322645	2\\
-2.6372745490982	2\\
-2.62124248496994	2\\
-2.60521042084168	2\\
-2.58917835671343	2\\
-2.57314629258517	2\\
-2.55711422845691	2\\
-2.54108216432866	2\\
-2.5250501002004	2\\
-2.50901803607214	2\\
-2.49298597194389	2\\
-2.47695390781563	2\\
-2.46092184368737	2\\
-2.44488977955912	2\\
-2.42885771543086	2\\
-2.41282565130261	2\\
-2.39679358717435	2\\
-2.38076152304609	2\\
-2.36472945891784	2\\
-2.34869739478958	2\\
-2.33266533066132	2\\
-2.31663326653307	2\\
-2.30060120240481	2\\
-2.28456913827655	2\\
-2.2685370741483	2\\
-2.25250501002004	2\\
-2.23647294589178	2\\
-2.22044088176353	2\\
-2.20440881763527	2\\
-2.18837675350701	2\\
-2.17234468937876	2\\
-2.1563126252505	2\\
-2.14028056112224	2\\
-2.12424849699399	2\\
-2.10821643286573	2\\
-2.09218436873747	2\\
-2.07615230460922	2\\
-2.06012024048096	2\\
-2.04408817635271	2\\
-2.02805611222445	2\\
-2.01202404809619	2\\
-1.99599198396794	2\\
-1.97995991983968	2\\
-1.96392785571142	2\\
-1.94789579158317	2\\
-1.93186372745491	2\\
-1.91583166332665	2\\
-1.8997995991984	2\\
-1.88376753507014	2\\
-1.86773547094188	2\\
-1.85170340681363	2\\
-1.83567134268537	2\\
-1.81963927855711	2\\
-1.80360721442886	2\\
-1.7875751503006	2\\
-1.77154308617234	2\\
-1.75551102204409	2\\
-1.73947895791583	2\\
-1.72344689378758	2\\
-1.70741482965932	2\\
-1.69138276553106	2\\
-1.67535070140281	2\\
-1.65931863727455	2\\
-1.64328657314629	2\\
-1.62725450901804	2\\
-1.61122244488978	2\\
-1.59519038076152	2\\
-1.57915831663327	2\\
-1.56312625250501	2\\
-1.54709418837675	2\\
-1.5310621242485	2\\
-1.51503006012024	2\\
-1.49899799599198	2\\
-1.48296593186373	2\\
-1.46693386773547	2\\
-1.45090180360721	2\\
-1.43486973947896	2\\
-1.4188376753507	2\\
-1.40280561122244	2\\
-1.38677354709419	2\\
-1.37074148296593	2\\
-1.35470941883768	2\\
-1.33867735470942	2\\
-1.32264529058116	2\\
-1.30661322645291	2\\
-1.29058116232465	2\\
-1.27454909819639	2\\
-1.25851703406814	2\\
-1.24248496993988	2\\
-1.22645290581162	2\\
-1.21042084168337	2\\
-1.19438877755511	2\\
-1.17835671342685	2\\
-1.1623246492986	2\\
-1.14629258517034	2\\
-1.13026052104208	2\\
-1.11422845691383	2\\
-1.09819639278557	2\\
-1.08216432865731	2\\
-1.06613226452906	2\\
-1.0501002004008	2\\
-1.03406813627255	2\\
-1.01803607214429	2\\
-1.00200400801603	2\\
-0.985971943887776	2\\
-0.969939879759519	2\\
-0.953907815631263	2\\
-0.937875751503006	2\\
-0.92184368737475	2\\
-0.905811623246493	2\\
-0.889779559118236	2\\
-0.87374749498998	2\\
-0.857715430861723	2\\
-0.841683366733467	2\\
-0.82565130260521	2\\
-0.809619238476954	2\\
-0.793587174348697	2\\
-0.777555110220441	2\\
-0.761523046092184	2\\
-0.745490981963928	2\\
-0.729458917835671	2\\
-0.713426853707415	2\\
-0.697394789579158	2\\
-0.681362725450902	2\\
-0.665330661322646	2\\
-0.649298597194389	2\\
-0.633266533066132	2\\
-0.617234468937876	2\\
-0.601202404809619	2\\
-0.585170340681363	2\\
-0.569138276553106	2\\
-0.55310621242485	2\\
-0.537074148296593	2\\
-0.521042084168337	2\\
-0.50501002004008	2\\
-0.488977955911824	2\\
-0.472945891783567	2\\
-0.456913827655311	2\\
-0.440881763527054	2\\
-0.424849699398798	2\\
-0.408817635270541	2\\
-0.392785571142285	2\\
-0.376753507014028	2\\
-0.360721442885771	2\\
-0.344689378757515	2\\
-0.328657314629258	2\\
-0.312625250501002	2\\
-0.296593186372745	2\\
-0.280561122244489	2\\
-0.264529058116232	2\\
-0.248496993987976	2\\
-0.232464929859719	2\\
-0.216432865731463	2\\
-0.200400801603207	2\\
-0.18436873747495	2\\
-0.168336673346694	2\\
-0.152304609218437	2\\
-0.13627254509018	2\\
-0.120240480961924	2\\
-0.104208416833667	2\\
-0.0881763527054109	2\\
-0.0721442885771544	2\\
-0.0561122244488979	2\\
-0.0400801603206413	2\\
-0.0240480961923848	2\\
-0.00801603206412826	2\\
0.00801603206412782	2\\
0.0240480961923843	2\\
0.0400801603206409	2\\
0.0561122244488974	2\\
0.0721442885771539	2\\
0.0881763527054105	2\\
0.104208416833667	2\\
0.120240480961924	2\\
0.13627254509018	2\\
0.152304609218437	2\\
0.168336673346693	2\\
0.18436873747495	2\\
0.200400801603206	2\\
0.216432865731463	2\\
0.232464929859719	2\\
0.248496993987976	2\\
0.264529058116232	2\\
0.280561122244489	2\\
0.296593186372745	2\\
0.312625250501002	2\\
0.328657314629258	2\\
0.344689378757515	2\\
0.360721442885771	2\\
0.376753507014028	2\\
0.392785571142285	2\\
0.408817635270541	2\\
0.424849699398798	2\\
0.440881763527054	2\\
0.456913827655311	2\\
0.472945891783567	2\\
0.488977955911824	2\\
0.50501002004008	2\\
0.521042084168337	2\\
0.537074148296593	2\\
0.55310621242485	2\\
0.569138276553106	2\\
0.585170340681363	2\\
0.601202404809619	2\\
0.617234468937876	2\\
0.633266533066132	2\\
0.649298597194389	2\\
0.665330661322646	2\\
0.681362725450902	2\\
0.697394789579159	2\\
0.713426853707415	2\\
0.729458917835672	2\\
0.745490981963928	2\\
0.761523046092185	2\\
0.777555110220441	2\\
0.793587174348698	2\\
0.809619238476954	2\\
0.825651302605211	2\\
0.841683366733467	2\\
0.857715430861724	2\\
0.87374749498998	2\\
0.889779559118236	2\\
0.905811623246493	2\\
0.921843687374749	2\\
0.937875751503006	2\\
0.953907815631262	2\\
0.969939879759519	2\\
0.985971943887775	2\\
1.00200400801603	2\\
1.01803607214429	2\\
1.03406813627254	2\\
1.0501002004008	2\\
1.06613226452906	2\\
1.08216432865731	2\\
1.09819639278557	2\\
1.11422845691383	2\\
1.13026052104208	2\\
1.14629258517034	2\\
1.1623246492986	2\\
1.17835671342685	2\\
1.19438877755511	2\\
1.21042084168337	2\\
1.22645290581162	2\\
1.24248496993988	2\\
1.25851703406814	2\\
1.27454909819639	2\\
1.29058116232465	2\\
1.30661322645291	2\\
1.32264529058116	2\\
1.33867735470942	2\\
1.35470941883768	2\\
1.37074148296593	2\\
1.38677354709419	2\\
1.40280561122244	2\\
1.4188376753507	2\\
1.43486973947896	2\\
1.45090180360721	2\\
1.46693386773547	2\\
1.48296593186373	2\\
1.49899799599198	2\\
1.51503006012024	2\\
1.5310621242485	2\\
1.54709418837675	2\\
1.56312625250501	2\\
1.57915831663327	2\\
1.59519038076152	2\\
1.61122244488978	2\\
1.62725450901804	2\\
1.64328657314629	2\\
1.65931863727455	2\\
1.67535070140281	2\\
1.69138276553106	2\\
1.70741482965932	2\\
1.72344689378758	2\\
1.73947895791583	2\\
1.75551102204409	2\\
1.77154308617235	2\\
1.7875751503006	2\\
1.80360721442886	2\\
1.81963927855711	2\\
1.83567134268537	2\\
1.85170340681363	2\\
1.86773547094188	2\\
1.88376753507014	2\\
1.8997995991984	2\\
1.91583166332665	2\\
1.93186372745491	2\\
1.94789579158317	2\\
1.96392785571142	2\\
1.97995991983968	2\\
1.99599198396794	2\\
2.01202404809619	2\\
2.02805611222445	2\\
2.04408817635271	2\\
2.06012024048096	2\\
2.07615230460922	2\\
2.09218436873747	2\\
2.10821643286573	2\\
2.12424849699399	2\\
2.14028056112224	2\\
2.1563126252505	2\\
2.17234468937876	2\\
2.18837675350701	2\\
2.20440881763527	2\\
2.22044088176353	2\\
2.23647294589178	2\\
2.25250501002004	2\\
2.2685370741483	2\\
2.28456913827655	2\\
2.30060120240481	2\\
2.31663326653307	2\\
2.33266533066132	2\\
2.34869739478958	2\\
2.36472945891784	2\\
2.38076152304609	2\\
2.39679358717435	2\\
2.41282565130261	2\\
2.42885771543086	2\\
2.44488977955912	2\\
2.46092184368737	2\\
2.47695390781563	2\\
2.49298597194389	2\\
2.50901803607214	2\\
2.5250501002004	2\\
2.54108216432866	2\\
2.55711422845691	2\\
2.57314629258517	2\\
2.58917835671343	2\\
2.60521042084168	2\\
2.62124248496994	2\\
2.6372745490982	2\\
2.65330661322645	2\\
2.66933867735471	2\\
2.68537074148297	2\\
2.70140280561122	2\\
2.71743486973948	2\\
2.73346693386774	2\\
2.74949899799599	2\\
2.76553106212425	2\\
2.7815631262525	2\\
2.79759519038076	2\\
2.81362725450902	2\\
2.82965931863727	2\\
2.84569138276553	2\\
2.86172344689379	2\\
2.87775551102204	2\\
2.8937875751503	2\\
2.90981963927856	2\\
2.92585170340681	2\\
2.94188376753507	2\\
2.95791583166333	2\\
2.97394789579158	2\\
2.98997995991984	2\\
3.0060120240481	2\\
3.02204408817635	2\\
3.03807615230461	2\\
3.05410821643287	2\\
3.07014028056112	2\\
3.08617234468938	2\\
3.10220440881764	2\\
3.11823647294589	2\\
3.13426853707415	2\\
3.1503006012024	2\\
3.16633266533066	2\\
3.18236472945892	2\\
3.19839679358717	2\\
3.21442885771543	2\\
3.23046092184369	2\\
3.24649298597194	2\\
3.2625250501002	2\\
3.27855711422846	2\\
3.29458917835671	2\\
3.31062124248497	2\\
3.32665330661323	2\\
3.34268537074148	2\\
3.35871743486974	2\\
3.374749498998	2\\
3.39078156312625	2\\
3.40681362725451	2\\
3.42284569138277	2\\
3.43887775551102	2\\
3.45490981963928	2\\
3.47094188376754	2\\
3.48697394789579	2\\
3.50300601202405	2\\
3.51903807615231	2\\
3.53507014028056	2\\
3.55110220440882	2\\
3.56713426853707	2\\
3.58316633266533	2\\
3.59919839679359	2\\
3.61523046092184	2\\
3.6312625250501	2\\
3.64729458917836	2\\
3.66332665330661	2\\
3.67935871743487	2\\
3.69539078156313	2\\
3.71142284569138	2\\
3.72745490981964	2\\
3.7434869739479	2\\
3.75951903807615	2\\
3.77555110220441	2\\
3.79158316633267	2\\
3.80761523046092	2\\
3.82364729458918	2\\
3.83967935871743	2\\
3.85571142284569	2\\
3.87174348697395	2\\
3.8877755511022	2\\
3.90380761523046	2\\
3.91983967935872	2\\
3.93587174348697	2\\
3.95190380761523	2\\
3.96793587174349	2\\
3.98396793587174	2\\
4	2\\
};
\end{axis}
\end{tikzpicture}%
    \caption{$p(\vec{x} \given \vec{\mu}, \mat{\Sigma})$}
    \label{conditional_2d_pdf}
  \end{subfigure}
  \begin{subfigure}[t]{0.49\textwidth}
    % This file was created by matlab2tikz.
% Minimal pgfplots version: 1.3
%
\tikzsetnextfilename{2d_conditional_pdf}
\definecolor{mycolor1}{rgb}{0.20000,0.62745,0.17255}%
\definecolor{mycolor2}{rgb}{0.12157,0.47059,0.70588}%
%
\begin{tikzpicture}

\begin{axis}[%
width=0.95092\smallfigurewidth,
height=\smallfigureheight,
at={(0\smallfigurewidth,0\smallfigureheight)},
scale only axis,
xmin=-4,
xmax=4,
xlabel={$x_1$},
ymin=0,
ymax=0.6,
axis x line*=bottom,
axis y line*=left,
legend style={legend cell align=left,align=left,fill=none,draw=none}
]
\addplot [color=mycolor1,solid]
  table[row sep=crcr]{%
-4	0.000117445129963364\\
-3.98396793587174	0.000127222662050478\\
-3.96793587174349	0.000137761069688905\\
-3.95190380761523	0.000149114918625066\\
-3.93587174348697	0.000161342301301458\\
-3.91983967935872	0.000174505034273774\\
-3.90380761523046	0.000188668864387138\\
-3.8877755511022	0.000203903683933546\\
-3.87174348697395	0.000220283755005535\\
-3.85571142284569	0.000237887943252822\\
-3.83967935871743	0.000256799961238978\\
-3.82364729458918	0.000277108621584173\\
-3.80761523046092	0.000298908100067476\\
-3.79158316633267	0.00032229820884809\\
-3.77555110220441	0.000347384679949196\\
-3.75951903807615	0.000374279459130612\\
-3.7434869739479	0.000403101010257316\\
-3.72745490981964	0.000433974630249758\\
-3.71142284569138	0.000467032774679003\\
-3.69539078156313	0.000502415394044737\\
-3.67935871743487	0.000540270280747202\\
-3.66332665330661	0.00058075342673505\\
-3.64729458917836	0.000624029391779829\\
-3.6312625250501	0.000670271682294396\\
-3.61523046092184	0.000719663140576801\\
-3.59919839679359	0.000772396344323203\\
-3.58316633266533	0.000828674016213033\\
-3.56713426853707	0.000888709443326916\\
-3.55110220440882	0.000952726906112775\\
-3.53507014028056	0.00102096211656811\\
-3.5190380761523	0.00109366266525648\\
-3.50300601202405	0.00117108847672411\\
-3.48697394789579	0.00125351227282766\\
-3.47094188376753	0.00134122004342752\\
-3.45490981963928	0.00143451152384135\\
-3.43887775551102	0.00153370067839131\\
-3.42284569138277	0.00163911618931433\\
-3.40681362725451	0.00175110195023922\\
-3.39078156312625	0.00187001756336611\\
-3.374749498998	0.00199623883941408\\
-3.35871743486974	0.00213015829933083\\
-3.34268537074148	0.00227218567668526\\
-3.32665330661323	0.00242274841958831\\
-3.31062124248497	0.00258229219091151\\
-3.29458917835671	0.00275128136549495\\
-3.27855711422846	0.00293019952295783\\
-3.2625250501002	0.0031195499346455\\
-3.24649298597194	0.00331985604316683\\
-3.23046092184369	0.00353166193289578\\
-3.21442885771543	0.00375553278973067\\
-3.19839679358717	0.0039920553483245\\
-3.18236472945892	0.00424183832492064\\
-3.16633266533066	0.00450551283384916\\
-3.1503006012024	0.00478373278566195\\
-3.13426853707415	0.00507717526480896\\
-3.11823647294589	0.00538654088468379\\
-3.10220440881764	0.00571255411779558\\
-3.08617234468938	0.00605596359875481\\
-3.07014028056112	0.00641754239769541\\
-3.05410821643287	0.00679808826169285\\
-3.03807615230461	0.00719842382167983\\
-3.02204408817635	0.00761939676230755\\
-3.0060120240481	0.00806187995215131\\
-2.98997995991984	0.00852677153161581\\
-2.97394789579158	0.00901499495585764\\
-2.95791583166333	0.00952749899001121\\
-2.94188376753507	0.0100652576539795\\
-2.92585170340681	0.0106292701140337\\
-2.90981963927856	0.011220560518456\\
-2.8937875751503	0.0118401777744586\\
-2.87775551102204	0.0124891952636195\\
-2.86172344689379	0.0131687104930902\\
-2.84569138276553	0.0138798446798602\\
-2.82965931863727	0.0146237422653945\\
-2.81362725450902	0.0154015703580108\\
-2.79759519038076	0.0162145181004178\\
-2.7815631262525	0.0170637959599068\\
-2.76553106212425	0.0179506349387662\\
-2.74949899799599	0.0188762857025847\\
-2.73346693386774	0.0198420176242085\\
-2.71743486973948	0.0208491177412395\\
-2.70140280561122	0.0218988896250882\\
-2.68537074148297	0.0229926521597385\\
-2.66933867735471	0.0241317382285378\\
-2.65330661322645	0.0253174933074955\\
-2.6372745490982	0.026551273963752\\
-2.62124248496994	0.0278344462580818\\
-2.60521042084168	0.0291683840504969\\
-2.58917835671343	0.0305544672082441\\
-2.57314629258517	0.0319940797157218\\
-2.55711422845691	0.0334886076860916\\
-2.54108216432866	0.0350394372746198\\
-2.5250501002004	0.0366479524940578\\
-2.50901803607214	0.0383155329326544\\
-2.49298597194389	0.0400435513756899\\
-2.47695390781563	0.041833371331729\\
-2.46092184368737	0.0436863444651058\\
-2.44488977955912	0.0456038079364852\\
-2.42885771543086	0.0475870816536754\\
-2.41282565130261	0.0496374654352153\\
-2.39679358717435	0.0517562360896105\\
-2.38076152304609	0.0539446444134496\\
-2.36472945891784	0.0562039121119953\\
-2.34869739478958	0.0585352286462132\\
-2.33266533066132	0.0609397480105739\\
-2.31663326653307	0.0634185854463319\\
-2.30060120240481	0.0659728140953648\\
-2.28456913827655	0.0686034616000252\\
-2.2685370741483	0.0713115066548308\\
-2.25250501002004	0.0740978755161859\\
-2.23647294589178	0.0769634384766929\\
-2.22044088176353	0.0799090063109667\\
-2.20440881763527	0.0829353267002173\\
-2.18837675350701	0.0860430806432066\\
-2.17234468937876	0.0892328788615129\\
-2.1563126252505	0.0925052582073576\\
-2.14028056112224	0.0958606780825505\\
-2.12424849699399	0.0992995168773978\\
-2.10821643286573	0.10282206843869\\
-2.09218436873747	0.106428538576138\\
-2.07615230460922	0.11011904161686\\
-2.06012024048096	0.113893597017727\\
-2.04408817635271	0.117752126045569\\
-2.02805611222445	0.121694448535406\\
-2.01202404809619	0.125720279736991\\
-1.99599198396794	0.129829227260085\\
-1.97995991983968	0.134020788128934\\
-1.96392785571142	0.138294345956499\\
-1.94789579158317	0.142649168248978\\
-1.93186372745491	0.147084403851169\\
-1.91583166332665	0.151599080543159\\
-1.8997995991984	0.156192102798748\\
-1.88376753507014	0.160862249715903\\
-1.86773547094188	0.165608173129371\\
-1.85170340681363	0.170428395915412\\
-1.83567134268537	0.17532131049836\\
-1.81963927855711	0.180285177568476\\
-1.80360721442886	0.185318125020256\\
-1.7875751503006	0.190418147119989\\
-1.77154308617234	0.195583103911041\\
-1.75551102204409	0.20081072086486\\
-1.73947895791583	0.206098588785302\\
-1.72344689378758	0.211444163973371\\
-1.70741482965932	0.216844768658931\\
-1.69138276553106	0.222297591705432\\
-1.67535070140281	0.227799689593048\\
-1.65931863727455	0.233347987685068\\
-1.64328657314629	0.238939281781671\\
-1.62725450901804	0.244570239964567\\
-1.61122244488978	0.25023740473528\\
-1.59519038076152	0.255937195449088\\
-1.57915831663327	0.261665911045904\\
-1.56312625250501	0.267419733078587\\
-1.54709418837675	0.273194729038368\\
-1.5310621242485	0.27898685597628\\
-1.51503006012024	0.284791964418616\\
-1.49899799599198	0.290605802573626\\
-1.48296593186373	0.296424020825796\\
-1.46693386773547	0.302242176513175\\
-1.45090180360721	0.3080557389824\\
-1.43486973947896	0.313860094915135\\
-1.4188376753507	0.319650553918846\\
-1.40280561122244	0.325422354373925\\
-1.38677354709419	0.331170669528352\\
-1.37074148296593	0.336890613830234\\
-1.35470941883768	0.342577249487751\\
-1.33867735470942	0.34822559324521\\
-1.32264529058116	0.353830623363147\\
-1.30661322645291	0.359387286789639\\
-1.29058116232465	0.36489050650926\\
-1.27454909819639	0.370335189055414\\
-1.25851703406814	0.375716232171097\\
-1.24248496993988	0.381028532602524\\
-1.22645290581162	0.386266994009427\\
-1.21042084168337	0.391426534975322\\
-1.19438877755511	0.39650209710048\\
-1.17835671342685	0.401488653159919\\
-1.1623246492986	0.406381215308277\\
-1.14629258517034	0.411174843313097\\
-1.13026052104208	0.415864652797709\\
-1.11422845691383	0.420445823474659\\
-1.09819639278557	0.424913607350425\\
-1.08216432865731	0.429263336882002\\
-1.06613226452906	0.433490433065874\\
-1.0501002004008	0.437590413439849\\
-1.03406813627255	0.441558899978256\\
-1.01803607214429	0.445391626861134\\
-1.00200400801603	0.449084448098139\\
-0.985971943887776	0.452633344988176\\
-0.969939879759519	0.45603443339598\\
-0.953907815631263	0.459283970827254\\
-0.937875751503006	0.462378363284336\\
-0.92184368737475	0.465314171884835\\
-0.905811623246493	0.468088119226196\\
-0.889779559118236	0.470697095479709\\
-0.87374749498998	0.473138164198116\\
-0.857715430861723	0.475408567821629\\
-0.841683366733467	0.477505732867927\\
-0.82565130260521	0.479427274792443\\
-0.809619238476954	0.481171002506118\\
-0.793587174348697	0.482734922538609\\
-0.777555110220441	0.484117242835916\\
-0.761523046092184	0.485316376182274\\
-0.745490981963928	0.486330943237168\\
-0.729458917835671	0.487159775179357\\
-0.713426853707415	0.48780191595078\\
-0.697394789579158	0.488256624094339\\
-0.681362725450902	0.48852337418061\\
-0.665330661322646	0.488601857819623\\
-0.649298597194389	0.488491984255008\\
-0.633266533066132	0.488193880538881\\
-0.617234468937876	0.487707891287019\\
-0.601202404809619	0.487034578014966\\
-0.585170340681363	0.48617471805688\\
-0.569138276553106	0.485129303070009\\
-0.55310621242485	0.483899537128855\\
-0.537074148296593	0.482486834414125\\
-0.521042084168337	0.480892816502696\\
-0.50501002004008	0.479119309265855\\
-0.488977955911824	0.477168339384123\\
-0.472945891783567	0.475042130487976\\
-0.456913827655311	0.472743098934741\\
-0.440881763527054	0.470273849232905\\
-0.424849699398798	0.467637169125958\\
-0.408817635270541	0.464836024348758\\
-0.392785571142285	0.461873553070221\\
-0.376753507014028	0.458753060036908\\
-0.360721442885771	0.455478010432809\\
-0.344689378757515	0.452052023471268\\
-0.328657314629258	0.448478865735645\\
-0.312625250501002	0.444762444285837\\
-0.296593186372745	0.440906799548306\\
-0.280561122244489	0.436916098007698\\
-0.264529058116232	0.432794624718525\\
-0.248496993987976	0.428546775655714\\
-0.232464929859719	0.424177049923075\\
-0.216432865731463	0.419690041838985\\
-0.200400801603207	0.415090432918668\\
-0.18436873747495	0.410382983772607\\
-0.168336673346694	0.405572525940577\\
-0.152304609218437	0.400663953680806\\
-0.13627254509018	0.395662215733632\\
-0.120240480961924	0.390572307078907\\
-0.104208416833667	0.385399260706149\\
-0.0881763527054109	0.380148139416206\\
-0.0721442885771544	0.374824027672854\\
-0.0561122244488979	0.369432023522392\\
-0.0400801603206413	0.363977230598857\\
-0.0240480961923848	0.358464750232017\\
-0.00801603206412826	0.352899673674783\\
0.00801603206412782	0.347287074466126\\
0.0240480961923843	0.341632000944943\\
0.0400801603206409	0.335939468929744\\
0.0561122244488974	0.330214454578286\\
0.0721442885771539	0.3244618874406\\
0.0881763527054105	0.318686643718147\\
0.104208416833667	0.312893539741007\\
0.120240480961924	0.307087325674278\\
0.13627254509018	0.301272679464027\\
0.152304609218437	0.295454201032297\\
0.168336673346693	0.289636406729862\\
0.18436873747495	0.283823724054549\\
0.200400801603206	0.27802048664209\\
0.216432865731463	0.272230929535613\\
0.232464929859719	0.26645918473901\\
0.248496993987976	0.260709277058544\\
0.264529058116232	0.254985120236227\\
0.280561122244489	0.24929051337761\\
0.296593186372745	0.243629137675817\\
0.312625250501002	0.238004553432813\\
0.328657314629258	0.23242019737806\\
0.344689378757515	0.22687938028397\\
0.360721442885771	0.221385284876718\\
0.376753507014028	0.215940964040297\\
0.392785571142285	0.210549339310889\\
0.408817635270541	0.205213199658003\\
0.424849699398798	0.199935200548079\\
0.440881763527054	0.194717863285665\\
0.456913827655311	0.189563574626628\\
0.472945891783567	0.184474586657292\\
0.488977955911824	0.179453016932837\\
0.50501002004008	0.174500848867783\\
0.521042084168337	0.169619932370901\\
0.537074148296593	0.164811984716458\\
0.55310621242485	0.160078591643272\\
0.569138276553106	0.155421208672723\\
0.585170340681363	0.150841162636488\\
0.601202404809619	0.146339653404519\\
0.617234468937876	0.141917755803476\\
0.633266533066132	0.137576421715659\\
0.649298597194389	0.133316482348257\\
0.665330661322646	0.129138650662606\\
0.681362725450902	0.125043523953029\\
0.697394789579159	0.121031586564762\\
0.713426853707415	0.117103212740417\\
0.729458917835672	0.113258669584431\\
0.745490981963928	0.109498120134982\\
0.761523046092185	0.105821626532874\\
0.777555110220441	0.102229153277031\\
0.793587174348698	0.0987205705562951\\
0.809619238476954	0.0952956576474205\\
0.825651302605211	0.0919541063692656\\
0.841683366733467	0.0886955245834284\\
0.857715430861724	0.0855194397317486\\
0.87374749498998	0.0824253024013527\\
0.889779559118236	0.0794124899081683\\
0.905811623246493	0.0764803098901112\\
0.921843687374749	0.0736280039014362\\
0.937875751503006	0.070854751000052\\
0.953907815631262	0.068159671319919\\
0.969939879759519	0.0655418296209804\\
0.985971943887775	0.0630002388094198\\
1.00200400801603	0.0605338634213905\\
1.01803607214429	0.0581416230637184\\
1.03406813627254	0.0558223958054472\\
1.0501002004008	0.053575021514461\\
1.06613226452906	0.051398305133794\\
1.08216432865731	0.049291019892608\\
1.09819639278557	0.0472519104471938\\
1.11422845691383	0.0452796959477241\\
1.13026052104208	0.0433730730268574\\
1.14629258517034	0.0415307187066586\\
1.1623246492986	0.0397512932206631\\
1.17835671342685	0.0380334427482723\\
1.19438877755511	0.0363758020590141\\
1.21042084168337	0.0347769970645482\\
1.22645290581162	0.0332356472766299\\
1.24248496993988	0.0317503681695706\\
1.25851703406814	0.0303197734460517\\
1.27454909819639	0.0289424772054501\\
1.29058116232465	0.0276170960141327\\
1.30661322645291	0.0263422508774556\\
1.32264529058116	0.0251165691134782\\
1.33867735470942	0.0239386861286589\\
1.35470941883768	0.022807247096046\\
1.37074148296593	0.0217209085367106\\
1.38677354709419	0.020678339805387\\
1.40280561122244	0.0196782244814955\\
1.4188376753507	0.0187192616669125\\
1.43486973947896	0.0178001671920361\\
1.45090180360721	0.0169196747318608\\
1.46693386773547	0.0160765368339273\\
1.48296593186373	0.015269525860158\\
1.49899799599198	0.0144974348447104\\
1.51503006012024	0.0137590782701012\\
1.5310621242485	0.0130532927639533\\
1.54709418837675	0.0123789377188074\\
1.56312625250501	0.0117348958375213\\
1.57915831663327	0.0111200736068429\\
1.59519038076152	0.0105334017018022\\
1.61122244488978	0.00997383532360827\\
1.62725450901804	0.00944035447377577\\
1.64328657314629	0.00893196416722667\\
1.65931863727455	0.00844769458712898\\
1.67535070140281	0.00798660118424005\\
1.69138276553106	0.00754776472351893\\
1.70741482965932	0.0071302912807615\\
1.72344689378758	0.00673331219199304\\
1.73947895791583	0.00635598395832763\\
1.75551102204409	0.00599748810897067\\
1.77154308617235	0.00565703102500252\\
1.7875751503006	0.00533384372653659\\
1.80360721442886	0.00502718162579517\\
1.81963927855711	0.00473632424859236\\
1.83567134268537	0.00446057492665377\\
1.85170340681363	0.0041992604631403\\
1.86773547094188	0.00395173077367688\\
1.88376753507014	0.0037173585051176\\
1.8997995991984	0.00349553863420671\\
1.91583166332665	0.00328568804822075\\
1.93186372745491	0.00308724510960083\\
1.94789579158317	0.00289966920650668\\
1.96392785571142	0.00272244029114513\\
1.97995991983968	0.00255505840764619\\
1.99599198396794	0.00239704321118001\\
2.01202404809619	0.00224793347992735\\
2.02805611222445	0.0021072866214365\\
2.04408817635271	0.00197467817481933\\
2.06012024048096	0.00184970131016033\\
2.07615230460922	0.0017319663264337\\
2.09218436873747	0.00162110014914652\\
2.10821643286573	0.00151674582884978\\
2.12424849699399	0.00141856204158417\\
2.14028056112224	0.00132622259225481\\
2.1563126252505	0.00123941592185707\\
2.17234468937876	0.00115784461940662\\
2.18837675350701	0.00108122493935886\\
2.20440881763527	0.00100928632523749\\
2.22044088176353	0.000941770940128415\\
2.23647294589178	0.000878433204634116\\
2.25250501002004	0.000819039342824684\\
2.2685370741483	0.00076336693666496\\
2.28456913827655	0.000711204489343158\\
2.30060120240481	0.000662350997874363\\
2.31663326653307	0.0006166155353029\\
2.33266533066132	0.000573816842780424\\
2.34869739478958	0.000533782931751961\\
2.36472945891784	0.000496350696439798\\
2.38076152304609	0.000461365536775155\\
2.39679358717435	0.000428680991889996\\
2.41282565130261	0.000398158384245936\\
2.42885771543086	0.000369666474444195\\
2.44488977955912	0.000343081126729589\\
2.46092184368737	0.000318284985172869\\
2.47695390781563	0.000295167160489058\\
2.49298597194389	0.000273622927424801\\
2.50901803607214	0.000253553432625146\\
2.5250501002004	0.000234865412869398\\
2.54108216432866	0.000217470923546791\\
2.55711422845691	0.000201287077225584\\
2.57314629258517	0.000186235792153734\\
2.58917835671343	0.00017224355051546\\
2.60521042084168	0.000159241166255733\\
2.62124248496994	0.000147163562273906\\
2.6372745490982	0.00013594955677828\\
2.65330661322645	0.000125541658585314\\
2.66933867735471	0.000115885871140344\\
2.68537074148297	0.000106931505031013\\
2.70140280561122	9.8630998760062e-05\\
2.71743486973948	9.09397475406338e-05\\
2.73346693386774	8.38159398746764e-05\\
2.74949899799599	7.72204016734268e-05\\
2.76553106212425	7.11164476781384e-05\\
2.7815631262525	6.54697399392167e-05\\
2.79759519038076	6.0248153112616e-05\\
2.81362725450902	5.54216463337182e-05\\
2.82965931863727	5.09621414308765e-05\\
2.84569138276553	4.68434072433078e-05\\
2.86172344689379	4.30409498110364e-05\\
2.87775551102204	3.95319082080304e-05\\
2.8937875751503	3.62949557935356e-05\\
2.90981963927856	3.33102066608047e-05\\
2.92585170340681	3.0559127066948e-05\\
2.94188376753507	2.80244516324077e-05\\
2.95791583166333	2.56901041035878e-05\\
2.97394789579158	2.35411224773838e-05\\
2.98997995991984	2.1563588291738e-05\\
3.0060120240481	1.9744559891863e-05\\
3.02204408817635	1.8072009487388e-05\\
3.03807615230461	1.6534763821377e-05\\
3.05410821643287	1.51224482779098e-05\\
3.07014028056112	1.38254342606862e-05\\
3.08617234468938	1.26347896808882e-05\\
3.10220440881764	1.15422323982966e-05\\
3.11823647294589	1.05400864653892e-05\\
3.13426853707415	9.62124102982533e-06\\
3.1503006012024	8.77911175634277e-06\\
3.16633266533066	8.00760463463436e-06\\
3.18236472945892	7.30108204522499e-06\\
3.19839679358717	6.65433096072608e-06\\
3.21442885771543	6.0625331650941e-06\\
3.23046092184369	5.52123737865054e-06\\
3.24649298597194	5.02633318163401e-06\\
3.2625250501002	4.57402663393557e-06\\
3.27855711422846	4.16081749341989e-06\\
3.29458917835671	3.78347793984667e-06\\
3.31062124248497	3.43903271588089e-06\\
3.32665330661323	3.1247406010111e-06\\
3.34268537074148	2.83807713838261e-06\\
3.35871743486974	2.57671853859603e-06\\
3.374749498998	2.33852668841948e-06\\
3.39078156312625	2.12153519611733e-06\\
3.40681362725451	1.92393640870865e-06\\
3.42284569138277	1.74406933993667e-06\\
3.43887775551102	1.58040845105803e-06\\
3.45490981963928	1.43155322974979e-06\\
3.47094188376754	1.29621851548408e-06\\
3.48697394789579	1.17322552263991e-06\\
3.50300601202405	1.06149351540935e-06\\
3.51903807615231	9.60032091215739e-07\\
3.53507014028056	8.67934031897843e-07\\
3.55110220440882	7.84368684328637e-07\\
3.56713426853707	7.08575834434872e-07\\
3.58316633266533	6.39860040767387e-07\\
3.59919839679359	5.77585395845381e-07\\
3.61523046092184	5.2117068546503e-07\\
3.6312625250501	4.70084918027108e-07\\
3.64729458917836	4.23843197703918e-07\\
3.66332665330661	3.82002916936139e-07\\
3.67935871743487	3.44160245329457e-07\\
3.69539078156313	3.09946893512289e-07\\
3.71142284569138	2.79027131923597e-07\\
3.72745490981964	2.51095045827143e-07\\
3.7434869739479	2.25872009099156e-07\\
3.75951903807615	2.0310436051388e-07\\
3.77555110220441	1.82561267359034e-07\\
3.79158316633267	1.64032762254422e-07\\
3.80761523046092	1.47327940024804e-07\\
3.82364729458918	1.32273302396024e-07\\
3.83967935871743	1.18711239144071e-07\\
3.85571142284569	1.06498635133528e-07\\
3.87174348697395	9.55055934372082e-08\\
3.8877755511022	8.56142654357241e-08\\
3.90380761523046	7.67177794567676e-08\\
3.91983967935872	6.87192601315789e-08\\
3.93587174348697	6.15309312229083e-08\\
3.95190380761523	5.50732952169927e-08\\
3.96793587174349	4.92743834740051e-08\\
3.98396793587174	4.40690711991154e-08\\
4	3.93984519318621e-08\\
};
\addlegendentry{$p(x_1 \given x_2 = 2)$};

\addplot [color=mycolor2,solid]
  table[row sep=crcr]{%
-4	0.000133830225764885\\
-3.98396793587174	0.000142675349741047\\
-3.96793587174349	0.000152065976657072\\
-3.95190380761523	0.000162033024993012\\
-3.93587174348697	0.000172608985016727\\
-3.91983967935872	0.000183827987316515\\
-3.90380761523046	0.000195725873640051\\
-3.8877755511022	0.000208340270077416\\
-3.87174348697395	0.000221710662624091\\
-3.85571142284569	0.000235878475157708\\
-3.83967935871743	0.00025088714986005\\
-3.82364729458918	0.000266782230113281\\
-3.80761523046092	0.000283611445896645\\
-3.79158316633267	0.000301424801706908\\
-3.77555110220441	0.000320274667022632\\
-3.75951903807615	0.00034021586932888\\
-3.7434869739479	0.000361305789715299\\
-3.72745490981964	0.000383604461056511\\
-3.71142284569138	0.000407174668779565\\
-3.69539078156313	0.000432082054218673\\
-3.67935871743487	0.000458395220552653\\
-3.66332665330661	0.000486185841315487\\
-3.64729458917836	0.000515528771464996\\
-3.6312625250501	0.000546502160988996\\
-3.61523046092184	0.000579187571022355\\
-3.59919839679359	0.000613670092442086\\
-3.58316633266533	0.000650038466901082\\
-3.56713426853707	0.00068838521025417\\
-3.55110220440882	0.000728806738323016\\
-3.53507014028056	0.000771403494938878\\
-3.5190380761523	0.000816280082194388\\
-3.50300601202405	0.000863545392827412\\
-3.48697394789579	0.000913312744651563\\
-3.47094188376753	0.000965700016939215\\
-3.45490981963928	0.00102082978865373\\
-3.43887775551102	0.00107882947841831\\
-3.42284569138277	0.00113983148609902\\
-3.40681362725451	0.00120397333586992\\
-3.39078156312625	0.00127139782061737\\
-3.374749498998	0.00134225314753069\\
-3.35871743486974	0.00141669308471503\\
-3.34268537074148	0.00149487710865133\\
-3.32665330661323	0.00157697055231716\\
-3.31062124248497	0.00166314475377022\\
-3.29458917835671	0.00175357720498483\\
-3.27855711422846	0.0018484517007196\\
-3.2625250501002	0.0019479584871821\\
-3.24649298597194	0.00205229441024435\\
-3.23046092184369	0.00216166306294995\\
-3.21442885771543	0.00227627493204146\\
-3.19839679358717	0.00239634754322351\\
-3.18236472945892	0.00252210560486446\\
-3.16633266533066	0.00265378114982641\\
-3.1503006012024	0.00279161367510061\\
-3.13426853707415	0.002935850278912\\
-3.11823647294589	0.00308674579494412\\
-3.10220440881764	0.00324456292332248\\
-3.08617234468938	0.00340957235798186\\
-3.07014028056112	0.00358205291003031\\
-3.05410821643287	0.00376229162671013\\
-3.03807615230461	0.00395058390554404\\
-3.02204408817635	0.0041472336032425\\
-3.0060120240481	0.00435255313893657\\
-2.98997995991984	0.00456686359128916\\
-2.97394789579158	0.00479049478902667\\
-2.95791583166333	0.00502378539442237\\
-2.94188376753507	0.00526708297925248\\
-2.92585170340681	0.00552074409273662\\
-2.90981963927856	0.005785134320965\\
-2.8937875751503	0.00606062833730627\\
-2.87775551102204	0.00634760994328225\\
-2.86172344689379	0.00664647209938829\\
-2.84569138276553	0.00695761694533229\\
-2.82965931863727	0.00728145580915903\\
-2.81362725450902	0.00761840920472244\\
-2.79759519038076	0.00796890681696439\\
-2.7815631262525	0.00833338747445562\\
-2.76553106212425	0.00871229910865282\\
-2.74949899799599	0.0091060986993251\\
-2.73346693386774	0.00951525220560328\\
-2.71743486973948	0.00994023448210711\\
-2.70140280561122	0.0103815291796084\\
-2.68537074148297	0.0108396286296914\\
-2.66933867735471	0.0113150337128785\\
-2.65330661322645	0.0118082537096938\\
-2.6372745490982	0.0123198061341476\\
-2.62124248496994	0.0128502165491321\\
-2.60521042084168	0.0134000183632325\\
-2.58917835671343	0.013969752608466\\
-2.57314629258517	0.0145599676984796\\
-2.55711422845691	0.015171219166751\\
-2.54108216432866	0.0158040693843541\\
-2.5250501002004	0.016459087256871\\
-2.50901803607214	0.0171368479000506\\
-2.49298597194389	0.0178379322938401\\
-2.47695390781563	0.0185629269144352\\
-2.46092184368737	0.0193124233440244\\
-2.44488977955912	0.0200870178579282\\
-2.42885771543086	0.0208873109888638\\
-2.41282565130261	0.021713907068098\\
-2.39679358717435	0.0225674137432813\\
-2.38076152304609	0.0234484414727945\\
-2.36472945891784	0.0243576029964712\\
-2.34869739478958	0.0252955127826016\\
-2.33266533066132	0.0262627864511589\\
-2.31663326653307	0.0272600401732346\\
-2.30060120240481	0.0282878900467097\\
-2.28456913827655	0.0293469514482331\\
-2.2685370741483	0.0304378383616278\\
-2.25250501002004	0.031561162682888\\
-2.23647294589178	0.0327175335019856\\
-2.22044088176353	0.033907556361749\\
-2.20440881763527	0.0351318324941335\\
-2.18837675350701	0.0363909580342532\\
-2.17234468937876	0.0376855232125996\\
-2.1563126252505	0.0390161115259268\\
-2.14028056112224	0.0403832988873405\\
-2.12424849699399	0.0417876527561837\\
-2.10821643286573	0.0432297312483712\\
-2.09218436873747	0.0447100822278847\\
-2.07615230460922	0.0462292423801957\\
-2.06012024048096	0.0477877362684481\\
-2.04408817635271	0.0493860753732887\\
-2.02805611222445	0.0510247571172962\\
-2.01202404809619	0.052704263875018\\
-1.99599198396794	0.0544250619696859\\
-1.97995991983968	0.0561876006577405\\
-1.96392785571142	0.0579923111023547\\
-1.94789579158317	0.0598396053372053\\
-1.93186372745491	0.0617298752217999\\
-1.91583166332665	0.0636634913897248\\
-1.8997995991984	0.0656408021912354\\
-1.88376753507014	0.0676621326316657\\
-1.86773547094188	0.0697277833071892\\
-1.85170340681363	0.0718380293395149\\
-1.83567134268537	0.0739931193111512\\
-1.81963927855711	0.0761932742029252\\
-1.80360721442886	0.078438686335485\\
-1.7875751503006	0.0807295183165622\\
-1.77154308617234	0.0830659019958129\\
-1.75551102204409	0.0854479374290955\\
-1.73947895791583	0.0878756918540804\\
-1.72344689378758	0.0903491986791223\\
-1.70741482965932	0.0928684564873552\\
-1.69138276553106	0.0954334280580022\\
-1.67535070140281	0.098044039406911\\
-1.65931863727455	0.100700178848355\\
-1.64328657314629	0.103401696080149\\
-1.62725450901804	0.106148401294153\\
-1.61122244488978	0.108940064314234\\
-1.59519038076152	0.111776413763773\\
-1.57915831663327	0.114657136264811\\
-1.56312625250501	0.117581875670895\\
-1.54709418837675	0.120550232335723\\
-1.5310621242485	0.12356176241964\\
-1.51503006012024	0.126615977236029\\
-1.49899799599198	0.129712342639645\\
-1.48296593186373	0.132850278458862\\
-1.46693386773547	0.136029157973839\\
-1.45090180360721	0.139248307442514\\
-1.43486973947896	0.142507005676342\\
-1.4188376753507	0.145804483667619\\
-1.40280561122244	0.149139924270191\\
-1.38677354709419	0.152512461935312\\
-1.37074148296593	0.155921182504311\\
-1.35470941883768	0.159365123059726\\
-1.33867735470942	0.162843271836423\\
-1.32264529058116	0.166354568194204\\
-1.30661322645291	0.169897902653299\\
-1.29058116232465	0.173472116994052\\
-1.27454909819639	0.177076004422035\\
-1.25851703406814	0.180708309799727\\
-1.24248496993988	0.184367729945797\\
-1.22645290581162	0.188052914002922\\
-1.21042084168337	0.191762463874985\\
-1.19438877755511	0.195494934734372\\
-1.17835671342685	0.199248835599966\\
-1.1623246492986	0.203022629986358\\
-1.14629258517034	0.206814736624628\\
-1.13026052104208	0.210623530254964\\
-1.11422845691383	0.214447342491225\\
-1.09819639278557	0.218284462757478\\
-1.08216432865731	0.222133139296347\\
-1.06613226452906	0.225991580248921\\
-1.0501002004008	0.229857954805832\\
-1.03406813627255	0.23373039442894\\
-1.01803607214429	0.237606994142988\\
-1.00200400801603	0.241485813896379\\
-0.985971943887776	0.245364879990149\\
-0.969939879759519	0.249242186574038\\
-0.953907815631263	0.253115697208426\\
-0.937875751503006	0.256983346490764\\
-0.92184368737475	0.260843041745\\
-0.905811623246493	0.264692664772344\\
-0.889779559118236	0.268530073661598\\
-0.87374749498998	0.272353104657127\\
-0.857715430861723	0.276159574082418\\
-0.841683366733467	0.279947280317065\\
-0.82565130260521	0.283714005824839\\
-0.809619238476954	0.287457519230438\\
-0.793587174348697	0.291175577442345\\
-0.777555110220441	0.29486592781912\\
-0.761523046092184	0.29852631037635\\
-0.745490981963928	0.302154460031336\\
-0.729458917835671	0.305748108882535\\
-0.713426853707415	0.309304988520637\\
-0.697394789579158	0.31282283236809\\
-0.681362725450902	0.316299378043765\\
-0.665330661322646	0.319732369749404\\
-0.649298597194389	0.323119560674393\\
-0.633266533066132	0.326458715415327\\
-0.617234468937876	0.329747612406789\\
-0.601202404809619	0.332984046359693\\
-0.585170340681363	0.336165830703478\\
-0.569138276553106	0.339290800028423\\
-0.55310621242485	0.342356812524294\\
-0.537074148296593	0.345361752411504\\
-0.521042084168337	0.348303532360969\\
-0.50501002004008	0.351180095898786\\
-0.488977955911824	0.353989419791894\\
-0.472945891783567	0.356729516410858\\
-0.456913827655311	0.359398436065926\\
-0.440881763527054	0.361994269312532\\
-0.424849699398798	0.364515149222442\\
-0.408817635270541	0.366959253616761\\
-0.392785571142285	0.369324807257091\\
-0.376753507014028	0.371610083991139\\
-0.360721442885771	0.373813408849153\\
-0.344689378757515	0.375933160087644\\
-0.328657314629258	0.377967771176883\\
-0.312625250501002	0.379915732728785\\
-0.296593186372745	0.381775594361851\\
-0.280561122244489	0.38354596649995\\
-0.264529058116232	0.385225522101812\\
-0.248496993987976	0.386812998318237\\
-0.232464929859719	0.388307198074096\\
-0.216432865731463	0.389706991572382\\
-0.200400801603207	0.391011317717639\\
-0.18436873747495	0.392219185456281\\
-0.168336673346694	0.393329675031411\\
-0.152304609218437	0.394341939149931\\
-0.13627254509018	0.395255204059869\\
-0.120240480961924	0.396068770535991\\
-0.104208416833667	0.39678201477196\\
-0.0881763527054109	0.397394389177437\\
-0.0721442885771544	0.397905423078697\\
-0.0561122244488979	0.398314723321516\\
-0.0400801603206413	0.398621974775231\\
-0.0240480961923848	0.398826940737091\\
-0.00801603206412826	0.398929463236143\\
0.00801603206412782	0.398929463236143\\
0.0240480961923843	0.398826940737091\\
0.0400801603206409	0.398621974775231\\
0.0561122244488974	0.398314723321516\\
0.0721442885771539	0.397905423078697\\
0.0881763527054105	0.397394389177437\\
0.104208416833667	0.39678201477196\\
0.120240480961924	0.396068770535991\\
0.13627254509018	0.395255204059869\\
0.152304609218437	0.394341939149931\\
0.168336673346693	0.393329675031411\\
0.18436873747495	0.392219185456281\\
0.200400801603206	0.391011317717639\\
0.216432865731463	0.389706991572382\\
0.232464929859719	0.388307198074096\\
0.248496993987976	0.386812998318237\\
0.264529058116232	0.385225522101812\\
0.280561122244489	0.38354596649995\\
0.296593186372745	0.381775594361851\\
0.312625250501002	0.379915732728785\\
0.328657314629258	0.377967771176883\\
0.344689378757515	0.375933160087644\\
0.360721442885771	0.373813408849153\\
0.376753507014028	0.371610083991139\\
0.392785571142285	0.369324807257091\\
0.408817635270541	0.366959253616761\\
0.424849699398798	0.364515149222442\\
0.440881763527054	0.361994269312532\\
0.456913827655311	0.359398436065926\\
0.472945891783567	0.356729516410858\\
0.488977955911824	0.353989419791894\\
0.50501002004008	0.351180095898786\\
0.521042084168337	0.348303532360969\\
0.537074148296593	0.345361752411504\\
0.55310621242485	0.342356812524294\\
0.569138276553106	0.339290800028423\\
0.585170340681363	0.336165830703478\\
0.601202404809619	0.332984046359693\\
0.617234468937876	0.329747612406789\\
0.633266533066132	0.326458715415327\\
0.649298597194389	0.323119560674393\\
0.665330661322646	0.319732369749404\\
0.681362725450902	0.316299378043765\\
0.697394789579159	0.31282283236809\\
0.713426853707415	0.309304988520637\\
0.729458917835672	0.305748108882535\\
0.745490981963928	0.302154460031336\\
0.761523046092185	0.29852631037635\\
0.777555110220441	0.29486592781912\\
0.793587174348698	0.291175577442344\\
0.809619238476954	0.287457519230438\\
0.825651302605211	0.283714005824839\\
0.841683366733467	0.279947280317065\\
0.857715430861724	0.276159574082418\\
0.87374749498998	0.272353104657126\\
0.889779559118236	0.268530073661598\\
0.905811623246493	0.264692664772344\\
0.921843687374749	0.260843041745\\
0.937875751503006	0.256983346490764\\
0.953907815631262	0.253115697208426\\
0.969939879759519	0.249242186574039\\
0.985971943887775	0.245364879990149\\
1.00200400801603	0.241485813896379\\
1.01803607214429	0.237606994142988\\
1.03406813627254	0.23373039442894\\
1.0501002004008	0.229857954805832\\
1.06613226452906	0.225991580248922\\
1.08216432865731	0.222133139296347\\
1.09819639278557	0.218284462757478\\
1.11422845691383	0.214447342491225\\
1.13026052104208	0.210623530254964\\
1.14629258517034	0.206814736624628\\
1.1623246492986	0.203022629986358\\
1.17835671342685	0.199248835599966\\
1.19438877755511	0.195494934734372\\
1.21042084168337	0.191762463874985\\
1.22645290581162	0.188052914002922\\
1.24248496993988	0.184367729945797\\
1.25851703406814	0.180708309799727\\
1.27454909819639	0.177076004422035\\
1.29058116232465	0.173472116994052\\
1.30661322645291	0.169897902653299\\
1.32264529058116	0.166354568194204\\
1.33867735470942	0.162843271836423\\
1.35470941883768	0.159365123059726\\
1.37074148296593	0.155921182504311\\
1.38677354709419	0.152512461935312\\
1.40280561122244	0.149139924270191\\
1.4188376753507	0.145804483667619\\
1.43486973947896	0.142507005676342\\
1.45090180360721	0.139248307442514\\
1.46693386773547	0.136029157973839\\
1.48296593186373	0.132850278458862\\
1.49899799599198	0.129712342639645\\
1.51503006012024	0.126615977236029\\
1.5310621242485	0.12356176241964\\
1.54709418837675	0.120550232335723\\
1.56312625250501	0.117581875670895\\
1.57915831663327	0.114657136264811\\
1.59519038076152	0.111776413763773\\
1.61122244488978	0.108940064314234\\
1.62725450901804	0.106148401294153\\
1.64328657314629	0.103401696080149\\
1.65931863727455	0.100700178848355\\
1.67535070140281	0.098044039406911\\
1.69138276553106	0.0954334280580021\\
1.70741482965932	0.0928684564873551\\
1.72344689378758	0.0903491986791222\\
1.73947895791583	0.0878756918540804\\
1.75551102204409	0.0854479374290954\\
1.77154308617235	0.0830659019958128\\
1.7875751503006	0.0807295183165622\\
1.80360721442886	0.0784386863354851\\
1.81963927855711	0.0761932742029253\\
1.83567134268537	0.0739931193111512\\
1.85170340681363	0.0718380293395149\\
1.86773547094188	0.0697277833071893\\
1.88376753507014	0.0676621326316657\\
1.8997995991984	0.0656408021912355\\
1.91583166332665	0.0636634913897249\\
1.93186372745491	0.0617298752217999\\
1.94789579158317	0.0598396053372053\\
1.96392785571142	0.0579923111023548\\
1.97995991983968	0.0561876006577406\\
1.99599198396794	0.054425061969686\\
2.01202404809619	0.052704263875018\\
2.02805611222445	0.0510247571172962\\
2.04408817635271	0.0493860753732887\\
2.06012024048096	0.0477877362684481\\
2.07615230460922	0.0462292423801957\\
2.09218436873747	0.0447100822278847\\
2.10821643286573	0.0432297312483712\\
2.12424849699399	0.0417876527561837\\
2.14028056112224	0.0403832988873405\\
2.1563126252505	0.0390161115259268\\
2.17234468937876	0.0376855232125996\\
2.18837675350701	0.0363909580342532\\
2.20440881763527	0.0351318324941335\\
2.22044088176353	0.033907556361749\\
2.23647294589178	0.0327175335019856\\
2.25250501002004	0.031561162682888\\
2.2685370741483	0.0304378383616278\\
2.28456913827655	0.0293469514482331\\
2.30060120240481	0.0282878900467097\\
2.31663326653307	0.0272600401732346\\
2.33266533066132	0.0262627864511589\\
2.34869739478958	0.0252955127826016\\
2.36472945891784	0.0243576029964712\\
2.38076152304609	0.0234484414727945\\
2.39679358717435	0.0225674137432813\\
2.41282565130261	0.021713907068098\\
2.42885771543086	0.0208873109888638\\
2.44488977955912	0.0200870178579282\\
2.46092184368737	0.0193124233440244\\
2.47695390781563	0.0185629269144352\\
2.49298597194389	0.0178379322938401\\
2.50901803607214	0.0171368479000506\\
2.5250501002004	0.016459087256871\\
2.54108216432866	0.0158040693843541\\
2.55711422845691	0.015171219166751\\
2.57314629258517	0.0145599676984796\\
2.58917835671343	0.013969752608466\\
2.60521042084168	0.0134000183632325\\
2.62124248496994	0.0128502165491321\\
2.6372745490982	0.0123198061341475\\
2.65330661322645	0.0118082537096938\\
2.66933867735471	0.0113150337128785\\
2.68537074148297	0.0108396286296914\\
2.70140280561122	0.0103815291796084\\
2.71743486973948	0.00994023448210712\\
2.73346693386774	0.00951525220560329\\
2.74949899799599	0.00910609869932512\\
2.76553106212425	0.00871229910865283\\
2.7815631262525	0.00833338747445562\\
2.79759519038076	0.00796890681696439\\
2.81362725450902	0.00761840920472244\\
2.82965931863727	0.00728145580915903\\
2.84569138276553	0.00695761694533229\\
2.86172344689379	0.00664647209938829\\
2.87775551102204	0.00634760994328225\\
2.8937875751503	0.00606062833730627\\
2.90981963927856	0.005785134320965\\
2.92585170340681	0.00552074409273662\\
2.94188376753507	0.00526708297925248\\
2.95791583166333	0.00502378539442237\\
2.97394789579158	0.00479049478902667\\
2.98997995991984	0.00456686359128916\\
3.0060120240481	0.00435255313893657\\
3.02204408817635	0.0041472336032425\\
3.03807615230461	0.00395058390554404\\
3.05410821643287	0.00376229162671013\\
3.07014028056112	0.00358205291003031\\
3.08617234468938	0.00340957235798186\\
3.10220440881764	0.00324456292332248\\
3.11823647294589	0.00308674579494412\\
3.13426853707415	0.002935850278912\\
3.1503006012024	0.00279161367510061\\
3.16633266533066	0.00265378114982641\\
3.18236472945892	0.00252210560486446\\
3.19839679358717	0.00239634754322351\\
3.21442885771543	0.00227627493204146\\
3.23046092184369	0.00216166306294995\\
3.24649298597194	0.00205229441024435\\
3.2625250501002	0.0019479584871821\\
3.27855711422846	0.0018484517007196\\
3.29458917835671	0.00175357720498483\\
3.31062124248497	0.00166314475377022\\
3.32665330661323	0.00157697055231716\\
3.34268537074148	0.00149487710865133\\
3.35871743486974	0.00141669308471503\\
3.374749498998	0.00134225314753069\\
3.39078156312625	0.00127139782061736\\
3.40681362725451	0.00120397333586992\\
3.42284569138277	0.00113983148609902\\
3.43887775551102	0.00107882947841831\\
3.45490981963928	0.00102082978865373\\
3.47094188376754	0.000965700016939214\\
3.48697394789579	0.000913312744651562\\
3.50300601202405	0.00086354539282741\\
3.51903807615231	0.000816280082194387\\
3.53507014028056	0.000771403494938876\\
3.55110220440882	0.000728806738323014\\
3.56713426853707	0.000688385210254171\\
3.58316633266533	0.000650038466901083\\
3.59919839679359	0.000613670092442087\\
3.61523046092184	0.000579187571022356\\
3.6312625250501	0.000546502160988997\\
3.64729458917836	0.000515528771464997\\
3.66332665330661	0.000486185841315487\\
3.67935871743487	0.000458395220552653\\
3.69539078156313	0.000432082054218674\\
3.71142284569138	0.000407174668779566\\
3.72745490981964	0.000383604461056511\\
3.7434869739479	0.0003613057897153\\
3.75951903807615	0.000340215869328881\\
3.77555110220441	0.000320274667022632\\
3.79158316633267	0.000301424801706908\\
3.80761523046092	0.000283611445896645\\
3.82364729458918	0.000266782230113281\\
3.83967935871743	0.00025088714986005\\
3.85571142284569	0.000235878475157708\\
3.87174348697395	0.000221710662624091\\
3.8877755511022	0.000208340270077416\\
3.90380761523046	0.000195725873640051\\
3.91983967935872	0.000183827987316515\\
3.93587174348697	0.000172608985016727\\
3.95190380761523	0.000162033024993012\\
3.96793587174349	0.000152065976657072\\
3.98396793587174	0.000142675349741047\\
4	0.000133830225764885\\
};
\addlegendentry{$p(x_1)$};

\end{axis}
\end{tikzpicture}%
    \caption{$p(x_1 \given x_2, \vec{\mu}, \mat{\Sigma}) =
      \mc{N}\bigl(x_1; -\nicefrac{2}{3}, (\nicefrac{2}{3})^2\bigr)$}
    \label{conditional_pdf}
  \end{subfigure}
  \caption{Conditioning example.  (\subref{conditional_2d_pdf}) shows
    the joint density over $\vec{x} = [x_1, x_2]\trans$, along with
    the observation value $x_2 = 2$; this is the same density as in
    Figure \ref{2d_examples}(\subref{2d_example_3}).
    (\subref{conditional_pdf}) shows the conditional density of
    $x_1$ given $x_2 = 2$.}
  \label{conditional_example}
\end{figure}

\subsection*{Pointwise multiplication}

Another remarkable fact about multivariate Gaussian density functions
is that pointwise multiplication gives another (unnormalized) Gaussian
\acro{PDF}:
\begin{equation*}
  \mc{N}(\vec{x}; \vec{\mu}, \mat{\Sigma})
  \,
  \mc{N}(\vec{x}; \vec{\nu}, \mat{P})
  =
  \frac{1}{Z}
  \mc{N}(\vec{x}; \vec{\omega}, \mat{T}),
\end{equation*}
where
\begin{align*}
  \mat{T}
  &=
  (\mat{\Sigma}\inv + \mat{P}\inv)\inv \\
  \vec{\omega}
  &=
  \mat{T}
  (\mat{\Sigma}\inv \vec{\mu} +
  \mat{P}\inv \vec{\nu}) \\
  Z^{-1}
  &=
  \mc{N}(\vec{\mu}; \vec{\nu}, \mat{\Sigma} + \mat{P})
  =
  \mc{N}(\vec{\nu}; \vec{\mu}, \mat{\Sigma} + \mat{P}).
\end{align*}

\subsection*{Convolutions}

Gaussian probability density functions are closed under convolutions.
Let $\vec{x}$ and $\vec{y}$ be $d$-dimensional vectors, with
distributions
\begin{equation*}
  p(\vec{x} \given \vec{\mu}, \mat{\Sigma})
  =
  \mc{N}(\vec{x}; \vec{\mu}, \mat{\Sigma});
  \qquad
  p(\vec{y} \given \vec{\nu}, \mat{P})
  =
  \mc{N}(\vec{y}; \vec{\nu}, \mat{P}).
\end{equation*}
Then the convolution of their density functions is another Gaussian
\acro{PDF}:
\begin{equation*}
  f(\vec{y}) =
  \int
  \mc{N}(\vec{y} - \vec{x}; \vec{\nu}, \mat{P})
  \,
  \mc{N}(\vec{x}; \vec{\mu}, \mat{\Sigma})
  \intd{\vec{x}}
  =
  \mc{N}(\vec{y}; \vec{\mu} + \vec{\nu}, \mat{\Sigma} + \mat{P}),
\end{equation*}
where the mean and covariances add in the result.

If we assume that $\vec{x}$ and $\vec{y}$ are independent, then the
distribution of their sum $\vec{z} = \vec{x} + \vec{y}$ will also have
a multivariate Gaussian distribution, whose density will precisely the
convolution of the individual densities:
\begin{equation*}
  p(\vec{z} \given \vec{\mu}, \vec{\nu}, \mat{\Sigma}, \mat{P})
  =
  \mc{N}(\vec{z}; \vec{\mu} + \vec{\nu}, \mat{\Sigma} + \mat{P}).
\end{equation*}

These results will often come in handy.

\subsection*{Affine transformations}

Consider a $d$-dimensional vector $\vec{x}$ with a multivariate
Gaussian distribution:
\begin{equation*}
  p(\vec{x} \given \vec{\mu}, \mat{\Sigma})
  =
  \mc{N}(\vec{x}, \vec{\mu}, \mat{\Sigma}).
\end{equation*}
Suppose we wish to reason about an affine transformation of $\vec{x}$
into $\R^D$, $\vec{y} = \mat{A}\vec{x} + \vec{b}$, where $\mat{A} \in
\R^{D \times d}$ and $\vec{b} \in \R^D$.  Then $\vec{y}$ has a
$D$-dimensional Gaussian distribution:
\begin{equation*}
  p(\vec{y} \given \vec{\mu}, \mat{\Sigma}, \mat{A}, \vec{b})
  =
  \mc{N}(\vec{y}, \mat{A}\vec{\mu} + \vec{b}, \mat{A}\mat{\Sigma}\mat{A}\trans).
\end{equation*}

\begin{figure}
  \centering
  \begin{subfigure}[t]{0.49\textwidth}
    % This file was created by matlab2tikz.
% Minimal pgfplots version: 1.3
%
\tikzsetnextfilename{2d_gaussian_pdf_3_large}
\definecolor{mycolor1}{rgb}{0.01430,0.01430,0.01430}%
\definecolor{mycolor2}{rgb}{0.15932,0.06827,0.17506}%
\definecolor{mycolor3}{rgb}{0.17345,0.11709,0.41691}%
\definecolor{mycolor4}{rgb}{0.10466,0.22842,0.49922}%
\definecolor{mycolor5}{rgb}{0.03136,0.34573,0.47968}%
\definecolor{mycolor6}{rgb}{0.00003,0.46181,0.36160}%
\definecolor{mycolor7}{rgb}{0.00000,0.57116,0.23204}%
\definecolor{mycolor8}{rgb}{0.09251,0.67012,0.06175}%
\definecolor{mycolor9}{rgb}{0.37724,0.75416,0.00000}%
\definecolor{mycolor10}{rgb}{0.74811,0.78629,0.07418}%
\definecolor{mycolor11}{rgb}{0.94890,0.82638,0.64748}%
\definecolor{mycolor12}{rgb}{0.96920,0.92730,0.89610}%
%
\begin{tikzpicture}

\begin{axis}[%
width=0.95092\squarefigurewidth,
height=\squarefigureheight,
at={(0\squarefigurewidth,0\squarefigureheight)},
scale only axis,
xmin=-4,
xmax=4,
xlabel={$x_1$},
ymin=-4,
ymax=4,
ylabel={$x_2$},
axis x line*=bottom,
axis y line*=left
]

\addplot[area legend,solid,fill=mycolor1,draw=black,forget plot]
table[row sep=crcr] {%
x	y\\
-4	4.00000000053783\\
-3.98396793587174	4.00000000057334\\
-3.96793587174349	4.00000000061096\\
-3.95190380761523	4.0000000006508\\
-3.93587174348697	4.00000000069296\\
-3.91983967935872	4.00000000073758\\
-3.90380761523046	4.00000000078476\\
-3.8877755511022	4.00000000083464\\
-3.87174348697395	4.00000000088735\\
-3.85571142284569	4.00000000094302\\
-3.83967935871743	4.0000000010018\\
-3.82364729458918	4.00000000106383\\
-3.80761523046092	4.00000000112927\\
-3.79158316633267	4.00000000119827\\
-3.77555110220441	4.000000001271\\
-3.75951903807615	4.00000000134762\\
-3.7434869739479	4.00000000142832\\
-3.72745490981964	4.00000000151325\\
-3.71142284569138	4.00000000160263\\
-3.69539078156313	4.00000000169662\\
-3.67935871743487	4.00000000179544\\
-3.66332665330661	4.00000000189928\\
-3.64729458917836	4.00000000200835\\
-3.6312625250501	4.00000000212287\\
-3.61523046092184	4.00000000224305\\
-3.59919839679359	4.00000000236912\\
-3.58316633266533	4.00000000250131\\
-3.56713426853707	4.00000000263986\\
-3.55110220440882	4.00000000278501\\
-3.53507014028056	4.00000000293701\\
-3.5190380761523	4.00000000309611\\
-3.50300601202405	4.00000000326257\\
-3.48697394789579	4.00000000343666\\
-3.47094188376753	4.00000000361864\\
-3.45490981963928	4.00000000380879\\
-3.43887775551102	4.00000000400738\\
-3.42284569138277	4.00000000421471\\
-3.40681362725451	4.00000000443105\\
-3.39078156312625	4.0000000046567\\
-3.374749498998	4.00000000489195\\
-3.35871743486974	4.00000000513711\\
-3.34268537074148	4.00000000539247\\
-3.32665330661323	4.00000000565835\\
-3.31062124248497	4.00000000593505\\
-3.29458917835671	4.00000000622288\\
-3.27855711422846	4.00000000652215\\
-3.2625250501002	4.00000000683318\\
-3.24649298597194	4.00000000715629\\
-3.23046092184369	4.00000000749178\\
-3.21442885771543	4.00000000783997\\
-3.19839679358717	4.00000000820119\\
-3.18236472945892	4.00000000857575\\
-3.16633266533066	4.00000000896395\\
-3.1503006012024	4.00000000936611\\
-3.13426853707415	4.00000000978255\\
-3.11823647294589	4.00000001021356\\
-3.10220440881764	4.00000001065945\\
-3.08617234468938	4.00000001112052\\
-3.07014028056112	4.00000001159706\\
-3.05410821643287	4.00000001208936\\
-3.03807615230461	4.0000000125977\\
-3.02204408817635	4.00000001312236\\
-3.0060120240481	4.00000001366359\\
-2.98997995991984	4.00000001422167\\
-2.97394789579158	4.00000001479684\\
-2.95791583166333	4.00000001538933\\
-2.94188376753507	4.00000001599937\\
-2.92585170340681	4.00000001662719\\
-2.90981963927856	4.00000001727298\\
-2.8937875751503	4.00000001793694\\
-2.87775551102204	4.00000001861924\\
-2.86172344689379	4.00000001932004\\
-2.84569138276553	4.00000002003949\\
-2.82965931863727	4.00000002077772\\
-2.81362725450902	4.00000002153485\\
-2.79759519038076	4.00000002231096\\
-2.7815631262525	4.00000002310613\\
-2.76553106212425	4.00000002392041\\
-2.74949899799599	4.00000002475385\\
-2.73346693386774	4.00000002560645\\
-2.71743486973948	4.0000000264782\\
-2.70140280561122	4.00000002736908\\
-2.68537074148297	4.00000002827904\\
-2.66933867735471	4.00000002920798\\
-2.65330661322645	4.0000000301558\\
-2.6372745490982	4.00000003112239\\
-2.62124248496994	4.00000003210757\\
-2.60521042084168	4.00000003311117\\
-2.58917835671343	4.00000003413298\\
-2.57314629258517	4.00000003517276\\
-2.55711422845691	4.00000003623025\\
-2.54108216432866	4.00000003730514\\
-2.5250501002004	4.00000003839712\\
-2.50901803607214	4.00000003950582\\
-2.49298597194389	4.00000004063087\\
-2.47695390781563	4.00000004177185\\
-2.46092184368737	4.00000004292832\\
-2.44488977955912	4.0000000440998\\
-2.42885771543086	4.00000004528579\\
-2.41282565130261	4.00000004648575\\
-2.39679358717435	4.0000000476991\\
-2.38076152304609	4.00000004892527\\
-2.36472945891784	4.0000000501636\\
-2.34869739478958	4.00000005141346\\
-2.33266533066132	4.00000005267415\\
-2.31663326653307	4.00000005394494\\
-2.30060120240481	4.0000000552251\\
-2.28456913827655	4.00000005651385\\
-2.2685370741483	4.00000005781037\\
-2.25250501002004	4.00000005911385\\
-2.23647294589178	4.00000006042342\\
-2.22044088176353	4.00000006173819\\
-2.20440881763527	4.00000006305725\\
-2.18837675350701	4.00000006437967\\
-2.17234468937876	4.00000006570449\\
-2.1563126252505	4.00000006703072\\
-2.14028056112224	4.00000006835736\\
-2.12424849699399	4.00000006968339\\
-2.10821643286573	4.00000007100775\\
-2.09218436873747	4.0000000723294\\
-2.07615230460922	4.00000007364725\\
-2.06012024048096	4.0000000749602\\
-2.04408817635271	4.00000007626715\\
-2.02805611222445	4.00000007756697\\
-2.01202404809619	4.00000007885854\\
-1.99599198396794	4.00000008014071\\
-1.97995991983968	4.00000008141234\\
-1.96392785571142	4.00000008267226\\
-1.94789579158317	4.00000008391932\\
-1.93186372745491	4.00000008515236\\
-1.91583166332665	4.0000000863702\\
-1.8997995991984	4.0000000875717\\
-1.88376753507014	4.00000008875568\\
-1.86773547094188	4.000000089921\\
-1.85170340681363	4.0000000910665\\
-1.83567134268537	4.00000009219105\\
-1.81963927855711	4.0000000932935\\
-1.80360721442886	4.00000009437275\\
-1.7875751503006	4.00000009542768\\
-1.77154308617234	4.00000009645721\\
-1.75551102204409	4.00000009746027\\
-1.73947895791583	4.0000000984358\\
-1.72344689378758	4.00000009938277\\
-1.70741482965932	4.00000010030017\\
-1.69138276553106	4.00000010118702\\
-1.67535070140281	4.00000010204237\\
-1.65931863727455	4.00000010286528\\
-1.64328657314629	4.00000010365485\\
-1.62725450901804	4.00000010441023\\
-1.61122244488978	4.00000010513056\\
-1.59519038076152	4.00000010581507\\
-1.57915831663327	4.00000010646298\\
-1.56312625250501	4.00000010707356\\
-1.54709418837675	4.00000010764614\\
-1.5310621242485	4.00000010818007\\
-1.51503006012024	4.00000010867473\\
-1.49899799599198	4.00000010912958\\
-1.48296593186373	4.00000010954409\\
-1.46693386773547	4.00000010991779\\
-1.45090180360721	4.00000011025024\\
-1.43486973947896	4.00000011054108\\
-1.4188376753507	4.00000011078997\\
-1.40280561122244	4.00000011099661\\
-1.38677354709419	4.00000011116077\\
-1.37074148296593	4.00000011128227\\
-1.35470941883768	4.00000011136095\\
-1.33867735470942	4.00000011139673\\
-1.32264529058116	4.00000011138958\\
-1.30661322645291	4.00000011133949\\
-1.29058116232465	4.00000011124652\\
-1.27454909819639	4.00000011111078\\
-1.25851703406814	4.00000011093244\\
-1.24248496993988	4.00000011071169\\
-1.22645290581162	4.00000011044878\\
-1.21042084168337	4.00000011014403\\
-1.19438877755511	4.00000010979778\\
-1.17835671342685	4.00000010941043\\
-1.1623246492986	4.00000010898242\\
-1.14629258517034	4.00000010851424\\
-1.13026052104208	4.00000010800642\\
-1.11422845691383	4.00000010745955\\
-1.09819639278557	4.00000010687423\\
-1.08216432865731	4.00000010625112\\
-1.06613226452906	4.00000010559093\\
-1.0501002004008	4.00000010489439\\
-1.03406813627255	4.00000010416229\\
-1.01803607214429	4.00000010339542\\
-1.00200400801603	4.00000010259463\\
-0.985971943887776	4.00000010176081\\
-0.969939879759519	4.00000010089486\\
-0.953907815631263	4.00000009999772\\
-0.937875751503006	4.00000009907035\\
-0.92184368737475	4.00000009811374\\
-0.905811623246493	4.00000009712892\\
-0.889779559118236	4.00000009611693\\
-0.87374749498998	4.00000009507881\\
-0.857715430861723	4.00000009401565\\
-0.841683366733467	4.00000009292854\\
-0.82565130260521	4.0000000918186\\
-0.809619238476954	4.00000009068694\\
-0.793587174348697	4.00000008953471\\
-0.777555110220441	4.00000008836304\\
-0.761523046092184	4.00000008717309\\
-0.745490981963928	4.00000008596601\\
-0.729458917835671	4.00000008474298\\
-0.713426853707415	4.00000008350514\\
-0.697394789579158	4.00000008225366\\
-0.681362725450902	4.00000008098971\\
-0.665330661322646	4.00000007971444\\
-0.649298597194389	4.00000007842901\\
-0.633266533066132	4.00000007713456\\
-0.617234468937876	4.00000007583223\\
-0.601202404809619	4.00000007452316\\
-0.585170340681363	4.00000007320845\\
-0.569138276553106	4.00000007188922\\
-0.55310621242485	4.00000007056655\\
-0.537074148296593	4.00000006924151\\
-0.521042084168337	4.00000006791516\\
-0.50501002004008	4.00000006658855\\
-0.488977955911824	4.00000006526268\\
-0.472945891783567	4.00000006393855\\
-0.456913827655311	4.00000006261714\\
-0.440881763527054	4.00000006129941\\
-0.424849699398798	4.00000005998627\\
-0.408817635270541	4.00000005867864\\
-0.392785571142285	4.00000005737738\\
-0.376753507014028	4.00000005608336\\
-0.360721442885771	4.00000005479739\\
-0.344689378757515	4.00000005352027\\
-0.328657314629258	4.00000005225276\\
-0.312625250501002	4.00000005099561\\
-0.296593186372745	4.00000004974951\\
-0.280561122244489	4.00000004851516\\
-0.264529058116232	4.0000000472932\\
-0.248496993987976	4.00000004608424\\
-0.232464929859719	4.00000004488888\\
-0.216432865731463	4.00000004370767\\
-0.200400801603207	4.00000004254114\\
-0.18436873747495	4.00000004138978\\
-0.168336673346694	4.00000004025406\\
-0.152304609218437	4.00000003913442\\
-0.13627254509018	4.00000003803125\\
-0.120240480961924	4.00000003694493\\
-0.104208416833667	4.0000000358758\\
-0.0881763527054109	4.00000003482419\\
-0.0721442885771544	4.00000003379037\\
-0.0561122244488979	4.00000003277461\\
-0.0400801603206413	4.00000003177712\\
-0.0240480961923848	4.00000003079812\\
-0.00801603206412826	4.00000002983777\\
0.00801603206412782	4.00000002889622\\
0.0240480961923843	4.0000000279736\\
0.0400801603206409	4.00000002707\\
0.0561122244488974	4.00000002618549\\
0.0721442885771539	4.00000002532012\\
0.0881763527054105	4.00000002447391\\
0.104208416833667	4.00000002364686\\
0.120240480961924	4.00000002283895\\
0.13627254509018	4.00000002205014\\
0.152304609218437	4.00000002128037\\
0.168336673346693	4.00000002052955\\
0.18436873747495	4.00000001979759\\
0.200400801603206	4.00000001908437\\
0.216432865731463	4.00000001838976\\
0.232464929859719	4.00000001771359\\
0.248496993987976	4.00000001705571\\
0.264529058116232	4.00000001641593\\
0.280561122244489	4.00000001579406\\
0.296593186372745	4.00000001518989\\
0.312625250501002	4.0000000146032\\
0.328657314629258	4.00000001403376\\
0.344689378757515	4.00000001348132\\
0.360721442885771	4.00000001294564\\
0.376753507014028	4.00000001242646\\
0.392785571142285	4.0000000119235\\
0.408817635270541	4.00000001143648\\
0.424849699398798	4.00000001096513\\
0.440881763527054	4.00000001050915\\
0.456913827655311	4.00000001006825\\
0.472945891783567	4.00000000964213\\
0.488977955911824	4.00000000923049\\
0.50501002004008	4.00000000883301\\
0.521042084168337	4.00000000844939\\
0.537074148296593	4.00000000807932\\
0.55310621242485	4.00000000772248\\
0.569138276553106	4.00000000737855\\
0.585170340681363	4.00000000704722\\
0.601202404809619	4.00000000672818\\
0.617234468937876	4.0000000064211\\
0.633266533066132	4.00000000612568\\
0.649298597194389	4.00000000584159\\
0.665330661322646	4.00000000556854\\
0.681362725450902	4.0000000053062\\
0.697394789579159	4.00000000505427\\
0.713426853707415	4.00000000481245\\
0.729458917835672	4.00000000458043\\
0.745490981963928	4.00000000435791\\
0.761523046092185	4.00000000414461\\
0.777555110220441	4.00000000394023\\
0.793587174348698	4.00000000374448\\
0.809619238476954	4.00000000355709\\
0.825651302605211	4.00000000337777\\
0.841683366733467	4.00000000320625\\
0.857715430861724	4.00000000304227\\
0.87374749498998	4.00000000288557\\
0.889779559118236	4.00000000273588\\
0.905811623246493	4.00000000259296\\
0.921843687374749	4.00000000245655\\
0.937875751503006	4.00000000232643\\
0.953907815631262	4.00000000220234\\
0.969939879759519	4.00000000208408\\
0.985971943887775	4.0000000019714\\
1.00200400801603	4.00000000186409\\
1.01803607214429	4.00000000176195\\
1.03406813627254	4.00000000166477\\
1.0501002004008	4.00000000157233\\
1.06613226452906	4.00000000148446\\
1.08216432865731	4.00000000140096\\
1.09819639278557	4.00000000132164\\
1.11422845691383	4.00000000124633\\
1.13026052104208	4.00000000117487\\
1.14629258517034	4.00000000110707\\
1.1623246492986	4.00000000104278\\
1.17835671342685	4.00000000098185\\
1.19438877755511	4.00000000092412\\
1.21042084168337	4.00000000086946\\
1.22645290581162	4.00000000081771\\
1.24248496993988	4.00000000076874\\
1.25851703406814	4.00000000072243\\
1.27454909819639	4.00000000067864\\
1.29058116232465	4.00000000063727\\
1.30661322645291	4.00000000059818\\
1.32264529058116	4.00000000056128\\
1.33867735470942	4.00000000052645\\
1.35470941883768	4.00000000049359\\
1.37074148296593	4.0000000004626\\
1.38677354709419	4.00000000043339\\
1.40280561122244	4.00000000040587\\
1.4188376753507	4.00000000037995\\
1.43486973947896	4.00000000035555\\
1.45090180360721	4.00000000033259\\
1.46693386773547	4.00000000031099\\
1.48296593186373	4.00000000029068\\
1.49899799599198	4.00000000027159\\
1.51503006012024	4.00000000025366\\
1.5310621242485	4.00000000023682\\
1.54709418837675	4.00000000022102\\
1.56312625250501	4.00000000020618\\
1.57915831663327	4.00000000019228\\
1.59519038076152	4.00000000017923\\
1.61122244488978	4.00000000016701\\
1.62725450901804	4.00000000015557\\
1.64328657314629	4.00000000014485\\
1.65931863727455	4.00000000013482\\
1.67535070140281	4.00000000012543\\
1.69138276553106	4.00000000011665\\
1.70741482965932	4.00000000010845\\
1.72344689378758	4.00000000010078\\
1.73947895791583	4.00000000009362\\
1.75551102204409	4.00000000008693\\
1.77154308617235	4.0000000000807\\
1.7875751503006	4.00000000007488\\
1.80360721442886	4.00000000006945\\
1.81963927855711	4.00000000006439\\
1.83567134268537	4.00000000005968\\
1.85170340681363	4.00000000005529\\
1.86773547094188	4.0000000000512\\
1.88376753507014	4.0000000000474\\
1.8997995991984	4.00000000004386\\
1.91583166332665	4.00000000004057\\
1.93186372745491	4.00000000003751\\
1.94789579158317	4.00000000003468\\
1.96392785571142	4.00000000003204\\
1.97995991983968	4.00000000002959\\
1.99599198396794	4.00000000002732\\
2.01202404809619	4.00000000002521\\
2.02805611222445	4.00000000002326\\
2.04408817635271	4.00000000002145\\
2.06012024048096	4.00000000001977\\
2.07615230460922	4.00000000001822\\
2.09218436873747	4.00000000001678\\
2.10821643286573	4.00000000001545\\
2.12424849699399	4.00000000001422\\
2.14028056112224	4.00000000001308\\
2.1563126252505	4.00000000001203\\
2.17234468937876	4.00000000001106\\
2.18837675350701	4.00000000001017\\
2.20440881763527	4.00000000000934\\
2.22044088176353	4.00000000000858\\
2.23647294589178	4.00000000000787\\
2.25250501002004	4.00000000000722\\
2.2685370741483	4.00000000000662\\
2.28456913827655	4.00000000000607\\
2.30060120240481	4.00000000000557\\
2.31663326653307	4.0000000000051\\
2.33266533066132	4.00000000000467\\
2.34869739478958	4.00000000000427\\
2.36472945891784	4.00000000000391\\
2.38076152304609	4.00000000000358\\
2.39679358717435	4.00000000000327\\
2.41282565130261	4.00000000000299\\
2.42885771543086	4.00000000000273\\
2.44488977955912	4.0000000000025\\
2.46092184368737	4.00000000000228\\
2.47695390781563	4.00000000000208\\
2.49298597194389	4.0000000000019\\
2.50901803607214	4.00000000000173\\
2.5250501002004	4.00000000000158\\
2.54108216432866	4.00000000000144\\
2.55711422845691	4.00000000000131\\
2.57314629258517	4.00000000000119\\
2.58917835671343	4.00000000000108\\
2.60521042084168	4.00000000000099\\
2.62124248496994	4.0000000000009\\
2.6372745490982	4.00000000000082\\
2.65330661322645	4.00000000000074\\
2.66933867735471	4.00000000000067\\
2.68537074148297	4.00000000000061\\
2.70140280561122	4.00000000000056\\
2.71743486973948	4.0000000000005\\
2.73346693386774	4.00000000000046\\
2.74949899799599	4.00000000000041\\
2.76553106212425	4.00000000000038\\
2.7815631262525	4.00000000000034\\
2.79759519038076	4.00000000000031\\
2.81362725450902	4.00000000000028\\
2.82965931863727	4.00000000000025\\
2.84569138276553	4.00000000000023\\
2.86172344689379	4.00000000000021\\
2.87775551102204	4.00000000000019\\
2.8937875751503	4.00000000000017\\
2.90981963927856	4.00000000000015\\
2.92585170340681	4.00000000000014\\
2.94188376753507	4.00000000000012\\
2.95791583166333	4.00000000000011\\
2.97394789579158	4.0000000000001\\
2.98997995991984	4.00000000000009\\
3.0060120240481	4.00000000000008\\
3.02204408817635	4.00000000000007\\
3.03807615230461	4.00000000000007\\
3.05410821643287	4.00000000000006\\
3.07014028056112	4.00000000000005\\
3.08617234468938	4.00000000000005\\
3.10220440881764	4.00000000000004\\
3.11823647294589	4.00000000000004\\
3.13426853707415	4.00000000000003\\
3.1503006012024	4.00000000000003\\
3.16633266533066	4.00000000000003\\
3.18236472945892	4.00000000000003\\
3.19839679358717	4.00000000000002\\
3.21442885771543	4.00000000000002\\
3.23046092184369	4.00000000000002\\
3.24649298597194	4.00000000000002\\
3.2625250501002	4.00000000000001\\
3.27855711422846	4.00000000000001\\
3.29458917835671	4.00000000000001\\
3.31062124248497	4.00000000000001\\
3.32665330661323	4.00000000000001\\
3.34268537074148	4.00000000000001\\
3.35871743486974	4.00000000000001\\
3.374749498998	4.00000000000001\\
3.39078156312625	4.00000000000001\\
3.40681362725451	4.00000000000001\\
3.42284569138277	4\\
3.43887775551102	4\\
3.45490981963928	4\\
3.47094188376754	4\\
3.48697394789579	4\\
3.50300601202405	4\\
3.51903807615231	4\\
3.53507014028056	4\\
3.55110220440882	4\\
3.56713426853707	4\\
3.58316633266533	4\\
3.59919839679359	4\\
3.61523046092184	4\\
3.6312625250501	4\\
3.64729458917836	4\\
3.66332665330661	4\\
3.67935871743487	4\\
3.69539078156313	4\\
3.71142284569138	4\\
3.72745490981964	4\\
3.7434869739479	4\\
3.75951903807615	4\\
3.77555110220441	4\\
3.79158316633267	4\\
3.80761523046092	4\\
3.82364729458918	4\\
3.83967935871743	4\\
3.85571142284569	4\\
3.87174348697395	4\\
3.8877755511022	4\\
3.90380761523046	4\\
3.91983967935872	4\\
3.93587174348697	4\\
3.95190380761523	4\\
3.96793587174349	4\\
3.98396793587174	4\\
4	4\\
4	3.98396793587174\\
4	3.96793587174349\\
4	3.95190380761523\\
4	3.93587174348697\\
4	3.91983967935872\\
4	3.90380761523046\\
4	3.8877755511022\\
4	3.87174348697395\\
4	3.85571142284569\\
4	3.83967935871743\\
4	3.82364729458918\\
4	3.80761523046092\\
4	3.79158316633267\\
4	3.77555110220441\\
4	3.75951903807615\\
4	3.7434869739479\\
4	3.72745490981964\\
4	3.71142284569138\\
4	3.69539078156313\\
4	3.67935871743487\\
4	3.66332665330661\\
4	3.64729458917836\\
4	3.6312625250501\\
4	3.61523046092184\\
4	3.59919839679359\\
4	3.58316633266533\\
4	3.56713426853707\\
4	3.55110220440882\\
4	3.53507014028056\\
4	3.51903807615231\\
4	3.50300601202405\\
4	3.48697394789579\\
4	3.47094188376754\\
4	3.45490981963928\\
4	3.43887775551102\\
4	3.42284569138277\\
4	3.40681362725451\\
4	3.39078156312625\\
4	3.374749498998\\
4	3.35871743486974\\
4	3.34268537074148\\
4	3.32665330661323\\
4	3.31062124248497\\
4	3.29458917835671\\
4	3.27855711422846\\
4	3.2625250501002\\
4	3.24649298597194\\
4	3.23046092184369\\
4	3.21442885771543\\
4	3.19839679358717\\
4	3.18236472945892\\
4	3.16633266533066\\
4	3.1503006012024\\
4	3.13426853707415\\
4	3.11823647294589\\
4	3.10220440881764\\
4	3.08617234468938\\
4	3.07014028056112\\
4	3.05410821643287\\
4	3.03807615230461\\
4	3.02204408817635\\
4	3.0060120240481\\
4	2.98997995991984\\
4	2.97394789579158\\
4	2.95791583166333\\
4	2.94188376753507\\
4	2.92585170340681\\
4	2.90981963927856\\
4	2.8937875751503\\
4	2.87775551102204\\
4	2.86172344689379\\
4	2.84569138276553\\
4	2.82965931863727\\
4	2.81362725450902\\
4.00000000000001	2.79759519038076\\
4.00000000000001	2.7815631262525\\
4.00000000000001	2.76553106212425\\
4.00000000000001	2.74949899799599\\
4.00000000000001	2.73346693386774\\
4.00000000000001	2.71743486973948\\
4.00000000000001	2.70140280561122\\
4.00000000000001	2.68537074148297\\
4.00000000000001	2.66933867735471\\
4.00000000000001	2.65330661322645\\
4.00000000000001	2.6372745490982\\
4.00000000000001	2.62124248496994\\
4.00000000000001	2.60521042084168\\
4.00000000000001	2.58917835671343\\
4.00000000000001	2.57314629258517\\
4.00000000000001	2.55711422845691\\
4.00000000000001	2.54108216432866\\
4.00000000000001	2.5250501002004\\
4.00000000000001	2.50901803607214\\
4.00000000000001	2.49298597194389\\
4.00000000000001	2.47695390781563\\
4.00000000000002	2.46092184368737\\
4.00000000000002	2.44488977955912\\
4.00000000000002	2.42885771543086\\
4.00000000000002	2.41282565130261\\
4.00000000000002	2.39679358717435\\
4.00000000000002	2.38076152304609\\
4.00000000000002	2.36472945891784\\
4.00000000000002	2.34869739478958\\
4.00000000000002	2.33266533066132\\
4.00000000000002	2.31663326653307\\
4.00000000000003	2.30060120240481\\
4.00000000000003	2.28456913827655\\
4.00000000000003	2.2685370741483\\
4.00000000000003	2.25250501002004\\
4.00000000000003	2.23647294589178\\
4.00000000000003	2.22044088176353\\
4.00000000000004	2.20440881763527\\
4.00000000000004	2.18837675350701\\
4.00000000000004	2.17234468937876\\
4.00000000000004	2.1563126252505\\
4.00000000000004	2.14028056112224\\
4.00000000000005	2.12424849699399\\
4.00000000000005	2.10821643286573\\
4.00000000000005	2.09218436873747\\
4.00000000000005	2.07615230460922\\
4.00000000000006	2.06012024048096\\
4.00000000000006	2.04408817635271\\
4.00000000000006	2.02805611222445\\
4.00000000000006	2.01202404809619\\
4.00000000000007	1.99599198396794\\
4.00000000000007	1.97995991983968\\
4.00000000000007	1.96392785571142\\
4.00000000000008	1.94789579158317\\
4.00000000000008	1.93186372745491\\
4.00000000000009	1.91583166332665\\
4.00000000000009	1.8997995991984\\
4.00000000000009	1.88376753507014\\
4.0000000000001	1.86773547094188\\
4.0000000000001	1.85170340681363\\
4.00000000000011	1.83567134268537\\
4.00000000000011	1.81963927855711\\
4.00000000000012	1.80360721442886\\
4.00000000000012	1.7875751503006\\
4.00000000000013	1.77154308617235\\
4.00000000000014	1.75551102204409\\
4.00000000000014	1.73947895791583\\
4.00000000000015	1.72344689378758\\
4.00000000000016	1.70741482965932\\
4.00000000000016	1.69138276553106\\
4.00000000000017	1.67535070140281\\
4.00000000000018	1.65931863727455\\
4.00000000000019	1.64328657314629\\
4.0000000000002	1.62725450901804\\
4.00000000000021	1.61122244488978\\
4.00000000000021	1.59519038076152\\
4.00000000000022	1.57915831663327\\
4.00000000000023	1.56312625250501\\
4.00000000000025	1.54709418837675\\
4.00000000000026	1.5310621242485\\
4.00000000000027	1.51503006012024\\
4.00000000000028	1.49899799599198\\
4.00000000000029	1.48296593186373\\
4.00000000000031	1.46693386773547\\
4.00000000000032	1.45090180360721\\
4.00000000000033	1.43486973947896\\
4.00000000000035	1.4188376753507\\
4.00000000000036	1.40280561122244\\
4.00000000000038	1.38677354709419\\
4.0000000000004	1.37074148296593\\
4.00000000000041	1.35470941883768\\
4.00000000000043	1.33867735470942\\
4.00000000000045	1.32264529058116\\
4.00000000000047	1.30661322645291\\
4.00000000000049	1.29058116232465\\
4.00000000000051	1.27454909819639\\
4.00000000000053	1.25851703406814\\
4.00000000000056	1.24248496993988\\
4.00000000000058	1.22645290581162\\
4.00000000000061	1.21042084168337\\
4.00000000000063	1.19438877755511\\
4.00000000000066	1.17835671342685\\
4.00000000000069	1.1623246492986\\
4.00000000000072	1.14629258517034\\
4.00000000000075	1.13026052104208\\
4.00000000000078	1.11422845691383\\
4.00000000000081	1.09819639278557\\
4.00000000000084	1.08216432865731\\
4.00000000000088	1.06613226452906\\
4.00000000000091	1.0501002004008\\
4.00000000000095	1.03406813627254\\
4.00000000000099	1.01803607214429\\
4.00000000000103	1.00200400801603\\
4.00000000000108	0.985971943887775\\
4.00000000000112	0.969939879759519\\
4.00000000000116	0.953907815631262\\
4.00000000000121	0.937875751503006\\
4.00000000000126	0.921843687374749\\
4.00000000000131	0.905811623246493\\
4.00000000000136	0.889779559118236\\
4.00000000000142	0.87374749498998\\
4.00000000000147	0.857715430861724\\
4.00000000000153	0.841683366733467\\
4.00000000000159	0.825651302605211\\
4.00000000000166	0.809619238476954\\
4.00000000000172	0.793587174348698\\
4.00000000000179	0.777555110220441\\
4.00000000000186	0.761523046092185\\
4.00000000000193	0.745490981963928\\
4.000000000002	0.729458917835672\\
4.00000000000208	0.713426853707415\\
4.00000000000216	0.697394789579159\\
4.00000000000225	0.681362725450902\\
4.00000000000233	0.665330661322646\\
4.00000000000242	0.649298597194389\\
4.00000000000251	0.633266533066132\\
4.00000000000261	0.617234468937876\\
4.0000000000027	0.601202404809619\\
4.00000000000281	0.585170340681363\\
4.00000000000291	0.569138276553106\\
4.00000000000302	0.55310621242485\\
4.00000000000313	0.537074148296593\\
4.00000000000325	0.521042084168337\\
4.00000000000337	0.50501002004008\\
4.00000000000349	0.488977955911824\\
4.00000000000362	0.472945891783567\\
4.00000000000375	0.456913827655311\\
4.00000000000388	0.440881763527054\\
4.00000000000403	0.424849699398798\\
4.00000000000417	0.408817635270541\\
4.00000000000432	0.392785571142285\\
4.00000000000448	0.376753507014028\\
4.00000000000464	0.360721442885771\\
4.0000000000048	0.344689378757515\\
4.00000000000497	0.328657314629258\\
4.00000000000514	0.312625250501002\\
4.00000000000532	0.296593186372745\\
4.00000000000551	0.280561122244489\\
4.0000000000057	0.264529058116232\\
4.0000000000059	0.248496993987976\\
4.00000000000611	0.232464929859719\\
4.00000000000632	0.216432865731463\\
4.00000000000653	0.200400801603206\\
4.00000000000676	0.18436873747495\\
4.00000000000699	0.168336673346693\\
4.00000000000722	0.152304609218437\\
4.00000000000747	0.13627254509018\\
4.00000000000772	0.120240480961924\\
4.00000000000798	0.104208416833667\\
4.00000000000824	0.0881763527054105\\
4.00000000000852	0.0721442885771539\\
4.0000000000088	0.0561122244488974\\
4.00000000000909	0.0400801603206409\\
4.00000000000939	0.0240480961923843\\
4.00000000000969	0.00801603206412782\\
4.00000000001001	-0.00801603206412826\\
4.00000000001033	-0.0240480961923848\\
4.00000000001067	-0.0400801603206413\\
4.00000000001101	-0.0561122244488979\\
4.00000000001137	-0.0721442885771544\\
4.00000000001173	-0.0881763527054109\\
4.0000000000121	-0.104208416833667\\
4.00000000001248	-0.120240480961924\\
4.00000000001288	-0.13627254509018\\
4.00000000001328	-0.152304609218437\\
4.0000000000137	-0.168336673346694\\
4.00000000001412	-0.18436873747495\\
4.00000000001456	-0.200400801603207\\
4.00000000001501	-0.216432865731463\\
4.00000000001547	-0.232464929859719\\
4.00000000001594	-0.248496993987976\\
4.00000000001643	-0.264529058116232\\
4.00000000001693	-0.280561122244489\\
4.00000000001744	-0.296593186372745\\
4.00000000001796	-0.312625250501002\\
4.0000000000185	-0.328657314629258\\
4.00000000001905	-0.344689378757515\\
4.00000000001962	-0.360721442885771\\
4.0000000000202	-0.376753507014028\\
4.00000000002079	-0.392785571142285\\
4.0000000000214	-0.408817635270541\\
4.00000000002202	-0.424849699398798\\
4.00000000002266	-0.440881763527054\\
4.00000000002332	-0.456913827655311\\
4.00000000002399	-0.472945891783567\\
4.00000000002467	-0.488977955911824\\
4.00000000002538	-0.50501002004008\\
4.0000000000261	-0.521042084168337\\
4.00000000002683	-0.537074148296593\\
4.00000000002759	-0.55310621242485\\
4.00000000002836	-0.569138276553106\\
4.00000000002914	-0.585170340681363\\
4.00000000002995	-0.601202404809619\\
4.00000000003078	-0.617234468937876\\
4.00000000003162	-0.633266533066132\\
4.00000000003248	-0.649298597194389\\
4.00000000003337	-0.665330661322646\\
4.00000000003427	-0.681362725450902\\
4.00000000003519	-0.697394789579158\\
4.00000000003613	-0.713426853707415\\
4.00000000003709	-0.729458917835671\\
4.00000000003808	-0.745490981963928\\
4.00000000003908	-0.761523046092184\\
4.00000000004011	-0.777555110220441\\
4.00000000004115	-0.793587174348697\\
4.00000000004222	-0.809619238476954\\
4.00000000004331	-0.82565130260521\\
4.00000000004442	-0.841683366733467\\
4.00000000004556	-0.857715430861723\\
4.00000000004672	-0.87374749498998\\
4.0000000000479	-0.889779559118236\\
4.00000000004911	-0.905811623246493\\
4.00000000005034	-0.92184368737475\\
4.00000000005159	-0.937875751503006\\
4.00000000005287	-0.953907815631263\\
4.00000000005418	-0.969939879759519\\
4.0000000000555	-0.985971943887776\\
4.00000000005686	-1.00200400801603\\
4.00000000005824	-1.01803607214429\\
4.00000000005964	-1.03406813627255\\
4.00000000006107	-1.0501002004008\\
4.00000000006253	-1.06613226452906\\
4.00000000006401	-1.08216432865731\\
4.00000000006552	-1.09819639278557\\
4.00000000006706	-1.11422845691383\\
4.00000000006863	-1.13026052104208\\
4.00000000007022	-1.14629258517034\\
4.00000000007184	-1.1623246492986\\
4.00000000007349	-1.17835671342685\\
4.00000000007516	-1.19438877755511\\
4.00000000007687	-1.21042084168337\\
4.0000000000786	-1.22645290581162\\
4.00000000008036	-1.24248496993988\\
4.00000000008216	-1.25851703406814\\
4.00000000008398	-1.27454909819639\\
4.00000000008583	-1.29058116232465\\
4.0000000000877	-1.30661322645291\\
4.00000000008961	-1.32264529058116\\
4.00000000009155	-1.33867735470942\\
4.00000000009352	-1.35470941883768\\
4.00000000009552	-1.37074148296593\\
4.00000000009755	-1.38677354709419\\
4.0000000000996	-1.40280561122244\\
4.00000000010169	-1.4188376753507\\
4.00000000010381	-1.43486973947896\\
4.00000000010596	-1.45090180360721\\
4.00000000010814	-1.46693386773547\\
4.00000000011035	-1.48296593186373\\
4.0000000001126	-1.49899799599198\\
4.00000000011487	-1.51503006012024\\
4.00000000011717	-1.5310621242485\\
4.00000000011951	-1.54709418837675\\
4.00000000012187	-1.56312625250501\\
4.00000000012427	-1.57915831663327\\
4.00000000012669	-1.59519038076152\\
4.00000000012915	-1.61122244488978\\
4.00000000013164	-1.62725450901804\\
4.00000000013416	-1.64328657314629\\
4.00000000013671	-1.65931863727455\\
4.00000000013929	-1.67535070140281\\
4.0000000001419	-1.69138276553106\\
4.00000000014454	-1.70741482965932\\
4.00000000014721	-1.72344689378758\\
4.00000000014991	-1.73947895791583\\
4.00000000015265	-1.75551102204409\\
4.00000000015541	-1.77154308617234\\
4.0000000001582	-1.7875751503006\\
4.00000000016102	-1.80360721442886\\
4.00000000016387	-1.81963927855711\\
4.00000000016675	-1.83567134268537\\
4.00000000016965	-1.85170340681363\\
4.00000000017259	-1.86773547094188\\
4.00000000017555	-1.88376753507014\\
4.00000000017854	-1.8997995991984\\
4.00000000018156	-1.91583166332665\\
4.00000000018461	-1.93186372745491\\
4.00000000018768	-1.94789579158317\\
4.00000000019079	-1.96392785571142\\
4.00000000019391	-1.97995991983968\\
4.00000000019707	-1.99599198396794\\
4.00000000020024	-2.01202404809619\\
4.00000000020345	-2.02805611222445\\
4.00000000020668	-2.04408817635271\\
4.00000000020993	-2.06012024048096\\
4.0000000002132	-2.07615230460922\\
4.0000000002165	-2.09218436873747\\
4.00000000021983	-2.10821643286573\\
4.00000000022317	-2.12424849699399\\
4.00000000022654	-2.14028056112224\\
4.00000000022992	-2.1563126252505\\
4.00000000023333	-2.17234468937876\\
4.00000000023676	-2.18837675350701\\
4.00000000024021	-2.20440881763527\\
4.00000000024368	-2.22044088176353\\
4.00000000024716	-2.23647294589178\\
4.00000000025066	-2.25250501002004\\
4.00000000025418	-2.2685370741483\\
4.00000000025772	-2.28456913827655\\
4.00000000026127	-2.30060120240481\\
4.00000000026484	-2.31663326653307\\
4.00000000026842	-2.33266533066132\\
4.00000000027201	-2.34869739478958\\
4.00000000027562	-2.36472945891784\\
4.00000000027924	-2.38076152304609\\
4.00000000028287	-2.39679358717435\\
4.00000000028651	-2.41282565130261\\
4.00000000029016	-2.42885771543086\\
4.00000000029382	-2.44488977955912\\
4.00000000029748	-2.46092184368737\\
4.00000000030116	-2.47695390781563\\
4.00000000030484	-2.49298597194389\\
4.00000000030852	-2.50901803607214\\
4.00000000031221	-2.5250501002004\\
4.0000000003159	-2.54108216432866\\
4.0000000003196	-2.55711422845691\\
4.0000000003233	-2.57314629258517\\
4.00000000032699	-2.58917835671343\\
4.00000000033069	-2.60521042084168\\
4.00000000033439	-2.62124248496994\\
4.00000000033808	-2.6372745490982\\
4.00000000034177	-2.65330661322645\\
4.00000000034546	-2.66933867735471\\
4.00000000034914	-2.68537074148297\\
4.00000000035282	-2.70140280561122\\
4.00000000035649	-2.71743486973948\\
4.00000000036015	-2.73346693386774\\
4.0000000003638	-2.74949899799599\\
4.00000000036744	-2.76553106212425\\
4.00000000037107	-2.7815631262525\\
4.00000000037469	-2.79759519038076\\
4.0000000003783	-2.81362725450902\\
4.00000000038189	-2.82965931863727\\
4.00000000038546	-2.84569138276553\\
4.00000000038902	-2.86172344689379\\
4.00000000039256	-2.87775551102204\\
4.00000000039608	-2.8937875751503\\
4.00000000039958	-2.90981963927856\\
4.00000000040307	-2.92585170340681\\
4.00000000040652	-2.94188376753507\\
4.00000000040996	-2.95791583166333\\
4.00000000041337	-2.97394789579158\\
4.00000000041676	-2.98997995991984\\
4.00000000042012	-3.0060120240481\\
4.00000000042346	-3.02204408817635\\
4.00000000042676	-3.03807615230461\\
4.00000000043004	-3.05410821643287\\
4.00000000043328	-3.07014028056112\\
4.0000000004365	-3.08617234468938\\
4.00000000043968	-3.10220440881764\\
4.00000000044282	-3.11823647294589\\
4.00000000044594	-3.13426853707415\\
4.00000000044901	-3.1503006012024\\
4.00000000045205	-3.16633266533066\\
4.00000000045505	-3.18236472945892\\
4.00000000045802	-3.19839679358717\\
4.00000000046094	-3.21442885771543\\
4.00000000046382	-3.23046092184369\\
4.00000000046666	-3.24649298597194\\
4.00000000046946	-3.2625250501002\\
4.00000000047221	-3.27855711422846\\
4.00000000047492	-3.29458917835671\\
4.00000000047758	-3.31062124248497\\
4.0000000004802	-3.32665330661323\\
4.00000000048277	-3.34268537074148\\
4.00000000048529	-3.35871743486974\\
4.00000000048776	-3.374749498998\\
4.00000000049018	-3.39078156312625\\
4.00000000049254	-3.40681362725451\\
4.00000000049486	-3.42284569138277\\
4.00000000049712	-3.43887775551102\\
4.00000000049933	-3.45490981963928\\
4.00000000050148	-3.47094188376753\\
4.00000000050358	-3.48697394789579\\
4.00000000050563	-3.50300601202405\\
4.00000000050761	-3.5190380761523\\
4.00000000050954	-3.53507014028056\\
4.00000000051141	-3.55110220440882\\
4.00000000051322	-3.56713426853707\\
4.00000000051497	-3.58316633266533\\
4.00000000051666	-3.59919839679359\\
4.00000000051829	-3.61523046092184\\
4.00000000051986	-3.6312625250501\\
4.00000000052136	-3.64729458917836\\
4.00000000052281	-3.66332665330661\\
4.00000000052419	-3.67935871743487\\
4.0000000005255	-3.69539078156313\\
4.00000000052675	-3.71142284569138\\
4.00000000052794	-3.72745490981964\\
4.00000000052906	-3.7434869739479\\
4.00000000053011	-3.75951903807615\\
4.0000000005311	-3.77555110220441\\
4.00000000053202	-3.79158316633267\\
4.00000000053288	-3.80761523046092\\
4.00000000053367	-3.82364729458918\\
4.00000000053439	-3.83967935871743\\
4.00000000053504	-3.85571142284569\\
4.00000000053563	-3.87174348697395\\
4.00000000053614	-3.8877755511022\\
4.00000000053659	-3.90380761523046\\
4.00000000053697	-3.91983967935872\\
4.00000000053728	-3.93587174348697\\
4.00000000053752	-3.95190380761523\\
4.00000000053769	-3.96793587174349\\
4.0000000005378	-3.98396793587174\\
4.00000000053783	-4\\
4	-4.00000000053783\\
3.98396793587174	-4.00000000057334\\
3.96793587174349	-4.00000000061096\\
3.95190380761523	-4.0000000006508\\
3.93587174348697	-4.00000000069296\\
3.91983967935872	-4.00000000073758\\
3.90380761523046	-4.00000000078476\\
3.8877755511022	-4.00000000083464\\
3.87174348697395	-4.00000000088735\\
3.85571142284569	-4.00000000094302\\
3.83967935871743	-4.0000000010018\\
3.82364729458918	-4.00000000106383\\
3.80761523046092	-4.00000000112927\\
3.79158316633267	-4.00000000119827\\
3.77555110220441	-4.000000001271\\
3.75951903807615	-4.00000000134762\\
3.7434869739479	-4.00000000142832\\
3.72745490981964	-4.00000000151325\\
3.71142284569138	-4.00000000160263\\
3.69539078156313	-4.00000000169662\\
3.67935871743487	-4.00000000179544\\
3.66332665330661	-4.00000000189928\\
3.64729458917836	-4.00000000200835\\
3.6312625250501	-4.00000000212287\\
3.61523046092184	-4.00000000224305\\
3.59919839679359	-4.00000000236912\\
3.58316633266533	-4.00000000250131\\
3.56713426853707	-4.00000000263986\\
3.55110220440882	-4.00000000278501\\
3.53507014028056	-4.00000000293701\\
3.51903807615231	-4.00000000309611\\
3.50300601202405	-4.00000000326257\\
3.48697394789579	-4.00000000343666\\
3.47094188376754	-4.00000000361864\\
3.45490981963928	-4.00000000380879\\
3.43887775551102	-4.00000000400738\\
3.42284569138277	-4.00000000421471\\
3.40681362725451	-4.00000000443105\\
3.39078156312625	-4.0000000046567\\
3.374749498998	-4.00000000489195\\
3.35871743486974	-4.00000000513711\\
3.34268537074148	-4.00000000539247\\
3.32665330661323	-4.00000000565835\\
3.31062124248497	-4.00000000593505\\
3.29458917835671	-4.00000000622288\\
3.27855711422846	-4.00000000652215\\
3.2625250501002	-4.00000000683318\\
3.24649298597194	-4.00000000715629\\
3.23046092184369	-4.00000000749178\\
3.21442885771543	-4.00000000783997\\
3.19839679358717	-4.00000000820119\\
3.18236472945892	-4.00000000857575\\
3.16633266533066	-4.00000000896395\\
3.1503006012024	-4.00000000936611\\
3.13426853707415	-4.00000000978255\\
3.11823647294589	-4.00000001021356\\
3.10220440881764	-4.00000001065945\\
3.08617234468938	-4.00000001112052\\
3.07014028056112	-4.00000001159706\\
3.05410821643287	-4.00000001208936\\
3.03807615230461	-4.0000000125977\\
3.02204408817635	-4.00000001312236\\
3.0060120240481	-4.0000000136636\\
2.98997995991984	-4.00000001422167\\
2.97394789579158	-4.00000001479684\\
2.95791583166333	-4.00000001538933\\
2.94188376753507	-4.00000001599937\\
2.92585170340681	-4.00000001662719\\
2.90981963927856	-4.00000001727298\\
2.8937875751503	-4.00000001793694\\
2.87775551102204	-4.00000001861924\\
2.86172344689379	-4.00000001932004\\
2.84569138276553	-4.00000002003949\\
2.82965931863727	-4.00000002077772\\
2.81362725450902	-4.00000002153485\\
2.79759519038076	-4.00000002231096\\
2.7815631262525	-4.00000002310613\\
2.76553106212425	-4.00000002392041\\
2.74949899799599	-4.00000002475385\\
2.73346693386774	-4.00000002560645\\
2.71743486973948	-4.0000000264782\\
2.70140280561122	-4.00000002736909\\
2.68537074148297	-4.00000002827904\\
2.66933867735471	-4.00000002920798\\
2.65330661322645	-4.0000000301558\\
2.6372745490982	-4.00000003112239\\
2.62124248496994	-4.00000003210757\\
2.60521042084168	-4.00000003311117\\
2.58917835671343	-4.00000003413298\\
2.57314629258517	-4.00000003517276\\
2.55711422845691	-4.00000003623025\\
2.54108216432866	-4.00000003730514\\
2.5250501002004	-4.00000003839712\\
2.50901803607214	-4.00000003950582\\
2.49298597194389	-4.00000004063087\\
2.47695390781563	-4.00000004177185\\
2.46092184368737	-4.00000004292832\\
2.44488977955912	-4.0000000440998\\
2.42885771543086	-4.00000004528579\\
2.41282565130261	-4.00000004648575\\
2.39679358717435	-4.0000000476991\\
2.38076152304609	-4.00000004892527\\
2.36472945891784	-4.0000000501636\\
2.34869739478958	-4.00000005141346\\
2.33266533066132	-4.00000005267415\\
2.31663326653307	-4.00000005394494\\
2.30060120240481	-4.0000000552251\\
2.28456913827655	-4.00000005651385\\
2.2685370741483	-4.00000005781037\\
2.25250501002004	-4.00000005911385\\
2.23647294589178	-4.00000006042342\\
2.22044088176353	-4.00000006173819\\
2.20440881763527	-4.00000006305725\\
2.18837675350701	-4.00000006437967\\
2.17234468937876	-4.00000006570449\\
2.1563126252505	-4.00000006703072\\
2.14028056112224	-4.00000006835736\\
2.12424849699399	-4.00000006968339\\
2.10821643286573	-4.00000007100775\\
2.09218436873747	-4.0000000723294\\
2.07615230460922	-4.00000007364725\\
2.06012024048096	-4.0000000749602\\
2.04408817635271	-4.00000007626715\\
2.02805611222445	-4.00000007756697\\
2.01202404809619	-4.00000007885854\\
1.99599198396794	-4.00000008014071\\
1.97995991983968	-4.00000008141234\\
1.96392785571142	-4.00000008267226\\
1.94789579158317	-4.00000008391932\\
1.93186372745491	-4.00000008515236\\
1.91583166332665	-4.0000000863702\\
1.8997995991984	-4.0000000875717\\
1.88376753507014	-4.00000008875568\\
1.86773547094188	-4.000000089921\\
1.85170340681363	-4.0000000910665\\
1.83567134268537	-4.00000009219105\\
1.81963927855711	-4.0000000932935\\
1.80360721442886	-4.00000009437275\\
1.7875751503006	-4.00000009542768\\
1.77154308617235	-4.00000009645721\\
1.75551102204409	-4.00000009746027\\
1.73947895791583	-4.0000000984358\\
1.72344689378758	-4.00000009938277\\
1.70741482965932	-4.00000010030017\\
1.69138276553106	-4.00000010118702\\
1.67535070140281	-4.00000010204237\\
1.65931863727455	-4.00000010286528\\
1.64328657314629	-4.00000010365485\\
1.62725450901804	-4.00000010441023\\
1.61122244488978	-4.00000010513056\\
1.59519038076152	-4.00000010581507\\
1.57915831663327	-4.00000010646298\\
1.56312625250501	-4.00000010707356\\
1.54709418837675	-4.00000010764614\\
1.5310621242485	-4.00000010818007\\
1.51503006012024	-4.00000010867473\\
1.49899799599198	-4.00000010912958\\
1.48296593186373	-4.00000010954409\\
1.46693386773547	-4.00000010991779\\
1.45090180360721	-4.00000011025025\\
1.43486973947896	-4.00000011054108\\
1.4188376753507	-4.00000011078997\\
1.40280561122244	-4.00000011099661\\
1.38677354709419	-4.00000011116077\\
1.37074148296593	-4.00000011128227\\
1.35470941883768	-4.00000011136095\\
1.33867735470942	-4.00000011139673\\
1.32264529058116	-4.00000011138958\\
1.30661322645291	-4.00000011133948\\
1.29058116232465	-4.00000011124652\\
1.27454909819639	-4.00000011111078\\
1.25851703406814	-4.00000011093244\\
1.24248496993988	-4.00000011071169\\
1.22645290581162	-4.00000011044878\\
1.21042084168337	-4.00000011014403\\
1.19438877755511	-4.00000010979778\\
1.17835671342685	-4.00000010941043\\
1.1623246492986	-4.00000010898242\\
1.14629258517034	-4.00000010851424\\
1.13026052104208	-4.00000010800642\\
1.11422845691383	-4.00000010745955\\
1.09819639278557	-4.00000010687422\\
1.08216432865731	-4.00000010625112\\
1.06613226452906	-4.00000010559093\\
1.0501002004008	-4.00000010489439\\
1.03406813627254	-4.00000010416229\\
1.01803607214429	-4.00000010339542\\
1.00200400801603	-4.00000010259463\\
0.985971943887775	-4.00000010176081\\
0.969939879759519	-4.00000010089486\\
0.953907815631262	-4.00000009999772\\
0.937875751503006	-4.00000009907035\\
0.921843687374749	-4.00000009811374\\
0.905811623246493	-4.00000009712892\\
0.889779559118236	-4.00000009611692\\
0.87374749498998	-4.00000009507881\\
0.857715430861724	-4.00000009401565\\
0.841683366733467	-4.00000009292854\\
0.825651302605211	-4.0000000918186\\
0.809619238476954	-4.00000009068694\\
0.793587174348698	-4.00000008953471\\
0.777555110220441	-4.00000008836304\\
0.761523046092185	-4.00000008717309\\
0.745490981963928	-4.00000008596601\\
0.729458917835672	-4.00000008474298\\
0.713426853707415	-4.00000008350514\\
0.697394789579159	-4.00000008225366\\
0.681362725450902	-4.00000008098971\\
0.665330661322646	-4.00000007971444\\
0.649298597194389	-4.00000007842901\\
0.633266533066132	-4.00000007713456\\
0.617234468937876	-4.00000007583223\\
0.601202404809619	-4.00000007452316\\
0.585170340681363	-4.00000007320845\\
0.569138276553106	-4.00000007188922\\
0.55310621242485	-4.00000007056655\\
0.537074148296593	-4.00000006924151\\
0.521042084168337	-4.00000006791516\\
0.50501002004008	-4.00000006658855\\
0.488977955911824	-4.00000006526268\\
0.472945891783567	-4.00000006393855\\
0.456913827655311	-4.00000006261714\\
0.440881763527054	-4.00000006129941\\
0.424849699398798	-4.00000005998627\\
0.408817635270541	-4.00000005867864\\
0.392785571142285	-4.00000005737738\\
0.376753507014028	-4.00000005608336\\
0.360721442885771	-4.00000005479739\\
0.344689378757515	-4.00000005352027\\
0.328657314629258	-4.00000005225276\\
0.312625250501002	-4.00000005099561\\
0.296593186372745	-4.00000004974951\\
0.280561122244489	-4.00000004851516\\
0.264529058116232	-4.0000000472932\\
0.248496993987976	-4.00000004608424\\
0.232464929859719	-4.00000004488888\\
0.216432865731463	-4.00000004370767\\
0.200400801603206	-4.00000004254114\\
0.18436873747495	-4.00000004138978\\
0.168336673346693	-4.00000004025406\\
0.152304609218437	-4.00000003913442\\
0.13627254509018	-4.00000003803125\\
0.120240480961924	-4.00000003694493\\
0.104208416833667	-4.0000000358758\\
0.0881763527054105	-4.00000003482419\\
0.0721442885771539	-4.00000003379037\\
0.0561122244488974	-4.00000003277461\\
0.0400801603206409	-4.00000003177712\\
0.0240480961923843	-4.00000003079812\\
0.00801603206412782	-4.00000002983777\\
-0.00801603206412826	-4.00000002889623\\
-0.0240480961923848	-4.0000000279736\\
-0.0400801603206413	-4.00000002707\\
-0.0561122244488979	-4.00000002618549\\
-0.0721442885771544	-4.00000002532012\\
-0.0881763527054109	-4.00000002447391\\
-0.104208416833667	-4.00000002364686\\
-0.120240480961924	-4.00000002283895\\
-0.13627254509018	-4.00000002205014\\
-0.152304609218437	-4.00000002128037\\
-0.168336673346694	-4.00000002052955\\
-0.18436873747495	-4.00000001979759\\
-0.200400801603207	-4.00000001908437\\
-0.216432865731463	-4.00000001838976\\
-0.232464929859719	-4.00000001771359\\
-0.248496993987976	-4.00000001705571\\
-0.264529058116232	-4.00000001641593\\
-0.280561122244489	-4.00000001579406\\
-0.296593186372745	-4.00000001518989\\
-0.312625250501002	-4.0000000146032\\
-0.328657314629258	-4.00000001403376\\
-0.344689378757515	-4.00000001348132\\
-0.360721442885771	-4.00000001294564\\
-0.376753507014028	-4.00000001242646\\
-0.392785571142285	-4.00000001192349\\
-0.408817635270541	-4.00000001143648\\
-0.424849699398798	-4.00000001096513\\
-0.440881763527054	-4.00000001050915\\
-0.456913827655311	-4.00000001006825\\
-0.472945891783567	-4.00000000964213\\
-0.488977955911824	-4.00000000923049\\
-0.50501002004008	-4.00000000883301\\
-0.521042084168337	-4.00000000844939\\
-0.537074148296593	-4.00000000807932\\
-0.55310621242485	-4.00000000772248\\
-0.569138276553106	-4.00000000737855\\
-0.585170340681363	-4.00000000704722\\
-0.601202404809619	-4.00000000672818\\
-0.617234468937876	-4.0000000064211\\
-0.633266533066132	-4.00000000612568\\
-0.649298597194389	-4.0000000058416\\
-0.665330661322646	-4.00000000556854\\
-0.681362725450902	-4.0000000053062\\
-0.697394789579158	-4.00000000505427\\
-0.713426853707415	-4.00000000481245\\
-0.729458917835671	-4.00000000458043\\
-0.745490981963928	-4.00000000435791\\
-0.761523046092184	-4.00000000414461\\
-0.777555110220441	-4.00000000394023\\
-0.793587174348697	-4.00000000374448\\
-0.809619238476954	-4.00000000355709\\
-0.82565130260521	-4.00000000337777\\
-0.841683366733467	-4.00000000320625\\
-0.857715430861723	-4.00000000304227\\
-0.87374749498998	-4.00000000288557\\
-0.889779559118236	-4.00000000273588\\
-0.905811623246493	-4.00000000259296\\
-0.92184368737475	-4.00000000245655\\
-0.937875751503006	-4.00000000232643\\
-0.953907815631263	-4.00000000220234\\
-0.969939879759519	-4.00000000208408\\
-0.985971943887776	-4.0000000019714\\
-1.00200400801603	-4.0000000018641\\
-1.01803607214429	-4.00000000176195\\
-1.03406813627255	-4.00000000166477\\
-1.0501002004008	-4.00000000157233\\
-1.06613226452906	-4.00000000148446\\
-1.08216432865731	-4.00000000140096\\
-1.09819639278557	-4.00000000132164\\
-1.11422845691383	-4.00000000124633\\
-1.13026052104208	-4.00000000117487\\
-1.14629258517034	-4.00000000110707\\
-1.1623246492986	-4.00000000104278\\
-1.17835671342685	-4.00000000098185\\
-1.19438877755511	-4.00000000092412\\
-1.21042084168337	-4.00000000086946\\
-1.22645290581162	-4.00000000081771\\
-1.24248496993988	-4.00000000076874\\
-1.25851703406814	-4.00000000072243\\
-1.27454909819639	-4.00000000067864\\
-1.29058116232465	-4.00000000063727\\
-1.30661322645291	-4.00000000059818\\
-1.32264529058116	-4.00000000056128\\
-1.33867735470942	-4.00000000052645\\
-1.35470941883768	-4.00000000049359\\
-1.37074148296593	-4.0000000004626\\
-1.38677354709419	-4.00000000043339\\
-1.40280561122244	-4.00000000040587\\
-1.4188376753507	-4.00000000037995\\
-1.43486973947896	-4.00000000035555\\
-1.45090180360721	-4.00000000033259\\
-1.46693386773547	-4.00000000031099\\
-1.48296593186373	-4.00000000029068\\
-1.49899799599198	-4.00000000027159\\
-1.51503006012024	-4.00000000025366\\
-1.5310621242485	-4.00000000023682\\
-1.54709418837675	-4.00000000022102\\
-1.56312625250501	-4.00000000020618\\
-1.57915831663327	-4.00000000019227\\
-1.59519038076152	-4.00000000017923\\
-1.61122244488978	-4.00000000016701\\
-1.62725450901804	-4.00000000015557\\
-1.64328657314629	-4.00000000014485\\
-1.65931863727455	-4.00000000013482\\
-1.67535070140281	-4.00000000012543\\
-1.69138276553106	-4.00000000011665\\
-1.70741482965932	-4.00000000010845\\
-1.72344689378758	-4.00000000010078\\
-1.73947895791583	-4.00000000009362\\
-1.75551102204409	-4.00000000008693\\
-1.77154308617234	-4.0000000000807\\
-1.7875751503006	-4.00000000007488\\
-1.80360721442886	-4.00000000006945\\
-1.81963927855711	-4.00000000006439\\
-1.83567134268537	-4.00000000005968\\
-1.85170340681363	-4.00000000005529\\
-1.86773547094188	-4.0000000000512\\
-1.88376753507014	-4.0000000000474\\
-1.8997995991984	-4.00000000004386\\
-1.91583166332665	-4.00000000004057\\
-1.93186372745491	-4.00000000003751\\
-1.94789579158317	-4.00000000003468\\
-1.96392785571142	-4.00000000003204\\
-1.97995991983968	-4.00000000002959\\
-1.99599198396794	-4.00000000002732\\
-2.01202404809619	-4.00000000002521\\
-2.02805611222445	-4.00000000002326\\
-2.04408817635271	-4.00000000002145\\
-2.06012024048096	-4.00000000001977\\
-2.07615230460922	-4.00000000001822\\
-2.09218436873747	-4.00000000001678\\
-2.10821643286573	-4.00000000001545\\
-2.12424849699399	-4.00000000001422\\
-2.14028056112224	-4.00000000001308\\
-2.1563126252505	-4.00000000001203\\
-2.17234468937876	-4.00000000001106\\
-2.18837675350701	-4.00000000001017\\
-2.20440881763527	-4.00000000000934\\
-2.22044088176353	-4.00000000000858\\
-2.23647294589178	-4.00000000000787\\
-2.25250501002004	-4.00000000000722\\
-2.2685370741483	-4.00000000000662\\
-2.28456913827655	-4.00000000000607\\
-2.30060120240481	-4.00000000000557\\
-2.31663326653307	-4.0000000000051\\
-2.33266533066132	-4.00000000000467\\
-2.34869739478958	-4.00000000000428\\
-2.36472945891784	-4.00000000000391\\
-2.38076152304609	-4.00000000000358\\
-2.39679358717435	-4.00000000000327\\
-2.41282565130261	-4.00000000000299\\
-2.42885771543086	-4.00000000000273\\
-2.44488977955912	-4.0000000000025\\
-2.46092184368737	-4.00000000000228\\
-2.47695390781563	-4.00000000000208\\
-2.49298597194389	-4.0000000000019\\
-2.50901803607214	-4.00000000000173\\
-2.5250501002004	-4.00000000000158\\
-2.54108216432866	-4.00000000000144\\
-2.55711422845691	-4.00000000000131\\
-2.57314629258517	-4.00000000000119\\
-2.58917835671343	-4.00000000000108\\
-2.60521042084168	-4.00000000000099\\
-2.62124248496994	-4.0000000000009\\
-2.6372745490982	-4.00000000000082\\
-2.65330661322645	-4.00000000000074\\
-2.66933867735471	-4.00000000000067\\
-2.68537074148297	-4.00000000000061\\
-2.70140280561122	-4.00000000000056\\
-2.71743486973948	-4.0000000000005\\
-2.73346693386774	-4.00000000000046\\
-2.74949899799599	-4.00000000000041\\
-2.76553106212425	-4.00000000000038\\
-2.7815631262525	-4.00000000000034\\
-2.79759519038076	-4.00000000000031\\
-2.81362725450902	-4.00000000000028\\
-2.82965931863727	-4.00000000000025\\
-2.84569138276553	-4.00000000000023\\
-2.86172344689379	-4.00000000000021\\
-2.87775551102204	-4.00000000000019\\
-2.8937875751503	-4.00000000000017\\
-2.90981963927856	-4.00000000000015\\
-2.92585170340681	-4.00000000000014\\
-2.94188376753507	-4.00000000000012\\
-2.95791583166333	-4.00000000000011\\
-2.97394789579158	-4.0000000000001\\
-2.98997995991984	-4.00000000000009\\
-3.0060120240481	-4.00000000000008\\
-3.02204408817635	-4.00000000000007\\
-3.03807615230461	-4.00000000000007\\
-3.05410821643287	-4.00000000000006\\
-3.07014028056112	-4.00000000000005\\
-3.08617234468938	-4.00000000000005\\
-3.10220440881764	-4.00000000000004\\
-3.11823647294589	-4.00000000000004\\
-3.13426853707415	-4.00000000000003\\
-3.1503006012024	-4.00000000000003\\
-3.16633266533066	-4.00000000000003\\
-3.18236472945892	-4.00000000000003\\
-3.19839679358717	-4.00000000000002\\
-3.21442885771543	-4.00000000000002\\
-3.23046092184369	-4.00000000000002\\
-3.24649298597194	-4.00000000000002\\
-3.2625250501002	-4.00000000000001\\
-3.27855711422846	-4.00000000000001\\
-3.29458917835671	-4.00000000000001\\
-3.31062124248497	-4.00000000000001\\
-3.32665330661323	-4.00000000000001\\
-3.34268537074148	-4.00000000000001\\
-3.35871743486974	-4.00000000000001\\
-3.374749498998	-4.00000000000001\\
-3.39078156312625	-4.00000000000001\\
-3.40681362725451	-4.00000000000001\\
-3.42284569138277	-4\\
-3.43887775551102	-4\\
-3.45490981963928	-4\\
-3.47094188376753	-4\\
-3.48697394789579	-4\\
-3.50300601202405	-4\\
-3.5190380761523	-4\\
-3.53507014028056	-4\\
-3.55110220440882	-4\\
-3.56713426853707	-4\\
-3.58316633266533	-4\\
-3.59919839679359	-4\\
-3.61523046092184	-4\\
-3.6312625250501	-4\\
-3.64729458917836	-4\\
-3.66332665330661	-4\\
-3.67935871743487	-4\\
-3.69539078156313	-4\\
-3.71142284569138	-4\\
-3.72745490981964	-4\\
-3.7434869739479	-4\\
-3.75951903807615	-4\\
-3.77555110220441	-4\\
-3.79158316633267	-4\\
-3.80761523046092	-4\\
-3.82364729458918	-4\\
-3.83967935871743	-4\\
-3.85571142284569	-4\\
-3.87174348697395	-4\\
-3.8877755511022	-4\\
-3.90380761523046	-4\\
-3.91983967935872	-4\\
-3.93587174348697	-4\\
-3.95190380761523	-4\\
-3.96793587174349	-4\\
-3.98396793587174	-4\\
-4	-4\\
-4	-3.98396793587174\\
-4	-3.96793587174349\\
-4	-3.95190380761523\\
-4	-3.93587174348697\\
-4	-3.91983967935872\\
-4	-3.90380761523046\\
-4	-3.8877755511022\\
-4	-3.87174348697395\\
-4	-3.85571142284569\\
-4	-3.83967935871743\\
-4	-3.82364729458918\\
-4	-3.80761523046092\\
-4	-3.79158316633267\\
-4	-3.77555110220441\\
-4	-3.75951903807615\\
-4	-3.7434869739479\\
-4	-3.72745490981964\\
-4	-3.71142284569138\\
-4	-3.69539078156313\\
-4	-3.67935871743487\\
-4	-3.66332665330661\\
-4	-3.64729458917836\\
-4	-3.6312625250501\\
-4	-3.61523046092184\\
-4	-3.59919839679359\\
-4	-3.58316633266533\\
-4	-3.56713426853707\\
-4	-3.55110220440882\\
-4	-3.53507014028056\\
-4	-3.5190380761523\\
-4	-3.50300601202405\\
-4	-3.48697394789579\\
-4	-3.47094188376753\\
-4	-3.45490981963928\\
-4	-3.43887775551102\\
-4	-3.42284569138277\\
-4	-3.40681362725451\\
-4	-3.39078156312625\\
-4	-3.374749498998\\
-4	-3.35871743486974\\
-4	-3.34268537074148\\
-4	-3.32665330661323\\
-4	-3.31062124248497\\
-4	-3.29458917835671\\
-4	-3.27855711422846\\
-4	-3.2625250501002\\
-4	-3.24649298597194\\
-4	-3.23046092184369\\
-4	-3.21442885771543\\
-4	-3.19839679358717\\
-4	-3.18236472945892\\
-4	-3.16633266533066\\
-4	-3.1503006012024\\
-4	-3.13426853707415\\
-4	-3.11823647294589\\
-4	-3.10220440881764\\
-4	-3.08617234468938\\
-4	-3.07014028056112\\
-4	-3.05410821643287\\
-4	-3.03807615230461\\
-4	-3.02204408817635\\
-4	-3.0060120240481\\
-4	-2.98997995991984\\
-4	-2.97394789579158\\
-4	-2.95791583166333\\
-4	-2.94188376753507\\
-4	-2.92585170340681\\
-4	-2.90981963927856\\
-4	-2.8937875751503\\
-4	-2.87775551102204\\
-4	-2.86172344689379\\
-4	-2.84569138276553\\
-4	-2.82965931863727\\
-4	-2.81362725450902\\
-4.00000000000001	-2.79759519038076\\
-4.00000000000001	-2.7815631262525\\
-4.00000000000001	-2.76553106212425\\
-4.00000000000001	-2.74949899799599\\
-4.00000000000001	-2.73346693386774\\
-4.00000000000001	-2.71743486973948\\
-4.00000000000001	-2.70140280561122\\
-4.00000000000001	-2.68537074148297\\
-4.00000000000001	-2.66933867735471\\
-4.00000000000001	-2.65330661322645\\
-4.00000000000001	-2.6372745490982\\
-4.00000000000001	-2.62124248496994\\
-4.00000000000001	-2.60521042084168\\
-4.00000000000001	-2.58917835671343\\
-4.00000000000001	-2.57314629258517\\
-4.00000000000001	-2.55711422845691\\
-4.00000000000001	-2.54108216432866\\
-4.00000000000001	-2.5250501002004\\
-4.00000000000001	-2.50901803607214\\
-4.00000000000001	-2.49298597194389\\
-4.00000000000002	-2.47695390781563\\
-4.00000000000002	-2.46092184368737\\
-4.00000000000002	-2.44488977955912\\
-4.00000000000002	-2.42885771543086\\
-4.00000000000002	-2.41282565130261\\
-4.00000000000002	-2.39679358717435\\
-4.00000000000002	-2.38076152304609\\
-4.00000000000002	-2.36472945891784\\
-4.00000000000002	-2.34869739478958\\
-4.00000000000002	-2.33266533066132\\
-4.00000000000002	-2.31663326653307\\
-4.00000000000003	-2.30060120240481\\
-4.00000000000003	-2.28456913827655\\
-4.00000000000003	-2.2685370741483\\
-4.00000000000003	-2.25250501002004\\
-4.00000000000003	-2.23647294589178\\
-4.00000000000003	-2.22044088176353\\
-4.00000000000004	-2.20440881763527\\
-4.00000000000004	-2.18837675350701\\
-4.00000000000004	-2.17234468937876\\
-4.00000000000004	-2.1563126252505\\
-4.00000000000004	-2.14028056112224\\
-4.00000000000005	-2.12424849699399\\
-4.00000000000005	-2.10821643286573\\
-4.00000000000005	-2.09218436873747\\
-4.00000000000005	-2.07615230460922\\
-4.00000000000006	-2.06012024048096\\
-4.00000000000006	-2.04408817635271\\
-4.00000000000006	-2.02805611222445\\
-4.00000000000006	-2.01202404809619\\
-4.00000000000007	-1.99599198396794\\
-4.00000000000007	-1.97995991983968\\
-4.00000000000007	-1.96392785571142\\
-4.00000000000008	-1.94789579158317\\
-4.00000000000008	-1.93186372745491\\
-4.00000000000009	-1.91583166332665\\
-4.00000000000009	-1.8997995991984\\
-4.00000000000009	-1.88376753507014\\
-4.0000000000001	-1.86773547094188\\
-4.0000000000001	-1.85170340681363\\
-4.00000000000011	-1.83567134268537\\
-4.00000000000011	-1.81963927855711\\
-4.00000000000012	-1.80360721442886\\
-4.00000000000012	-1.7875751503006\\
-4.00000000000013	-1.77154308617234\\
-4.00000000000014	-1.75551102204409\\
-4.00000000000014	-1.73947895791583\\
-4.00000000000015	-1.72344689378758\\
-4.00000000000016	-1.70741482965932\\
-4.00000000000016	-1.69138276553106\\
-4.00000000000017	-1.67535070140281\\
-4.00000000000018	-1.65931863727455\\
-4.00000000000019	-1.64328657314629\\
-4.0000000000002	-1.62725450901804\\
-4.00000000000021	-1.61122244488978\\
-4.00000000000021	-1.59519038076152\\
-4.00000000000022	-1.57915831663327\\
-4.00000000000023	-1.56312625250501\\
-4.00000000000025	-1.54709418837675\\
-4.00000000000026	-1.5310621242485\\
-4.00000000000027	-1.51503006012024\\
-4.00000000000028	-1.49899799599198\\
-4.00000000000029	-1.48296593186373\\
-4.00000000000031	-1.46693386773547\\
-4.00000000000032	-1.45090180360721\\
-4.00000000000033	-1.43486973947896\\
-4.00000000000035	-1.4188376753507\\
-4.00000000000036	-1.40280561122244\\
-4.00000000000038	-1.38677354709419\\
-4.0000000000004	-1.37074148296593\\
-4.00000000000041	-1.35470941883768\\
-4.00000000000043	-1.33867735470942\\
-4.00000000000045	-1.32264529058116\\
-4.00000000000047	-1.30661322645291\\
-4.00000000000049	-1.29058116232465\\
-4.00000000000051	-1.27454909819639\\
-4.00000000000053	-1.25851703406814\\
-4.00000000000056	-1.24248496993988\\
-4.00000000000058	-1.22645290581162\\
-4.00000000000061	-1.21042084168337\\
-4.00000000000063	-1.19438877755511\\
-4.00000000000066	-1.17835671342685\\
-4.00000000000069	-1.1623246492986\\
-4.00000000000072	-1.14629258517034\\
-4.00000000000075	-1.13026052104208\\
-4.00000000000078	-1.11422845691383\\
-4.00000000000081	-1.09819639278557\\
-4.00000000000084	-1.08216432865731\\
-4.00000000000088	-1.06613226452906\\
-4.00000000000092	-1.0501002004008\\
-4.00000000000095	-1.03406813627255\\
-4.00000000000099	-1.01803607214429\\
-4.00000000000103	-1.00200400801603\\
-4.00000000000108	-0.985971943887776\\
-4.00000000000112	-0.969939879759519\\
-4.00000000000116	-0.953907815631263\\
-4.00000000000121	-0.937875751503006\\
-4.00000000000126	-0.92184368737475\\
-4.00000000000131	-0.905811623246493\\
-4.00000000000136	-0.889779559118236\\
-4.00000000000142	-0.87374749498998\\
-4.00000000000147	-0.857715430861723\\
-4.00000000000153	-0.841683366733467\\
-4.00000000000159	-0.82565130260521\\
-4.00000000000166	-0.809619238476954\\
-4.00000000000172	-0.793587174348697\\
-4.00000000000179	-0.777555110220441\\
-4.00000000000186	-0.761523046092184\\
-4.00000000000193	-0.745490981963928\\
-4.000000000002	-0.729458917835671\\
-4.00000000000208	-0.713426853707415\\
-4.00000000000216	-0.697394789579158\\
-4.00000000000225	-0.681362725450902\\
-4.00000000000233	-0.665330661322646\\
-4.00000000000242	-0.649298597194389\\
-4.00000000000251	-0.633266533066132\\
-4.00000000000261	-0.617234468937876\\
-4.0000000000027	-0.601202404809619\\
-4.00000000000281	-0.585170340681363\\
-4.00000000000291	-0.569138276553106\\
-4.00000000000302	-0.55310621242485\\
-4.00000000000313	-0.537074148296593\\
-4.00000000000325	-0.521042084168337\\
-4.00000000000337	-0.50501002004008\\
-4.00000000000349	-0.488977955911824\\
-4.00000000000362	-0.472945891783567\\
-4.00000000000375	-0.456913827655311\\
-4.00000000000388	-0.440881763527054\\
-4.00000000000403	-0.424849699398798\\
-4.00000000000417	-0.408817635270541\\
-4.00000000000432	-0.392785571142285\\
-4.00000000000448	-0.376753507014028\\
-4.00000000000463	-0.360721442885771\\
-4.0000000000048	-0.344689378757515\\
-4.00000000000497	-0.328657314629258\\
-4.00000000000514	-0.312625250501002\\
-4.00000000000532	-0.296593186372745\\
-4.00000000000551	-0.280561122244489\\
-4.0000000000057	-0.264529058116232\\
-4.0000000000059	-0.248496993987976\\
-4.00000000000611	-0.232464929859719\\
-4.00000000000631	-0.216432865731463\\
-4.00000000000653	-0.200400801603207\\
-4.00000000000676	-0.18436873747495\\
-4.00000000000699	-0.168336673346694\\
-4.00000000000722	-0.152304609218437\\
-4.00000000000747	-0.13627254509018\\
-4.00000000000772	-0.120240480961924\\
-4.00000000000798	-0.104208416833667\\
-4.00000000000824	-0.0881763527054109\\
-4.00000000000852	-0.0721442885771544\\
-4.0000000000088	-0.0561122244488979\\
-4.00000000000909	-0.0400801603206413\\
-4.00000000000939	-0.0240480961923848\\
-4.00000000000969	-0.00801603206412826\\
-4.00000000001001	0.00801603206412782\\
-4.00000000001033	0.0240480961923843\\
-4.00000000001067	0.0400801603206409\\
-4.00000000001101	0.0561122244488974\\
-4.00000000001137	0.0721442885771539\\
-4.00000000001173	0.0881763527054105\\
-4.0000000000121	0.104208416833667\\
-4.00000000001248	0.120240480961924\\
-4.00000000001288	0.13627254509018\\
-4.00000000001328	0.152304609218437\\
-4.0000000000137	0.168336673346693\\
-4.00000000001412	0.18436873747495\\
-4.00000000001456	0.200400801603206\\
-4.00000000001501	0.216432865731463\\
-4.00000000001547	0.232464929859719\\
-4.00000000001594	0.248496993987976\\
-4.00000000001643	0.264529058116232\\
-4.00000000001693	0.280561122244489\\
-4.00000000001744	0.296593186372745\\
-4.00000000001796	0.312625250501002\\
-4.0000000000185	0.328657314629258\\
-4.00000000001905	0.344689378757515\\
-4.00000000001962	0.360721442885771\\
-4.0000000000202	0.376753507014028\\
-4.00000000002079	0.392785571142285\\
-4.0000000000214	0.408817635270541\\
-4.00000000002202	0.424849699398798\\
-4.00000000002266	0.440881763527054\\
-4.00000000002332	0.456913827655311\\
-4.00000000002399	0.472945891783567\\
-4.00000000002467	0.488977955911824\\
-4.00000000002538	0.50501002004008\\
-4.0000000000261	0.521042084168337\\
-4.00000000002683	0.537074148296593\\
-4.00000000002759	0.55310621242485\\
-4.00000000002836	0.569138276553106\\
-4.00000000002914	0.585170340681363\\
-4.00000000002995	0.601202404809619\\
-4.00000000003078	0.617234468937876\\
-4.00000000003162	0.633266533066132\\
-4.00000000003248	0.649298597194389\\
-4.00000000003337	0.665330661322646\\
-4.00000000003427	0.681362725450902\\
-4.00000000003519	0.697394789579159\\
-4.00000000003613	0.713426853707415\\
-4.00000000003709	0.729458917835672\\
-4.00000000003808	0.745490981963928\\
-4.00000000003908	0.761523046092185\\
-4.00000000004011	0.777555110220441\\
-4.00000000004115	0.793587174348698\\
-4.00000000004222	0.809619238476954\\
-4.00000000004331	0.825651302605211\\
-4.00000000004442	0.841683366733467\\
-4.00000000004556	0.857715430861724\\
-4.00000000004672	0.87374749498998\\
-4.0000000000479	0.889779559118236\\
-4.00000000004911	0.905811623246493\\
-4.00000000005034	0.921843687374749\\
-4.00000000005159	0.937875751503006\\
-4.00000000005287	0.953907815631262\\
-4.00000000005418	0.969939879759519\\
-4.0000000000555	0.985971943887775\\
-4.00000000005686	1.00200400801603\\
-4.00000000005824	1.01803607214429\\
-4.00000000005964	1.03406813627254\\
-4.00000000006107	1.0501002004008\\
-4.00000000006253	1.06613226452906\\
-4.00000000006401	1.08216432865731\\
-4.00000000006553	1.09819639278557\\
-4.00000000006706	1.11422845691383\\
-4.00000000006863	1.13026052104208\\
-4.00000000007022	1.14629258517034\\
-4.00000000007184	1.1623246492986\\
-4.00000000007349	1.17835671342685\\
-4.00000000007516	1.19438877755511\\
-4.00000000007687	1.21042084168337\\
-4.0000000000786	1.22645290581162\\
-4.00000000008037	1.24248496993988\\
-4.00000000008216	1.25851703406814\\
-4.00000000008398	1.27454909819639\\
-4.00000000008582	1.29058116232465\\
-4.0000000000877	1.30661322645291\\
-4.00000000008961	1.32264529058116\\
-4.00000000009155	1.33867735470942\\
-4.00000000009352	1.35470941883768\\
-4.00000000009552	1.37074148296593\\
-4.00000000009754	1.38677354709419\\
-4.0000000000996	1.40280561122244\\
-4.00000000010169	1.4188376753507\\
-4.00000000010381	1.43486973947896\\
-4.00000000010596	1.45090180360721\\
-4.00000000010814	1.46693386773547\\
-4.00000000011035	1.48296593186373\\
-4.00000000011259	1.49899799599198\\
-4.00000000011487	1.51503006012024\\
-4.00000000011717	1.5310621242485\\
-4.00000000011951	1.54709418837675\\
-4.00000000012187	1.56312625250501\\
-4.00000000012427	1.57915831663327\\
-4.00000000012669	1.59519038076152\\
-4.00000000012915	1.61122244488978\\
-4.00000000013164	1.62725450901804\\
-4.00000000013416	1.64328657314629\\
-4.00000000013671	1.65931863727455\\
-4.00000000013929	1.67535070140281\\
-4.0000000001419	1.69138276553106\\
-4.00000000014454	1.70741482965932\\
-4.00000000014721	1.72344689378758\\
-4.00000000014991	1.73947895791583\\
-4.00000000015265	1.75551102204409\\
-4.00000000015541	1.77154308617235\\
-4.0000000001582	1.7875751503006\\
-4.00000000016102	1.80360721442886\\
-4.00000000016387	1.81963927855711\\
-4.00000000016675	1.83567134268537\\
-4.00000000016965	1.85170340681363\\
-4.00000000017259	1.86773547094188\\
-4.00000000017555	1.88376753507014\\
-4.00000000017854	1.8997995991984\\
-4.00000000018156	1.91583166332665\\
-4.00000000018461	1.93186372745491\\
-4.00000000018769	1.94789579158317\\
-4.00000000019079	1.96392785571142\\
-4.00000000019391	1.97995991983968\\
-4.00000000019707	1.99599198396794\\
-4.00000000020024	2.01202404809619\\
-4.00000000020345	2.02805611222445\\
-4.00000000020668	2.04408817635271\\
-4.00000000020993	2.06012024048096\\
-4.0000000002132	2.07615230460922\\
-4.0000000002165	2.09218436873747\\
-4.00000000021983	2.10821643286573\\
-4.00000000022317	2.12424849699399\\
-4.00000000022654	2.14028056112224\\
-4.00000000022992	2.1563126252505\\
-4.00000000023333	2.17234468937876\\
-4.00000000023676	2.18837675350701\\
-4.00000000024021	2.20440881763527\\
-4.00000000024368	2.22044088176353\\
-4.00000000024716	2.23647294589178\\
-4.00000000025066	2.25250501002004\\
-4.00000000025418	2.2685370741483\\
-4.00000000025772	2.28456913827655\\
-4.00000000026127	2.30060120240481\\
-4.00000000026484	2.31663326653307\\
-4.00000000026842	2.33266533066132\\
-4.00000000027201	2.34869739478958\\
-4.00000000027562	2.36472945891784\\
-4.00000000027924	2.38076152304609\\
-4.00000000028287	2.39679358717435\\
-4.00000000028651	2.41282565130261\\
-4.00000000029016	2.42885771543086\\
-4.00000000029382	2.44488977955912\\
-4.00000000029748	2.46092184368737\\
-4.00000000030116	2.47695390781563\\
-4.00000000030484	2.49298597194389\\
-4.00000000030852	2.50901803607214\\
-4.00000000031221	2.5250501002004\\
-4.0000000003159	2.54108216432866\\
-4.0000000003196	2.55711422845691\\
-4.0000000003233	2.57314629258517\\
-4.00000000032699	2.58917835671343\\
-4.00000000033069	2.60521042084168\\
-4.00000000033439	2.62124248496994\\
-4.00000000033808	2.6372745490982\\
-4.00000000034177	2.65330661322645\\
-4.00000000034546	2.66933867735471\\
-4.00000000034914	2.68537074148297\\
-4.00000000035282	2.70140280561122\\
-4.00000000035649	2.71743486973948\\
-4.00000000036015	2.73346693386774\\
-4.0000000003638	2.74949899799599\\
-4.00000000036744	2.76553106212425\\
-4.00000000037107	2.7815631262525\\
-4.00000000037469	2.79759519038076\\
-4.0000000003783	2.81362725450902\\
-4.00000000038189	2.82965931863727\\
-4.00000000038546	2.84569138276553\\
-4.00000000038902	2.86172344689379\\
-4.00000000039256	2.87775551102204\\
-4.00000000039608	2.8937875751503\\
-4.00000000039958	2.90981963927856\\
-4.00000000040307	2.92585170340681\\
-4.00000000040653	2.94188376753507\\
-4.00000000040996	2.95791583166333\\
-4.00000000041337	2.97394789579158\\
-4.00000000041676	2.98997995991984\\
-4.00000000042012	3.0060120240481\\
-4.00000000042346	3.02204408817635\\
-4.00000000042676	3.03807615230461\\
-4.00000000043004	3.05410821643287\\
-4.00000000043328	3.07014028056112\\
-4.0000000004365	3.08617234468938\\
-4.00000000043968	3.10220440881764\\
-4.00000000044282	3.11823647294589\\
-4.00000000044594	3.13426853707415\\
-4.00000000044901	3.1503006012024\\
-4.00000000045205	3.16633266533066\\
-4.00000000045505	3.18236472945892\\
-4.00000000045802	3.19839679358717\\
-4.00000000046094	3.21442885771543\\
-4.00000000046382	3.23046092184369\\
-4.00000000046666	3.24649298597194\\
-4.00000000046946	3.2625250501002\\
-4.00000000047221	3.27855711422846\\
-4.00000000047492	3.29458917835671\\
-4.00000000047758	3.31062124248497\\
-4.0000000004802	3.32665330661323\\
-4.00000000048277	3.34268537074148\\
-4.00000000048529	3.35871743486974\\
-4.00000000048776	3.374749498998\\
-4.00000000049017	3.39078156312625\\
-4.00000000049254	3.40681362725451\\
-4.00000000049486	3.42284569138277\\
-4.00000000049712	3.43887775551102\\
-4.00000000049933	3.45490981963928\\
-4.00000000050148	3.47094188376754\\
-4.00000000050358	3.48697394789579\\
-4.00000000050563	3.50300601202405\\
-4.00000000050761	3.51903807615231\\
-4.00000000050954	3.53507014028056\\
-4.00000000051141	3.55110220440882\\
-4.00000000051322	3.56713426853707\\
-4.00000000051497	3.58316633266533\\
-4.00000000051666	3.59919839679359\\
-4.00000000051829	3.61523046092184\\
-4.00000000051986	3.6312625250501\\
-4.00000000052136	3.64729458917836\\
-4.00000000052281	3.66332665330661\\
-4.00000000052419	3.67935871743487\\
-4.0000000005255	3.69539078156313\\
-4.00000000052675	3.71142284569138\\
-4.00000000052794	3.72745490981964\\
-4.00000000052906	3.7434869739479\\
-4.00000000053011	3.75951903807615\\
-4.0000000005311	3.77555110220441\\
-4.00000000053202	3.79158316633267\\
-4.00000000053288	3.80761523046092\\
-4.00000000053367	3.82364729458918\\
-4.00000000053439	3.83967935871743\\
-4.00000000053504	3.85571142284569\\
-4.00000000053563	3.87174348697395\\
-4.00000000053614	3.8877755511022\\
-4.00000000053659	3.90380761523046\\
-4.00000000053697	3.91983967935872\\
-4.00000000053728	3.93587174348697\\
-4.00000000053752	3.95190380761523\\
-4.0000000005377	3.96793587174349\\
-4.0000000005378	3.98396793587174\\
-4.00000000053783	4\\
-4	4.00000000053783\\
}--cycle;


\addplot[area legend,solid,fill=mycolor2,draw=black,forget plot]
table[row sep=crcr] {%
x	y\\
-2.18837675350701	2.51218829191193\\
-2.18733967634176	2.5250501002004\\
-2.18597120113148	2.54108216432866\\
-2.18452376321569	2.55711422845691\\
-2.18299582363611	2.57314629258517\\
-2.18138579001696	2.58917835671343\\
-2.179692015072	2.60521042084168\\
-2.17791279505507	2.62124248496994\\
-2.17604636815184	2.6372745490982\\
-2.17409091281085	2.65330661322645\\
-2.17234468937876	2.66700321609212\\
-2.17205000108829	2.66933867735471\\
-2.169949223652	2.68537074148297\\
-2.16775450582966	2.70140280561122\\
-2.1654637519929	2.71743486973948\\
-2.1630747968375	2.73346693386774\\
-2.16058540325749	2.74949899799599\\
-2.15799326013909	2.76553106212425\\
-2.1563126252505	2.77555497956145\\
-2.15531263058558	2.7815631262525\\
-2.1525529883469	2.79759519038076\\
-2.14968393265089	2.81362725450902\\
-2.14670280909838	2.82965931863727\\
-2.1436068759187	2.84569138276553\\
-2.14039330115298	2.86172344689379\\
-2.14028056112224	2.86227095588746\\
-2.13710656700386	2.87775551102204\\
-2.13369847096792	2.8937875751503\\
-2.13016418146547	2.90981963927856\\
-2.12650044307887	2.92585170340681\\
-2.12424849699399	2.9354039450745\\
-2.12272457763207	2.94188376753507\\
-2.11884335835732	2.95791583166333\\
-2.11482256942741	2.97394789579158\\
-2.11065846187091	2.98997995991984\\
-2.10821643286573	2.99910755738531\\
-2.10636965934285	3.0060120240481\\
-2.10195962779741	3.02204408817635\\
-2.09739431105722	3.03807615230461\\
-2.09266938106967	3.05410821643287\\
-2.09218436873747	3.05571690098827\\
-2.08782724256108	3.07014028056112\\
-2.08282144178327	3.08617234468938\\
-2.07764187201208	3.10220440881764\\
-2.07615230460922	3.10670417198628\\
-2.0723199728081	3.11823647294589\\
-2.0668279322443	3.13426853707415\\
-2.06114596808467	3.1503006012024\\
-2.06012024048096	3.15313095886327\\
-2.05530808263616	3.16633266533066\\
-2.04927652476668	3.18236472945892\\
-2.04408817635271	3.19572743614558\\
-2.04304403780948	3.19839679358717\\
-2.03663353865448	3.21442885771543\\
-2.03000039851701	3.23046092184369\\
-2.02805611222445	3.23505272624421\\
-2.02316612665281	3.24649298597194\\
-2.01610504929561	3.2625250501002\\
-2.01202404809619	3.27154119214257\\
-2.00881293353938	3.27855711422846\\
-2.00128462031002	3.29458917835671\\
-1.99599198396794	3.30553466284577\\
-1.99350063178841	3.31062124248497\\
-1.98546057570419	3.32665330661323\\
-1.97995991983968	3.33730692442868\\
-1.97714210753077	3.34268537074148\\
-1.96853981202308	3.35871743486974\\
-1.96392785571142	3.36708567002893\\
-1.95963426252646	3.374749498998\\
-1.9504123103506	3.39078156312625\\
-1.94789579158317	3.39506013426401\\
-1.9408553430388	3.40681362725451\\
-1.93186372745491	3.42138894870863\\
-1.93094713526962	3.42284569138277\\
-1.92066168164344	3.43887775551102\\
-1.91583166332665	3.44621303528219\\
-1.90997946975807	3.45490981963928\\
-1.8997995991984	3.46964207855806\\
-1.89888029101296	3.47094188376754\\
-1.8873210583821	3.48697394789579\\
-1.88376753507014	3.4917892515538\\
-1.87527724755564	3.50300601202405\\
-1.86773547094188	3.51273640977956\\
-1.86271726148988	3.51903807615231\\
-1.85170340681363	3.53256061063652\\
-1.84959972635276	3.53507014028056\\
-1.8358740682194	3.55110220440882\\
-1.83567134268537	3.55133486431535\\
-1.82146947117734	3.56713426853707\\
-1.81963927855711	3.56913032217164\\
-1.80633755317204	3.58316633266533\\
-1.80360721442886	3.58599389569402\\
-1.79040271332929	3.59919839679359\\
-1.7875751503006	3.6019764910627\\
-1.77357601502004	3.61523046092184\\
-1.77154308617234	3.6171235514231\\
-1.75575261167513	3.6312625250501\\
-1.75551102204409	3.63147552060405\\
-1.73947895791583	3.64508353874812\\
-1.73675998401568	3.64729458917836\\
-1.72344689378758	3.65797252578098\\
-1.71645946836329	3.66332665330661\\
-1.70741482965932	3.67016942626541\\
-1.69466730704812	3.67935871743487\\
-1.69138276553106	3.68169890685915\\
-1.67535070140281	3.69260532634724\\
-1.67104927101087	3.69539078156313\\
-1.65931863727455	3.70291133356174\\
-1.64527066679244	3.71142284569138\\
-1.64328657314629	3.71261420294734\\
-1.62725450901804	3.72177920887112\\
-1.61670775334827	3.72745490981964\\
-1.61122244488978	3.7303851331788\\
-1.59519038076152	3.73847872555087\\
-1.5846136617279	3.7434869739479\\
-1.57915831663327	3.74605536787514\\
-1.56312625250501	3.75315746679068\\
-1.54767777357342	3.75951903807615\\
-1.54709418837675	3.75975836220358\\
-1.5310621242485	3.76593753192175\\
-1.51503006012024	3.77164679629925\\
-1.50316156487214	3.77555110220441\\
-1.49899799599198	3.77691813026863\\
-1.48296593186373	3.7817922824152\\
-1.46693386773547	3.78624079510059\\
-1.45090180360721	3.7902776572105\\
-1.44519057425423	3.79158316633267\\
-1.43486973947896	3.79394442068093\\
-1.4188376753507	3.79723728898389\\
-1.40280561122244	3.80015222065808\\
-1.38677354709419	3.80270011516812\\
-1.37074148296593	3.80489119658113\\
-1.35470941883768	3.80673504142009\\
-1.34536965782306	3.80761523046092\\
-1.33867735470942	3.80824897603755\\
-1.32264529058116	3.80944066381979\\
-1.30661322645291	3.81030619218637\\
-1.29058116232465	3.81085272362304\\
-1.27454909819639	3.81108688513842\\
-1.25851703406814	3.81101478617959\\
-1.24248496993988	3.81064203520423\\
-1.22645290581162	3.80997375496874\\
-1.21042084168337	3.80901459658709\\
-1.19438877755511	3.80776875241063\\
-1.19278466694236	3.80761523046092\\
-1.17835671342685	3.80626041352538\\
-1.1623246492986	3.80447940168065\\
-1.14629258517034	3.80242643223183\\
-1.13026052104208	3.80010406348015\\
-1.11422845691383	3.79751444724034\\
-1.09819639278557	3.79465933573617\\
-1.08238610784513	3.79158316633267\\
-1.08216432865731	3.79154076292006\\
-1.06613226452906	3.78821192551388\\
-1.0501002004008	3.7846257924548\\
-1.03406813627255	3.78078268667116\\
-1.01803607214429	3.77668255792351\\
-1.01387250326413	3.77555110220441\\
-1.00200400801603	3.7723776273203\\
-0.985971943887776	3.76783838447288\\
-0.969939879759519	3.76304552242583\\
-0.958735177018894	3.75951903807615\\
-0.953907815631263	3.75802296121048\\
-0.937875751503006	3.75280765424696\\
-0.92184368737475	3.74733952723095\\
-0.91104327932147	3.7434869739479\\
-0.905811623246493	3.74164795998862\\
-0.889779559118236	3.73576945779861\\
-0.87374749498998	3.72963642279804\\
-0.868262186531491	3.72745490981964\\
-0.857715430861723	3.7233182685757\\
-0.841683366733467	3.71678256208697\\
-0.829025370107675	3.71142284569138\\
-0.82565130260521	3.71001314770832\\
-0.809619238476954	3.70308011609804\\
-0.793587174348697	3.69588653505746\\
-0.792516323769336	3.69539078156313\\
-0.777555110220441	3.68855112833831\\
-0.761523046092184	3.68096111671956\\
-0.758238504575129	3.67935871743487\\
-0.745490981963928	3.6732143754226\\
-0.729458917835671	3.66522904433948\\
-0.725746813913297	3.66332665330661\\
-0.713426853707415	3.65708548600821\\
-0.697394789579158	3.64870434829676\\
-0.69476868281116	3.64729458917836\\
-0.681362725450902	3.64017704005774\\
-0.665330661322646	3.63139803025136\\
-0.665089071691607	3.6312625250501\\
-0.649298597194389	3.62249864928523\\
-0.636584308250442	3.61523046092184\\
-0.633266533066132	3.61335323719904\\
-0.617234468937876	3.60405701947759\\
-0.609073991017453	3.59919839679359\\
-0.601202404809619	3.59455732732911\\
-0.585170340681363	3.58485599522903\\
-0.582440001938185	3.58316633266533\\
-0.569138276553106	3.57501041351771\\
-0.556626181278272	3.56713426853707\\
-0.55310621242485	3.56493835321383\\
-0.537074148296593	3.55471227402176\\
-0.531542492394103	3.55110220440882\\
-0.521042084168337	3.5443073060698\\
-0.507113700500945	3.53507014028056\\
-0.50501002004008	3.53368642806086\\
-0.488977955911824	3.52292736966742\\
-0.483308498413444	3.51903807615231\\
-0.472945891783567	3.51198384860222\\
-0.46005621128414	3.50300601202405\\
-0.456913827655311	3.50083355292382\\
-0.440881763527054	3.48952899261935\\
-0.437328240215089	3.48697394789579\\
-0.424849699398798	3.47806371092255\\
-0.415086534647536	3.47094188376754\\
-0.408817635270541	3.46639943697746\\
-0.393288980010621	3.45490981963928\\
-0.392785571142285	3.454539734248\\
-0.376753507014028	3.44255838458529\\
-0.371923488697241	3.43887775551102\\
-0.360721442885771	3.43039175147638\\
-0.350951914584676	3.42284569138277\\
-0.344689378757515	3.41803573511502\\
-0.330353172608619	3.40681362725451\\
-0.328657314629258	3.40549328796713\\
-0.312625250501002	3.39280623394943\\
-0.310108731733566	3.39078156312625\\
-0.296593186372745	3.37996057446672\\
-0.29019815203592	3.374749498998\\
-0.280561122244489	3.36693288932542\\
-0.270605055257431	3.35871743486974\\
-0.264529058116232	3.35372555976707\\
-0.251314806296885	3.34268537074148\\
-0.248496993987976	3.34034081811575\\
-0.232464929859719	3.32678327605539\\
-0.232313243736369	3.32665330661323\\
-0.216432865731463	3.31309537364555\\
-0.213579442999436	3.31062124248497\\
-0.200400801603207	3.2992324911091\\
-0.195108165261117	3.29458917835671\\
-0.18436873747495	3.28519632550059\\
-0.17688838962525	3.27855711422846\\
-0.168336673346694	3.27098843500974\\
-0.158909757082702	3.2625250501002\\
-0.152304609218437	3.25661024098888\\
-0.141162530661818	3.24649298597194\\
-0.13627254509018	3.24206302979891\\
-0.12363755846904	3.23046092184369\\
-0.120240480961924	3.22734795447686\\
-0.10632623939084	3.21442885771543\\
-0.104208416833667	3.21246603622831\\
-0.0892204912486374	3.19839679358717\\
-0.0881763527054109	3.19741816574723\\
-0.0723127212168553	3.18236472945892\\
-0.0721442885771544	3.18220510436568\\
-0.0561122244488979	3.16683736834846\\
-0.0555916565288081	3.16633266533066\\
-0.0400801603206413	3.15130549133715\\
-0.0390544327169323	3.1503006012024\\
-0.0240480961923848	3.13560667108172\\
-0.0226962325176485	3.13426853707415\\
-0.00801603206412826	3.11974116466135\\
-0.00651134034866899	3.11823647294589\\
0.00801603206412782	3.10370910053309\\
0.00950559946698457	3.10220440881764\\
0.0240480961923843	3.08751047869695\\
0.0253596009134551	3.08617234468938\\
0.0400801603206409	3.07114517069587\\
0.0410553532443366	3.07014028056112\\
0.0561122244488974	3.05461291945066\\
0.0565972367810957	3.05410821643287\\
0.0719910800676468	3.03807615230461\\
0.0721442885771539	3.03791652721137\\
0.087246414967761	3.02204408817635\\
0.0881763527054105	3.0210654603364\\
0.102361643310788	3.0060120240481\\
0.104208416833667	3.00404920256098\\
0.117340166624626	2.98997995991984\\
0.120240480961924	2.98686699255302\\
0.1321851375395	2.97394789579158\\
0.13627254509018	2.96951793961855\\
0.146899470581772	2.95791583166333\\
0.152304609218437	2.95200102255201\\
0.161485852207839	2.94188376753507\\
0.168336673346693	2.93431508831635\\
0.175946750125508	2.92585170340681\\
0.18436873747495	2.91645885055069\\
0.190284421946428	2.90981963927856\\
0.200400801603206	2.89843088790269\\
0.204500923209734	2.8937875751503\\
0.216432865731463	2.88022964218263\\
0.218598114813846	2.87775551102204\\
0.232464929859719	2.86185341633595\\
0.232577669890459	2.86172344689379\\
0.246472484612783	2.84569138276553\\
0.248496993987976	2.84334683013979\\
0.260257551260827	2.82965931863727\\
0.264529058116232	2.8246674435346\\
0.273932429644883	2.81362725450902\\
0.280561122244489	2.80581064483644\\
0.28749819455161	2.79759519038076\\
0.296593186372745	2.78677420172124\\
0.300955760353324	2.7815631262525\\
0.312625250501002	2.76755573294743\\
0.314305885389588	2.76553106212425\\
0.327567928056745	2.74949899799599\\
0.328657314629258	2.74817865870861\\
0.340755276724305	2.73346693386774\\
0.344689378757515	2.72865697759999\\
0.353840505499915	2.71743486973948\\
0.360721442885771	2.70894886570484\\
0.366823878678944	2.70140280561122\\
0.376753507014028	2.68905137055723\\
0.37970552252767	2.68537074148297\\
0.39249088285182	2.66933867735471\\
0.392785571142285	2.66896859196343\\
0.405231158769934	2.65330661322645\\
0.408817635270541	2.64876416643638\\
0.417873568327678	2.6372745490982\\
0.424849699398798	2.62836431212495\\
0.430417805075108	2.62124248496994\\
0.440881763527054	2.60776546556524\\
0.442863436587081	2.60521042084168\\
0.45524297062471	2.58917835671343\\
0.456913827655311	2.5870058976132\\
0.467564961912667	2.57314629258517\\
0.472945891783567	2.56609206503508\\
0.479790945645323	2.55711422845691\\
0.488977955911824	2.54497145784377\\
0.491920100199112	2.54108216432866\\
0.503972942874826	2.5250501002004\\
0.50501002004008	2.5236663879807\\
0.515990373212707	2.50901803607214\\
0.521042084168337	2.50222313773313\\
0.527912695607077	2.49298597194389\\
0.537074148296593	2.48056397742858\\
0.539738724534669	2.47695390781563\\
0.551501787790471	2.46092184368737\\
0.55310621242485	2.45872592836413\\
0.563226279677954	2.44488977955912\\
0.569138276553106	2.43673386041149\\
0.574855395806361	2.42885771543086\\
0.585170340681363	2.4145153138663\\
0.586387604466831	2.41282565130261\\
0.597895556901213	2.39679358717435\\
0.601202404809619	2.39215251770987\\
0.609334863426597	2.38076152304609\\
0.617234468937876	2.36958808160184\\
0.620677433211101	2.36472945891784\\
0.631951634761235	2.34869739478958\\
0.633266533066132	2.34682017106677\\
0.643206984131002	2.33266533066132\\
0.649298597194389	2.32390145489646\\
0.654365277727699	2.31663326653307\\
0.665330661322646	2.30073670760607\\
0.665424433461496	2.30060120240481\\
0.676498350458976	2.28456913827655\\
0.681362725450902	2.27745158915594\\
0.687476579602111	2.2685370741483\\
0.697394789579159	2.25391476913844\\
0.698354673058552	2.25250501002004\\
0.709232506939536	2.23647294589178\\
0.713426853707415	2.23023177859338\\
0.720033795682041	2.22044088176353\\
0.729458917835672	2.20631120866813\\
0.730733503473829	2.20440881763527\\
0.7414301412799	2.18837675350701\\
0.745490981963928	2.18223241149474\\
0.752056579978297	2.17234468937876\\
0.761523046092185	2.15791502453519\\
0.762579557512944	2.1563126252505\\
0.773109252646421	2.14028056112224\\
0.777555110220441	2.13344090789742\\
0.783561943357505	2.12424849699399\\
0.793587174348698	2.10871218636007\\
0.793908866757872	2.10821643286573\\
0.804285298864114	2.09218436873747\\
0.809619238476954	2.08384163914413\\
0.814564392751905	2.07615230460922\\
0.824758302252478	2.06012024048096\\
0.825651302605211	2.0587105424979\\
0.834971324123861	2.04408817635271\\
0.841683366733467	2.03341582862003\\
0.845076051383763	2.02805611222445\\
0.855137738611259	2.01202404809619\\
0.857715430861724	2.00788740685225\\
0.865178068571839	1.99599198396794\\
0.87374749498998	1.98214143281808\\
0.875106761295386	1.97995991983968\\
0.885048557799838	1.96392785571142\\
0.889779559118236	1.95621033956214\\
0.89491406096065	1.94789579158317\\
0.904694649819828	1.93186372745491\\
0.905811623246493	1.93002471349564\\
0.914498021268976	1.91583166332665\\
0.921843687374749	1.90365215248145\\
0.924185695363177	1.8997995991984\\
0.933865262670698	1.88376753507014\\
0.937875751503006	1.87705615124094\\
0.943491311068582	1.86773547094188\\
0.953022059741463	1.85170340681363\\
0.953907815631262	1.85020732994795\\
0.962587153792168	1.83567134268537\\
0.969939879759519	1.8231657629068\\
0.972031583136088	1.81963927855711\\
0.981479033512833	1.80360721442886\\
0.985971943887775	1.79589449669732\\
0.990862346907323	1.7875751503006\\
1.00017244570538	1.77154308617235\\
1.00200400801603	1.76836961128823\\
1.00949495205705	1.75551102204409\\
1.01803607214429	1.74061041363493\\
1.01869091350944	1.73947895791583\\
1.02793446764556	1.72344689378758\\
1.03406813627254	1.71264641412607\\
1.03706881500154	1.70741482965932\\
1.04618567492923	1.69138276553106\\
1.0501002004008	1.6844253916532\\
1.05525837210245	1.67535070140281\\
1.06425307777468	1.65931863727455\\
1.06613226452906	1.65594739645577\\
1.07326397108547	1.64328657314629\\
1.08214091242481	1.62725450901804\\
1.08216432865731	1.62721210560544\\
1.09108973294102	1.61122244488978\\
1.09819639278557	1.59826655016503\\
1.09990251902498	1.59519038076152\\
1.10873952224162	1.57915831663327\\
1.11422845691383	1.56905753341269\\
1.11748836383966	1.56312625250501\\
1.12621695542465	1.54709418837675\\
1.13026052104208	1.53958302139598\\
1.13490125784064	1.5310621242485\\
1.14352540851714	1.51503006012024\\
1.14629258517034	1.50984126189114\\
1.15214446671673	1.49899799599198\\
1.16066802432549	1.48296593186373\\
1.1623246492986	1.47983010308346\\
1.16922102401272	1.46693386773547\\
1.17764771911111	1.45090180360721\\
1.17835671342685	1.44954698667167\\
1.18613373755529	1.43486973947896\\
1.19438877755511	1.41899119730041\\
1.19446961549923	1.4188376753507\\
1.2028851953898	1.40280561122244\\
1.21042084168337	1.38817291322035\\
1.21115094474724	1.38677354709419\\
1.21947777124482	1.37074148296593\\
1.22645290581162	1.35706794334549\\
1.22767214575648	1.35470941883768\\
1.23591362954052	1.33867735470942\\
1.24248496993988	1.32567209532447\\
1.24403527399133	1.32264529058116\\
1.25219472995507	1.30661322645291\\
1.25851703406814	1.29398071804332\\
1.26024217985847	1.29058116232465\\
1.26832283156246	1.27454909819639\\
1.27454909819639	1.26198868874564\\
1.27629451261218	1.25851703406814\\
1.28429949655346	1.24248496993988\\
1.29058116232465	1.22969039897374\\
1.29219372384407	1.22645290581162\\
1.30012609355036	1.21042084168337\\
1.30661322645291	1.19707973928056\\
1.30794107056714	1.19438877755511\\
1.31580380052522	1.17835671342685\\
1.32264529058116	1.16415008265747\\
1.32353761790295	1.1623246492986\\
1.33133360732971	1.14629258517034\\
1.33867735470942	1.13089426661871\\
1.33898424137981	1.13026052104208\\
1.34671631784397	1.11422845691383\\
1.35429512759832	1.09819639278557\\
1.35470941883768	1.09731620374474\\
1.36195255175066	1.08216432865731\\
1.36947185626311	1.06613226452906\\
1.37074148296593	1.06340823064926\\
1.37704274593921	1.0501002004008\\
1.38450516462339	1.03406813627254\\
1.38677354709419	1.02915302097975\\
1.39198715554453	1.01803607214429\\
1.39939524815508	1.00200400801603\\
1.40280561122244	0.994540998213194\\
1.40678585462307	0.985971943887775\\
1.41414212425145	0.969939879759519\\
1.4188376753507	0.959561938282485\\
1.42143873646836	0.953907815631262\\
1.42874563242771	0.937875751503006\\
1.43486973947896	0.924204941723015\\
1.43594551356707	0.921843687374749\\
1.44320543417176	0.905811623246493\\
1.45032497946464	0.889779559118236\\
1.45090180360721	0.888474049996076\\
1.4575210124409	0.87374749498998\\
1.46459710752866	0.857715430861724\\
1.46693386773547	0.852373059629644\\
1.47169167080326	0.841683366733467\\
1.4787265569285	0.825651302605211\\
1.48296593186373	0.815860418687749\\
1.48571653222177	0.809619238476954\\
1.49271240871521	0.793587174348698\\
1.49899799599198	0.778922138284666\\
1.49959453747732	0.777555110220441\\
1.50655356408688	0.761523046092185\\
1.51338033327571	0.745490981963928\\
1.51503006012024	0.741586676058771\\
1.52024874236831	0.729458917835672\\
1.52704167434769	0.713426853707415\\
1.5310621242485	0.703813283424752\\
1.53379647862441	0.697394789579159\\
1.540557604138	0.681362725450902\\
1.54709418837675	0.665569985450078\\
1.54719512090086	0.665330661322646\\
1.55392643861466	0.649298597194389\\
1.56053156729777	0.633266533066132\\
1.56312625250501	0.626904961780657\\
1.56714630555442	0.617234468937876\\
1.57372428420662	0.601202404809619\\
1.57915831663327	0.587738734608609\\
1.58021514095198	0.585170340681363\\
1.58676785760723	0.569138276553106\\
1.59319910861719	0.55310621242485\\
1.59519038076152	0.548097964027826\\
1.59966000209789	0.537074148296593\\
1.60606840850847	0.521042084168337\\
1.61122244488978	0.507940243399239\\
1.61239823384089	0.50501002004008\\
1.61878558460074	0.488977955911824\\
1.62505577178471	0.472945891783567\\
1.62725450901804	0.467270190835045\\
1.63134792809836	0.456913827655311\\
1.63759925671641	0.440881763527054\\
1.64328657314629	0.42604105665475\\
1.64375252317298	0.424849699398798\\
1.64998669320543	0.408817635270541\\
1.65610761012657	0.392785571142285\\
1.65931863727455	0.384274059012638\\
1.66221493444902	0.376753507014028\\
1.66832068705787	0.360721442885771\\
1.67431648597408	0.344689378757515\\
1.67535070140281	0.34190392354163\\
1.68037282410443	0.328657314629258\\
1.6863553242951	0.312625250501002\\
1.69138276553106	0.298933375797022\\
1.6922604173067	0.296593186372745\\
1.69823117961289	0.280561122244489\\
1.70409529753085	0.264529058116232\\
1.70741482965932	0.255339766946777\\
1.70994020395826	0.248496993987976\\
1.71579427084906	0.232464929859719\\
1.72154459568374	0.216432865731463\\
1.72344689378758	0.211078738205828\\
1.72732382048331	0.200400801603206\\
1.73306566673049	0.18436873747495\\
1.73870651950803	0.168336673346693\\
1.73947895791583	0.166125622916457\\
1.74441441207152	0.152304609218437\\
1.75004824908652	0.13627254509018\\
1.75551102204409	0.12045347651587\\
1.75558622659312	0.120240480961924\\
1.76121443676716	0.104208416833667\\
1.76674422382966	0.0881763527054105\\
1.77154308617235	0.0740373790784134\\
1.77220020625581	0.0721442885771539\\
1.77772567498475	0.0561122244488974\\
1.78315512949002	0.0400801603206409\\
1.7875751503006	0.0268261904615026\\
1.78852289179611	0.0240480961923843\\
1.7939492363902	0.00801603206412782\\
1.79928184315911	-0.00801603206412826\\
1.80360721442886	-0.0212205331636935\\
1.80455495592437	-0.0240480961923848\\
1.80988556418485	-0.0400801603206413\\
1.81512458387508	-0.0561122244488979\\
1.81963927855711	-0.0701482349425848\\
1.82029639864058	-0.0721442885771544\\
1.82553443681551	-0.0881763527054109\\
1.83068291346776	-0.104208416833667\\
1.83567134268537	-0.120007821055388\\
1.8357465472344	-0.120240480961924\\
1.84089496711966	-0.13627254509018\\
1.84595573486811	-0.152304609218437\\
1.85093096840582	-0.168336673346694\\
1.85170340681363	-0.17084620299073\\
1.85596559890135	-0.18436873747495\\
1.86094128787276	-0.200400801603207\\
1.86583317283805	-0.216432865731463\\
1.86773547094188	-0.222734532104207\\
1.87074410091915	-0.232464929859719\\
1.875637142341	-0.248496993987976\\
1.88044800294167	-0.264529058116232\\
1.88376753507014	-0.275745818586478\\
1.88522755825375	-0.280561122244489\\
1.89004018878716	-0.296593186372745\\
1.89477215796244	-0.312625250501002\\
1.89942534149261	-0.328657314629258\\
1.8997995991984	-0.329957119838733\\
1.90414662631756	-0.344689378757515\\
1.90880164898172	-0.360721442885771\\
1.9133792480818	-0.376753507014028\\
1.91583166332665	-0.385450291371113\\
1.91795194784638	-0.392785571142285\\
1.92253178338579	-0.408817635270541\\
1.92703546142494	-0.424849699398798\\
1.93146469803591	-0.440881763527054\\
1.93186372745491	-0.44233850620119\\
1.93595714653524	-0.456913827655311\\
1.94038838530739	-0.472945891783567\\
1.94474630364003	-0.488977955911824\\
1.94789579158317	-0.500731448902317\\
1.94907158053428	-0.50501002004008\\
1.95343168096556	-0.521042084168337\\
1.95771949174352	-0.537074148296593\\
1.96193658356709	-0.55310621242485\\
1.96392785571142	-0.560770041393917\\
1.96615824208962	-0.569138276553106\\
1.97037697660293	-0.585170340681363\\
1.97452588741304	-0.601202404809619\\
1.97860645615369	-0.617234468937876\\
1.97995991983968	-0.622612915250674\\
1.98271067567622	-0.633266533066132\\
1.98679217007759	-0.649298597194389\\
1.99080608711256	-0.665330661322646\\
1.99475382388421	-0.681362725450902\\
1.99599198396794	-0.686449305090096\\
1.99872633834385	-0.697394789579158\\
2.00267410246425	-0.713426853707415\\
2.0065563245599	-0.729458917835671\\
2.01037432125166	-0.745490981963928\\
2.01202404809619	-0.752506904049819\\
2.0142003569969	-0.761523046092184\\
2.01801729618351	-0.777555110220441\\
2.02177052494767	-0.793587174348697\\
2.02546128301537	-0.809619238476954\\
2.02805611222445	-0.821059498204689\\
2.02912549303033	-0.82565130260521\\
2.03281391529224	-0.841683366733467\\
2.03644026093137	-0.857715430861723\\
2.04000569592562	-0.87374749498998\\
2.04351135221013	-0.889779559118236\\
2.04408817635271	-0.892448916559835\\
2.04705370876431	-0.905811623246493\\
2.05055469160957	-0.92184368737475\\
2.05399613342972	-0.937875751503006\\
2.05737909650658	-0.953907815631263\\
2.06012024048096	-0.96710952209865\\
2.06072393619058	-0.969939879759519\\
2.06410048388159	-0.985971943887776\\
2.06741867270145	-1.00200400801603\\
2.07067949744408	-1.01803607214429\\
2.07388392213842	-1.03406813627255\\
2.07615230460922	-1.04560043723216\\
2.0770617153881	-1.0501002004008\\
2.08025679315444	-1.06613226452906\\
2.0833954375222	-1.08216432865731\\
2.08647854783053	-1.09819639278557\\
2.08950699428547	-1.11422845691383\\
2.09218436873747	-1.12865183648668\\
2.09249125540787	-1.13026052104208\\
2.09550706610889	-1.14629258517034\\
2.09846802604284	-1.1623246492986\\
2.10137494280979	-1.17835671342685\\
2.10422859624857	-1.19438877755511\\
2.10702973894416	-1.21042084168337\\
2.10821643286573	-1.21732530834615\\
2.10982899438516	-1.22645290581162\\
2.11261272985613	-1.24248496993988\\
2.11534357785558	-1.25851703406814\\
2.11802223036006	-1.27454909819639\\
2.12064935314585	-1.29058116232465\\
2.1232255862079	-1.30661322645291\\
2.12424849699399	-1.31309304891348\\
2.12579880104544	-1.32264529058116\\
2.12835227738831	-1.33867735470942\\
2.13085429246183	-1.35470941883768\\
2.13330542655544	-1.37074148296593\\
2.13570623492495	-1.38677354709419\\
2.13805724812759	-1.40280561122244\\
2.14028056112224	-1.41829016635703\\
2.14036139906636	-1.4188376753507\\
2.14268365816133	-1.43486973947896\\
2.14495534686681	-1.45090180360721\\
2.14717693583636	-1.46693386773547\\
2.14934887147192	-1.48296593186373\\
2.15147157617786	-1.49899799599198\\
2.1535454485973	-1.51503006012024\\
2.15557086383083	-1.5310621242485\\
2.1563126252505	-1.53707027093956\\
2.15758539475005	-1.54709418837675\\
2.15957253217634	-1.56312625250501\\
2.16151014528104	-1.57915831663327\\
2.1633985476971	-1.59519038076152\\
2.1652380295342	-1.61122244488978\\
2.16702885751804	-1.62725450901804\\
2.16877127511243	-1.64328657314629\\
2.17046550262438	-1.65931863727455\\
2.17211173729216	-1.67535070140281\\
2.17234468937876	-1.67768616266539\\
2.17374993028851	-1.69138276553106\\
2.17534536810776	-1.70741482965932\\
2.17689138091722	-1.72344689378758\\
2.17838810008491	-1.73947895791583\\
2.17983563341978	-1.75551102204409\\
2.18123406517753	-1.77154308617234\\
2.18258345604918	-1.7875751503006\\
2.18388384313207	-1.80360721442886\\
2.1851352398834	-1.81963927855711\\
2.18633763605613	-1.83567134268537\\
2.18749099761721	-1.85170340681363\\
2.18837675350701	-1.8645652151021\\
2.18860133989036	-1.86773547094188\\
2.18968554512439	-1.88376753507014\\
2.19071876149544	-1.8997995991984\\
2.19170087930555	-1.91583166332665\\
2.19263176387134	-1.93186372745491\\
2.19351125534943	-1.94789579158317\\
2.19433916854262	-1.96392785571142\\
2.19511529268665	-1.97995991983968\\
2.19583939121713	-1.99599198396794\\
2.19651120151655	-2.01202404809619\\
2.19713043464091	-2.02805611222445\\
2.19769677502566	-2.04408817635271\\
2.19820988017068	-2.06012024048096\\
2.19866938030378	-2.07615230460922\\
2.19907487802243	-2.09218436873747\\
2.19942594791327	-2.10821643286573\\
2.19972213614885	-2.12424849699399\\
2.19996296006125	-2.14028056112224\\
2.20014790769201	-2.1563126252505\\
2.20027643731779	-2.17234468937876\\
2.20034797695124	-2.18837675350701\\
2.20036192381657	-2.20440881763527\\
2.20031764379895	-2.22044088176353\\
2.20021447086739	-2.23647294589178\\
2.20005170647016	-2.25250501002004\\
2.19982861890213	-2.2685370741483\\
2.19954444264335	-2.28456913827655\\
2.19919837766783	-2.30060120240481\\
2.19878958872204	-2.31663326653307\\
2.19831720457188	-2.33266533066132\\
2.19778031721757	-2.34869739478958\\
2.19717798107516	-2.36472945891784\\
2.19650921212399	-2.38076152304609\\
2.19577298701883	-2.39679358717435\\
2.1949682421657	-2.41282565130261\\
2.19409387276027	-2.42885771543086\\
2.19314873178755	-2.44488977955912\\
2.19213162898178	-2.46092184368737\\
2.19104132974509	-2.47695390781563\\
2.1898765540237	-2.49298597194389\\
2.18863597514014	-2.50901803607214\\
2.18837675350701	-2.51218829191193\\
2.18733967634176	-2.5250501002004\\
2.18597120113148	-2.54108216432866\\
2.18452376321569	-2.55711422845691\\
2.18299582363611	-2.57314629258517\\
2.18138579001696	-2.58917835671343\\
2.179692015072	-2.60521042084168\\
2.17791279505507	-2.62124248496994\\
2.17604636815184	-2.6372745490982\\
2.17409091281085	-2.65330661322645\\
2.17234468937876	-2.66700321609212\\
2.17205000108829	-2.66933867735471\\
2.169949223652	-2.68537074148297\\
2.16775450582966	-2.70140280561122\\
2.1654637519929	-2.71743486973948\\
2.1630747968375	-2.73346693386774\\
2.16058540325749	-2.74949899799599\\
2.15799326013909	-2.76553106212425\\
2.1563126252505	-2.77555497956145\\
2.15531263058558	-2.7815631262525\\
2.1525529883469	-2.79759519038076\\
2.14968393265089	-2.81362725450902\\
2.14670280909838	-2.82965931863727\\
2.1436068759187	-2.84569138276553\\
2.14039330115298	-2.86172344689379\\
2.14028056112224	-2.86227095588746\\
2.13710656700386	-2.87775551102204\\
2.13369847096792	-2.8937875751503\\
2.13016418146547	-2.90981963927856\\
2.12650044307887	-2.92585170340681\\
2.12424849699399	-2.9354039450745\\
2.12272457763207	-2.94188376753507\\
2.11884335835732	-2.95791583166333\\
2.11482256942741	-2.97394789579158\\
2.11065846187091	-2.98997995991984\\
2.10821643286573	-2.99910755738531\\
2.10636965934285	-3.0060120240481\\
2.10195962779741	-3.02204408817635\\
2.09739431105722	-3.03807615230461\\
2.09266938106967	-3.05410821643287\\
2.09218436873747	-3.05571690098827\\
2.08782724256107	-3.07014028056112\\
2.08282144178327	-3.08617234468938\\
2.07764187201208	-3.10220440881764\\
2.07615230460922	-3.10670417198628\\
2.0723199728081	-3.11823647294589\\
2.0668279322443	-3.13426853707415\\
2.06114596808467	-3.1503006012024\\
2.06012024048096	-3.15313095886327\\
2.05530808263616	-3.16633266533066\\
2.04927652476668	-3.18236472945892\\
2.04408817635271	-3.19572743614558\\
2.04304403780948	-3.19839679358717\\
2.03663353865448	-3.21442885771543\\
2.03000039851701	-3.23046092184369\\
2.02805611222445	-3.23505272624421\\
2.02316612665281	-3.24649298597194\\
2.01610504929561	-3.2625250501002\\
2.01202404809619	-3.27154119214257\\
2.00881293353938	-3.27855711422846\\
2.00128462031003	-3.29458917835671\\
1.99599198396794	-3.30553466284577\\
1.99350063178841	-3.31062124248497\\
1.98546057570419	-3.32665330661323\\
1.97995991983968	-3.33730692442868\\
1.97714210753077	-3.34268537074148\\
1.96853981202308	-3.35871743486974\\
1.96392785571142	-3.36708567002893\\
1.95963426252646	-3.374749498998\\
1.9504123103506	-3.39078156312625\\
1.94789579158317	-3.39506013426401\\
1.94085534303881	-3.40681362725451\\
1.93186372745491	-3.42138894870863\\
1.93094713526962	-3.42284569138277\\
1.92066168164344	-3.43887775551102\\
1.91583166332665	-3.44621303528219\\
1.90997946975807	-3.45490981963928\\
1.8997995991984	-3.46964207855806\\
1.89888029101296	-3.47094188376753\\
1.88732105838211	-3.48697394789579\\
1.88376753507014	-3.4917892515538\\
1.87527724755564	-3.50300601202405\\
1.86773547094188	-3.51273640977956\\
1.86271726148988	-3.5190380761523\\
1.85170340681363	-3.53256061063652\\
1.84959972635276	-3.53507014028056\\
1.8358740682194	-3.55110220440882\\
1.83567134268537	-3.55133486431535\\
1.82146947117734	-3.56713426853707\\
1.81963927855711	-3.56913032217164\\
1.80633755317203	-3.58316633266533\\
1.80360721442886	-3.58599389569402\\
1.79040271332929	-3.59919839679359\\
1.7875751503006	-3.6019764910627\\
1.77357601502004	-3.61523046092184\\
1.77154308617235	-3.6171235514231\\
1.75575261167513	-3.6312625250501\\
1.75551102204409	-3.63147552060405\\
1.73947895791583	-3.64508353874812\\
1.73675998401568	-3.64729458917836\\
1.72344689378758	-3.65797252578098\\
1.71645946836328	-3.66332665330661\\
1.70741482965932	-3.67016942626541\\
1.69466730704812	-3.67935871743487\\
1.69138276553106	-3.68169890685915\\
1.67535070140281	-3.69260532634724\\
1.67104927101087	-3.69539078156313\\
1.65931863727455	-3.70291133356174\\
1.64527066679244	-3.71142284569138\\
1.64328657314629	-3.71261420294734\\
1.62725450901804	-3.72177920887112\\
1.61670775334827	-3.72745490981964\\
1.61122244488978	-3.7303851331788\\
1.59519038076152	-3.73847872555087\\
1.5846136617279	-3.7434869739479\\
1.57915831663327	-3.74605536787514\\
1.56312625250501	-3.75315746679068\\
1.54767777357342	-3.75951903807615\\
1.54709418837675	-3.75975836220358\\
1.5310621242485	-3.76593753192175\\
1.51503006012024	-3.77164679629925\\
1.50316156487214	-3.77555110220441\\
1.49899799599198	-3.77691813026864\\
1.48296593186373	-3.7817922824152\\
1.46693386773547	-3.78624079510059\\
1.45090180360721	-3.7902776572105\\
1.44519057425423	-3.79158316633267\\
1.43486973947896	-3.79394442068093\\
1.4188376753507	-3.79723728898389\\
1.40280561122244	-3.80015222065808\\
1.38677354709419	-3.80270011516812\\
1.37074148296593	-3.80489119658113\\
1.35470941883768	-3.80673504142009\\
1.34536965782306	-3.80761523046092\\
1.33867735470942	-3.80824897603755\\
1.32264529058116	-3.80944066381979\\
1.30661322645291	-3.81030619218637\\
1.29058116232465	-3.81085272362304\\
1.27454909819639	-3.81108688513842\\
1.25851703406814	-3.81101478617959\\
1.24248496993988	-3.81064203520423\\
1.22645290581162	-3.80997375496874\\
1.21042084168337	-3.80901459658709\\
1.19438877755511	-3.80776875241063\\
1.19278466694236	-3.80761523046092\\
1.17835671342685	-3.80626041352538\\
1.1623246492986	-3.80447940168065\\
1.14629258517034	-3.80242643223183\\
1.13026052104208	-3.80010406348015\\
1.11422845691383	-3.79751444724034\\
1.09819639278557	-3.79465933573617\\
1.08238610784513	-3.79158316633267\\
1.08216432865731	-3.79154076292006\\
1.06613226452906	-3.78821192551389\\
1.0501002004008	-3.7846257924548\\
1.03406813627254	-3.78078268667116\\
1.01803607214429	-3.77668255792351\\
1.01387250326413	-3.77555110220441\\
1.00200400801603	-3.7723776273203\\
0.985971943887775	-3.76783838447288\\
0.969939879759519	-3.76304552242583\\
0.958735177018894	-3.75951903807615\\
0.953907815631262	-3.75802296121048\\
0.937875751503006	-3.75280765424696\\
0.921843687374749	-3.74733952723095\\
0.911043279321471	-3.7434869739479\\
0.905811623246493	-3.74164795998862\\
0.889779559118236	-3.73576945779861\\
0.87374749498998	-3.72963642279804\\
0.868262186531492	-3.72745490981964\\
0.857715430861724	-3.7233182685757\\
0.841683366733467	-3.71678256208697\\
0.829025370107676	-3.71142284569138\\
0.825651302605211	-3.71001314770832\\
0.809619238476954	-3.70308011609804\\
0.793587174348698	-3.69588653505746\\
0.792516323769337	-3.69539078156313\\
0.777555110220441	-3.68855112833831\\
0.761523046092185	-3.68096111671956\\
0.75823850457513	-3.67935871743487\\
0.745490981963928	-3.6732143754226\\
0.729458917835672	-3.66522904433948\\
0.725746813913297	-3.66332665330661\\
0.713426853707415	-3.65708548600821\\
0.697394789579159	-3.64870434829676\\
0.694768682811162	-3.64729458917836\\
0.681362725450902	-3.64017704005774\\
0.665330661322646	-3.63139803025136\\
0.665089071691608	-3.6312625250501\\
0.649298597194389	-3.62249864928523\\
0.636584308250443	-3.61523046092184\\
0.633266533066132	-3.61335323719904\\
0.617234468937876	-3.60405701947759\\
0.609073991017455	-3.59919839679359\\
0.601202404809619	-3.59455732732911\\
0.585170340681363	-3.58485599522903\\
0.582440001938186	-3.58316633266533\\
0.569138276553106	-3.57501041351771\\
0.556626181278274	-3.56713426853707\\
0.55310621242485	-3.56493835321383\\
0.537074148296593	-3.55471227402176\\
0.531542492394102	-3.55110220440882\\
0.521042084168337	-3.5443073060698\\
0.507113700500945	-3.53507014028056\\
0.50501002004008	-3.53368642806086\\
0.488977955911824	-3.52292736966742\\
0.483308498413443	-3.5190380761523\\
0.472945891783567	-3.51198384860222\\
0.460056211284139	-3.50300601202405\\
0.456913827655311	-3.50083355292383\\
0.440881763527054	-3.48952899261935\\
0.437328240215089	-3.48697394789579\\
0.424849699398798	-3.47806371092255\\
0.415086534647535	-3.47094188376754\\
0.408817635270541	-3.46639943697746\\
0.393288980010621	-3.45490981963928\\
0.392785571142285	-3.454539734248\\
0.376753507014028	-3.44255838458529\\
0.371923488697241	-3.43887775551102\\
0.360721442885771	-3.43039175147638\\
0.350951914584676	-3.42284569138276\\
0.344689378757515	-3.41803573511502\\
0.330353172608618	-3.40681362725451\\
0.328657314629258	-3.40549328796713\\
0.312625250501002	-3.39280623394943\\
0.310108731733566	-3.39078156312625\\
0.296593186372745	-3.37996057446673\\
0.29019815203592	-3.374749498998\\
0.280561122244489	-3.36693288932542\\
0.270605055257431	-3.35871743486974\\
0.264529058116232	-3.35372555976707\\
0.251314806296885	-3.34268537074148\\
0.248496993987976	-3.34034081811575\\
0.232464929859719	-3.32678327605539\\
0.232313243736369	-3.32665330661323\\
0.216432865731463	-3.31309537364555\\
0.213579442999435	-3.31062124248497\\
0.200400801603206	-3.2992324911091\\
0.195108165261117	-3.29458917835671\\
0.18436873747495	-3.28519632550059\\
0.176888389625251	-3.27855711422846\\
0.168336673346693	-3.27098843500974\\
0.158909757082701	-3.2625250501002\\
0.152304609218437	-3.25661024098888\\
0.141162530661818	-3.24649298597194\\
0.13627254509018	-3.24206302979891\\
0.123637558469039	-3.23046092184369\\
0.120240480961924	-3.22734795447686\\
0.10632623939084	-3.21442885771543\\
0.104208416833667	-3.21246603622831\\
0.0892204912486378	-3.19839679358717\\
0.0881763527054105	-3.19741816574722\\
0.0723127212168553	-3.18236472945892\\
0.0721442885771539	-3.18220510436568\\
0.0561122244488974	-3.16683736834846\\
0.055591656528809	-3.16633266533066\\
0.0400801603206409	-3.15130549133715\\
0.0390544327169332	-3.1503006012024\\
0.0240480961923843	-3.13560667108172\\
0.0226962325176485	-3.13426853707415\\
0.00801603206412782	-3.11974116466135\\
0.00651134034866946	-3.11823647294589\\
-0.00801603206412826	-3.10370910053309\\
-0.00950559946698502	-3.10220440881764\\
-0.0240480961923848	-3.08751047869695\\
-0.0253596009134542	-3.08617234468938\\
-0.0400801603206413	-3.07114517069587\\
-0.0410553532443371	-3.07014028056112\\
-0.0561122244488979	-3.05461291945066\\
-0.0565972367810961	-3.05410821643287\\
-0.0719910800676459	-3.03807615230461\\
-0.0721442885771544	-3.03791652721137\\
-0.087246414967761	-3.02204408817635\\
-0.0881763527054109	-3.0210654603364\\
-0.102361643310788	-3.0060120240481\\
-0.104208416833667	-3.00404920256098\\
-0.117340166624625	-2.98997995991984\\
-0.120240480961924	-2.98686699255302\\
-0.1321851375395	-2.97394789579158\\
-0.13627254509018	-2.96951793961855\\
-0.146899470581772	-2.95791583166333\\
-0.152304609218437	-2.95200102255201\\
-0.161485852207838	-2.94188376753507\\
-0.168336673346694	-2.93431508831635\\
-0.175946750125508	-2.92585170340681\\
-0.18436873747495	-2.91645885055069\\
-0.190284421946428	-2.90981963927856\\
-0.200400801603207	-2.89843088790269\\
-0.204500923209734	-2.8937875751503\\
-0.216432865731463	-2.88022964218263\\
-0.218598114813846	-2.87775551102204\\
-0.232464929859719	-2.86185341633595\\
-0.232577669890459	-2.86172344689379\\
-0.246472484612783	-2.84569138276553\\
-0.248496993987976	-2.84334683013979\\
-0.260257551260827	-2.82965931863727\\
-0.264529058116232	-2.8246674435346\\
-0.273932429644883	-2.81362725450902\\
-0.280561122244489	-2.80581064483644\\
-0.28749819455161	-2.79759519038076\\
-0.296593186372745	-2.78677420172123\\
-0.300955760353324	-2.7815631262525\\
-0.312625250501002	-2.76755573294743\\
-0.314305885389588	-2.76553106212425\\
-0.327567928056745	-2.74949899799599\\
-0.328657314629258	-2.74817865870861\\
-0.340755276724304	-2.73346693386774\\
-0.344689378757515	-2.72865697759999\\
-0.353840505499914	-2.71743486973948\\
-0.360721442885771	-2.70894886570484\\
-0.366823878678944	-2.70140280561122\\
-0.376753507014028	-2.68905137055723\\
-0.379705522527669	-2.68537074148297\\
-0.39249088285182	-2.66933867735471\\
-0.392785571142285	-2.66896859196343\\
-0.405231158769934	-2.65330661322645\\
-0.408817635270541	-2.64876416643638\\
-0.417873568327679	-2.6372745490982\\
-0.424849699398798	-2.62836431212495\\
-0.430417805075109	-2.62124248496994\\
-0.440881763527054	-2.60776546556524\\
-0.442863436587081	-2.60521042084168\\
-0.455242970624711	-2.58917835671343\\
-0.456913827655311	-2.5870058976132\\
-0.467564961912668	-2.57314629258517\\
-0.472945891783567	-2.56609206503508\\
-0.479790945645323	-2.55711422845691\\
-0.488977955911824	-2.54497145784377\\
-0.491920100199112	-2.54108216432866\\
-0.503972942874826	-2.5250501002004\\
-0.50501002004008	-2.5236663879807\\
-0.515990373212706	-2.50901803607214\\
-0.521042084168337	-2.50222313773313\\
-0.527912695607077	-2.49298597194389\\
-0.537074148296593	-2.48056397742858\\
-0.539738724534669	-2.47695390781563\\
-0.551501787790471	-2.46092184368737\\
-0.55310621242485	-2.45872592836413\\
-0.563226279677955	-2.44488977955912\\
-0.569138276553106	-2.43673386041149\\
-0.574855395806361	-2.42885771543086\\
-0.585170340681363	-2.4145153138663\\
-0.586387604466832	-2.41282565130261\\
-0.597895556901214	-2.39679358717435\\
-0.601202404809619	-2.39215251770987\\
-0.609334863426596	-2.38076152304609\\
-0.617234468937876	-2.36958808160184\\
-0.620677433211101	-2.36472945891784\\
-0.631951634761235	-2.34869739478958\\
-0.633266533066132	-2.34682017106677\\
-0.643206984131002	-2.33266533066132\\
-0.649298597194389	-2.32390145489646\\
-0.654365277727698	-2.31663326653307\\
-0.665330661322646	-2.30073670760607\\
-0.665424433461496	-2.30060120240481\\
-0.676498350458976	-2.28456913827655\\
-0.681362725450902	-2.27745158915594\\
-0.687476579602111	-2.2685370741483\\
-0.697394789579158	-2.25391476913844\\
-0.698354673058551	-2.25250501002004\\
-0.709232506939537	-2.23647294589178\\
-0.713426853707415	-2.23023177859338\\
-0.720033795682041	-2.22044088176353\\
-0.729458917835671	-2.20631120866813\\
-0.730733503473829	-2.20440881763527\\
-0.741430141279901	-2.18837675350701\\
-0.745490981963928	-2.18223241149475\\
-0.752056579978297	-2.17234468937876\\
-0.761523046092184	-2.15791502453519\\
-0.762579557512944	-2.1563126252505\\
-0.773109252646421	-2.14028056112224\\
-0.777555110220441	-2.13344090789742\\
-0.783561943357505	-2.12424849699399\\
-0.793587174348697	-2.10871218636007\\
-0.793908866757872	-2.10821643286573\\
-0.804285298864114	-2.09218436873747\\
-0.809619238476954	-2.08384163914414\\
-0.814564392751905	-2.07615230460922\\
-0.824758302252477	-2.06012024048096\\
-0.82565130260521	-2.0587105424979\\
-0.834971324123861	-2.04408817635271\\
-0.841683366733467	-2.03341582862003\\
-0.845076051383762	-2.02805611222445\\
-0.855137738611258	-2.01202404809619\\
-0.857715430861723	-2.00788740685225\\
-0.865178068571839	-1.99599198396794\\
-0.87374749498998	-1.98214143281808\\
-0.875106761295386	-1.97995991983968\\
-0.885048557799838	-1.96392785571142\\
-0.889779559118236	-1.95621033956214\\
-0.89491406096065	-1.94789579158317\\
-0.904694649819828	-1.93186372745491\\
-0.905811623246493	-1.93002471349564\\
-0.914498021268976	-1.91583166332665\\
-0.92184368737475	-1.90365215248145\\
-0.924185695363177	-1.8997995991984\\
-0.933865262670698	-1.88376753507014\\
-0.937875751503006	-1.87705615124094\\
-0.943491311068581	-1.86773547094188\\
-0.953022059741462	-1.85170340681363\\
-0.953907815631263	-1.85020732994795\\
-0.962587153792167	-1.83567134268537\\
-0.969939879759519	-1.8231657629068\\
-0.972031583136088	-1.81963927855711\\
-0.981479033512832	-1.80360721442886\\
-0.985971943887776	-1.79589449669732\\
-0.990862346907323	-1.7875751503006\\
-1.00017244570538	-1.77154308617234\\
-1.00200400801603	-1.76836961128823\\
-1.00949495205705	-1.75551102204409\\
-1.01803607214429	-1.74061041363493\\
-1.01869091350944	-1.73947895791583\\
-1.02793446764556	-1.72344689378758\\
-1.03406813627255	-1.71264641412607\\
-1.03706881500154	-1.70741482965932\\
-1.04618567492923	-1.69138276553106\\
-1.0501002004008	-1.68442539165319\\
-1.05525837210245	-1.67535070140281\\
-1.06425307777468	-1.65931863727455\\
-1.06613226452906	-1.65594739645577\\
-1.07326397108547	-1.64328657314629\\
-1.08214091242481	-1.62725450901804\\
-1.08216432865731	-1.62721210560543\\
-1.09108973294102	-1.61122244488978\\
-1.09819639278557	-1.59826655016503\\
-1.09990251902498	-1.59519038076152\\
-1.10873952224162	-1.57915831663327\\
-1.11422845691383	-1.56905753341268\\
-1.11748836383966	-1.56312625250501\\
-1.12621695542465	-1.54709418837675\\
-1.13026052104208	-1.53958302139598\\
-1.13490125784064	-1.5310621242485\\
-1.14352540851714	-1.51503006012024\\
-1.14629258517034	-1.50984126189114\\
-1.15214446671673	-1.49899799599198\\
-1.16066802432549	-1.48296593186373\\
-1.1623246492986	-1.47983010308346\\
-1.16922102401271	-1.46693386773547\\
-1.17764771911111	-1.45090180360721\\
-1.17835671342685	-1.44954698667167\\
-1.18613373755529	-1.43486973947896\\
-1.19438877755511	-1.41899119730041\\
-1.19446961549923	-1.4188376753507\\
-1.2028851953898	-1.40280561122244\\
-1.21042084168337	-1.38817291322035\\
-1.21115094474724	-1.38677354709419\\
-1.21947777124482	-1.37074148296593\\
-1.22645290581162	-1.35706794334549\\
-1.22767214575648	-1.35470941883768\\
-1.23591362954052	-1.33867735470942\\
-1.24248496993988	-1.32567209532447\\
-1.24403527399133	-1.32264529058116\\
-1.25219472995507	-1.30661322645291\\
-1.25851703406814	-1.29398071804332\\
-1.26024217985847	-1.29058116232465\\
-1.26832283156246	-1.27454909819639\\
-1.27454909819639	-1.26198868874564\\
-1.27629451261218	-1.25851703406814\\
-1.28429949655346	-1.24248496993988\\
-1.29058116232465	-1.22969039897374\\
-1.29219372384407	-1.22645290581162\\
-1.30012609355036	-1.21042084168337\\
-1.30661322645291	-1.19707973928056\\
-1.30794107056714	-1.19438877755511\\
-1.31580380052522	-1.17835671342685\\
-1.32264529058116	-1.16415008265747\\
-1.32353761790295	-1.1623246492986\\
-1.33133360732971	-1.14629258517034\\
-1.33867735470942	-1.13089426661871\\
-1.33898424137981	-1.13026052104208\\
-1.34671631784397	-1.11422845691383\\
-1.35429512759832	-1.09819639278557\\
-1.35470941883768	-1.09731620374474\\
-1.36195255175066	-1.08216432865731\\
-1.36947185626311	-1.06613226452906\\
-1.37074148296593	-1.06340823064926\\
-1.37704274593921	-1.0501002004008\\
-1.38450516462339	-1.03406813627255\\
-1.38677354709419	-1.02915302097975\\
-1.39198715554453	-1.01803607214429\\
-1.39939524815508	-1.00200400801603\\
-1.40280561122244	-0.994540998213195\\
-1.40678585462307	-0.985971943887776\\
-1.41414212425145	-0.969939879759519\\
-1.4188376753507	-0.959561938282485\\
-1.42143873646836	-0.953907815631263\\
-1.42874563242771	-0.937875751503006\\
-1.43486973947896	-0.924204941723015\\
-1.43594551356707	-0.92184368737475\\
-1.44320543417176	-0.905811623246493\\
-1.45032497946464	-0.889779559118236\\
-1.45090180360721	-0.888474049996076\\
-1.4575210124409	-0.87374749498998\\
-1.46459710752866	-0.857715430861723\\
-1.46693386773547	-0.852373059629644\\
-1.47169167080326	-0.841683366733467\\
-1.4787265569285	-0.82565130260521\\
-1.48296593186373	-0.815860418687748\\
-1.48571653222177	-0.809619238476954\\
-1.49271240871521	-0.793587174348697\\
-1.49899799599198	-0.778922138284666\\
-1.49959453747732	-0.777555110220441\\
-1.50655356408688	-0.761523046092184\\
-1.51338033327571	-0.745490981963928\\
-1.51503006012024	-0.74158667605877\\
-1.52024874236831	-0.729458917835671\\
-1.52704167434769	-0.713426853707415\\
-1.5310621242485	-0.703813283424752\\
-1.53379647862441	-0.697394789579158\\
-1.54055760413801	-0.681362725450902\\
-1.54709418837675	-0.665569985450078\\
-1.54719512090086	-0.665330661322646\\
-1.55392643861466	-0.649298597194389\\
-1.56053156729777	-0.633266533066132\\
-1.56312625250501	-0.626904961780658\\
-1.56714630555442	-0.617234468937876\\
-1.57372428420663	-0.601202404809619\\
-1.57915831663327	-0.587738734608611\\
-1.58021514095198	-0.585170340681363\\
-1.58676785760723	-0.569138276553106\\
-1.59319910861719	-0.55310621242485\\
-1.59519038076152	-0.548097964027827\\
-1.59966000209789	-0.537074148296593\\
-1.60606840850847	-0.521042084168337\\
-1.61122244488978	-0.50794024339924\\
-1.61239823384089	-0.50501002004008\\
-1.61878558460074	-0.488977955911824\\
-1.62505577178471	-0.472945891783567\\
-1.62725450901804	-0.467270190835046\\
-1.63134792809836	-0.456913827655311\\
-1.63759925671641	-0.440881763527054\\
-1.64328657314629	-0.426041056654751\\
-1.64375252317298	-0.424849699398798\\
-1.64998669320543	-0.408817635270541\\
-1.65610761012657	-0.392785571142285\\
-1.65931863727455	-0.38427405901264\\
-1.66221493444902	-0.376753507014028\\
-1.66832068705787	-0.360721442885771\\
-1.67431648597408	-0.344689378757515\\
-1.67535070140281	-0.341903923541631\\
-1.68037282410443	-0.328657314629258\\
-1.6863553242951	-0.312625250501002\\
-1.69138276553106	-0.298933375797024\\
-1.69226041730671	-0.296593186372745\\
-1.69823117961289	-0.280561122244489\\
-1.70409529753085	-0.264529058116232\\
-1.70741482965932	-0.255339766946777\\
-1.70994020395826	-0.248496993987976\\
-1.71579427084906	-0.232464929859719\\
-1.72154459568374	-0.216432865731463\\
-1.72344689378758	-0.211078738205828\\
-1.72732382048331	-0.200400801603207\\
-1.73306566673049	-0.18436873747495\\
-1.73870651950802	-0.168336673346694\\
-1.73947895791583	-0.166125622916459\\
-1.74441441207152	-0.152304609218437\\
-1.75004824908651	-0.13627254509018\\
-1.75551102204409	-0.120453476515871\\
-1.75558622659312	-0.120240480961924\\
-1.76121443676716	-0.104208416833667\\
-1.76674422382965	-0.0881763527054109\\
-1.77154308617234	-0.0740373790784148\\
-1.77220020625581	-0.0721442885771544\\
-1.77772567498475	-0.0561122244488979\\
-1.78315512949002	-0.0400801603206413\\
-1.7875751503006	-0.0268261904615012\\
-1.78852289179611	-0.0240480961923848\\
-1.7939492363902	-0.00801603206412826\\
-1.79928184315911	0.00801603206412782\\
-1.80360721442886	0.0212205331636953\\
-1.80455495592437	0.0240480961923843\\
-1.80988556418485	0.0400801603206409\\
-1.81512458387508	0.0561122244488974\\
-1.81963927855711	0.0701482349425853\\
-1.82029639864058	0.0721442885771539\\
-1.82553443681551	0.0881763527054105\\
-1.83068291346776	0.104208416833667\\
-1.83567134268537	0.120007821055389\\
-1.8357465472344	0.120240480961924\\
-1.84089496711966	0.13627254509018\\
-1.84595573486811	0.152304609218437\\
-1.85093096840582	0.168336673346693\\
-1.85170340681363	0.170846202990731\\
-1.85596559890135	0.18436873747495\\
-1.86094128787276	0.200400801603206\\
-1.86583317283805	0.216432865731463\\
-1.86773547094188	0.222734532104208\\
-1.87074410091915	0.232464929859719\\
-1.875637142341	0.248496993987976\\
-1.88044800294167	0.264529058116232\\
-1.88376753507014	0.27574581858648\\
-1.88522755825375	0.280561122244489\\
-1.89004018878716	0.296593186372745\\
-1.89477215796244	0.312625250501002\\
-1.89942534149261	0.328657314629258\\
-1.8997995991984	0.329957119838733\\
-1.90414662631756	0.344689378757515\\
-1.90880164898172	0.360721442885771\\
-1.9133792480818	0.376753507014028\\
-1.91583166332665	0.385450291371114\\
-1.91795194784638	0.392785571142285\\
-1.92253178338579	0.408817635270541\\
-1.92703546142494	0.424849699398798\\
-1.93146469803591	0.440881763527054\\
-1.93186372745491	0.442338506201192\\
-1.93595714653524	0.456913827655311\\
-1.94038838530739	0.472945891783567\\
-1.94474630364003	0.488977955911824\\
-1.94789579158317	0.500731448902319\\
-1.94907158053428	0.50501002004008\\
-1.95343168096556	0.521042084168337\\
-1.95771949174352	0.537074148296593\\
-1.96193658356709	0.55310621242485\\
-1.96392785571142	0.560770041393919\\
-1.96615824208962	0.569138276553106\\
-1.97037697660293	0.585170340681363\\
-1.97452588741304	0.601202404809619\\
-1.97860645615369	0.617234468937876\\
-1.97995991983968	0.622612915250676\\
-1.98271067567622	0.633266533066132\\
-1.98679217007759	0.649298597194389\\
-1.99080608711256	0.665330661322646\\
-1.99475382388421	0.681362725450902\\
-1.99599198396794	0.686449305090098\\
-1.99872633834385	0.697394789579159\\
-2.00267410246425	0.713426853707415\\
-2.0065563245599	0.729458917835672\\
-2.01037432125166	0.745490981963928\\
-2.01202404809619	0.752506904049819\\
-2.0142003569969	0.761523046092185\\
-2.01801729618351	0.777555110220441\\
-2.02177052494767	0.793587174348698\\
-2.02546128301537	0.809619238476954\\
-2.02805611222445	0.821059498204689\\
-2.02912549303033	0.825651302605211\\
-2.03281391529224	0.841683366733467\\
-2.03644026093137	0.857715430861724\\
-2.04000569592562	0.87374749498998\\
-2.04351135221013	0.889779559118236\\
-2.04408817635271	0.892448916559834\\
-2.04705370876431	0.905811623246493\\
-2.05055469160957	0.921843687374749\\
-2.05399613342972	0.937875751503006\\
-2.05737909650658	0.953907815631262\\
-2.06012024048096	0.96710952209865\\
-2.06072393619058	0.969939879759519\\
-2.06410048388159	0.985971943887775\\
-2.06741867270145	1.00200400801603\\
-2.07067949744408	1.01803607214429\\
-2.07388392213842	1.03406813627254\\
-2.07615230460922	1.04560043723216\\
-2.0770617153881	1.0501002004008\\
-2.08025679315444	1.06613226452906\\
-2.0833954375222	1.08216432865731\\
-2.08647854783053	1.09819639278557\\
-2.08950699428547	1.11422845691383\\
-2.09218436873747	1.12865183648668\\
-2.09249125540787	1.13026052104208\\
-2.09550706610889	1.14629258517034\\
-2.09846802604284	1.1623246492986\\
-2.10137494280979	1.17835671342685\\
-2.10422859624857	1.19438877755511\\
-2.10702973894416	1.21042084168337\\
-2.10821643286573	1.21732530834615\\
-2.10982899438516	1.22645290581162\\
-2.11261272985613	1.24248496993988\\
-2.11534357785558	1.25851703406814\\
-2.11802223036006	1.27454909819639\\
-2.12064935314585	1.29058116232465\\
-2.1232255862079	1.30661322645291\\
-2.12424849699399	1.31309304891348\\
-2.12579880104544	1.32264529058116\\
-2.12835227738831	1.33867735470942\\
-2.13085429246183	1.35470941883768\\
-2.13330542655544	1.37074148296593\\
-2.13570623492495	1.38677354709419\\
-2.13805724812759	1.40280561122244\\
-2.14028056112224	1.41829016635702\\
-2.14036139906636	1.4188376753507\\
-2.14268365816133	1.43486973947896\\
-2.14495534686681	1.45090180360721\\
-2.14717693583636	1.46693386773547\\
-2.14934887147192	1.48296593186373\\
-2.15147157617786	1.49899799599198\\
-2.1535454485973	1.51503006012024\\
-2.15557086383083	1.5310621242485\\
-2.1563126252505	1.53707027093956\\
-2.15758539475005	1.54709418837675\\
-2.15957253217634	1.56312625250501\\
-2.16151014528104	1.57915831663327\\
-2.1633985476971	1.59519038076152\\
-2.1652380295342	1.61122244488978\\
-2.16702885751804	1.62725450901804\\
-2.16877127511243	1.64328657314629\\
-2.17046550262438	1.65931863727455\\
-2.17211173729216	1.67535070140281\\
-2.17234468937876	1.67768616266539\\
-2.17374993028851	1.69138276553106\\
-2.17534536810776	1.70741482965932\\
-2.17689138091722	1.72344689378758\\
-2.17838810008491	1.73947895791583\\
-2.17983563341978	1.75551102204409\\
-2.18123406517753	1.77154308617235\\
-2.18258345604918	1.7875751503006\\
-2.18388384313207	1.80360721442886\\
-2.1851352398834	1.81963927855711\\
-2.18633763605613	1.83567134268537\\
-2.18749099761721	1.85170340681363\\
-2.18837675350701	1.8645652151021\\
-2.18860133989036	1.86773547094188\\
-2.18968554512439	1.88376753507014\\
-2.19071876149544	1.8997995991984\\
-2.19170087930555	1.91583166332665\\
-2.19263176387134	1.93186372745491\\
-2.19351125534943	1.94789579158317\\
-2.19433916854262	1.96392785571142\\
-2.19511529268665	1.97995991983968\\
-2.19583939121713	1.99599198396794\\
-2.19651120151655	2.01202404809619\\
-2.19713043464091	2.02805611222445\\
-2.19769677502566	2.04408817635271\\
-2.19820988017068	2.06012024048096\\
-2.19866938030378	2.07615230460922\\
-2.19907487802243	2.09218436873747\\
-2.19942594791327	2.10821643286573\\
-2.19972213614885	2.12424849699399\\
-2.19996296006125	2.14028056112224\\
-2.20014790769201	2.1563126252505\\
-2.20027643731779	2.17234468937876\\
-2.20034797695124	2.18837675350701\\
-2.20036192381657	2.20440881763527\\
-2.20031764379895	2.22044088176353\\
-2.20021447086739	2.23647294589178\\
-2.20005170647016	2.25250501002004\\
-2.19982861890213	2.2685370741483\\
-2.19954444264334	2.28456913827655\\
-2.19919837766783	2.30060120240481\\
-2.19878958872204	2.31663326653307\\
-2.19831720457188	2.33266533066132\\
-2.19778031721757	2.34869739478958\\
-2.19717798107516	2.36472945891784\\
-2.19650921212399	2.38076152304609\\
-2.19577298701883	2.39679358717435\\
-2.1949682421657	2.41282565130261\\
-2.19409387276027	2.42885771543086\\
-2.19314873178755	2.44488977955912\\
-2.19213162898178	2.46092184368737\\
-2.19104132974509	2.47695390781563\\
-2.1898765540237	2.49298597194389\\
-2.18863597514014	2.50901803607214\\
-2.18837675350701	2.51218829191193\\
}--cycle;


\addplot[area legend,solid,fill=mycolor3,draw=black,forget plot]
table[row sep=crcr] {%
x	y\\
-1.85170340681363	2.08110478474631\\
-1.85096402425816	2.09218436873747\\
-1.84980986194504	2.10821643286573\\
-1.84856756467358	2.12424849699399\\
-1.84723538393176	2.14028056112224\\
-1.84581150654969	2.1563126252505\\
-1.84429405280818	2.17234468937876\\
-1.84268107447124	2.18837675350701\\
-1.8409705527399	2.20440881763527\\
-1.83916039612411	2.22044088176353\\
-1.83724843822968	2.23647294589178\\
-1.83567134268537	2.2490366856574\\
-1.83523737101778	2.25250501002004\\
-1.83313835568119	2.2685370741483\\
-1.83093089661092	2.28456913827655\\
-1.82861245985542	2.30060120240481\\
-1.82618042239087	2.31663326653307\\
-1.82363206922794	2.33266533066132\\
-1.82096459040213	2.34869739478958\\
-1.81963927855711	2.35635177958395\\
-1.81818857080282	2.36472945891784\\
-1.81529996453305	2.38076152304609\\
-1.8122831122404	2.39679358717435\\
-1.80913473079258	2.41282565130261\\
-1.80585142247804	2.42885771543086\\
-1.80360721442886	2.43941001692876\\
-1.8024385369892	2.44488977955912\\
-1.79890051672139	2.46092184368737\\
-1.79521614279353	2.47695390781563\\
-1.79138143889474	2.49298597194389\\
-1.7875751503006	2.50829095273659\\
-1.78739338660917	2.50901803607214\\
-1.78326946333912	2.5250501002004\\
-1.77898147439258	2.54108216432866\\
-1.7745247538334	2.55711422845691\\
-1.77154308617234	2.56748238129583\\
-1.76990175838651	2.57314629258517\\
-1.76511267720294	2.58917835671343\\
-1.7601383404456	2.60521042084168\\
-1.75551102204409	2.61959233253044\\
-1.75497487581311	2.62124248496994\\
-1.7496281134607	2.6372745490982\\
-1.74407694612616	2.65330661322645\\
-1.73947895791583	2.66613514873682\\
-1.73831692392196	2.66933867735471\\
-1.73234548685442	2.68537074148297\\
-1.72614739434424	2.70140280561122\\
-1.72344689378758	2.70818987746438\\
-1.71971566625049	2.71743486973948\\
-1.71304013232617	2.73346693386774\\
-1.70741482965932	2.74652047249504\\
-1.7061103722011	2.74949899799599\\
-1.69891012383408	2.76553106212425\\
-1.69143610370506	2.7815631262525\\
-1.69138276553106	2.78167509565841\\
-1.68365691788803	2.79759519038076\\
-1.67557984389159	2.81362725450902\\
-1.67535070140281	2.81407237793107\\
-1.66715968749157	2.82965931863727\\
-1.65931863727455	2.84405724240068\\
-1.65840866361447	2.84569138276553\\
-1.64927335199124	2.86172344689379\\
-1.64328657314629	2.87190089013585\\
-1.6397574132464	2.87775551102204\\
-1.62982373029581	2.8937875751503\\
-1.62725450901804	2.89783144559011\\
-1.61943156879626	2.90981963927856\\
-1.61122244488978	2.92202643431353\\
-1.60857531675665	2.92585170340681\\
-1.59719561432098	2.94188376753507\\
-1.59519038076152	2.94464271289486\\
-1.58523875260076	2.95791583166333\\
-1.57915831663327	2.9658107404219\\
-1.5726802746891	2.97394789579158\\
-1.56312625250501	2.98564600747063\\
-1.55945965112415	2.98997995991984\\
-1.54709418837675	3.00424549977693\\
-1.5455044615911	3.0060120240481\\
-1.5310621242485	3.02169538090901\\
-1.5307279572433	3.02204408817635\\
-1.51503006012024	3.03807216018363\\
-1.51502598115792	3.03807615230461\\
-1.49899799599198	3.05344370640559\\
-1.49827304395488	3.05410821643287\\
-1.48296593186373	3.06787012941409\\
-1.48031727468776	3.07014028056112\\
-1.46693386773547	3.08140454787408\\
-1.46097385579681	3.08617234468938\\
-1.45090180360721	3.09409375882733\\
-1.44001638721062	3.10220440881764\\
-1.43486973947896	3.10597882209824\\
-1.4188376753507	3.11710033903772\\
-1.41710903881226	3.11823647294589\\
-1.40280561122244	3.12750636962587\\
-1.39167734969043	3.13426853707415\\
-1.38677354709419	3.1372101514459\\
-1.37074148296593	3.14625626530284\\
-1.36307656718187	3.1503006012024\\
-1.35470941883768	3.15466661391826\\
-1.33867735470942	3.16246635164411\\
-1.33012237525181	3.16633266533066\\
-1.32264529058116	3.16968031772411\\
-1.30661322645291	3.17633463965954\\
-1.29075325444766	3.18236472945892\\
-1.29058116232465	3.18242966225713\\
-1.27454909819639	3.18802432785573\\
-1.25851703406814	3.19309930910985\\
-1.24248496993988	3.19767335714685\\
-1.23967836172005	3.19839679358717\\
-1.22645290581162	3.2017900554151\\
-1.21042084168337	3.20544337144136\\
-1.19438877755511	3.20864327520901\\
-1.17835671342685	3.21140421553854\\
-1.1623246492986	3.21373973178891\\
-1.15661502706466	3.21442885771543\\
-1.14629258517034	3.215673431823\\
-1.13026052104208	3.21720929975852\\
-1.11422845691383	3.21835227700143\\
-1.09819639278557	3.21911252681467\\
-1.08216432865731	3.21949948728803\\
-1.06613226452906	3.21952189989994\\
-1.0501002004008	3.21918783601784\\
-1.03406813627255	3.21850472144465\\
-1.01803607214429	3.21747935911114\\
-1.00200400801603	3.21611795000566\\
-0.985998095749708	3.21442885771543\\
-0.985971943887776	3.21442614155896\\
-0.969939879759519	3.21243069533663\\
-0.953907815631263	3.21011861233941\\
-0.937875751503006	3.2074939838205\\
-0.92184368737475	3.20456038185535\\
-0.905811623246493	3.20132087162849\\
-0.892586167338064	3.19839679358717\\
-0.889779559118236	3.19778513500252\\
-0.87374749498998	3.19398599487374\\
-0.857715430861723	3.18989208776193\\
-0.841683366733467	3.18550468284531\\
-0.830931707074993	3.18236472945892\\
-0.82565130260521	3.18084309491341\\
-0.809619238476954	3.17593153597101\\
-0.793587174348697	3.17073286629194\\
-0.780729021259947	3.16633266533066\\
-0.777555110220441	3.16526011168692\\
-0.761523046092184	3.15955813639492\\
-0.745490981963928	3.15357218820123\\
-0.737123833619732	3.1503006012024\\
-0.729458917835671	3.14733881557017\\
-0.713426853707415	3.14086475809335\\
-0.697777947004	3.13426853707415\\
-0.697394789579158	3.13410884736716\\
-0.681362725450902	3.12715612856513\\
-0.665330661322646	3.11991987857457\\
-0.661735583422272	3.11823647294589\\
-0.649298597194389	3.11247386624583\\
-0.633266533066132	3.10476552161437\\
-0.628119885334472	3.10220440881764\\
-0.617234468937876	3.09684155551826\\
-0.601202404809619	3.08866524528832\\
-0.596468517478442	3.08617234468938\\
-0.585170340681363	3.08027895349741\\
-0.569138276553106	3.07163682472855\\
-0.566442776085726	3.07014028056112\\
-0.55310621242485	3.06280212052261\\
-0.537799100333697	3.05410821643287\\
-0.537074148296593	3.05370005159222\\
-0.521042084168337	3.0444235273177\\
-0.510393744430537	3.03807615230461\\
-0.50501002004008	3.03489325179561\\
-0.488977955911824	3.02515213976752\\
-0.483995866906375	3.02204408817635\\
-0.472945891783567	3.01520356570113\\
-0.458503554440966	3.0060120240481\\
-0.456913827655311	3.00500778702454\\
-0.440881763527054	2.99463500693657\\
-0.433868346994613	2.98997995991984\\
-0.424849699398798	2.98403536906207\\
-0.409932488193913	2.97394789579158\\
-0.408817635270541	2.97319901951363\\
-0.392785571142285	2.96219094658839\\
-0.386705135174792	2.95791583166333\\
-0.376753507014028	2.95096193545089\\
-0.364065662450606	2.94188376753507\\
-0.360721442885771	2.93950507801007\\
-0.344689378757515	2.92785282855784\\
-0.341992196266275	2.92585170340681\\
-0.328657314629258	2.91601138957255\\
-0.32044819072278	2.90981963927856\\
-0.312625250501002	2.90394942268957\\
-0.299368989727407	2.8937875751503\\
-0.296593186372745	2.89167009450756\\
-0.280561122244489	2.87919735205685\\
-0.278741749184476	2.87775551102204\\
-0.264529058116232	2.86654115588278\\
-0.258542279271288	2.86172344689379\\
-0.248496993987976	2.85367293277856\\
-0.238727092594649	2.84569138276553\\
-0.232464929859719	2.84059517615725\\
-0.219276137482059	2.82965931863727\\
-0.216432865731463	2.82731020824735\\
-0.200400801603207	2.81382306215925\\
-0.200171659114423	2.81362725450902\\
-0.18436873747495	2.80016487245945\\
-0.181403709616206	2.79759519038076\\
-0.168336673346694	2.78630243725173\\
-0.162943743482435	2.7815631262525\\
-0.152304609218437	2.77223744705891\\
-0.14477725091542	2.76553106212425\\
-0.13627254509018	2.75797143184814\\
-0.126890685934725	2.74949899799599\\
-0.120240480961924	2.74350576301438\\
-0.109271399852711	2.73346693386774\\
-0.104208416833667	2.72884165515535\\
-0.0919075802425003	2.71743486973948\\
-0.0881763527054109	2.71398016764136\\
-0.0747881948941846	2.70140280561122\\
-0.0721442885771544	2.69892220598275\\
-0.0579029403073767	2.68537074148297\\
-0.0561122244488979	2.68366852299741\\
-0.0412421943145111	2.66933867735471\\
-0.0400801603206413	2.66821971978054\\
-0.0247969724703408	2.65330661322645\\
-0.0240480961923848	2.65257624647837\\
-0.00855888787954408	2.6372745490982\\
-0.00801603206412826	2.63673840286722\\
0.00747988583314737	2.62124248496994\\
0.00801603206412782	2.62070633873896\\
0.0233266486770152	2.60521042084168\\
0.0240480961923843	2.6044800540936\\
0.0389882058144757	2.58917835671343\\
0.0400801603206409	2.58805939913926\\
0.0544708966630596	2.57314629258517\\
0.0561122244488974	2.57144407409961\\
0.0697806217774118	2.55711422845691\\
0.0721442885771539	2.55463362882844\\
0.0849228676899447	2.54108216432866\\
0.0881763527054105	2.53762746223054\\
0.0999027298721873	2.5250501002004\\
0.104208416833667	2.52042482148801\\
0.114724933963928	2.50901803607214\\
0.120240480961924	2.50302480109053\\
0.129393855403837	2.49298597194389\\
0.13627254509018	2.48542634166778\\
0.143913537583111	2.47695390781563\\
0.152304609218437	2.46762822862203\\
0.158287708632811	2.46092184368737\\
0.168336673346693	2.44962909055835\\
0.172519796945656	2.44488977955912\\
0.18436873747495	2.43142739750955\\
0.186612945524132	2.42885771543086\\
0.200400801603206	2.41302145895284\\
0.200570025238683	2.41282565130261\\
0.214411163717496	2.39679358717435\\
0.216432865731463	2.39444447678443\\
0.22812561583566	2.38076152304609\\
0.232464929859719	2.37566531643781\\
0.241714242995323	2.36472945891784\\
0.248496993987976	2.35667894480261\\
0.255179057332447	2.34869739478958\\
0.264529058116232	2.33748303965031\\
0.268521848787055	2.33266533066132\\
0.280561122244489	2.31807510756787\\
0.281744193712368	2.31663326653307\\
0.294865693023195	2.30060120240481\\
0.296593186372745	2.29848372176207\\
0.307884804426556	2.28456913827655\\
0.312625250501002	2.27869892168757\\
0.320790737016945	2.2685370741483\\
0.328657314629258	2.25869676031403\\
0.333584388029848	2.25250501002004\\
0.344689378757515	2.23847407104281\\
0.346266474301825	2.23647294589178\\
0.358859841022653	2.22044088176353\\
0.360721442885771	2.21806219223852\\
0.37136467745965	2.20440881763527\\
0.376753507014028	2.19745492142283\\
0.383763238799898	2.18837675350701\\
0.392785571142285	2.17661980430382\\
0.396055667160588	2.17234468937876\\
0.408249324480346	2.1563126252505\\
0.408817635270541	2.15556374897255\\
0.420381676516927	2.14028056112224\\
0.424849699398798	2.13433597026447\\
0.432411864046547	2.12424849699399\\
0.440881763527054	2.11287147988246\\
0.444339516511447	2.10821643286573\\
0.456174445099843	2.09218436873747\\
0.456913827655311	2.09118013171392\\
0.467956638280627	2.07615230460922\\
0.472945891783567	2.069311782134\\
0.47963909184413	2.06012024048096\\
0.488977955911824	2.04719622794388\\
0.491220970321726	2.04408817635271\\
0.502735279239629	2.02805611222445\\
0.50501002004008	2.02487321171545\\
0.514184056143404	2.01202404809619\\
0.521042084168337	2.00233935898103\\
0.525533998792645	1.99599198396794\\
0.536788276952951	1.97995991983968\\
0.537074148296593	1.97955175499903\\
0.548013140606755	1.96392785571142\\
0.55310621242485	1.95658969567291\\
0.559140227809762	1.94789579158317\\
0.569138276553106	1.93336027162234\\
0.570167995371698	1.93186372745491\\
0.581160780256625	1.91583166332665\\
0.585170340681363	1.90993827213469\\
0.592072506252331	1.8997995991984\\
0.601202404809619	1.88626043566908\\
0.60288506512914	1.88376753507014\\
0.613657428555809	1.86773547094188\\
0.617234468937876	1.86237261764251\\
0.624359807395399	1.85170340681363\\
0.633266533066132	1.8382324554821\\
0.634962578770195	1.83567134268537\\
0.645529835186418	1.81963927855711\\
0.649298597194389	1.81387667185705\\
0.65602749226157	1.80360721442886\\
0.665330661322646	1.78925855592928\\
0.666424536732041	1.7875751503006\\
0.676801290439637	1.77154308617235\\
0.681362725450902	1.76443067766332\\
0.68709754087772	1.75551102204409\\
0.697293488050765	1.73947895791583\\
0.697394789579159	1.73931926820884\\
0.707491839219817	1.72344689378758\\
0.713426853707415	1.71401105067851\\
0.717588754260136	1.70741482965932\\
0.727616224980459	1.69138276553106\\
0.729458917835672	1.68842097989882\\
0.737618467272952	1.67535070140281\\
0.745490981963928	1.66259022427337\\
0.747516929577984	1.65931863727455\\
0.757388887898833	1.64328657314629\\
0.761523046092185	1.63651204421055\\
0.767195261899139	1.62725450901804\\
0.776907704806536	1.61122244488978\\
0.777555110220441	1.61014989124604\\
0.786623917378866	1.59519038076152\\
0.793587174348698	1.58355851759455\\
0.796233549095041	1.57915831663327\\
0.8058099752019	1.56312625250501\\
0.809619238476954	1.5566930590171\\
0.815331077458949	1.54709418837675\\
0.824760127386751	1.5310621242485\\
0.825651302605211	1.52954048970299\\
0.834194099609312	1.51503006012024\\
0.841683366733467	1.50213794937837\\
0.843517559081993	1.49899799599198\\
0.852828753689511	1.48296593186373\\
0.857715430861724	1.47446122603849\\
0.862065825221398	1.46693386773547\\
0.871240825915441	1.45090180360721\\
0.87374749498998	1.44649100489378\\
0.880392419679554	1.43486973947896\\
0.889435763559499	1.4188376753507\\
0.889779559118236	1.41822601676604\\
0.898502620912698	1.40280561122244\\
0.905811623246493	1.38969762513551\\
0.907453086973702	1.38677354709419\\
0.916401385107147	1.37074148296593\\
0.921843687374749	1.36087300710585\\
0.925266721059645	1.35470941883768\\
0.934093357167528	1.33867735470942\\
0.937875751503006	1.33174248081449\\
0.942873854060321	1.32264529058116\\
0.951582880981771	1.30661322645291\\
0.953907815631262	1.30230298107688\\
0.960278674444841	1.29058116232465\\
0.968874008993936	1.27454909819639\\
0.969939879759519	1.27255093581759\\
0.977485082284088	1.25851703406814\\
0.985970511113727	1.24248496993988\\
0.985971943887775	1.24248225378341\\
0.99449669779832	1.22645290581162\\
1.00200400801603	1.21210993397359\\
1.00289562534923	1.21042084168337\\
1.01131686927922	1.19438877755511\\
1.01803607214429	1.18140721482257\\
1.0196288895712	1.17835671342685\\
1.02794868041008	1.1623246492986\\
1.03406813627254	1.15036844889956\\
1.03617320976842	1.14629258517034\\
1.04439495700578	1.13026052104208\\
1.0501002004008	1.11898743521624\\
1.05253126766382	1.11422845691383\\
1.06065827319264	1.09819639278557\\
1.06613226452906	1.08725737084183\\
1.06870549438125	1.08216432865731\\
1.07674095704631	1.06613226452906\\
1.08216432865731	1.0551708299734\\
1.0846980757674	1.0501002004008\\
1.09264509570445	1.03406813627254\\
1.09819639278557	1.02271974124353\\
1.10051095719356	1.01803607214429\\
1.10837253996919	1.00200400801603\\
1.11422845691383	0.989895363173775\\
1.11614584785142	0.985971943887775\\
1.12392490841295	0.969939879759519\\
1.13026052104208	0.956688257674352\\
1.13160422455531	0.953907815631262\\
1.13930359099971	0.937875751503006\\
1.14629258517034	0.92308826148232\\
1.14688733506242	0.921843687374749\\
1.15450975223249	0.905811623246493\\
1.16200401996448	0.889779559118236\\
1.1623246492986	0.889090433191721\\
1.16954433383618	0.87374749498998\\
1.17696578019152	0.857715430861724\\
1.17835671342685	0.854690788684829\\
1.18440805698392	0.841683366733467\\
1.19175920173928	0.825651302605211\\
1.19438877755511	0.819865720098788\\
1.19910142407344	0.809619238476954\\
1.20638471539175	0.793587174348698\\
1.21042084168337	0.784601688074628\\
1.21362472005891	0.777555110220441\\
1.22084253737021	0.761523046092185\\
1.22645290581162	0.748884243791853\\
1.22797801334242	0.745490981963928\\
1.23513267009027	0.729458917835672\\
1.24216907752019	0.713426853707415\\
1.24248496993988	0.712703417267096\\
1.24925490248686	0.697394789579159\\
1.2562314894554	0.681362725450902\\
1.25851703406814	0.676065240973576\\
1.26320880990842	0.665330661322646\\
1.27012770666234	0.649298597194389\\
1.27454909819639	0.63892613146295\\
1.27699375358044	0.633266533066132\\
1.28385703316352	0.617234468937876\\
1.29058116232465	0.60126733760783\\
1.29060887963702	0.601202404809619\\
1.29741856006962	0.585170340681363\\
1.30411699672397	0.569138276553106\\
1.30661322645291	0.56310818675373\\
1.31081116406888	0.55310621242485\\
1.31745878918003	0.537074148296593\\
1.32264529058116	0.524389736561787\\
1.32403350548239	0.521042084168337\\
1.33063218027391	0.50501002004008\\
1.33712404097122	0.488977955911824\\
1.33867735470942	0.485111642225269\\
1.34363556339904	0.472945891783567\\
1.3500809692004	0.456913827655311\\
1.35470941883768	0.445247776242905\\
1.35646711206576	0.440881763527054\\
1.36286777974099	0.424849699398798\\
1.3691656409182	0.408817635270541\\
1.37074148296593	0.404773299370973\\
1.37548237873076	0.392785571142285\\
1.38173772597015	0.376753507014028\\
1.38677354709419	0.363663057257521\\
1.3879224466588	0.360721442885771\\
1.39413686273957	0.344689378757515\\
1.40025208659613	0.328657314629258\\
1.40280561122244	0.32189514718098\\
1.40636046095951	0.312625250501002\\
1.41243672444534	0.296593186372745\\
1.41841668542161	0.280561122244489\\
1.4188376753507	0.279424988336313\\
1.42444445727106	0.264529058116232\\
1.43038727181292	0.248496993987976\\
1.43486973947896	0.236239343140326\\
1.43627218400033	0.232464929859719\\
1.44217922799749	0.216432865731463\\
1.44799298112941	0.200400801603206\\
1.45090180360721	0.192290151612905\\
1.4537891721569	0.18436873747495\\
1.45956873876195	0.168336673346693\\
1.46525751441528	0.152304609218437\\
1.46693386773547	0.147536812403136\\
1.47096011403291	0.13627254509018\\
1.47661614380373	0.120240480961924\\
1.48218373363104	0.104208416833667\\
1.48296593186373	0.101938265686634\\
1.48778765267569	0.0881763527054105\\
1.49332379819697	0.0721442885771539\\
1.49877371175641	0.0561122244488974\\
1.49899799599198	0.0554477144216261\\
1.50427363149831	0.0400801603206409\\
1.50969326695727	0.0240480961923843\\
1.51502874080229	0.00801603206412782\\
1.51503006012024	0.00801203994315158\\
1.52041910140116	-0.00801603206412826\\
1.52572533108552	-0.0240480961923848\\
1.53094933679874	-0.0400801603206413\\
1.5310621242485	-0.0404288675879876\\
1.53622432482812	-0.0561122244488979\\
1.54141999058174	-0.0721442885771544\\
1.546535241786	-0.0881763527054109\\
1.54709418837675	-0.0899428769765749\\
1.551688776779	-0.104208416833667\\
1.55677646444501	-0.120240480961924\\
1.56178542281927	-0.13627254509018\\
1.56312625250501	-0.140606497539393\\
1.56681114278396	-0.152304609218437\\
1.57179318765974	-0.168336673346694\\
1.57669806798192	-0.18436873747495\\
1.57915831663327	-0.192505892844636\\
1.58158931382882	-0.200400801603207\\
1.5864678051518	-0.216432865731463\\
1.59127057938552	-0.232464929859719\\
1.59519038076152	-0.245738048628188\\
1.59602037820382	-0.248496993987976\\
1.60079716268189	-0.264529058116232\\
1.6054995631189	-0.280561122244489\\
1.61012914769989	-0.296593186372745\\
1.61122244488978	-0.300418455466027\\
1.61477729462685	-0.312625250501002\\
1.61938081609172	-0.328657314629258\\
1.62391268479772	-0.344689378757515\\
1.62725450901804	-0.356677572445968\\
1.62840340858265	-0.360721442885771\\
1.63290930970114	-0.376753507014028\\
1.63734461298571	-0.392785571142285\\
1.64171073109856	-0.408817635270541\\
1.64328657314629	-0.414672256156732\\
1.64607917023272	-0.424849699398798\\
1.65041882425813	-0.440881763527054\\
1.65469018763728	-0.456913827655311\\
1.6588945800742	-0.472945891783567\\
1.65931863727455	-0.474580032148421\\
1.66312834154604	-0.488977955911824\\
1.66730552696729	-0.50501002004008\\
1.67141648023444	-0.521042084168337\\
1.67535070140281	-0.536629024874543\\
1.67546529128451	-0.537074148296593\\
1.67954863901878	-0.55310621242485\\
1.68356639100834	-0.569138276553106\\
1.68751972819312	-0.585170340681363\\
1.69138276553106	-0.601090435403715\\
1.69141048284343	-0.601202404809619\\
1.69533503186013	-0.617234468937876\\
1.69919565132172	-0.633266533066132\\
1.70299343812526	-0.649298597194389\\
1.70672945483409	-0.665330661322646\\
1.70741482965932	-0.668309186823601\\
1.71047970488146	-0.681362725450902\\
1.7141847622063	-0.697394789579158\\
1.71782833817753	-0.713426853707415\\
1.7214114151401	-0.729458917835671\\
1.72344689378758	-0.738703910110774\\
1.72497200131837	-0.745490981963928\\
1.72852233483592	-0.761523046092184\\
1.73201231025909	-0.777555110220441\\
1.73544283162421	-0.793587174348697\\
1.7388147715124	-0.809619238476954\\
1.73947895791583	-0.812822767094841\\
1.74219405406295	-0.82565130260521\\
1.7455303014729	-0.841683366733467\\
1.74880791438502	-0.857715430861723\\
1.7520276893305	-0.87374749498998\\
1.75519039270997	-0.889779559118236\\
1.75551102204409	-0.891429711557734\\
1.75836421385233	-0.905811623246493\\
1.7614878128815	-0.92184368737475\\
1.76455409200172	-0.937875751503006\\
1.76756374294779	-0.953907815631263\\
1.77051742834487	-0.969939879759519\\
1.77154308617235	-0.975603791048864\\
1.77346047710993	-0.985971943887776\\
1.77637115270639	-1.00200400801603\\
1.77922541525254	-1.01803607214429\\
1.78202385321948	-1.03406813627255\\
1.78476702669091	-1.0501002004008\\
1.78745546766551	-1.06613226452906\\
1.7875751503006	-1.06685934786461\\
1.7901483801528	-1.08216432865731\\
1.79278818988801	-1.09819639278557\\
1.79537256765055	-1.11422845691383\\
1.79790197103384	-1.13026052104208\\
1.80037682989223	-1.14629258517034\\
1.80279754654409	-1.1623246492986\\
1.80360721442886	-1.16780441192895\\
1.80520003185577	-1.17835671342685\\
1.80756567487247	-1.19438877755511\\
1.80987621161925	-1.21042084168337\\
1.8121319683394	-1.22645290581162\\
1.81433324370427	-1.24248496993988\\
1.81648030891641	-1.25851703406814\\
1.81857340779153	-1.27454909819639\\
1.81963927855711	-1.28292677753027\\
1.82063428513926	-1.29058116232465\\
1.82266346550923	-1.30661322645291\\
1.82463738111443	-1.32264529058116\\
1.82655619601566	-1.33867735470942\\
1.82842004637171	-1.35470941883768\\
1.83022904041777	-1.37074148296593\\
1.83198325842196	-1.38677354709419\\
1.83368275261984	-1.40280561122244\\
1.83532754712663	-1.4188376753507\\
1.83567134268537	-1.42230599971333\\
1.83694387071105	-1.43486973947896\\
1.83851104478601	-1.45090180360721\\
1.84002173704505	-1.46693386773547\\
1.84147588385069	-1.48296593186373\\
1.84287339223335	-1.49899799599198\\
1.84421413968947	-1.51503006012024\\
1.84549797395561	-1.5310621242485\\
1.84672471275821	-1.54709418837675\\
1.84789414353857	-1.56312625250501\\
1.84900602315273	-1.57915831663327\\
1.8500600775457	-1.59519038076152\\
1.85105600139972	-1.61122244488978\\
1.85170340681363	-1.62230202888094\\
1.85199909229475	-1.62725450901804\\
1.85289456737126	-1.64328657314629\\
1.85372935442768	-1.65931863727455\\
1.85450304230896	-1.67535070140281\\
1.85521518624693	-1.69138276553106\\
1.85586530736635	-1.70741482965932\\
1.85645289216113	-1.72344689378758\\
1.85697739193992	-1.73947895791583\\
1.85743822224045	-1.75551102204409\\
1.85783476221172	-1.77154308617234\\
1.8581663539633	-1.7875751503006\\
1.85843230188081	-1.80360721442886\\
1.85863187190667	-1.81963927855711\\
1.85876429078523	-1.83567134268537\\
1.85882874527115	-1.85170340681363\\
1.85882438130004	-1.86773547094188\\
1.85875030312024	-1.88376753507014\\
1.8586055723846	-1.8997995991984\\
1.85838920720086	-1.91583166332665\\
1.85810018113967	-1.93186372745491\\
1.85773742219854	-1.94789579158317\\
1.85729981172059	-1.96392785571142\\
1.85678618326647	-1.97995991983968\\
1.85619532143793	-1.99599198396794\\
1.85552596065151	-2.01202404809619\\
1.85477678386049	-2.02805611222445\\
1.85394642122353	-2.04408817635271\\
1.85303344871797	-2.06012024048096\\
1.85203638669594	-2.07615230460922\\
1.85170340681363	-2.08110478474632\\
1.85096402425816	-2.09218436873747\\
1.84980986194504	-2.10821643286573\\
1.84856756467358	-2.12424849699399\\
1.84723538393176	-2.14028056112224\\
1.84581150654969	-2.1563126252505\\
1.84429405280818	-2.17234468937876\\
1.84268107447124	-2.18837675350701\\
1.8409705527399	-2.20440881763527\\
1.83916039612411	-2.22044088176353\\
1.83724843822968	-2.23647294589178\\
1.83567134268537	-2.24903668565741\\
1.83523737101778	-2.25250501002004\\
1.83313835568119	-2.2685370741483\\
1.83093089661092	-2.28456913827655\\
1.82861245985542	-2.30060120240481\\
1.82618042239087	-2.31663326653307\\
1.82363206922794	-2.33266533066132\\
1.82096459040213	-2.34869739478958\\
1.81963927855711	-2.35635177958396\\
1.81818857080282	-2.36472945891784\\
1.81529996453305	-2.38076152304609\\
1.8122831122404	-2.39679358717435\\
1.80913473079258	-2.41282565130261\\
1.80585142247804	-2.42885771543086\\
1.80360721442886	-2.43941001692876\\
1.8024385369892	-2.44488977955912\\
1.79890051672139	-2.46092184368737\\
1.79521614279353	-2.47695390781563\\
1.79138143889474	-2.49298597194389\\
1.7875751503006	-2.50829095273659\\
1.78739338660917	-2.50901803607214\\
1.78326946333912	-2.5250501002004\\
1.77898147439258	-2.54108216432866\\
1.7745247538334	-2.55711422845691\\
1.77154308617235	-2.56748238129583\\
1.76990175838651	-2.57314629258517\\
1.76511267720294	-2.58917835671343\\
1.7601383404456	-2.60521042084168\\
1.75551102204409	-2.61959233253044\\
1.75497487581311	-2.62124248496994\\
1.7496281134607	-2.6372745490982\\
1.74407694612616	-2.65330661322645\\
1.73947895791583	-2.66613514873682\\
1.73831692392196	-2.66933867735471\\
1.73234548685442	-2.68537074148297\\
1.72614739434424	-2.70140280561122\\
1.72344689378758	-2.70818987746438\\
1.71971566625049	-2.71743486973948\\
1.71304013232617	-2.73346693386774\\
1.70741482965932	-2.74652047249504\\
1.7061103722011	-2.74949899799599\\
1.69891012383408	-2.76553106212425\\
1.69143610370506	-2.7815631262525\\
1.69138276553106	-2.78167509565841\\
1.68365691788803	-2.79759519038076\\
1.67557984389159	-2.81362725450902\\
1.67535070140281	-2.81407237793107\\
1.66715968749157	-2.82965931863727\\
1.65931863727455	-2.84405724240068\\
1.65840866361447	-2.84569138276553\\
1.64927335199124	-2.86172344689379\\
1.64328657314629	-2.87190089013585\\
1.6397574132464	-2.87775551102204\\
1.62982373029581	-2.8937875751503\\
1.62725450901804	-2.8978314455901\\
1.61943156879626	-2.90981963927856\\
1.61122244488978	-2.92202643431353\\
1.60857531675665	-2.92585170340681\\
1.59719561432098	-2.94188376753507\\
1.59519038076152	-2.94464271289486\\
1.58523875260076	-2.95791583166333\\
1.57915831663327	-2.9658107404219\\
1.5726802746891	-2.97394789579158\\
1.56312625250501	-2.98564600747063\\
1.55945965112415	-2.98997995991984\\
1.54709418837675	-3.00424549977693\\
1.5455044615911	-3.0060120240481\\
1.5310621242485	-3.02169538090901\\
1.5307279572433	-3.02204408817635\\
1.51503006012024	-3.03807216018363\\
1.51502598115792	-3.03807615230461\\
1.49899799599198	-3.05344370640559\\
1.49827304395488	-3.05410821643287\\
1.48296593186373	-3.06787012941409\\
1.48031727468776	-3.07014028056112\\
1.46693386773547	-3.08140454787408\\
1.46097385579681	-3.08617234468938\\
1.45090180360721	-3.09409375882733\\
1.44001638721062	-3.10220440881764\\
1.43486973947896	-3.10597882209824\\
1.4188376753507	-3.11710033903772\\
1.41710903881226	-3.11823647294589\\
1.40280561122244	-3.12750636962587\\
1.39167734969043	-3.13426853707415\\
1.38677354709419	-3.1372101514459\\
1.37074148296593	-3.14625626530284\\
1.36307656718187	-3.1503006012024\\
1.35470941883768	-3.15466661391826\\
1.33867735470942	-3.16246635164411\\
1.33012237525181	-3.16633266533066\\
1.32264529058116	-3.16968031772411\\
1.30661322645291	-3.17633463965954\\
1.29075325444766	-3.18236472945892\\
1.29058116232465	-3.18242966225713\\
1.27454909819639	-3.18802432785573\\
1.25851703406814	-3.19309930910985\\
1.24248496993988	-3.19767335714685\\
1.23967836172005	-3.19839679358717\\
1.22645290581162	-3.2017900554151\\
1.21042084168337	-3.20544337144136\\
1.19438877755511	-3.20864327520901\\
1.17835671342685	-3.21140421553854\\
1.1623246492986	-3.21373973178892\\
1.15661502706466	-3.21442885771543\\
1.14629258517034	-3.215673431823\\
1.13026052104208	-3.21720929975852\\
1.11422845691383	-3.21835227700143\\
1.09819639278557	-3.21911252681467\\
1.08216432865731	-3.21949948728802\\
1.06613226452906	-3.21952189989994\\
1.0501002004008	-3.21918783601784\\
1.03406813627254	-3.21850472144465\\
1.01803607214429	-3.21747935911114\\
1.00200400801603	-3.21611795000566\\
0.985998095749708	-3.21442885771543\\
0.985971943887775	-3.21442614155896\\
0.969939879759519	-3.21243069533663\\
0.953907815631262	-3.21011861233941\\
0.937875751503006	-3.2074939838205\\
0.921843687374749	-3.20456038185535\\
0.905811623246493	-3.20132087162849\\
0.892586167338063	-3.19839679358717\\
0.889779559118236	-3.19778513500252\\
0.87374749498998	-3.19398599487374\\
0.857715430861724	-3.18989208776193\\
0.841683366733467	-3.18550468284531\\
0.830931707074993	-3.18236472945892\\
0.825651302605211	-3.18084309491341\\
0.809619238476954	-3.17593153597101\\
0.793587174348698	-3.17073286629194\\
0.780729021259948	-3.16633266533066\\
0.777555110220441	-3.16526011168692\\
0.761523046092185	-3.15955813639492\\
0.745490981963928	-3.15357218820123\\
0.737123833619732	-3.1503006012024\\
0.729458917835672	-3.14733881557017\\
0.713426853707415	-3.14086475809335\\
0.697777947004	-3.13426853707415\\
0.697394789579159	-3.13410884736716\\
0.681362725450902	-3.12715612856513\\
0.665330661322646	-3.11991987857457\\
0.661735583422272	-3.11823647294589\\
0.649298597194389	-3.11247386624583\\
0.633266533066132	-3.10476552161437\\
0.628119885334472	-3.10220440881764\\
0.617234468937876	-3.09684155551826\\
0.601202404809619	-3.08866524528832\\
0.596468517478442	-3.08617234468938\\
0.585170340681363	-3.08027895349741\\
0.569138276553106	-3.07163682472855\\
0.566442776085726	-3.07014028056112\\
0.55310621242485	-3.06280212052261\\
0.537799100333697	-3.05410821643287\\
0.537074148296593	-3.05370005159222\\
0.521042084168337	-3.0444235273177\\
0.510393744430537	-3.03807615230461\\
0.50501002004008	-3.03489325179561\\
0.488977955911824	-3.02515213976752\\
0.483995866906375	-3.02204408817635\\
0.472945891783567	-3.01520356570113\\
0.458503554440966	-3.0060120240481\\
0.456913827655311	-3.00500778702454\\
0.440881763527054	-2.99463500693657\\
0.433868346994613	-2.98997995991984\\
0.424849699398798	-2.98403536906207\\
0.409932488193913	-2.97394789579158\\
0.408817635270541	-2.97319901951363\\
0.392785571142285	-2.96219094658839\\
0.386705135174793	-2.95791583166333\\
0.376753507014028	-2.95096193545089\\
0.364065662450606	-2.94188376753507\\
0.360721442885771	-2.93950507801007\\
0.344689378757515	-2.92785282855784\\
0.341992196266275	-2.92585170340681\\
0.328657314629258	-2.91601138957255\\
0.32044819072278	-2.90981963927856\\
0.312625250501002	-2.90394942268957\\
0.299368989727407	-2.8937875751503\\
0.296593186372745	-2.89167009450756\\
0.280561122244489	-2.87919735205685\\
0.278741749184475	-2.87775551102204\\
0.264529058116232	-2.86654115588278\\
0.258542279271288	-2.86172344689379\\
0.248496993987976	-2.85367293277856\\
0.238727092594649	-2.84569138276553\\
0.232464929859719	-2.84059517615724\\
0.219276137482059	-2.82965931863727\\
0.216432865731463	-2.82731020824735\\
0.200400801603206	-2.81382306215925\\
0.200171659114422	-2.81362725450902\\
0.18436873747495	-2.80016487245945\\
0.181403709616206	-2.79759519038076\\
0.168336673346693	-2.78630243725173\\
0.162943743482434	-2.7815631262525\\
0.152304609218437	-2.77223744705891\\
0.144777250915421	-2.76553106212425\\
0.13627254509018	-2.75797143184814\\
0.126890685934725	-2.74949899799599\\
0.120240480961924	-2.74350576301438\\
0.109271399852711	-2.73346693386774\\
0.104208416833667	-2.72884165515535\\
0.0919075802425008	-2.71743486973948\\
0.0881763527054105	-2.71398016764136\\
0.0747881948941855	-2.70140280561122\\
0.0721442885771539	-2.69892220598275\\
0.0579029403073767	-2.68537074148297\\
0.0561122244488974	-2.68366852299741\\
0.0412421943145115	-2.66933867735471\\
0.0400801603206409	-2.66821971978054\\
0.0247969724703404	-2.65330661322645\\
0.0240480961923843	-2.65257624647837\\
0.00855888787954364	-2.6372745490982\\
0.00801603206412782	-2.63673840286722\\
-0.00747988583314781	-2.62124248496994\\
-0.00801603206412826	-2.62070633873896\\
-0.0233266486770157	-2.60521042084168\\
-0.0240480961923848	-2.6044800540936\\
-0.0389882058144761	-2.58917835671343\\
-0.0400801603206413	-2.58805939913926\\
-0.05447089666306	-2.57314629258517\\
-0.0561122244488979	-2.57144407409961\\
-0.0697806217774123	-2.55711422845691\\
-0.0721442885771544	-2.55463362882844\\
-0.0849228676899451	-2.54108216432866\\
-0.0881763527054109	-2.53762746223054\\
-0.0999027298721868	-2.5250501002004\\
-0.104208416833667	-2.52042482148801\\
-0.114724933963928	-2.50901803607214\\
-0.120240480961924	-2.50302480109053\\
-0.129393855403837	-2.49298597194389\\
-0.13627254509018	-2.48542634166778\\
-0.14391353758311	-2.47695390781563\\
-0.152304609218437	-2.46762822862203\\
-0.158287708632811	-2.46092184368737\\
-0.168336673346694	-2.44962909055835\\
-0.172519796945657	-2.44488977955912\\
-0.18436873747495	-2.43142739750955\\
-0.186612945524132	-2.42885771543086\\
-0.200400801603207	-2.41302145895284\\
-0.200570025238682	-2.41282565130261\\
-0.214411163717496	-2.39679358717435\\
-0.216432865731463	-2.39444447678443\\
-0.228125615835659	-2.38076152304609\\
-0.232464929859719	-2.37566531643781\\
-0.241714242995323	-2.36472945891784\\
-0.248496993987976	-2.35667894480261\\
-0.255179057332446	-2.34869739478958\\
-0.264529058116232	-2.33748303965031\\
-0.268521848787055	-2.33266533066132\\
-0.280561122244489	-2.31807510756787\\
-0.281744193712368	-2.31663326653307\\
-0.294865693023194	-2.30060120240481\\
-0.296593186372745	-2.29848372176207\\
-0.307884804426556	-2.28456913827655\\
-0.312625250501002	-2.27869892168757\\
-0.320790737016945	-2.2685370741483\\
-0.328657314629258	-2.25869676031403\\
-0.333584388029848	-2.25250501002004\\
-0.344689378757515	-2.23847407104281\\
-0.346266474301825	-2.23647294589178\\
-0.358859841022652	-2.22044088176353\\
-0.360721442885771	-2.21806219223852\\
-0.371364677459651	-2.20440881763527\\
-0.376753507014028	-2.19745492142283\\
-0.383763238799898	-2.18837675350701\\
-0.392785571142285	-2.17661980430382\\
-0.396055667160588	-2.17234468937876\\
-0.408249324480346	-2.1563126252505\\
-0.408817635270541	-2.15556374897255\\
-0.420381676516927	-2.14028056112224\\
-0.424849699398798	-2.13433597026447\\
-0.432411864046547	-2.12424849699399\\
-0.440881763527054	-2.11287147988246\\
-0.444339516511446	-2.10821643286573\\
-0.456174445099843	-2.09218436873747\\
-0.456913827655311	-2.09118013171392\\
-0.467956638280628	-2.07615230460922\\
-0.472945891783567	-2.069311782134\\
-0.47963909184413	-2.06012024048096\\
-0.488977955911824	-2.04719622794388\\
-0.491220970321727	-2.04408817635271\\
-0.502735279239629	-2.02805611222445\\
-0.50501002004008	-2.02487321171545\\
-0.514184056143404	-2.01202404809619\\
-0.521042084168337	-2.00233935898103\\
-0.525533998792644	-1.99599198396794\\
-0.53678827695295	-1.97995991983968\\
-0.537074148296593	-1.97955175499903\\
-0.548013140606755	-1.96392785571142\\
-0.55310621242485	-1.95658969567291\\
-0.559140227809761	-1.94789579158317\\
-0.569138276553106	-1.93336027162234\\
-0.570167995371698	-1.93186372745491\\
-0.581160780256625	-1.91583166332665\\
-0.585170340681363	-1.90993827213469\\
-0.592072506252331	-1.8997995991984\\
-0.601202404809619	-1.88626043566908\\
-0.60288506512914	-1.88376753507014\\
-0.613657428555808	-1.86773547094188\\
-0.617234468937876	-1.86237261764251\\
-0.624359807395399	-1.85170340681363\\
-0.633266533066132	-1.8382324554821\\
-0.634962578770196	-1.83567134268537\\
-0.645529835186418	-1.81963927855711\\
-0.649298597194389	-1.81387667185705\\
-0.656027492261569	-1.80360721442886\\
-0.665330661322646	-1.78925855592928\\
-0.666424536732041	-1.7875751503006\\
-0.676801290439637	-1.77154308617234\\
-0.681362725450902	-1.76443067766332\\
-0.687097540877721	-1.75551102204409\\
-0.697293488050765	-1.73947895791583\\
-0.697394789579158	-1.73931926820884\\
-0.707491839219817	-1.72344689378758\\
-0.713426853707415	-1.71401105067852\\
-0.717588754260136	-1.70741482965932\\
-0.727616224980459	-1.69138276553106\\
-0.729458917835671	-1.68842097989882\\
-0.737618467272952	-1.67535070140281\\
-0.745490981963928	-1.66259022427337\\
-0.747516929577984	-1.65931863727455\\
-0.757388887898833	-1.64328657314629\\
-0.761523046092184	-1.63651204421055\\
-0.767195261899139	-1.62725450901804\\
-0.776907704806536	-1.61122244488978\\
-0.777555110220441	-1.61014989124604\\
-0.786623917378867	-1.59519038076152\\
-0.793587174348697	-1.58355851759455\\
-0.796233549095041	-1.57915831663327\\
-0.8058099752019	-1.56312625250501\\
-0.809619238476954	-1.5566930590171\\
-0.815331077458949	-1.54709418837675\\
-0.824760127386751	-1.5310621242485\\
-0.82565130260521	-1.52954048970299\\
-0.834194099609312	-1.51503006012024\\
-0.841683366733467	-1.50213794937837\\
-0.843517559081993	-1.49899799599198\\
-0.852828753689511	-1.48296593186373\\
-0.857715430861723	-1.47446122603849\\
-0.862065825221398	-1.46693386773547\\
-0.871240825915441	-1.45090180360721\\
-0.87374749498998	-1.44649100489378\\
-0.880392419679554	-1.43486973947896\\
-0.889435763559499	-1.4188376753507\\
-0.889779559118236	-1.41822601676604\\
-0.898502620912698	-1.40280561122244\\
-0.905811623246493	-1.38969762513551\\
-0.907453086973702	-1.38677354709419\\
-0.916401385107147	-1.37074148296593\\
-0.92184368737475	-1.36087300710585\\
-0.925266721059645	-1.35470941883768\\
-0.934093357167528	-1.33867735470942\\
-0.937875751503006	-1.33174248081448\\
-0.942873854060321	-1.32264529058116\\
-0.951582880981771	-1.30661322645291\\
-0.953907815631263	-1.30230298107688\\
-0.96027867444484	-1.29058116232465\\
-0.968874008993936	-1.27454909819639\\
-0.969939879759519	-1.27255093581759\\
-0.977485082284089	-1.25851703406814\\
-0.985970511113727	-1.24248496993988\\
-0.985971943887776	-1.24248225378341\\
-0.99449669779832	-1.22645290581162\\
-1.00200400801603	-1.21210993397359\\
-1.00289562534923	-1.21042084168337\\
-1.01131686927922	-1.19438877755511\\
-1.01803607214429	-1.18140721482257\\
-1.0196288895712	-1.17835671342685\\
-1.02794868041008	-1.1623246492986\\
-1.03406813627255	-1.15036844889956\\
-1.03617320976842	-1.14629258517034\\
-1.04439495700578	-1.13026052104208\\
-1.0501002004008	-1.11898743521624\\
-1.05253126766383	-1.11422845691383\\
-1.06065827319264	-1.09819639278557\\
-1.06613226452906	-1.08725737084183\\
-1.06870549438125	-1.08216432865731\\
-1.07674095704631	-1.06613226452906\\
-1.08216432865731	-1.0551708299734\\
-1.0846980757674	-1.0501002004008\\
-1.09264509570445	-1.03406813627255\\
-1.09819639278557	-1.02271974124353\\
-1.10051095719356	-1.01803607214429\\
-1.10837253996919	-1.00200400801603\\
-1.11422845691383	-0.989895363173774\\
-1.11614584785142	-0.985971943887776\\
-1.12392490841295	-0.969939879759519\\
-1.13026052104208	-0.956688257674352\\
-1.13160422455531	-0.953907815631263\\
-1.13930359099971	-0.937875751503006\\
-1.14629258517034	-0.92308826148232\\
-1.14688733506242	-0.92184368737475\\
-1.15450975223249	-0.905811623246493\\
-1.16200401996448	-0.889779559118236\\
-1.1623246492986	-0.889090433191721\\
-1.16954433383618	-0.87374749498998\\
-1.17696578019152	-0.857715430861723\\
-1.17835671342685	-0.854690788684829\\
-1.18440805698392	-0.841683366733467\\
-1.19175920173928	-0.82565130260521\\
-1.19438877755511	-0.819865720098788\\
-1.19910142407344	-0.809619238476954\\
-1.20638471539175	-0.793587174348697\\
-1.21042084168337	-0.784601688074629\\
-1.21362472005891	-0.777555110220441\\
-1.22084253737021	-0.761523046092184\\
-1.22645290581162	-0.748884243791853\\
-1.22797801334242	-0.745490981963928\\
-1.23513267009027	-0.729458917835671\\
-1.24216907752019	-0.713426853707415\\
-1.24248496993988	-0.712703417267096\\
-1.24925490248686	-0.697394789579158\\
-1.2562314894554	-0.681362725450902\\
-1.25851703406814	-0.676065240973576\\
-1.26320880990842	-0.665330661322646\\
-1.27012770666234	-0.649298597194389\\
-1.27454909819639	-0.638926131462949\\
-1.27699375358044	-0.633266533066132\\
-1.28385703316352	-0.617234468937876\\
-1.29058116232465	-0.601267337607829\\
-1.29060887963702	-0.601202404809619\\
-1.29741856006962	-0.585170340681363\\
-1.30411699672397	-0.569138276553106\\
-1.30661322645291	-0.56310818675373\\
-1.31081116406888	-0.55310621242485\\
-1.31745878918003	-0.537074148296593\\
-1.32264529058116	-0.524389736561787\\
-1.32403350548239	-0.521042084168337\\
-1.33063218027391	-0.50501002004008\\
-1.33712404097122	-0.488977955911824\\
-1.33867735470942	-0.48511164222527\\
-1.34363556339904	-0.472945891783567\\
-1.3500809692004	-0.456913827655311\\
-1.35470941883768	-0.445247776242905\\
-1.35646711206576	-0.440881763527054\\
-1.36286777974099	-0.424849699398798\\
-1.3691656409182	-0.408817635270541\\
-1.37074148296593	-0.404773299370972\\
-1.37548237873076	-0.392785571142285\\
-1.38173772597015	-0.376753507014028\\
-1.38677354709419	-0.363663057257521\\
-1.3879224466588	-0.360721442885771\\
-1.39413686273957	-0.344689378757515\\
-1.40025208659613	-0.328657314629258\\
-1.40280561122244	-0.32189514718098\\
-1.40636046095951	-0.312625250501002\\
-1.41243672444534	-0.296593186372745\\
-1.41841668542161	-0.280561122244489\\
-1.4188376753507	-0.279424988336314\\
-1.42444445727107	-0.264529058116232\\
-1.43038727181292	-0.248496993987976\\
-1.43486973947896	-0.236239343140326\\
-1.43627218400033	-0.232464929859719\\
-1.44217922799749	-0.216432865731463\\
-1.44799298112941	-0.200400801603207\\
-1.45090180360721	-0.192290151612905\\
-1.4537891721569	-0.18436873747495\\
-1.45956873876195	-0.168336673346694\\
-1.46525751441528	-0.152304609218437\\
-1.46693386773547	-0.147536812403136\\
-1.47096011403291	-0.13627254509018\\
-1.47661614380372	-0.120240480961924\\
-1.48218373363104	-0.104208416833667\\
-1.48296593186373	-0.101938265686635\\
-1.48778765267569	-0.0881763527054109\\
-1.49332379819697	-0.0721442885771544\\
-1.49877371175641	-0.0561122244488979\\
-1.49899799599198	-0.0554477144216257\\
-1.50427363149831	-0.0400801603206413\\
-1.50969326695727	-0.0240480961923848\\
-1.51502874080229	-0.00801603206412826\\
-1.51503006012024	-0.00801203994315157\\
-1.52041910140116	0.00801603206412782\\
-1.52572533108552	0.0240480961923843\\
-1.53094933679874	0.0400801603206409\\
-1.5310621242485	0.0404288675879877\\
-1.53622432482812	0.0561122244488974\\
-1.54141999058174	0.0721442885771539\\
-1.546535241786	0.0881763527054105\\
-1.54709418837675	0.0899428769765749\\
-1.551688776779	0.104208416833667\\
-1.55677646444501	0.120240480961924\\
-1.56178542281927	0.13627254509018\\
-1.56312625250501	0.140606497539392\\
-1.56681114278396	0.152304609218437\\
-1.57179318765974	0.168336673346693\\
-1.57669806798192	0.18436873747495\\
-1.57915831663327	0.192505892844634\\
-1.58158931382882	0.200400801603206\\
-1.5864678051518	0.216432865731463\\
-1.59127057938552	0.232464929859719\\
-1.59519038076152	0.245738048628186\\
-1.59602037820382	0.248496993987976\\
-1.60079716268189	0.264529058116232\\
-1.6054995631189	0.280561122244489\\
-1.61012914769989	0.296593186372745\\
-1.61122244488978	0.300418455466026\\
-1.61477729462685	0.312625250501002\\
-1.61938081609172	0.328657314629258\\
-1.62391268479772	0.344689378757515\\
-1.62725450901804	0.356677572445966\\
-1.62840340858265	0.360721442885771\\
-1.63290930970114	0.376753507014028\\
-1.63734461298571	0.392785571142285\\
-1.64171073109856	0.408817635270541\\
-1.64328657314629	0.414672256156731\\
-1.64607917023272	0.424849699398798\\
-1.65041882425813	0.440881763527054\\
-1.65469018763728	0.456913827655311\\
-1.6588945800742	0.472945891783567\\
-1.65931863727455	0.474580032148419\\
-1.66312834154604	0.488977955911824\\
-1.66730552696729	0.50501002004008\\
-1.67141648023444	0.521042084168337\\
-1.67535070140281	0.536629024874541\\
-1.67546529128451	0.537074148296593\\
-1.67954863901878	0.55310621242485\\
-1.68356639100834	0.569138276553106\\
-1.68751972819312	0.585170340681363\\
-1.69138276553106	0.601090435403712\\
-1.69141048284343	0.601202404809619\\
-1.69533503186013	0.617234468937876\\
-1.69919565132171	0.633266533066132\\
-1.70299343812526	0.649298597194389\\
-1.70672945483409	0.665330661322646\\
-1.70741482965932	0.668309186823599\\
-1.71047970488146	0.681362725450902\\
-1.7141847622063	0.697394789579159\\
-1.71782833817753	0.713426853707415\\
-1.7214114151401	0.729458917835672\\
-1.72344689378758	0.738703910110772\\
-1.72497200131837	0.745490981963928\\
-1.72852233483592	0.761523046092185\\
-1.73201231025909	0.777555110220441\\
-1.73544283162421	0.793587174348698\\
-1.7388147715124	0.809619238476954\\
-1.73947895791583	0.812822767094839\\
-1.74219405406295	0.825651302605211\\
-1.7455303014729	0.841683366733467\\
-1.74880791438502	0.857715430861724\\
-1.7520276893305	0.87374749498998\\
-1.75519039270997	0.889779559118236\\
-1.75551102204409	0.891429711557732\\
-1.75836421385233	0.905811623246493\\
-1.7614878128815	0.921843687374749\\
-1.76455409200172	0.937875751503006\\
-1.76756374294779	0.953907815631262\\
-1.77051742834487	0.969939879759519\\
-1.77154308617234	0.975603791048862\\
-1.77346047710993	0.985971943887775\\
-1.77637115270639	1.00200400801603\\
-1.77922541525254	1.01803607214429\\
-1.78202385321948	1.03406813627254\\
-1.78476702669091	1.0501002004008\\
-1.78745546766551	1.06613226452906\\
-1.7875751503006	1.06685934786461\\
-1.7901483801528	1.08216432865731\\
-1.79278818988801	1.09819639278557\\
-1.79537256765055	1.11422845691383\\
-1.79790197103384	1.13026052104208\\
-1.80037682989223	1.14629258517034\\
-1.80279754654409	1.1623246492986\\
-1.80360721442886	1.16780441192895\\
-1.80520003185577	1.17835671342685\\
-1.80756567487247	1.19438877755511\\
-1.80987621161925	1.21042084168337\\
-1.8121319683394	1.22645290581162\\
-1.81433324370427	1.24248496993988\\
-1.81648030891641	1.25851703406814\\
-1.81857340779153	1.27454909819639\\
-1.81963927855711	1.28292677753027\\
-1.82063428513926	1.29058116232465\\
-1.82266346550923	1.30661322645291\\
-1.82463738111443	1.32264529058116\\
-1.82655619601566	1.33867735470942\\
-1.82842004637171	1.35470941883768\\
-1.83022904041777	1.37074148296593\\
-1.83198325842196	1.38677354709419\\
-1.83368275261984	1.40280561122244\\
-1.83532754712663	1.4188376753507\\
-1.83567134268537	1.42230599971334\\
-1.83694387071105	1.43486973947896\\
-1.83851104478601	1.45090180360721\\
-1.84002173704505	1.46693386773547\\
-1.84147588385069	1.48296593186373\\
-1.84287339223335	1.49899799599198\\
-1.84421413968947	1.51503006012024\\
-1.84549797395561	1.5310621242485\\
-1.84672471275821	1.54709418837675\\
-1.84789414353857	1.56312625250501\\
-1.84900602315273	1.57915831663327\\
-1.8500600775457	1.59519038076152\\
-1.85105600139972	1.61122244488978\\
-1.85170340681363	1.62230202888095\\
-1.85199909229475	1.62725450901804\\
-1.85289456737126	1.64328657314629\\
-1.85372935442768	1.65931863727455\\
-1.85450304230896	1.67535070140281\\
-1.85521518624693	1.69138276553106\\
-1.85586530736635	1.70741482965932\\
-1.85645289216113	1.72344689378758\\
-1.85697739193992	1.73947895791583\\
-1.85743822224045	1.75551102204409\\
-1.85783476221172	1.77154308617235\\
-1.8581663539633	1.7875751503006\\
-1.85843230188081	1.80360721442886\\
-1.85863187190667	1.81963927855711\\
-1.85876429078523	1.83567134268537\\
-1.85882874527115	1.85170340681363\\
-1.85882438130004	1.86773547094188\\
-1.85875030312025	1.88376753507014\\
-1.8586055723846	1.8997995991984\\
-1.85838920720086	1.91583166332665\\
-1.85810018113967	1.93186372745491\\
-1.85773742219854	1.94789579158317\\
-1.85729981172059	1.96392785571142\\
-1.85678618326647	1.97995991983968\\
-1.85619532143793	1.99599198396794\\
-1.85552596065151	2.01202404809619\\
-1.85477678386049	2.02805611222445\\
-1.85394642122353	2.04408817635271\\
-1.85303344871797	2.06012024048096\\
-1.85203638669594	2.07615230460922\\
-1.85170340681363	2.08110478474631\\
}--cycle;


\addplot[area legend,solid,fill=mycolor4,draw=black,forget plot]
table[row sep=crcr] {%
x	y\\
-1.61122244488978	1.92154062635485\\
-1.61011248878115	1.93186372745491\\
-1.60828067438538	1.94789579158317\\
-1.60633322202377	1.96392785571142\\
-1.60426738422376	1.97995991983968\\
-1.60208031121155	1.99599198396794\\
-1.59976904743487	2.01202404809619\\
-1.59733052793776	2.02805611222445\\
-1.59519038076152	2.04143268533249\\
-1.59476347033196	2.04408817635271\\
-1.59207201391084	2.06012024048096\\
-1.58924270019754	2.07615230460922\\
-1.58627189389273	2.09218436873747\\
-1.58315582550126	2.10821643286573\\
-1.57989058644274	2.12424849699399\\
-1.57915831663327	2.1277110612458\\
-1.5764782936904	2.14028056112224\\
-1.57290909694685	2.1563126252505\\
-1.56917665717801	2.17234468937876\\
-1.56527632738903	2.18837675350701\\
-1.56312625250501	2.19688521514885\\
-1.56120437078298	2.20440881763527\\
-1.55695444683608	2.22044088176353\\
-1.55251945803753	2.23647294589178\\
-1.5478938590338	2.25250501002004\\
-1.54709418837675	2.25519119398684\\
-1.54306685727617	2.2685370741483\\
-1.53803384825403	2.28456913827655\\
-1.53278921849702	2.30060120240481\\
-1.5310621242485	2.30571529601306\\
-1.52731554649612	2.31663326653307\\
-1.52160843687928	2.33266533066132\\
-1.5156649653976	2.34869739478958\\
-1.51503006012024	2.35036312387492\\
-1.50945191118913	2.36472945891784\\
-1.50297916329548	2.38076152304609\\
-1.49899799599198	2.39029006435056\\
-1.49622401985135	2.39679358717435\\
-1.48916536249221	2.41282565130261\\
-1.48296593186373	2.42638221231704\\
-1.48180767585431	2.42885771543086\\
-1.47409706992632	2.44488977955912\\
-1.46693386773547	2.4592173411896\\
-1.4660599246591	2.46092184368737\\
-1.45762008602662	2.47695390781563\\
-1.45090180360721	2.48925412846316\\
-1.4488063422536	2.49298597194389\\
-1.43954643592755	2.50901803607214\\
-1.43486973947896	2.51685864994032\\
-1.42983385031263	2.5250501002004\\
-1.41964657378708	2.54108216432866\\
-1.4188376753507	2.54232403453555\\
-1.40888141830834	2.55711422845691\\
-1.40280561122244	2.56585381624442\\
-1.39755434365011	2.57314629258517\\
-1.38677354709419	2.58766528351743\\
-1.38560707639808	2.58917835671343\\
-1.37292503730557	2.60521042084168\\
-1.37074148296593	2.6078980187191\\
-1.35944897462705	2.62124248496994\\
-1.35470941883768	2.62669324850733\\
-1.34510050105639	2.6372745490982\\
-1.33867735470942	2.64416791174315\\
-1.32975869254771	2.65330661322645\\
-1.32264529058116	2.660420015193\\
-1.31327547851173	2.66933867735471\\
-1.30661322645291	2.67553578751015\\
-1.29546845826235	2.68537074148297\\
-1.29058116232465	2.68959102694447\\
-1.27611129417885	2.70140280561122\\
-1.27454909819639	2.70265225874734\\
-1.25851703406814	2.71477960525793\\
-1.25478519058741	2.71743486973948\\
-1.24248496993988	2.7260271997387\\
-1.23108671830602	2.73346693386774\\
-1.22645290581162	2.73644003213387\\
-1.21042084168337	2.74606380126248\\
-1.20427225725145	2.74949899799599\\
-1.19438877755511	2.75493757065414\\
-1.17835671342685	2.76309270994689\\
-1.17317428761382	2.76553106212425\\
-1.1623246492986	2.7705684654317\\
-1.14629258517034	2.77738784786229\\
-1.13552081220309	2.7815631262525\\
-1.13026052104208	2.78357893720756\\
-1.11422845691383	2.78917463166277\\
-1.09819639278557	2.79418489914963\\
-1.08594047435841	2.79759519038076\\
-1.08216432865731	2.79863663408102\\
-1.06613226452906	2.80255991511815\\
-1.0501002004008	2.80595976525476\\
-1.03406813627255	2.80885428964474\\
-1.01803607214429	2.81126042517164\\
-1.00200400801603	2.81319399879284\\
-0.997331417501761	2.81362725450902\\
-0.985971943887776	2.814675928256\\
-0.969939879759519	2.81571419710024\\
-0.953907815631263	2.81631895101368\\
-0.937875751503006	2.81650189349093\\
-0.92184368737475	2.8162738607838\\
-0.905811623246493	2.81564485848873\\
-0.889779559118236	2.81462409541754\\
-0.878420085504249	2.81362725450902\\
-0.87374749498998	2.81322297105368\\
-0.857715430861723	2.81145662464762\\
-0.841683366733467	2.80932517220469\\
-0.82565130260521	2.80683508253215\\
-0.809619238476954	2.80399215955377\\
-0.793587174348697	2.80080156334239\\
-0.779046159184035	2.79759519038076\\
-0.777555110220441	2.79727052082353\\
-0.761523046092184	2.79343021580375\\
-0.745490981963928	2.78925838468812\\
-0.729458917835671	2.78475793913733\\
-0.718857090061259	2.7815631262525\\
-0.713426853707415	2.7799455667311\\
-0.697394789579158	2.77484091227589\\
-0.681362725450902	2.76941790865144\\
-0.670513087135676	2.76553106212425\\
-0.665330661322646	2.76369440978076\\
-0.649298597194389	2.75769375545286\\
-0.633266533066132	2.75138106085277\\
-0.628713705911684	2.74949899799599\\
-0.617234468937876	2.74480100256469\\
-0.601202404809619	2.73793042458416\\
-0.591235090961048	2.73346693386774\\
-0.585170340681363	2.7307767665998\\
-0.569138276553106	2.72336002335721\\
-0.556838055905579	2.71743486973948\\
-0.55310621242485	2.71565329829787\\
-0.537074148296593	2.70769959745392\\
-0.524849509067469	2.70140280561122\\
-0.521042084168337	2.69945820243091\\
-0.50501002004008	2.69097432997933\\
-0.494783771844951	2.68537074148297\\
-0.488977955911824	2.68221462676709\\
-0.472945891783567	2.67320501962568\\
-0.466283639724746	2.66933867735471\\
-0.456913827655311	2.66394142181914\\
-0.440881763527054	2.65440822496289\\
-0.439082402647768	2.65330661322645\\
-0.424849699398798	2.64465327283536\\
-0.413074526782555	2.6372745490982\\
-0.408817635270541	2.63462480283205\\
-0.392785571142285	2.6243608052396\\
-0.388046015352913	2.62124248496994\\
-0.376753507014028	2.6138584373754\\
-0.363888837953899	2.60521042084168\\
-0.360721442885771	2.60309375284415\\
-0.344689378757515	2.59210255044851\\
-0.340529533563966	2.58917835671343\\
-0.328657314629258	2.5808776181457\\
-0.317876518073337	2.57314629258517\\
-0.312625250501002	2.5693997148328\\
-0.296593186372745	2.55767866334882\\
-0.295837892673827	2.55711422845691\\
-0.280561122244489	2.54575051469963\\
-0.274429644828789	2.54108216432866\\
-0.264529058116232	2.53357697573616\\
-0.253532883154308	2.5250501002004\\
-0.248496993987976	2.52116109576389\\
-0.233116340787461	2.50901803607214\\
-0.232464929859719	2.50850572468606\\
-0.216432865731463	2.49564312778937\\
-0.213188106974364	2.49298597194389\\
-0.200400801603207	2.48254928829144\\
-0.193682519183802	2.47695390781563\\
-0.18436873747495	2.46922073142719\\
-0.174568524887323	2.46092184368737\\
-0.168336673346694	2.45565956781352\\
-0.155824596672748	2.44488977955912\\
-0.152304609218437	2.44186772271686\\
-0.137430801099601	2.42885771543086\\
-0.13627254509018	2.42784693855345\\
-0.120240480961924	2.41360762885972\\
-0.119374742175998	2.41282565130261\\
-0.104208416833667	2.39915131035514\\
-0.101637517500693	2.39679358717435\\
-0.0881763527054109	2.38446763068815\\
-0.0841951854019193	2.38076152304609\\
-0.0721442885771544	2.36955763291399\\
-0.0670336762908581	2.36472945891784\\
-0.0561122244488979	2.35442218420181\\
-0.0501399221213309	2.34869739478958\\
-0.0400801603206413	2.33906197695177\\
-0.0335017835616004	2.33266533066132\\
-0.0240480961923848	2.32347752968626\\
-0.0171079835705498	2.31663326653307\\
-0.00801603206412826	2.30766918771692\\
-0.000948046752016809	2.30060120240481\\
0.00801603206412782	2.29163712358867\\
0.0149877560696653	2.28456913827655\\
0.0240480961923843	2.27538133730149\\
0.0307084585490926	2.2685370741483\\
0.0400801603206409	2.25890165631049\\
0.046222445236839	2.25250501002004\\
0.0561122244488974	2.24219773530401\\
0.0615374941096744	2.23647294589178\\
0.0721442885771539	2.22526905575968\\
0.0766608155678959	2.22044088176353\\
0.0881763527054105	2.20811492527733\\
0.0915990882169153	2.20440881763527\\
0.104208416833667	2.19073447668781\\
0.106358491717682	2.18837675350701\\
0.120240480961924	2.17312666693587\\
0.120944736961574	2.17234468937876\\
0.135364739286021	2.1563126252505\\
0.13627254509018	2.15530184837309\\
0.149624586275568	2.14028056112224\\
0.152304609218437	2.13725850427999\\
0.163728008132806	2.12424849699399\\
0.168336673346693	2.11898622112013\\
0.177679279651716	2.10821643286573\\
0.18436873747495	2.10048325647729\\
0.191482314734416	2.09218436873747\\
0.200400801603206	2.08174768508503\\
0.205140685797516	2.07615230460922\\
0.216432865731463	2.06277739632644\\
0.218657641608958	2.06012024048096\\
0.232038019430153	2.04408817635271\\
0.232464929859719	2.04357586496662\\
0.245294482052848	2.02805611222445\\
0.248496993987976	2.02416710778794\\
0.258420727231885	2.01202404809619\\
0.264529058116232	2.0045188595037\\
0.271418988566259	1.99599198396794\\
0.280561122244489	1.98462827021065\\
0.284291241551083	1.97995991983968\\
0.296593186372745	1.96449229060333\\
0.297039214899656	1.96392785571142\\
0.309683479996603	1.94789579158317\\
0.312625250501002	1.94414921383079\\
0.322212717668509	1.93186372745491\\
0.328657314629258	1.92356298888719\\
0.334625405535997	1.91583166332665\\
0.344689378757515	1.90272379293348\\
0.346922486628156	1.8997995991984\\
0.359116872563305	1.88376753507014\\
0.360721442885771	1.88165086707261\\
0.371216990235094	1.86773547094188\\
0.376753507014028	1.86035142334735\\
0.383207437170299	1.85170340681363\\
0.392785571142285	1.83878966295503\\
0.395088476620054	1.83567134268537\\
0.406877006331786	1.81963927855711\\
0.408817635270541	1.81698953229097\\
0.418579485774504	1.80360721442886\\
0.424849699398798	1.79495387403776\\
0.430176934846032	1.7875751503006\\
0.440881763527054	1.77264469790878\\
0.441669019392435	1.77154308617235\\
0.453092393214548	1.75551102204409\\
0.456913827655311	1.75011376650852\\
0.464421380727239	1.73947895791583\\
0.472945891783567	1.72731323605855\\
0.475648009534167	1.72344689378758\\
0.486794006752962	1.70741482965932\\
0.488977955911824	1.70425871494344\\
0.497867416095764	1.69138276553106\\
0.50501002004008	1.68095428989917\\
0.508840595929194	1.67535070140281\\
0.519726753421353	1.65931863727455\\
0.521042084168337	1.65737403409424\\
0.530555404980253	1.64328657314629\\
0.537074148296593	1.63355130086073\\
0.54128514558374	1.62725450901804\\
0.551928074821154	1.61122244488978\\
0.55310621242485	1.60944087344817\\
0.562520884658426	1.59519038076152\\
0.569138276553106	1.585083470251\\
0.573015347633412	1.57915831663327\\
0.583430809687441	1.56312625250501\\
0.585170340681363	1.56043608523708\\
0.593794926995524	1.54709418837675\\
0.601202404809619	1.53552561496492\\
0.604060554477165	1.5310621242485\\
0.614263529626301	1.51503006012024\\
0.617234468937876	1.51033206468894\\
0.62440445512755	1.49899799599198\\
0.633266533066132	1.4848479947205\\
0.634446080277041	1.48296593186373\\
0.644450832711952	1.46693386773547\\
0.649298597194389	1.45909656106408\\
0.654372519869492	1.45090180360721\\
0.664208689869032	1.43486973947896\\
0.665330661322646	1.43303308713547\\
0.67401359899695	1.4188376753507\\
0.681362725450902	1.40669245774963\\
0.683718539634568	1.40280561122244\\
0.69337756591569	1.38677354709419\\
0.697394789579159	1.38005133311758\\
0.702969211433921	1.37074148296593\\
0.712472211428738	1.35470941883768\\
0.713426853707415	1.35309185931627\\
0.721953017014737	1.33867735470942\\
0.729458917835672	1.32584010346599\\
0.7313317452183	1.32264529058116\\
0.740677067007647	1.30661322645291\\
0.745490981963928	1.29827642076026\\
0.749947267981695	1.29058116232465\\
0.759148058525284	1.27454909819639\\
0.761523046092185	1.27038412361938\\
0.76831154460939	1.25851703406814\\
0.777372290686372	1.24248496993988\\
0.777555110220441	1.24216030038265\\
0.786430653651559	1.22645290581162\\
0.793587174348698	1.213627214645\\
0.795383112446186	1.21042084168337\\
0.804310296988349	1.19438877755511\\
0.809619238476954	1.18475368259986\\
0.813158487753533	1.17835671342685\\
0.821955812894223	1.1623246492986\\
0.825651302605211	1.15553247732173\\
0.830700749518797	1.14629258517034\\
0.839372188236657	1.13026052104208\\
0.841683366733467	1.12595843873776\\
0.848014685479281	1.11422845691383\\
0.856564069847169	1.09819639278557\\
0.857715430861724	1.09602576292418\\
0.865104748160702	1.08216432865731\\
0.873535775099905	1.06613226452906\\
0.87374749498998	1.06572798107372\\
0.881975064979866	1.0501002004008\\
0.889779559118236	1.03506497718107\\
0.890299918799153	1.03406813627254\\
0.898629447604153	1.01803607214429\\
0.905811623246493	1.00402161199574\\
0.906851749946105	1.00200400801603\\
0.915071400596516	0.985971943887775\\
0.921843687374749	0.972586486034307\\
0.923191066272462	0.969939879759519\\
0.931304129372398	0.953907815631262\\
0.937875751503006	0.94075039048492\\
0.939320894316716	0.937875751503006\\
0.947330547492817	0.921843687374749\\
0.953907815631262	0.908503319751152\\
0.955243971041258	0.905811623246493\\
0.96315328331582	0.889779559118236\\
0.969939879759519	0.8758344375812\\
0.970962750081582	0.87374749498998\\
0.978774686026535	0.857715430861724\\
0.985971943887775	0.842732040480446\\
0.986479407382706	0.841683366733467\\
0.994196831064096	0.825651302605211\\
1.00179955554845	0.809619238476954\\
1.00200400801603	0.809185982760772\\
1.00942152496197	0.793587174348698\\
1.01693390146419	0.777555110220441\\
1.01803607214429	0.775188280883066\\
1.02445030961641	0.761523046092185\\
1.03187494369613	0.745490981963928\\
1.03406813627254	0.740718017099647\\
1.03928446599602	0.729458917835672\\
1.04662387524361	0.713426853707415\\
1.0501002004008	0.705759364453157\\
1.05392501730387	0.697394789579159\\
1.0611816344267	0.681362725450902\\
1.06613226452906	0.670295386060036\\
1.06837273160182	0.665330661322646\\
1.07554890734466	0.649298597194389\\
1.08216432865731	0.634307976766395\\
1.0826281239054	0.633266533066132\\
1.08972612981108	0.617234468937876\\
1.09671960727013	0.601202404809619\\
1.09819639278557	0.597792113578492\\
1.10371348877188	0.585170340681363\\
1.11063195689279	0.569138276553106\\
1.11422845691383	0.560717717835118\\
1.11751092321085	0.55310621242485\\
1.12435644545861	0.537074148296593\\
1.13026052104208	0.523057895123392\\
1.13111812454601	0.521042084168337\\
1.13789269380047	0.50501002004008\\
1.14456803763354	0.488977955911824\\
1.14629258517034	0.484802677521606\\
1.15124007716546	0.472945891783567\\
1.1578471485081	0.456913827655311\\
1.1623246492986	0.445919166834502\\
1.16439772446391	0.440881763527054\\
1.17093832854351	0.424849699398798\\
1.17738359767137	0.408817635270541\\
1.17835671342685	0.406379283093181\\
1.18384039529476	0.392785571142285\\
1.19022155276666	0.376753507014028\\
1.19438877755511	0.366160015543915\\
1.19655191858787	0.360721442885771\\
1.20287059404896	0.344689378757515\\
1.20909740780311	0.328657314629258\\
1.21042084168337	0.325222117895742\\
1.21532898133727	0.312625250501002\\
1.22149536367499	0.296593186372745\\
1.22645290581162	0.283534220510624\\
1.2275947227263	0.280561122244489\\
1.2337021375955	0.264529058116232\\
1.23972077261114	0.248496993987976\\
1.24248496993988	0.241057259858942\\
1.24571542558249	0.232464929859719\\
1.25167685511604	0.216432865731463\\
1.25755193846325	0.200400801603206\\
1.25851703406814	0.197745537121656\\
1.2634381954357	0.18436873747495\\
1.26925765836088	0.168336673346693\\
1.27454909819639	0.153554062354551\\
1.27500191367548	0.152304609218437\\
1.28076696369929	0.13627254509018\\
1.28644814606728	0.120240480961924\\
1.29058116232465	0.108428702295168\\
1.29207665346788	0.104208416833667\\
1.29770473969571	0.0881763527054105\\
1.30325100518834	0.0721442885771539\\
1.30661322645291	0.0623093346043338\\
1.30875930199207	0.0561122244488974\\
1.31425362229937	0.0400801603206409\\
1.3196680212829	0.0240480961923843\\
1.32264529058116	0.0151294340306766\\
1.32505136586401	0.00801603206412782\\
1.3304148080451	-0.00801603206412826\\
1.33570008541116	-0.0240480961923848\\
1.33867735470942	-0.0331867976756923\\
1.34095344682717	-0.0400801603206413\\
1.34618859574639	-0.0561122244488979\\
1.35134719757311	-0.0721442885771544\\
1.35470941883768	-0.0827255891680218\\
1.35646524407426	-0.0881763527054109\\
1.36157438764822	-0.104208416833667\\
1.36660846670857	-0.120240480961924\\
1.37074148296593	-0.133584947212763\\
1.37158555308672	-0.13627254509018\\
1.3765706871207	-0.152304609218437\\
1.38148210725868	-0.168336673346694\\
1.38632129259891	-0.18436873747495\\
1.38677354709419	-0.185881810670942\\
1.39117509288114	-0.200400801603207\\
1.39596543227035	-0.216432865731463\\
1.40068469352238	-0.232464929859719\\
1.40280561122244	-0.23975740620047\\
1.40538428971324	-0.248496993987976\\
1.41005484300632	-0.264529058116232\\
1.41465534948129	-0.280561122244489\\
1.4188376753507	-0.295351316165849\\
1.4191940356332	-0.296593186372745\\
1.4237458150046	-0.312625250501002\\
1.4282284552075	-0.328657314629258\\
1.43264320670604	-0.344689378757515\\
1.43486973947896	-0.352880829017598\\
1.43703288051172	-0.360721442885771\\
1.44139826164899	-0.376753507014028\\
1.44569647964082	-0.392785571142285\\
1.44992868785174	-0.408817635270541\\
1.45090180360721	-0.412549478751272\\
1.45415805356282	-0.424849699398798\\
1.45833994169558	-0.440881763527054\\
1.46245636694497	-0.456913827655311\\
1.46650838862278	-0.472945891783567\\
1.46693386773547	-0.474650394281346\\
1.47056559973766	-0.488977955911824\\
1.47456604049386	-0.50501002004008\\
1.47850244808027	-0.521042084168337\\
1.48237579019866	-0.537074148296593\\
1.48296593186373	-0.539549651410414\\
1.48624839816075	-0.55310621242485\\
1.49006858026906	-0.569138276553106\\
1.49382589172678	-0.585170340681363\\
1.49752121047654	-0.601202404809619\\
1.49899799599198	-0.607705927633412\\
1.50119610909336	-0.617234468937876\\
1.50483636086096	-0.633266533066132\\
1.50841463880758	-0.649298597194389\\
1.51193173238232	-0.665330661322646\\
1.51503006012024	-0.679696996365562\\
1.5153950975408	-0.681362725450902\\
1.51885487702331	-0.697394789579158\\
1.52225331347104	-0.713426853707415\\
1.52559110852095	-0.729458917835671\\
1.52886893167208	-0.745490981963928\\
1.5310621242485	-0.756408952483932\\
1.53210621144258	-0.761523046092184\\
1.53532310962606	-0.777555110220441\\
1.53847964119444	-0.793587174348697\\
1.54157638824944	-0.809619238476954\\
1.54461390135166	-0.82565130260521\\
1.54709418837675	-0.838997182766671\\
1.54760165187168	-0.841683366733467\\
1.5505740365611	-0.857715430861723\\
1.55348655045289	-0.87374749498998\\
1.55633965606131	-0.889779559118236\\
1.55913378459536	-0.905811623246493\\
1.56186933611373	-0.92184368737475\\
1.56312625250501	-0.929367289861169\\
1.56457139531872	-0.937875751503006\\
1.56723569397651	-0.953907815631263\\
1.56984046882894	-0.969939879759519\\
1.57238602985503	-0.985971943887776\\
1.57487265565427	-1.00200400801603\\
1.57730059347975	-1.01803607214429\\
1.57915831663327	-1.03060557202073\\
1.57967867631418	-1.03406813627255\\
1.58202824084217	-1.0501002004008\\
1.58431784226701	-1.06613226452906\\
1.58654763393225	-1.08216432865731\\
1.58871773711821	-1.09819639278557\\
1.59082824094773	-1.11422845691383\\
1.59287920226471	-1.13026052104208\\
1.59487064548539	-1.14629258517034\\
1.59519038076152	-1.14894807619056\\
1.59682818575385	-1.1623246492986\\
1.5987296300381	-1.17835671342685\\
1.6005697919074	-1.19438877755511\\
1.60234859378189	-1.21042084168337\\
1.60406592419264	-1.22645290581162\\
1.60572163749197	-1.24248496993988\\
1.60731555353372	-1.25851703406814\\
1.60884745732288	-1.27454909819639\\
1.61031709863407	-1.29058116232465\\
1.61122244488978	-1.30090426342471\\
1.61173158890952	-1.30661322645291\\
1.61309527227241	-1.32264529058116\\
1.61439429698395	-1.33867735470942\\
1.61562829653605	-1.35470941883768\\
1.61679686674454	-1.37074148296593\\
1.61789956520109	-1.38677354709419\\
1.61893591068928	-1.40280561122244\\
1.61990538256408	-1.4188376753507\\
1.62080742009368	-1.43486973947896\\
1.62164142176267	-1.45090180360721\\
1.6224067445356	-1.46693386773547\\
1.62310270307978	-1.48296593186373\\
1.62372856894605	-1.49899799599198\\
1.62428356970646	-1.51503006012024\\
1.62476688804736	-1.5310621242485\\
1.62517766081672	-1.54709418837675\\
1.62551497802411	-1.56312625250501\\
1.62577788179188	-1.57915831663327\\
1.62596536525592	-1.59519038076152\\
1.62607637141434	-1.61122244488978\\
1.62610979192237	-1.62725450901804\\
1.62606446583149	-1.64328657314629\\
1.62593917827105	-1.65931863727455\\
1.62573265907016	-1.67535070140281\\
1.62544358131785	-1.69138276553106\\
1.62507055985917	-1.70741482965932\\
1.62461214972503	-1.72344689378758\\
1.62406684449307	-1.73947895791583\\
1.62343307457727	-1.75551102204409\\
1.62270920544335	-1.77154308617234\\
1.62189353574725	-1.7875751503006\\
1.62098429539374	-1.80360721442886\\
1.61997964351187	-1.81963927855711\\
1.61887766634416	-1.83567134268537\\
1.61767637504605	-1.85170340681363\\
1.61637370339194	-1.86773547094188\\
1.61496750538409	-1.88376753507014\\
1.61345555276042	-1.8997995991984\\
1.61183553239709	-1.91583166332665\\
1.61122244488978	-1.92154062635485\\
1.61011248878115	-1.93186372745491\\
1.60828067438538	-1.94789579158317\\
1.60633322202377	-1.96392785571142\\
1.60426738422377	-1.97995991983968\\
1.60208031121155	-1.99599198396794\\
1.59976904743487	-2.01202404809619\\
1.59733052793776	-2.02805611222445\\
1.59519038076152	-2.04143268533248\\
1.59476347033196	-2.04408817635271\\
1.59207201391084	-2.06012024048096\\
1.58924270019754	-2.07615230460922\\
1.58627189389273	-2.09218436873747\\
1.58315582550126	-2.10821643286573\\
1.57989058644274	-2.12424849699399\\
1.57915831663327	-2.1277110612458\\
1.5764782936904	-2.14028056112224\\
1.57290909694685	-2.1563126252505\\
1.56917665717801	-2.17234468937876\\
1.56527632738902	-2.18837675350701\\
1.56312625250501	-2.19688521514885\\
1.56120437078298	-2.20440881763527\\
1.55695444683608	-2.22044088176353\\
1.55251945803753	-2.23647294589178\\
1.5478938590338	-2.25250501002004\\
1.54709418837675	-2.25519119398684\\
1.54306685727617	-2.2685370741483\\
1.53803384825403	-2.28456913827655\\
1.53278921849702	-2.30060120240481\\
1.5310621242485	-2.30571529601306\\
1.52731554649612	-2.31663326653307\\
1.52160843687928	-2.33266533066132\\
1.5156649653976	-2.34869739478958\\
1.51503006012024	-2.35036312387492\\
1.50945191118913	-2.36472945891784\\
1.50297916329547	-2.38076152304609\\
1.49899799599198	-2.39029006435056\\
1.49622401985135	-2.39679358717435\\
1.48916536249221	-2.41282565130261\\
1.48296593186373	-2.42638221231704\\
1.48180767585431	-2.42885771543086\\
1.47409706992632	-2.44488977955912\\
1.46693386773547	-2.4592173411896\\
1.4660599246591	-2.46092184368737\\
1.45762008602662	-2.47695390781563\\
1.45090180360721	-2.48925412846316\\
1.4488063422536	-2.49298597194389\\
1.43954643592755	-2.50901803607214\\
1.43486973947896	-2.51685864994032\\
1.42983385031263	-2.5250501002004\\
1.41964657378708	-2.54108216432866\\
1.4188376753507	-2.54232403453555\\
1.40888141830834	-2.55711422845691\\
1.40280561122244	-2.56585381624442\\
1.39755434365011	-2.57314629258517\\
1.38677354709419	-2.58766528351743\\
1.38560707639808	-2.58917835671343\\
1.37292503730557	-2.60521042084168\\
1.37074148296593	-2.6078980187191\\
1.35944897462705	-2.62124248496994\\
1.35470941883768	-2.62669324850733\\
1.34510050105639	-2.6372745490982\\
1.33867735470942	-2.64416791174315\\
1.32975869254771	-2.65330661322645\\
1.32264529058116	-2.660420015193\\
1.31327547851173	-2.66933867735471\\
1.30661322645291	-2.67553578751015\\
1.29546845826235	-2.68537074148297\\
1.29058116232465	-2.68959102694447\\
1.27611129417885	-2.70140280561122\\
1.27454909819639	-2.70265225874734\\
1.25851703406814	-2.71477960525793\\
1.2547851905874	-2.71743486973948\\
1.24248496993988	-2.7260271997387\\
1.23108671830602	-2.73346693386774\\
1.22645290581162	-2.73644003213387\\
1.21042084168337	-2.74606380126247\\
1.20427225725145	-2.74949899799599\\
1.19438877755511	-2.75493757065414\\
1.17835671342685	-2.76309270994689\\
1.17317428761382	-2.76553106212425\\
1.1623246492986	-2.7705684654317\\
1.14629258517034	-2.77738784786229\\
1.13552081220309	-2.7815631262525\\
1.13026052104208	-2.78357893720756\\
1.11422845691383	-2.78917463166277\\
1.09819639278557	-2.79418489914963\\
1.08594047435841	-2.79759519038076\\
1.08216432865731	-2.79863663408102\\
1.06613226452906	-2.80255991511815\\
1.0501002004008	-2.80595976525476\\
1.03406813627254	-2.80885428964474\\
1.01803607214429	-2.81126042517164\\
1.00200400801603	-2.81319399879284\\
0.997331417501759	-2.81362725450902\\
0.985971943887775	-2.814675928256\\
0.969939879759519	-2.81571419710024\\
0.953907815631262	-2.81631895101368\\
0.937875751503006	-2.81650189349093\\
0.921843687374749	-2.8162738607838\\
0.905811623246493	-2.81564485848873\\
0.889779559118236	-2.81462409541754\\
0.878420085504253	-2.81362725450902\\
0.87374749498998	-2.81322297105368\\
0.857715430861724	-2.81145662464762\\
0.841683366733467	-2.80932517220469\\
0.825651302605211	-2.80683508253215\\
0.809619238476954	-2.80399215955377\\
0.793587174348698	-2.80080156334239\\
0.779046159184034	-2.79759519038076\\
0.777555110220441	-2.79727052082353\\
0.761523046092185	-2.79343021580375\\
0.745490981963928	-2.78925838468812\\
0.729458917835672	-2.78475793913733\\
0.718857090061259	-2.7815631262525\\
0.713426853707415	-2.7799455667311\\
0.697394789579159	-2.77484091227589\\
0.681362725450902	-2.76941790865144\\
0.670513087135677	-2.76553106212425\\
0.665330661322646	-2.76369440978076\\
0.649298597194389	-2.75769375545286\\
0.633266533066132	-2.75138106085277\\
0.628713705911686	-2.74949899799599\\
0.617234468937876	-2.74480100256469\\
0.601202404809619	-2.73793042458416\\
0.591235090961049	-2.73346693386774\\
0.585170340681363	-2.7307767665998\\
0.569138276553106	-2.72336002335721\\
0.556838055905581	-2.71743486973948\\
0.55310621242485	-2.71565329829787\\
0.537074148296593	-2.70769959745392\\
0.524849509067469	-2.70140280561122\\
0.521042084168337	-2.69945820243091\\
0.50501002004008	-2.69097432997933\\
0.494783771844953	-2.68537074148297\\
0.488977955911824	-2.68221462676709\\
0.472945891783567	-2.67320501962568\\
0.466283639724747	-2.66933867735471\\
0.456913827655311	-2.66394142181914\\
0.440881763527054	-2.65440822496289\\
0.439082402647767	-2.65330661322645\\
0.424849699398798	-2.64465327283535\\
0.413074526782554	-2.6372745490982\\
0.408817635270541	-2.63462480283205\\
0.392785571142285	-2.6243608052396\\
0.388046015352912	-2.62124248496994\\
0.376753507014028	-2.6138584373754\\
0.363888837953899	-2.60521042084168\\
0.360721442885771	-2.60309375284415\\
0.344689378757515	-2.59210255044851\\
0.340529533563966	-2.58917835671343\\
0.328657314629258	-2.5808776181457\\
0.317876518073336	-2.57314629258517\\
0.312625250501002	-2.5693997148328\\
0.296593186372745	-2.55767866334882\\
0.295837892673827	-2.55711422845691\\
0.280561122244489	-2.54575051469963\\
0.274429644828789	-2.54108216432866\\
0.264529058116232	-2.53357697573616\\
0.253532883154308	-2.5250501002004\\
0.248496993987976	-2.52116109576389\\
0.233116340787461	-2.50901803607214\\
0.232464929859719	-2.50850572468606\\
0.216432865731463	-2.49564312778937\\
0.213188106974364	-2.49298597194389\\
0.200400801603206	-2.48254928829144\\
0.193682519183801	-2.47695390781563\\
0.18436873747495	-2.46922073142719\\
0.174568524887323	-2.46092184368737\\
0.168336673346693	-2.45565956781352\\
0.155824596672748	-2.44488977955912\\
0.152304609218437	-2.44186772271686\\
0.137430801099602	-2.42885771543086\\
0.13627254509018	-2.42784693855345\\
0.120240480961924	-2.41360762885972\\
0.119374742175997	-2.41282565130261\\
0.104208416833667	-2.39915131035514\\
0.101637517500693	-2.39679358717435\\
0.0881763527054105	-2.38446763068815\\
0.0841951854019198	-2.38076152304609\\
0.0721442885771539	-2.36955763291399\\
0.0670336762908576	-2.36472945891784\\
0.0561122244488974	-2.35442218420181\\
0.0501399221213309	-2.34869739478958\\
0.0400801603206409	-2.33906197695177\\
0.0335017835616004	-2.33266533066132\\
0.0240480961923843	-2.32347752968626\\
0.0171079835705493	-2.31663326653307\\
0.00801603206412782	-2.30766918771692\\
0.00094804675201682	-2.30060120240481\\
-0.00801603206412826	-2.29163712358866\\
-0.0149877560696652	-2.28456913827655\\
-0.0240480961923848	-2.27538133730149\\
-0.0307084585490926	-2.2685370741483\\
-0.0400801603206413	-2.25890165631049\\
-0.046222445236839	-2.25250501002004\\
-0.0561122244488979	-2.24219773530401\\
-0.0615374941096744	-2.23647294589178\\
-0.0721442885771544	-2.22526905575968\\
-0.0766608155678955	-2.22044088176353\\
-0.0881763527054109	-2.20811492527733\\
-0.0915990882169157	-2.20440881763527\\
-0.104208416833667	-2.19073447668781\\
-0.106358491717682	-2.18837675350701\\
-0.120240480961924	-2.17312666693587\\
-0.120944736961575	-2.17234468937876\\
-0.13536473928602	-2.1563126252505\\
-0.13627254509018	-2.15530184837309\\
-0.149624586275567	-2.14028056112224\\
-0.152304609218437	-2.13725850427999\\
-0.163728008132807	-2.12424849699399\\
-0.168336673346694	-2.11898622112013\\
-0.177679279651716	-2.10821643286573\\
-0.18436873747495	-2.10048325647729\\
-0.191482314734416	-2.09218436873747\\
-0.200400801603207	-2.08174768508503\\
-0.205140685797516	-2.07615230460922\\
-0.216432865731463	-2.06277739632644\\
-0.218657641608958	-2.06012024048096\\
-0.232038019430153	-2.04408817635271\\
-0.232464929859719	-2.04357586496662\\
-0.245294482052848	-2.02805611222445\\
-0.248496993987976	-2.02416710778794\\
-0.258420727231884	-2.01202404809619\\
-0.264529058116232	-2.0045188595037\\
-0.271418988566259	-1.99599198396794\\
-0.280561122244489	-1.98462827021066\\
-0.284291241551083	-1.97995991983968\\
-0.296593186372745	-1.96449229060333\\
-0.297039214899656	-1.96392785571142\\
-0.309683479996603	-1.94789579158317\\
-0.312625250501002	-1.94414921383079\\
-0.322212717668508	-1.93186372745491\\
-0.328657314629258	-1.92356298888719\\
-0.334625405535997	-1.91583166332665\\
-0.344689378757515	-1.90272379293348\\
-0.346922486628155	-1.8997995991984\\
-0.359116872563304	-1.88376753507014\\
-0.360721442885771	-1.88165086707261\\
-0.371216990235094	-1.86773547094188\\
-0.376753507014028	-1.86035142334735\\
-0.383207437170299	-1.85170340681363\\
-0.392785571142285	-1.83878966295503\\
-0.395088476620055	-1.83567134268537\\
-0.406877006331786	-1.81963927855711\\
-0.408817635270541	-1.81698953229097\\
-0.418579485774505	-1.80360721442886\\
-0.424849699398798	-1.79495387403776\\
-0.430176934846032	-1.7875751503006\\
-0.440881763527054	-1.77264469790878\\
-0.441669019392435	-1.77154308617234\\
-0.453092393214548	-1.75551102204409\\
-0.456913827655311	-1.75011376650852\\
-0.46442138072724	-1.73947895791583\\
-0.472945891783567	-1.72731323605855\\
-0.475648009534168	-1.72344689378758\\
-0.486794006752962	-1.70741482965932\\
-0.488977955911824	-1.70425871494344\\
-0.497867416095764	-1.69138276553106\\
-0.50501002004008	-1.68095428989917\\
-0.508840595929195	-1.67535070140281\\
-0.519726753421353	-1.65931863727455\\
-0.521042084168337	-1.65737403409424\\
-0.530555404980253	-1.64328657314629\\
-0.537074148296593	-1.63355130086073\\
-0.54128514558374	-1.62725450901804\\
-0.551928074821154	-1.61122244488978\\
-0.55310621242485	-1.60944087344817\\
-0.562520884658426	-1.59519038076152\\
-0.569138276553106	-1.585083470251\\
-0.573015347633412	-1.57915831663327\\
-0.583430809687441	-1.56312625250501\\
-0.585170340681363	-1.56043608523708\\
-0.593794926995524	-1.54709418837675\\
-0.601202404809619	-1.53552561496492\\
-0.604060554477165	-1.5310621242485\\
-0.614263529626301	-1.51503006012024\\
-0.617234468937876	-1.51033206468894\\
-0.62440445512755	-1.49899799599198\\
-0.633266533066132	-1.4848479947205\\
-0.634446080277042	-1.48296593186373\\
-0.644450832711952	-1.46693386773547\\
-0.649298597194389	-1.45909656106408\\
-0.654372519869492	-1.45090180360721\\
-0.664208689869032	-1.43486973947896\\
-0.665330661322646	-1.43303308713547\\
-0.67401359899695	-1.4188376753507\\
-0.681362725450902	-1.40669245774963\\
-0.683718539634568	-1.40280561122244\\
-0.693377565915689	-1.38677354709419\\
-0.697394789579158	-1.38005133311758\\
-0.702969211433921	-1.37074148296593\\
-0.712472211428738	-1.35470941883768\\
-0.713426853707415	-1.35309185931627\\
-0.721953017014737	-1.33867735470942\\
-0.729458917835671	-1.32584010346599\\
-0.7313317452183	-1.32264529058116\\
-0.740677067007647	-1.30661322645291\\
-0.745490981963928	-1.29827642076026\\
-0.749947267981696	-1.29058116232465\\
-0.759148058525283	-1.27454909819639\\
-0.761523046092184	-1.27038412361938\\
-0.76831154460939	-1.25851703406814\\
-0.777372290686371	-1.24248496993988\\
-0.777555110220441	-1.24216030038265\\
-0.786430653651558	-1.22645290581162\\
-0.793587174348697	-1.213627214645\\
-0.795383112446185	-1.21042084168337\\
-0.804310296988349	-1.19438877755511\\
-0.809619238476954	-1.18475368259986\\
-0.813158487753533	-1.17835671342685\\
-0.821955812894222	-1.1623246492986\\
-0.82565130260521	-1.15553247732173\\
-0.830700749518797	-1.14629258517034\\
-0.839372188236656	-1.13026052104208\\
-0.841683366733467	-1.12595843873776\\
-0.848014685479281	-1.11422845691383\\
-0.856564069847169	-1.09819639278557\\
-0.857715430861723	-1.09602576292418\\
-0.865104748160702	-1.08216432865731\\
-0.873535775099906	-1.06613226452906\\
-0.87374749498998	-1.06572798107372\\
-0.881975064979865	-1.0501002004008\\
-0.889779559118236	-1.03506497718107\\
-0.890299918799153	-1.03406813627255\\
-0.898629447604153	-1.01803607214429\\
-0.905811623246493	-1.00402161199574\\
-0.906851749946105	-1.00200400801603\\
-0.915071400596516	-0.985971943887776\\
-0.92184368737475	-0.972586486034305\\
-0.923191066272462	-0.969939879759519\\
-0.931304129372398	-0.953907815631263\\
-0.937875751503006	-0.94075039048492\\
-0.939320894316716	-0.937875751503006\\
-0.947330547492816	-0.92184368737475\\
-0.953907815631263	-0.908503319751151\\
-0.955243971041259	-0.905811623246493\\
-0.96315328331582	-0.889779559118236\\
-0.969939879759519	-0.875834437581199\\
-0.970962750081582	-0.87374749498998\\
-0.978774686026536	-0.857715430861723\\
-0.985971943887776	-0.842732040480445\\
-0.986479407382707	-0.841683366733467\\
-0.994196831064096	-0.82565130260521\\
-1.00179955554845	-0.809619238476954\\
-1.00200400801603	-0.809185982760772\\
-1.00942152496197	-0.793587174348697\\
-1.01693390146419	-0.777555110220441\\
-1.01803607214429	-0.775188280883064\\
-1.02445030961641	-0.761523046092184\\
-1.03187494369613	-0.745490981963928\\
-1.03406813627255	-0.740718017099646\\
-1.03928446599602	-0.729458917835671\\
-1.04662387524361	-0.713426853707415\\
-1.0501002004008	-0.705759364453155\\
-1.05392501730387	-0.697394789579158\\
-1.0611816344267	-0.681362725450902\\
-1.06613226452906	-0.670295386060035\\
-1.06837273160182	-0.665330661322646\\
-1.07554890734466	-0.649298597194389\\
-1.08216432865731	-0.634307976766393\\
-1.0826281239054	-0.633266533066132\\
-1.08972612981108	-0.617234468937876\\
-1.09671960727013	-0.601202404809619\\
-1.09819639278557	-0.59779211357849\\
-1.10371348877188	-0.585170340681363\\
-1.11063195689279	-0.569138276553106\\
-1.11422845691383	-0.560717717835118\\
-1.11751092321085	-0.55310621242485\\
-1.12435644545861	-0.537074148296593\\
-1.13026052104208	-0.523057895123392\\
-1.13111812454601	-0.521042084168337\\
-1.13789269380047	-0.50501002004008\\
-1.14456803763354	-0.488977955911824\\
-1.14629258517034	-0.484802677521605\\
-1.15124007716546	-0.472945891783567\\
-1.1578471485081	-0.456913827655311\\
-1.1623246492986	-0.445919166834502\\
-1.16439772446391	-0.440881763527054\\
-1.17093832854351	-0.424849699398798\\
-1.17738359767137	-0.408817635270541\\
-1.17835671342685	-0.406379283093181\\
-1.18384039529476	-0.392785571142285\\
-1.19022155276666	-0.376753507014028\\
-1.19438877755511	-0.366160015543915\\
-1.19655191858787	-0.360721442885771\\
-1.20287059404896	-0.344689378757515\\
-1.20909740780311	-0.328657314629258\\
-1.21042084168337	-0.325222117895742\\
-1.21532898133727	-0.312625250501002\\
-1.22149536367499	-0.296593186372745\\
-1.22645290581162	-0.283534220510624\\
-1.2275947227263	-0.280561122244489\\
-1.2337021375955	-0.264529058116232\\
-1.23972077261114	-0.248496993987976\\
-1.24248496993988	-0.241057259858942\\
-1.24571542558249	-0.232464929859719\\
-1.25167685511604	-0.216432865731463\\
-1.25755193846325	-0.200400801603207\\
-1.25851703406814	-0.197745537121656\\
-1.2634381954357	-0.18436873747495\\
-1.26925765836088	-0.168336673346694\\
-1.27454909819639	-0.153554062354551\\
-1.27500191367548	-0.152304609218437\\
-1.28076696369929	-0.13627254509018\\
-1.28644814606728	-0.120240480961924\\
-1.29058116232465	-0.108428702295168\\
-1.29207665346788	-0.104208416833667\\
-1.29770473969571	-0.0881763527054109\\
-1.30325100518834	-0.0721442885771544\\
-1.30661322645291	-0.0623093346043342\\
-1.30875930199207	-0.0561122244488979\\
-1.31425362229937	-0.0400801603206413\\
-1.3196680212829	-0.0240480961923848\\
-1.32264529058116	-0.0151294340306766\\
-1.32505136586401	-0.00801603206412826\\
-1.3304148080451	0.00801603206412782\\
-1.33570008541116	0.0240480961923843\\
-1.33867735470942	0.0331867976756918\\
-1.34095344682717	0.0400801603206409\\
-1.34618859574639	0.0561122244488974\\
-1.35134719757311	0.0721442885771539\\
-1.35470941883768	0.0827255891680218\\
-1.35646524407426	0.0881763527054105\\
-1.36157438764822	0.104208416833667\\
-1.36660846670857	0.120240480961924\\
-1.37074148296593	0.133584947212763\\
-1.37158555308672	0.13627254509018\\
-1.3765706871207	0.152304609218437\\
-1.38148210725868	0.168336673346693\\
-1.38632129259891	0.18436873747495\\
-1.38677354709419	0.185881810670943\\
-1.39117509288114	0.200400801603206\\
-1.39596543227035	0.216432865731463\\
-1.40068469352238	0.232464929859719\\
-1.40280561122244	0.23975740620047\\
-1.40538428971324	0.248496993987976\\
-1.41005484300632	0.264529058116232\\
-1.41465534948129	0.280561122244489\\
-1.4188376753507	0.29535131616585\\
-1.4191940356332	0.296593186372745\\
-1.4237458150046	0.312625250501002\\
-1.4282284552075	0.328657314629258\\
-1.43264320670604	0.344689378757515\\
-1.43486973947896	0.352880829017598\\
-1.43703288051172	0.360721442885771\\
-1.44139826164899	0.376753507014028\\
-1.44569647964082	0.392785571142285\\
-1.44992868785174	0.408817635270541\\
-1.45090180360721	0.412549478751272\\
-1.45415805356282	0.424849699398798\\
-1.45833994169558	0.440881763527054\\
-1.46245636694497	0.456913827655311\\
-1.46650838862278	0.472945891783567\\
-1.46693386773547	0.474650394281346\\
-1.47056559973766	0.488977955911824\\
-1.47456604049386	0.50501002004008\\
-1.47850244808027	0.521042084168337\\
-1.48237579019867	0.537074148296593\\
-1.48296593186373	0.539549651410414\\
-1.48624839816075	0.55310621242485\\
-1.49006858026906	0.569138276553106\\
-1.49382589172678	0.585170340681363\\
-1.49752121047654	0.601202404809619\\
-1.49899799599198	0.607705927633412\\
-1.50119610909336	0.617234468937876\\
-1.50483636086096	0.633266533066132\\
-1.50841463880758	0.649298597194389\\
-1.51193173238232	0.665330661322646\\
-1.51503006012024	0.679696996365563\\
-1.5153950975408	0.681362725450902\\
-1.51885487702331	0.697394789579159\\
-1.52225331347104	0.713426853707415\\
-1.52559110852095	0.729458917835672\\
-1.52886893167208	0.745490981963928\\
-1.5310621242485	0.756408952483931\\
-1.53210621144258	0.761523046092185\\
-1.53532310962606	0.777555110220441\\
-1.53847964119444	0.793587174348698\\
-1.54157638824944	0.809619238476954\\
-1.54461390135166	0.825651302605211\\
-1.54709418837675	0.838997182766671\\
-1.54760165187168	0.841683366733467\\
-1.5505740365611	0.857715430861724\\
-1.55348655045289	0.87374749498998\\
-1.55633965606131	0.889779559118236\\
-1.55913378459536	0.905811623246493\\
-1.56186933611373	0.921843687374749\\
-1.56312625250501	0.929367289861167\\
-1.56457139531872	0.937875751503006\\
-1.56723569397651	0.953907815631262\\
-1.56984046882894	0.969939879759519\\
-1.57238602985503	0.985971943887775\\
-1.57487265565427	1.00200400801603\\
-1.57730059347975	1.01803607214429\\
-1.57915831663327	1.03060557202073\\
-1.57967867631418	1.03406813627254\\
-1.58202824084217	1.0501002004008\\
-1.58431784226701	1.06613226452906\\
-1.58654763393225	1.08216432865731\\
-1.58871773711821	1.09819639278557\\
-1.59082824094773	1.11422845691383\\
-1.59287920226471	1.13026052104208\\
-1.59487064548539	1.14629258517034\\
-1.59519038076152	1.14894807619056\\
-1.59682818575385	1.1623246492986\\
-1.5987296300381	1.17835671342685\\
-1.6005697919074	1.19438877755511\\
-1.60234859378189	1.21042084168337\\
-1.60406592419264	1.22645290581162\\
-1.60572163749197	1.24248496993988\\
-1.60731555353372	1.25851703406814\\
-1.60884745732288	1.27454909819639\\
-1.61031709863407	1.29058116232465\\
-1.61122244488978	1.30090426342471\\
-1.61173158890952	1.30661322645291\\
-1.61309527227241	1.32264529058116\\
-1.61439429698395	1.33867735470942\\
-1.61562829653605	1.35470941883768\\
-1.61679686674454	1.37074148296593\\
-1.61789956520109	1.38677354709419\\
-1.61893591068928	1.40280561122244\\
-1.61990538256409	1.4188376753507\\
-1.62080742009368	1.43486973947896\\
-1.62164142176267	1.45090180360721\\
-1.6224067445356	1.46693386773547\\
-1.62310270307978	1.48296593186373\\
-1.62372856894605	1.49899799599198\\
-1.62428356970646	1.51503006012024\\
-1.62476688804736	1.5310621242485\\
-1.62517766081672	1.54709418837675\\
-1.62551497802411	1.56312625250501\\
-1.62577788179189	1.57915831663327\\
-1.62596536525592	1.59519038076152\\
-1.62607637141434	1.61122244488978\\
-1.62610979192237	1.62725450901804\\
-1.62606446583149	1.64328657314629\\
-1.62593917827105	1.65931863727455\\
-1.62573265907016	1.67535070140281\\
-1.62544358131785	1.69138276553106\\
-1.62507055985917	1.70741482965932\\
-1.62461214972503	1.72344689378758\\
-1.62406684449307	1.73947895791583\\
-1.62343307457727	1.75551102204409\\
-1.62270920544335	1.77154308617235\\
-1.62189353574725	1.7875751503006\\
-1.62098429539374	1.80360721442886\\
-1.61997964351187	1.81963927855711\\
-1.61887766634416	1.83567134268537\\
-1.61767637504605	1.85170340681363\\
-1.61637370339194	1.86773547094188\\
-1.61496750538409	1.88376753507014\\
-1.61345555276042	1.8997995991984\\
-1.61183553239709	1.91583166332665\\
-1.61122244488978	1.92154062635485\\
}--cycle;


\addplot[area legend,solid,fill=mycolor5,draw=black,forget plot]
table[row sep=crcr] {%
x	y\\
-1.43486973947896	1.57614782301339\\
-1.43471943107953	1.57915831663327\\
-1.43381826014665	1.59519038076152\\
-1.43280952025882	1.61122244488978\\
-1.43169085204059	1.62725450901804\\
-1.4304598001939	1.64328657314629\\
-1.42911381027157	1.65931863727455\\
-1.4276502253059	1.67535070140281\\
-1.42606628228583	1.69138276553106\\
-1.42435910847566	1.70741482965932\\
-1.4225257175681	1.72344689378758\\
-1.42056300566399	1.73947895791583\\
-1.4188376753507	1.75270116524991\\
-1.41846804884804	1.75551102204409\\
-1.41623805524873	1.77154308617235\\
-1.41386758907841	1.7875751503006\\
-1.41135291389638	1.80360721442886\\
-1.40869014715433	1.81963927855711\\
-1.40587525464174	1.83567134268537\\
-1.40290404467388	1.85170340681363\\
-1.40280561122244	1.85221236475073\\
-1.39976673853409	1.86773547094188\\
-1.39646204917176	1.88376753507014\\
-1.39298525898563	1.8997995991984\\
-1.38933130901527	1.91583166332665\\
-1.38677354709419	1.92656076872958\\
-1.38549017444567	1.93186372745491\\
-1.38144940352448	1.94789579158317\\
-1.37721190834076	1.96392785571142\\
-1.37277155563299	1.97995991983968\\
-1.37074148296593	1.98700845858851\\
-1.36810682049629	1.99599198396794\\
-1.36321095195326	2.01202404809619\\
-1.35808788971171	2.02805611222445\\
-1.35470941883768	2.03821482835606\\
-1.35271501593282	2.04408817635271\\
-1.34706954954247	2.06012024048096\\
-1.34116778578055	2.07615230460922\\
-1.33867735470942	2.08268121363305\\
-1.33496551159711	2.09218436873747\\
-1.3284588477304	2.10821643286573\\
-1.32264529058116	2.12195623172791\\
-1.32164969012634	2.12424849699399\\
-1.31446628330718	2.14028056112224\\
-1.30696119426195	2.1563126252505\\
-1.30661322645291	2.1570351362201\\
-1.29902049060337	2.17234468937876\\
-1.29071804569593	2.18837675350701\\
-1.29058116232465	2.18863394640411\\
-1.28191145320043	2.20440881763527\\
-1.27454909819639	2.2172723129112\\
-1.27267209655242	2.22044088176353\\
-1.26287839719742	2.23647294589178\\
-1.25851703406814	2.2433713531395\\
-1.25252169552617	2.25250501002004\\
-1.24248496993988	2.26725508244213\\
-1.24157692121844	2.2685370741483\\
-1.2298926066466	2.28456913827655\\
-1.22645290581162	2.28914032936524\\
-1.2174438451519	2.30060120240481\\
-1.21042084168337	2.3092557276721\\
-1.20414790670024	2.31663326653307\\
-1.19438877755511	2.32776952279695\\
-1.18987869651441	2.33266533066132\\
-1.17835671342685	2.34481988990147\\
-1.17447920853874	2.34869739478958\\
-1.1623246492986	2.36052717493355\\
-1.15775251694412	2.36472945891784\\
-1.14629258517034	2.37499623458497\\
-1.13944940232465	2.38076152304609\\
-1.13026052104208	2.38831840762883\\
-1.11925105511328	2.39679358717435\\
-1.11422845691383	2.40057318173519\\
-1.09819639278557	2.41182795528761\\
-1.09667379717982	2.41282565130261\\
-1.08216432865731	2.42214004892704\\
-1.07080768800794	2.42885771543086\\
-1.06613226452906	2.4315707992682\\
-1.0501002004008	2.4401641685971\\
-1.0404549230817	2.44488977955912\\
-1.03406813627255	2.4479659469966\\
-1.01803607214429	2.45501471015489\\
-1.00309160510989	2.46092184368737\\
-1.00200400801603	2.46134532663581\\
-0.985971943887776	2.46699364640741\\
-0.969939879759519	2.47198542314756\\
-0.953907815631263	2.47634725722471\\
-0.951348053022255	2.47695390781563\\
-0.937875751503006	2.48011007091374\\
-0.92184368737475	2.48329258491622\\
-0.905811623246493	2.48591508836163\\
-0.889779559118236	2.48799746033493\\
-0.87374749498998	2.4895582254375\\
-0.857715430861723	2.49061462510628\\
-0.841683366733467	2.49118268369602\\
-0.82565130260521	2.49127726968805\\
-0.809619238476954	2.49091215235856\\
-0.793587174348697	2.49010005421127\\
-0.777555110220441	2.48885269945374\\
-0.761523046092184	2.48718085877326\\
-0.745490981963928	2.48509439064656\\
-0.729458917835671	2.48260227939784\\
-0.713426853707415	2.47971267020162\\
-0.699954552188166	2.47695390781563\\
-0.697394789579158	2.47643563232263\\
-0.681362725450902	2.47279233708659\\
-0.665330661322646	2.46877490054165\\
-0.649298597194389	2.46438859305215\\
-0.637613172394768	2.46092184368737\\
-0.633266533066132	2.45964550325484\\
-0.617234468937876	2.45456571746513\\
-0.601202404809619	2.44913301662021\\
-0.589448840589472	2.44488977955912\\
-0.585170340681363	2.44335978778237\\
-0.569138276553106	2.43726879886481\\
-0.55310621242485	2.43083597158032\\
-0.548430788945969	2.42885771543086\\
-0.537074148296593	2.42409432489073\\
-0.521042084168337	2.41703015422659\\
-0.511937632205333	2.41282565130261\\
-0.50501002004008	2.40965263427974\\
-0.488977955911824	2.40197261032635\\
-0.47861633688666	2.39679358717435\\
-0.472945891783567	2.39398111250709\\
-0.456913827655311	2.38569774828604\\
-0.447724946372745	2.38076152304609\\
-0.440881763527054	2.37711175756346\\
-0.424849699398798	2.36823470705649\\
-0.418742219050504	2.36472945891784\\
-0.408817635270541	2.35907140958386\\
-0.392785571142285	2.34960756594041\\
-0.391293580145418	2.34869739478958\\
-0.376753507014028	2.33988194566119\\
-0.365231523926469	2.33266533066132\\
-0.360721442885771	2.3298571615125\\
-0.344689378757515	2.31956044169135\\
-0.340265649230689	2.31663326653307\\
-0.328657314629258	2.30899339723786\\
-0.316276698443884	2.30060120240481\\
-0.312625250501002	2.298138761015\\
-0.296593186372745	2.28701987652264\\
-0.293153485537765	2.28456913827655\\
-0.280561122244489	2.27563866976863\\
-0.270812795449883	2.2685370741483\\
-0.264529058116232	2.26397936875109\\
-0.249116841161576	2.25250501002004\\
-0.248496993987976	2.25204542916615\\
-0.232464929859719	2.23986734300685\\
-0.228103566730434	2.23647294589178\\
-0.216432865731463	2.22742240388654\\
-0.20764169030169	2.22044088176353\\
-0.200400801603207	2.21470970355198\\
-0.187683817264239	2.20440881763527\\
-0.18436873747495	2.20173183890034\\
-0.168336673346694	2.18849214306497\\
-0.16819978997541	2.18837675350701\\
-0.152304609218437	2.17501187444054\\
-0.149198527430876	2.17234468937876\\
-0.13627254509018	2.16127082520807\\
-0.13060223360748	2.1563126252505\\
-0.120240480961924	2.14727077441581\\
-0.112387424107648	2.14028056112224\\
-0.104208416833667	2.13301329770614\\
-0.0945324789276519	2.12424849699399\\
-0.0881763527054109	2.11849976935194\\
-0.0770174803552489	2.10821643286573\\
-0.0721442885771544	2.10373136403057\\
-0.0598240675612109	2.09218436873747\\
-0.0561122244488979	2.08870905833842\\
-0.042935305942095	2.07615230460922\\
-0.0400801603206413	2.07343363204895\\
-0.0263355695258069	2.06012024048096\\
-0.0240480961923848	2.05790566911636\\
-0.0100104349689827	2.04408817635271\\
-0.00801603206412826	2.04212555842685\\
0.00605341413827421	2.02805611222445\\
0.00801603206412782	2.02609349429859\\
0.0218682737844108	2.01202404809619\\
0.0240480961923843	2.0098094767316\\
0.0374454978510013	1.99599198396794\\
0.0400801603206409	1.99327331140766\\
0.0527955690962666	1.97995991983968\\
0.0561122244488974	1.97648460944062\\
0.0679281635703945	1.96392785571142\\
0.0721442885771539	1.95944278687626\\
0.082852209135698	1.94789579158317\\
0.0881763527054105	1.94214706394112\\
0.0975759386864971	1.93186372745491\\
0.104208416833667	1.9245964640388\\
0.112106938595429	1.91583166332665\\
0.120240480961924	1.90678981249196\\
0.126452192853365	1.8997995991984\\
0.13627254509018	1.88872573502771\\
0.14061812331798	1.88376753507014\\
0.152304609218437	1.87040265600366\\
0.154610626440267	1.86773547094188\\
0.168336673346693	1.85181879637159\\
0.16843510679813	1.85170340681363\\
0.182094263918885	1.83567134268537\\
0.18436873747495	1.83299436395044\\
0.195597114389196	1.81963927855711\\
0.200400801603206	1.81390810034557\\
0.208948104277145	1.80360721442886\\
0.216432865731463	1.79455667242361\\
0.222151133760788	1.7875751503006\\
0.232464929859719	1.77493748328741\\
0.235209729534101	1.77154308617235\\
0.248127367485314	1.75551102204409\\
0.248496993987976	1.7550514411902\\
0.260910749817074	1.73947895791583\\
0.264529058116232	1.73492125251862\\
0.273561692861503	1.72344689378758\\
0.280561122244489	1.71451642527965\\
0.286082555369444	1.70741482965932\\
0.296593186372745	1.69383350377715\\
0.298475405129103	1.69138276553106\\
0.31074681341069	1.67535070140281\\
0.312625250501002	1.672888260013\\
0.322901385421872	1.65931863727455\\
0.328657314629258	1.65167876797935\\
0.334936639563524	1.64328657314629\\
0.344689378757515	1.63018168417632\\
0.346853715889065	1.62725450901804\\
0.358661223665631	1.61122244488978\\
0.360721442885771	1.60841427574096\\
0.370363623606131	1.59519038076152\\
0.376753507014028	1.58637493163314\\
0.381954431452667	1.57915831663327\\
0.392785571142285	1.56403642365584\\
0.393433997955205	1.56312625250501\\
0.404822061190901	1.54709418837675\\
0.408817635270541	1.54143613904278\\
0.416106673164823	1.5310621242485\\
0.424849699398798	1.51853530825889\\
0.427284817757179	1.51503006012024\\
0.438370195933025	1.49899799599198\\
0.440881763527054	1.49534823050935\\
0.44936577793113	1.48296593186373\\
0.456913827655311	1.47187009297541\\
0.460258527737727	1.46693386773547\\
0.47105948890644	1.45090180360721\\
0.472945891783567	1.44808932893996\\
0.481780563993476	1.43486973947896\\
0.488977955911824	1.42401669850271\\
0.492401403989198	1.4188376753507\\
0.502935212529043	1.40280561122244\\
0.50501002004008	1.39963259419958\\
0.513393889713681	1.38677354709419\\
0.521042084168337	1.37494598588992\\
0.523753976742535	1.37074148296593\\
0.534036970397823	1.35470941883768\\
0.537074148296593	1.34994602829754\\
0.54424314245709	1.33867735470942\\
0.55310621242485	1.32462354673062\\
0.554351490046198	1.32264529058116\\
0.564399151523312	1.30661322645291\\
0.569138276553106	1.29899224575859\\
0.574360665418839	1.29058116232465\\
0.584232230108001	1.27454909819639\\
0.585170340681363	1.27301910641964\\
0.594051326219487	1.25851703406814\\
0.601202404809619	1.24672820700097\\
0.603774129129969	1.24248496993988\\
0.613432791996306	1.22645290581162\\
0.617234468937876	1.22009677958938\\
0.62301858987926	1.21042084168337\\
0.632513854057083	1.19438877755511\\
0.633266533066132	1.19311243712257\\
0.641966158922827	1.17835671342685\\
0.649298597194389	1.16579139866338\\
0.651321859979967	1.1623246492986\\
0.660624815805226	1.14629258517034\\
0.665330661322646	1.13811357789636\\
0.669850938713261	1.13026052104208\\
0.679002064889276	1.11422845691383\\
0.681362725450902	1.11006688618478\\
0.688101307851878	1.09819639278557\\
0.69710495342446	1.08216432865731\\
0.697394789579159	1.08164605316432\\
0.706079738533738	1.06613226452906\\
0.713426853707415	1.05285896278679\\
0.714956057633846	1.0501002004008\\
0.723792570976108	1.03406813627254\\
0.729458917835672	1.0236844437265\\
0.732546518759912	1.01803607214429\\
0.741245729576007	1.00200400801603\\
0.745490981963928	0.994112426718699\\
0.749879009420214	0.985971943887775\\
0.75844473699409	0.969939879759519\\
0.761523046092185	0.964134766588891\\
0.766958806572806	0.953907815631262\\
0.775394727267325	0.937875751503006\\
0.777555110220441	0.933742479012859\\
0.783790805290208	0.921843687374749\\
0.79210045799249	0.905811623246493\\
0.793587174348698	0.902925705513879\\
0.800379530336358	0.889779559118236\\
0.808566321619408	0.87374749498998\\
0.809619238476954	0.871673675404656\\
0.816729146876445	0.857715430861724\\
0.824796355889865	0.841683366733467\\
0.825651302605211	0.839974664477626\\
0.832843470353536	0.825651302605211\\
0.840794253455332	0.809619238476954\\
0.841683366733467	0.807815950229083\\
0.848725975563253	0.793587174348698\\
0.856563370704004	0.777555110220441\\
0.857715430861724	0.775183763382832\\
0.864379804955127	0.761523046092185\\
0.872106735825037	0.745490981963928\\
0.87374749498998	0.742063235457539\\
0.879807776186825	0.729458917835672\\
0.887427056135494	0.713426853707415\\
0.889779559118236	0.708438342098458\\
0.895012388955042	0.697394789579159\\
0.902526724693145	0.681362725450902\\
0.905811623246493	0.674291841868642\\
0.909995831124626	0.665330661322646\\
0.917407826216055	0.649298597194389\\
0.921843687374749	0.639605210166722\\
0.924759984175293	0.633266533066132\\
0.932072142327726	0.617234468937876\\
0.937875751503006	0.604358567907728\\
0.939306427983218	0.601202404809619\\
0.946521156144502	0.585170340681363\\
0.953640079044779	0.569138276553106\\
0.953907815631262	0.568531625962186\\
0.960756056219921	0.55310621242485\\
0.967780754132099	0.537074148296593\\
0.969939879759519	0.532105663628523\\
0.974777739858769	0.521042084168337\\
0.981710376076964	0.50501002004008\\
0.985971943887775	0.495049758631863\\
0.988586815811709	0.488977955911824\\
0.99542946636084	0.472945891783567\\
1.00200400801603	0.45733731060375\\
1.00218360635947	0.456913827655311\\
1.00893826228546	0.440881763527054\\
1.01560260849886	0.424849699398798\\
1.01803607214429	0.418942565866309\\
1.02223671829137	0.408817635270541\\
1.02881559873643	0.392785571142285\\
1.03406813627254	0.379829674451507\\
1.03532450668279	0.376753507014028\\
1.04181969582964	0.360721442885771\\
1.04822789550428	0.344689378757515\\
1.0501002004008	0.339963767795497\\
1.05461421190746	0.328657314629258\\
1.06094089149501	0.312625250501002\\
1.06613226452906	0.299306270210088\\
1.06719817822434	0.296593186372745\\
1.07344490733363	0.280561122244489\\
1.07960754402498	0.264529058116232\\
1.08216432865731	0.257811391612406\\
1.08573861773676	0.248496993987976\\
1.09182313616956	0.232464929859719\\
1.09782580665023	0.216432865731463\\
1.09819639278557	0.215435169716469\\
1.10382818894802	0.200400801603206\\
1.10975436746823	0.18436873747495\\
1.11422845691383	0.172116267907533\\
1.11562073759942	0.168336673346693\\
1.12147167987081	0.152304609218437\\
1.12724301738029	0.13627254509018\\
1.13026052104208	0.127797365544665\\
1.13297542122262	0.120240480961924\\
1.13867283245662	0.104208416833667\\
1.14429246627308	0.0881763527054105\\
1.14629258517034	0.0824110642442843\\
1.14988753346395	0.0721442885771539\\
1.15543435708166	0.0561122244488974\\
1.16090507180902	0.0400801603206409\\
1.1623246492986	0.0358778763363548\\
1.16635911296884	0.0240480961923843\\
1.17175794302227	0.00801603206412782\\
1.17708217805388	-0.00801603206412826\\
1.17835671342685	-0.0118935369522394\\
1.1823911770971	-0.0240480961923848\\
1.1876442649766	-0.0400801603206413\\
1.19282412066439	-0.0561122244488979\\
1.19438877755511	-0.0610080323132715\\
1.19798372584872	-0.0721442885771544\\
1.20309298566784	-0.0881763527054109\\
1.20813022818325	-0.104208416833667\\
1.21042084168337	-0.111585955694638\\
1.21313574186391	-0.120240480961924\\
1.21810275454298	-0.13627254509018\\
1.22299881944837	-0.152304609218437\\
1.22645290581162	-0.163765482258007\\
1.22784518649722	-0.168336673346694\\
1.23267120256016	-0.18436873747495\\
1.23742719710257	-0.200400801603207\\
1.24211438380454	-0.216432865731463\\
1.24248496993988	-0.217714857437629\\
1.24679493303642	-0.232464929859719\\
1.25141163716841	-0.248496993987976\\
1.2559602494358	-0.264529058116232\\
1.25851703406814	-0.273662714996768\\
1.26046950850622	-0.280561122244489\\
1.26494737463194	-0.296593186372745\\
1.26935772516235	-0.312625250501002\\
1.27370159972485	-0.328657314629258\\
1.27454909819639	-0.331825883481582\\
1.2780285849571	-0.344689378757515\\
1.28230065775349	-0.360721442885771\\
1.28650662502303	-0.376753507014028\\
1.29058116232465	-0.392528378245189\\
1.29064836143032	-0.392785571142285\\
1.29478180847173	-0.408817635270541\\
1.29884938139001	-0.424849699398798\\
1.30285195075063	-0.440881763527054\\
1.30661322645291	-0.456191316685715\\
1.30679282479634	-0.456913827655311\\
1.31072117881957	-0.472945891783567\\
1.31458455512255	-0.488977955911824\\
1.31838372277035	-0.50501002004008\\
1.32211941541806	-0.521042084168337\\
1.32264529058116	-0.523334349434413\\
1.32583504765005	-0.537074148296593\\
1.32949353116982	-0.55310621242485\\
1.33308829272707	-0.569138276553106\\
1.33661996515266	-0.585170340681363\\
1.33867735470942	-0.594673495785783\\
1.34010803118963	-0.601202404809619\\
1.34355990970964	-0.617234468937876\\
1.34694824139766	-0.633266533066132\\
1.35027355767898	-0.649298597194389\\
1.35353635514788	-0.665330661322646\\
1.35470941883768	-0.671204009319294\\
1.35676341650596	-0.681362725450902\\
1.35994224867448	-0.697394789579158\\
1.36305782068158	-0.713426853707415\\
1.36611052617763	-0.729458917835671\\
1.36910072380099	-0.745490981963928\\
1.37074148296593	-0.754474507343357\\
1.37204493428342	-0.761523046092184\\
1.37494628488202	-0.777555110220441\\
1.37778409179572	-0.793587174348697\\
1.380558607853	-0.809619238476954\\
1.38327005023437	-0.82565130260521\\
1.38591860037884	-0.841683366733467\\
1.38677354709419	-0.846986325458802\\
1.38852514086549	-0.857715430861723\\
1.39107767773152	-0.87374749498998\\
1.39356590308185	-0.889779559118236\\
1.39598988725695	-0.905811623246493\\
1.39834966361338	-0.92184368737475\\
1.40064522826933	-0.937875751503006\\
1.40280561122244	-0.953398857694159\\
1.40287734648678	-0.953907815631263\\
1.40506851430965	-0.969939879759519\\
1.40719363867873	-0.985971943887776\\
1.40925259695101	-1.00200400801603\\
1.41124522734124	-1.01803607214429\\
1.41317132849114	-1.03406813627255\\
1.41503065900035	-1.0501002004008\\
1.4168229369183	-1.06613226452906\\
1.418547839196	-1.08216432865731\\
1.4188376753507	-1.0849741854515\\
1.42021886398182	-1.09819639278557\\
1.42182337322298	-1.11422845691383\\
1.42335795274132	-1.13026052104208\\
1.42482214355233	-1.14629258517034\\
1.42621544244241	-1.1623246492986\\
1.4275373012074	-1.17835671342685\\
1.42878712584424	-1.19438877755511\\
1.4299642756944	-1.21042084168337\\
1.43106806253739	-1.22645290581162\\
1.43209774963311	-1.24248496993988\\
1.43305255071117	-1.25851703406814\\
1.4339316289056	-1.27454909819639\\
1.43473409563297	-1.29058116232465\\
1.43486973947896	-1.29359165594453\\
1.435463735431	-1.30661322645291\\
1.43611501710031	-1.32264529058116\\
1.43668581376201	-1.33867735470942\\
1.43717505973574	-1.35470941883768\\
1.43758163205316	-1.37074148296593\\
1.43790434902837	-1.38677354709419\\
1.43814196875914	-1.40280561122244\\
1.43829318755633	-1.4188376753507\\
1.43835663829887	-1.43486973947896\\
1.43833088871135	-1.45090180360721\\
1.43821443956137	-1.46693386773547\\
1.43800572277337	-1.48296593186373\\
1.43770309945568	-1.49899799599198\\
1.43730485783734	-1.51503006012024\\
1.43680921111101	-1.5310621242485\\
1.43621429517806	-1.54709418837675\\
1.43551816629188	-1.56312625250501\\
1.43486973947896	-1.57614782301339\\
1.43471943107953	-1.57915831663327\\
1.43381826014665	-1.59519038076152\\
1.43280952025882	-1.61122244488978\\
1.43169085204059	-1.62725450901804\\
1.4304598001939	-1.64328657314629\\
1.42911381027157	-1.65931863727455\\
1.4276502253059	-1.67535070140281\\
1.42606628228583	-1.69138276553106\\
1.42435910847566	-1.70741482965932\\
1.4225257175681	-1.72344689378758\\
1.42056300566399	-1.73947895791583\\
1.4188376753507	-1.75270116524991\\
1.41846804884804	-1.75551102204409\\
1.41623805524873	-1.77154308617234\\
1.41386758907841	-1.7875751503006\\
1.41135291389638	-1.80360721442886\\
1.40869014715433	-1.81963927855711\\
1.40587525464174	-1.83567134268537\\
1.40290404467388	-1.85170340681363\\
1.40280561122244	-1.85221236475073\\
1.39976673853409	-1.86773547094188\\
1.39646204917176	-1.88376753507014\\
1.39298525898563	-1.8997995991984\\
1.38933130901527	-1.91583166332665\\
1.38677354709419	-1.92656076872958\\
1.38549017444567	-1.93186372745491\\
1.38144940352448	-1.94789579158317\\
1.37721190834076	-1.96392785571142\\
1.37277155563299	-1.97995991983968\\
1.37074148296593	-1.98700845858851\\
1.36810682049629	-1.99599198396794\\
1.36321095195326	-2.01202404809619\\
1.35808788971171	-2.02805611222445\\
1.35470941883768	-2.03821482835606\\
1.35271501593282	-2.04408817635271\\
1.34706954954247	-2.06012024048096\\
1.34116778578055	-2.07615230460922\\
1.33867735470942	-2.08268121363305\\
1.33496551159711	-2.09218436873747\\
1.3284588477304	-2.10821643286573\\
1.32264529058116	-2.12195623172791\\
1.32164969012635	-2.12424849699399\\
1.31446628330718	-2.14028056112224\\
1.30696119426195	-2.1563126252505\\
1.30661322645291	-2.1570351362201\\
1.29902049060337	-2.17234468937876\\
1.29071804569593	-2.18837675350701\\
1.29058116232465	-2.18863394640411\\
1.28191145320043	-2.20440881763527\\
1.27454909819639	-2.2172723129112\\
1.27267209655242	-2.22044088176353\\
1.26287839719742	-2.23647294589178\\
1.25851703406814	-2.2433713531395\\
1.25252169552617	-2.25250501002004\\
1.24248496993988	-2.26725508244213\\
1.24157692121844	-2.2685370741483\\
1.2298926066466	-2.28456913827655\\
1.22645290581162	-2.28914032936524\\
1.2174438451519	-2.30060120240481\\
1.21042084168337	-2.3092557276721\\
1.20414790670024	-2.31663326653307\\
1.19438877755511	-2.32776952279695\\
1.18987869651441	-2.33266533066132\\
1.17835671342685	-2.34481988990147\\
1.17447920853874	-2.34869739478958\\
1.1623246492986	-2.36052717493355\\
1.15775251694412	-2.36472945891784\\
1.14629258517034	-2.37499623458497\\
1.13944940232465	-2.38076152304609\\
1.13026052104208	-2.38831840762883\\
1.11925105511328	-2.39679358717435\\
1.11422845691383	-2.40057318173519\\
1.09819639278557	-2.41182795528761\\
1.09667379717982	-2.41282565130261\\
1.08216432865731	-2.42214004892704\\
1.07080768800794	-2.42885771543086\\
1.06613226452906	-2.4315707992682\\
1.0501002004008	-2.4401641685971\\
1.0404549230817	-2.44488977955912\\
1.03406813627254	-2.4479659469966\\
1.01803607214429	-2.45501471015489\\
1.00309160510989	-2.46092184368737\\
1.00200400801603	-2.46134532663581\\
0.985971943887775	-2.46699364640741\\
0.969939879759519	-2.47198542314756\\
0.953907815631262	-2.47634725722471\\
0.951348053022254	-2.47695390781563\\
0.937875751503006	-2.48011007091374\\
0.921843687374749	-2.48329258491622\\
0.905811623246493	-2.48591508836163\\
0.889779559118236	-2.48799746033493\\
0.87374749498998	-2.4895582254375\\
0.857715430861724	-2.49061462510628\\
0.841683366733467	-2.49118268369602\\
0.825651302605211	-2.49127726968805\\
0.809619238476954	-2.49091215235856\\
0.793587174348698	-2.49010005421127\\
0.777555110220441	-2.48885269945374\\
0.761523046092185	-2.48718085877326\\
0.745490981963928	-2.48509439064656\\
0.729458917835672	-2.48260227939784\\
0.713426853707415	-2.47971267020162\\
0.699954552188167	-2.47695390781563\\
0.697394789579159	-2.47643563232263\\
0.681362725450902	-2.47279233708659\\
0.665330661322646	-2.46877490054165\\
0.649298597194389	-2.46438859305215\\
0.637613172394769	-2.46092184368737\\
0.633266533066132	-2.45964550325484\\
0.617234468937876	-2.45456571746513\\
0.601202404809619	-2.44913301662021\\
0.589448840589471	-2.44488977955912\\
0.585170340681363	-2.44335978778237\\
0.569138276553106	-2.43726879886481\\
0.55310621242485	-2.43083597158032\\
0.548430788945969	-2.42885771543086\\
0.537074148296593	-2.42409432489073\\
0.521042084168337	-2.41703015422659\\
0.511937632205333	-2.41282565130261\\
0.50501002004008	-2.40965263427974\\
0.488977955911824	-2.40197261032635\\
0.47861633688666	-2.39679358717435\\
0.472945891783567	-2.39398111250709\\
0.456913827655311	-2.38569774828604\\
0.447724946372745	-2.38076152304609\\
0.440881763527054	-2.37711175756346\\
0.424849699398798	-2.36823470705649\\
0.418742219050504	-2.36472945891784\\
0.408817635270541	-2.35907140958386\\
0.392785571142285	-2.34960756594041\\
0.391293580145418	-2.34869739478958\\
0.376753507014028	-2.33988194566119\\
0.365231523926469	-2.33266533066132\\
0.360721442885771	-2.3298571615125\\
0.344689378757515	-2.31956044169135\\
0.340265649230688	-2.31663326653307\\
0.328657314629258	-2.30899339723786\\
0.316276698443884	-2.30060120240481\\
0.312625250501002	-2.298138761015\\
0.296593186372745	-2.28701987652264\\
0.293153485537765	-2.28456913827655\\
0.280561122244489	-2.27563866976863\\
0.270812795449883	-2.2685370741483\\
0.264529058116232	-2.26397936875109\\
0.249116841161576	-2.25250501002004\\
0.248496993987976	-2.25204542916615\\
0.232464929859719	-2.23986734300685\\
0.228103566730433	-2.23647294589178\\
0.216432865731463	-2.22742240388654\\
0.20764169030169	-2.22044088176353\\
0.200400801603206	-2.21470970355198\\
0.187683817264239	-2.20440881763527\\
0.18436873747495	-2.20173183890034\\
0.168336673346693	-2.18849214306497\\
0.168199789975409	-2.18837675350701\\
0.152304609218437	-2.17501187444054\\
0.149198527430876	-2.17234468937876\\
0.13627254509018	-2.16127082520807\\
0.13060223360748	-2.1563126252505\\
0.120240480961924	-2.14727077441581\\
0.112387424107648	-2.14028056112224\\
0.104208416833667	-2.13301329770613\\
0.0945324789276524	-2.12424849699399\\
0.0881763527054105	-2.11849976935194\\
0.0770174803552489	-2.10821643286573\\
0.0721442885771539	-2.10373136403057\\
0.0598240675612114	-2.09218436873747\\
0.0561122244488974	-2.08870905833842\\
0.0429353059420955	-2.07615230460922\\
0.0400801603206409	-2.07343363204895\\
0.0263355695258069	-2.06012024048096\\
0.0240480961923843	-2.05790566911636\\
0.0100104349689827	-2.04408817635271\\
0.00801603206412782	-2.04212555842685\\
-0.0060534141382742	-2.02805611222445\\
-0.00801603206412826	-2.02609349429859\\
-0.0218682737844103	-2.01202404809619\\
-0.0240480961923848	-2.00980947673159\\
-0.0374454978510004	-1.99599198396794\\
-0.0400801603206413	-1.99327331140766\\
-0.0527955690962666	-1.97995991983968\\
-0.0561122244488979	-1.97648460944062\\
-0.067928163570394	-1.96392785571142\\
-0.0721442885771544	-1.95944278687626\\
-0.082852209135697	-1.94789579158317\\
-0.0881763527054109	-1.94214706394111\\
-0.0975759386864961	-1.93186372745491\\
-0.104208416833667	-1.9245964640388\\
-0.112106938595428	-1.91583166332665\\
-0.120240480961924	-1.90678981249196\\
-0.126452192853364	-1.8997995991984\\
-0.13627254509018	-1.8887257350277\\
-0.140618123317979	-1.88376753507014\\
-0.152304609218437	-1.87040265600366\\
-0.154610626440267	-1.86773547094188\\
-0.168336673346694	-1.85181879637159\\
-0.168435106798128	-1.85170340681363\\
-0.182094263918885	-1.83567134268537\\
-0.18436873747495	-1.83299436395044\\
-0.195597114389196	-1.81963927855711\\
-0.200400801603207	-1.81390810034557\\
-0.208948104277145	-1.80360721442886\\
-0.216432865731463	-1.79455667242361\\
-0.222151133760787	-1.7875751503006\\
-0.232464929859719	-1.77493748328741\\
-0.235209729534102	-1.77154308617234\\
-0.248127367485314	-1.75551102204409\\
-0.248496993987976	-1.7550514411902\\
-0.260910749817074	-1.73947895791583\\
-0.264529058116232	-1.73492125251862\\
-0.273561692861503	-1.72344689378758\\
-0.280561122244489	-1.71451642527965\\
-0.286082555369444	-1.70741482965932\\
-0.296593186372745	-1.69383350377715\\
-0.298475405129103	-1.69138276553106\\
-0.31074681341069	-1.67535070140281\\
-0.312625250501002	-1.672888260013\\
-0.322901385421872	-1.65931863727455\\
-0.328657314629258	-1.65167876797935\\
-0.334936639563524	-1.64328657314629\\
-0.344689378757515	-1.63018168417632\\
-0.346853715889066	-1.62725450901804\\
-0.358661223665631	-1.61122244488978\\
-0.360721442885771	-1.60841427574096\\
-0.370363623606131	-1.59519038076152\\
-0.376753507014028	-1.58637493163314\\
-0.381954431452668	-1.57915831663327\\
-0.392785571142285	-1.56403642365584\\
-0.393433997955205	-1.56312625250501\\
-0.404822061190901	-1.54709418837675\\
-0.408817635270541	-1.54143613904278\\
-0.416106673164823	-1.5310621242485\\
-0.424849699398798	-1.51853530825889\\
-0.427284817757179	-1.51503006012024\\
-0.438370195933025	-1.49899799599198\\
-0.440881763527054	-1.49534823050935\\
-0.44936577793113	-1.48296593186373\\
-0.456913827655311	-1.47187009297541\\
-0.460258527737727	-1.46693386773547\\
-0.47105948890644	-1.45090180360721\\
-0.472945891783567	-1.44808932893996\\
-0.481780563993476	-1.43486973947896\\
-0.488977955911824	-1.42401669850271\\
-0.492401403989197	-1.4188376753507\\
-0.502935212529043	-1.40280561122244\\
-0.50501002004008	-1.39963259419958\\
-0.513393889713682	-1.38677354709419\\
-0.521042084168337	-1.37494598588992\\
-0.523753976742535	-1.37074148296593\\
-0.534036970397823	-1.35470941883768\\
-0.537074148296593	-1.34994602829754\\
-0.54424314245709	-1.33867735470942\\
-0.55310621242485	-1.32462354673062\\
-0.554351490046198	-1.32264529058116\\
-0.564399151523312	-1.30661322645291\\
-0.569138276553106	-1.29899224575859\\
-0.57436066541884	-1.29058116232465\\
-0.584232230108001	-1.27454909819639\\
-0.585170340681363	-1.27301910641964\\
-0.594051326219487	-1.25851703406814\\
-0.601202404809619	-1.24672820700097\\
-0.603774129129969	-1.24248496993988\\
-0.613432791996307	-1.22645290581162\\
-0.617234468937876	-1.22009677958938\\
-0.62301858987926	-1.21042084168337\\
-0.632513854057083	-1.19438877755511\\
-0.633266533066132	-1.19311243712257\\
-0.641966158922827	-1.17835671342685\\
-0.649298597194389	-1.16579139866338\\
-0.651321859979967	-1.1623246492986\\
-0.660624815805227	-1.14629258517034\\
-0.665330661322646	-1.13811357789636\\
-0.669850938713261	-1.13026052104208\\
-0.679002064889275	-1.11422845691383\\
-0.681362725450902	-1.11006688618478\\
-0.688101307851878	-1.09819639278557\\
-0.69710495342446	-1.08216432865731\\
-0.697394789579158	-1.08164605316432\\
-0.706079738533738	-1.06613226452906\\
-0.713426853707415	-1.05285896278679\\
-0.714956057633846	-1.0501002004008\\
-0.723792570976108	-1.03406813627255\\
-0.729458917835671	-1.0236844437265\\
-0.732546518759912	-1.01803607214429\\
-0.741245729576007	-1.00200400801603\\
-0.745490981963928	-0.9941124267187\\
-0.749879009420214	-0.985971943887776\\
-0.75844473699409	-0.969939879759519\\
-0.761523046092184	-0.964134766588893\\
-0.766958806572806	-0.953907815631263\\
-0.775394727267324	-0.937875751503006\\
-0.777555110220441	-0.93374247901286\\
-0.783790805290207	-0.92184368737475\\
-0.792100457992489	-0.905811623246493\\
-0.793587174348697	-0.90292570551388\\
-0.800379530336358	-0.889779559118236\\
-0.808566321619409	-0.87374749498998\\
-0.809619238476954	-0.871673675404657\\
-0.816729146876445	-0.857715430861723\\
-0.824796355889865	-0.841683366733467\\
-0.82565130260521	-0.839974664477627\\
-0.832843470353537	-0.82565130260521\\
-0.840794253455332	-0.809619238476954\\
-0.841683366733467	-0.807815950229083\\
-0.848725975563253	-0.793587174348697\\
-0.856563370704004	-0.777555110220441\\
-0.857715430861723	-0.775183763382833\\
-0.864379804955127	-0.761523046092184\\
-0.872106735825038	-0.745490981963928\\
-0.87374749498998	-0.74206323545754\\
-0.879807776186825	-0.729458917835671\\
-0.887427056135495	-0.713426853707415\\
-0.889779559118236	-0.708438342098458\\
-0.895012388955043	-0.697394789579158\\
-0.902526724693145	-0.681362725450902\\
-0.905811623246493	-0.67429184186864\\
-0.909995831124627	-0.665330661322646\\
-0.917407826216054	-0.649298597194389\\
-0.92184368737475	-0.639605210166721\\
-0.924759984175294	-0.633266533066132\\
-0.932072142327726	-0.617234468937876\\
-0.937875751503006	-0.604358567907728\\
-0.939306427983218	-0.601202404809619\\
-0.946521156144503	-0.585170340681363\\
-0.953640079044779	-0.569138276553106\\
-0.953907815631263	-0.568531625962185\\
-0.960756056219921	-0.55310621242485\\
-0.967780754132098	-0.537074148296593\\
-0.969939879759519	-0.532105663628523\\
-0.97477773985877	-0.521042084168337\\
-0.981710376076964	-0.50501002004008\\
-0.985971943887776	-0.495049758631863\\
-0.988586815811709	-0.488977955911824\\
-0.99542946636084	-0.472945891783567\\
-1.00200400801603	-0.45733731060375\\
-1.00218360635947	-0.456913827655311\\
-1.00893826228546	-0.440881763527054\\
-1.01560260849885	-0.424849699398798\\
-1.01803607214429	-0.418942565866308\\
-1.02223671829137	-0.408817635270541\\
-1.02881559873643	-0.392785571142285\\
-1.03406813627255	-0.379829674451506\\
-1.03532450668279	-0.376753507014028\\
-1.04181969582964	-0.360721442885771\\
-1.04822789550428	-0.344689378757515\\
-1.0501002004008	-0.339963767795496\\
-1.05461421190746	-0.328657314629258\\
-1.06094089149501	-0.312625250501002\\
-1.06613226452906	-0.299306270210088\\
-1.06719817822434	-0.296593186372745\\
-1.07344490733363	-0.280561122244489\\
-1.07960754402498	-0.264529058116232\\
-1.08216432865731	-0.257811391612405\\
-1.08573861773676	-0.248496993987976\\
-1.09182313616956	-0.232464929859719\\
-1.09782580665023	-0.216432865731463\\
-1.09819639278557	-0.215435169716468\\
-1.10382818894802	-0.200400801603207\\
-1.10975436746823	-0.18436873747495\\
-1.11422845691383	-0.172116267907533\\
-1.11562073759942	-0.168336673346694\\
-1.12147167987081	-0.152304609218437\\
-1.12724301738029	-0.13627254509018\\
-1.13026052104208	-0.127797365544665\\
-1.13297542122262	-0.120240480961924\\
-1.13867283245662	-0.104208416833667\\
-1.14429246627308	-0.0881763527054109\\
-1.14629258517034	-0.0824110642442843\\
-1.14988753346395	-0.0721442885771544\\
-1.15543435708166	-0.0561122244488979\\
-1.16090507180902	-0.0400801603206413\\
-1.1623246492986	-0.0358778763363548\\
-1.16635911296884	-0.0240480961923848\\
-1.17175794302227	-0.00801603206412826\\
-1.17708217805388	0.00801603206412782\\
-1.17835671342685	0.0118935369522394\\
-1.1823911770971	0.0240480961923843\\
-1.1876442649766	0.0400801603206409\\
-1.19282412066439	0.0561122244488974\\
-1.19438877755511	0.0610080323132719\\
-1.19798372584872	0.0721442885771539\\
-1.20309298566784	0.0881763527054105\\
-1.20813022818325	0.104208416833667\\
-1.21042084168337	0.111585955694638\\
-1.21313574186391	0.120240480961924\\
-1.21810275454298	0.13627254509018\\
-1.22299881944837	0.152304609218437\\
-1.22645290581162	0.163765482258007\\
-1.22784518649722	0.168336673346693\\
-1.23267120256016	0.18436873747495\\
-1.23742719710257	0.200400801603206\\
-1.24211438380454	0.216432865731463\\
-1.24248496993988	0.217714857437629\\
-1.24679493303642	0.232464929859719\\
-1.25141163716841	0.248496993987976\\
-1.2559602494358	0.264529058116232\\
-1.25851703406814	0.273662714996768\\
-1.26046950850622	0.280561122244489\\
-1.26494737463194	0.296593186372745\\
-1.26935772516235	0.312625250501002\\
-1.27370159972485	0.328657314629258\\
-1.27454909819639	0.331825883481581\\
-1.2780285849571	0.344689378757515\\
-1.28230065775349	0.360721442885771\\
-1.28650662502303	0.376753507014028\\
-1.29058116232465	0.392528378245189\\
-1.29064836143032	0.392785571142285\\
-1.29478180847173	0.408817635270541\\
-1.29884938139001	0.424849699398798\\
-1.30285195075063	0.440881763527054\\
-1.30661322645291	0.456191316685715\\
-1.30679282479634	0.456913827655311\\
-1.31072117881957	0.472945891783567\\
-1.31458455512255	0.488977955911824\\
-1.31838372277035	0.50501002004008\\
-1.32211941541806	0.521042084168337\\
-1.32264529058116	0.523334349434413\\
-1.32583504765005	0.537074148296593\\
-1.32949353116982	0.55310621242485\\
-1.33308829272707	0.569138276553106\\
-1.33661996515266	0.585170340681363\\
-1.33867735470942	0.594673495785783\\
-1.34010803118963	0.601202404809619\\
-1.34355990970964	0.617234468937876\\
-1.34694824139766	0.633266533066132\\
-1.35027355767898	0.649298597194389\\
-1.35353635514788	0.665330661322646\\
-1.35470941883768	0.671204009319294\\
-1.35676341650596	0.681362725450902\\
-1.35994224867448	0.697394789579159\\
-1.36305782068158	0.713426853707415\\
-1.36611052617763	0.729458917835672\\
-1.36910072380099	0.745490981963928\\
-1.37074148296593	0.754474507343357\\
-1.37204493428342	0.761523046092185\\
-1.37494628488202	0.777555110220441\\
-1.37778409179572	0.793587174348698\\
-1.380558607853	0.809619238476954\\
-1.38327005023437	0.825651302605211\\
-1.38591860037884	0.841683366733467\\
-1.38677354709419	0.846986325458802\\
-1.38852514086549	0.857715430861724\\
-1.39107767773152	0.87374749498998\\
-1.39356590308185	0.889779559118236\\
-1.39598988725695	0.905811623246493\\
-1.39834966361338	0.921843687374749\\
-1.40064522826933	0.937875751503006\\
-1.40280561122244	0.953398857694158\\
-1.40287734648678	0.953907815631262\\
-1.40506851430965	0.969939879759519\\
-1.40719363867873	0.985971943887775\\
-1.40925259695101	1.00200400801603\\
-1.41124522734124	1.01803607214429\\
-1.41317132849114	1.03406813627254\\
-1.41503065900035	1.0501002004008\\
-1.4168229369183	1.06613226452906\\
-1.418547839196	1.08216432865731\\
-1.4188376753507	1.0849741854515\\
-1.42021886398182	1.09819639278557\\
-1.42182337322298	1.11422845691383\\
-1.42335795274132	1.13026052104208\\
-1.42482214355234	1.14629258517034\\
-1.42621544244241	1.1623246492986\\
-1.4275373012074	1.17835671342685\\
-1.42878712584424	1.19438877755511\\
-1.4299642756944	1.21042084168337\\
-1.43106806253739	1.22645290581162\\
-1.43209774963311	1.24248496993988\\
-1.43305255071117	1.25851703406814\\
-1.4339316289056	1.27454909819639\\
-1.43473409563298	1.29058116232465\\
-1.43486973947896	1.29359165594453\\
-1.435463735431	1.30661322645291\\
-1.43611501710031	1.32264529058116\\
-1.43668581376201	1.33867735470942\\
-1.43717505973574	1.35470941883768\\
-1.43758163205316	1.37074148296593\\
-1.43790434902837	1.38677354709419\\
-1.43814196875914	1.40280561122244\\
-1.43829318755633	1.4188376753507\\
-1.43835663829887	1.43486973947896\\
-1.43833088871135	1.45090180360721\\
-1.43821443956137	1.46693386773547\\
-1.43800572277337	1.48296593186373\\
-1.43770309945568	1.49899799599198\\
-1.43730485783734	1.51503006012024\\
-1.43680921111101	1.5310621242485\\
-1.43621429517806	1.54709418837675\\
-1.43551816629188	1.56312625250501\\
-1.43486973947896	1.57614782301339\\
}--cycle;


\addplot[area legend,solid,fill=mycolor6,draw=black,forget plot]
table[row sep=crcr] {%
x	y\\
-1.25851703406814	1.5366182780483\\
-1.2571939945978	1.54709418837675\\
-1.25502463969909	1.56312625250501\\
-1.25269977496859	1.57915831663327\\
-1.25021500021237	1.59519038076152\\
-1.24756573216555	1.61122244488978\\
-1.24474719693108	1.62725450901804\\
-1.24248496993988	1.63940453353308\\
-1.24174975232708	1.64328657314629\\
-1.23855607385912	1.65931863727455\\
-1.23517430122925	1.67535070140281\\
-1.23159857003181	1.69138276553106\\
-1.22782276802783	1.70741482965932\\
-1.22645290581162	1.71297294500933\\
-1.22381713524383	1.72344689378758\\
-1.21958196768543	1.73947895791583\\
-1.21512182688343	1.75551102204409\\
-1.21042914983683	1.77154308617235\\
-1.21042084168337	1.77157044071457\\
-1.20544003002271	1.7875751503006\\
-1.2001956701131	1.80360721442886\\
-1.19468712586423	1.81963927855711\\
-1.19438877755511	1.82047757060002\\
-1.18883032863662	1.83567134268537\\
-1.18267763612223	1.85170340681363\\
-1.17835671342685	1.86248676464928\\
-1.17618880789234	1.86773547094188\\
-1.16930974881892	1.88376753507014\\
-1.1623246492986	1.89929379840314\\
-1.16208939750908	1.8997995991984\\
-1.15438504264813	1.91583166332665\\
-1.14630279283667	1.93186372745491\\
-1.14629258517034	1.93188336299896\\
-1.13765350560881	1.94789579158317\\
-1.13026052104208	1.96099723421974\\
-1.12853963234662	1.96392785571142\\
-1.11879866336283	1.97995991983968\\
-1.11422845691383	1.98719867302287\\
-1.10842999965929	1.99599198396794\\
-1.09819639278557	2.01090995449779\\
-1.09739566362948	2.01202404809619\\
-1.08549980486119	2.02805611222445\\
-1.08216432865731	2.03239562534614\\
-1.07270745353508	2.04408817635271\\
-1.06613226452906	2.05194028088218\\
-1.05889760819504	2.06012024048096\\
-1.0501002004008	2.06974519316727\\
-1.04388952525939	2.07615230460922\\
-1.03406813627255	2.08597369359606\\
-1.02745181211216	2.09218436873747\\
-1.01803607214429	2.10076645757832\\
-1.00928481999218	2.10821643286573\\
-1.00200400801603	2.11424476645324\\
-0.988995122323532	2.12424849699399\\
-0.985971943887776	2.12651320994356\\
-0.969939879759519	2.1376506054179\\
-0.965832546162879	2.14028056112224\\
-0.953907815631263	2.14773689968831\\
-0.93884409198786	2.1563126252505\\
-0.937875751503006	2.15685176728282\\
-0.92184368737475	2.16503328233086\\
-0.905838977788721	2.17234468937876\\
-0.905811623246493	2.17235693884157\\
-0.889779559118236	2.1788492796017\\
-0.87374749498998	2.18456798436975\\
-0.861530437300388	2.18837675350701\\
-0.857715430861723	2.18954616042773\\
-0.841683366733467	2.19381497273679\\
-0.82565130260521	2.19741021755645\\
-0.809619238476954	2.20035935280452\\
-0.793587174348697	2.20268792893981\\
-0.777657159274506	2.20440881763527\\
-0.777555110220441	2.20441970542425\\
-0.761523046092184	2.20557727650906\\
-0.745490981963928	2.20618078314376\\
-0.729458917835671	2.20624879964812\\
-0.713426853707415	2.20579850771998\\
-0.697394789579158	2.20484577011933\\
-0.692563647657071	2.20440881763527\\
-0.681362725450902	2.20340688550526\\
-0.665330661322646	2.20149581598421\\
-0.649298597194389	2.19912452309507\\
-0.633266533066132	2.19630450520987\\
-0.617234468937876	2.1930462183839\\
-0.601202404809619	2.18935912257919\\
-0.597387398370955	2.18837675350701\\
-0.585170340681363	2.18526062119864\\
-0.569138276553106	2.18075511445476\\
-0.55310621242485	2.17584697356038\\
-0.542541012275099	2.17234468937876\\
-0.537074148296593	2.17054823307935\\
-0.521042084168337	2.16487272763267\\
-0.50501002004008	2.15881437208123\\
-0.498795492688808	2.1563126252505\\
-0.488977955911824	2.15239180075642\\
-0.472945891783567	2.14560656143752\\
-0.461021161251951	2.14028056112224\\
-0.456913827655311	2.13845970574222\\
-0.440881763527054	2.1309704822989\\
-0.427159101112417	2.12424849699399\\
-0.424849699398798	2.12312507020458\\
-0.408817635270541	2.11495108169751\\
-0.396163769225858	2.10821643286573\\
-0.392785571142285	2.10643000206709\\
-0.376753507014028	2.09758696454167\\
-0.367337767046153	2.09218436873747\\
-0.360721442885771	2.0884104077147\\
-0.344689378757515	2.07891068588303\\
-0.340200968789764	2.07615230460922\\
-0.328657314629258	2.06909626596799\\
-0.314476241216867	2.06012024048096\\
-0.312625250501002	2.05895464809079\\
-0.296593186372745	2.04851186060859\\
-0.290017997366719	2.04408817635271\\
-0.280561122244489	2.03775510652911\\
-0.266525006674562	2.02805611222445\\
-0.264529058116232	2.02668291954234\\
-0.248496993987976	2.01531578552215\\
-0.24398720621718	2.01202404809619\\
-0.232464929859719	2.00364619026176\\
-0.222231322986003	1.99599198396794\\
-0.216432865731463	1.99167061710524\\
-0.201144387012885	1.97995991983968\\
-0.200400801603207	1.97939224687704\\
-0.18436873747495	1.96682922522964\\
-0.180759277272084	1.96392785571142\\
-0.168336673346694	1.95397046507693\\
-0.160943688779969	1.94789579158317\\
-0.152304609218437	1.94081544756269\\
-0.141641876092624	1.93186372745491\\
-0.13627254509018	1.92736641042022\\
-0.122819082958197	1.91583166332665\\
-0.120240480961924	1.91362535945474\\
-0.104443668623181	1.8997995991984\\
-0.104208416833667	1.89959407113954\\
-0.0881763527054109	1.88528235039355\\
-0.0865148160541831	1.88376753507014\\
-0.0721442885771544	1.8706828177036\\
-0.0689729344898847	1.86773547094188\\
-0.0561122244488979	1.85579545394615\\
-0.0517913017535205	1.85170340681363\\
-0.0400801603206413	1.8406211431019\\
-0.0349491946272899	1.83567134268537\\
-0.0240480961923848	1.82516054716343\\
-0.0184276157516171	1.81963927855711\\
-0.00801603206412826	1.80941410698685\\
-0.00220913950613561	1.80360721442886\\
0.00801603206412782	1.79338204285859\\
0.0137222256779407	1.7875751503006\\
0.0240480961923843	1.77706435477866\\
0.0293811605494023	1.77154308617235\\
0.0400801603206409	1.76046082246061\\
0.044781145520701	1.75551102204409\\
0.0561122244488974	1.74357100504835\\
0.0599345598845288	1.73947895791583\\
0.0721442885771539	1.72639424054929\\
0.0748527710502992	1.72344689378758\\
0.0881763527054105	1.70892964498273\\
0.0895462149216145	1.70741482965932\\
0.104023002497253	1.69138276553106\\
0.104208416833667	1.6911772374722\\
0.118282065685979	1.67535070140281\\
0.120240480961924	1.67314439753089\\
0.132343649009419	1.65931863727455\\
0.13627254509018	1.65482132023986\\
0.146215470783112	1.64328657314629\\
0.152304609218437	1.63620622912581\\
0.159904586324255	1.62725450901804\\
0.168336673346693	1.61729711838354\\
0.173417435572365	1.61122244488978\\
0.18436873747495	1.59809175027974\\
0.186759886243764	1.59519038076152\\
0.199935144005117	1.57915831663327\\
0.200400801603206	1.57859064367063\\
0.212940471362413	1.56312625250501\\
0.216432865731463	1.55880488564231\\
0.225792837809639	1.54709418837675\\
0.232464929859719	1.53871633054232\\
0.238496370489154	1.5310621242485\\
0.248496993987976	1.5183217975462\\
0.251054773962973	1.51503006012024\\
0.263468599425322	1.49899799599198\\
0.264529058116232	1.49762480330988\\
0.275739145546932	1.48296593186373\\
0.280561122244489	1.47663286204013\\
0.287877223636193	1.46693386773547\\
0.296593186372745	1.4553254878631\\
0.299885216061356	1.45090180360721\\
0.311764107097419	1.43486973947896\\
0.312625250501002	1.43370414708879\\
0.323515162650359	1.4188376753507\\
0.328657314629258	1.41178163670947\\
0.335145841858041	1.40280561122244\\
0.344689378757515	1.389531928368\\
0.346657440132767	1.38677354709419\\
0.358051131200599	1.37074148296593\\
0.360721442885771	1.36696752194316\\
0.369331448980606	1.35470941883768\\
0.376753507014028	1.34407995051361\\
0.380499925436401	1.33867735470942\\
0.391557950811915	1.32264529058116\\
0.392785571142285	1.32085885978252\\
0.402510992260561	1.30661322645291\\
0.408817635270541	1.29731581115643\\
0.41335791184124	1.29058116232465\\
0.424099699060212	1.27454909819639\\
0.424849699398798	1.27342567140698\\
0.434745077414153	1.25851703406814\\
0.440881763527054	1.2492069552448\\
0.445288689743263	1.24248496993988\\
0.455732556726431	1.22645290581162\\
0.456913827655311	1.2246320504316\\
0.46608675354468	1.21042084168337\\
0.472945891783567	1.19971477787038\\
0.476342309784794	1.19438877755511\\
0.486505544007245	1.17835671342685\\
0.488977955911824	1.17443588893277\\
0.496582201926817	1.1623246492986\\
0.50501002004008	1.14879433200106\\
0.506562227335664	1.14629258517034\\
0.516461159868609	1.13026052104208\\
0.521042084168337	1.12278855929599\\
0.526271334030238	1.11422845691383\\
0.535990169562516	1.09819639278557\\
0.537074148296593	1.09639993648616\\
0.545635936496428	1.08216432865731\\
0.55310621242485	1.06963454871068\\
0.555188309789687	1.06613226452906\\
0.564665976417858	1.0501002004008\\
0.569138276553106	1.04247856134855\\
0.574060918999784	1.03406813627254\\
0.583370807925782	1.01803607214429\\
0.585170340681363	1.01491993983592\\
0.592612661348774	1.00200400801603\\
0.601202404809619	0.986954312959955\\
0.601762078998257	0.985971943887775\\
0.610851960394462	0.969939879759519\\
0.617234468937876	0.95857728050815\\
0.619853429263515	0.953907815631262\\
0.628786712597152	0.937875751503006\\
0.633266533066132	0.929771439077608\\
0.637643590364386	0.921843687374749\\
0.646424307191761	0.905811623246493\\
0.649298597194389	0.900527328706294\\
0.655139621623025	0.889779559118236\\
0.663771644745461	0.87374749498998\\
0.665330661322646	0.870834493338914\\
0.672348105225822	0.857715430861724\\
0.680835154463941	0.841683366733467\\
0.681362725450902	0.840681434603457\\
0.689275162647607	0.825651302605211\\
0.697394789579159	0.810056190961008\\
0.697622326369228	0.809619238476954\\
0.705926469923734	0.793587174348698\\
0.713426853707415	0.778944800305149\\
0.714139046579946	0.777555110220441\\
0.722307271821182	0.761523046092185\\
0.729458917835672	0.747330963976777\\
0.730386855299095	0.745490981963928\\
0.738422394955954	0.729458917835672\\
0.745490981963928	0.715198819215905\\
0.746370303804596	0.713426853707415\\
0.754276259900431	0.697394789579159\\
0.761523046092185	0.682531184324693\\
0.762093544238379	0.681362725450902\\
0.76987289232088	0.665330661322646\\
0.777555110220441	0.649309484983373\\
0.77756034005763	0.649298597194389\\
0.785215933182017	0.633266533066132\\
0.792780544847256	0.617234468937876\\
0.793587174348698	0.615513580242413\\
0.800308648052462	0.601202404809619\\
0.807753133342204	0.585170340681363\\
0.809619238476954	0.581120875850607\\
0.815153935541872	0.569138276553106\\
0.822480850118346	0.55310621242485\\
0.825651302605211	0.546107612346024\\
0.829754334897784	0.537074148296593\\
0.836966124067748	0.521042084168337\\
0.841683366733467	0.510448239269849\\
0.844112032787387	0.50501002004008\\
0.851211034475311	0.488977955911824\\
0.857715430861724	0.474115298704278\\
0.858228869286926	0.472945891783567\\
0.865217316805395	0.456913827655311\\
0.872120783966104	0.440881763527054\\
0.87374749498998	0.437072994389789\\
0.878986367722158	0.424849699398798\\
0.885781952461454	0.408817635270541\\
0.889779559118236	0.399290161365227\\
0.892519249360777	0.392785571142285\\
0.899208927971631	0.376753507014028\\
0.905811623246493	0.36073369234858\\
0.905816692864383	0.360721442885771\\
0.912402345178452	0.344689378757515\\
0.918906420569259	0.328657314629258\\
0.921843687374749	0.321345907581362\\
0.925362512507619	0.312625250501002\\
0.931764677049471	0.296593186372745\\
0.937875751503006	0.281100264276807\\
0.93808941405656	0.280561122244489\\
0.944391308699561	0.264529058116232\\
0.950614039736601	0.248496993987976\\
0.953907815631262	0.239921268425783\\
0.956785885963635	0.232464929859719\\
0.962910080477966	0.216432865731463\\
0.96895700668126	0.200400801603206\\
0.969939879759519	0.197770845898868\\
0.974974727817517	0.18436873747495\\
0.980924619413404	0.168336673346693\\
0.985971943887775	0.154569322168009\\
0.986806821734876	0.152304609218437\\
0.992660954792984	0.13627254509018\\
0.998439532095766	0.120240480961924\\
1.00200400801603	0.110236750421178\\
1.00416444469553	0.104208416833667\\
1.00984840578382	0.0881763527054105\\
1.01545825797897	0.0721442885771539\\
1.01803607214429	0.0646943132897401\\
1.02102360004871	0.0561122244488974\\
1.02653975825509	0.0400801603206409\\
1.03198308268065	0.0240480961923843\\
1.03406813627254	0.0178374210509739\\
1.03738620316287	0.00801603206412782\\
1.04273653858697	-0.00801603206412826\\
1.04801514680891	-0.0240480961923848\\
1.0501002004008	-0.0304552076343362\\
1.05325301903754	-0.0400801603206413\\
1.05843912796602	-0.0561122244488979\\
1.06355445036374	-0.0721442885771544\\
1.06613226452906	-0.0803242481759373\\
1.06862366525489	-0.0881763527054109\\
1.07364676385577	-0.104208416833667\\
1.07859985273705	-0.120240480961924\\
1.08216432865731	-0.131933031968492\\
1.08349661050439	-0.13627254509018\\
1.08835753705469	-0.152304609218437\\
1.0931490683112	-0.168336673346694\\
1.09787225175611	-0.18436873747495\\
1.09819639278557	-0.18548283107335\\
1.10256838432536	-0.200400801603207\\
1.10719865763227	-0.216432865731463\\
1.1117609582029	-0.232464929859719\\
1.11422845691383	-0.241258240804784\\
1.11627507655532	-0.248496993987976\\
1.12074401411038	-0.264529058116232\\
1.12514519320558	-0.280561122244489\\
1.12947946693003	-0.296593186372745\\
1.13026052104208	-0.29952380786443\\
1.13377934617495	-0.312625250501002\\
1.13801878451301	-0.328657314629258\\
1.14219129312577	-0.344689378757515\\
1.14629258517034	-0.360701807341723\\
1.14629765478823	-0.360721442885771\\
1.15037434760639	-0.376753507014028\\
1.15438392376633	-0.392785571142285\\
1.15832704264181	-0.408817635270541\\
1.162204325077	-0.424849699398798\\
1.1623246492986	-0.425355500194049\\
1.16604833959957	-0.440881763527054\\
1.16982653524227	-0.456913827655311\\
1.17353838602857	-0.472945891783567\\
1.17718439541235	-0.488977955911824\\
1.17835671342685	-0.494226662204424\\
1.18078537948077	-0.50501002004008\\
1.18432949877743	-0.521042084168337\\
1.1878070192803	-0.537074148296593\\
1.19121832506825	-0.55310621242485\\
1.19438877755511	-0.568299984510198\\
1.19456520206247	-0.569138276553106\\
1.19787134074402	-0.585170340681363\\
1.20111025125887	-0.601202404809619\\
1.20428219543686	-0.617234468937876\\
1.20738739493193	-0.633266533066132\\
1.21042084168337	-0.649271242652159\\
1.21042607152056	-0.649298597194389\\
1.21342083185755	-0.665330661322646\\
1.21634748309878	-0.681362725450902\\
1.21920611961987	-0.697394789579158\\
1.22199679407919	-0.713426853707415\\
1.2247195170619	-0.729458917835671\\
1.22645290581162	-0.739932866613912\\
1.22738084327505	-0.745490981963928\\
1.22998524429604	-0.761523046092184\\
1.23251986702919	-0.777555110220441\\
1.2349845861562	-0.793587174348697\\
1.23737923194097	-0.809619238476954\\
1.23970358965474	-0.82565130260521\\
1.24195739895292	-0.841683366733467\\
1.24248496993988	-0.845565406346682\\
1.24415072011282	-0.857715430861723\\
1.24627484657362	-0.87374749498998\\
1.24832599436852	-0.889779559118236\\
1.25030375289917	-0.905811623246493\\
1.25220766250782	-0.92184368737475\\
1.25403721359916	-0.937875751503006\\
1.25579184570513	-0.953907815631263\\
1.25747094649076	-0.969939879759519\\
1.25851703406814	-0.980415790087977\\
1.25907670825677	-0.985971943887776\\
1.26061005025889	-1.00200400801603\\
1.26206463497995	-1.01803607214429\\
1.26343967651481	-1.03406813627255\\
1.2647343321079	-1.0501002004008\\
1.26594770084412	-1.06613226452906\\
1.26707882226797	-1.08216432865731\\
1.26812667492813	-1.09819639278557\\
1.2690901748448	-1.11422845691383\\
1.26996817389666	-1.13026052104208\\
1.2707594581245	-1.14629258517034\\
1.2714627459479	-1.1623246492986\\
1.27207668629181	-1.17835671342685\\
1.27259985661893	-1.19438877755511\\
1.27303076086424	-1.21042084168337\\
1.27336782726751	-1.22645290581162\\
1.27360940609925	-1.24248496993988\\
1.27375376727568	-1.25851703406814\\
1.27379909785781	-1.27454909819639\\
1.27374349942935	-1.29058116232465\\
1.27358498534827	-1.30661322645291\\
1.27332147786602	-1.32264529058116\\
1.27295080510859	-1.33867735470942\\
1.27247069791279	-1.35470941883768\\
1.27187878651122	-1.37074148296593\\
1.27117259705861	-1.38677354709419\\
1.27034954799212	-1.40280561122244\\
1.26940694621749	-1.4188376753507\\
1.2683419831128	-1.43486973947896\\
1.2671517303407	-1.45090180360721\\
1.26583313545984	-1.46693386773547\\
1.26438301732549	-1.48296593186373\\
1.26279806126875	-1.49899799599198\\
1.26107481404313	-1.51503006012024\\
1.25920967852684	-1.5310621242485\\
1.25851703406814	-1.5366182780483\\
1.2571939945978	-1.54709418837675\\
1.25502463969909	-1.56312625250501\\
1.25269977496859	-1.57915831663327\\
1.25021500021237	-1.59519038076152\\
1.24756573216555	-1.61122244488978\\
1.24474719693108	-1.62725450901804\\
1.24248496993988	-1.63940453353308\\
1.24174975232708	-1.64328657314629\\
1.23855607385912	-1.65931863727455\\
1.23517430122925	-1.67535070140281\\
1.23159857003181	-1.69138276553106\\
1.22782276802783	-1.70741482965932\\
1.22645290581162	-1.71297294500933\\
1.22381713524383	-1.72344689378758\\
1.21958196768543	-1.73947895791583\\
1.21512182688343	-1.75551102204409\\
1.21042914983683	-1.77154308617234\\
1.21042084168337	-1.77157044071457\\
1.20544003002271	-1.7875751503006\\
1.2001956701131	-1.80360721442886\\
1.19468712586423	-1.81963927855711\\
1.19438877755511	-1.82047757060002\\
1.18883032863661	-1.83567134268537\\
1.18267763612223	-1.85170340681363\\
1.17835671342685	-1.86248676464928\\
1.17618880789234	-1.86773547094188\\
1.16930974881892	-1.88376753507014\\
1.1623246492986	-1.89929379840315\\
1.16208939750908	-1.8997995991984\\
1.15438504264813	-1.91583166332665\\
1.14630279283667	-1.93186372745491\\
1.14629258517034	-1.93188336299896\\
1.13765350560881	-1.94789579158317\\
1.13026052104208	-1.96099723421974\\
1.12853963234662	-1.96392785571142\\
1.11879866336283	-1.97995991983968\\
1.11422845691383	-1.98719867302287\\
1.10842999965929	-1.99599198396794\\
1.09819639278557	-2.01090995449779\\
1.09739566362948	-2.01202404809619\\
1.08549980486119	-2.02805611222445\\
1.08216432865731	-2.03239562534614\\
1.07270745353508	-2.04408817635271\\
1.06613226452906	-2.05194028088218\\
1.05889760819504	-2.06012024048096\\
1.0501002004008	-2.06974519316727\\
1.04388952525939	-2.07615230460922\\
1.03406813627254	-2.08597369359606\\
1.02745181211216	-2.09218436873747\\
1.01803607214429	-2.10076645757832\\
1.00928481999218	-2.10821643286573\\
1.00200400801603	-2.11424476645324\\
0.988995122323532	-2.12424849699399\\
0.985971943887775	-2.12651320994356\\
0.969939879759519	-2.1376506054179\\
0.965832546162879	-2.14028056112224\\
0.953907815631262	-2.14773689968831\\
0.938844091987861	-2.1563126252505\\
0.937875751503006	-2.15685176728282\\
0.921843687374749	-2.16503328233086\\
0.905838977788721	-2.17234468937876\\
0.905811623246493	-2.17235693884156\\
0.889779559118236	-2.1788492796017\\
0.87374749498998	-2.18456798436975\\
0.861530437300388	-2.18837675350701\\
0.857715430861724	-2.18954616042773\\
0.841683366733467	-2.19381497273678\\
0.825651302605211	-2.19741021755645\\
0.809619238476954	-2.20035935280452\\
0.793587174348698	-2.20268792893981\\
0.777657159274506	-2.20440881763527\\
0.777555110220441	-2.20441970542426\\
0.761523046092185	-2.20557727650906\\
0.745490981963928	-2.20618078314376\\
0.729458917835672	-2.20624879964812\\
0.713426853707415	-2.20579850771998\\
0.697394789579159	-2.20484577011933\\
0.692563647657066	-2.20440881763527\\
0.681362725450902	-2.20340688550526\\
0.665330661322646	-2.20149581598421\\
0.649298597194389	-2.19912452309507\\
0.633266533066132	-2.19630450520987\\
0.617234468937876	-2.1930462183839\\
0.601202404809619	-2.18935912257919\\
0.597387398370955	-2.18837675350701\\
0.585170340681363	-2.18526062119864\\
0.569138276553106	-2.18075511445476\\
0.55310621242485	-2.17584697356038\\
0.542541012275099	-2.17234468937876\\
0.537074148296593	-2.17054823307935\\
0.521042084168337	-2.16487272763267\\
0.50501002004008	-2.15881437208122\\
0.498795492688808	-2.1563126252505\\
0.488977955911824	-2.15239180075642\\
0.472945891783567	-2.14560656143752\\
0.46102116125195	-2.14028056112224\\
0.456913827655311	-2.13845970574222\\
0.440881763527054	-2.1309704822989\\
0.427159101112417	-2.12424849699399\\
0.424849699398798	-2.12312507020458\\
0.408817635270541	-2.11495108169751\\
0.396163769225858	-2.10821643286573\\
0.392785571142285	-2.10643000206709\\
0.376753507014028	-2.09758696454167\\
0.367337767046153	-2.09218436873747\\
0.360721442885771	-2.0884104077147\\
0.344689378757515	-2.07891068588303\\
0.340200968789764	-2.07615230460922\\
0.328657314629258	-2.06909626596799\\
0.314476241216868	-2.06012024048096\\
0.312625250501002	-2.05895464809079\\
0.296593186372745	-2.04851186060859\\
0.29001799736672	-2.04408817635271\\
0.280561122244489	-2.03775510652911\\
0.266525006674562	-2.02805611222445\\
0.264529058116232	-2.02668291954234\\
0.248496993987976	-2.01531578552215\\
0.243987206217179	-2.01202404809619\\
0.232464929859719	-2.00364619026176\\
0.222231322986004	-1.99599198396794\\
0.216432865731463	-1.99167061710524\\
0.201144387012887	-1.97995991983968\\
0.200400801603206	-1.97939224687704\\
0.18436873747495	-1.96682922522964\\
0.180759277272084	-1.96392785571142\\
0.168336673346693	-1.95397046507693\\
0.160943688779969	-1.94789579158317\\
0.152304609218437	-1.94081544756269\\
0.141641876092625	-1.93186372745491\\
0.13627254509018	-1.92736641042022\\
0.122819082958197	-1.91583166332665\\
0.120240480961924	-1.91362535945474\\
0.104443668623181	-1.8997995991984\\
0.104208416833667	-1.89959407113954\\
0.0881763527054105	-1.88528235039355\\
0.0865148160541835	-1.88376753507014\\
0.0721442885771539	-1.8706828177036\\
0.0689729344898856	-1.86773547094188\\
0.0561122244488974	-1.85579545394615\\
0.051791301753521	-1.85170340681363\\
0.0400801603206409	-1.8406211431019\\
0.0349491946272904	-1.83567134268537\\
0.0240480961923843	-1.82516054716343\\
0.0184276157516176	-1.81963927855711\\
0.00801603206412782	-1.80941410698685\\
0.00220913950613607	-1.80360721442886\\
-0.00801603206412826	-1.79338204285859\\
-0.0137222256779403	-1.7875751503006\\
-0.0240480961923848	-1.77706435477866\\
-0.0293811605494027	-1.77154308617234\\
-0.0400801603206413	-1.76046082246061\\
-0.0447811455207014	-1.75551102204409\\
-0.0561122244488979	-1.74357100504835\\
-0.0599345598845293	-1.73947895791583\\
-0.0721442885771544	-1.72639424054929\\
-0.0748527710502996	-1.72344689378758\\
-0.0881763527054109	-1.70892964498273\\
-0.089546214921614	-1.70741482965932\\
-0.104023002497253	-1.69138276553106\\
-0.104208416833667	-1.6911772374722\\
-0.11828206568598	-1.67535070140281\\
-0.120240480961924	-1.67314439753089\\
-0.13234364900942	-1.65931863727455\\
-0.13627254509018	-1.65482132023986\\
-0.146215470783112	-1.64328657314629\\
-0.152304609218437	-1.63620622912581\\
-0.159904586324256	-1.62725450901804\\
-0.168336673346694	-1.61729711838354\\
-0.173417435572365	-1.61122244488978\\
-0.18436873747495	-1.59809175027974\\
-0.186759886243765	-1.59519038076152\\
-0.199935144005117	-1.57915831663327\\
-0.200400801603207	-1.57859064367062\\
-0.212940471362413	-1.56312625250501\\
-0.216432865731463	-1.55880488564231\\
-0.225792837809638	-1.54709418837675\\
-0.232464929859719	-1.53871633054232\\
-0.238496370489154	-1.5310621242485\\
-0.248496993987976	-1.5183217975462\\
-0.251054773962973	-1.51503006012024\\
-0.263468599425322	-1.49899799599198\\
-0.264529058116232	-1.49762480330988\\
-0.275739145546932	-1.48296593186373\\
-0.280561122244489	-1.47663286204013\\
-0.287877223636193	-1.46693386773547\\
-0.296593186372745	-1.4553254878631\\
-0.299885216061356	-1.45090180360721\\
-0.311764107097419	-1.43486973947896\\
-0.312625250501002	-1.43370414708879\\
-0.323515162650359	-1.4188376753507\\
-0.328657314629258	-1.41178163670947\\
-0.335145841858041	-1.40280561122244\\
-0.344689378757515	-1.389531928368\\
-0.346657440132767	-1.38677354709419\\
-0.358051131200599	-1.37074148296593\\
-0.360721442885771	-1.36696752194316\\
-0.369331448980606	-1.35470941883768\\
-0.376753507014028	-1.34407995051361\\
-0.380499925436401	-1.33867735470942\\
-0.391557950811915	-1.32264529058116\\
-0.392785571142285	-1.32085885978252\\
-0.402510992260561	-1.30661322645291\\
-0.408817635270541	-1.29731581115643\\
-0.41335791184124	-1.29058116232465\\
-0.424099699060212	-1.27454909819639\\
-0.424849699398798	-1.27342567140698\\
-0.434745077414153	-1.25851703406814\\
-0.440881763527054	-1.2492069552448\\
-0.445288689743263	-1.24248496993988\\
-0.455732556726431	-1.22645290581162\\
-0.456913827655311	-1.2246320504316\\
-0.466086753544681	-1.21042084168337\\
-0.472945891783567	-1.19971477787038\\
-0.476342309784793	-1.19438877755511\\
-0.486505544007245	-1.17835671342685\\
-0.488977955911824	-1.17443588893277\\
-0.496582201926817	-1.1623246492986\\
-0.50501002004008	-1.14879433200106\\
-0.506562227335664	-1.14629258517034\\
-0.516461159868609	-1.13026052104208\\
-0.521042084168337	-1.12278855929599\\
-0.526271334030238	-1.11422845691383\\
-0.535990169562516	-1.09819639278557\\
-0.537074148296593	-1.09639993648616\\
-0.545635936496428	-1.08216432865731\\
-0.55310621242485	-1.06963454871068\\
-0.555188309789687	-1.06613226452906\\
-0.564665976417857	-1.0501002004008\\
-0.569138276553106	-1.04247856134855\\
-0.574060918999784	-1.03406813627255\\
-0.583370807925782	-1.01803607214429\\
-0.585170340681363	-1.01491993983592\\
-0.592612661348773	-1.00200400801603\\
-0.601202404809619	-0.986954312959955\\
-0.601762078998257	-0.985971943887776\\
-0.610851960394462	-0.969939879759519\\
-0.617234468937876	-0.95857728050815\\
-0.619853429263515	-0.953907815631263\\
-0.628786712597152	-0.937875751503006\\
-0.633266533066132	-0.929771439077608\\
-0.637643590364386	-0.92184368737475\\
-0.646424307191761	-0.905811623246493\\
-0.649298597194389	-0.900527328706294\\
-0.655139621623025	-0.889779559118236\\
-0.663771644745462	-0.87374749498998\\
-0.665330661322646	-0.870834493338914\\
-0.672348105225823	-0.857715430861723\\
-0.680835154463942	-0.841683366733467\\
-0.681362725450902	-0.840681434603458\\
-0.689275162647607	-0.82565130260521\\
-0.697394789579158	-0.81005619096101\\
-0.697622326369228	-0.809619238476954\\
-0.705926469923734	-0.793587174348697\\
-0.713426853707415	-0.77894480030515\\
-0.714139046579946	-0.777555110220441\\
-0.722307271821182	-0.761523046092184\\
-0.729458917835671	-0.747330963976778\\
-0.730386855299095	-0.745490981963928\\
-0.738422394955954	-0.729458917835671\\
-0.745490981963928	-0.715198819215906\\
-0.746370303804596	-0.713426853707415\\
-0.754276259900432	-0.697394789579158\\
-0.761523046092184	-0.682531184324695\\
-0.76209354423838	-0.681362725450902\\
-0.76987289232088	-0.665330661322646\\
-0.777555110220441	-0.649309484983374\\
-0.77756034005763	-0.649298597194389\\
-0.785215933182017	-0.633266533066132\\
-0.792780544847256	-0.617234468937876\\
-0.793587174348697	-0.615513580242414\\
-0.800308648052461	-0.601202404809619\\
-0.807753133342203	-0.585170340681363\\
-0.809619238476954	-0.581120875850608\\
-0.815153935541872	-0.569138276553106\\
-0.822480850118345	-0.55310621242485\\
-0.82565130260521	-0.546107612346026\\
-0.829754334897784	-0.537074148296593\\
-0.836966124067748	-0.521042084168337\\
-0.841683366733467	-0.510448239269851\\
-0.844112032787387	-0.50501002004008\\
-0.851211034475311	-0.488977955911824\\
-0.857715430861723	-0.474115298704279\\
-0.858228869286925	-0.472945891783567\\
-0.865217316805395	-0.456913827655311\\
-0.872120783966104	-0.440881763527054\\
-0.87374749498998	-0.437072994389791\\
-0.878986367722158	-0.424849699398798\\
-0.885781952461454	-0.408817635270541\\
-0.889779559118236	-0.399290161365226\\
-0.892519249360777	-0.392785571142285\\
-0.899208927971631	-0.376753507014028\\
-0.905811623246493	-0.360733692348578\\
-0.905816692864383	-0.360721442885771\\
-0.912402345178453	-0.344689378757515\\
-0.918906420569258	-0.328657314629258\\
-0.92184368737475	-0.321345907581361\\
-0.925362512507619	-0.312625250501002\\
-0.93176467704947	-0.296593186372745\\
-0.937875751503006	-0.281100264276806\\
-0.938089414056559	-0.280561122244489\\
-0.944391308699561	-0.264529058116232\\
-0.950614039736601	-0.248496993987976\\
-0.953907815631263	-0.239921268425782\\
-0.956785885963635	-0.232464929859719\\
-0.962910080477966	-0.216432865731463\\
-0.968957006681261	-0.200400801603207\\
-0.969939879759519	-0.197770845898866\\
-0.974974727817517	-0.18436873747495\\
-0.980924619413404	-0.168336673346694\\
-0.985971943887776	-0.154569322168008\\
-0.986806821734876	-0.152304609218437\\
-0.992660954792983	-0.13627254509018\\
-0.998439532095766	-0.120240480961924\\
-1.00200400801603	-0.110236750421177\\
-1.00416444469553	-0.104208416833667\\
-1.00984840578382	-0.0881763527054109\\
-1.01545825797897	-0.0721442885771544\\
-1.01803607214429	-0.0646943132897392\\
-1.02102360004871	-0.0561122244488979\\
-1.02653975825509	-0.0400801603206413\\
-1.03198308268065	-0.0240480961923848\\
-1.03406813627255	-0.0178374210509725\\
-1.03738620316287	-0.00801603206412826\\
-1.04273653858697	0.00801603206412782\\
-1.04801514680891	0.0240480961923843\\
-1.0501002004008	0.0304552076343385\\
-1.05325301903754	0.0400801603206409\\
-1.05843912796601	0.0561122244488974\\
-1.06355445036374	0.0721442885771539\\
-1.06613226452906	0.0803242481759391\\
-1.06862366525489	0.0881763527054105\\
-1.07364676385577	0.104208416833667\\
-1.07859985273705	0.120240480961924\\
-1.08216432865731	0.131933031968493\\
-1.08349661050439	0.13627254509018\\
-1.08835753705469	0.152304609218437\\
-1.0931490683112	0.168336673346693\\
-1.09787225175611	0.18436873747495\\
-1.09819639278557	0.185482831073351\\
-1.10256838432536	0.200400801603206\\
-1.10719865763227	0.216432865731463\\
-1.1117609582029	0.232464929859719\\
-1.11422845691383	0.241258240804785\\
-1.11627507655532	0.248496993987976\\
-1.12074401411038	0.264529058116232\\
-1.12514519320558	0.280561122244489\\
-1.12947946693003	0.296593186372745\\
-1.13026052104208	0.299523807864431\\
-1.13377934617495	0.312625250501002\\
-1.13801878451301	0.328657314629258\\
-1.14219129312577	0.344689378757515\\
-1.14629258517034	0.360701807341723\\
-1.14629765478823	0.360721442885771\\
-1.15037434760639	0.376753507014028\\
-1.15438392376633	0.392785571142285\\
-1.15832704264181	0.408817635270541\\
-1.162204325077	0.424849699398798\\
-1.1623246492986	0.425355500194049\\
-1.16604833959957	0.440881763527054\\
-1.16982653524227	0.456913827655311\\
-1.17353838602857	0.472945891783567\\
-1.17718439541235	0.488977955911824\\
-1.17835671342685	0.494226662204425\\
-1.18078537948077	0.50501002004008\\
-1.18432949877743	0.521042084168337\\
-1.1878070192803	0.537074148296593\\
-1.19121832506825	0.55310621242485\\
-1.19438877755511	0.568299984510198\\
-1.19456520206247	0.569138276553106\\
-1.19787134074402	0.585170340681363\\
-1.20111025125887	0.601202404809619\\
-1.20428219543685	0.617234468937876\\
-1.20738739493193	0.633266533066132\\
-1.21042084168337	0.649271242652159\\
-1.21042607152056	0.649298597194389\\
-1.21342083185755	0.665330661322646\\
-1.21634748309878	0.681362725450902\\
-1.21920611961987	0.697394789579159\\
-1.22199679407919	0.713426853707415\\
-1.2247195170619	0.729458917835672\\
-1.22645290581162	0.739932866613911\\
-1.22738084327505	0.745490981963928\\
-1.22998524429604	0.761523046092185\\
-1.23251986702919	0.777555110220441\\
-1.2349845861562	0.793587174348698\\
-1.23737923194097	0.809619238476954\\
-1.23970358965474	0.825651302605211\\
-1.24195739895292	0.841683366733467\\
-1.24248496993988	0.845565406346681\\
-1.24415072011282	0.857715430861724\\
-1.24627484657362	0.87374749498998\\
-1.24832599436852	0.889779559118236\\
-1.25030375289917	0.905811623246493\\
-1.25220766250782	0.921843687374749\\
-1.25403721359916	0.937875751503006\\
-1.25579184570513	0.953907815631262\\
-1.25747094649076	0.969939879759519\\
-1.25851703406814	0.980415790087977\\
-1.25907670825677	0.985971943887775\\
-1.26061005025888	1.00200400801603\\
-1.26206463497995	1.01803607214429\\
-1.26343967651481	1.03406813627254\\
-1.2647343321079	1.0501002004008\\
-1.26594770084412	1.06613226452906\\
-1.26707882226797	1.08216432865731\\
-1.26812667492813	1.09819639278557\\
-1.2690901748448	1.11422845691383\\
-1.26996817389666	1.13026052104208\\
-1.2707594581245	1.14629258517034\\
-1.2714627459479	1.1623246492986\\
-1.27207668629181	1.17835671342685\\
-1.27259985661893	1.19438877755511\\
-1.27303076086424	1.21042084168337\\
-1.27336782726751	1.22645290581162\\
-1.27360940609925	1.24248496993988\\
-1.27375376727568	1.25851703406814\\
-1.27379909785781	1.27454909819639\\
-1.27374349942935	1.29058116232465\\
-1.27358498534827	1.30661322645291\\
-1.27332147786602	1.32264529058116\\
-1.27295080510859	1.33867735470942\\
-1.27247069791279	1.35470941883768\\
-1.27187878651122	1.37074148296593\\
-1.27117259705861	1.38677354709419\\
-1.27034954799212	1.40280561122244\\
-1.26940694621749	1.4188376753507\\
-1.2683419831128	1.43486973947896\\
-1.2671517303407	1.45090180360721\\
-1.26583313545984	1.46693386773547\\
-1.26438301732549	1.48296593186373\\
-1.26279806126875	1.49899799599198\\
-1.26107481404313	1.51503006012024\\
-1.25920967852684	1.5310621242485\\
-1.25851703406814	1.5366182780483\\
}--cycle;


\addplot[area legend,solid,fill=mycolor7,draw=black,forget plot]
table[row sep=crcr] {%
x	y\\
-1.11422845691383	1.29527490685972\\
-1.11319708140095	1.30661322645291\\
-1.11158897321772	1.32264529058116\\
-1.1098226773115	1.33867735470942\\
-1.10789361042715	1.35470941883768\\
-1.10579698497173	1.37074148296593\\
-1.10352780003294	1.38677354709419\\
-1.1010808319104	1.40280561122244\\
-1.0984506241303	1.4188376753507\\
-1.09819639278557	1.42029804511158\\
-1.09560322010549	1.43486973947896\\
-1.09255398071092	1.45090180360721\\
-1.08929907735243	1.46693386773547\\
-1.08583167329455	1.48296593186373\\
-1.08216432865731	1.49891333546637\\
-1.0821443520793	1.49899799599198\\
-1.07817537844418	1.51503006012024\\
-1.07396500034531	1.5310621242485\\
-1.06950457834855	1.54709418837675\\
-1.06613226452906	1.55859377815824\\
-1.0647629193205	1.56312625250501\\
-1.05969013932373	1.57915831663327\\
-1.05432972957213	1.59519038076152\\
-1.0501002004008	1.60721650529931\\
-1.04864334990002	1.61122244488978\\
-1.04255657480825	1.62725450901804\\
-1.0361352655747	1.64328657314629\\
-1.03406813627255	1.64823587357896\\
-1.0292623072434	1.65931863727455\\
-1.02196731801052	1.67535070140281\\
-1.01803607214429	1.68361504268286\\
-1.01418588617855	1.69138276553106\\
-1.00587990953774	1.70741482965932\\
-1.00200400801603	1.71458298935537\\
-0.996990227098053	1.72344689378758\\
-0.987502407962091	1.73947895791583\\
-0.985971943887776	1.74197122833202\\
-0.977236362534941	1.75551102204409\\
-0.969939879759519	1.7663285202369\\
-0.966230684303345	1.77154308617235\\
-0.954346209377059	1.7875751503006\\
-0.953907815631263	1.78814656274547\\
-0.94133327615024	1.80360721442886\\
-0.937875751503006	1.8076965221177\\
-0.927121552749067	1.81963927855711\\
-0.92184368737475	1.82528934752995\\
-0.911461692219334	1.83567134268537\\
-0.905811623246493	1.84112894737434\\
-0.89402410506561	1.85170340681363\\
-0.889779559118236	1.85538858208849\\
-0.874365875445441	1.86773547094188\\
-0.87374749498998	1.86821576553839\\
-0.857715430861723	1.87970629675285\\
-0.851492446151481	1.88376753507014\\
-0.841683366733467	1.8899920287837\\
-0.82565130260521	1.89916697663983\\
-0.824429387048633	1.8997995991984\\
-0.809619238476954	1.90727502109806\\
-0.793587174348697	1.91442224207739\\
-0.790015546419613	1.91583166332665\\
-0.777555110220441	1.92063769330967\\
-0.761523046092184	1.92599125684182\\
-0.745490981963928	1.93053135242795\\
-0.739874814442953	1.93186372745491\\
-0.729458917835671	1.93428701368306\\
-0.713426853707415	1.9373019131192\\
-0.697394789579158	1.9396132115098\\
-0.681362725450902	1.94125040392693\\
-0.665330661322646	1.94224079591663\\
-0.649298597194389	1.94260964305413\\
-0.633266533066132	1.94238027890235\\
-0.617234468937876	1.94157423236356\\
-0.601202404809619	1.94021133531567\\
-0.585170340681363	1.93830982133684\\
-0.569138276553106	1.93588641624351\\
-0.55310621242485	1.93295642109648\\
-0.54801938157678	1.93186372745491\\
-0.537074148296593	1.92953305696762\\
-0.521042084168337	1.92563085023898\\
-0.50501002004008	1.92126222414596\\
-0.488977955911824	1.91643779725655\\
-0.487146074453037	1.91583166332665\\
-0.472945891783567	1.91116918961357\\
-0.456913827655311	1.9054657821794\\
-0.442103679083631	1.8997995991984\\
-0.440881763527054	1.89933546148151\\
-0.424849699398798	1.89279061863763\\
-0.408817635270541	1.8858346643723\\
-0.404332999824192	1.88376753507014\\
-0.392785571142285	1.8784790028984\\
-0.376753507014028	1.87072758265841\\
-0.370881128353451	1.86773547094188\\
-0.360721442885771	1.86258938978954\\
-0.344689378757515	1.85406785039725\\
-0.340444832810142	1.85170340681363\\
-0.328657314629258	1.84517246283947\\
-0.312625250501002	1.83590190159421\\
-0.312243254575273	1.83567134268537\\
-0.296593186372745	1.82627154452174\\
-0.285972676222855	1.81963927855711\\
-0.280561122244489	1.81627548532647\\
-0.264529058116232	1.80592295097609\\
-0.261071533468998	1.80360721442886\\
-0.248496993987976	1.795219764379\\
-0.237425008585017	1.7875751503006\\
-0.232464929859719	1.78416365914826\\
-0.216432865731463	1.77276117036572\\
-0.21477460953327	1.77154308617235\\
-0.200400801603207	1.76101992105668\\
-0.193104318827785	1.75551102204409\\
-0.18436873747495	1.7489359039608\\
-0.172173345241227	1.73947895791583\\
-0.168336673346694	1.73651220130204\\
-0.152304609218437	1.72375235381353\\
-0.151931278795002	1.72344689378758\\
-0.13627254509018	1.71066447591853\\
-0.132396643568474	1.70741482965932\\
-0.120240480961924	1.69724375478096\\
-0.113414571508778	1.69138276553106\\
-0.104208416833667	1.68349221526087\\
-0.0949445761708533	1.67535070140281\\
-0.0881763527054109	1.66941162343815\\
-0.0769501176063007	1.65931863727455\\
-0.0721442885771544	1.6550034888922\\
-0.0593981639867885	1.64328657314629\\
-0.0561122244488979	1.64026906664833\\
-0.0422588104508495	1.62725450901804\\
-0.0400801603206413	1.62520935879441\\
-0.025504946693164	1.61122244488978\\
-0.0240480961923848	1.60982511577077\\
-0.00911196622221128	1.59519038076152\\
-0.00801603206412826	1.59411683733554\\
0.00694248863814595	1.57915831663327\\
0.00801603206412782	1.57808477320728\\
0.0226787509838308	1.56312625250501\\
0.0240480961923843	1.561728923386\\
0.0381153270050993	1.54709418837675\\
0.0400801603206409	1.54504903815313\\
0.0532690706591171	1.5310621242485\\
0.0561122244488974	1.52804461775053\\
0.0681553383640204	1.51503006012024\\
0.0721442885771539	1.5107149117379\\
0.082788126291551	1.49899799599198\\
0.0881763527054105	1.49305891802733\\
0.0971801924616865	1.48296593186373\\
0.104208416833667	1.47507538159354\\
0.111343165528784	1.46693386773547\\
0.120240480961924	1.45676279285712\\
0.125287641884237	1.45090180360721\\
0.13627254509018	1.43811938573817\\
0.139023272476939	1.43486973947896\\
0.152304609218437	1.41914313537665\\
0.152558840563168	1.4188376753507\\
0.165878065541669	1.40280561122244\\
0.168336673346693	1.39983885460865\\
0.179011697335403	1.38677354709419\\
0.18436873747495	1.3801984290109\\
0.191969329661104	1.37074148296593\\
0.200400801603206	1.36021831785027\\
0.20475718723836	1.35470941883768\\
0.216432865731463	1.33989543890279\\
0.21738088618959	1.33867735470942\\
0.229825446163612	1.32264529058116\\
0.232464929859719	1.31923379942882\\
0.242111014129311	1.30661322645291\\
0.248496993987976	1.29822577640305\\
0.254249679111395	1.29058116232465\\
0.264529058116232	1.27686483474363\\
0.266245197116871	1.27454909819639\\
0.278085913715293	1.25851703406814\\
0.280561122244489	1.25515324083749\\
0.289782528321357	1.24248496993988\\
0.296593186372745	1.23308517177625\\
0.301349186753381	1.22645290581162\\
0.312625250501002	1.2106514005922\\
0.312788164902552	1.21042084168337\\
0.324080466620025	1.19438877755511\\
0.328657314629258	1.18785783358095\\
0.335252744928811	1.17835671342685\\
0.344689378757515	1.16468909288222\\
0.346307219645143	1.1623246492986\\
0.357232011798649	1.14629258517034\\
0.360721442885771	1.141146504018\\
0.368039327265191	1.13026052104208\\
0.376753507014028	1.11722056863035\\
0.378736718869431	1.11422845691383\\
0.389314546384562	1.09819639278557\\
0.392785571142285	1.09290786061383\\
0.399781828147799	1.08216432865731\\
0.408817635270541	1.06819939383122\\
0.410145301707386	1.06613226452906\\
0.420395503992964	1.0501002004008\\
0.424849699398798	1.04309121984003\\
0.430543682829265	1.03406813627254\\
0.440592084630359	1.01803607214429\\
0.440881763527054	1.0175719344274\\
0.450533390184912	1.00200400801603\\
0.456913827655311	0.991638126868775\\
0.460379816391221	0.985971943887775\\
0.470127348333343	0.969939879759519\\
0.472945891783567	0.965277406046434\\
0.479778661095603	0.953907815631262\\
0.488977955911824	0.938481885432903\\
0.489337460739446	0.937875751503006\\
0.498800763532824	0.921843687374749\\
0.50501002004008	0.911242184065805\\
0.508174479282237	0.905811623246493\\
0.517457049057759	0.889779559118236\\
0.521042084168337	0.883546681902311\\
0.526651618119988	0.87374749498998\\
0.535757770323694	0.857715430861724\\
0.537074148296593	0.855384760374438\\
0.544778659699726	0.841683366733467\\
0.55310621242485	0.826743996246779\\
0.553712876870233	0.825651302605211\\
0.562564761636584	0.809619238476954\\
0.569138276553106	0.797609863137293\\
0.571332121958972	0.793587174348698\\
0.5800184813075	0.777555110220441\\
0.585170340681363	0.767969139974112\\
0.588623238808741	0.761523046092185\\
0.597147798796265	0.745490981963928\\
0.601202404809619	0.737806525696426\\
0.605593806770739	0.729458917835672\\
0.613960138273222	0.713426853707415\\
0.617234468937876	0.707105294487803\\
0.62225086427468	0.697394789579159\\
0.630462387886829	0.681362725450902\\
0.633266533066132	0.675847212770085\\
0.638600927584093	0.665330661322646\\
0.64666091823862	0.649298597194389\\
0.649298597194389	0.644012448665355\\
0.654650008190866	0.633266533066132\\
0.662561599507746	0.617234468937876\\
0.665330661322646	0.611579473271343\\
0.670403628910887	0.601202404809619\\
0.678169817286189	0.585170340681363\\
0.681362725450902	0.578524953025124\\
0.685866838737787	0.569138276553106\\
0.69349048718108	0.55310621242485\\
0.697394789579159	0.544823632351482\\
0.701044226507351	0.537074148296593\\
0.708528068235948	0.521042084168337\\
0.713426853707415	0.510448205704373\\
0.715939933420816	0.50501002004008\\
0.723286575218568	0.488977955911824\\
0.729458917835672	0.475369178011716\\
0.7305576644713	0.472945891783567\\
0.737769589818998	0.456913827655311\\
0.744902759144508	0.440881763527054\\
0.745490981963928	0.439549388500092\\
0.751980270797614	0.424849699398798\\
0.758981402159146	0.408817635270541\\
0.761523046092185	0.402945164657454\\
0.765921363119241	0.392785571142285\\
0.772792646047509	0.376753507014028\\
0.777555110220441	0.365527472868786\\
0.779595206106097	0.360721442885771\\
0.78633871591627	0.344689378757515\\
0.79300605003884	0.328657314629258\\
0.793587174348698	0.327247893379989\\
0.799621440602657	0.312625250501002\\
0.806163058730893	0.296593186372745\\
0.809619238476954	0.288036544144152\\
0.81264225875116	0.280561122244489\\
0.819059962112337	0.264529058116232\\
0.825403267700109	0.248496993987976\\
0.825651302605211	0.247864371429403\\
0.831697706153768	0.232464929859719\\
0.837918726609621	0.216432865731463\\
0.841683366733467	0.206625295316764\\
0.84407685759974	0.200400801603206\\
0.850177130537457	0.18436873747495\\
0.856204317512007	0.168336673346693\\
0.857715430861724	0.164275435029403\\
0.862178565765762	0.152304609218437\\
0.868086222771197	0.13627254509018\\
0.87374749498998	0.120720775558425\\
0.873922746025479	0.120240480961924\\
0.879712154483288	0.104208416833667\\
0.88542933896269	0.0881763527054105\\
0.889779559118236	0.075829463852014\\
0.891081352733647	0.0721442885771539\\
0.896681104837513	0.0561122244488974\\
0.902209517311712	0.0400801603206409\\
0.905811623246493	0.0295057008813569\\
0.907676036994777	0.0240480961923843\\
0.913087586458924	0.00801603206412782\\
0.918428489847537	-0.00801603206412826\\
0.921843687374749	-0.0183980272195427\\
0.923708101123034	-0.0240480961923848\\
0.928932465899462	-0.0400801603206413\\
0.934086689113959	-0.0561122244488979\\
0.937875751503006	-0.0680549808883071\\
0.939177545118417	-0.0721442885771544\\
0.944215309975964	-0.0881763527054109\\
0.949183249838047	-0.104208416833667\\
0.953907815631262	-0.119669068517051\\
0.954083066666762	-0.120240480961924\\
0.958934384098946	-0.13627254509018\\
0.963716005591637	-0.152304609218437\\
0.968428766409803	-0.168336673346694\\
0.969939879759519	-0.173551239282138\\
0.973086645577692	-0.18436873747495\\
0.977681480427534	-0.200400801603207\\
0.982207303763929	-0.216432865731463\\
0.985971943887775	-0.229972659443533\\
0.986667731865482	-0.232464929859719\\
0.991074875445203	-0.248496993987976\\
0.995412667523159	-0.264529058116232\\
0.99968171568218	-0.280561122244489\\
1.00200400801603	-0.289425026676692\\
1.00389005021323	-0.296593186372745\\
1.00803827426999	-0.312625250501002\\
1.01211712930511	-0.328657314629258\\
1.01612708530091	-0.344689378757515\\
1.01803607214429	-0.352457101605711\\
1.02007616802994	-0.360721442885771\\
1.0239626482088	-0.376753507014028\\
1.02777931323051	-0.392785571142285\\
1.03152649233951	-0.408817635270541\\
1.03406813627254	-0.419900398966124\\
1.03520845019296	-0.424849699398798\\
1.03882897978193	-0.440881763527054\\
1.04237880825587	-0.456913827655311\\
1.04585811988899	-0.472945891783567\\
1.04926705264233	-0.488977955911824\\
1.0501002004008	-0.492983895502286\\
1.0526132801142	-0.50501002004008\\
1.05589018644029	-0.521042084168337\\
1.0590950852347	-0.537074148296593\\
1.06222796213098	-0.55310621242485\\
1.06528875386917	-0.569138276553106\\
1.06613226452906	-0.573670750899876\\
1.06828280652761	-0.585170340681363\\
1.0712052321173	-0.601202404809619\\
1.07405350670123	-0.617234468937876\\
1.07682740507046	-0.633266533066132\\
1.07952664970155	-0.649298597194389\\
1.08215090986002	-0.665330661322646\\
1.08216432865731	-0.66541532184826\\
1.08470439642679	-0.681362725450902\\
1.08718072399412	-0.697394789579158\\
1.08957941337184	-0.713426853707415\\
1.09189995632039	-0.729458917835671\\
1.09414178677222	-0.745490981963928\\
1.09630427959414	-0.761523046092184\\
1.09819639278557	-0.776094740459558\\
1.09838695251236	-0.777555110220441\\
1.10039023819144	-0.793587174348697\\
1.10231088468654	-0.809619238476954\\
1.10414806928106	-0.82565130260521\\
1.1059009041887	-0.841683366733467\\
1.10756843491448	-0.857715430861723\\
1.10914963852203	-0.87374749498998\\
1.11064342180325	-0.889779559118236\\
1.11204861934615	-0.905811623246493\\
1.11336399149656	-0.92184368737475\\
1.11422845691383	-0.933182006967936\\
1.11458796174145	-0.937875751503006\\
1.1157184454276	-0.953907815631263\\
1.11675443604705	-0.969939879759519\\
1.11769444564974	-0.985971943887776\\
1.11853690276846	-1.00200400801603\\
1.11928014981559	-1.01803607214429\\
1.11992244034429	-1.03406813627255\\
1.12046193616776	-1.0501002004008\\
1.12089670432974	-1.06613226452906\\
1.12122471391934	-1.08216432865731\\
1.12144383272227	-1.09819639278557\\
1.12155182370072	-1.11422845691383\\
1.12154634129325	-1.13026052104208\\
1.12142492752561	-1.14629258517034\\
1.12118500792299	-1.1623246492986\\
1.12082388721338	-1.17835671342685\\
1.12033874481139	-1.19438877755511\\
1.1197266300709	-1.21042084168337\\
1.11898445729446	-1.22645290581162\\
1.11810900048634	-1.24248496993988\\
1.11709688783561	-1.25851703406814\\
1.11594459591447	-1.27454909819639\\
1.11464844357632	-1.29058116232465\\
1.11422845691383	-1.29527490685972\\
1.11319708140095	-1.30661322645291\\
1.11158897321772	-1.32264529058116\\
1.1098226773115	-1.33867735470942\\
1.10789361042715	-1.35470941883768\\
1.10579698497173	-1.37074148296593\\
1.10352780003294	-1.38677354709419\\
1.1010808319104	-1.40280561122244\\
1.0984506241303	-1.4188376753507\\
1.09819639278557	-1.42029804511158\\
1.09560322010549	-1.43486973947896\\
1.09255398071092	-1.45090180360721\\
1.08929907735243	-1.46693386773547\\
1.08583167329455	-1.48296593186373\\
1.08216432865731	-1.49891333546637\\
1.0821443520793	-1.49899799599198\\
1.07817537844418	-1.51503006012024\\
1.07396500034531	-1.5310621242485\\
1.06950457834855	-1.54709418837675\\
1.06613226452906	-1.55859377815824\\
1.0647629193205	-1.56312625250501\\
1.05969013932373	-1.57915831663327\\
1.05432972957213	-1.59519038076152\\
1.0501002004008	-1.60721650529932\\
1.04864334990002	-1.61122244488978\\
1.04255657480825	-1.62725450901804\\
1.0361352655747	-1.64328657314629\\
1.03406813627254	-1.64823587357897\\
1.0292623072434	-1.65931863727455\\
1.02196731801052	-1.67535070140281\\
1.01803607214429	-1.68361504268287\\
1.01418588617855	-1.69138276553106\\
1.00587990953774	-1.70741482965932\\
1.00200400801603	-1.71458298935537\\
0.996990227098052	-1.72344689378758\\
0.987502407962091	-1.73947895791583\\
0.985971943887775	-1.74197122833202\\
0.977236362534941	-1.75551102204409\\
0.969939879759519	-1.7663285202369\\
0.966230684303344	-1.77154308617234\\
0.954346209377058	-1.7875751503006\\
0.953907815631262	-1.78814656274547\\
0.94133327615024	-1.80360721442886\\
0.937875751503006	-1.8076965221177\\
0.927121552749066	-1.81963927855711\\
0.921843687374749	-1.82528934752995\\
0.911461692219333	-1.83567134268537\\
0.905811623246493	-1.84112894737434\\
0.894024105065609	-1.85170340681363\\
0.889779559118236	-1.85538858208849\\
0.874365875445442	-1.86773547094188\\
0.87374749498998	-1.86821576553839\\
0.857715430861724	-1.87970629675285\\
0.851492446151481	-1.88376753507014\\
0.841683366733467	-1.8899920287837\\
0.825651302605211	-1.89916697663982\\
0.824429387048632	-1.8997995991984\\
0.809619238476954	-1.90727502109806\\
0.793587174348698	-1.91442224207739\\
0.790015546419612	-1.91583166332665\\
0.777555110220441	-1.92063769330967\\
0.761523046092185	-1.92599125684182\\
0.745490981963928	-1.93053135242795\\
0.739874814442951	-1.93186372745491\\
0.729458917835672	-1.93428701368306\\
0.713426853707415	-1.9373019131192\\
0.697394789579159	-1.9396132115098\\
0.681362725450902	-1.94125040392693\\
0.665330661322646	-1.94224079591663\\
0.649298597194389	-1.94260964305413\\
0.633266533066132	-1.94238027890235\\
0.617234468937876	-1.94157423236356\\
0.601202404809619	-1.94021133531566\\
0.585170340681363	-1.93830982133684\\
0.569138276553106	-1.93588641624351\\
0.55310621242485	-1.93295642109648\\
0.548019381576782	-1.93186372745491\\
0.537074148296593	-1.92953305696762\\
0.521042084168337	-1.92563085023898\\
0.50501002004008	-1.92126222414596\\
0.488977955911824	-1.91643779725655\\
0.487146074453037	-1.91583166332665\\
0.472945891783567	-1.91116918961357\\
0.456913827655311	-1.9054657821794\\
0.442103679083632	-1.8997995991984\\
0.440881763527054	-1.89933546148151\\
0.424849699398798	-1.89279061863763\\
0.408817635270541	-1.8858346643723\\
0.404332999824194	-1.88376753507014\\
0.392785571142285	-1.8784790028984\\
0.376753507014028	-1.87072758265841\\
0.370881128353451	-1.86773547094188\\
0.360721442885771	-1.86258938978954\\
0.344689378757515	-1.85406785039725\\
0.340444832810142	-1.85170340681363\\
0.328657314629258	-1.84517246283947\\
0.312625250501002	-1.83590190159421\\
0.312243254575274	-1.83567134268537\\
0.296593186372745	-1.82627154452174\\
0.285972676222856	-1.81963927855711\\
0.280561122244489	-1.81627548532647\\
0.264529058116232	-1.8059229509761\\
0.261071533468998	-1.80360721442886\\
0.248496993987976	-1.795219764379\\
0.237425008585017	-1.7875751503006\\
0.232464929859719	-1.78416365914826\\
0.216432865731463	-1.77276117036572\\
0.21477460953327	-1.77154308617234\\
0.200400801603206	-1.76101992105668\\
0.193104318827784	-1.75551102204409\\
0.18436873747495	-1.7489359039608\\
0.172173345241226	-1.73947895791583\\
0.168336673346693	-1.73651220130204\\
0.152304609218437	-1.72375235381353\\
0.151931278795001	-1.72344689378758\\
0.13627254509018	-1.71066447591853\\
0.132396643568474	-1.70741482965932\\
0.120240480961924	-1.69724375478096\\
0.113414571508777	-1.69138276553106\\
0.104208416833667	-1.68349221526087\\
0.0949445761708529	-1.67535070140281\\
0.0881763527054105	-1.66941162343815\\
0.0769501176063007	-1.65931863727455\\
0.0721442885771539	-1.6550034888922\\
0.0593981639867881	-1.64328657314629\\
0.0561122244488974	-1.64026906664833\\
0.042258810450849	-1.62725450901804\\
0.0400801603206409	-1.62520935879441\\
0.0255049466931636	-1.61122244488978\\
0.0240480961923843	-1.60982511577077\\
0.00911196622221084	-1.59519038076152\\
0.00801603206412782	-1.59411683733554\\
-0.0069424886381464	-1.57915831663327\\
-0.00801603206412826	-1.57808477320728\\
-0.0226787509838312	-1.56312625250501\\
-0.0240480961923848	-1.561728923386\\
-0.0381153270050997	-1.54709418837675\\
-0.0400801603206413	-1.54504903815313\\
-0.0532690706591176	-1.5310621242485\\
-0.0561122244488979	-1.52804461775053\\
-0.0681553383640199	-1.51503006012024\\
-0.0721442885771544	-1.5107149117379\\
-0.082788126291551	-1.49899799599198\\
-0.0881763527054109	-1.49305891802733\\
-0.0971801924616865	-1.48296593186373\\
-0.104208416833667	-1.47507538159353\\
-0.111343165528784	-1.46693386773547\\
-0.120240480961924	-1.45676279285711\\
-0.125287641884237	-1.45090180360721\\
-0.13627254509018	-1.43811938573817\\
-0.13902327247694	-1.43486973947896\\
-0.152304609218437	-1.41914313537665\\
-0.152558840563168	-1.4188376753507\\
-0.165878065541669	-1.40280561122244\\
-0.168336673346694	-1.39983885460865\\
-0.179011697335402	-1.38677354709419\\
-0.18436873747495	-1.3801984290109\\
-0.191969329661105	-1.37074148296593\\
-0.200400801603207	-1.36021831785027\\
-0.204757187238359	-1.35470941883768\\
-0.216432865731463	-1.33989543890279\\
-0.21738088618959	-1.33867735470942\\
-0.229825446163611	-1.32264529058116\\
-0.232464929859719	-1.31923379942883\\
-0.242111014129311	-1.30661322645291\\
-0.248496993987976	-1.29822577640305\\
-0.254249679111395	-1.29058116232465\\
-0.264529058116232	-1.27686483474363\\
-0.266245197116871	-1.27454909819639\\
-0.278085913715293	-1.25851703406814\\
-0.280561122244489	-1.25515324083749\\
-0.289782528321357	-1.24248496993988\\
-0.296593186372745	-1.23308517177625\\
-0.301349186753381	-1.22645290581162\\
-0.312625250501002	-1.2106514005922\\
-0.312788164902552	-1.21042084168337\\
-0.324080466620025	-1.19438877755511\\
-0.328657314629258	-1.18785783358095\\
-0.33525274492881	-1.17835671342685\\
-0.344689378757515	-1.16468909288222\\
-0.346307219645143	-1.1623246492986\\
-0.357232011798649	-1.14629258517034\\
-0.360721442885771	-1.141146504018\\
-0.368039327265191	-1.13026052104208\\
-0.376753507014028	-1.11722056863035\\
-0.37873671886943	-1.11422845691383\\
-0.389314546384562	-1.09819639278557\\
-0.392785571142285	-1.09290786061383\\
-0.399781828147799	-1.08216432865731\\
-0.408817635270541	-1.06819939383122\\
-0.410145301707386	-1.06613226452906\\
-0.420395503992964	-1.0501002004008\\
-0.424849699398798	-1.04309121984003\\
-0.430543682829265	-1.03406813627255\\
-0.440592084630359	-1.01803607214429\\
-0.440881763527054	-1.0175719344274\\
-0.450533390184912	-1.00200400801603\\
-0.456913827655311	-0.991638126868775\\
-0.46037981639122	-0.985971943887776\\
-0.470127348333342	-0.969939879759519\\
-0.472945891783567	-0.965277406046435\\
-0.479778661095603	-0.953907815631263\\
-0.488977955911824	-0.938481885432904\\
-0.489337460739446	-0.937875751503006\\
-0.498800763532823	-0.92184368737475\\
-0.50501002004008	-0.911242184065805\\
-0.508174479282237	-0.905811623246493\\
-0.517457049057759	-0.889779559118236\\
-0.521042084168337	-0.883546681902311\\
-0.526651618119988	-0.87374749498998\\
-0.535757770323694	-0.857715430861723\\
-0.537074148296593	-0.855384760374438\\
-0.544778659699726	-0.841683366733467\\
-0.55310621242485	-0.826743996246779\\
-0.553712876870233	-0.82565130260521\\
-0.562564761636584	-0.809619238476954\\
-0.569138276553106	-0.797609863137293\\
-0.571332121958972	-0.793587174348697\\
-0.5800184813075	-0.777555110220441\\
-0.585170340681363	-0.767969139974112\\
-0.588623238808741	-0.761523046092184\\
-0.597147798796266	-0.745490981963928\\
-0.601202404809619	-0.737806525696426\\
-0.60559380677074	-0.729458917835671\\
-0.613960138273222	-0.713426853707415\\
-0.617234468937876	-0.707105294487803\\
-0.622250864274681	-0.697394789579158\\
-0.630462387886829	-0.681362725450902\\
-0.633266533066132	-0.675847212770085\\
-0.638600927584093	-0.665330661322646\\
-0.646660918238621	-0.649298597194389\\
-0.649298597194389	-0.644012448665355\\
-0.654650008190866	-0.633266533066132\\
-0.662561599507746	-0.617234468937876\\
-0.665330661322646	-0.611579473271343\\
-0.670403628910887	-0.601202404809619\\
-0.678169817286189	-0.585170340681363\\
-0.681362725450902	-0.578524953025125\\
-0.685866838737788	-0.569138276553106\\
-0.693490487181081	-0.55310621242485\\
-0.697394789579158	-0.544823632351483\\
-0.701044226507351	-0.537074148296593\\
-0.708528068235948	-0.521042084168337\\
-0.713426853707415	-0.510448205704374\\
-0.715939933420816	-0.50501002004008\\
-0.723286575218568	-0.488977955911824\\
-0.729458917835671	-0.475369178011717\\
-0.730557664471301	-0.472945891783567\\
-0.737769589818998	-0.456913827655311\\
-0.744902759144508	-0.440881763527054\\
-0.745490981963928	-0.439549388500093\\
-0.751980270797614	-0.424849699398798\\
-0.758981402159147	-0.408817635270541\\
-0.761523046092184	-0.402945164657456\\
-0.765921363119241	-0.392785571142285\\
-0.772792646047509	-0.376753507014028\\
-0.777555110220441	-0.365527472868787\\
-0.779595206106096	-0.360721442885771\\
-0.78633871591627	-0.344689378757515\\
-0.79300605003884	-0.328657314629258\\
-0.793587174348697	-0.327247893379991\\
-0.799621440602657	-0.312625250501002\\
-0.806163058730893	-0.296593186372745\\
-0.809619238476954	-0.288036544144153\\
-0.81264225875116	-0.280561122244489\\
-0.819059962112338	-0.264529058116232\\
-0.825403267700109	-0.248496993987976\\
-0.82565130260521	-0.247864371429404\\
-0.831697706153768	-0.232464929859719\\
-0.83791872660962	-0.216432865731463\\
-0.841683366733467	-0.206625295316765\\
-0.84407685759974	-0.200400801603207\\
-0.850177130537458	-0.18436873747495\\
-0.856204317512007	-0.168336673346694\\
-0.857715430861723	-0.164275435029404\\
-0.862178565765761	-0.152304609218437\\
-0.868086222771196	-0.13627254509018\\
-0.87374749498998	-0.120720775558426\\
-0.873922746025479	-0.120240480961924\\
-0.879712154483288	-0.104208416833667\\
-0.885429338962689	-0.0881763527054109\\
-0.889779559118236	-0.075829463852013\\
-0.891081352733647	-0.0721442885771544\\
-0.896681104837513	-0.0561122244488979\\
-0.902209517311712	-0.0400801603206413\\
-0.905811623246493	-0.0295057008813555\\
-0.907676036994777	-0.0240480961923848\\
-0.913087586458924	-0.00801603206412826\\
-0.918428489847537	0.00801603206412782\\
-0.92184368737475	0.0183980272195441\\
-0.923708101123034	0.0240480961923843\\
-0.928932465899462	0.0400801603206409\\
-0.934086689113959	0.0561122244488974\\
-0.937875751503006	0.0680549808883089\\
-0.939177545118416	0.0721442885771539\\
-0.944215309975965	0.0881763527054105\\
-0.949183249838047	0.104208416833667\\
-0.953907815631263	0.119669068517051\\
-0.954083066666762	0.120240480961924\\
-0.958934384098946	0.13627254509018\\
-0.963716005591637	0.152304609218437\\
-0.968428766409803	0.168336673346693\\
-0.969939879759519	0.173551239282139\\
-0.973086645577692	0.18436873747495\\
-0.977681480427534	0.200400801603206\\
-0.982207303763929	0.216432865731463\\
-0.985971943887776	0.229972659443534\\
-0.986667731865482	0.232464929859719\\
-0.991074875445203	0.248496993987976\\
-0.995412667523159	0.264529058116232\\
-0.99968171568218	0.280561122244489\\
-1.00200400801603	0.289425026676694\\
-1.00389005021323	0.296593186372745\\
-1.00803827426999	0.312625250501002\\
-1.01211712930511	0.328657314629258\\
-1.01612708530091	0.344689378757515\\
-1.01803607214429	0.352457101605714\\
-1.02007616802994	0.360721442885771\\
-1.0239626482088	0.376753507014028\\
-1.02777931323051	0.392785571142285\\
-1.03152649233951	0.408817635270541\\
-1.03406813627255	0.419900398966127\\
-1.03520845019296	0.424849699398798\\
-1.03882897978193	0.440881763527054\\
-1.04237880825587	0.456913827655311\\
-1.04585811988899	0.472945891783567\\
-1.04926705264232	0.488977955911824\\
-1.0501002004008	0.492983895502289\\
-1.0526132801142	0.50501002004008\\
-1.05589018644029	0.521042084168337\\
-1.0590950852347	0.537074148296593\\
-1.06222796213098	0.55310621242485\\
-1.06528875386917	0.569138276553106\\
-1.06613226452906	0.57367075089988\\
-1.06828280652761	0.585170340681363\\
-1.0712052321173	0.601202404809619\\
-1.07405350670123	0.617234468937876\\
-1.07682740507046	0.633266533066132\\
-1.07952664970155	0.649298597194389\\
-1.08215090986002	0.665330661322646\\
-1.08216432865731	0.665415321848262\\
-1.08470439642679	0.681362725450902\\
-1.08718072399412	0.697394789579159\\
-1.08957941337184	0.713426853707415\\
-1.09189995632039	0.729458917835672\\
-1.09414178677222	0.745490981963928\\
-1.09630427959414	0.761523046092185\\
-1.09819639278557	0.776094740459559\\
-1.09838695251236	0.777555110220441\\
-1.10039023819144	0.793587174348698\\
-1.10231088468654	0.809619238476954\\
-1.10414806928106	0.825651302605211\\
-1.1059009041887	0.841683366733467\\
-1.10756843491448	0.857715430861724\\
-1.10914963852203	0.87374749498998\\
-1.11064342180325	0.889779559118236\\
-1.11204861934615	0.905811623246493\\
-1.11336399149656	0.921843687374749\\
-1.11422845691383	0.933182006967935\\
-1.11458796174145	0.937875751503006\\
-1.1157184454276	0.953907815631262\\
-1.11675443604705	0.969939879759519\\
-1.11769444564974	0.985971943887775\\
-1.11853690276846	1.00200400801603\\
-1.11928014981559	1.01803607214429\\
-1.11992244034429	1.03406813627254\\
-1.12046193616776	1.0501002004008\\
-1.12089670432974	1.06613226452906\\
-1.12122471391934	1.08216432865731\\
-1.12144383272227	1.09819639278557\\
-1.12155182370072	1.11422845691383\\
-1.12154634129325	1.13026052104208\\
-1.12142492752561	1.14629258517034\\
-1.12118500792299	1.1623246492986\\
-1.12082388721338	1.17835671342685\\
-1.12033874481139	1.19438877755511\\
-1.1197266300709	1.21042084168337\\
-1.11898445729446	1.22645290581162\\
-1.11810900048634	1.24248496993988\\
-1.11709688783561	1.25851703406814\\
-1.11594459591447	1.27454909819639\\
-1.11464844357632	1.29058116232465\\
-1.11422845691383	1.29527490685972\\
}--cycle;


\addplot[area legend,solid,fill=mycolor8,draw=black,forget plot]
table[row sep=crcr] {%
x	y\\
-0.969939879759519	1.10267832172919\\
-0.969057219367736	1.11422845691383\\
-0.967666023500817	1.13026052104208\\
-0.96610008118631	1.14629258517034\\
-0.964353882669638	1.1623246492986\\
-0.962421651452387	1.17835671342685\\
-0.960297331194264	1.19438877755511\\
-0.957974571822777	1.21042084168337\\
-0.955446714796597	1.22645290581162\\
-0.953907815631263	1.23550430311173\\
-0.95268652032031	1.24248496993988\\
-0.949675465068525	1.25851703406814\\
-0.946430689407037	1.27454909819639\\
-0.942943626489223	1.29058116232465\\
-0.939205278892995	1.30661322645291\\
-0.937875751503006	1.31199129212065\\
-0.935151423287963	1.32264529058116\\
-0.930790674073424	1.33867735470942\\
-0.926138506431433	1.35470941883768\\
-0.92184368737475	1.36863159698073\\
-0.921167596555201	1.37074148296593\\
-0.91576922275681	1.38677354709419\\
-0.910028348601568	1.40280561122244\\
-0.905811623246493	1.4139460396668\\
-0.903879403874429	1.4188376753507\\
-0.89722860759457	1.43486973947896\\
-0.890169387584158	1.45090180360721\\
-0.889779559118236	1.45175054669336\\
-0.882464023832977	1.46693386773547\\
-0.874275126038673	1.48296593186373\\
-0.87374749498998	1.4839573570158\\
-0.865310295946134	1.49899799599198\\
-0.857715430861723	1.51182056797403\\
-0.85570327730774	1.51503006012024\\
-0.845197697745205	1.5310621242485\\
-0.841683366733467	1.5361856585203\\
-0.83371642719017	1.54709418837675\\
-0.82565130260521	1.55763141692518\\
-0.821149424312379	1.56312625250501\\
-0.809619238476954	1.57659144839573\\
-0.807252347724585	1.57915831663327\\
-0.793587174348697	1.59337379320406\\
-0.791693855339852	1.59519038076152\\
-0.777555110220441	1.60823431958121\\
-0.774015623754877	1.61122244488978\\
-0.761523046092184	1.62138648351191\\
-0.753568053904924	1.62725450901804\\
-0.745490981963928	1.6330089628635\\
-0.729458917835671	1.64325132499136\\
-0.729397545610944	1.64328657314629\\
-0.713426853707415	1.65217400092951\\
-0.698733223512349	1.65931863727455\\
-0.697394789579158	1.65995043575415\\
-0.681362725450902	1.66660099071443\\
-0.665330661322646	1.67225062416767\\
-0.654827748809892	1.67535070140281\\
-0.649298597194389	1.67694127241858\\
-0.633266533066132	1.68072001757187\\
-0.617234468937876	1.6836551377129\\
-0.601202404809619	1.68578743348757\\
-0.585170340681363	1.68715450310275\\
-0.569138276553106	1.6877909773566\\
-0.55310621242485	1.68772873313515\\
-0.537074148296593	1.68699708749149\\
-0.521042084168337	1.68562297418631\\
-0.50501002004008	1.68363110436025\\
-0.488977955911824	1.681044112825\\
-0.472945891783567	1.67788269129889\\
-0.462062759340567	1.67535070140281\\
-0.456913827655311	1.67416168445254\\
-0.440881763527054	1.66989348069114\\
-0.424849699398798	1.66510347263781\\
-0.408817635270541	1.65980612744266\\
-0.40747920133735	1.65931863727455\\
-0.392785571142285	1.65400113858226\\
-0.376753507014028	1.64771432457223\\
-0.366276833404529	1.64328657314629\\
-0.360721442885771	1.64095253996619\\
-0.344689378757515	1.63372219754025\\
-0.331202431040213	1.62725450901804\\
-0.328657314629258	1.62604056421676\\
-0.312625250501002	1.61790654165211\\
-0.300132672838309	1.61122244488978\\
-0.296593186372745	1.60933788456875\\
-0.280561122244489	1.60033434295526\\
-0.271830152307451	1.59519038076152\\
-0.264529058116232	1.59090760245724\\
-0.248496993987976	1.5810633694857\\
-0.245529404938141	1.57915831663327\\
-0.232464929859719	1.57080380915392\\
-0.220934744024294	1.56312625250501\\
-0.216432865731463	1.56013937290885\\
-0.200400801603207	1.54907207324439\\
-0.19764255158661	1.54709418837675\\
-0.18436873747495	1.53760511938098\\
-0.175535084883226	1.5310621242485\\
-0.168336673346694	1.52574538852246\\
-0.15431676277242	1.51503006012024\\
-0.152304609218437	1.51349613165084\\
-0.13627254509018	1.50085872488534\\
-0.133985795284029	1.49899799599198\\
-0.120240480961924	1.48783640452733\\
-0.11442157751627	1.48296593186373\\
-0.104208416833667	1.47443279432367\\
-0.0954918879906701	1.46693386773547\\
-0.0881763527054109	1.46064992979705\\
-0.0771441629499948	1.45090180360721\\
-0.0721442885771544	1.44648955083659\\
-0.0593315894323775	1.43486973947896\\
-0.0561122244488979	1.4319531039411\\
-0.0420123796927054	1.4188376753507\\
-0.0400801603206413	1.41704174408248\\
-0.0251491528545071	1.40280561122244\\
-0.0240480961923848	1.40175633619233\\
-0.00870840581046833	1.38677354709419\\
-0.00801603206412826	1.38609745627464\\
0.00733994124458037	1.37074148296593\\
0.00801603206412782	1.37006539214638\\
0.0230229365910116	1.35470941883768\\
0.0240480961923843	1.35366014380756\\
0.0383649358792694	1.33867735470942\\
0.0400801603206409	1.33688142344119\\
0.0533878962338543	1.32264529058116\\
0.0561122244488974	1.3197286550433\\
0.0681116322727954	1.30661322645291\\
0.0721442885771539	1.30220097368228\\
0.0825540396247736	1.29058116232465\\
0.0881763527054105	1.28429722438623\\
0.0967312906094415	1.27454909819639\\
0.104208416833667	1.26601596065633\\
0.110658005995674	1.25851703406814\\
0.120240480961924	1.24735544260349\\
0.124347406134783	1.24248496993988\\
0.13627254509018	1.22831363470498\\
0.137811444255516	1.22645290581162\\
0.151041518246361	1.21042084168337\\
0.152304609218437	1.20888691321397\\
0.164043027344138	1.19438877755511\\
0.168336673346693	1.18907204182907\\
0.176850509167818	1.17835671342685\\
0.18436873747495	1.16886764443108\\
0.189472282332026	1.1623246492986\\
0.200400801603206	1.14827047003798\\
0.201915825520064	1.14629258517034\\
0.21415900947276	1.13026052104208\\
0.216432865731463	1.12727364144593\\
0.226219685290747	1.11422845691383\\
0.232464929859719	1.10587394943448\\
0.238124052009175	1.09819639278557\\
0.248496993987976	1.08406938150975\\
0.249877068771455	1.08216432865731\\
0.261449565491779	1.06613226452906\\
0.264529058116232	1.06184948622477\\
0.272867526498327	1.0501002004008\\
0.280561122244489	1.03921209846628\\
0.284150637741463	1.03406813627254\\
0.295289049977966	1.01803607214429\\
0.296593186372745	1.01615151182326\\
0.306265440129244	1.00200400801603\\
0.312625250501002	0.992656040650101\\
0.317120335276949	0.985971943887775\\
0.327848425282381	0.969939879759519\\
0.328657314629258	0.968725934958247\\
0.338421193847936	0.953907815631262\\
0.344689378757515	0.944343440025218\\
0.348883112072032	0.937875751503006\\
0.359223219238986	0.921843687374749\\
0.360721442885771	0.919509654194643\\
0.369424311362763	0.905811623246493\\
0.376753507014028	0.894207310544178\\
0.37952284977305	0.889779559118236\\
0.389495963689498	0.87374749498998\\
0.392785571142285	0.86842999629769\\
0.399352061568488	0.857715430861724\\
0.408817635270541	0.842170856901581\\
0.409111855743252	0.841683366733467\\
0.418737689933317	0.825651302605211\\
0.424849699398798	0.81540407384021\\
0.428270836853321	0.809619238476954\\
0.437693808705884	0.793587174348698\\
0.440881763527054	0.788129953637036\\
0.447009199199949	0.777555110220441\\
0.456233617637009	0.761523046092185\\
0.456913827655311	0.760334029141922\\
0.465339483189213	0.745490981963928\\
0.472945891783567	0.731990907731753\\
0.474362142761551	0.729458917835672\\
0.48327342673314	0.713426853707415\\
0.488977955911824	0.703088201001349\\
0.492097725170469	0.697394789579159\\
0.500821999933126	0.681362725450902\\
0.50501002004008	0.673611064280093\\
0.509454470646171	0.665330661322646\\
0.51799543742193	0.649298597194389\\
0.521042084168337	0.643538805849636\\
0.526442067692356	0.633266533066132\\
0.534803268494776	0.617234468937876\\
0.537074148296593	0.612848790898299\\
0.543069523425128	0.601202404809619\\
0.551254345150408	0.585170340681363\\
0.55310621242485	0.581516308285451\\
0.559345189869164	0.569138276553106\\
0.567356868153898	0.55310621242485\\
0.569138276553106	0.549514424250385\\
0.575276788374131	0.537074148296593\\
0.583118411224556	0.521042084168337\\
0.585170340681363	0.516813821740019\\
0.590871432247094	0.50501002004008\\
0.598545943444414	0.488977955911824\\
0.601202404809619	0.483382623868328\\
0.606135647689298	0.472945891783567\\
0.613645849975363	0.456913827655311\\
0.617234468937876	0.44918619983715\\
0.621075393118784	0.440881763527054\\
0.62842395116611	0.424849699398798\\
0.633266533066132	0.414186951439601\\
0.635696076953843	0.408817635270541\\
0.642885520123645	0.392785571142285\\
0.649298597194389	0.378344078029799\\
0.650002573926212	0.376753507014028\\
0.657035298817787	0.360721442885771\\
0.663996859551543	0.344689378757515\\
0.665330661322646	0.341589301522375\\
0.670877512781642	0.328657314629258\\
0.677684346364305	0.312625250501002\\
0.681362725450902	0.303875539812623\\
0.684415884465412	0.296593186372745\\
0.691070078445277	0.280561122244489\\
0.697394789579159	0.265160856595837\\
0.697653645295825	0.264529058116232\\
0.704157161442419	0.248496993987976\\
0.71058999818202	0.232464929859719\\
0.713426853707415	0.225320293514676\\
0.716948222425681	0.216432865731463\\
0.723231725412169	0.200400801603206\\
0.729445402395971	0.18436873747495\\
0.729458917835672	0.184333489320014\\
0.7355812556841	0.168336673346693\\
0.741646656558198	0.152304609218437\\
0.745490981963928	0.142026998935648\\
0.747640164685551	0.13627254509018\\
0.753558788829143	0.120240480961924\\
0.759407488220804	0.104208416833667\\
0.761523046092185	0.0983403913275378\\
0.765182801678966	0.0881763527054105\\
0.77088533435874	0.0721442885771539\\
0.776518228317745	0.0561122244488974\\
0.777555110220441	0.053124099140325\\
0.782076830845543	0.0400801603206409\\
0.787563890959639	0.0240480961923843\\
0.792981374834583	0.00801603206412782\\
0.793587174348698	0.00619944450666274\\
0.798324247801242	-0.00801603206412826\\
0.803595955087895	-0.0240480961923848\\
0.80879792546851	-0.0400801603206413\\
0.809619238476954	-0.0426470285581829\\
0.81392555091599	-0.0561122244488979\\
0.818981526743509	-0.0721442885771544\\
0.823967381477785	-0.0881763527054109\\
0.825651302605211	-0.0936711882852378\\
0.828879743236386	-0.104208416833667\\
0.833719109470425	-0.120240480961924\\
0.838487745757591	-0.13627254509018\\
0.841683366733467	-0.14718107494664\\
0.843184328639133	-0.152304609218437\\
0.847805704581895	-0.168336673346694\\
0.852355515131915	-0.18436873747495\\
0.856834199960817	-0.200400801603207\\
0.857715430861724	-0.203610293749422\\
0.86123679957999	-0.216432865731463\\
0.865565666819784	-0.232464929859719\\
0.869822225003994	-0.248496993987976\\
0.87374749498998	-0.263537632964165\\
0.874006350706646	-0.264529058116232\\
0.878111639002643	-0.280561122244489\\
0.8821431997404	-0.296593186372745\\
0.88610118003164	-0.312625250501002\\
0.889779559118236	-0.32780857154311\\
0.889985305387594	-0.328657314629258\\
0.893788465297644	-0.344689378757515\\
0.897516260741634	-0.360721442885771\\
0.901168657328444	-0.376753507014028\\
0.904745563887746	-0.392785571142285\\
0.905811623246493	-0.397677206826184\\
0.908241167134204	-0.408817635270541\\
0.911657017285245	-0.424849699398798\\
0.914995079503241	-0.440881763527054\\
0.918255068412237	-0.456913827655311\\
0.921436637000515	-0.472945891783567\\
0.921843687374749	-0.47505577776877\\
0.924531625014911	-0.488977955911824\\
0.92754477894048	-0.50501002004008\\
0.93047666597542	-0.521042084168337\\
0.933326728992571	-0.537074148296593\\
0.936094343103798	-0.55310621242485\\
0.937875751503006	-0.563760210885365\\
0.938775714503375	-0.569138276553106\\
0.941366364051953	-0.585170340681363\\
0.943871126631541	-0.601202404809619\\
0.946289147749201	-0.617234468937876\\
0.94861949758841	-0.633266533066132\\
0.950861168884856	-0.649298597194389\\
0.953013074674022	-0.665330661322646\\
0.953907815631262	-0.672311328150791\\
0.955068713952377	-0.681362725450902\\
0.957027584889907	-0.697394789579158\\
0.958892252810273	-0.713426853707415\\
0.960661363087028	-0.729458917835671\\
0.962333471165166	-0.745490981963928\\
0.963907039627009	-0.761523046092184\\
0.965380435091499	-0.777555110220441\\
0.966751924938348	-0.793587174348697\\
0.968019673847939	-0.809619238476954\\
0.969181740147287	-0.82565130260521\\
0.969939879759519	-0.837201437789848\\
0.97023410023223	-0.841683366733467\\
0.971171862150369	-0.857715430861723\\
0.971997666920664	-0.87374749498998\\
0.972709222518539	-0.889779559118236\\
0.973304114743161	-0.905811623246493\\
0.973779802541407	-0.92184368737475\\
0.974133613074036	-0.937875751503006\\
0.974362736509592	-0.953907815631263\\
0.974464220530582	-0.969939879759519\\
0.974434964535466	-0.985971943887776\\
0.974271713518829	-1.00200400801603\\
0.973971051610933	-1.01803607214429\\
0.973529395256493	-1.03406813627255\\
0.972942986011204	-1.0501002004008\\
0.97220788293294	-1.06613226452906\\
0.971319954542998	-1.08216432865731\\
0.970274870330963	-1.09819639278557\\
0.969939879759519	-1.10267832172919\\
0.969057219367736	-1.11422845691383\\
0.967666023500816	-1.13026052104208\\
0.96610008118631	-1.14629258517034\\
0.964353882669638	-1.1623246492986\\
0.962421651452387	-1.17835671342685\\
0.960297331194265	-1.19438877755511\\
0.957974571822777	-1.21042084168337\\
0.955446714796598	-1.22645290581162\\
0.953907815631262	-1.23550430311173\\
0.95268652032031	-1.24248496993988\\
0.949675465068525	-1.25851703406814\\
0.946430689407037	-1.27454909819639\\
0.942943626489224	-1.29058116232465\\
0.939205278892995	-1.30661322645291\\
0.937875751503006	-1.31199129212065\\
0.935151423287963	-1.32264529058116\\
0.930790674073424	-1.33867735470942\\
0.926138506431433	-1.35470941883768\\
0.921843687374749	-1.36863159698073\\
0.921167596555201	-1.37074148296593\\
0.91576922275681	-1.38677354709419\\
0.910028348601568	-1.40280561122244\\
0.905811623246493	-1.4139460396668\\
0.903879403874429	-1.4188376753507\\
0.897228607594571	-1.43486973947896\\
0.890169387584157	-1.45090180360721\\
0.889779559118236	-1.45175054669336\\
0.882464023832978	-1.46693386773547\\
0.874275126038673	-1.48296593186373\\
0.87374749498998	-1.4839573570158\\
0.865310295946134	-1.49899799599198\\
0.857715430861724	-1.51182056797403\\
0.85570327730774	-1.51503006012024\\
0.845197697745205	-1.5310621242485\\
0.841683366733467	-1.53618565852029\\
0.83371642719017	-1.54709418837675\\
0.825651302605211	-1.55763141692518\\
0.821149424312379	-1.56312625250501\\
0.809619238476954	-1.57659144839572\\
0.807252347724586	-1.57915831663327\\
0.793587174348698	-1.59337379320406\\
0.791693855339852	-1.59519038076152\\
0.777555110220441	-1.60823431958121\\
0.774015623754877	-1.61122244488978\\
0.761523046092185	-1.62138648351191\\
0.753568053904925	-1.62725450901804\\
0.745490981963928	-1.6330089628635\\
0.729458917835672	-1.64325132499136\\
0.729397545610944	-1.64328657314629\\
0.713426853707415	-1.65217400092951\\
0.698733223512349	-1.65931863727455\\
0.697394789579159	-1.65995043575415\\
0.681362725450902	-1.66660099071443\\
0.665330661322646	-1.67225062416767\\
0.654827748809895	-1.67535070140281\\
0.649298597194389	-1.67694127241858\\
0.633266533066132	-1.68072001757187\\
0.617234468937876	-1.6836551377129\\
0.601202404809619	-1.68578743348757\\
0.585170340681363	-1.68715450310274\\
0.569138276553106	-1.6877909773566\\
0.55310621242485	-1.68772873313515\\
0.537074148296593	-1.68699708749149\\
0.521042084168337	-1.68562297418631\\
0.50501002004008	-1.68363110436025\\
0.488977955911824	-1.681044112825\\
0.472945891783567	-1.67788269129889\\
0.462062759340563	-1.67535070140281\\
0.456913827655311	-1.67416168445254\\
0.440881763527054	-1.66989348069114\\
0.424849699398798	-1.66510347263781\\
0.408817635270541	-1.65980612744266\\
0.407479201337349	-1.65931863727455\\
0.392785571142285	-1.65400113858226\\
0.376753507014028	-1.64771432457223\\
0.366276833404527	-1.64328657314629\\
0.360721442885771	-1.64095253996619\\
0.344689378757515	-1.63372219754025\\
0.331202431040212	-1.62725450901804\\
0.328657314629258	-1.62604056421676\\
0.312625250501002	-1.61790654165211\\
0.300132672838309	-1.61122244488978\\
0.296593186372745	-1.60933788456875\\
0.280561122244489	-1.60033434295526\\
0.27183015230745	-1.59519038076152\\
0.264529058116232	-1.59090760245724\\
0.248496993987976	-1.5810633694857\\
0.245529404938141	-1.57915831663327\\
0.232464929859719	-1.57080380915392\\
0.220934744024294	-1.56312625250501\\
0.216432865731463	-1.56013937290885\\
0.200400801603206	-1.54907207324439\\
0.19764255158661	-1.54709418837675\\
0.18436873747495	-1.53760511938098\\
0.175535084883226	-1.5310621242485\\
0.168336673346693	-1.52574538852245\\
0.15431676277242	-1.51503006012024\\
0.152304609218437	-1.51349613165084\\
0.13627254509018	-1.50085872488534\\
0.133985795284029	-1.49899799599198\\
0.120240480961924	-1.48783640452733\\
0.11442157751627	-1.48296593186373\\
0.104208416833667	-1.47443279432367\\
0.0954918879906697	-1.46693386773547\\
0.0881763527054105	-1.46064992979705\\
0.0771441629499948	-1.45090180360721\\
0.0721442885771539	-1.44648955083659\\
0.0593315894323771	-1.43486973947896\\
0.0561122244488974	-1.4319531039411\\
0.042012379692705	-1.4188376753507\\
0.0400801603206409	-1.41704174408247\\
0.025149152854508	-1.40280561122244\\
0.0240480961923843	-1.40175633619233\\
0.0087084058104688	-1.38677354709419\\
0.00801603206412782	-1.38609745627464\\
-0.0073399412445799	-1.37074148296593\\
-0.00801603206412826	-1.37006539214638\\
-0.023022936591012	-1.35470941883768\\
-0.0240480961923848	-1.35366014380756\\
-0.0383649358792694	-1.33867735470942\\
-0.0400801603206413	-1.33688142344119\\
-0.0533878962338543	-1.32264529058116\\
-0.0561122244488979	-1.3197286550433\\
-0.0681116322727958	-1.30661322645291\\
-0.0721442885771544	-1.30220097368228\\
-0.0825540396247731	-1.29058116232465\\
-0.0881763527054109	-1.28429722438623\\
-0.0967312906094415	-1.27454909819639\\
-0.104208416833667	-1.26601596065633\\
-0.110658005995674	-1.25851703406814\\
-0.120240480961924	-1.24735544260349\\
-0.124347406134783	-1.24248496993988\\
-0.13627254509018	-1.22831363470498\\
-0.137811444255515	-1.22645290581162\\
-0.151041518246361	-1.21042084168337\\
-0.152304609218437	-1.20888691321397\\
-0.164043027344137	-1.19438877755511\\
-0.168336673346694	-1.18907204182907\\
-0.176850509167818	-1.17835671342685\\
-0.18436873747495	-1.16886764443108\\
-0.189472282332025	-1.1623246492986\\
-0.200400801603207	-1.14827047003798\\
-0.201915825520063	-1.14629258517034\\
-0.214159009472761	-1.13026052104208\\
-0.216432865731463	-1.12727364144593\\
-0.226219685290747	-1.11422845691383\\
-0.232464929859719	-1.10587394943448\\
-0.238124052009175	-1.09819639278557\\
-0.248496993987976	-1.08406938150975\\
-0.249877068771455	-1.08216432865731\\
-0.261449565491779	-1.06613226452906\\
-0.264529058116232	-1.06184948622477\\
-0.272867526498328	-1.0501002004008\\
-0.280561122244489	-1.03921209846628\\
-0.284150637741463	-1.03406813627255\\
-0.295289049977966	-1.01803607214429\\
-0.296593186372745	-1.01615151182326\\
-0.306265440129243	-1.00200400801603\\
-0.312625250501002	-0.992656040650101\\
-0.317120335276949	-0.985971943887776\\
-0.327848425282381	-0.969939879759519\\
-0.328657314629258	-0.968725934958247\\
-0.338421193847936	-0.953907815631263\\
-0.344689378757515	-0.944343440025219\\
-0.348883112072033	-0.937875751503006\\
-0.359223219238986	-0.92184368737475\\
-0.360721442885771	-0.919509654194644\\
-0.369424311362763	-0.905811623246493\\
-0.376753507014028	-0.894207310544178\\
-0.379522849773049	-0.889779559118236\\
-0.389495963689499	-0.87374749498998\\
-0.392785571142285	-0.86842999629769\\
-0.399352061568489	-0.857715430861723\\
-0.408817635270541	-0.84217085690158\\
-0.409111855743253	-0.841683366733467\\
-0.418737689933318	-0.82565130260521\\
-0.424849699398798	-0.81540407384021\\
-0.428270836853321	-0.809619238476954\\
-0.437693808705884	-0.793587174348697\\
-0.440881763527054	-0.788129953637036\\
-0.447009199199949	-0.777555110220441\\
-0.45623361763701	-0.761523046092184\\
-0.456913827655311	-0.760334029141922\\
-0.465339483189214	-0.745490981963928\\
-0.472945891783567	-0.731990907731753\\
-0.474362142761551	-0.729458917835671\\
-0.48327342673314	-0.713426853707415\\
-0.488977955911824	-0.703088201001348\\
-0.49209772517047	-0.697394789579158\\
-0.500821999933126	-0.681362725450902\\
-0.50501002004008	-0.673611064280093\\
-0.509454470646172	-0.665330661322646\\
-0.51799543742193	-0.649298597194389\\
-0.521042084168337	-0.643538805849636\\
-0.526442067692356	-0.633266533066132\\
-0.534803268494776	-0.617234468937876\\
-0.537074148296593	-0.612848790898299\\
-0.543069523425128	-0.601202404809619\\
-0.551254345150408	-0.585170340681363\\
-0.55310621242485	-0.581516308285452\\
-0.559345189869164	-0.569138276553106\\
-0.567356868153898	-0.55310621242485\\
-0.569138276553106	-0.549514424250385\\
-0.57527678837413	-0.537074148296593\\
-0.583118411224556	-0.521042084168337\\
-0.585170340681363	-0.516813821740019\\
-0.590871432247094	-0.50501002004008\\
-0.598545943444414	-0.488977955911824\\
-0.601202404809619	-0.483382623868328\\
-0.606135647689298	-0.472945891783567\\
-0.613645849975363	-0.456913827655311\\
-0.617234468937876	-0.449186199837151\\
-0.621075393118784	-0.440881763527054\\
-0.62842395116611	-0.424849699398798\\
-0.633266533066132	-0.414186951439601\\
-0.635696076953843	-0.408817635270541\\
-0.642885520123645	-0.392785571142285\\
-0.649298597194389	-0.378344078029798\\
-0.650002573926212	-0.376753507014028\\
-0.657035298817786	-0.360721442885771\\
-0.663996859551543	-0.344689378757515\\
-0.665330661322646	-0.341589301522375\\
-0.670877512781641	-0.328657314629258\\
-0.677684346364305	-0.312625250501002\\
-0.681362725450902	-0.303875539812625\\
-0.684415884465412	-0.296593186372745\\
-0.691070078445277	-0.280561122244489\\
-0.697394789579158	-0.265160856595838\\
-0.697653645295824	-0.264529058116232\\
-0.704157161442419	-0.248496993987976\\
-0.710589998182019	-0.232464929859719\\
-0.713426853707415	-0.225320293514677\\
-0.716948222425681	-0.216432865731463\\
-0.723231725412169	-0.200400801603207\\
-0.729445402395971	-0.18436873747495\\
-0.729458917835671	-0.184333489320016\\
-0.735581255684099	-0.168336673346694\\
-0.741646656558198	-0.152304609218437\\
-0.745490981963928	-0.142026998935649\\
-0.747640164685551	-0.13627254509018\\
-0.753558788829143	-0.120240480961924\\
-0.759407488220804	-0.104208416833667\\
-0.761523046092184	-0.0983403913275397\\
-0.765182801678965	-0.0881763527054109\\
-0.770885334358739	-0.0721442885771544\\
-0.776518228317744	-0.0561122244488979\\
-0.777555110220441	-0.0531240991403255\\
-0.782076830845542	-0.0400801603206413\\
-0.787563890959638	-0.0240480961923848\\
-0.792981374834583	-0.00801603206412826\\
-0.793587174348697	-0.00619944450666455\\
-0.798324247801242	0.00801603206412782\\
-0.803595955087895	0.0240480961923843\\
-0.80879792546851	0.0400801603206409\\
-0.809619238476954	0.0426470285581824\\
-0.81392555091599	0.0561122244488974\\
-0.818981526743509	0.0721442885771539\\
-0.823967381477785	0.0881763527054105\\
-0.82565130260521	0.0936711882852369\\
-0.828879743236386	0.104208416833667\\
-0.833719109470425	0.120240480961924\\
-0.838487745757591	0.13627254509018\\
-0.841683366733467	0.147181074946638\\
-0.843184328639133	0.152304609218437\\
-0.847805704581895	0.168336673346693\\
-0.852355515131915	0.18436873747495\\
-0.856834199960817	0.200400801603206\\
-0.857715430861723	0.20361029374942\\
-0.861236799579989	0.216432865731463\\
-0.865565666819784	0.232464929859719\\
-0.869822225003994	0.248496993987976\\
-0.87374749498998	0.263537632964162\\
-0.874006350706646	0.264529058116232\\
-0.878111639002643	0.280561122244489\\
-0.8821431997404	0.296593186372745\\
-0.88610118003164	0.312625250501002\\
-0.889779559118236	0.327808571543111\\
-0.889985305387594	0.328657314629258\\
-0.893788465297644	0.344689378757515\\
-0.897516260741634	0.360721442885771\\
-0.901168657328444	0.376753507014028\\
-0.904745563887746	0.392785571142285\\
-0.905811623246493	0.397677206826187\\
-0.908241167134204	0.408817635270541\\
-0.911657017285245	0.424849699398798\\
-0.914995079503241	0.440881763527054\\
-0.918255068412237	0.456913827655311\\
-0.921436637000515	0.472945891783567\\
-0.92184368737475	0.475055777768773\\
-0.924531625014912	0.488977955911824\\
-0.927544778940481	0.50501002004008\\
-0.930476665975421	0.521042084168337\\
-0.933326728992572	0.537074148296593\\
-0.936094343103798	0.55310621242485\\
-0.937875751503006	0.563760210885367\\
-0.938775714503375	0.569138276553106\\
-0.941366364051952	0.585170340681363\\
-0.943871126631541	0.601202404809619\\
-0.9462891477492	0.617234468937876\\
-0.94861949758841	0.633266533066132\\
-0.950861168884856	0.649298597194389\\
-0.953013074674022	0.665330661322646\\
-0.953907815631263	0.672311328150798\\
-0.955068713952377	0.681362725450902\\
-0.957027584889908	0.697394789579159\\
-0.958892252810273	0.713426853707415\\
-0.960661363087028	0.729458917835672\\
-0.962333471165166	0.745490981963928\\
-0.963907039627009	0.761523046092185\\
-0.965380435091499	0.777555110220441\\
-0.966751924938348	0.793587174348698\\
-0.968019673847939	0.809619238476954\\
-0.969181740147288	0.825651302605211\\
-0.969939879759519	0.837201437789848\\
-0.970234100232231	0.841683366733467\\
-0.971171862150368	0.857715430861724\\
-0.971997666920663	0.87374749498998\\
-0.97270922251854	0.889779559118236\\
-0.973304114743161	0.905811623246493\\
-0.973779802541406	0.921843687374749\\
-0.974133613074036	0.937875751503006\\
-0.974362736509593	0.953907815631262\\
-0.974464220530582	0.969939879759519\\
-0.974434964535466	0.985971943887775\\
-0.97427171351883	1.00200400801603\\
-0.973971051610933	1.01803607214429\\
-0.973529395256494	1.03406813627254\\
-0.972942986011204	1.0501002004008\\
-0.97220788293294	1.06613226452906\\
-0.971319954542998	1.08216432865731\\
-0.970274870330964	1.09819639278557\\
-0.969939879759519	1.10267832172919\\
}--cycle;


\addplot[area legend,solid,fill=mycolor9,draw=black,forget plot]
table[row sep=crcr] {%
x	y\\
-0.82565130260521	0.866245169292337\\
-0.825462060494201	0.87374749498998\\
-0.824887980644687	0.889779559118236\\
-0.824136376831622	0.905811623246493\\
-0.823201350070427	0.921843687374749\\
-0.822076686092449	0.937875751503006\\
-0.820755838230285	0.953907815631262\\
-0.819231909149522	0.969939879759519\\
-0.817497631338341	0.985971943887775\\
-0.815545346258589	1.00200400801603\\
-0.813366982053292	1.01803607214429\\
-0.810954029695988	1.03406813627254\\
-0.809619238476954	1.042176361739\\
-0.80826563518729	1.0501002004008\\
-0.805283581699165	1.06613226452906\\
-0.802028601208808	1.08216432865731\\
-0.79848934370967	1.09819639278557\\
-0.794653812367824	1.11422845691383\\
-0.793587174348697	1.11840086240547\\
-0.79042052310809	1.13026052104208\\
-0.785819903998319	1.14629258517034\\
-0.780866613022514	1.1623246492986\\
-0.777555110220441	1.17236205496192\\
-0.775478014964858	1.17835671342685\\
-0.769575432731745	1.19438877755511\\
-0.763245346289689	1.21042084168337\\
-0.761523046092184	1.21454979020275\\
-0.75627569446411	1.22645290581162\\
-0.748740773012284	1.24248496993988\\
-0.745490981963928	1.24902269292138\\
-0.7404736102047	1.25851703406814\\
-0.731474334627966	1.27454909819639\\
-0.729458917835671	1.27796128310748\\
-0.721480456734734	1.29058116232465\\
-0.713426853707415	1.30259251812186\\
-0.710523412722237	1.30661322645291\\
-0.698314123476152	1.32264529058116\\
-0.697394789579158	1.32379662789258\\
-0.684481340561341	1.33867735470942\\
-0.681362725450902	1.34209568133149\\
-0.66874898794471	1.35470941883768\\
-0.665330661322646	1.35797099976621\\
-0.650511183447168	1.37074148296593\\
-0.649298597194389	1.37174138065302\\
-0.633266533066132	1.38361953634077\\
-0.62845300893667	1.38677354709419\\
-0.617234468937876	1.39383666296571\\
-0.601202404809619	1.4025892937035\\
-0.600745994797927	1.40280561122244\\
-0.585170340681363	1.40992573720582\\
-0.569138276553106	1.4160452009309\\
-0.560232538149745	1.4188376753507\\
-0.55310621242485	1.42100045734942\\
-0.537074148296593	1.42486816501932\\
-0.521042084168337	1.42774204570758\\
-0.50501002004008	1.42967718233409\\
-0.488977955911824	1.43072393472941\\
-0.472945891783567	1.43092834074326\\
-0.456913827655311	1.43033247579527\\
-0.440881763527054	1.42897477560459\\
-0.424849699398798	1.42689032622384\\
-0.408817635270541	1.42411112497851\\
-0.392785571142285	1.42066631546198\\
-0.38565924541739	1.4188376753507\\
-0.376753507014028	1.41656648519904\\
-0.360721442885771	1.41183688755087\\
-0.344689378757515	1.40651358659104\\
-0.334651973094187	1.40280561122244\\
-0.328657314629258	1.40060301102082\\
-0.312625250501002	1.39411330148802\\
-0.296593186372745	1.38708617343689\\
-0.295935787071416	1.38677354709419\\
-0.280561122244489	1.37949587086625\\
-0.264529058116232	1.37139710129056\\
-0.263316471863453	1.37074148296593\\
-0.248496993987976	1.36276180547724\\
-0.234356124466997	1.35470941883768\\
-0.232464929859719	1.35363663963936\\
-0.216432865731463	1.3439988072933\\
-0.208016305591172	1.33867735470942\\
-0.200400801603207	1.33387843940809\\
-0.18436873747495	1.32328265171227\\
-0.183449403577956	1.32264529058116\\
-0.168336673346694	1.31219743040739\\
-0.160593790422456	1.30661322645291\\
-0.152304609218437	1.30065037543827\\
-0.138863158302592	1.29058116232465\\
-0.13627254509018	1.28864497519273\\
-0.120240480961924	1.27617679261093\\
-0.118225064169629	1.27454909819639\\
-0.104208416833667	1.26324929788337\\
-0.0985418869905856	1.25851703406814\\
-0.0881763527054109	1.24987385108974\\
-0.0796107483583451	1.24248496993988\\
-0.0721442885771544	1.23605250728087\\
-0.0613595760769719	1.22645290581162\\
-0.0561122244488979	1.22178697497293\\
-0.043725344827961	1.21042084168337\\
-0.0400801603206413	1.20707861834823\\
-0.0266527533644603	1.19438877755511\\
-0.0240480961923848	1.19192845901123\\
-0.0100931273197109	1.17835671342685\\
-0.00801603206412826	1.17633717730267\\
0.00599649593994431	1.1623246492986\\
0.00801603206412782	1.16030511317441\\
0.0216541346291227	1.14629258517034\\
0.0240480961923843	1.14383226662646\\
0.0369135090800333	1.13026052104208\\
0.0400801603206409	1.12691829770695\\
0.0518045890419841	1.11422845691383\\
0.0561122244488974	1.10956252607513\\
0.0663540590020255	1.09819639278557\\
0.0721442885771539	1.09176393012657\\
0.0805857154372647	1.08216432865731\\
0.0881763527054105	1.07352114567892\\
0.0945208072747595	1.06613226452906\\
0.104208416833667	1.05483246421604\\
0.108178328748497	1.0501002004008\\
0.120240480961924	1.03569583068708\\
0.121575272180958	1.03406813627254\\
0.134692032149139	1.01803607214429\\
0.13627254509018	1.01609988501237\\
0.147546506819147	1.00200400801603\\
0.152304609218437	0.9960411570014\\
0.160183002079825	0.985971943887775\\
0.168336673346693	0.975524083714005\\
0.172612977213375	0.969939879759519\\
0.18436873747495	0.954545176762373\\
0.184846659229807	0.953907815631262\\
0.196826185090445	0.937875751503006\\
0.200400801603206	0.933076836201674\\
0.208621100028441	0.921843687374749\\
0.216432865731463	0.91113307583037\\
0.220248467550492	0.905811623246493\\
0.231701607899197	0.889779559118236\\
0.232464929859719	0.888706779919921\\
0.242934002994659	0.87374749498998\\
0.248496993987976	0.865767817501285\\
0.254022025382232	0.857715430861724\\
0.264529058116232	0.842338985058091\\
0.264970265322122	0.841683366733467\\
0.275708137982974	0.825651302605211\\
0.280561122244489	0.818373626377271\\
0.286312909771483	0.809619238476954\\
0.296593186372745	0.793899800691397\\
0.296794755951398	0.793587174348698\\
0.307077408297698	0.777555110220441\\
0.312625250501002	0.768862800486012\\
0.317245772034545	0.761523046092185\\
0.327285519079521	0.745490981963928\\
0.328657314629258	0.743288381762303\\
0.337156300817286	0.729458917835672\\
0.344689378757515	0.717134829076012\\
0.346927181285696	0.713426853707415\\
0.356545602273577	0.697394789579159\\
0.360721442885771	0.690394001779324\\
0.366042486545644	0.681362725450902\\
0.375431709792827	0.665330661322646\\
0.376753507014028	0.663059471170983\\
0.384667400047787	0.649298597194389\\
0.392785571142285	0.635095173177409\\
0.393818986979759	0.633266533066132\\
0.402817842545186	0.617234468937876\\
0.408817635270541	0.60647585443743\\
0.411726576130805	0.601202404809619\\
0.42050868119499	0.585170340681363\\
0.424849699398798	0.577190927426249\\
0.429184778407611	0.569138276553106\\
0.437753779839534	0.55310621242485\\
0.440881763527054	0.547211248550484\\
0.446206654881929	0.537074148296593\\
0.454566045756848	0.521042084168337\\
0.456913827655311	0.516504820484648\\
0.462804351611308	0.50501002004008\\
0.470957473100776	0.488977955911824\\
0.472945891783567	0.48503655717613\\
0.478989140028764	0.472945891783567\\
0.486939183233613	0.456913827655311\\
0.488977955911824	0.452768022905766\\
0.494771454425069	0.440881763527054\\
0.502521462138104	0.424849699398798\\
0.50501002004008	0.419657142253924\\
0.510160926694367	0.408817635270541\\
0.517713795080866	0.392785571142285\\
0.521042084168337	0.385657877370906\\
0.525166418500278	0.376753507014028\\
0.532524898685601	0.360721442885771\\
0.537074148296593	0.350719868426134\\
0.539796051007148	0.344689378757515\\
0.546962750561728	0.328657314629258\\
0.55310621242485	0.314788032499723\\
0.554057232309417	0.312625250501002\\
0.561034616622361	0.296593186372745\\
0.567946882818409	0.280561122244489\\
0.569138276553106	0.277768647824683\\
0.574747076214539	0.264529058116232\\
0.581471212238471	0.248496993987976\\
0.585170340681363	0.23958505584309\\
0.588106045174488	0.232464929859719\\
0.594644196129463	0.216432865731463\\
0.601116114696609	0.200400801603206\\
0.601202404809619	0.200184484084263\\
0.607470969037548	0.18436873747495\\
0.613757593813339	0.168336673346693\\
0.617234468937876	0.159367725089954\\
0.619956043021835	0.152304609218437\\
0.626059204199459	0.13627254509018\\
0.632094429226009	0.120240480961924\\
0.633266533066132	0.117086470208509\\
0.63802471474842	0.104208416833667\\
0.643876682886971	0.0881763527054105\\
0.649298597194389	0.0731441862642451\\
0.649657305025131	0.0721442885771539\\
0.655327533694953	0.0561122244488974\\
0.660927542575827	0.0400801603206409\\
0.665330661322646	0.0273096771209221\\
0.666449447450091	0.0240480961923843\\
0.671867442661018	0.00801603206412782\\
0.677214141975015	-0.00801603206412826\\
0.681362725450902	-0.0206297695703205\\
0.682481511578348	-0.0240480961923848\\
0.687645617002839	-0.0400801603206413\\
0.692737070130544	-0.0561122244488979\\
0.697394789579159	-0.0709929512657397\\
0.6977534974099	-0.0721442885771544\\
0.702661469168185	-0.0881763527054109\\
0.707495148697988	-0.104208416833667\\
0.712254749867292	-0.120240480961924\\
0.713426853707415	-0.124261189292971\\
0.7169126464296	-0.13627254509018\\
0.721485427396051	-0.152304609218437\\
0.725982042711135	-0.168336673346694\\
0.729458917835672	-0.180956552563859\\
0.730395021803703	-0.18436873747495\\
0.734703172616581	-0.200400801603207\\
0.738932773283772	-0.216432865731463\\
0.743083645568415	-0.232464929859719\\
0.745490981963928	-0.241959271006479\\
0.747141849129499	-0.248496993987976\\
0.751099781625361	-0.264529058116232\\
0.754976124731029	-0.280561122244489\\
0.758770453206528	-0.296593186372745\\
0.761523046092185	-0.308496301981619\\
0.762474065976751	-0.312625250501002\\
0.766070269536165	-0.328657314629258\\
0.769581097463983	-0.344689378757515\\
0.773005860609449	-0.360721442885771\\
0.776343786249278	-0.376753507014028\\
0.777555110220441	-0.382748165478959\\
0.779575252649355	-0.392785571142285\\
0.782706016874727	-0.408817635270541\\
0.785745855233877	-0.424849699398798\\
0.788693703292883	-0.440881763527054\\
0.791548401670487	-0.456913827655311\\
0.793587174348698	-0.468773486291924\\
0.794301571550162	-0.472945891783567\\
0.796938650811452	-0.488977955911824\\
0.799477698304696	-0.50501002004008\\
0.801917229327064	-0.521042084168337\\
0.804255650319159	-0.537074148296593\\
0.806491254789434	-0.55310621242485\\
0.808622218986226	-0.569138276553106\\
0.809619238476954	-0.577062115214904\\
0.810635043483407	-0.585170340681363\\
0.812528179337217	-0.601202404809619\\
0.814310405770573	-0.617234468937876\\
0.815979507096244	-0.633266533066132\\
0.817533131510713	-0.649298597194389\\
0.818968785424053	-0.665330661322646\\
0.820283827444078	-0.681362725450902\\
0.821475461993017	-0.697394789579158\\
0.822540732533354	-0.713426853707415\\
0.82347651437768	-0.729458917835671\\
0.824279507055472	-0.745490981963928\\
0.824946226207663	-0.761523046092184\\
0.825472994977549	-0.777555110220441\\
0.825651302605211	-0.785057435918098\\
0.825852872183863	-0.793587174348697\\
0.826084201198702	-0.809619238476954\\
0.826165509867008	-0.82565130260521\\
0.826092509811099	-0.841683366733467\\
0.825860684610487	-0.857715430861723\\
0.825651302605211	-0.866245169292336\\
0.825462060494201	-0.87374749498998\\
0.824887980644688	-0.889779559118236\\
0.824136376831622	-0.905811623246493\\
0.823201350070428	-0.92184368737475\\
0.822076686092449	-0.937875751503006\\
0.820755838230285	-0.953907815631263\\
0.819231909149522	-0.969939879759519\\
0.817497631338341	-0.985971943887776\\
0.815545346258589	-1.00200400801603\\
0.813366982053292	-1.01803607214429\\
0.810954029695989	-1.03406813627255\\
0.809619238476954	-1.042176361739\\
0.80826563518729	-1.0501002004008\\
0.805283581699165	-1.06613226452906\\
0.802028601208808	-1.08216432865731\\
0.79848934370967	-1.09819639278557\\
0.794653812367823	-1.11422845691383\\
0.793587174348698	-1.11840086240547\\
0.790420523108089	-1.13026052104208\\
0.785819903998319	-1.14629258517034\\
0.780866613022514	-1.1623246492986\\
0.777555110220441	-1.17236205496192\\
0.775478014964857	-1.17835671342685\\
0.769575432731745	-1.19438877755511\\
0.763245346289689	-1.21042084168337\\
0.761523046092185	-1.21454979020275\\
0.75627569446411	-1.22645290581162\\
0.748740773012284	-1.24248496993988\\
0.745490981963928	-1.24902269292138\\
0.7404736102047	-1.25851703406814\\
0.731474334627967	-1.27454909819639\\
0.729458917835672	-1.27796128310748\\
0.721480456734734	-1.29058116232465\\
0.713426853707415	-1.30259251812186\\
0.710523412722236	-1.30661322645291\\
0.698314123476151	-1.32264529058116\\
0.697394789579159	-1.32379662789258\\
0.68448134056134	-1.33867735470942\\
0.681362725450902	-1.34209568133148\\
0.66874898794471	-1.35470941883768\\
0.665330661322646	-1.35797099976621\\
0.650511183447168	-1.37074148296593\\
0.649298597194389	-1.37174138065302\\
0.633266533066132	-1.38361953634077\\
0.62845300893667	-1.38677354709419\\
0.617234468937876	-1.39383666296571\\
0.601202404809619	-1.4025892937035\\
0.600745994797927	-1.40280561122244\\
0.585170340681363	-1.40992573720582\\
0.569138276553106	-1.4160452009309\\
0.560232538149745	-1.4188376753507\\
0.55310621242485	-1.42100045734942\\
0.537074148296593	-1.42486816501932\\
0.521042084168337	-1.42774204570758\\
0.50501002004008	-1.42967718233408\\
0.488977955911824	-1.43072393472941\\
0.472945891783567	-1.43092834074326\\
0.456913827655311	-1.43033247579527\\
0.440881763527054	-1.42897477560459\\
0.424849699398798	-1.42689032622384\\
0.408817635270541	-1.42411112497851\\
0.392785571142285	-1.42066631546198\\
0.385659245417389	-1.4188376753507\\
0.376753507014028	-1.41656648519904\\
0.360721442885771	-1.41183688755087\\
0.344689378757515	-1.40651358659104\\
0.334651973094188	-1.40280561122244\\
0.328657314629258	-1.40060301102082\\
0.312625250501002	-1.39411330148802\\
0.296593186372745	-1.38708617343689\\
0.295935787071416	-1.38677354709419\\
0.280561122244489	-1.37949587086625\\
0.264529058116232	-1.37139710129056\\
0.263316471863453	-1.37074148296593\\
0.248496993987976	-1.36276180547724\\
0.234356124466996	-1.35470941883768\\
0.232464929859719	-1.35363663963936\\
0.216432865731463	-1.3439988072933\\
0.208016305591172	-1.33867735470942\\
0.200400801603206	-1.33387843940809\\
0.18436873747495	-1.32328265171227\\
0.183449403577957	-1.32264529058116\\
0.168336673346693	-1.31219743040739\\
0.160593790422456	-1.30661322645291\\
0.152304609218437	-1.30065037543827\\
0.138863158302593	-1.29058116232465\\
0.13627254509018	-1.28864497519273\\
0.120240480961924	-1.27617679261093\\
0.118225064169628	-1.27454909819639\\
0.104208416833667	-1.26324929788337\\
0.098541886990586	-1.25851703406814\\
0.0881763527054105	-1.24987385108974\\
0.0796107483583451	-1.24248496993988\\
0.0721442885771539	-1.23605250728087\\
0.0613595760769714	-1.22645290581162\\
0.0561122244488974	-1.22178697497293\\
0.043725344827961	-1.21042084168337\\
0.0400801603206409	-1.20707861834823\\
0.0266527533644607	-1.19438877755511\\
0.0240480961923843	-1.19192845901123\\
0.0100931273197114	-1.17835671342685\\
0.00801603206412782	-1.17633717730267\\
-0.00599649593994475	-1.1623246492986\\
-0.00801603206412826	-1.16030511317441\\
-0.0216541346291227	-1.14629258517034\\
-0.0240480961923848	-1.14383226662646\\
-0.0369135090800337	-1.13026052104208\\
-0.0400801603206413	-1.12691829770695\\
-0.0518045890419846	-1.11422845691383\\
-0.0561122244488979	-1.10956252607513\\
-0.0663540590020255	-1.09819639278557\\
-0.0721442885771544	-1.09176393012657\\
-0.0805857154372647	-1.08216432865731\\
-0.0881763527054109	-1.07352114567892\\
-0.0945208072747585	-1.06613226452906\\
-0.104208416833667	-1.05483246421604\\
-0.108178328748496	-1.0501002004008\\
-0.120240480961924	-1.03569583068708\\
-0.121575272180958	-1.03406813627255\\
-0.134692032149139	-1.01803607214429\\
-0.13627254509018	-1.01609988501237\\
-0.147546506819146	-1.00200400801603\\
-0.152304609218437	-0.996041157001399\\
-0.160183002079824	-0.985971943887776\\
-0.168336673346694	-0.975524083714005\\
-0.172612977213376	-0.969939879759519\\
-0.18436873747495	-0.954545176762373\\
-0.184846659229807	-0.953907815631263\\
-0.196826185090445	-0.937875751503006\\
-0.200400801603207	-0.933076836201674\\
-0.208621100028441	-0.92184368737475\\
-0.216432865731463	-0.91113307583037\\
-0.220248467550491	-0.905811623246493\\
-0.231701607899196	-0.889779559118236\\
-0.232464929859719	-0.888706779919921\\
-0.242934002994659	-0.87374749498998\\
-0.248496993987976	-0.865767817501284\\
-0.254022025382232	-0.857715430861723\\
-0.264529058116232	-0.842338985058091\\
-0.264970265322122	-0.841683366733467\\
-0.275708137982974	-0.82565130260521\\
-0.280561122244489	-0.818373626377271\\
-0.286312909771483	-0.809619238476954\\
-0.296593186372745	-0.793899800691397\\
-0.296794755951398	-0.793587174348697\\
-0.307077408297698	-0.777555110220441\\
-0.312625250501002	-0.768862800486012\\
-0.317245772034545	-0.761523046092184\\
-0.327285519079521	-0.745490981963928\\
-0.328657314629258	-0.743288381762303\\
-0.337156300817286	-0.729458917835671\\
-0.344689378757515	-0.717134829076012\\
-0.346927181285696	-0.713426853707415\\
-0.356545602273577	-0.697394789579158\\
-0.360721442885771	-0.690394001779324\\
-0.366042486545644	-0.681362725450902\\
-0.375431709792827	-0.665330661322646\\
-0.376753507014028	-0.663059471170983\\
-0.384667400047787	-0.649298597194389\\
-0.392785571142285	-0.635095173177409\\
-0.393818986979759	-0.633266533066132\\
-0.402817842545187	-0.617234468937876\\
-0.408817635270541	-0.60647585443743\\
-0.411726576130805	-0.601202404809619\\
-0.42050868119499	-0.585170340681363\\
-0.424849699398798	-0.577190927426249\\
-0.429184778407611	-0.569138276553106\\
-0.437753779839535	-0.55310621242485\\
-0.440881763527054	-0.547211248550484\\
-0.446206654881929	-0.537074148296593\\
-0.454566045756848	-0.521042084168337\\
-0.456913827655311	-0.516504820484648\\
-0.462804351611308	-0.50501002004008\\
-0.470957473100776	-0.488977955911824\\
-0.472945891783567	-0.485036557176129\\
-0.478989140028764	-0.472945891783567\\
-0.486939183233613	-0.456913827655311\\
-0.488977955911824	-0.452768022905766\\
-0.494771454425069	-0.440881763527054\\
-0.502521462138104	-0.424849699398798\\
-0.50501002004008	-0.419657142253924\\
-0.510160926694366	-0.408817635270541\\
-0.517713795080866	-0.392785571142285\\
-0.521042084168337	-0.385657877370906\\
-0.525166418500279	-0.376753507014028\\
-0.532524898685601	-0.360721442885771\\
-0.537074148296593	-0.350719868426134\\
-0.539796051007148	-0.344689378757515\\
-0.546962750561728	-0.328657314629258\\
-0.55310621242485	-0.314788032499723\\
-0.554057232309416	-0.312625250501002\\
-0.561034616622361	-0.296593186372745\\
-0.56794688281841	-0.280561122244489\\
-0.569138276553106	-0.277768647824683\\
-0.574747076214539	-0.264529058116232\\
-0.581471212238471	-0.248496993987976\\
-0.585170340681363	-0.23958505584309\\
-0.588106045174488	-0.232464929859719\\
-0.594644196129463	-0.216432865731463\\
-0.601116114696608	-0.200400801603207\\
-0.601202404809619	-0.200184484084262\\
-0.607470969037547	-0.18436873747495\\
-0.613757593813339	-0.168336673346694\\
-0.617234468937876	-0.159367725089954\\
-0.619956043021835	-0.152304609218437\\
-0.626059204199459	-0.13627254509018\\
-0.632094429226009	-0.120240480961924\\
-0.633266533066132	-0.117086470208509\\
-0.638024714748419	-0.104208416833667\\
-0.643876682886971	-0.0881763527054109\\
-0.649298597194389	-0.0731441862642456\\
-0.649657305025131	-0.0721442885771544\\
-0.655327533694952	-0.0561122244488979\\
-0.660927542575827	-0.0400801603206413\\
-0.665330661322646	-0.0273096771209217\\
-0.666449447450091	-0.0240480961923848\\
-0.671867442661018	-0.00801603206412826\\
-0.677214141975015	0.00801603206412782\\
-0.681362725450902	0.0206297695703186\\
-0.682481511578348	0.0240480961923843\\
-0.687645617002839	0.0400801603206409\\
-0.692737070130543	0.0561122244488974\\
-0.697394789579158	0.0709929512657383\\
-0.6977534974099	0.0721442885771539\\
-0.702661469168184	0.0881763527054105\\
-0.707495148697989	0.104208416833667\\
-0.712254749867292	0.120240480961924\\
-0.713426853707415	0.124261189292968\\
-0.7169126464296	0.13627254509018\\
-0.72148542739605	0.152304609218437\\
-0.725982042711135	0.168336673346693\\
-0.729458917835671	0.180956552563858\\
-0.730395021803702	0.18436873747495\\
-0.734703172616582	0.200400801603206\\
-0.738932773283771	0.216432865731463\\
-0.743083645568415	0.232464929859719\\
-0.745490981963928	0.241959271006478\\
-0.747141849129499	0.248496993987976\\
-0.75109978162536	0.264529058116232\\
-0.754976124731029	0.280561122244489\\
-0.758770453206528	0.296593186372745\\
-0.761523046092184	0.308496301981619\\
-0.762474065976751	0.312625250501002\\
-0.766070269536166	0.328657314629258\\
-0.769581097463983	0.344689378757515\\
-0.773005860609449	0.360721442885771\\
-0.776343786249279	0.376753507014028\\
-0.777555110220441	0.382748165478957\\
-0.779575252649355	0.392785571142285\\
-0.782706016874727	0.408817635270541\\
-0.785745855233877	0.424849699398798\\
-0.788693703292883	0.440881763527054\\
-0.791548401670487	0.456913827655311\\
-0.793587174348697	0.468773486291922\\
-0.794301571550162	0.472945891783567\\
-0.796938650811452	0.488977955911824\\
-0.799477698304695	0.50501002004008\\
-0.801917229327064	0.521042084168337\\
-0.804255650319159	0.537074148296593\\
-0.806491254789434	0.55310621242485\\
-0.808622218986225	0.569138276553106\\
-0.809619238476954	0.577062115214904\\
-0.810635043483407	0.585170340681363\\
-0.812528179337218	0.601202404809619\\
-0.814310405770573	0.617234468937876\\
-0.815979507096244	0.633266533066132\\
-0.817533131510713	0.649298597194389\\
-0.818968785424053	0.665330661322646\\
-0.820283827444077	0.681362725450902\\
-0.821475461993016	0.697394789579159\\
-0.822540732533354	0.713426853707415\\
-0.82347651437768	0.729458917835672\\
-0.824279507055473	0.745490981963928\\
-0.824946226207662	0.761523046092185\\
-0.825472994977549	0.777555110220441\\
-0.82565130260521	0.785057435918085\\
-0.825852872183862	0.793587174348698\\
-0.826084201198703	0.809619238476954\\
-0.826165509867008	0.825651302605211\\
-0.8260925098111	0.841683366733467\\
-0.825860684610486	0.857715430861724\\
-0.82565130260521	0.866245169292337\\
}--cycle;


\addplot[area legend,solid,fill=mycolor10,draw=black,forget plot]
table[row sep=crcr] {%
x	y\\
-0.665330661322646	0.758105025403793\\
-0.665082784862698	0.761523046092185\\
-0.663686564675468	0.777555110220441\\
-0.662033485917416	0.793587174348698\\
-0.660112238187751	0.809619238476954\\
-0.657910790516671	0.825651302605211\\
-0.655416339751623	0.841683366733467\\
-0.652615254428646	0.857715430861724\\
-0.649493013666892	0.87374749498998\\
-0.649298597194389	0.87466276666894\\
-0.645911158536872	0.889779559118236\\
-0.641949453571151	0.905811623246493\\
-0.637595573522108	0.921843687374749\\
-0.633266533066132	0.936427568599341\\
-0.632809737264636	0.937875751503006\\
-0.62737690649475	0.953907815631262\\
-0.621452974746538	0.969939879759519\\
-0.617234468937876	0.980509918222073\\
-0.614898112565123	0.985971943887775\\
-0.607542947700778	1.00200400801603\\
-0.601202404809619	1.0147880599505\\
-0.599461200225299	1.01803607214429\\
-0.590292763112442	1.03406813627254\\
-0.585170340681363	1.04241611958894\\
-0.58002212340773	1.0501002004008\\
-0.569138276553106	1.06525453592202\\
-0.568443290727499	1.06613226452906\\
-0.55496007841122	1.08216432865731\\
-0.55310621242485	1.08423901886186\\
-0.539148838501137	1.09819639278557\\
-0.537074148296593	1.10015445046976\\
-0.521042084168337	1.11349529538232\\
-0.520031608379621	1.11422845691383\\
-0.50501002004008	1.1245759618867\\
-0.495273095705868	1.13026052104208\\
-0.488977955911824	1.13376186558903\\
-0.472945891783567	1.14121295074715\\
-0.459322825664768	1.14629258517034\\
-0.456913827655311	1.1471525085698\\
-0.440881763527054	1.15164454166159\\
-0.424849699398798	1.1548759669644\\
-0.408817635270541	1.15693170935783\\
-0.392785571142285	1.15788836523612\\
-0.376753507014028	1.15781506554153\\
-0.360721442885771	1.15677423201626\\
-0.344689378757515	1.15482224148008\\
-0.328657314629258	1.15201001063494\\
-0.312625250501002	1.14838351198749\\
-0.305073103182926	1.14629258517034\\
-0.296593186372745	1.14395622879759\\
-0.280561122244489	1.13876287781614\\
-0.264529058116232	1.13286477697221\\
-0.258233918322189	1.13026052104208\\
-0.248496993987976	1.12624895384855\\
-0.232464929859719	1.11895667271266\\
-0.222939668618048	1.11422845691383\\
-0.216432865731463	1.11101016607699\\
-0.200400801603207	1.10241489859423\\
-0.193078381798149	1.09819639278557\\
-0.18436873747495	1.09319414797708\\
-0.168336673346694	1.0833749248919\\
-0.166482807360323	1.08216432865731\\
-0.152304609218437	1.07292956533563\\
-0.142421230303577	1.06613226452906\\
-0.13627254509018	1.06191327279785\\
-0.120240480961924	1.05032770442359\\
-0.119941671271634	1.0501002004008\\
-0.104208416833667	1.03814294264968\\
-0.0990859944025878	1.03406813627254\\
-0.0881763527054109	1.02540306136772\\
-0.0792960980873046	1.01803607214429\\
-0.0721442885771544	1.0121106072761\\
-0.0604426536847802	1.00200400801603\\
-0.0561122244488979	0.998267697634059\\
-0.0424165166933938	0.985971943887775\\
-0.0400801603206413	0.983876022989052\\
-0.0251250206077705	0.969939879759519\\
-0.0240480961923848	0.968936849352776\\
-0.00848905848976323	0.953907815631262\\
-0.00801603206412826	0.953451019829765\\
0.007559236262631	0.937875751503006\\
0.00801603206412782	0.937418955701508\\
0.023078348608166	0.921843687374749\\
0.0240480961923843	0.920840656968006\\
0.038118883178968	0.905811623246493\\
0.0400801603206409	0.903715702347771\\
0.052724785791382	0.889779559118236\\
0.0561122244488974	0.886043248736265\\
0.0669343440570812	0.87374749498998\\
0.0721442885771539	0.867822030121793\\
0.0807810122405978	0.857715430861724\\
0.0881763527054105	0.849050355956901\\
0.0942940952626445	0.841683366733467\\
0.104208416833667	0.829726108982349\\
0.107499319045926	0.825651302605211\\
0.120240480961924	0.809846742499742\\
0.12041930855841	0.809619238476954\\
0.132975369684951	0.793587174348698\\
0.13627254509018	0.789368182617485\\
0.145279652829894	0.777555110220441\\
0.152304609218437	0.768320346898754\\
0.157354243807146	0.761523046092185\\
0.168336673346693	0.746701578198513\\
0.169213498649643	0.745490981963928\\
0.180773489041761	0.729458917835672\\
0.18436873747495	0.724456673027176\\
0.192122093660993	0.713426853707415\\
0.200400801603206	0.70161329538782\\
0.203294662397068	0.697394789579159\\
0.214244932658426	0.681362725450902\\
0.216432865731463	0.67814443461406\\
0.224965119394612	0.665330661322646\\
0.232464929859719	0.654026812993221\\
0.235540035818519	0.649298597194389\\
0.245914642067158	0.633266533066132\\
0.248496993987976	0.629254965872603\\
0.256086167319485	0.617234468937876\\
0.264529058116232	0.603806660739741\\
0.266136311158703	0.601202404809619\\
0.275967306441132	0.585170340681363\\
0.280561122244489	0.577640633327164\\
0.285655289054162	0.569138276553106\\
0.295210419473976	0.55310621242485\\
0.296593186372745	0.550769856052097\\
0.30455760395013	0.537074148296593\\
0.312625250501002	0.523133010985484\\
0.313815042172045	0.521042084168337\\
0.322866732980162	0.50501002004008\\
0.328657314629258	0.494695381376427\\
0.331814981981456	0.488977955911824\\
0.340604525579819	0.472945891783567\\
0.344689378757515	0.465443483965051\\
0.349260440368795	0.456913827655311\\
0.357791295522285	0.440881763527054\\
0.360721442885771	0.435331346244714\\
0.366170395731711	0.424849699398798\\
0.374445905957461	0.408817635270541\\
0.376753507014028	0.404308051513471\\
0.382562459543581	0.392785571142285\\
0.39058584797749	0.376753507014028\\
0.392785571142285	0.372317222951554\\
0.398452948292133	0.360721442885771\\
0.406227313185292	0.344689378757515\\
0.408817635270541	0.339296438816749\\
0.413856949942716	0.328657314629258\\
0.421385260700624	0.312625250501002\\
0.424849699398798	0.305176568166802\\
0.428788385313387	0.296593186372745\\
0.436073478999931	0.280561122244489\\
0.440881763527054	0.269881014607484\\
0.443260064715237	0.264529058116232\\
0.450304642951515	0.248496993987976\\
0.456913827655311	0.233324853259175\\
0.457283740180685	0.232464929859719\\
0.464090366375889	0.216432865731463\\
0.470835143423245	0.200400801603206\\
0.472945891783567	0.195321167180018\\
0.477441250432316	0.18436873747495\\
0.483947723367293	0.168336673346693\\
0.488977955911824	0.155805953765381\\
0.490366928106163	0.152304609218437\\
0.496637367872314	0.13627254509018\\
0.502841024354953	0.120240480961924\\
0.50501002004008	0.114555921806544\\
0.508912636015967	0.104208416833667\\
0.514879435468517	0.0881763527054105\\
0.520776905952092	0.0721442885771539\\
0.521042084168337	0.0714111270456462\\
0.526513104852846	0.0561122244488974\\
0.532172574628555	0.0400801603206409\\
0.537074148296593	0.0260061538765745\\
0.537748689204562	0.0240480961923843\\
0.54317196671062	0.00801603206412782\\
0.5485197049853	-0.00801603206412826\\
0.55310621242485	-0.0219734059878415\\
0.553780753332819	-0.0240480961923848\\
0.558890672250114	-0.0400801603206413\\
0.56392155566562	-0.0561122244488979\\
0.568873098336861	-0.0721442885771544\\
0.569138276553106	-0.0730220171841901\\
0.573668496296285	-0.0881763527054109\\
0.57837666810335	-0.104208416833667\\
0.583001344996236	-0.120240480961924\\
0.585170340681363	-0.127924561773787\\
0.587502535040489	-0.13627254509018\\
0.591881390411667	-0.152304609218437\\
0.596172172265089	-0.168336673346694\\
0.60037395363977	-0.18436873747495\\
0.601202404809619	-0.187616749668742\\
0.60442986328447	-0.200400801603207\\
0.608378943530198	-0.216432865731463\\
0.612233727171058	-0.232464929859719\\
0.615992884122811	-0.248496993987976\\
0.617234468937876	-0.25395901965368\\
0.619612770126059	-0.264529058116232\\
0.623111489402566	-0.280561122244489\\
0.626508521446038	-0.296593186372745\\
0.629802094367958	-0.312625250501002\\
0.632990291864623	-0.328657314629258\\
0.633266533066132	-0.330105497532926\\
0.6360197366617	-0.344689378757515\\
0.638933910215981	-0.360721442885771\\
0.641735488262777	-0.376753507014028\\
0.644422048509105	-0.392785571142285\\
0.646990996137822	-0.408817635270541\\
0.649298597194389	-0.423934427719837\\
0.649436850751654	-0.424849699398798\\
0.651716318887212	-0.440881763527054\\
0.653869658805669	-0.456913827655311\\
0.655893692198952	-0.472945891783567\\
0.657785033616345	-0.488977955911824\\
0.659540079673549	-0.50501002004008\\
0.661154997500123	-0.521042084168337\\
0.662625712366185	-0.537074148296593\\
0.663947894423875	-0.55310621242485\\
0.665116944493114	-0.569138276553106\\
0.665330661322646	-0.572556297241498\\
0.666110017040159	-0.585170340681363\\
0.666937914365116	-0.601202404809619\\
0.667600197623856	-0.617234468937876\\
0.668091410176274	-0.633266533066132\\
0.668405767281445	-0.649298597194389\\
0.668537136825009	-0.665330661322646\\
0.66847901861384	-0.681362725450902\\
0.668224522116507	-0.697394789579158\\
0.6677663425161	-0.713426853707415\\
0.66709673492865	-0.729458917835671\\
0.666207486625594	-0.745490981963928\\
0.665330661322646	-0.758105025403793\\
0.665082784862698	-0.761523046092184\\
0.663686564675469	-0.777555110220441\\
0.662033485917416	-0.793587174348697\\
0.660112238187751	-0.809619238476954\\
0.657910790516671	-0.82565130260521\\
0.655416339751623	-0.841683366733467\\
0.652615254428647	-0.857715430861723\\
0.649493013666891	-0.87374749498998\\
0.649298597194389	-0.87466276666894\\
0.645911158536872	-0.889779559118236\\
0.641949453571151	-0.905811623246493\\
0.637595573522108	-0.92184368737475\\
0.633266533066132	-0.936427568599342\\
0.632809737264635	-0.937875751503006\\
0.627376906494749	-0.953907815631263\\
0.621452974746537	-0.969939879759519\\
0.617234468937876	-0.980509918222073\\
0.614898112565123	-0.985971943887776\\
0.607542947700777	-1.00200400801603\\
0.601202404809619	-1.0147880599505\\
0.599461200225299	-1.01803607214429\\
0.590292763112442	-1.03406813627255\\
0.585170340681363	-1.04241611958894\\
0.580022123407729	-1.0501002004008\\
0.569138276553106	-1.06525453592202\\
0.5684432907275	-1.06613226452906\\
0.55496007841122	-1.08216432865731\\
0.55310621242485	-1.08423901886186\\
0.539148838501136	-1.09819639278557\\
0.537074148296593	-1.10015445046976\\
0.521042084168337	-1.11349529538232\\
0.520031608379621	-1.11422845691383\\
0.50501002004008	-1.12457596188671\\
0.495273095705867	-1.13026052104208\\
0.488977955911824	-1.13376186558903\\
0.472945891783567	-1.14121295074715\\
0.459322825664769	-1.14629258517034\\
0.456913827655311	-1.1471525085698\\
0.440881763527054	-1.15164454166159\\
0.424849699398798	-1.1548759669644\\
0.408817635270541	-1.15693170935783\\
0.392785571142285	-1.15788836523612\\
0.376753507014028	-1.15781506554153\\
0.360721442885771	-1.15677423201626\\
0.344689378757515	-1.15482224148008\\
0.328657314629258	-1.15201001063494\\
0.312625250501002	-1.14838351198749\\
0.305073103182926	-1.14629258517034\\
0.296593186372745	-1.14395622879759\\
0.280561122244489	-1.13876287781614\\
0.264529058116232	-1.13286477697221\\
0.258233918322189	-1.13026052104208\\
0.248496993987976	-1.12624895384855\\
0.232464929859719	-1.11895667271266\\
0.222939668618048	-1.11422845691383\\
0.216432865731463	-1.11101016607699\\
0.200400801603206	-1.10241489859423\\
0.19307838179815	-1.09819639278557\\
0.18436873747495	-1.09319414797708\\
0.168336673346693	-1.0833749248919\\
0.166482807360324	-1.08216432865731\\
0.152304609218437	-1.07292956533563\\
0.142421230303577	-1.06613226452906\\
0.13627254509018	-1.06191327279785\\
0.120240480961924	-1.05032770442359\\
0.119941671271634	-1.0501002004008\\
0.104208416833667	-1.03814294264968\\
0.0990859944025882	-1.03406813627255\\
0.0881763527054105	-1.02540306136772\\
0.079296098087305	-1.01803607214429\\
0.0721442885771539	-1.0121106072761\\
0.0604426536847797	-1.00200400801603\\
0.0561122244488974	-0.99826769763406\\
0.0424165166933943	-0.985971943887776\\
0.0400801603206409	-0.983876022989052\\
0.0251250206077718	-0.969939879759519\\
0.0240480961923843	-0.968936849352774\\
0.00848905848976415	-0.953907815631263\\
0.00801603206412782	-0.953451019829765\\
-0.00755923626263144	-0.937875751503006\\
-0.00801603206412826	-0.937418955701509\\
-0.0230783486081656	-0.92184368737475\\
-0.0240480961923848	-0.920840656968006\\
-0.0381188831789675	-0.905811623246493\\
-0.0400801603206413	-0.90371570234777\\
-0.0527247857913811	-0.889779559118236\\
-0.0561122244488979	-0.886043248736264\\
-0.0669343440570817	-0.87374749498998\\
-0.0721442885771544	-0.867822030121793\\
-0.0807810122405983	-0.857715430861723\\
-0.0881763527054109	-0.8490503559569\\
-0.0942940952626444	-0.841683366733467\\
-0.104208416833667	-0.829726108982349\\
-0.107499319045926	-0.82565130260521\\
-0.120240480961924	-0.809846742499741\\
-0.120419308558411	-0.809619238476954\\
-0.132975369684952	-0.793587174348697\\
-0.13627254509018	-0.789368182617485\\
-0.145279652829894	-0.777555110220441\\
-0.152304609218437	-0.768320346898754\\
-0.157354243807146	-0.761523046092184\\
-0.168336673346694	-0.746701578198513\\
-0.169213498649643	-0.745490981963928\\
-0.180773489041761	-0.729458917835671\\
-0.18436873747495	-0.724456673027176\\
-0.192122093660993	-0.713426853707415\\
-0.200400801603207	-0.70161329538782\\
-0.203294662397068	-0.697394789579158\\
-0.214244932658426	-0.681362725450902\\
-0.216432865731463	-0.678144434614059\\
-0.224965119394612	-0.665330661322646\\
-0.232464929859719	-0.654026812993221\\
-0.235540035818519	-0.649298597194389\\
-0.245914642067158	-0.633266533066132\\
-0.248496993987976	-0.629254965872603\\
-0.256086167319485	-0.617234468937876\\
-0.264529058116232	-0.603806660739742\\
-0.266136311158703	-0.601202404809619\\
-0.275967306441132	-0.585170340681363\\
-0.280561122244489	-0.577640633327163\\
-0.285655289054161	-0.569138276553106\\
-0.295210419473976	-0.55310621242485\\
-0.296593186372745	-0.550769856052097\\
-0.30455760395013	-0.537074148296593\\
-0.312625250501002	-0.523133010985485\\
-0.313815042172045	-0.521042084168337\\
-0.322866732980163	-0.50501002004008\\
-0.328657314629258	-0.494695381376427\\
-0.331814981981456	-0.488977955911824\\
-0.34060452557982	-0.472945891783567\\
-0.344689378757515	-0.465443483965052\\
-0.349260440368795	-0.456913827655311\\
-0.357791295522285	-0.440881763527054\\
-0.360721442885771	-0.435331346244714\\
-0.366170395731711	-0.424849699398798\\
-0.374445905957461	-0.408817635270541\\
-0.376753507014028	-0.404308051513471\\
-0.382562459543581	-0.392785571142285\\
-0.390585847977489	-0.376753507014028\\
-0.392785571142285	-0.372317222951554\\
-0.398452948292133	-0.360721442885771\\
-0.406227313185292	-0.344689378757515\\
-0.408817635270541	-0.339296438816749\\
-0.413856949942716	-0.328657314629258\\
-0.421385260700624	-0.312625250501002\\
-0.424849699398798	-0.305176568166802\\
-0.428788385313387	-0.296593186372745\\
-0.436073478999931	-0.280561122244489\\
-0.440881763527054	-0.269881014607484\\
-0.443260064715237	-0.264529058116232\\
-0.450304642951515	-0.248496993987976\\
-0.456913827655311	-0.233324853259175\\
-0.457283740180684	-0.232464929859719\\
-0.46409036637589	-0.216432865731463\\
-0.470835143423245	-0.200400801603207\\
-0.472945891783567	-0.195321167180017\\
-0.477441250432316	-0.18436873747495\\
-0.483947723367294	-0.168336673346694\\
-0.488977955911824	-0.155805953765381\\
-0.490366928106163	-0.152304609218437\\
-0.496637367872314	-0.13627254509018\\
-0.502841024354953	-0.120240480961924\\
-0.50501002004008	-0.114555921806545\\
-0.508912636015967	-0.104208416833667\\
-0.514879435468517	-0.0881763527054109\\
-0.520776905952091	-0.0721442885771544\\
-0.521042084168337	-0.0714111270456461\\
-0.526513104852846	-0.0561122244488979\\
-0.532172574628555	-0.0400801603206413\\
-0.537074148296593	-0.026006153876575\\
-0.537748689204563	-0.0240480961923848\\
-0.54317196671062	-0.00801603206412826\\
-0.548519704985299	0.00801603206412782\\
-0.55310621242485	0.0219734059878419\\
-0.553780753332819	0.0240480961923843\\
-0.558890672250114	0.0400801603206409\\
-0.563921555665619	0.0561122244488974\\
-0.56887309833686	0.0721442885771539\\
-0.569138276553106	0.073022017184192\\
-0.573668496296284	0.0881763527054105\\
-0.57837666810335	0.104208416833667\\
-0.583001344996236	0.120240480961924\\
-0.585170340681363	0.127924561773787\\
-0.58750253504049	0.13627254509018\\
-0.591881390411667	0.152304609218437\\
-0.596172172265089	0.168336673346693\\
-0.60037395363977	0.18436873747495\\
-0.601202404809619	0.187616749668742\\
-0.60442986328447	0.200400801603206\\
-0.608378943530199	0.216432865731463\\
-0.612233727171058	0.232464929859719\\
-0.615992884122811	0.248496993987976\\
-0.617234468937876	0.25395901965368\\
-0.619612770126059	0.264529058116232\\
-0.623111489402566	0.280561122244489\\
-0.626508521446038	0.296593186372745\\
-0.629802094367959	0.312625250501002\\
-0.632990291864623	0.328657314629258\\
-0.633266533066132	0.330105497532926\\
-0.6360197366617	0.344689378757515\\
-0.638933910215981	0.360721442885771\\
-0.641735488262777	0.376753507014028\\
-0.644422048509105	0.392785571142285\\
-0.646990996137822	0.408817635270541\\
-0.649298597194389	0.423934427719838\\
-0.649436850751654	0.424849699398798\\
-0.651716318887212	0.440881763527054\\
-0.653869658805669	0.456913827655311\\
-0.655893692198952	0.472945891783567\\
-0.657785033616345	0.488977955911824\\
-0.659540079673549	0.50501002004008\\
-0.661154997500123	0.521042084168337\\
-0.662625712366185	0.537074148296593\\
-0.663947894423875	0.55310621242485\\
-0.665116944493114	0.569138276553106\\
-0.665330661322646	0.572556297241498\\
-0.666110017040159	0.585170340681363\\
-0.666937914365116	0.601202404809619\\
-0.667600197623856	0.617234468937876\\
-0.668091410176274	0.633266533066132\\
-0.668405767281445	0.649298597194389\\
-0.668537136825008	0.665330661322646\\
-0.668479018613839	0.681362725450902\\
-0.668224522116507	0.697394789579159\\
-0.6677663425161	0.713426853707415\\
-0.66709673492865	0.729458917835672\\
-0.666207486625595	0.745490981963928\\
-0.665330661322646	0.758105025403793\\
}--cycle;


\addplot[area legend,solid,fill=mycolor11,draw=black,forget plot]
table[row sep=crcr] {%
x	y\\
-0.472945891783567	0.631558403398626\\
-0.472605099508047	0.633266533066132\\
-0.468985504333452	0.649298597194389\\
-0.464849004208713	0.665330661322646\\
-0.460159075615718	0.681362725450902\\
-0.456913827655311	0.691303111073879\\
-0.454736499135691	0.697394789579159\\
-0.448395418577774	0.713426853707415\\
-0.441293873286597	0.729458917835672\\
-0.440881763527054	0.730315450863864\\
-0.432752526676281	0.745490981963928\\
-0.424849699398798	0.758778909934585\\
-0.423006465444712	0.761523046092185\\
-0.411260209628666	0.777555110220441\\
-0.408817635270541	0.780620139334836\\
-0.396869159056258	0.793587174348698\\
-0.392785571142285	0.797670762262671\\
-0.378610477861005	0.809619238476954\\
-0.376753507014028	0.811071611595405\\
-0.360721442885771	0.821430877328627\\
-0.352418655016157	0.825651302605211\\
-0.344689378757515	0.829329654544471\\
-0.328657314629258	0.835068349076468\\
-0.312625250501002	0.838968684198747\\
-0.296593186372745	0.841199718536656\\
-0.285769468609448	0.841683366733467\\
-0.280561122244489	0.841903918836987\\
-0.275352775879529	0.841683366733467\\
-0.264529058116232	0.841227025015592\\
-0.248496993987976	0.83926680531526\\
-0.232464929859719	0.836128895371847\\
-0.216432865731463	0.831901692994961\\
-0.200400801603207	0.82666322786356\\
-0.197825995821692	0.825651302605211\\
-0.18436873747495	0.820378748496846\\
-0.168336673346694	0.813184089207794\\
-0.161302958637682	0.809619238476954\\
-0.152304609218437	0.805070371891943\\
-0.13627254509018	0.796098946117739\\
-0.132188957176207	0.793587174348698\\
-0.120240480961924	0.786252889299523\\
-0.107126535443859	0.777555110220441\\
-0.104208416833667	0.77562315720775\\
-0.0881763527054109	0.764164839203355\\
-0.084723917528645	0.761523046092185\\
-0.0721442885771544	0.751909514130287\\
-0.0642414612996724	0.745490981963928\\
-0.0561122244488979	0.738895348584444\\
-0.045128831381008	0.729458917835672\\
-0.0400801603206413	0.725124673100652\\
-0.0271702987996977	0.713426853707415\\
-0.0240480961923848	0.710599232962271\\
-0.0101933605837475	0.697394789579159\\
-0.00801603206412826	0.695320190446341\\
0.00594143293131114	0.681362725450902\\
0.00801603206412782	0.679288126318085\\
0.0213479068822207	0.665330661322646\\
0.0240480961923843	0.662503040577503\\
0.0361197728705266	0.649298597194389\\
0.0400801603206409	0.64496435245937\\
0.0503347757565049	0.633266533066132\\
0.0561122244488974	0.626670899686649\\
0.0640578544475958	0.617234468937876\\
0.0721442885771539	0.607620936975979\\
0.0773435796145164	0.601202404809619\\
0.0881763527054105	0.587812133792533\\
0.0902380540757649	0.585170340681363\\
0.102710955361553	0.569138276553106\\
0.104208416833667	0.567206323540415\\
0.114761222290392	0.55310621242485\\
0.120240480961924	0.545771927375676\\
0.126524458918498	0.537074148296593\\
0.13627254509018	0.523553855937378\\
0.138026478403903	0.521042084168337\\
0.149159038093379	0.50501002004008\\
0.152304609218437	0.50046115345507\\
0.160001583014829	0.488977955911824\\
0.168336673346693	0.476510742514408\\
0.170649960884252	0.472945891783567\\
0.180987089305425	0.456913827655311\\
0.18436873747495	0.451641273546946\\
0.191071979166073	0.440881763527054\\
0.200400801603206	0.425861624657147\\
0.201012065831786	0.424849699398798\\
0.210603949389608	0.408817635270541\\
0.216432865731463	0.399035961532035\\
0.220058021957355	0.392785571142285\\
0.229290611656081	0.376753507014028\\
0.232464929859719	0.371199035652408\\
0.238297033847969	0.360721442885771\\
0.247173167878561	0.344689378757515\\
0.248496993987976	0.342272817339308\\
0.255766886224913	0.328657314629258\\
0.264289549707929	0.312625250501002\\
0.264529058116232	0.312168908783128\\
0.272502383661635	0.296593186372745\\
0.280561122244489	0.28078167434801\\
0.280670880448816	0.280561122244489\\
0.288535564991394	0.264529058116232\\
0.296360600228989	0.248496993987976\\
0.296593186372745	0.248013345791166\\
0.303895898217337	0.232464929859719\\
0.311376856476042	0.216432865731463\\
0.312625250501002	0.213718183196744\\
0.318610457811489	0.200400801603206\\
0.325750495336266	0.18436873747495\\
0.328657314629258	0.177753719817951\\
0.332704086079932	0.168336673346693\\
0.339506322152324	0.152304609218437\\
0.344689378757515	0.139950897029441\\
0.346199540087413	0.13627254509018\\
0.352667032756419	0.120240480961924\\
0.359067068258363	0.104208416833667\\
0.360721442885771	0.0999879915570851\\
0.36525335024532	0.0881763527054105\\
0.371314638075837	0.0721442885771539\\
0.376753507014028	0.0575645975673485\\
0.377284149442735	0.0561122244488974\\
0.383009686993799	0.0400801603206409\\
0.388652960980946	0.0240480961923843\\
0.392785571142285	0.0120996199781027\\
0.394169229744255	0.00801603206412782\\
0.399472542357823	-0.00801603206412826\\
0.404685025109202	-0.0240480961923848\\
0.408817635270541	-0.0370151312062448\\
0.409774840597944	-0.0400801603206413\\
0.414642913275298	-0.0561122244488979\\
0.419410830460606	-0.0721442885771544\\
0.424076362139965	-0.0881763527054109\\
0.424849699398798	-0.0909204888630107\\
0.428518805443235	-0.104208416833667\\
0.432827353397701	-0.120240480961924\\
0.437022813017709	-0.13627254509018\\
0.440881763527054	-0.151448076190245\\
0.441095225623083	-0.152304609218437\\
0.444928534977728	-0.168336673346694\\
0.448637216736993	-0.18436873747495\\
0.452217531449852	-0.200400801603207\\
0.45566543363035	-0.216432865731463\\
0.456913827655311	-0.222524544236743\\
0.458909321024828	-0.232464929859719\\
0.46197191826691	-0.248496993987976\\
0.464888270402216	-0.264529058116232\\
0.467653230225484	-0.280561122244489\\
0.470261256084326	-0.296593186372745\\
0.472706383375263	-0.312625250501002\\
0.472945891783567	-0.314333380168509\\
0.474911396503024	-0.328657314629258\\
0.476934843536039	-0.344689378757515\\
0.478777995771817	-0.360721442885771\\
0.48043346647087	-0.376753507014028\\
0.481893321940788	-0.392785571142285\\
0.483149039569969	-0.408817635270541\\
0.484191461832447	-0.424849699398798\\
0.485010745812909	-0.440881763527054\\
0.485596307742298	-0.456913827655311\\
0.48593676196695	-0.472945891783567\\
0.486019853696573	-0.488977955911824\\
0.485832384786766	-0.50501002004008\\
0.485360131708188	-0.521042084168337\\
0.484587754734332	-0.537074148296593\\
0.483498697240291	-0.55310621242485\\
0.482075073842231	-0.569138276553106\\
0.480297545917277	-0.585170340681363\\
0.47814518282093	-0.601202404809619\\
0.475595306857631	-0.617234468937876\\
0.472945891783567	-0.631558403398626\\
0.472605099508047	-0.633266533066132\\
0.468985504333452	-0.649298597194389\\
0.464849004208713	-0.665330661322646\\
0.46015907561572	-0.681362725450902\\
0.456913827655311	-0.69130311107388\\
0.454736499135691	-0.697394789579158\\
0.448395418577773	-0.713426853707415\\
0.441293873286598	-0.729458917835671\\
0.440881763527054	-0.730315450863864\\
0.432752526676281	-0.745490981963928\\
0.424849699398798	-0.758778909934585\\
0.423006465444711	-0.761523046092184\\
0.411260209628669	-0.777555110220441\\
0.408817635270541	-0.780620139334837\\
0.396869159056258	-0.793587174348697\\
0.392785571142285	-0.797670762262671\\
0.378610477861005	-0.809619238476954\\
0.376753507014028	-0.811071611595405\\
0.360721442885771	-0.821430877328629\\
0.352418655016159	-0.82565130260521\\
0.344689378757515	-0.829329654544471\\
0.328657314629258	-0.835068349076468\\
0.312625250501002	-0.838968684198748\\
0.296593186372745	-0.841199718536656\\
0.285769468609449	-0.841683366733467\\
0.280561122244489	-0.841903918836987\\
0.275352775879529	-0.841683366733467\\
0.264529058116232	-0.841227025015592\\
0.248496993987976	-0.839266805315259\\
0.232464929859719	-0.836128895371846\\
0.216432865731463	-0.83190169299496\\
0.200400801603206	-0.826663227863559\\
0.197825995821691	-0.82565130260521\\
0.18436873747495	-0.820378748496845\\
0.168336673346693	-0.813184089207795\\
0.161302958637679	-0.809619238476954\\
0.152304609218437	-0.805070371891943\\
0.13627254509018	-0.796098946117737\\
0.132188957176207	-0.793587174348697\\
0.120240480961924	-0.786252889299523\\
0.107126535443859	-0.777555110220441\\
0.104208416833667	-0.77562315720775\\
0.0881763527054105	-0.764164839203354\\
0.0847239175286446	-0.761523046092184\\
0.0721442885771539	-0.751909514130287\\
0.064241461299671	-0.745490981963928\\
0.0561122244488974	-0.738895348584444\\
0.0451288313810076	-0.729458917835671\\
0.0400801603206409	-0.725124673100651\\
0.0271702987996958	-0.713426853707415\\
0.0240480961923843	-0.710599232962272\\
0.010193360583747	-0.697394789579158\\
0.00801603206412782	-0.69532019044634\\
-0.00594143293131158	-0.681362725450902\\
-0.00801603206412826	-0.679288126318085\\
-0.0213479068822197	-0.665330661322646\\
-0.0240480961923848	-0.662503040577502\\
-0.0361197728705266	-0.649298597194389\\
-0.0400801603206413	-0.644964352459369\\
-0.050334775756504	-0.633266533066132\\
-0.0561122244488979	-0.626670899686648\\
-0.0640578544475958	-0.617234468937876\\
-0.0721442885771544	-0.607620936975978\\
-0.0773435796145168	-0.601202404809619\\
-0.0881763527054109	-0.587812133792533\\
-0.0902380540757654	-0.585170340681363\\
-0.102710955361554	-0.569138276553106\\
-0.104208416833667	-0.567206323540416\\
-0.114761222290391	-0.55310621242485\\
-0.120240480961924	-0.545771927375675\\
-0.126524458918498	-0.537074148296593\\
-0.13627254509018	-0.523553855937377\\
-0.138026478403903	-0.521042084168337\\
-0.149159038093379	-0.50501002004008\\
-0.152304609218437	-0.500461153455069\\
-0.160001583014829	-0.488977955911824\\
-0.168336673346694	-0.476510742514407\\
-0.170649960884251	-0.472945891783567\\
-0.180987089305424	-0.456913827655311\\
-0.18436873747495	-0.451641273546945\\
-0.191071979166074	-0.440881763527054\\
-0.200400801603207	-0.425861624657147\\
-0.201012065831786	-0.424849699398798\\
-0.210603949389609	-0.408817635270541\\
-0.216432865731463	-0.399035961532034\\
-0.220058021957355	-0.392785571142285\\
-0.229290611656081	-0.376753507014028\\
-0.232464929859719	-0.371199035652408\\
-0.238297033847968	-0.360721442885771\\
-0.247173167878561	-0.344689378757515\\
-0.248496993987976	-0.342272817339309\\
-0.255766886224913	-0.328657314629258\\
-0.264289549707929	-0.312625250501002\\
-0.264529058116232	-0.312168908783128\\
-0.272502383661634	-0.296593186372745\\
-0.280561122244489	-0.28078167434801\\
-0.280670880448816	-0.280561122244489\\
-0.288535564991394	-0.264529058116232\\
-0.296360600228989	-0.248496993987976\\
-0.296593186372745	-0.248013345791166\\
-0.303895898217337	-0.232464929859719\\
-0.311376856476042	-0.216432865731463\\
-0.312625250501002	-0.213718183196745\\
-0.318610457811488	-0.200400801603207\\
-0.325750495336266	-0.18436873747495\\
-0.328657314629258	-0.177753719817951\\
-0.332704086079932	-0.168336673346694\\
-0.339506322152324	-0.152304609218437\\
-0.344689378757515	-0.139950897029441\\
-0.346199540087413	-0.13627254509018\\
-0.352667032756418	-0.120240480961924\\
-0.359067068258364	-0.104208416833667\\
-0.360721442885771	-0.0999879915570855\\
-0.365253350245319	-0.0881763527054109\\
-0.371314638075837	-0.0721442885771544\\
-0.376753507014028	-0.0575645975673489\\
-0.377284149442735	-0.0561122244488979\\
-0.383009686993798	-0.0400801603206413\\
-0.388652960980946	-0.0240480961923848\\
-0.392785571142285	-0.0120996199781018\\
-0.394169229744254	-0.00801603206412826\\
-0.399472542357823	0.00801603206412782\\
-0.404685025109202	0.0240480961923843\\
-0.408817635270541	0.0370151312062466\\
-0.409774840597943	0.0400801603206409\\
-0.414642913275298	0.0561122244488974\\
-0.419410830460605	0.0721442885771539\\
-0.424076362139966	0.0881763527054105\\
-0.424849699398798	0.0909204888630098\\
-0.428518805443234	0.104208416833667\\
-0.432827353397701	0.120240480961924\\
-0.437022813017709	0.13627254509018\\
-0.440881763527054	0.151448076190245\\
-0.441095225623083	0.152304609218437\\
-0.444928534977728	0.168336673346693\\
-0.448637216736993	0.18436873747495\\
-0.452217531449852	0.200400801603206\\
-0.455665433630351	0.216432865731463\\
-0.456913827655311	0.222524544236743\\
-0.458909321024828	0.232464929859719\\
-0.461971918266909	0.248496993987976\\
-0.464888270402216	0.264529058116232\\
-0.467653230225484	0.280561122244489\\
-0.470261256084326	0.296593186372745\\
-0.472706383375263	0.312625250501002\\
-0.472945891783567	0.314333380168508\\
-0.474911396503024	0.328657314629258\\
-0.476934843536038	0.344689378757515\\
-0.478777995771816	0.360721442885771\\
-0.48043346647087	0.376753507014028\\
-0.481893321940788	0.392785571142285\\
-0.483149039569969	0.408817635270541\\
-0.484191461832447	0.424849699398798\\
-0.485010745812909	0.440881763527054\\
-0.485596307742298	0.456913827655311\\
-0.48593676196695	0.472945891783567\\
-0.486019853696573	0.488977955911824\\
-0.485832384786766	0.50501002004008\\
-0.485360131708188	0.521042084168337\\
-0.484587754734332	0.537074148296593\\
-0.483498697240291	0.55310621242485\\
-0.482075073842231	0.569138276553106\\
-0.480297545917278	0.585170340681363\\
-0.47814518282093	0.601202404809619\\
-0.475595306857631	0.617234468937876\\
-0.472945891783567	0.631558403398626\\
}--cycle;


\addplot[area legend,solid,fill=mycolor12,draw=black,forget plot]
table[row sep=crcr] {%
x	y\\
-0.200400801603207	0.304862370138295\\
-0.197344662109513	0.312625250501002\\
-0.189390682595018	0.328657314629258\\
-0.18436873747495	0.336847857012185\\
-0.177891791257849	0.344689378757515\\
-0.168336673346694	0.35424449666867\\
-0.158402802348644	0.360721442885771\\
-0.152304609218437	0.364110634083375\\
-0.13627254509018	0.368853940670599\\
-0.120240480961924	0.369881171754941\\
-0.104208416833667	0.367869101570326\\
-0.0881763527054109	0.363336633675013\\
-0.082078159575204	0.360721442885771\\
-0.0721442885771544	0.356466805104686\\
-0.0561122244488979	0.34759700820266\\
-0.0518105910934108	0.344689378757515\\
-0.0400801603206413	0.336766468226409\\
-0.0297366567970107	0.328657314629258\\
-0.0240480961923848	0.324199859313259\\
-0.0110721715578217	0.312625250501002\\
-0.00801603206412826	0.309899854870135\\
0.00529063643326164	0.296593186372745\\
0.00801603206412782	0.293867790741879\\
0.0200260317046286	0.280561122244489\\
0.0240480961923843	0.276103666928489\\
0.0335674421125732	0.264529058116232\\
0.0400801603206409	0.256606147585127\\
0.0462057588758648	0.248496993987976\\
0.0561122244488974	0.235372559304864\\
0.0581431221874816	0.232464929859719\\
0.0692259403854356	0.216432865731463\\
0.0721442885771539	0.212178227950381\\
0.0796438295284159	0.200400801603206\\
0.0881763527054105	0.186983928264192\\
0.0897289331740241	0.18436873747495\\
0.099068780199145	0.168336673346693\\
0.104208416833667	0.159452267902991\\
0.108078661342575	0.152304609218437\\
0.116632841383845	0.13627254509018\\
0.120240480961924	0.129400209831093\\
0.124752283040789	0.120240480961924\\
0.132520874517804	0.104208416833667\\
0.13627254509018	0.0963088504902381\\
0.139905087192154	0.0881763527054105\\
0.146892501509108	0.0721442885771539\\
0.152304609218437	0.0595014156465018\\
0.153672158159379	0.0561122244488974\\
0.159885987743175	0.0400801603206409\\
0.165975112210303	0.0240480961923843\\
0.168336673346693	0.0175711499752883\\
0.171621426422992	0.00801603206412782\\
0.176904626791464	-0.00801603206412826\\
0.182007176338558	-0.0240480961923848\\
0.18436873747495	-0.0318896179377154\\
0.186695011676916	-0.0400801603206413\\
0.190963287470592	-0.0561122244488979\\
0.194988693893878	-0.0721442885771544\\
0.198748967636206	-0.0881763527054109\\
0.200400801603206	-0.0959392330681169\\
0.202057402535111	-0.104208416833667\\
0.204912603682072	-0.120240480961924\\
0.207426878711987	-0.13627254509018\\
0.209566606502076	-0.152304609218437\\
0.211293229096941	-0.168336673346694\\
0.212562354681171	-0.18436873747495\\
0.213322657300812	-0.200400801603207\\
0.213514517539743	-0.216432865731463\\
0.213068330138315	-0.232464929859719\\
0.211902379345737	-0.248496993987976\\
0.209920147523395	-0.264529058116232\\
0.207006872475959	-0.280561122244489\\
0.203025096977945	-0.296593186372745\\
0.200400801603206	-0.304862370138296\\
0.197344662109512	-0.312625250501002\\
0.189390682595018	-0.328657314629258\\
0.18436873747495	-0.336847857012184\\
0.17789179125785	-0.344689378757515\\
0.168336673346693	-0.354244496668672\\
0.158402802348643	-0.360721442885771\\
0.152304609218437	-0.364110634083375\\
0.13627254509018	-0.3688539406706\\
0.120240480961924	-0.369881171754941\\
0.104208416833667	-0.367869101570326\\
0.0881763527054105	-0.363336633675014\\
0.0820781595752036	-0.360721442885771\\
0.0721442885771539	-0.356466805104686\\
0.0561122244488974	-0.347597008202659\\
0.0518105910934104	-0.344689378757515\\
0.0400801603206409	-0.336766468226409\\
0.0297366567970103	-0.328657314629258\\
0.0240480961923843	-0.324199859313259\\
0.0110721715578212	-0.312625250501002\\
0.00801603206412782	-0.309899854870136\\
-0.00529063643326208	-0.296593186372745\\
-0.00801603206412826	-0.293867790741879\\
-0.020026031704629	-0.280561122244489\\
-0.0240480961923848	-0.276103666928489\\
-0.0335674421125736	-0.264529058116232\\
-0.0400801603206413	-0.256606147585127\\
-0.0462057588758648	-0.248496993987976\\
-0.0561122244488979	-0.235372559304863\\
-0.058143122187482	-0.232464929859719\\
-0.0692259403854342	-0.216432865731463\\
-0.0721442885771544	-0.212178227950378\\
-0.0796438295284154	-0.200400801603207\\
-0.0881763527054109	-0.186983928264193\\
-0.0897289331740245	-0.18436873747495\\
-0.0990687801991455	-0.168336673346694\\
-0.104208416833667	-0.159452267902992\\
-0.108078661342576	-0.152304609218437\\
-0.116632841383844	-0.13627254509018\\
-0.120240480961924	-0.129400209831092\\
-0.124752283040789	-0.120240480961924\\
-0.132520874517804	-0.104208416833667\\
-0.13627254509018	-0.096308850490239\\
-0.139905087192154	-0.0881763527054109\\
-0.146892501509108	-0.0721442885771544\\
-0.152304609218437	-0.0595014156465017\\
-0.15367215815938	-0.0561122244488979\\
-0.159885987743175	-0.0400801603206413\\
-0.165975112210302	-0.0240480961923848\\
-0.168336673346694	-0.0175711499752838\\
-0.171621426422991	-0.00801603206412826\\
-0.176904626791464	0.00801603206412782\\
-0.182007176338559	0.0240480961923843\\
-0.18436873747495	0.031889617937715\\
-0.186695011676916	0.0400801603206409\\
-0.190963287470593	0.0561122244488974\\
-0.194988693893878	0.0721442885771539\\
-0.198748967636206	0.0881763527054105\\
-0.200400801603207	0.0959392330681169\\
-0.202057402535111	0.104208416833667\\
-0.204912603682072	0.120240480961924\\
-0.207426878711987	0.13627254509018\\
-0.209566606502074	0.152304609218437\\
-0.211293229096941	0.168336673346693\\
-0.212562354681169	0.18436873747495\\
-0.213322657300812	0.200400801603206\\
-0.213514517539743	0.216432865731463\\
-0.213068330138315	0.232464929859719\\
-0.211902379345734	0.248496993987976\\
-0.209920147523395	0.264529058116232\\
-0.207006872475959	0.280561122244489\\
-0.203025096977946	0.296593186372745\\
-0.200400801603207	0.304862370138295\\
}--cycle;

\end{axis}
\end{tikzpicture}%
    \caption{$p(\vec{x} \given \vec{\mu}, \mat{\Sigma})$}
    \label{transformed_2d_pdf}
  \end{subfigure}
  \begin{subfigure}[t]{0.49\textwidth}
    % This file was created by matlab2tikz.
% Minimal pgfplots version: 1.3
%
\tikzsetnextfilename{2d_transformed_pdf}
\definecolor{mycolor1}{rgb}{0.01430,0.01430,0.01430}%
\definecolor{mycolor2}{rgb}{0.17488,0.10789,0.37159}%
\definecolor{mycolor3}{rgb}{0.06860,0.30440,0.50680}%
\definecolor{mycolor4}{rgb}{0.00000,0.51886,0.29379}%
\definecolor{mycolor5}{rgb}{0.21670,0.69660,0.00000}%
\definecolor{mycolor6}{rgb}{0.78713,0.79171,0.20277}%
\definecolor{mycolor7}{rgb}{0.96920,0.92730,0.89610}%
%
\begin{tikzpicture}

\begin{axis}[%
width=0.95092\squarefigurewidth,
height=\squarefigureheight,
at={(0\squarefigurewidth,0\squarefigureheight)},
scale only axis,
xmin=-4,
xmax=4,
xlabel={$x_1$},
ymin=-4,
ymax=4,
ylabel={$x_2$},
axis x line*=bottom,
axis y line*=left
]

\addplot[area legend,solid,fill=mycolor1,draw=black,forget plot]
table[row sep=crcr] {%
x	y\\
-4	4\\
-3.98396793587174	4\\
-3.96793587174349	4\\
-3.95190380761523	4\\
-3.93587174348697	4\\
-3.91983967935872	4\\
-3.90380761523046	4\\
-3.8877755511022	4\\
-3.87174348697395	4\\
-3.85571142284569	4\\
-3.83967935871743	4\\
-3.82364729458918	4\\
-3.80761523046092	4\\
-3.79158316633267	4\\
-3.77555110220441	4\\
-3.75951903807615	4\\
-3.7434869739479	4\\
-3.72745490981964	4\\
-3.71142284569138	4\\
-3.69539078156313	4\\
-3.67935871743487	4\\
-3.66332665330661	4\\
-3.64729458917836	4\\
-3.6312625250501	4\\
-3.61523046092184	4\\
-3.59919839679359	4\\
-3.58316633266533	4\\
-3.56713426853707	4\\
-3.55110220440882	4\\
-3.53507014028056	4\\
-3.5190380761523	4\\
-3.50300601202405	4\\
-3.48697394789579	4\\
-3.47094188376753	4\\
-3.45490981963928	4\\
-3.43887775551102	4\\
-3.42284569138277	4\\
-3.40681362725451	4\\
-3.39078156312625	4\\
-3.374749498998	4\\
-3.35871743486974	4\\
-3.34268537074148	4\\
-3.32665330661323	4\\
-3.31062124248497	4\\
-3.29458917835671	4\\
-3.27855711422846	4\\
-3.2625250501002	4\\
-3.24649298597194	4\\
-3.23046092184369	4\\
-3.21442885771543	4\\
-3.19839679358717	4\\
-3.18236472945892	4\\
-3.16633266533066	4\\
-3.1503006012024	4\\
-3.13426853707415	4\\
-3.11823647294589	4\\
-3.10220440881764	4\\
-3.08617234468938	4\\
-3.07014028056112	4\\
-3.05410821643287	4\\
-3.03807615230461	4\\
-3.02204408817635	4\\
-3.0060120240481	4\\
-2.98997995991984	4\\
-2.97394789579158	4\\
-2.95791583166333	4\\
-2.94188376753507	4\\
-2.92585170340681	4\\
-2.90981963927856	4\\
-2.8937875751503	4\\
-2.87775551102204	4\\
-2.86172344689379	4\\
-2.84569138276553	4\\
-2.82965931863727	4\\
-2.81362725450902	4\\
-2.79759519038076	4\\
-2.7815631262525	4\\
-2.76553106212425	4\\
-2.74949899799599	4\\
-2.73346693386774	4\\
-2.71743486973948	4\\
-2.70140280561122	4\\
-2.68537074148297	4\\
-2.66933867735471	4\\
-2.65330661322645	4\\
-2.6372745490982	4\\
-2.62124248496994	4\\
-2.60521042084168	4\\
-2.58917835671343	4\\
-2.57314629258517	4\\
-2.55711422845691	4\\
-2.54108216432866	4\\
-2.5250501002004	4\\
-2.50901803607214	4\\
-2.49298597194389	4\\
-2.47695390781563	4\\
-2.46092184368737	4\\
-2.44488977955912	4\\
-2.42885771543086	4\\
-2.41282565130261	4\\
-2.39679358717435	4\\
-2.38076152304609	4\\
-2.36472945891784	4\\
-2.34869739478958	4\\
-2.33266533066132	4\\
-2.31663326653307	4\\
-2.30060120240481	4\\
-2.28456913827655	4\\
-2.2685370741483	4\\
-2.25250501002004	4\\
-2.23647294589178	4\\
-2.22044088176353	4\\
-2.20440881763527	4\\
-2.18837675350701	4\\
-2.17234468937876	4\\
-2.1563126252505	4\\
-2.14028056112224	4\\
-2.12424849699399	4\\
-2.10821643286573	4\\
-2.09218436873747	4\\
-2.07615230460922	4\\
-2.06012024048096	4\\
-2.04408817635271	4\\
-2.02805611222445	4\\
-2.01202404809619	4\\
-1.99599198396794	4\\
-1.97995991983968	4\\
-1.96392785571142	4\\
-1.94789579158317	4\\
-1.93186372745491	4\\
-1.91583166332665	4\\
-1.8997995991984	4\\
-1.88376753507014	4\\
-1.86773547094188	4\\
-1.85170340681363	4\\
-1.83567134268537	4\\
-1.81963927855711	4\\
-1.80360721442886	4\\
-1.7875751503006	4\\
-1.77154308617234	4\\
-1.75551102204409	4\\
-1.73947895791583	4\\
-1.72344689378758	4\\
-1.70741482965932	4\\
-1.69138276553106	4\\
-1.67535070140281	4\\
-1.65931863727455	4\\
-1.64328657314629	4\\
-1.62725450901804	4\\
-1.61122244488978	4\\
-1.59519038076152	4\\
-1.57915831663327	4\\
-1.56312625250501	4\\
-1.54709418837675	4\\
-1.5310621242485	4\\
-1.51503006012024	4\\
-1.49899799599198	4\\
-1.48296593186373	4\\
-1.46693386773547	4\\
-1.45090180360721	4\\
-1.43486973947896	4\\
-1.4188376753507	4\\
-1.40280561122244	4\\
-1.38677354709419	4\\
-1.37074148296593	4\\
-1.35470941883768	4\\
-1.33867735470942	4\\
-1.32264529058116	4\\
-1.30661322645291	4\\
-1.29058116232465	4\\
-1.27454909819639	4\\
-1.25851703406814	4\\
-1.24248496993988	4\\
-1.22645290581162	4\\
-1.21042084168337	4\\
-1.19438877755511	4\\
-1.17835671342685	4\\
-1.1623246492986	4\\
-1.14629258517034	4\\
-1.13026052104208	4\\
-1.11422845691383	4\\
-1.09819639278557	4\\
-1.08216432865731	4\\
-1.06613226452906	4\\
-1.0501002004008	4\\
-1.03406813627255	4\\
-1.01803607214429	4\\
-1.00200400801603	4\\
-0.985971943887776	4\\
-0.969939879759519	4\\
-0.953907815631263	4\\
-0.937875751503006	4\\
-0.92184368737475	4\\
-0.905811623246493	4\\
-0.889779559118236	4\\
-0.87374749498998	4\\
-0.857715430861723	4\\
-0.841683366733467	4\\
-0.82565130260521	4\\
-0.809619238476954	4\\
-0.793587174348697	4\\
-0.777555110220441	4\\
-0.761523046092184	4\\
-0.745490981963928	4\\
-0.729458917835671	4\\
-0.713426853707415	4\\
-0.697394789579158	4\\
-0.681362725450902	4\\
-0.665330661322646	4\\
-0.649298597194389	4\\
-0.633266533066132	4\\
-0.617234468937876	4\\
-0.601202404809619	4\\
-0.585170340681363	4\\
-0.569138276553106	4\\
-0.55310621242485	4\\
-0.537074148296593	4\\
-0.521042084168337	4\\
-0.50501002004008	4\\
-0.488977955911824	4\\
-0.472945891783567	4\\
-0.456913827655311	4\\
-0.440881763527054	4\\
-0.424849699398798	4\\
-0.408817635270541	4\\
-0.392785571142285	4\\
-0.376753507014028	4\\
-0.360721442885771	4\\
-0.344689378757515	4\\
-0.328657314629258	4\\
-0.312625250501002	4\\
-0.296593186372745	4\\
-0.280561122244489	4\\
-0.264529058116232	4\\
-0.248496993987976	4\\
-0.232464929859719	4\\
-0.216432865731463	4\\
-0.200400801603207	4\\
-0.18436873747495	4\\
-0.168336673346694	4\\
-0.152304609218437	4\\
-0.13627254509018	4\\
-0.120240480961924	4\\
-0.104208416833667	4\\
-0.0881763527054109	4\\
-0.0721442885771544	4\\
-0.0561122244488979	4\\
-0.0400801603206413	4\\
-0.0240480961923848	4\\
-0.00801603206412826	4\\
0.00801603206412782	4\\
0.0240480961923843	4\\
0.0400801603206409	4\\
0.0561122244488974	4\\
0.0721442885771539	4\\
0.0881763527054105	4\\
0.104208416833667	4\\
0.120240480961924	4\\
0.13627254509018	4\\
0.152304609218437	4\\
0.168336673346693	4\\
0.18436873747495	4\\
0.200400801603206	4\\
0.216432865731463	4\\
0.232464929859719	4\\
0.248496993987976	4\\
0.264529058116232	4\\
0.280561122244489	4\\
0.296593186372745	4\\
0.312625250501002	4\\
0.328657314629258	4\\
0.344689378757515	4\\
0.360721442885771	4\\
0.376753507014028	4\\
0.392785571142285	4\\
0.408817635270541	4\\
0.424849699398798	4\\
0.440881763527054	4\\
0.456913827655311	4\\
0.472945891783567	4\\
0.488977955911824	4\\
0.50501002004008	4\\
0.521042084168337	4\\
0.537074148296593	4\\
0.55310621242485	4\\
0.569138276553106	4\\
0.585170340681363	4\\
0.601202404809619	4\\
0.617234468937876	4\\
0.633266533066132	4\\
0.649298597194389	4\\
0.665330661322646	4\\
0.681362725450902	4\\
0.697394789579159	4\\
0.713426853707415	4\\
0.729458917835672	4\\
0.745490981963928	4\\
0.761523046092185	4\\
0.777555110220441	4\\
0.793587174348698	4\\
0.809619238476954	4\\
0.825651302605211	4\\
0.841683366733467	4\\
0.857715430861724	4\\
0.87374749498998	4\\
0.889779559118236	4\\
0.905811623246493	4\\
0.921843687374749	4\\
0.937875751503006	4\\
0.953907815631262	4\\
0.969939879759519	4\\
0.985971943887775	4\\
1.00200400801603	4\\
1.01803607214429	4\\
1.03406813627254	4\\
1.0501002004008	4\\
1.06613226452906	4\\
1.08216432865731	4\\
1.09819639278557	4\\
1.11422845691383	4\\
1.13026052104208	4\\
1.14629258517034	4\\
1.1623246492986	4\\
1.17835671342685	4\\
1.19438877755511	4\\
1.21042084168337	4\\
1.22645290581162	4\\
1.24248496993988	4\\
1.25851703406814	4\\
1.27454909819639	4\\
1.29058116232465	4\\
1.30661322645291	4\\
1.32264529058116	4\\
1.33867735470942	4\\
1.35470941883768	4\\
1.37074148296593	4\\
1.38677354709419	4\\
1.40280561122244	4\\
1.4188376753507	4\\
1.43486973947896	4\\
1.45090180360721	4\\
1.46693386773547	4\\
1.48296593186373	4\\
1.49899799599198	4\\
1.51503006012024	4\\
1.5310621242485	4\\
1.54709418837675	4\\
1.56312625250501	4\\
1.57915831663327	4\\
1.59519038076152	4\\
1.61122244488978	4\\
1.62725450901804	4\\
1.64328657314629	4\\
1.65931863727455	4\\
1.67535070140281	4\\
1.69138276553106	4\\
1.70741482965932	4\\
1.72344689378758	4\\
1.73947895791583	4\\
1.75551102204409	4\\
1.77154308617235	4\\
1.7875751503006	4\\
1.80360721442886	4\\
1.81963927855711	4\\
1.83567134268537	4\\
1.85170340681363	4\\
1.86773547094188	4\\
1.88376753507014	4\\
1.8997995991984	4\\
1.91583166332665	4\\
1.93186372745491	4\\
1.94789579158317	4\\
1.96392785571142	4\\
1.97995991983968	4\\
1.99599198396794	4\\
2.01202404809619	4\\
2.02805611222445	4\\
2.04408817635271	4\\
2.06012024048096	4\\
2.07615230460922	4\\
2.09218436873747	4\\
2.10821643286573	4\\
2.12424849699399	4\\
2.14028056112224	4\\
2.1563126252505	4\\
2.17234468937876	4\\
2.18837675350701	4\\
2.20440881763527	4\\
2.22044088176353	4\\
2.23647294589178	4\\
2.25250501002004	4\\
2.2685370741483	4\\
2.28456913827655	4\\
2.30060120240481	4\\
2.31663326653307	4\\
2.33266533066132	4\\
2.34869739478958	4\\
2.36472945891784	4\\
2.38076152304609	4\\
2.39679358717435	4\\
2.41282565130261	4\\
2.42885771543086	4\\
2.44488977955912	4\\
2.46092184368737	4\\
2.47695390781563	4\\
2.49298597194389	4\\
2.50901803607214	4\\
2.5250501002004	4\\
2.54108216432866	4\\
2.55711422845691	4\\
2.57314629258517	4\\
2.58917835671343	4\\
2.60521042084168	4\\
2.62124248496994	4\\
2.6372745490982	4\\
2.65330661322645	4\\
2.66933867735471	4\\
2.68537074148297	4\\
2.70140280561122	4\\
2.71743486973948	4\\
2.73346693386774	4\\
2.74949899799599	4\\
2.76553106212425	4\\
2.7815631262525	4\\
2.79759519038076	4\\
2.81362725450902	4\\
2.82965931863727	4\\
2.84569138276553	4\\
2.86172344689379	4\\
2.87775551102204	4\\
2.8937875751503	4\\
2.90981963927856	4\\
2.92585170340681	4\\
2.94188376753507	4\\
2.95791583166333	4\\
2.97394789579158	4\\
2.98997995991984	4\\
3.0060120240481	4\\
3.02204408817635	4\\
3.03807615230461	4\\
3.05410821643287	4\\
3.07014028056112	4\\
3.08617234468938	4\\
3.10220440881764	4\\
3.11823647294589	4\\
3.13426853707415	4\\
3.1503006012024	4\\
3.16633266533066	4\\
3.18236472945892	4\\
3.19839679358717	4\\
3.21442885771543	4\\
3.23046092184369	4\\
3.24649298597194	4\\
3.2625250501002	4\\
3.27855711422846	4\\
3.29458917835671	4\\
3.31062124248497	4\\
3.32665330661323	4\\
3.34268537074148	4\\
3.35871743486974	4\\
3.374749498998	4\\
3.39078156312625	4\\
3.40681362725451	4\\
3.42284569138277	4\\
3.43887775551102	4\\
3.45490981963928	4\\
3.47094188376754	4\\
3.48697394789579	4\\
3.50300601202405	4\\
3.51903807615231	4\\
3.53507014028056	4\\
3.55110220440882	4\\
3.56713426853707	4\\
3.58316633266533	4\\
3.59919839679359	4\\
3.61523046092184	4\\
3.6312625250501	4\\
3.64729458917836	4\\
3.66332665330661	4\\
3.67935871743487	4\\
3.69539078156313	4\\
3.71142284569138	4\\
3.72745490981964	4\\
3.7434869739479	4\\
3.75951903807615	4\\
3.77555110220441	4\\
3.79158316633267	4\\
3.80761523046092	4\\
3.82364729458918	4\\
3.83967935871743	4\\
3.85571142284569	4\\
3.87174348697395	4\\
3.8877755511022	4\\
3.90380761523046	4\\
3.91983967935872	4\\
3.93587174348697	4\\
3.95190380761523	4\\
3.96793587174349	4\\
3.98396793587174	4\\
4	4\\
4	3.98396793587174\\
4	3.96793587174349\\
4	3.95190380761523\\
4	3.93587174348697\\
4	3.91983967935872\\
4	3.90380761523046\\
4	3.8877755511022\\
4	3.87174348697395\\
4	3.85571142284569\\
4	3.83967935871743\\
4	3.82364729458918\\
4	3.80761523046092\\
4	3.79158316633267\\
4	3.77555110220441\\
4	3.75951903807615\\
4	3.7434869739479\\
4	3.72745490981964\\
4	3.71142284569138\\
4	3.69539078156313\\
4	3.67935871743487\\
4	3.66332665330661\\
4	3.64729458917836\\
4	3.6312625250501\\
4	3.61523046092184\\
4	3.59919839679359\\
4	3.58316633266533\\
4	3.56713426853707\\
4	3.55110220440882\\
4	3.53507014028056\\
4	3.51903807615231\\
4	3.50300601202405\\
4	3.48697394789579\\
4	3.47094188376754\\
4	3.45490981963928\\
4	3.43887775551102\\
4	3.42284569138277\\
4	3.40681362725451\\
4	3.39078156312625\\
4	3.374749498998\\
4	3.35871743486974\\
4	3.34268537074148\\
4	3.32665330661323\\
4	3.31062124248497\\
4	3.29458917835671\\
4	3.27855711422846\\
4	3.2625250501002\\
4	3.24649298597194\\
4	3.23046092184369\\
4	3.21442885771543\\
4	3.19839679358717\\
4	3.18236472945892\\
4	3.16633266533066\\
4	3.1503006012024\\
4	3.13426853707415\\
4	3.11823647294589\\
4	3.10220440881764\\
4	3.08617234468938\\
4	3.07014028056112\\
4	3.05410821643287\\
4	3.03807615230461\\
4	3.02204408817635\\
4	3.0060120240481\\
4	2.98997995991984\\
4	2.97394789579158\\
4	2.95791583166333\\
4	2.94188376753507\\
4	2.92585170340681\\
4	2.90981963927856\\
4	2.8937875751503\\
4	2.87775551102204\\
4	2.86172344689379\\
4	2.84569138276553\\
4	2.82965931863727\\
4	2.81362725450902\\
4	2.79759519038076\\
4	2.7815631262525\\
4	2.76553106212425\\
4	2.74949899799599\\
4	2.73346693386774\\
4	2.71743486973948\\
4	2.70140280561122\\
4	2.68537074148297\\
4	2.66933867735471\\
4	2.65330661322645\\
4	2.6372745490982\\
4	2.62124248496994\\
4	2.60521042084168\\
4	2.58917835671343\\
4	2.57314629258517\\
4	2.55711422845691\\
4	2.54108216432866\\
4	2.5250501002004\\
4	2.50901803607214\\
4	2.49298597194389\\
4	2.47695390781563\\
4	2.46092184368737\\
4	2.44488977955912\\
4	2.42885771543086\\
4	2.41282565130261\\
4	2.39679358717435\\
4	2.38076152304609\\
4	2.36472945891784\\
4	2.34869739478958\\
4	2.33266533066132\\
4	2.31663326653307\\
4	2.30060120240481\\
4	2.28456913827655\\
4	2.2685370741483\\
4	2.25250501002004\\
4	2.23647294589178\\
4	2.22044088176353\\
4	2.20440881763527\\
4	2.18837675350701\\
4	2.17234468937876\\
4	2.1563126252505\\
4	2.14028056112224\\
4	2.12424849699399\\
4	2.10821643286573\\
4	2.09218436873747\\
4	2.07615230460922\\
4	2.06012024048096\\
4	2.04408817635271\\
4	2.02805611222445\\
4	2.01202404809619\\
4	1.99599198396794\\
4	1.97995991983968\\
4	1.96392785571142\\
4	1.94789579158317\\
4	1.93186372745491\\
4	1.91583166332665\\
4	1.8997995991984\\
4	1.88376753507014\\
4	1.86773547094188\\
4	1.85170340681363\\
4	1.83567134268537\\
4	1.81963927855711\\
4	1.80360721442886\\
4	1.7875751503006\\
4	1.77154308617235\\
4	1.75551102204409\\
4	1.73947895791583\\
4	1.72344689378758\\
4	1.70741482965932\\
4	1.69138276553106\\
4	1.67535070140281\\
4	1.65931863727455\\
4	1.64328657314629\\
4	1.62725450901804\\
4	1.61122244488978\\
4	1.59519038076152\\
4	1.57915831663327\\
4	1.56312625250501\\
4	1.54709418837675\\
4	1.5310621242485\\
4	1.51503006012024\\
4	1.49899799599198\\
4	1.48296593186373\\
4	1.46693386773547\\
4	1.45090180360721\\
4	1.43486973947896\\
4	1.4188376753507\\
4	1.40280561122244\\
4	1.38677354709419\\
4	1.37074148296593\\
4	1.35470941883768\\
4	1.33867735470942\\
4	1.32264529058116\\
4	1.30661322645291\\
4	1.29058116232465\\
4	1.27454909819639\\
4	1.25851703406814\\
4	1.24248496993988\\
4	1.22645290581162\\
4	1.21042084168337\\
4	1.19438877755511\\
4	1.17835671342685\\
4	1.1623246492986\\
4	1.14629258517034\\
4	1.13026052104208\\
4	1.11422845691383\\
4	1.09819639278557\\
4	1.08216432865731\\
4	1.06613226452906\\
4	1.0501002004008\\
4.00000000000001	1.03406813627254\\
4.00000000000001	1.01803607214429\\
4.00000000000001	1.00200400801603\\
4.00000000000001	0.985971943887775\\
4.00000000000001	0.969939879759519\\
4.00000000000002	0.953907815631262\\
4.00000000000002	0.937875751503006\\
4.00000000000003	0.921843687374749\\
4.00000000000003	0.905811623246493\\
4.00000000000004	0.889779559118236\\
4.00000000000005	0.87374749498998\\
4.00000000000006	0.857715430861724\\
4.00000000000007	0.841683366733467\\
4.00000000000008	0.825651302605211\\
4.0000000000001	0.809619238476954\\
4.00000000000012	0.793587174348698\\
4.00000000000014	0.777555110220441\\
4.00000000000017	0.761523046092185\\
4.00000000000021	0.745490981963928\\
4.00000000000025	0.729458917835672\\
4.0000000000003	0.713426853707415\\
4.00000000000036	0.697394789579159\\
4.00000000000043	0.681362725450902\\
4.00000000000052	0.665330661322646\\
4.00000000000062	0.649298597194389\\
4.00000000000073	0.633266533066132\\
4.00000000000087	0.617234468937876\\
4.00000000000104	0.601202404809619\\
4.00000000000123	0.585170340681363\\
4.00000000000146	0.569138276553106\\
4.00000000000172	0.55310621242485\\
4.00000000000204	0.537074148296593\\
4.00000000000241	0.521042084168337\\
4.00000000000284	0.50501002004008\\
4.00000000000334	0.488977955911824\\
4.00000000000392	0.472945891783567\\
4.00000000000461	0.456913827655311\\
4.0000000000054	0.440881763527054\\
4.00000000000632	0.424849699398798\\
4.00000000000739	0.408817635270541\\
4.00000000000863	0.392785571142285\\
4.00000000001006	0.376753507014028\\
4.00000000001171	0.360721442885771\\
4.00000000001362	0.344689378757515\\
4.00000000001582	0.328657314629258\\
4.00000000001835	0.312625250501002\\
4.00000000002125	0.296593186372745\\
4.00000000002458	0.280561122244489\\
4.00000000002839	0.264529058116232\\
4.00000000003274	0.248496993987976\\
4.00000000003772	0.232464929859719\\
4.00000000004339	0.216432865731463\\
4.00000000004985	0.200400801603206\\
4.0000000000572	0.18436873747495\\
4.00000000006553	0.168336673346693\\
4.00000000007498	0.152304609218437\\
4.00000000008568	0.13627254509018\\
4.00000000009777	0.120240480961924\\
4.00000000011142	0.104208416833667\\
4.0000000001268	0.0881763527054105\\
4.00000000014412	0.0721442885771539\\
4.00000000016357	0.0561122244488974\\
4.0000000001854	0.0400801603206409\\
4.00000000020986	0.0240480961923843\\
4.00000000023723	0.00801603206412782\\
4.0000000002678	-0.00801603206412826\\
4.00000000030191	-0.0240480961923848\\
4.00000000033991	-0.0400801603206413\\
4.00000000038217	-0.0561122244488979\\
4.0000000004291	-0.0721442885771544\\
4.00000000048115	-0.0881763527054109\\
4.00000000053878	-0.104208416833667\\
4.00000000060251	-0.120240480961924\\
4.00000000067287	-0.13627254509018\\
4.00000000075043	-0.152304609218437\\
4.0000000008358	-0.168336673346694\\
4.00000000092962	-0.18436873747495\\
4.00000000103259	-0.200400801603207\\
4.00000000114542	-0.216432865731463\\
4.00000000126886	-0.232464929859719\\
4.00000000140371	-0.248496993987976\\
4.0000000015508	-0.264529058116232\\
4.000000001711	-0.280561122244489\\
4.0000000018852	-0.296593186372745\\
4.00000000207433	-0.312625250501002\\
4.00000000227937	-0.328657314629258\\
4.00000000250129	-0.344689378757515\\
4.00000000274112	-0.360721442885771\\
4.0000000029999	-0.376753507014028\\
4.00000000327869	-0.392785571142285\\
4.00000000357855	-0.408817635270541\\
4.00000000390058	-0.424849699398798\\
4.00000000424585	-0.440881763527054\\
4.00000000461546	-0.456913827655311\\
4.00000000501049	-0.472945891783567\\
4.00000000543199	-0.488977955911824\\
4.00000000588101	-0.50501002004008\\
4.00000000635857	-0.521042084168337\\
4.00000000686564	-0.537074148296593\\
4.00000000740317	-0.55310621242485\\
4.00000000797201	-0.569138276553106\\
4.000000008573	-0.585170340681363\\
4.00000000920686	-0.601202404809619\\
4.00000000987427	-0.617234468937876\\
4.00000001057579	-0.633266533066132\\
4.00000001131188	-0.649298597194389\\
4.00000001208289	-0.665330661322646\\
4.00000001288907	-0.681362725450902\\
4.0000000137305	-0.697394789579158\\
4.00000001460716	-0.713426853707415\\
4.00000001551884	-0.729458917835671\\
4.00000001646521	-0.745490981963928\\
4.00000001744574	-0.761523046092184\\
4.00000001845975	-0.777555110220441\\
4.00000001950637	-0.793587174348697\\
4.00000002058456	-0.809619238476954\\
4.00000002169307	-0.82565130260521\\
4.00000002283046	-0.841683366733467\\
4.00000002399511	-0.857715430861723\\
4.00000002518518	-0.87374749498998\\
4.00000002639865	-0.889779559118236\\
4.00000002763329	-0.905811623246493\\
4.0000000288867	-0.92184368737475\\
4.00000003015626	-0.937875751503006\\
4.00000003143919	-0.953907815631263\\
4.00000003273253	-0.969939879759519\\
4.00000003403314	-0.985971943887776\\
4.00000003533775	-1.00200400801603\\
4.00000003664291	-1.01803607214429\\
4.00000003794508	-1.03406813627255\\
4.00000003924056	-1.0501002004008\\
4.00000004052558	-1.06613226452906\\
4.00000004179628	-1.08216432865731\\
4.00000004304873	-1.09819639278557\\
4.00000004427895	-1.11422845691383\\
4.00000004548295	-1.13026052104208\\
4.00000004665672	-1.14629258517034\\
4.00000004779628	-1.1623246492986\\
4.00000004889769	-1.17835671342685\\
4.00000004995707	-1.19438877755511\\
4.0000000509706	-1.21042084168337\\
4.00000005193462	-1.22645290581162\\
4.00000005284555	-1.24248496993988\\
4.0000000537	-1.25851703406814\\
4.00000005449471	-1.27454909819639\\
4.00000005522666	-1.29058116232465\\
4.00000005589301	-1.30661322645291\\
4.00000005649117	-1.32264529058116\\
4.00000005701878	-1.33867735470942\\
4.00000005747376	-1.35470941883768\\
4.0000000578543	-1.37074148296593\\
4.00000005815886	-1.38677354709419\\
4.00000005838624	-1.40280561122244\\
4.00000005853552	-1.4188376753507\\
4.00000005860608	-1.43486973947896\\
4.00000005859766	-1.45090180360721\\
4.00000005851027	-1.46693386773547\\
4.00000005834428	-1.48296593186373\\
4.00000005810036	-1.49899799599198\\
4.00000005777948	-1.51503006012024\\
4.00000005738293	-1.5310621242485\\
4.0000000569123	-1.54709418837675\\
4.00000005636947	-1.56312625250501\\
4.00000005575656	-1.57915831663327\\
4.00000005507599	-1.59519038076152\\
4.00000005433042	-1.61122244488978\\
4.0000000535227	-1.62725450901804\\
4.00000005265594	-1.64328657314629\\
4.00000005173339	-1.65931863727455\\
4.00000005075851	-1.67535070140281\\
4.00000004973489	-1.69138276553106\\
4.00000004866623	-1.70741482965932\\
4.00000004755636	-1.72344689378758\\
4.00000004640917	-1.73947895791583\\
4.00000004522861	-1.75551102204409\\
4.00000004401868	-1.77154308617234\\
4.00000004278339	-1.7875751503006\\
4.00000004152672	-1.80360721442886\\
4.00000004025264	-1.81963927855711\\
4.00000003896506	-1.83567134268537\\
4.00000003766784	-1.85170340681363\\
4.00000003636474	-1.86773547094188\\
4.0000000350594	-1.88376753507014\\
4.00000003375536	-1.8997995991984\\
4.00000003245603	-1.91583166332665\\
4.00000003116465	-1.93186372745491\\
4.00000002988433	-1.94789579158317\\
4.00000002861798	-1.96392785571142\\
4.00000002736837	-1.97995991983968\\
4.00000002613804	-1.99599198396794\\
4.00000002492938	-2.01202404809619\\
4.00000002374457	-2.02805611222445\\
4.00000002258559	-2.04408817635271\\
4.00000002145422	-2.06012024048096\\
4.00000002035207	-2.07615230460922\\
4.00000001928051	-2.09218436873747\\
4.00000001824076	-2.10821643286573\\
4.00000001723382	-2.12424849699399\\
4.00000001626052	-2.14028056112224\\
4.00000001532151	-2.1563126252505\\
4.00000001441728	-2.17234468937876\\
4.00000001354812	-2.18837675350701\\
4.00000001271421	-2.20440881763527\\
4.00000001191554	-2.22044088176353\\
4.000000011152	-2.23647294589178\\
4.00000001042331	-2.25250501002004\\
4.00000000972911	-2.2685370741483\\
4.00000000906891	-2.28456913827655\\
4.00000000844211	-2.30060120240481\\
4.00000000784804	-2.31663326653307\\
4.00000000728595	-2.33266533066132\\
4.00000000675499	-2.34869739478958\\
4.00000000625429	-2.36472945891784\\
4.0000000057829	-2.38076152304609\\
4.00000000533983	-2.39679358717435\\
4.00000000492407	-2.41282565130261\\
4.00000000453455	-2.42885771543086\\
4.00000000417022	-2.44488977955912\\
4.00000000383	-2.46092184368737\\
4.00000000351279	-2.47695390781563\\
4.00000000321751	-2.49298597194389\\
4.00000000294308	-2.50901803607214\\
4.00000000268843	-2.5250501002004\\
4.0000000024525	-2.54108216432866\\
4.00000000223426	-2.55711422845691\\
4.0000000020327	-2.57314629258517\\
4.00000000184683	-2.58917835671343\\
4.0000000016757	-2.60521042084168\\
4.00000000151837	-2.62124248496994\\
4.00000000137396	-2.6372745490982\\
4.00000000124161	-2.65330661322645\\
4.00000000112049	-2.66933867735471\\
4.00000000100983	-2.68537074148297\\
4.00000000090887	-2.70140280561122\\
4.0000000008169	-2.71743486973948\\
4.00000000073325	-2.73346693386774\\
4.00000000065728	-2.74949899799599\\
4.00000000058838	-2.76553106212425\\
4.000000000526	-2.7815631262525\\
4.0000000004696	-2.79759519038076\\
4.00000000041868	-2.81362725450902\\
4.00000000037278	-2.82965931863727\\
4.00000000033146	-2.84569138276553\\
4.00000000029433	-2.86172344689379\\
4.000000000261	-2.87775551102204\\
4.00000000023114	-2.8937875751503\\
4.00000000020441	-2.90981963927856\\
4.00000000018053	-2.92585170340681\\
4.00000000015923	-2.94188376753507\\
4.00000000014025	-2.95791583166333\\
4.00000000012337	-2.97394789579158\\
4.00000000010837	-2.98997995991984\\
4.00000000009507	-3.0060120240481\\
4.00000000008329	-3.02204408817635\\
4.00000000007287	-3.03807615230461\\
4.00000000006367	-3.05410821643287\\
4.00000000005555	-3.07014028056112\\
4.0000000000484	-3.08617234468938\\
4.00000000004212	-3.10220440881764\\
4.0000000000366	-3.11823647294589\\
4.00000000003177	-3.13426853707415\\
4.00000000002753	-3.1503006012024\\
4.00000000002383	-3.16633266533066\\
4.0000000000206	-3.18236472945892\\
4.00000000001778	-3.19839679358717\\
4.00000000001532	-3.21442885771543\\
4.00000000001319	-3.23046092184369\\
4.00000000001134	-3.24649298597194\\
4.00000000000974	-3.2625250501002\\
4.00000000000835	-3.27855711422846\\
4.00000000000715	-3.29458917835671\\
4.00000000000611	-3.31062124248497\\
4.00000000000522	-3.32665330661323\\
4.00000000000445	-3.34268537074148\\
4.00000000000379	-3.35871743486974\\
4.00000000000322	-3.374749498998\\
4.00000000000274	-3.39078156312625\\
4.00000000000232	-3.40681362725451\\
4.00000000000197	-3.42284569138277\\
4.00000000000166	-3.43887775551102\\
4.00000000000141	-3.45490981963928\\
4.00000000000119	-3.47094188376753\\
4.000000000001	-3.48697394789579\\
4.00000000000084	-3.50300601202405\\
4.00000000000071	-3.5190380761523\\
4.00000000000059	-3.53507014028056\\
4.0000000000005	-3.55110220440882\\
4.00000000000042	-3.56713426853707\\
4.00000000000035	-3.58316633266533\\
4.00000000000029	-3.59919839679359\\
4.00000000000024	-3.61523046092184\\
4.0000000000002	-3.6312625250501\\
4.00000000000017	-3.64729458917836\\
4.00000000000014	-3.66332665330661\\
4.00000000000011	-3.67935871743487\\
4.0000000000001	-3.69539078156313\\
4.00000000000008	-3.71142284569138\\
4.00000000000006	-3.72745490981964\\
4.00000000000005	-3.7434869739479\\
4.00000000000004	-3.75951903807615\\
4.00000000000004	-3.77555110220441\\
4.00000000000003	-3.79158316633267\\
4.00000000000002	-3.80761523046092\\
4.00000000000002	-3.82364729458918\\
4.00000000000002	-3.83967935871743\\
4.00000000000001	-3.85571142284569\\
4.00000000000001	-3.87174348697395\\
4.00000000000001	-3.8877755511022\\
4.00000000000001	-3.90380761523046\\
4.00000000000001	-3.91983967935872\\
4	-3.93587174348697\\
4	-3.95190380761523\\
4	-3.96793587174349\\
4	-3.98396793587174\\
4	-4\\
4	-4\\
3.98396793587174	-4\\
3.96793587174349	-4\\
3.95190380761523	-4\\
3.93587174348697	-4\\
3.91983967935872	-4\\
3.90380761523046	-4\\
3.8877755511022	-4\\
3.87174348697395	-4\\
3.85571142284569	-4\\
3.83967935871743	-4\\
3.82364729458918	-4\\
3.80761523046092	-4\\
3.79158316633267	-4\\
3.77555110220441	-4\\
3.75951903807615	-4\\
3.7434869739479	-4\\
3.72745490981964	-4\\
3.71142284569138	-4\\
3.69539078156313	-4\\
3.67935871743487	-4\\
3.66332665330661	-4\\
3.64729458917836	-4\\
3.6312625250501	-4\\
3.61523046092184	-4\\
3.59919839679359	-4\\
3.58316633266533	-4\\
3.56713426853707	-4\\
3.55110220440882	-4\\
3.53507014028056	-4\\
3.51903807615231	-4\\
3.50300601202405	-4\\
3.48697394789579	-4\\
3.47094188376754	-4\\
3.45490981963928	-4\\
3.43887775551102	-4\\
3.42284569138277	-4\\
3.40681362725451	-4\\
3.39078156312625	-4\\
3.374749498998	-4\\
3.35871743486974	-4\\
3.34268537074148	-4\\
3.32665330661323	-4\\
3.31062124248497	-4\\
3.29458917835671	-4\\
3.27855711422846	-4\\
3.2625250501002	-4\\
3.24649298597194	-4\\
3.23046092184369	-4\\
3.21442885771543	-4\\
3.19839679358717	-4\\
3.18236472945892	-4\\
3.16633266533066	-4\\
3.1503006012024	-4\\
3.13426853707415	-4\\
3.11823647294589	-4\\
3.10220440881764	-4\\
3.08617234468938	-4\\
3.07014028056112	-4\\
3.05410821643287	-4\\
3.03807615230461	-4\\
3.02204408817635	-4\\
3.0060120240481	-4\\
2.98997995991984	-4\\
2.97394789579158	-4\\
2.95791583166333	-4\\
2.94188376753507	-4\\
2.92585170340681	-4\\
2.90981963927856	-4\\
2.8937875751503	-4\\
2.87775551102204	-4\\
2.86172344689379	-4\\
2.84569138276553	-4\\
2.82965931863727	-4\\
2.81362725450902	-4\\
2.79759519038076	-4\\
2.7815631262525	-4\\
2.76553106212425	-4\\
2.74949899799599	-4\\
2.73346693386774	-4\\
2.71743486973948	-4\\
2.70140280561122	-4\\
2.68537074148297	-4\\
2.66933867735471	-4\\
2.65330661322645	-4\\
2.6372745490982	-4\\
2.62124248496994	-4\\
2.60521042084168	-4\\
2.58917835671343	-4\\
2.57314629258517	-4\\
2.55711422845691	-4\\
2.54108216432866	-4\\
2.5250501002004	-4\\
2.50901803607214	-4\\
2.49298597194389	-4\\
2.47695390781563	-4\\
2.46092184368737	-4\\
2.44488977955912	-4\\
2.42885771543086	-4\\
2.41282565130261	-4\\
2.39679358717435	-4\\
2.38076152304609	-4\\
2.36472945891784	-4\\
2.34869739478958	-4\\
2.33266533066132	-4\\
2.31663326653307	-4\\
2.30060120240481	-4\\
2.28456913827655	-4\\
2.2685370741483	-4\\
2.25250501002004	-4\\
2.23647294589178	-4\\
2.22044088176353	-4\\
2.20440881763527	-4\\
2.18837675350701	-4\\
2.17234468937876	-4\\
2.1563126252505	-4\\
2.14028056112224	-4\\
2.12424849699399	-4\\
2.10821643286573	-4\\
2.09218436873747	-4\\
2.07615230460922	-4\\
2.06012024048096	-4\\
2.04408817635271	-4\\
2.02805611222445	-4\\
2.01202404809619	-4\\
1.99599198396794	-4\\
1.97995991983968	-4\\
1.96392785571142	-4\\
1.94789579158317	-4\\
1.93186372745491	-4\\
1.91583166332665	-4\\
1.8997995991984	-4\\
1.88376753507014	-4\\
1.86773547094188	-4\\
1.85170340681363	-4\\
1.83567134268537	-4\\
1.81963927855711	-4\\
1.80360721442886	-4\\
1.7875751503006	-4\\
1.77154308617235	-4\\
1.75551102204409	-4\\
1.73947895791583	-4\\
1.72344689378758	-4\\
1.70741482965932	-4\\
1.69138276553106	-4\\
1.67535070140281	-4\\
1.65931863727455	-4\\
1.64328657314629	-4\\
1.62725450901804	-4\\
1.61122244488978	-4\\
1.59519038076152	-4\\
1.57915831663327	-4\\
1.56312625250501	-4\\
1.54709418837675	-4\\
1.5310621242485	-4\\
1.51503006012024	-4\\
1.49899799599198	-4\\
1.48296593186373	-4\\
1.46693386773547	-4\\
1.45090180360721	-4\\
1.43486973947896	-4\\
1.4188376753507	-4\\
1.40280561122244	-4\\
1.38677354709419	-4\\
1.37074148296593	-4\\
1.35470941883768	-4\\
1.33867735470942	-4\\
1.32264529058116	-4\\
1.30661322645291	-4\\
1.29058116232465	-4\\
1.27454909819639	-4\\
1.25851703406814	-4\\
1.24248496993988	-4\\
1.22645290581162	-4\\
1.21042084168337	-4\\
1.19438877755511	-4\\
1.17835671342685	-4\\
1.1623246492986	-4\\
1.14629258517034	-4\\
1.13026052104208	-4\\
1.11422845691383	-4\\
1.09819639278557	-4\\
1.08216432865731	-4\\
1.06613226452906	-4\\
1.0501002004008	-4\\
1.03406813627254	-4\\
1.01803607214429	-4\\
1.00200400801603	-4\\
0.985971943887775	-4\\
0.969939879759519	-4\\
0.953907815631262	-4\\
0.937875751503006	-4\\
0.921843687374749	-4\\
0.905811623246493	-4\\
0.889779559118236	-4\\
0.87374749498998	-4\\
0.857715430861724	-4\\
0.841683366733467	-4\\
0.825651302605211	-4\\
0.809619238476954	-4\\
0.793587174348698	-4\\
0.777555110220441	-4\\
0.761523046092185	-4\\
0.745490981963928	-4\\
0.729458917835672	-4\\
0.713426853707415	-4\\
0.697394789579159	-4\\
0.681362725450902	-4\\
0.665330661322646	-4\\
0.649298597194389	-4\\
0.633266533066132	-4\\
0.617234468937876	-4\\
0.601202404809619	-4\\
0.585170340681363	-4\\
0.569138276553106	-4\\
0.55310621242485	-4\\
0.537074148296593	-4\\
0.521042084168337	-4\\
0.50501002004008	-4\\
0.488977955911824	-4\\
0.472945891783567	-4\\
0.456913827655311	-4\\
0.440881763527054	-4\\
0.424849699398798	-4\\
0.408817635270541	-4\\
0.392785571142285	-4\\
0.376753507014028	-4\\
0.360721442885771	-4\\
0.344689378757515	-4\\
0.328657314629258	-4\\
0.312625250501002	-4\\
0.296593186372745	-4\\
0.280561122244489	-4\\
0.264529058116232	-4\\
0.248496993987976	-4\\
0.232464929859719	-4\\
0.216432865731463	-4\\
0.200400801603206	-4\\
0.18436873747495	-4\\
0.168336673346693	-4\\
0.152304609218437	-4\\
0.13627254509018	-4\\
0.120240480961924	-4\\
0.104208416833667	-4\\
0.0881763527054105	-4\\
0.0721442885771539	-4\\
0.0561122244488974	-4\\
0.0400801603206409	-4\\
0.0240480961923843	-4\\
0.00801603206412782	-4\\
-0.00801603206412826	-4\\
-0.0240480961923848	-4\\
-0.0400801603206413	-4\\
-0.0561122244488979	-4\\
-0.0721442885771544	-4\\
-0.0881763527054109	-4\\
-0.104208416833667	-4\\
-0.120240480961924	-4\\
-0.13627254509018	-4\\
-0.152304609218437	-4\\
-0.168336673346694	-4\\
-0.18436873747495	-4\\
-0.200400801603207	-4\\
-0.216432865731463	-4\\
-0.232464929859719	-4\\
-0.248496993987976	-4\\
-0.264529058116232	-4\\
-0.280561122244489	-4\\
-0.296593186372745	-4\\
-0.312625250501002	-4\\
-0.328657314629258	-4\\
-0.344689378757515	-4\\
-0.360721442885771	-4\\
-0.376753507014028	-4\\
-0.392785571142285	-4\\
-0.408817635270541	-4\\
-0.424849699398798	-4\\
-0.440881763527054	-4\\
-0.456913827655311	-4\\
-0.472945891783567	-4\\
-0.488977955911824	-4\\
-0.50501002004008	-4\\
-0.521042084168337	-4\\
-0.537074148296593	-4\\
-0.55310621242485	-4\\
-0.569138276553106	-4\\
-0.585170340681363	-4\\
-0.601202404809619	-4\\
-0.617234468937876	-4\\
-0.633266533066132	-4\\
-0.649298597194389	-4\\
-0.665330661322646	-4\\
-0.681362725450902	-4\\
-0.697394789579158	-4\\
-0.713426853707415	-4\\
-0.729458917835671	-4\\
-0.745490981963928	-4\\
-0.761523046092184	-4\\
-0.777555110220441	-4\\
-0.793587174348697	-4\\
-0.809619238476954	-4\\
-0.82565130260521	-4\\
-0.841683366733467	-4\\
-0.857715430861723	-4\\
-0.87374749498998	-4\\
-0.889779559118236	-4\\
-0.905811623246493	-4\\
-0.92184368737475	-4\\
-0.937875751503006	-4\\
-0.953907815631263	-4\\
-0.969939879759519	-4\\
-0.985971943887776	-4\\
-1.00200400801603	-4\\
-1.01803607214429	-4\\
-1.03406813627255	-4\\
-1.0501002004008	-4\\
-1.06613226452906	-4\\
-1.08216432865731	-4\\
-1.09819639278557	-4\\
-1.11422845691383	-4\\
-1.13026052104208	-4\\
-1.14629258517034	-4\\
-1.1623246492986	-4\\
-1.17835671342685	-4\\
-1.19438877755511	-4\\
-1.21042084168337	-4\\
-1.22645290581162	-4\\
-1.24248496993988	-4\\
-1.25851703406814	-4\\
-1.27454909819639	-4\\
-1.29058116232465	-4\\
-1.30661322645291	-4\\
-1.32264529058116	-4\\
-1.33867735470942	-4\\
-1.35470941883768	-4\\
-1.37074148296593	-4\\
-1.38677354709419	-4\\
-1.40280561122244	-4\\
-1.4188376753507	-4\\
-1.43486973947896	-4\\
-1.45090180360721	-4\\
-1.46693386773547	-4\\
-1.48296593186373	-4\\
-1.49899799599198	-4\\
-1.51503006012024	-4\\
-1.5310621242485	-4\\
-1.54709418837675	-4\\
-1.56312625250501	-4\\
-1.57915831663327	-4\\
-1.59519038076152	-4\\
-1.61122244488978	-4\\
-1.62725450901804	-4\\
-1.64328657314629	-4\\
-1.65931863727455	-4\\
-1.67535070140281	-4\\
-1.69138276553106	-4\\
-1.70741482965932	-4\\
-1.72344689378758	-4\\
-1.73947895791583	-4\\
-1.75551102204409	-4\\
-1.77154308617234	-4\\
-1.7875751503006	-4\\
-1.80360721442886	-4\\
-1.81963927855711	-4\\
-1.83567134268537	-4\\
-1.85170340681363	-4\\
-1.86773547094188	-4\\
-1.88376753507014	-4\\
-1.8997995991984	-4\\
-1.91583166332665	-4\\
-1.93186372745491	-4\\
-1.94789579158317	-4\\
-1.96392785571142	-4\\
-1.97995991983968	-4\\
-1.99599198396794	-4\\
-2.01202404809619	-4\\
-2.02805611222445	-4\\
-2.04408817635271	-4\\
-2.06012024048096	-4\\
-2.07615230460922	-4\\
-2.09218436873747	-4\\
-2.10821643286573	-4\\
-2.12424849699399	-4\\
-2.14028056112224	-4\\
-2.1563126252505	-4\\
-2.17234468937876	-4\\
-2.18837675350701	-4\\
-2.20440881763527	-4\\
-2.22044088176353	-4\\
-2.23647294589178	-4\\
-2.25250501002004	-4\\
-2.2685370741483	-4\\
-2.28456913827655	-4\\
-2.30060120240481	-4\\
-2.31663326653307	-4\\
-2.33266533066132	-4\\
-2.34869739478958	-4\\
-2.36472945891784	-4\\
-2.38076152304609	-4\\
-2.39679358717435	-4\\
-2.41282565130261	-4\\
-2.42885771543086	-4\\
-2.44488977955912	-4\\
-2.46092184368737	-4\\
-2.47695390781563	-4\\
-2.49298597194389	-4\\
-2.50901803607214	-4\\
-2.5250501002004	-4\\
-2.54108216432866	-4\\
-2.55711422845691	-4\\
-2.57314629258517	-4\\
-2.58917835671343	-4\\
-2.60521042084168	-4\\
-2.62124248496994	-4\\
-2.6372745490982	-4\\
-2.65330661322645	-4\\
-2.66933867735471	-4\\
-2.68537074148297	-4\\
-2.70140280561122	-4\\
-2.71743486973948	-4\\
-2.73346693386774	-4\\
-2.74949899799599	-4\\
-2.76553106212425	-4\\
-2.7815631262525	-4\\
-2.79759519038076	-4\\
-2.81362725450902	-4\\
-2.82965931863727	-4\\
-2.84569138276553	-4\\
-2.86172344689379	-4\\
-2.87775551102204	-4\\
-2.8937875751503	-4\\
-2.90981963927856	-4\\
-2.92585170340681	-4\\
-2.94188376753507	-4\\
-2.95791583166333	-4\\
-2.97394789579158	-4\\
-2.98997995991984	-4\\
-3.0060120240481	-4\\
-3.02204408817635	-4\\
-3.03807615230461	-4\\
-3.05410821643287	-4\\
-3.07014028056112	-4\\
-3.08617234468938	-4\\
-3.10220440881764	-4\\
-3.11823647294589	-4\\
-3.13426853707415	-4\\
-3.1503006012024	-4\\
-3.16633266533066	-4\\
-3.18236472945892	-4\\
-3.19839679358717	-4\\
-3.21442885771543	-4\\
-3.23046092184369	-4\\
-3.24649298597194	-4\\
-3.2625250501002	-4\\
-3.27855711422846	-4\\
-3.29458917835671	-4\\
-3.31062124248497	-4\\
-3.32665330661323	-4\\
-3.34268537074148	-4\\
-3.35871743486974	-4\\
-3.374749498998	-4\\
-3.39078156312625	-4\\
-3.40681362725451	-4\\
-3.42284569138277	-4\\
-3.43887775551102	-4\\
-3.45490981963928	-4\\
-3.47094188376753	-4\\
-3.48697394789579	-4\\
-3.50300601202405	-4\\
-3.5190380761523	-4\\
-3.53507014028056	-4\\
-3.55110220440882	-4\\
-3.56713426853707	-4\\
-3.58316633266533	-4\\
-3.59919839679359	-4\\
-3.61523046092184	-4\\
-3.6312625250501	-4\\
-3.64729458917836	-4\\
-3.66332665330661	-4\\
-3.67935871743487	-4\\
-3.69539078156313	-4\\
-3.71142284569138	-4\\
-3.72745490981964	-4\\
-3.7434869739479	-4\\
-3.75951903807615	-4\\
-3.77555110220441	-4\\
-3.79158316633267	-4\\
-3.80761523046092	-4\\
-3.82364729458918	-4\\
-3.83967935871743	-4\\
-3.85571142284569	-4\\
-3.87174348697395	-4\\
-3.8877755511022	-4\\
-3.90380761523046	-4\\
-3.91983967935872	-4\\
-3.93587174348697	-4\\
-3.95190380761523	-4\\
-3.96793587174349	-4\\
-3.98396793587174	-4\\
-4	-4\\
-4	-3.98396793587174\\
-4	-3.96793587174349\\
-4	-3.95190380761523\\
-4	-3.93587174348697\\
-4	-3.91983967935872\\
-4	-3.90380761523046\\
-4	-3.8877755511022\\
-4	-3.87174348697395\\
-4	-3.85571142284569\\
-4	-3.83967935871743\\
-4	-3.82364729458918\\
-4	-3.80761523046092\\
-4	-3.79158316633267\\
-4	-3.77555110220441\\
-4	-3.75951903807615\\
-4	-3.7434869739479\\
-4	-3.72745490981964\\
-4	-3.71142284569138\\
-4	-3.69539078156313\\
-4	-3.67935871743487\\
-4	-3.66332665330661\\
-4	-3.64729458917836\\
-4	-3.6312625250501\\
-4	-3.61523046092184\\
-4	-3.59919839679359\\
-4	-3.58316633266533\\
-4	-3.56713426853707\\
-4	-3.55110220440882\\
-4	-3.53507014028056\\
-4	-3.5190380761523\\
-4	-3.50300601202405\\
-4	-3.48697394789579\\
-4	-3.47094188376753\\
-4	-3.45490981963928\\
-4	-3.43887775551102\\
-4	-3.42284569138277\\
-4	-3.40681362725451\\
-4	-3.39078156312625\\
-4	-3.374749498998\\
-4	-3.35871743486974\\
-4	-3.34268537074148\\
-4	-3.32665330661323\\
-4	-3.31062124248497\\
-4	-3.29458917835671\\
-4	-3.27855711422846\\
-4	-3.2625250501002\\
-4	-3.24649298597194\\
-4	-3.23046092184369\\
-4	-3.21442885771543\\
-4	-3.19839679358717\\
-4	-3.18236472945892\\
-4	-3.16633266533066\\
-4	-3.1503006012024\\
-4	-3.13426853707415\\
-4	-3.11823647294589\\
-4	-3.10220440881764\\
-4	-3.08617234468938\\
-4	-3.07014028056112\\
-4	-3.05410821643287\\
-4	-3.03807615230461\\
-4	-3.02204408817635\\
-4	-3.0060120240481\\
-4	-2.98997995991984\\
-4	-2.97394789579158\\
-4	-2.95791583166333\\
-4	-2.94188376753507\\
-4	-2.92585170340681\\
-4	-2.90981963927856\\
-4	-2.8937875751503\\
-4	-2.87775551102204\\
-4	-2.86172344689379\\
-4	-2.84569138276553\\
-4	-2.82965931863727\\
-4	-2.81362725450902\\
-4	-2.79759519038076\\
-4	-2.7815631262525\\
-4	-2.76553106212425\\
-4	-2.74949899799599\\
-4	-2.73346693386774\\
-4	-2.71743486973948\\
-4	-2.70140280561122\\
-4	-2.68537074148297\\
-4	-2.66933867735471\\
-4	-2.65330661322645\\
-4	-2.6372745490982\\
-4	-2.62124248496994\\
-4	-2.60521042084168\\
-4	-2.58917835671343\\
-4	-2.57314629258517\\
-4	-2.55711422845691\\
-4	-2.54108216432866\\
-4	-2.5250501002004\\
-4	-2.50901803607214\\
-4	-2.49298597194389\\
-4	-2.47695390781563\\
-4	-2.46092184368737\\
-4	-2.44488977955912\\
-4	-2.42885771543086\\
-4	-2.41282565130261\\
-4	-2.39679358717435\\
-4	-2.38076152304609\\
-4	-2.36472945891784\\
-4	-2.34869739478958\\
-4	-2.33266533066132\\
-4	-2.31663326653307\\
-4	-2.30060120240481\\
-4	-2.28456913827655\\
-4	-2.2685370741483\\
-4.00000000000001	-2.25250501002004\\
-4.00000000000001	-2.23647294589178\\
-4.00000000000001	-2.22044088176353\\
-4.00000000000001	-2.20440881763527\\
-4.00000000000001	-2.18837675350701\\
-4.00000000000001	-2.17234468937876\\
-4.00000000000001	-2.1563126252505\\
-4.00000000000002	-2.14028056112224\\
-4.00000000000002	-2.12424849699399\\
-4.00000000000002	-2.10821643286573\\
-4.00000000000003	-2.09218436873747\\
-4.00000000000003	-2.07615230460922\\
-4.00000000000003	-2.06012024048096\\
-4.00000000000004	-2.04408817635271\\
-4.00000000000005	-2.02805611222445\\
-4.00000000000005	-2.01202404809619\\
-4.00000000000006	-1.99599198396794\\
-4.00000000000007	-1.97995991983968\\
-4.00000000000008	-1.96392785571142\\
-4.0000000000001	-1.94789579158317\\
-4.00000000000011	-1.93186372745491\\
-4.00000000000013	-1.91583166332665\\
-4.00000000000015	-1.8997995991984\\
-4.00000000000017	-1.88376753507014\\
-4.00000000000019	-1.86773547094188\\
-4.00000000000022	-1.85170340681363\\
-4.00000000000025	-1.83567134268537\\
-4.00000000000029	-1.81963927855711\\
-4.00000000000033	-1.80360721442886\\
-4.00000000000037	-1.7875751503006\\
-4.00000000000042	-1.77154308617234\\
-4.00000000000048	-1.75551102204409\\
-4.00000000000054	-1.73947895791583\\
-4.00000000000061	-1.72344689378758\\
-4.00000000000069	-1.70741482965932\\
-4.00000000000078	-1.69138276553106\\
-4.00000000000088	-1.67535070140281\\
-4.00000000000099	-1.65931863727455\\
-4.00000000000112	-1.64328657314629\\
-4.00000000000125	-1.62725450901804\\
-4.0000000000014	-1.61122244488978\\
-4.00000000000157	-1.59519038076152\\
-4.00000000000176	-1.57915831663327\\
-4.00000000000196	-1.56312625250501\\
-4.00000000000219	-1.54709418837675\\
-4.00000000000243	-1.5310621242485\\
-4.0000000000027	-1.51503006012024\\
-4.000000000003	-1.49899799599198\\
-4.00000000000333	-1.48296593186373\\
-4.00000000000369	-1.46693386773547\\
-4.00000000000408	-1.45090180360721\\
-4.0000000000045	-1.43486973947896\\
-4.00000000000496	-1.4188376753507\\
-4.00000000000546	-1.40280561122244\\
-4.00000000000601	-1.38677354709419\\
-4.0000000000066	-1.37074148296593\\
-4.00000000000724	-1.35470941883768\\
-4.00000000000793	-1.33867735470942\\
-4.00000000000867	-1.32264529058116\\
-4.00000000000947	-1.30661322645291\\
-4.00000000001033	-1.29058116232465\\
-4.00000000001126	-1.27454909819639\\
-4.00000000001225	-1.25851703406814\\
-4.00000000001331	-1.24248496993988\\
-4.00000000001444	-1.22645290581162\\
-4.00000000001564	-1.21042084168337\\
-4.00000000001693	-1.19438877755511\\
-4.00000000001829	-1.17835671342685\\
-4.00000000001974	-1.1623246492986\\
-4.00000000002127	-1.14629258517034\\
-4.00000000002289	-1.13026052104208\\
-4.00000000002461	-1.11422845691383\\
-4.00000000002641	-1.09819639278557\\
-4.00000000002831	-1.08216432865731\\
-4.00000000003031	-1.06613226452906\\
-4.0000000000324	-1.0501002004008\\
-4.00000000003459	-1.03406813627255\\
-4.00000000003687	-1.01803607214429\\
-4.00000000003926	-1.00200400801603\\
-4.00000000004174	-0.985971943887776\\
-4.00000000004433	-0.969939879759519\\
-4.000000000047	-0.953907815631263\\
-4.00000000004977	-0.937875751503006\\
-4.00000000005264	-0.92184368737475\\
-4.00000000005559	-0.905811623246493\\
-4.00000000005863	-0.889779559118236\\
-4.00000000006176	-0.87374749498998\\
-4.00000000006496	-0.857715430861723\\
-4.00000000006824	-0.841683366733467\\
-4.00000000007158	-0.82565130260521\\
-4.00000000007499	-0.809619238476954\\
-4.00000000007846	-0.793587174348697\\
-4.00000000008197	-0.777555110220441\\
-4.00000000008553	-0.761523046092184\\
-4.00000000008911	-0.745490981963928\\
-4.00000000009273	-0.729458917835671\\
-4.00000000009636	-0.713426853707415\\
-4.0000000001	-0.697394789579158\\
-4.00000000010364	-0.681362725450902\\
-4.00000000010726	-0.665330661322646\\
-4.00000000011086	-0.649298597194389\\
-4.00000000011443	-0.633266533066132\\
-4.00000000011796	-0.617234468937876\\
-4.00000000012142	-0.601202404809619\\
-4.00000000012483	-0.585170340681363\\
-4.00000000012815	-0.569138276553106\\
-4.00000000013139	-0.55310621242485\\
-4.00000000013452	-0.537074148296593\\
-4.00000000013755	-0.521042084168337\\
-4.00000000014045	-0.50501002004008\\
-4.00000000014322	-0.488977955911824\\
-4.00000000014585	-0.472945891783567\\
-4.00000000014833	-0.456913827655311\\
-4.00000000015064	-0.440881763527054\\
-4.00000000015279	-0.424849699398798\\
-4.00000000015476	-0.408817635270541\\
-4.00000000015654	-0.392785571142285\\
-4.00000000015813	-0.376753507014028\\
-4.00000000015952	-0.360721442885771\\
-4.0000000001607	-0.344689378757515\\
-4.00000000016168	-0.328657314629258\\
-4.00000000016244	-0.312625250501002\\
-4.00000000016299	-0.296593186372745\\
-4.00000000016332	-0.280561122244489\\
-4.00000000016342	-0.264529058116232\\
-4.00000000016331	-0.248496993987976\\
-4.00000000016298	-0.232464929859719\\
-4.00000000016243	-0.216432865731463\\
-4.00000000016166	-0.200400801603207\\
-4.00000000016068	-0.18436873747495\\
-4.00000000015949	-0.168336673346694\\
-4.0000000001581	-0.152304609218437\\
-4.0000000001565	-0.13627254509018\\
-4.00000000015472	-0.120240480961924\\
-4.00000000015275	-0.104208416833667\\
-4.0000000001506	-0.0881763527054109\\
-4.00000000014828	-0.0721442885771544\\
-4.00000000014579	-0.0561122244488979\\
-4.00000000014316	-0.0400801603206413\\
-4.00000000014039	-0.0240480961923848\\
-4.00000000013748	-0.00801603206412826\\
-4.00000000013445	0.00801603206412782\\
-4.00000000013132	0.0240480961923843\\
-4.00000000012808	0.0400801603206409\\
-4.00000000012475	0.0561122244488974\\
-4.00000000012135	0.0721442885771539\\
-4.00000000011788	0.0881763527054105\\
-4.00000000011435	0.104208416833667\\
-4.00000000011079	0.120240480961924\\
-4.00000000010718	0.13627254509018\\
-4.00000000010356	0.152304609218437\\
-4.00000000009992	0.168336673346693\\
-4.00000000009628	0.18436873747495\\
-4.00000000009265	0.200400801603206\\
-4.00000000008904	0.216432865731463\\
-4.00000000008545	0.232464929859719\\
-4.00000000008189	0.248496993987976\\
-4.00000000007838	0.264529058116232\\
-4.00000000007492	0.280561122244489\\
-4.00000000007151	0.296593186372745\\
-4.00000000006816	0.312625250501002\\
-4.00000000006489	0.328657314629258\\
-4.00000000006169	0.344689378757515\\
-4.00000000005857	0.360721442885771\\
-4.00000000005553	0.376753507014028\\
-4.00000000005257	0.392785571142285\\
-4.00000000004971	0.408817635270541\\
-4.00000000004694	0.424849699398798\\
-4.00000000004427	0.440881763527054\\
-4.00000000004169	0.456913827655311\\
-4.00000000003921	0.472945891783567\\
-4.00000000003682	0.488977955911824\\
-4.00000000003454	0.50501002004008\\
-4.00000000003235	0.521042084168337\\
-4.00000000003026	0.537074148296593\\
-4.00000000002827	0.55310621242485\\
-4.00000000002637	0.569138276553106\\
-4.00000000002457	0.585170340681363\\
-4.00000000002286	0.601202404809619\\
-4.00000000002124	0.617234468937876\\
-4.00000000001971	0.633266533066132\\
-4.00000000001826	0.649298597194389\\
-4.0000000000169	0.665330661322646\\
-4.00000000001562	0.681362725450902\\
-4.00000000001441	0.697394789579159\\
-4.00000000001328	0.713426853707415\\
-4.00000000001222	0.729458917835672\\
-4.00000000001124	0.745490981963928\\
-4.00000000001031	0.761523046092185\\
-4.00000000000945	0.777555110220441\\
-4.00000000000865	0.793587174348698\\
-4.00000000000791	0.809619238476954\\
-4.00000000000722	0.825651302605211\\
-4.00000000000659	0.841683366733467\\
-4.000000000006	0.857715430861724\\
-4.00000000000545	0.87374749498998\\
-4.00000000000495	0.889779559118236\\
-4.00000000000449	0.905811623246493\\
-4.00000000000407	0.921843687374749\\
-4.00000000000368	0.937875751503006\\
-4.00000000000332	0.953907815631262\\
-4.000000000003	0.969939879759519\\
-4.0000000000027	0.985971943887775\\
-4.00000000000243	1.00200400801603\\
-4.00000000000218	1.01803607214429\\
-4.00000000000196	1.03406813627254\\
-4.00000000000175	1.0501002004008\\
-4.00000000000157	1.06613226452906\\
-4.0000000000014	1.08216432865731\\
-4.00000000000125	1.09819639278557\\
-4.00000000000111	1.11422845691383\\
-4.00000000000099	1.13026052104208\\
-4.00000000000088	1.14629258517034\\
-4.00000000000078	1.1623246492986\\
-4.00000000000069	1.17835671342685\\
-4.00000000000061	1.19438877755511\\
-4.00000000000054	1.21042084168337\\
-4.00000000000048	1.22645290581162\\
-4.00000000000042	1.24248496993988\\
-4.00000000000037	1.25851703406814\\
-4.00000000000033	1.27454909819639\\
-4.00000000000029	1.29058116232465\\
-4.00000000000025	1.30661322645291\\
-4.00000000000022	1.32264529058116\\
-4.00000000000019	1.33867735470942\\
-4.00000000000017	1.35470941883768\\
-4.00000000000015	1.37074148296593\\
-4.00000000000013	1.38677354709419\\
-4.00000000000011	1.40280561122244\\
-4.0000000000001	1.4188376753507\\
-4.00000000000008	1.43486973947896\\
-4.00000000000007	1.45090180360721\\
-4.00000000000006	1.46693386773547\\
-4.00000000000005	1.48296593186373\\
-4.00000000000005	1.49899799599198\\
-4.00000000000004	1.51503006012024\\
-4.00000000000003	1.5310621242485\\
-4.00000000000003	1.54709418837675\\
-4.00000000000003	1.56312625250501\\
-4.00000000000002	1.57915831663327\\
-4.00000000000002	1.59519038076152\\
-4.00000000000002	1.61122244488978\\
-4.00000000000001	1.62725450901804\\
-4.00000000000001	1.64328657314629\\
-4.00000000000001	1.65931863727455\\
-4.00000000000001	1.67535070140281\\
-4.00000000000001	1.69138276553106\\
-4.00000000000001	1.70741482965932\\
-4.00000000000001	1.72344689378758\\
-4	1.73947895791583\\
-4	1.75551102204409\\
-4	1.77154308617235\\
-4	1.7875751503006\\
-4	1.80360721442886\\
-4	1.81963927855711\\
-4	1.83567134268537\\
-4	1.85170340681363\\
-4	1.86773547094188\\
-4	1.88376753507014\\
-4	1.8997995991984\\
-4	1.91583166332665\\
-4	1.93186372745491\\
-4	1.94789579158317\\
-4	1.96392785571142\\
-4	1.97995991983968\\
-4	1.99599198396794\\
-4	2.01202404809619\\
-4	2.02805611222445\\
-4	2.04408817635271\\
-4	2.06012024048096\\
-4	2.07615230460922\\
-4	2.09218436873747\\
-4	2.10821643286573\\
-4	2.12424849699399\\
-4	2.14028056112224\\
-4	2.1563126252505\\
-4	2.17234468937876\\
-4	2.18837675350701\\
-4	2.20440881763527\\
-4	2.22044088176353\\
-4	2.23647294589178\\
-4	2.25250501002004\\
-4	2.2685370741483\\
-4	2.28456913827655\\
-4	2.30060120240481\\
-4	2.31663326653307\\
-4	2.33266533066132\\
-4	2.34869739478958\\
-4	2.36472945891784\\
-4	2.38076152304609\\
-4	2.39679358717435\\
-4	2.41282565130261\\
-4	2.42885771543086\\
-4	2.44488977955912\\
-4	2.46092184368737\\
-4	2.47695390781563\\
-4	2.49298597194389\\
-4	2.50901803607214\\
-4	2.5250501002004\\
-4	2.54108216432866\\
-4	2.55711422845691\\
-4	2.57314629258517\\
-4	2.58917835671343\\
-4	2.60521042084168\\
-4	2.62124248496994\\
-4	2.6372745490982\\
-4	2.65330661322645\\
-4	2.66933867735471\\
-4	2.68537074148297\\
-4	2.70140280561122\\
-4	2.71743486973948\\
-4	2.73346693386774\\
-4	2.74949899799599\\
-4	2.76553106212425\\
-4	2.7815631262525\\
-4	2.79759519038076\\
-4	2.81362725450902\\
-4	2.82965931863727\\
-4	2.84569138276553\\
-4	2.86172344689379\\
-4	2.87775551102204\\
-4	2.8937875751503\\
-4	2.90981963927856\\
-4	2.92585170340681\\
-4	2.94188376753507\\
-4	2.95791583166333\\
-4	2.97394789579158\\
-4	2.98997995991984\\
-4	3.0060120240481\\
-4	3.02204408817635\\
-4	3.03807615230461\\
-4	3.05410821643287\\
-4	3.07014028056112\\
-4	3.08617234468938\\
-4	3.10220440881764\\
-4	3.11823647294589\\
-4	3.13426853707415\\
-4	3.1503006012024\\
-4	3.16633266533066\\
-4	3.18236472945892\\
-4	3.19839679358717\\
-4	3.21442885771543\\
-4	3.23046092184369\\
-4	3.24649298597194\\
-4	3.2625250501002\\
-4	3.27855711422846\\
-4	3.29458917835671\\
-4	3.31062124248497\\
-4	3.32665330661323\\
-4	3.34268537074148\\
-4	3.35871743486974\\
-4	3.374749498998\\
-4	3.39078156312625\\
-4	3.40681362725451\\
-4	3.42284569138277\\
-4	3.43887775551102\\
-4	3.45490981963928\\
-4	3.47094188376754\\
-4	3.48697394789579\\
-4	3.50300601202405\\
-4	3.51903807615231\\
-4	3.53507014028056\\
-4	3.55110220440882\\
-4	3.56713426853707\\
-4	3.58316633266533\\
-4	3.59919839679359\\
-4	3.61523046092184\\
-4	3.6312625250501\\
-4	3.64729458917836\\
-4	3.66332665330661\\
-4	3.67935871743487\\
-4	3.69539078156313\\
-4	3.71142284569138\\
-4	3.72745490981964\\
-4	3.7434869739479\\
-4	3.75951903807615\\
-4	3.77555110220441\\
-4	3.79158316633267\\
-4	3.80761523046092\\
-4	3.82364729458918\\
-4	3.83967935871743\\
-4	3.85571142284569\\
-4	3.87174348697395\\
-4	3.8877755511022\\
-4	3.90380761523046\\
-4	3.91983967935872\\
-4	3.93587174348697\\
-4	3.95190380761523\\
-4	3.96793587174349\\
-4	3.98396793587174\\
-4	4\\
}--cycle;


\addplot[area legend,solid,fill=mycolor2,draw=black,forget plot]
table[row sep=crcr] {%
x	y\\
3.23046092184369	-1.28961521088211\\
3.23059995598456	-1.29058116232465\\
3.23210609102459	-1.30661322645291\\
3.23280676744868	-1.32264529058116\\
3.2326867994279	-1.33867735470942\\
3.23172714709587	-1.35470941883768\\
3.23046092184369	-1.36588428566277\\
3.22991430377184	-1.37074148296593\\
3.22724763113916	-1.38677354709419\\
3.22367328893673	-1.40280561122244\\
3.21915487183931	-1.4188376753507\\
3.21442885771543	-1.43263578866559\\
3.21366338217958	-1.43486973947896\\
3.20722911766733	-1.45090180360721\\
3.19972394887254	-1.46693386773547\\
3.19839679358717	-1.46944198485306\\
3.19119729409304	-1.48296593186373\\
3.18236472945892	-1.49761155231003\\
3.18152160526817	-1.49899799599198\\
3.17073281643393	-1.51503006012024\\
3.16633266533066	-1.52095165894214\\
3.15872696621743	-1.5310621242485\\
3.1503006012024	-1.5412896994202\\
3.14544806715447	-1.54709418837675\\
3.13426853707415	-1.55941216212146\\
3.13084025278646	-1.56312625250501\\
3.11823647294589	-1.57580219565082\\
3.11483468649029	-1.57915831663327\\
3.10220440881764	-1.59080719486309\\
3.09734836585729	-1.59519038076152\\
3.08617234468938	-1.60468094524144\\
3.07828269514471	-1.61122244488978\\
3.07014028056112	-1.61761092731766\\
3.05752179244145	-1.62725450901804\\
3.05410821643287	-1.6297365139743\\
3.03807615230461	-1.6411232825282\\
3.03495787438418	-1.64328657314629\\
3.02204408817635	-1.65185959076075\\
3.01044618832334	-1.65931863727455\\
3.0060120240481	-1.66206035277174\\
2.98997995991984	-1.67173289907098\\
2.98381459788665	-1.67535070140281\\
2.97394789579158	-1.6809455423404\\
2.95791583166333	-1.68974395662669\\
2.95486069021482	-1.69138276553106\\
2.94188376753507	-1.69814202290229\\
2.92585170340681	-1.70619029099022\\
2.92335820508253	-1.70741482965932\\
2.90981963927856	-1.71390062121684\\
2.8937875751503	-1.72130107948628\\
2.88902037386654	-1.72344689378758\\
2.87775551102204	-1.72841497244893\\
2.86172344689379	-1.73525273192731\\
2.85148803287348	-1.73947895791583\\
2.84569138276553	-1.74183394014393\\
2.82965931863727	-1.74818025670303\\
2.81362725450902	-1.75428102956184\\
2.8103180354147	-1.75551102204409\\
2.79759519038076	-1.76018563550569\\
2.7815631262525	-1.76587479430612\\
2.76553106212425	-1.77134635983112\\
2.76494260229327	-1.77154308617234\\
2.74949899799599	-1.77666966306709\\
2.73346693386774	-1.78179405851302\\
2.71743486973948	-1.78672426022703\\
2.71460101850803	-1.7875751503006\\
2.70140280561122	-1.79152726867831\\
2.68537074148297	-1.7961625168226\\
2.66933867735471	-1.80062325128438\\
2.65825094783178	-1.80360721442886\\
2.65330661322645	-1.80493983626817\\
2.6372745490982	-1.80914799668318\\
2.62124248496994	-1.81319825436862\\
2.60521042084168	-1.8170957750681\\
2.59439102974932	-1.81963927855711\\
2.58917835671343	-1.82087225675428\\
2.57314629258517	-1.82456045509625\\
2.55711422845691	-1.82810935487827\\
2.54108216432866	-1.83152333179034\\
2.5250501002004	-1.83480659401005\\
2.52070358707822	-1.83567134268537\\
2.50901803607214	-1.83802238249737\\
2.49298597194389	-1.84113618965292\\
2.47695390781563	-1.84412967006196\\
2.46092184368737	-1.84700640628282\\
2.44488977955912	-1.84976984479635\\
2.43324811505561	-1.85170340681363\\
2.42885771543086	-1.8524444636601\\
2.41282565130261	-1.85506696553866\\
2.39679358717435	-1.85758414842917\\
2.38076152304609	-1.85999895633512\\
2.36472945891784	-1.86231422134456\\
2.34869739478958	-1.86453266769829\\
2.33266533066132	-1.86665691568166\\
2.32420181961595	-1.86773547094188\\
2.31663326653307	-1.86872140988157\\
2.30060120240481	-1.87073059657625\\
2.28456913827655	-1.87265102633414\\
2.2685370741483	-1.8744849376924\\
2.25250501002004	-1.87623448218679\\
2.23647294589178	-1.87790172729458\\
2.22044088176353	-1.87948865924891\\
2.20440881763527	-1.88099718572993\\
2.18837675350701	-1.88242913843758\\
2.17256795610032	-1.88376753507014\\
2.17234468937876	-1.88378698766152\\
2.1563126252505	-1.88512015550221\\
2.14028056112224	-1.88637978112544\\
2.12424849699399	-1.88756741389019\\
2.10821643286573	-1.88868453784598\\
2.09218436873747	-1.88973257366453\\
2.07615230460922	-1.89071288048256\\
2.06012024048096	-1.89162675765876\\
2.04408817635271	-1.89247544644824\\
2.02805611222445	-1.89326013159737\\
2.01202404809619	-1.89398194286175\\
1.99599198396794	-1.89464195645007\\
1.97995991983968	-1.89524119639628\\
1.96392785571142	-1.8957806358626\\
1.94789579158317	-1.89626119837546\\
1.93186372745491	-1.89668375899667\\
1.91583166332665	-1.89704914543169\\
1.8997995991984	-1.89735813907712\\
1.88376753507014	-1.89761147600888\\
1.86773547094188	-1.89780984791316\\
1.85170340681363	-1.89795390296138\\
1.83567134268537	-1.89804424663092\\
1.81963927855711	-1.89808144247283\\
1.80360721442886	-1.89806601282793\\
1.7875751503006	-1.8979984394925\\
1.77154308617235	-1.89787916433461\\
1.75551102204409	-1.8977085898623\\
1.73947895791583	-1.89748707974438\\
1.72344689378758	-1.89721495928486\\
1.70741482965932	-1.89689251585185\\
1.69138276553106	-1.89651999926146\\
1.67535070140281	-1.89609762211765\\
1.65931863727455	-1.89562556010829\\
1.64328657314629	-1.89510395225829\\
1.62725450901804	-1.89453290113989\\
1.61122244488978	-1.89391247304077\\
1.59519038076152	-1.89324269809012\\
1.57915831663327	-1.89252357034295\\
1.56312625250501	-1.89175504782283\\
1.54709418837675	-1.89093705252312\\
1.5310621242485	-1.89006947036679\\
1.51503006012024	-1.88915215112467\\
1.49899799599198	-1.88818490829224\\
1.48296593186373	-1.88716751892457\\
1.46693386773547	-1.88609972342932\\
1.45090180360721	-1.88498122531747\\
1.43486973947896	-1.88381169091138\\
1.43428860309123	-1.88376753507014\\
1.4188376753507	-1.88264775228456\\
1.40280561122244	-1.88143723806231\\
1.38677354709419	-1.8801776306824\\
1.37074148296593	-1.87886850424023\\
1.35470941883768	-1.87750939583729\\
1.33867735470942	-1.87609980509628\\
1.32264529058116	-1.87463919363194\\
1.30661322645291	-1.87312698447686\\
1.29058116232465	-1.87156256146129\\
1.27454909819639	-1.86994526854605\\
1.25851703406814	-1.86827440910744\\
1.25349274501678	-1.86773547094188\\
1.24248496993988	-1.86660790713538\\
1.22645290581162	-1.86491640910667\\
1.21042084168337	-1.8631726485267\\
1.19438877755511	-1.86137584009584\\
1.17835671342685	-1.85952515658903\\
1.1623246492986	-1.85761972793021\\
1.14629258517034	-1.85565864021309\\
1.13026052104208	-1.85364093466694\\
1.11528685895472	-1.85170340681363\\
1.11422845691383	-1.8515724608641\\
1.09819639278557	-1.84954515019404\\
1.08216432865731	-1.84746208877715\\
1.06613226452906	-1.84532222555333\\
1.0501002004008	-1.84312446211694\\
1.03406813627254	-1.84086765144259\\
1.01803607214429	-1.83855059654719\\
1.00200400801603	-1.8361720490861\\
0.998695675675477	-1.83567134268537\\
0.985971943887775	-1.83382779500692\\
0.969939879759519	-1.83144871482859\\
0.953907815631262	-1.82900858218361\\
0.937875751503006	-1.82650604676602\\
0.921843687374749	-1.8239397031366\\
0.905811623246493	-1.82130808903139\\
0.895863331779166	-1.81963927855711\\
0.889779559118236	-1.81866099529851\\
0.87374749498998	-1.81603301437109\\
0.857715430861724	-1.8133398638981\\
0.841683366733467	-1.81057996946897\\
0.825651302605211	-1.80775169504553\\
0.809619238476954	-1.80485334093889\\
0.802854279498902	-1.80360721442886\\
0.793587174348698	-1.80196873296383\\
0.777555110220441	-1.79907711189586\\
0.761523046092185	-1.79611515499217\\
0.745490981963928	-1.79308103646781\\
0.729458917835672	-1.78997286089763\\
0.717354580316501	-1.7875751503006\\
0.713426853707415	-1.78682736555778\\
0.697394789579159	-1.78372661570974\\
0.681362725450902	-1.78055123172838\\
0.665330661322646	-1.77729917849036\\
0.649298597194389	-1.77396834418843\\
0.637863890518416	-1.77154308617234\\
0.633266533066132	-1.77060465955592\\
0.617234468937876	-1.76728117018617\\
0.601202404809619	-1.76387797628382\\
0.585170340681363	-1.76039280973233\\
0.569138276553106	-1.75682331748332\\
0.563337640807776	-1.75551102204409\\
0.55310621242485	-1.7532803726554\\
0.537074148296593	-1.74971735872807\\
0.521042084168337	-1.74606871502868\\
0.50501002004008	-1.74233191216145\\
0.493020160977245	-1.73947895791583\\
0.488977955911824	-1.7385507808087\\
0.472945891783567	-1.73481777285915\\
0.456913827655311	-1.73099497547316\\
0.440881763527054	-1.72707966623084\\
0.426341328300433	-1.72344689378758\\
0.424849699398798	-1.72308677391813\\
0.408817635270541	-1.7191720002689\\
0.392785571142285	-1.71516273084878\\
0.376753507014028	-1.71105603114855\\
0.362851204538666	-1.70741482965932\\
0.360721442885771	-1.70687505909521\\
0.344689378757515	-1.70276502078737\\
0.328657314629258	-1.69855518168202\\
0.312625250501002	-1.69424237342747\\
0.302195367801572	-1.69138276553106\\
0.296593186372745	-1.68989443233561\\
0.280561122244489	-1.68557374456136\\
0.264529058116232	-1.6811472914142\\
0.248496993987976	-1.67661164590486\\
0.244097738957461	-1.67535070140281\\
0.232464929859719	-1.67211540462806\\
0.216432865731463	-1.66756660610403\\
0.200400801603206	-1.66290534725454\\
0.18831602131676	-1.65931863727455\\
0.18436873747495	-1.6581802741877\\
0.168336673346693	-1.65349968416365\\
0.152304609218437	-1.64870301058371\\
0.13627254509018	-1.64378636918556\\
0.13465855787968	-1.64328657314629\\
0.120240480961924	-1.63894222814516\\
0.104208416833667	-1.63399950218153\\
0.0881763527054105	-1.62893261901265\\
0.0829369897190766	-1.62725450901804\\
0.0721442885771539	-1.62388619826427\\
0.0561122244488974	-1.61878586874869\\
0.0400801603206409	-1.61355676610086\\
0.0330311128682826	-1.61122244488978\\
0.0240480961923843	-1.60831968963625\\
0.00801603206412782	-1.60304920425247\\
-0.00801603206412826	-1.59764486856376\\
-0.0151872422751879	-1.59519038076152\\
-0.0240480961923848	-1.59222676272478\\
-0.0400801603206413	-1.58677246885299\\
-0.0561122244488979	-1.58117874717943\\
-0.0618253906258919	-1.57915831663327\\
-0.0721442885771544	-1.57558722396177\\
-0.0881763527054109	-1.56993425666789\\
-0.104208416833667	-1.5641357398035\\
-0.106971951051998	-1.56312625250501\\
-0.120240480961924	-1.55837635262843\\
-0.13627254509018	-1.55250850824985\\
-0.150718953194898	-1.54709418837675\\
-0.152304609218437	-1.54651092503132\\
-0.168336673346694	-1.54056456913024\\
-0.18436873747495	-1.53446416266356\\
-0.193159285396915	-1.5310621242485\\
-0.200400801603207	-1.52830754603955\\
-0.216432865731463	-1.52211703872842\\
-0.232464929859719	-1.51576474294339\\
-0.234303765062478	-1.51503006012024\\
-0.248496993987976	-1.5094482638141\\
-0.264529058116232	-1.50299320358208\\
-0.274274032769787	-1.49899799599198\\
-0.280561122244489	-1.4964570164149\\
-0.296593186372745	-1.48988820280757\\
-0.312625250501002	-1.48314623457938\\
-0.31305133922246	-1.48296593186373\\
-0.328657314629258	-1.47644625708424\\
-0.344689378757515	-1.46957579998439\\
-0.350772622677735	-1.46693386773547\\
-0.360721442885771	-1.46266161809656\\
-0.376753507014028	-1.45565018920506\\
-0.387404607532609	-1.45090180360721\\
-0.392785571142285	-1.44852616098423\\
-0.408817635270541	-1.44136057868542\\
-0.422998359488645	-1.43486973947896\\
-0.424849699398798	-1.43402924560527\\
-0.440881763527054	-1.4266955505133\\
-0.456913827655311	-1.41916660613899\\
-0.457610094455509	-1.4188376753507\\
-0.472945891783567	-1.41164091664999\\
-0.488977955911824	-1.4039241233092\\
-0.491283309968169	-1.40280561122244\\
-0.50501002004008	-1.39617950933902\\
-0.521042084168337	-1.38825863394789\\
-0.524021784882639	-1.38677354709419\\
-0.537074148296593	-1.38029093372379\\
-0.55310621242485	-1.37214864905154\\
-0.555853993635378	-1.37074148296593\\
-0.569138276553106	-1.36395127795074\\
-0.585170340681363	-1.35556904431561\\
-0.586803573374233	-1.35470941883768\\
-0.601202404809619	-1.34713277515455\\
-0.616895642836883	-1.33867735470942\\
-0.617234468937876	-1.33849452965622\\
-0.633266533066132	-1.32980341068702\\
-0.646183583404857	-1.32264529058116\\
-0.649298597194389	-1.32091365820452\\
-0.665330661322646	-1.3119264667285\\
-0.674649797431261	-1.30661322645291\\
-0.681362725450902	-1.30276741258878\\
-0.697394789579158	-1.29345999496945\\
-0.702300948890021	-1.29058116232465\\
-0.713426853707415	-1.28400997311574\\
-0.729145574621661	-1.27454909819639\\
-0.729458917835671	-1.27435892001985\\
-0.745490981963928	-1.26458918906142\\
-0.755284879972821	-1.25851703406814\\
-0.761523046092184	-1.2546101194984\\
-0.777555110220441	-1.24444578626272\\
-0.7806241808328	-1.24248496993988\\
-0.793587174348697	-1.23410389891954\\
-0.805240924452793	-1.22645290581162\\
-0.809619238476954	-1.22353844250848\\
-0.82565130260521	-1.21276675145543\\
-0.829115242856569	-1.21042084168337\\
-0.841683366733467	-1.20177489351671\\
-0.85228185589804	-1.19438877755511\\
-0.857715430861723	-1.19053461550663\\
-0.87374749498998	-1.17904311740075\\
-0.874701214362226	-1.17835671342685\\
-0.889779559118236	-1.16729000547233\\
-0.896473524943851	-1.1623246492986\\
-0.905811623246493	-1.15524611350815\\
-0.917512421144481	-1.14629258517034\\
-0.92184368737475	-1.14289828669822\\
-0.937837418717266	-1.13026052104208\\
-0.937875751503006	-1.13022943151893\\
-0.953907815631263	-1.11719957623773\\
-0.957537988230582	-1.11422845691383\\
-0.969939879759519	-1.10378789690598\\
-0.976542787390598	-1.09819639278557\\
-0.985971943887776	-1.08996381496024\\
-0.994863009677645	-1.08216432865731\\
-1.00200400801603	-1.07568984723592\\
-1.01250816539724	-1.06613226452906\\
-1.01803607214429	-1.06092046374046\\
-1.0294853561977	-1.0501002004008\\
-1.03406813627255	-1.04560051479515\\
-1.04579931683519	-1.03406813627255\\
-1.0501002004008	-1.02966324992676\\
-1.06145244296743	-1.01803607214429\\
-1.06613226452906	-1.01302780423393\\
-1.07644480520421	-1.00200400801603\\
-1.08216432865731	-0.995595982537537\\
-1.09077414953035	-0.985971943887776\\
-1.09819639278557	-0.977248106704591\\
-1.10443588410436	-0.969939879759519\\
-1.11422845691383	-0.957837597329472\\
-1.1174230523247	-0.953907815631263\\
-1.12973712563456	-0.937875751503006\\
-1.13026052104208	-0.937153613325981\\
-1.1414344627811	-0.92184368737475\\
-1.14629258517034	-0.914732213295877\\
-1.15243634880785	-0.905811623246493\\
-1.1623246492986	-0.890399641739081\\
-1.16272621186394	-0.889779559118236\\
-1.17240689915918	-0.87374749498998\\
-1.17835671342685	-0.863049360795811\\
-1.18135787195892	-0.857715430861723\\
-1.18964297833635	-0.841683366733467\\
-1.19438877755511	-0.831582094477961\\
-1.19721429494284	-0.82565130260521\\
-1.20410894208703	-0.809619238476954\\
-1.21024313549248	-0.793587174348697\\
-1.21042084168337	-0.79305888897335\\
-1.21572756665504	-0.777555110220441\\
-1.22045253560373	-0.761523046092184\\
-1.22441739916725	-0.745490981963928\\
-1.22645290581162	-0.735298985507058\\
-1.227643843671	-0.729458917835671\\
-1.23012885885582	-0.713426853707415\\
-1.23182613558288	-0.697394789579158\\
-1.2327250346495	-0.681362725450902\\
-1.23281122281346	-0.665330661322646\\
-1.23206655780207	-0.649298597194389\\
-1.23046894507415	-0.633266533066132\\
-1.22799216483383	-0.617234468937876\\
-1.22645290581162	-0.60989579031488\\
-1.22463637147691	-0.601202404809619\\
-1.22037541581624	-0.585170340681363\\
-1.21514323831581	-0.569138276553106\\
-1.21042084168337	-0.556970478221209\\
-1.20891836079986	-0.55310621242485\\
-1.20171400441608	-0.537074148296593\\
-1.19438877755511	-0.522916246575606\\
-1.19341348897151	-0.521042084168337\\
-1.18406152950669	-0.50501002004008\\
-1.17835671342685	-0.496273466067223\\
-1.17354629378645	-0.488977955911824\\
-1.1623246492986	-0.473624840176101\\
-1.16182240491084	-0.472945891783567\\
-1.14888095349199	-0.456913827655311\\
-1.14629258517034	-0.453965114666237\\
-1.13462257143131	-0.440881763527054\\
-1.13026052104208	-0.436352947917628\\
-1.11897579086372	-0.424849699398798\\
-1.11422845691383	-0.42033568393905\\
-1.10186209229604	-0.408817635270541\\
-1.09819639278557	-0.405611983691325\\
-1.08318779310296	-0.392785571142285\\
-1.08216432865731	-0.391959428237392\\
-1.06613226452906	-0.379270499863795\\
-1.06287424700731	-0.376753507014028\\
-1.0501002004008	-0.367372849170581\\
-1.04078189098744	-0.360721442885771\\
-1.03406813627255	-0.356142942441398\\
-1.01803607214429	-0.345519122814556\\
-1.0167582355621	-0.344689378757515\\
-1.00200400801603	-0.335485728868845\\
-0.990680312709161	-0.328657314629258\\
-0.985971943887776	-0.325917570256128\\
-0.969939879759519	-0.316817203622289\\
-0.962332115374341	-0.312625250501002\\
-0.953907815631263	-0.308123992866641\\
-0.937875751503006	-0.299812511405945\\
-0.931492033888463	-0.296593186372745\\
-0.92184368737475	-0.291852854026678\\
-0.905811623246493	-0.284221440715854\\
-0.897891126775563	-0.280561122244489\\
-0.889779559118236	-0.276892806911203\\
-0.87374749498998	-0.269851144434226\\
-0.861191691049024	-0.264529058116232\\
-0.857715430861723	-0.263080970145336\\
-0.841683366733467	-0.256553576370402\\
-0.82565130260521	-0.250279544210271\\
-0.820980159917562	-0.248496993987976\\
-0.809619238476954	-0.244216256155398\\
-0.793587174348697	-0.238372209167776\\
-0.777555110220441	-0.232752279202886\\
-0.776718141755782	-0.232464929859719\\
-0.761523046092184	-0.227290568006898\\
-0.745490981963928	-0.222032664374439\\
-0.729458917835671	-0.216974426661787\\
-0.727703356964025	-0.216432865731463\\
-0.713426853707415	-0.212045479714019\\
-0.697394789579158	-0.207294245117741\\
-0.681362725450902	-0.20272213372618\\
-0.672971995943429	-0.200400801603207\\
-0.665330661322646	-0.198285840526224\\
-0.649298597194389	-0.193976083677792\\
-0.633266533066132	-0.189828124614682\\
-0.617234468937876	-0.185836627329769\\
-0.611166861606183	-0.18436873747495\\
-0.601202404809619	-0.181945727804442\\
-0.585170340681363	-0.178171047337197\\
-0.569138276553106	-0.174538818226188\\
-0.55310621242485	-0.171044526007193\\
-0.540225793925939	-0.168336673346694\\
-0.537074148296593	-0.167667660617935\\
-0.521042084168337	-0.164355158474811\\
-0.50501002004008	-0.161169256789747\\
-0.488977955911824	-0.158106117227103\\
-0.472945891783567	-0.155162047461685\\
-0.456913827655311	-0.152333495567803\\
-0.456745539136409	-0.152304609218437\\
-0.440881763527054	-0.149540010801065\\
-0.424849699398798	-0.146856043117213\\
-0.408817635270541	-0.144279260024262\\
-0.392785571142285	-0.141806629348554\\
-0.376753507014028	-0.139435234112738\\
-0.360721442885771	-0.137162268314306\\
-0.354214135689647	-0.13627254509018\\
-0.344689378757515	-0.134943244623806\\
-0.328657314629258	-0.132790514660219\\
-0.312625250501002	-0.130730201071956\\
-0.296593186372745	-0.128759905597946\\
-0.280561122244489	-0.126877322422596\\
-0.264529058116232	-0.125080234984817\\
-0.248496993987976	-0.123366512925925\\
-0.232464929859719	-0.121734109170593\\
-0.217051797086759	-0.120240480961924\\
-0.216432865731463	-0.120178874314433\\
-0.200400801603207	-0.118648644592585\\
-0.18436873747495	-0.117196046149554\\
-0.168336673346694	-0.115819339393302\\
-0.152304609218437	-0.114516855696396\\
-0.13627254509018	-0.113286995195251\\
-0.120240480961924	-0.11212822468841\\
-0.104208416833667	-0.111039075630082\\
-0.0881763527054109	-0.11001814221532\\
-0.0721442885771544	-0.109064079553407\\
-0.0561122244488979	-0.108175601926193\\
-0.0400801603206413	-0.107351481128298\\
-0.0240480961923848	-0.106590544886223\\
-0.00801603206412826	-0.105891675353611\\
0.00801603206412782	-0.105253807679987\\
0.0240480961923843	-0.104675928650494\\
0.0384860915488141	-0.104208416833667\\
0.0400801603206409	-0.104154861059979\\
0.0561122244488974	-0.103674106958934\\
0.0721442885771539	-0.103252845750643\\
0.0881763527054105	-0.102890270253362\\
0.104208416833667	-0.102585620105656\\
0.120240480961924	-0.102338180810791\\
0.13627254509018	-0.102147282838598\\
0.152304609218437	-0.10201230078332\\
0.168336673346693	-0.10193265257598\\
0.18436873747495	-0.101907798749958\\
0.200400801603206	-0.10193724175852\\
0.216432865731463	-0.102020525343162\\
0.232464929859719	-0.102157233951674\\
0.248496993987976	-0.102346992204979\\
0.264529058116232	-0.102589464411798\\
0.280561122244489	-0.102884354130366\\
0.296593186372745	-0.103231403776418\\
0.312625250501002	-0.103630394276798\\
0.328657314629258	-0.10408114476809\\
0.332716612237025	-0.104208416833667\\
0.344689378757515	-0.10456622499153\\
0.360721442885771	-0.105094349394231\\
0.376753507014028	-0.105671250672454\\
0.392785571142285	-0.106296862979368\\
0.408817635270541	-0.106971156214083\\
0.424849699398798	-0.107694135936757\\
0.440881763527054	-0.108465843324835\\
0.456913827655311	-0.109286355170345\\
0.472945891783567	-0.110155783918181\\
0.488977955911824	-0.111074277745425\\
0.50501002004008	-0.112042020681798\\
0.521042084168337	-0.113059232771386\\
0.537074148296593	-0.114126170275889\\
0.55310621242485	-0.115243125919667\\
0.569138276553106	-0.11641042917696\\
0.585170340681363	-0.117628446601689\\
0.601202404809619	-0.118897582200367\\
0.617234468937876	-0.120218277848649\\
0.617494556466999	-0.120240480961924\\
0.633266533066132	-0.121525088420303\\
0.649298597194389	-0.122880139385738\\
0.665330661322646	-0.124285003623758\\
0.681362725450902	-0.125740213111226\\
0.697394789579159	-0.127246337956668\\
0.713426853707415	-0.128803987012182\\
0.729458917835672	-0.130413808531744\\
0.745490981963928	-0.132076490876939\\
0.761523046092185	-0.133792763271252\\
0.777555110220441	-0.135563396604083\\
0.783801282451325	-0.13627254509018\\
0.793587174348698	-0.137333834991438\\
0.809619238476954	-0.139121643598175\\
0.825651302605211	-0.14096263267455\\
0.841683366733467	-0.142857662322218\\
0.857715430861724	-0.144807635966932\\
0.87374749498998	-0.146813501380113\\
0.889779559118236	-0.148876251756569\\
0.905811623246493	-0.150996926850086\\
0.915455342748828	-0.152304609218437\\
0.921843687374749	-0.15313312487587\\
0.937875751503006	-0.155260508531662\\
0.953907815631262	-0.15744506795938\\
0.969939879759519	-0.159687937423864\\
0.985971943887775	-0.161990300581893\\
1.00200400801603	-0.16435339186966\\
1.01803607214429	-0.166778497957794\\
1.02810450037453	-0.168336673346694\\
1.03406813627254	-0.169220518744618\\
1.0501002004008	-0.171645438504814\\
1.06613226452906	-0.174132004406286\\
1.08216432865731	-0.176681606700931\\
1.09819639278557	-0.179295691836934\\
1.11422845691383	-0.181975764208748\\
1.12820743196796	-0.18436873747495\\
1.13026052104208	-0.184705751632533\\
1.14629258517034	-0.187381715908896\\
1.1623246492986	-0.190123640007622\\
1.17835671342685	-0.192933142806197\\
1.19438877755511	-0.195811906147217\\
1.21042084168337	-0.198761676929586\\
1.21915527505219	-0.200400801603207\\
1.22645290581162	-0.20171573683135\\
1.24248496993988	-0.204658528653169\\
1.25851703406814	-0.207672662366728\\
1.27454909819639	-0.210760014104636\\
1.29058116232465	-0.213922531296441\\
1.30303461552212	-0.216432865731463\\
1.30661322645291	-0.21712642044707\\
1.32264529058116	-0.220281459535333\\
1.33867735470942	-0.223512325046621\\
1.35470941883768	-0.226821107964769\\
1.37074148296593	-0.230209977900019\\
1.38119955171829	-0.232464929859719\\
1.38677354709419	-0.233621991425851\\
1.40280561122244	-0.237003658389757\\
1.4188376753507	-0.24046642741907\\
1.43486973947896	-0.244012629102991\\
1.45090180360721	-0.247644681233421\\
1.45461093983895	-0.248496993987976\\
1.46693386773547	-0.25122675209509\\
1.48296593186373	-0.254852669142662\\
1.49899799599198	-0.258565842333484\\
1.51503006012024	-0.262368871411008\\
1.52397338188145	-0.264529058116232\\
1.5310621242485	-0.266181927726857\\
1.54709418837675	-0.269981682194586\\
1.56312625250501	-0.273873033407793\\
1.57915831663327	-0.277858780516938\\
1.58981949077842	-0.280561122244489\\
1.59519038076152	-0.281877111478385\\
1.61122244488978	-0.285863031556741\\
1.62725450901804	-0.28994545272518\\
1.64328657314629	-0.294127393480663\\
1.65257073444299	-0.296593186372745\\
1.65931863727455	-0.298327982563034\\
1.67535070140281	-0.302514173822077\\
1.69138276553106	-0.306802389075245\\
1.70741482965932	-0.311195888906087\\
1.71255835550467	-0.312625250501002\\
1.72344689378758	-0.315558252571614\\
1.73947895791583	-0.319960766466128\\
1.75551102204409	-0.324471509825136\\
1.77005005683371	-0.328657314629258\\
1.77154308617235	-0.32907455313386\\
1.7875751503006	-0.333600082495856\\
1.80360721442886	-0.338237118111008\\
1.81963927855711	-0.342989345630808\\
1.82529448076157	-0.344689378757515\\
1.83567134268537	-0.347721498191544\\
1.85170340681363	-0.352494647422147\\
1.86773547094188	-0.3573867905862\\
1.87845560715137	-0.360721442885771\\
1.88376753507014	-0.36232982031682\\
1.8997995991984	-0.367249745655317\\
1.91583166332665	-0.37229286251328\\
1.92969533560309	-0.376753507014028\\
1.93186372745491	-0.377433585862376\\
1.94789579158317	-0.382511868911486\\
1.96392785571142	-0.387717967589056\\
1.97916047409932	-0.392785571142285\\
1.97995991983968	-0.393045191080948\\
1.99599198396794	-0.39829442099365\\
2.01202404809619	-0.403676552092698\\
2.02697494258574	-0.408817635270541\\
2.02805611222445	-0.409181067420323\\
2.04408817635271	-0.414614940961463\\
2.06012024048096	-0.420187302017891\\
2.07324228137456	-0.424849699398798\\
2.07615230460922	-0.425861907510531\\
2.09218436873747	-0.431495342091625\\
2.10821643286573	-0.437273395508012\\
2.11804774549988	-0.440881763527054\\
2.12424849699399	-0.443112945236902\\
2.14028056112224	-0.448962206494252\\
2.1563126252505	-0.45496281259183\\
2.16146004350126	-0.456913827655311\\
2.17234468937876	-0.460964294899084\\
2.18837675350701	-0.46704714063442\\
2.20354441663539	-0.472945891783567\\
2.20440881763527	-0.473276433981669\\
2.22044088176353	-0.479451355540376\\
2.23647294589178	-0.485787198128501\\
2.2444197751215	-0.488977955911824\\
2.25250501002004	-0.492174622005283\\
2.2685370741483	-0.498615287757921\\
2.28405691287249	-0.50501002004008\\
2.28456913827655	-0.505218161956765\\
2.30060120240481	-0.511774693963323\\
2.31663326653307	-0.518503571702674\\
2.32260188881461	-0.521042084168337\\
2.33266533066132	-0.52526931597279\\
2.34869739478958	-0.532128905387535\\
2.36002563379325	-0.537074148296593\\
2.36472945891784	-0.539105326602662\\
2.38076152304609	-0.546108190561773\\
2.39637673693721	-0.55310621242485\\
2.39679358717435	-0.55329128907943\\
2.41282565130261	-0.560450703256046\\
2.42885771543086	-0.567800183419584\\
2.43175059379652	-0.569138276553106\\
2.44488977955912	-0.575168334392808\\
2.46092184368737	-0.582692973371629\\
2.4661375939215	-0.585170340681363\\
2.47695390781563	-0.59027577175072\\
2.49298597194389	-0.597990943962867\\
2.49956881926398	-0.601202404809619\\
2.50901803607214	-0.605790716426717\\
2.5250501002004	-0.613712801736243\\
2.53207662734772	-0.617234468937876\\
2.54108216432866	-0.621734137911362\\
2.55711422845691	-0.62988063077605\\
2.56368836356926	-0.633266533066132\\
2.57314629258517	-0.638130572670557\\
2.58917835671343	-0.646520202535423\\
2.59442660288372	-0.649298597194389\\
2.60521042084168	-0.655008472037091\\
2.62124248496994	-0.663661336444696\\
2.62430935165147	-0.665330661322646\\
2.6372745490982	-0.672400606287379\\
2.65330661322645	-0.681338318376581\\
2.65335021200776	-0.681362725450902\\
2.66933867735471	-0.690344533046737\\
2.6816283839591	-0.697394789579158\\
2.68537074148297	-0.699553006362883\\
2.70140280561122	-0.708883139673524\\
2.7090976923053	-0.713426853707415\\
2.71743486973948	-0.718384304556037\\
2.73346693386774	-0.72806527107101\\
2.73575909646037	-0.729458917835671\\
2.74949899799599	-0.737885966965321\\
2.76168466808382	-0.745490981963928\\
2.76553106212425	-0.747917021243419\\
2.7815631262525	-0.758118574644316\\
2.78685602799635	-0.761523046092184\\
2.79759519038076	-0.768516595258497\\
2.81126741125086	-0.777555110220441\\
2.81362725450902	-0.77913764208083\\
2.82965931863727	-0.78995808756234\\
2.83498017437936	-0.793587174348697\\
2.84569138276553	-0.801011745440611\\
2.85795280620848	-0.809619238476954\\
2.86172344689379	-0.812314768379373\\
2.87775551102204	-0.823868485547144\\
2.88021422052494	-0.82565130260521\\
2.8937875751503	-0.835693255417653\\
2.90180119755875	-0.841683366733467\\
2.90981963927856	-0.84781175307244\\
2.92265989573371	-0.857715430861723\\
2.92585170340681	-0.860238068391342\\
2.94188376753507	-0.872993614114716\\
2.94282802612816	-0.87374749498998\\
2.95791583166333	-0.886116465166599\\
2.96235397925269	-0.889779559118236\\
2.97394789579158	-0.8996281213868\\
2.98118660890843	-0.905811623246493\\
2.98997995991984	-0.913560870034076\\
2.99933726789753	-0.92184368737475\\
3.0060120240481	-0.927954077145858\\
3.01681485966983	-0.937875751503006\\
3.02204408817635	-0.942855516017133\\
3.03362589078586	-0.953907815631263\\
3.03807615230461	-0.95832303241286\\
3.04977450916141	-0.969939879759519\\
3.05410821643287	-0.974426645647578\\
3.0652625284084	-0.985971943887776\\
3.07014028056112	-0.991251219722232\\
3.08008943847214	-1.00200400801603\\
3.08617234468938	-1.00889988830034\\
3.09425240265218	-1.01803607214429\\
3.10220440881764	-1.02749848845569\\
3.10774624098251	-1.03406813627255\\
3.11823647294589	-1.04720136177528\\
3.12056339983439	-1.0501002004008\\
3.13272584932801	-1.06613226452906\\
3.13426853707415	-1.06829179104506\\
3.14425073569081	-1.08216432865731\\
3.1503006012024	-1.09115715490849\\
3.15507627168849	-1.09819639278557\\
3.16520827887118	-1.11422845691383\\
3.16633266533066	-1.1161427044615\\
3.17471471082106	-1.13026052104208\\
3.18236472945892	-1.14427770569465\\
3.18347754743401	-1.14629258517034\\
3.19160942401605	-1.1623246492986\\
3.19839679358717	-1.17709902819093\\
3.19898285528927	-1.17835671342685\\
3.20571977385265	-1.19438877755511\\
3.21169426809874	-1.21042084168337\\
3.21442885771543	-1.21881219331586\\
3.2169632640572	-1.22645290581162\\
3.22152011503098	-1.24248496993988\\
3.22531285095565	-1.25851703406814\\
3.22834019319437	-1.27454909819639\\
3.23046092184369	-1.28961521088211\\
}--cycle;


\addplot[area legend,solid,fill=mycolor3,draw=black,forget plot]
table[row sep=crcr] {%
x	y\\
2.74949899799599	-1.18326472763881\\
2.75260754194712	-1.19438877755511\\
2.75612309650223	-1.21042084168337\\
2.75866599003552	-1.22645290581162\\
2.76021903155964	-1.24248496993988\\
2.76076010569051	-1.25851703406814\\
2.7602619777296	-1.27454909819639\\
2.75869205822277	-1.29058116232465\\
2.75601212433699	-1.30661322645291\\
2.75217799490544	-1.32264529058116\\
2.74949899799599	-1.33123319658754\\
2.74715911798893	-1.33867735470942\\
2.74090891296912	-1.35470941883768\\
2.73346693386774	-1.37047492503076\\
2.7333396172674	-1.37074148296593\\
2.72444049721794	-1.38677354709419\\
2.71743486973948	-1.39768370072979\\
2.71409291534319	-1.40280561122244\\
2.70223404531362	-1.4188376753507\\
2.70140280561122	-1.41984743589894\\
2.68878284738327	-1.43486973947896\\
2.68537074148297	-1.43854094295319\\
2.67361715333422	-1.45090180360721\\
2.66933867735471	-1.45501335337336\\
2.65661559695331	-1.46693386773547\\
2.65330661322645	-1.46979359430419\\
2.63763313386318	-1.48296593186373\\
2.6372745490982	-1.48324622386572\\
2.62124248496994	-1.49549138280907\\
2.61650812880541	-1.49899799599198\\
2.60521042084168	-1.50684677781941\\
2.59302420116193	-1.51503006012024\\
2.58917835671343	-1.51746839291743\\
2.57314629258517	-1.52735391063724\\
2.56693774903721	-1.5310621242485\\
2.55711422845691	-1.53664022119633\\
2.54108216432866	-1.54542174962705\\
2.53794408900653	-1.54709418837675\\
2.5250501002004	-1.55366872448046\\
2.50901803607214	-1.56152567514196\\
2.50566034730948	-1.56312625250501\\
2.49298597194389	-1.56894049328515\\
2.47695390781563	-1.57601499019121\\
2.46959381054555	-1.57915831663327\\
2.46092184368737	-1.58274192733288\\
2.44488977955912	-1.58914788250036\\
2.42911326439086	-1.59519038076152\\
2.42885771543086	-1.59528557259378\\
2.41282565130261	-1.60111518356387\\
2.39679358717435	-1.60670410532438\\
2.38330801104153	-1.61122244488978\\
2.38076152304609	-1.61205683119059\\
2.36472945891784	-1.61716708414087\\
2.34869739478958	-1.62206668090268\\
2.33266533066132	-1.62676289008264\\
2.33093845342387	-1.62725450901804\\
2.31663326653307	-1.63126066652016\\
2.30060120240481	-1.63557187437577\\
2.28456913827655	-1.63970312216702\\
2.27007167249606	-1.64328657314629\\
2.2685370741483	-1.64366156587616\\
2.25250501002004	-1.64746455226666\\
2.23647294589178	-1.65110682639766\\
2.22044088176353	-1.65459344077343\\
2.20440881763527	-1.65792924711447\\
2.19747729133886	-1.65931863727455\\
2.18837675350701	-1.66113275972503\\
2.17234468937876	-1.66420595186482\\
2.1563126252505	-1.66714279999314\\
2.14028056112224	-1.66994730486461\\
2.12424849699399	-1.67262330880815\\
2.10821643286573	-1.67517450175016\\
2.10706612054305	-1.67535070140281\\
2.09218436873747	-1.67763196612065\\
2.07615230460922	-1.67997385088361\\
2.06012024048096	-1.682201242181\\
2.04408817635271	-1.68431719109936\\
2.02805611222445	-1.68632462610407\\
2.01202404809619	-1.68822635739643\\
1.99599198396794	-1.69002508107448\\
1.98319952900127	-1.69138276553106\\
1.97995991983968	-1.69172913095202\\
1.96392785571142	-1.69335713625728\\
1.94789579158317	-1.69488895210973\\
1.93186372745491	-1.6963267982689\\
1.91583166332665	-1.69767280039442\\
1.8997995991984	-1.69892899305954\\
1.88376753507014	-1.70009732262513\\
1.86773547094188	-1.70117964997975\\
1.85170340681363	-1.70217775315098\\
1.83567134268537	-1.70309332979314\\
1.81963927855711	-1.70392799955601\\
1.80360721442886	-1.70468330633918\\
1.7875751503006	-1.70536072043631\\
1.77154308617235	-1.70596164057314\\
1.75551102204409	-1.70648739584342\\
1.73947895791583	-1.70693924754604\\
1.72344689378758	-1.70731839092705\\
1.71843738150381	-1.70741482965932\\
1.70741482965932	-1.70763112374955\\
1.69138276553106	-1.70787412022587\\
1.67535070140281	-1.70804601607068\\
1.65931863727455	-1.70814771731554\\
1.64328657314629	-1.70818007128771\\
1.62725450901804	-1.70814386777236\\
1.61122244488978	-1.70803984009991\\
1.59519038076152	-1.70786866616041\\
1.57915831663327	-1.70763096934689\\
1.56775534575173	-1.70741482965932\\
1.56312625250501	-1.70732963916665\\
1.54709418837675	-1.70697046793427\\
1.5310621242485	-1.70654829612779\\
1.51503006012024	-1.70606358067379\\
1.49899799599198	-1.70551672835325\\
1.48296593186373	-1.7049080964152\\
1.46693386773547	-1.70423799312882\\
1.45090180360721	-1.70350667827507\\
1.43486973947896	-1.70271436357871\\
1.4188376753507	-1.70186121308157\\
1.40280561122244	-1.7009473434578\\
1.38677354709419	-1.69997282427174\\
1.37074148296593	-1.69893767817887\\
1.35470941883768	-1.69784188107032\\
1.33867735470942	-1.69668536216133\\
1.32264529058116	-1.69546800402378\\
1.30661322645291	-1.69418964256299\\
1.29058116232465	-1.69285006693896\\
1.27454909819639	-1.6914490194318\\
1.27382196966232	-1.69138276553106\\
1.25851703406814	-1.69002803286628\\
1.24248496993988	-1.68854957546958\\
1.22645290581162	-1.68701134200726\\
1.21042084168337	-1.68541294730658\\
1.19438877755511	-1.68375395958453\\
1.17835671342685	-1.68203390005504\\
1.1623246492986	-1.68025224247994\\
1.14629258517034	-1.6784084126627\\
1.13026052104208	-1.67650178788443\\
1.12088065905737	-1.67535070140281\\
1.11422845691383	-1.67455673154897\\
1.09819639278557	-1.6725854237553\\
1.08216432865731	-1.67055229801051\\
1.06613226452906	-1.66845660605421\\
1.0501002004008	-1.66629754973825\\
1.03406813627254	-1.66407428019374\\
1.01803607214429	-1.66178589693549\\
1.00200400801603	-1.65943144690233\\
1.00125370371338	-1.65931863727455\\
0.985971943887775	-1.65708126153868\\
0.969939879759519	-1.65466985485516\\
0.953907815631262	-1.65219274151581\\
0.937875751503006	-1.64964883652478\\
0.921843687374749	-1.64703699944133\\
0.905811623246493	-1.64435603311995\\
0.899557379466247	-1.64328657314629\\
0.889779559118236	-1.6416562803369\\
0.87374749498998	-1.63892108154879\\
0.857715430861724	-1.63611664439505\\
0.841683366733467	-1.63324163072518\\
0.825651302605211	-1.63029464164771\\
0.809619238476954	-1.62727421593532\\
0.809516456343981	-1.62725450901804\\
0.793587174348698	-1.62427250481385\\
0.777555110220441	-1.62119890115366\\
0.761523046092185	-1.61805126006268\\
0.745490981963928	-1.61482796560761\\
0.729458917835672	-1.61152733417762\\
0.72800224322814	-1.61122244488978\\
0.713426853707415	-1.60823972649097\\
0.697394789579159	-1.60488457757833\\
0.681362725450902	-1.60145105800788\\
0.665330661322646	-1.59793731512235\\
0.653059055799319	-1.59519038076152\\
0.649298597194389	-1.59436619086757\\
0.633266533066132	-1.59079477913598\\
0.617234468937876	-1.58714175002672\\
0.601202404809619	-1.5834050690165\\
0.585170340681363	-1.57958262216529\\
0.583416144990789	-1.57915831663327\\
0.569138276553106	-1.57577215961472\\
0.55310621242485	-1.5718883119034\\
0.537074148296593	-1.56791680232633\\
0.521042084168337	-1.56385531389739\\
0.518204909499616	-1.56312625250501\\
0.50501002004008	-1.55979692016909\\
0.488977955911824	-1.55566878969993\\
0.472945891783567	-1.5514483580455\\
0.456913827655311	-1.54713308655087\\
0.456771004731442	-1.54709418837675\\
0.440881763527054	-1.54283879074733\\
0.424849699398798	-1.53845059137728\\
0.408817635270541	-1.53396476441611\\
0.398629758827111	-1.5310621242485\\
0.392785571142285	-1.52942228392378\\
0.376753507014028	-1.52485646939428\\
0.360721442885771	-1.52018988620391\\
0.344689378757515	-1.51541959330595\\
0.343395038987932	-1.51503006012024\\
0.328657314629258	-1.51065552575352\\
0.312625250501002	-1.50579686445284\\
0.296593186372745	-1.50083080263735\\
0.29076041332747	-1.49899799599198\\
0.280561122244489	-1.49583217707382\\
0.264529058116232	-1.4907689048451\\
0.248496993987976	-1.48559412084379\\
0.240481399173424	-1.48296593186373\\
0.232464929859719	-1.48036546040542\\
0.216432865731463	-1.47508372952271\\
0.200400801603206	-1.4696859210827\\
0.192352574134425	-1.46693386773547\\
0.18436873747495	-1.46422869800438\\
0.168336673346693	-1.458713226219\\
0.152304609218437	-1.45307661776678\\
0.14620289192794	-1.45090180360721\\
0.13627254509018	-1.44738908264105\\
0.120240480961924	-1.44162301616266\\
0.104208416833667	-1.43573021652507\\
0.101891245356959	-1.43486973947896\\
0.0881763527054105	-1.4298071750806\\
0.0721442885771539	-1.42377193208359\\
0.0592839881535078	-1.4188376753507\\
0.0561122244488974	-1.41762596735156\\
0.0400801603206409	-1.41143630355962\\
0.0240480961923843	-1.40511139579183\\
0.0182774071501571	-1.40280561122244\\
0.00801603206412782	-1.39871655419775\\
-0.00801603206412826	-1.39222185299629\\
-0.0212101310917292	-1.38677354709419\\
-0.0240480961923848	-1.38560280547769\\
-0.0400801603206413	-1.37892435582129\\
-0.0561122244488979	-1.37210042925093\\
-0.0592733399355444	-1.37074148296593\\
-0.0721442885771544	-1.36520468913413\\
-0.0881763527054109	-1.35817874602116\\
-0.0959827596567683	-1.35470941883768\\
-0.104208416833667	-1.35104492150329\\
-0.120240480961924	-1.34380057043271\\
-0.131388017094275	-1.33867735470942\\
-0.13627254509018	-1.33642305817234\\
-0.152304609218437	-1.3289426552312\\
-0.165550377561985	-1.32264529058116\\
-0.168336673346694	-1.3213126274169\\
-0.18436873747495	-1.31357714092764\\
-0.198523229332871	-1.30661322645291\\
-0.200400801603207	-1.3056821897858\\
-0.216432865731463	-1.29767104929049\\
-0.230352610473349	-1.29058116232465\\
-0.232464929859719	-1.28949475749144\\
-0.248496993987976	-1.28118568483955\\
-0.261077664673576	-1.27454909819639\\
-0.264529058116232	-1.27270710848085\\
-0.280561122244489	-1.26407592855336\\
-0.290731032066843	-1.25851703406814\\
-0.296593186372745	-1.25526897635711\\
-0.312625250501002	-1.24628940458971\\
-0.319339180154265	-1.24248496993988\\
-0.328657314629258	-1.23712209316453\\
-0.344689378757515	-1.22776549660303\\
-0.346922679161979	-1.22645290581162\\
-0.360721442885771	-1.21819905688061\\
-0.373532419084184	-1.21042084168337\\
-0.376753507014028	-1.20842600940475\\
-0.392785571142285	-1.19842198899159\\
-0.399180918769461	-1.19438877755511\\
-0.408817635270541	-1.18817656140745\\
-0.423860679260853	-1.17835671342685\\
-0.424849699398798	-1.17769520165555\\
-0.440881763527054	-1.16691961685951\\
-0.447647500059588	-1.1623246492986\\
-0.456913827655311	-1.15586188580738\\
-0.470495327575244	-1.14629258517034\\
-0.472945891783567	-1.14451492734921\\
-0.488977955911824	-1.13281857110997\\
-0.492461024182903	-1.13026052104208\\
-0.50501002004008	-1.12074893765486\\
-0.513545806149487	-1.11422845691383\\
-0.521042084168337	-1.10830227182791\\
-0.53374083042227	-1.09819639278557\\
-0.537074148296593	-1.0954431889178\\
-0.553063246130047	-1.08216432865731\\
-0.55310621242485	-1.08212718076834\\
-0.569138276553106	-1.06823651811433\\
-0.571556671111825	-1.06613226452906\\
-0.585170340681363	-1.05376534811553\\
-0.589192502500737	-1.0501002004008\\
-0.601202404809619	-1.03863574624376\\
-0.605978114260035	-1.03406813627255\\
-0.617234468937876	-1.02274949217108\\
-0.621918403438138	-1.01803607214429\\
-0.633266533066132	-1.00598374074828\\
-0.637015287134524	-1.00200400801603\\
-0.649298597194389	-0.988183911459061\\
-0.651267713759763	-0.985971943887776\\
-0.664679843843466	-0.969939879759519\\
-0.665330661322646	-0.969108182787294\\
-0.677271673230838	-0.953907815631263\\
-0.681362725450902	-0.948301328666419\\
-0.689008927968801	-0.937875751503006\\
-0.697394789579158	-0.925489634177226\\
-0.699878682271941	-0.92184368737475\\
-0.709908422007189	-0.905811623246493\\
-0.713426853707415	-0.899626883299733\\
-0.719077010649911	-0.889779559118236\\
-0.727357330207911	-0.87374749498998\\
-0.729458917835671	-0.869171274521562\\
-0.734777388626289	-0.857715430861723\\
-0.74129062152629	-0.841683366733467\\
-0.745490981963928	-0.829615259009827\\
-0.746888958483518	-0.82565130260521\\
-0.751589367884652	-0.809619238476954\\
-0.755335255446113	-0.793587174348697\\
-0.758115195531997	-0.777555110220441\\
-0.75991303480034	-0.761523046092184\\
-0.760707743337056	-0.745490981963928\\
-0.760473227940785	-0.729458917835671\\
-0.759178105440824	-0.713426853707415\\
-0.756785433485468	-0.697394789579158\\
-0.753252395755858	-0.681362725450902\\
-0.748529938033544	-0.665330661322646\\
-0.745490981963928	-0.657097361978569\\
-0.7425861964364	-0.649298597194389\\
-0.735366697548476	-0.633266533066132\\
-0.729458917835671	-0.622173078926685\\
-0.726793740715446	-0.617234468937876\\
-0.716817180348186	-0.601202404809619\\
-0.713426853707415	-0.596402327754499\\
-0.705353384906925	-0.585170340681363\\
-0.697394789579158	-0.575315823152729\\
-0.6923048730225	-0.569138276553106\\
-0.681362725450902	-0.557160252170715\\
-0.677574870872246	-0.55310621242485\\
-0.665330661322646	-0.541153296786094\\
-0.661045960551464	-0.537074148296593\\
-0.649298597194389	-0.526774239209995\\
-0.642577535313855	-0.521042084168337\\
-0.633266533066132	-0.513666478720857\\
-0.622002743472563	-0.50501002004008\\
-0.617234468937876	-0.501580692059362\\
-0.601202404809619	-0.490388740765583\\
-0.599127651236604	-0.488977955911824\\
-0.585170340681363	-0.480027122785411\\
-0.573719072526812	-0.472945891783567\\
-0.569138276553106	-0.470257971237539\\
-0.55310621242485	-0.461116772549802\\
-0.54548384496252	-0.456913827655311\\
-0.537074148296593	-0.452485119346148\\
-0.521042084168337	-0.444322192492087\\
-0.514062920771755	-0.440881763527054\\
-0.50501002004008	-0.436594076710081\\
-0.488977955911824	-0.429256628021445\\
-0.478997027818847	-0.424849699398798\\
-0.472945891783567	-0.422268390462778\\
-0.456913827655311	-0.41563433480124\\
-0.440881763527054	-0.409277744450356\\
-0.439691097851732	-0.408817635270541\\
-0.424849699398798	-0.403243772915913\\
-0.408817635270541	-0.397463772669068\\
-0.39531641265975	-0.392785571142285\\
-0.392785571142285	-0.391929123729416\\
-0.376753507014028	-0.386649989272733\\
-0.360721442885771	-0.38158845229712\\
-0.34474506445297	-0.376753507014028\\
-0.344689378757515	-0.376736961297139\\
-0.328657314629258	-0.372095089431188\\
-0.312625250501002	-0.367645025904362\\
-0.296593186372745	-0.363380351221717\\
-0.286214129254117	-0.360721442885771\\
-0.280561122244489	-0.359291512439729\\
-0.264529058116232	-0.355368027561503\\
-0.248496993987976	-0.35160972239275\\
-0.232464929859719	-0.348011343811235\\
-0.217006209675089	-0.344689378757515\\
-0.216432865731463	-0.344567065477537\\
-0.200400801603207	-0.341251153547249\\
-0.18436873747495	-0.338079269540548\\
-0.168336673346694	-0.335047087794837\\
-0.152304609218437	-0.33215045382108\\
-0.13627254509018	-0.329385377688802\\
-0.131884743099851	-0.328657314629258\\
-0.120240480961924	-0.326726984909464\\
-0.104208416833667	-0.324184439195626\\
-0.0881763527054109	-0.321762074682888\\
-0.0721442885771544	-0.319456592846781\\
-0.0561122244488979	-0.317264826827548\\
-0.0400801603206413	-0.315183736653393\\
-0.0240480961923848	-0.3132104046787\\
-0.019061007132892	-0.312625250501002\\
-0.00801603206412826	-0.311321845907351\\
0.00801603206412782	-0.309526948499704\\
0.0240480961923843	-0.307832210973503\\
0.0400801603206409	-0.306235226401617\\
0.0561122244488974	-0.304733688075022\\
0.0721442885771539	-0.303325386196306\\
0.0881763527054105	-0.302008204724655\\
0.104208416833667	-0.300780118366136\\
0.120240480961924	-0.299639189703393\\
0.13627254509018	-0.2985835664592\\
0.152304609218437	-0.297611478888563\\
0.168336673346693	-0.296721237294368\\
0.170856830652408	-0.296593186372745\\
0.18436873747495	-0.295896342428621\\
0.200400801603206	-0.29514862137764\\
0.216432865731463	-0.29447940913472\\
0.232464929859719	-0.293887356478515\\
0.248496993987976	-0.293371182970103\\
0.264529058116232	-0.292929675202511\\
0.280561122244489	-0.292561685142857\\
0.296593186372745	-0.29226612856404\\
0.312625250501002	-0.292041983563091\\
0.328657314629258	-0.291888289163494\\
0.344689378757515	-0.291804143998893\\
0.360721442885771	-0.291788705075809\\
0.376753507014028	-0.291841186613088\\
0.392785571142285	-0.291960858955994\\
0.408817635270541	-0.29214704756295\\
0.424849699398798	-0.292399132063091\\
0.440881763527054	-0.29271654538292\\
0.456913827655311	-0.293098772940476\\
0.472945891783567	-0.293545351905534\\
0.488977955911824	-0.294055870524498\\
0.50501002004008	-0.29462996750874\\
0.521042084168337	-0.295267331485259\\
0.537074148296593	-0.295967700508638\\
0.550211521056733	-0.296593186372745\\
0.55310621242485	-0.296726994183804\\
0.569138276553106	-0.297529313395123\\
0.585170340681363	-0.29839188748884\\
0.601202404809619	-0.299314599589349\\
0.617234468937876	-0.300297379624081\\
0.633266533066132	-0.30134020411989\\
0.649298597194389	-0.302443096055604\\
0.665330661322646	-0.303606124770402\\
0.681362725450902	-0.304829405927759\\
0.697394789579159	-0.306113101534817\\
0.713426853707415	-0.3074574200171\\
0.729458917835672	-0.30886261634862\\
0.745490981963928	-0.310328992237489\\
0.761523046092185	-0.311856896367243\\
0.769278854878664	-0.312625250501002\\
0.777555110220441	-0.313421989826799\\
0.793587174348698	-0.315023751219481\\
0.809619238476954	-0.316685530207057\\
0.825651302605211	-0.318407801497733\\
0.841683366733467	-0.320191086914323\\
0.857715430861724	-0.322035955893901\\
0.87374749498998	-0.323943026044923\\
0.889779559118236	-0.325912963762634\\
0.905811623246493	-0.327946484903709\\
0.911257105807711	-0.328657314629258\\
0.921843687374749	-0.330001745616119\\
0.937875751503006	-0.332097354319755\\
0.953907815631262	-0.334255743152237\\
0.969939879759519	-0.33647775460789\\
0.985971943887775	-0.338764282425752\\
1.00200400801603	-0.341116272540041\\
1.01803607214429	-0.343534724095772\\
1.02550251437477	-0.344689378757515\\
1.03406813627254	-0.345979719056959\\
1.0501002004008	-0.348454618968539\\
1.06613226452906	-0.350995716684761\\
1.08216432865731	-0.353604144653186\\
1.09819639278557	-0.356281091623539\\
1.11422845691383	-0.359027803969216\\
1.12388812329667	-0.360721442885771\\
1.13026052104208	-0.361811239910152\\
1.14629258517034	-0.36461204200786\\
1.1623246492986	-0.367482817798886\\
1.17835671342685	-0.370424956932151\\
1.19438877755511	-0.373439910961293\\
1.21042084168337	-0.376529195009296\\
1.21156465935191	-0.376753507014028\\
1.22645290581162	-0.379605185426995\\
1.24248496993988	-0.382747864695452\\
1.25851703406814	-0.385965580318528\\
1.27454909819639	-0.389260005384335\\
1.29058116232465	-0.392632882170167\\
1.29129556431976	-0.392785571142285\\
1.30661322645291	-0.39598756472163\\
1.32264529058116	-0.399415491732901\\
1.33867735470942	-0.402923017415068\\
1.35470941883768	-0.406512057240299\\
1.36480683124592	-0.408817635270541\\
1.37074148296593	-0.410144893237664\\
1.38677354709419	-0.413792407269475\\
1.40280561122244	-0.417522950337495\\
1.4188376753507	-0.421338624672566\\
1.43327341753146	-0.424849699398798\\
1.43486973947896	-0.425230538176461\\
1.45090180360721	-0.429109039478898\\
1.46693386773547	-0.433074580783541\\
1.48296593186373	-0.43712946706133\\
1.49748716150505	-0.440881763527054\\
1.49899799599198	-0.441265260293378\\
1.51503006012024	-0.445388306405733\\
1.5310621242485	-0.449603032405745\\
1.54709418837675	-0.453911964666915\\
1.55805460716506	-0.456913827655311\\
1.56312625250501	-0.458280288087583\\
1.57915831663327	-0.462663763041387\\
1.59519038076152	-0.46714424478626\\
1.61122244488978	-0.471724502761668\\
1.61543886050251	-0.472945891783567\\
1.62725450901804	-0.476317896356828\\
1.64328657314629	-0.480980219928278\\
1.65931863727455	-0.485745633239383\\
1.6699978232689	-0.488977955911824\\
1.67535070140281	-0.490576595612283\\
1.69138276553106	-0.495432182727003\\
1.70741482965932	-0.50039456423516\\
1.72201847542196	-0.50501002004008\\
1.72344689378758	-0.505456155064339\\
1.73947895791583	-0.510517648117318\\
1.75551102204409	-0.515690061468157\\
1.77154308617235	-0.520976803725175\\
1.77173917924778	-0.521042084168337\\
1.7875751503006	-0.52625964068516\\
1.80360721442886	-0.531656510205004\\
1.81935701604353	-0.537074148296593\\
1.81963927855711	-0.537170398342595\\
1.83567134268537	-0.542687057035592\\
1.85170340681363	-0.548324293921871\\
1.86503918983753	-0.55310621242485\\
1.86773547094188	-0.554066157580682\\
1.88376753507014	-0.55983511046815\\
1.8997995991984	-0.565730257147557\\
1.90892119230688	-0.569138276553106\\
1.91583166332665	-0.571705871094049\\
1.93186372745491	-0.577745866910482\\
1.94789579158317	-0.583918261763007\\
1.95111341242887	-0.585170340681363\\
1.96392785571142	-0.59013716893203\\
1.97995991983968	-0.596468882960891\\
1.99173198400165	-0.601202404809619\\
1.99599198396794	-0.602911487611387\\
2.01202404809619	-0.609415718935836\\
2.02805611222445	-0.616061959081576\\
2.03085688089103	-0.617234468937876\\
2.04408817635271	-0.622770155692562\\
2.06012024048096	-0.62960608223014\\
2.06857953022466	-0.633266533066132\\
2.07615230460922	-0.636547048754015\\
2.09218436873747	-0.643588116950522\\
2.10494818785325	-0.649298597194389\\
2.10821643286573	-0.650765023227161\\
2.12424849699399	-0.658027843952168\\
2.14003067252478	-0.665330661322646\\
2.14028056112224	-0.665446833471186\\
2.1563126252505	-0.672949297390174\\
2.17234468937876	-0.680613337810843\\
2.17390029259931	-0.681362725450902\\
2.18837675350701	-0.688381207448194\\
2.20440881763527	-0.696308777722044\\
2.20658745843663	-0.697394789579158\\
2.22044088176353	-0.704357524444677\\
2.23647294589178	-0.712569909280306\\
2.23813363633999	-0.713426853707415\\
2.25250501002004	-0.72091803804414\\
2.2685370741483	-0.729438296608975\\
2.26857566186315	-0.729458917835671\\
2.28456913827655	-0.738109108066444\\
2.29797140124603	-0.745490981963928\\
2.30060120240481	-0.746960028463725\\
2.31663326653307	-0.755984526969413\\
2.32633122735752	-0.761523046092184\\
2.33266533066132	-0.765199153861344\\
2.34869739478958	-0.774606538513275\\
2.35367410577749	-0.777555110220441\\
2.36472945891784	-0.784224962105189\\
2.38002203526913	-0.793587174348697\\
2.38076152304609	-0.794049217719981\\
2.39679358717435	-0.804117445181521\\
2.40544364626548	-0.809619238476954\\
2.41282565130261	-0.814421490795985\\
2.42885771543086	-0.824969023010803\\
2.42988953076037	-0.82565130260521\\
2.44488977955912	-0.835818474938177\\
2.45344743962524	-0.841683366733467\\
2.46092184368737	-0.846947141018387\\
2.47606203396086	-0.857715430861723\\
2.47695390781563	-0.858368940783156\\
2.49298597194389	-0.87016513032894\\
2.49781669408265	-0.87374749498998\\
2.50901803607214	-0.882326139130336\\
2.51867725212272	-0.889779559118236\\
2.5250501002004	-0.894872266277838\\
2.5386526417773	-0.905811623246493\\
2.54108216432866	-0.9078410098254\\
2.55711422845691	-0.921294150547114\\
2.55776719946121	-0.92184368737475\\
2.57314629258517	-0.935324148951232\\
2.57604571760297	-0.937875751503006\\
2.58917835671343	-0.949951853634216\\
2.59346837110476	-0.953907815631263\\
2.60521042084168	-0.965260550449017\\
2.61004230988098	-0.969939879759519\\
2.62124248496994	-0.981354060946348\\
2.62577168398352	-0.985971943887776\\
2.6372745490982	-0.998362410503902\\
2.64065766796415	-1.00200400801603\\
2.65330661322645	-1.01644951121299\\
2.65469846778161	-1.01803607214429\\
2.66790732630689	-1.03406813627255\\
2.66933867735471	-1.03592872873387\\
2.68028655540202	-1.0501002004008\\
2.68537074148297	-1.05719586980094\\
2.69180826785138	-1.06613226452906\\
2.70140280561122	-1.08058413989984\\
2.70245876191266	-1.08216432865731\\
2.71228495540956	-1.09819639278557\\
2.71743486973948	-1.10746387167625\\
2.72122733225868	-1.11422845691383\\
2.72930038595183	-1.13026052104208\\
2.73346693386774	-1.13960259196637\\
2.73648414127856	-1.14629258517034\\
2.742783141836	-1.1623246492986\\
2.7481473822652	-1.17835671342685\\
2.74949899799599	-1.18326472763881\\
}--cycle;


\addplot[area legend,solid,fill=mycolor4,draw=black,forget plot]
table[row sep=crcr] {%
x	y\\
2.41282565130261	-1.1944120331923\\
2.41326088352019	-1.21042084168337\\
2.41282565130261	-1.218471264915\\
2.41238998269463	-1.22645290581162\\
2.4101647713995	-1.24248496993988\\
2.40652353882578	-1.25851703406814\\
2.40139743988524	-1.27454909819639\\
2.39679358717435	-1.28564874313426\\
2.39471093784373	-1.29058116232465\\
2.38637706757187	-1.30661322645291\\
2.38076152304609	-1.31562485081699\\
2.37628745276453	-1.32264529058116\\
2.36472945891784	-1.33814481650825\\
2.36432220385613	-1.33867735470942\\
2.35033359497007	-1.35470941883768\\
2.34869739478958	-1.35637445492644\\
2.33414917324921	-1.37074148296593\\
2.33266533066132	-1.37206065365589\\
2.31663326653307	-1.38587607029119\\
2.31555861166133	-1.38677354709419\\
2.30060120240481	-1.39817407092151\\
2.29429680229451	-1.40280561122244\\
2.28456913827655	-1.40939457770825\\
2.27006073937107	-1.4188376753507\\
2.2685370741483	-1.41976015518994\\
2.25250501002004	-1.42918240687263\\
2.24241320344335	-1.43486973947896\\
2.23647294589178	-1.43801170712916\\
2.22044088176353	-1.44619574012528\\
2.21083805405892	-1.45090180360721\\
2.20440881763527	-1.45388254748922\\
2.18837675350701	-1.46105205070422\\
2.17461414983657	-1.46693386773547\\
2.17234468937876	-1.4678580292417\\
2.1563126252505	-1.47418374300018\\
2.14028056112224	-1.48021650611815\\
2.13268465251362	-1.48296593186373\\
2.12424849699399	-1.48589699450008\\
2.10821643286573	-1.49125289543617\\
2.09218436873747	-1.49635976121213\\
2.08355653997585	-1.49899799599198\\
2.07615230460922	-1.50118597594021\\
2.06012024048096	-1.50574042838379\\
2.04408817635271	-1.51008027063698\\
2.02805611222445	-1.51421286296382\\
2.02476181321789	-1.51503006012024\\
2.01202404809619	-1.51810506114378\\
1.99599198396794	-1.52179898968587\\
1.97995991983968	-1.52531164748454\\
1.96392785571142	-1.5286488460696\\
1.95174954119182	-1.5310621242485\\
1.94789579158317	-1.53181014672284\\
1.93186372745491	-1.53479062969444\\
1.91583166332665	-1.53761541129451\\
1.8997995991984	-1.54028911755706\\
1.88376753507014	-1.54281618007427\\
1.86773547094188	-1.54520084363668\\
1.8542455607675	-1.54709418837675\\
1.85170340681363	-1.54744628218132\\
1.83567134268537	-1.549553845547\\
1.81963927855711	-1.55153263421603\\
1.80360721442886	-1.55338603843301\\
1.7875751503006	-1.55511730252261\\
1.77154308617235	-1.55672953014929\\
1.75551102204409	-1.55822568932676\\
1.73947895791583	-1.55960861718824\\
1.72344689378758	-1.56088102452811\\
1.70741482965932	-1.56204550012528\\
1.69138276553106	-1.56310451485748\\
1.69102089695952	-1.56312625250501\\
1.67535070140281	-1.56406458380315\\
1.65931863727455	-1.56492411401095\\
1.64328657314629	-1.56568505032161\\
1.62725450901804	-1.56634932646037\\
1.61122244488978	-1.56691877942678\\
1.59519038076152	-1.56739515230621\\
1.57915831663327	-1.56778009693581\\
1.56312625250501	-1.56807517643052\\
1.54709418837675	-1.56828186757416\\
1.5310621242485	-1.56840156308071\\
1.51503006012024	-1.56843557373026\\
1.49899799599198	-1.56838513038404\\
1.48296593186373	-1.56825138588266\\
1.46693386773547	-1.56803541683136\\
1.45090180360721	-1.56773822527598\\
1.43486973947896	-1.56736074027292\\
1.4188376753507	-1.5669038193564\\
1.40280561122244	-1.56636824990595\\
1.38677354709419	-1.56575475041687\\
1.37074148296593	-1.56506397167631\\
1.35470941883768	-1.56429649784729\\
1.33867735470942	-1.56345284746297\\
1.33298932686219	-1.56312625250501\\
1.32264529058116	-1.56254071009163\\
1.30661322645291	-1.56155867881555\\
1.29058116232465	-1.5605032713749\\
1.27454909819639	-1.55937481935918\\
1.25851703406814	-1.5581735923737\\
1.24248496993988	-1.556899798643\\
1.22645290581162	-1.55555358553389\\
1.21042084168337	-1.55413503999946\\
1.19438877755511	-1.55264418894497\\
1.17835671342685	-1.55108099951661\\
1.1623246492986	-1.54944537931385\\
1.14629258517034	-1.54773717652602\\
1.14050244419177	-1.54709418837675\\
1.13026052104208	-1.54597204029556\\
1.11422845691383	-1.54414480266571\\
1.09819639278557	-1.54224645423564\\
1.08216432865731	-1.54027667597855\\
1.06613226452906	-1.53823509000439\\
1.0501002004008	-1.53612125935353\\
1.03406813627254	-1.53393468771586\\
1.01803607214429	-1.53167481907508\\
1.01382046102815	-1.5310621242485\\
1.00200400801603	-1.52936570236384\\
0.985971943887775	-1.52699319599491\\
0.969939879759519	-1.52454805763158\\
0.953907815631262	-1.52202956394829\\
0.937875751503006	-1.51943693060621\\
0.921843687374749	-1.51676931154935\\
0.911668100350907	-1.51503006012024\\
0.905811623246493	-1.51403992668718\\
0.889779559118236	-1.51126048066167\\
0.87374749498998	-1.508406121586\\
0.857715430861724	-1.50547582989822\\
0.841683366733467	-1.50246852165183\\
0.825651302605211	-1.49938304744346\\
0.823693324244152	-1.49899799599198\\
0.809619238476954	-1.49625656666176\\
0.793587174348698	-1.49305783259855\\
0.777555110220441	-1.48978039039152\\
0.761523046092185	-1.48642290539057\\
0.745490981963928	-1.48298397335973\\
0.74540842866127	-1.48296593186373\\
0.729458917835672	-1.47950840178624\\
0.713426853707415	-1.47595207132862\\
0.697394789579159	-1.47231321962394\\
0.681362725450902	-1.46859024335391\\
0.674367737579212	-1.46693386773547\\
0.665330661322646	-1.4648078822834\\
0.649298597194389	-1.46096139623414\\
0.633266533066132	-1.45702927484691\\
0.617234468937876	-1.45300970223674\\
0.608984186037125	-1.45090180360721\\
0.601202404809619	-1.44892344898969\\
0.585170340681363	-1.44477288687452\\
0.569138276553106	-1.44053289553551\\
0.55310621242485	-1.43620143018105\\
0.548257697020157	-1.43486973947896\\
0.537074148296593	-1.43180843750419\\
0.521042084168337	-1.42733704033054\\
0.50501002004008	-1.42277168503033\\
0.491464664020823	-1.4188376753507\\
0.488977955911824	-1.41811666361491\\
0.472945891783567	-1.41340142041776\\
0.456913827655311	-1.40858935585315\\
0.440881763527054	-1.40367800176941\\
0.43807334068574	-1.40280561122244\\
0.424849699398798	-1.39869818591332\\
0.408817635270541	-1.39362500515067\\
0.392785571142285	-1.38844907789457\\
0.387670207982509	-1.38677354709419\\
0.376753507014028	-1.38319182220986\\
0.360721442885771	-1.37784139931886\\
0.344689378757515	-1.37238432183222\\
0.339928740916952	-1.37074148296593\\
0.328657314629258	-1.36683849647845\\
0.312625250501002	-1.36119283406111\\
0.296593186372745	-1.35543612040452\\
0.294592836813773	-1.35470941883768\\
0.280561122244489	-1.34958526612237\\
0.264529058116232	-1.34362431935362\\
0.251455697033697	-1.33867735470942\\
0.248496993987976	-1.33754974850892\\
0.232464929859719	-1.3313691301482\\
0.216432865731463	-1.32507060450178\\
0.210340196857588	-1.32264529058116\\
0.200400801603206	-1.31865301245155\\
0.18436873747495	-1.31211590035153\\
0.171102117311288	-1.30661322645291\\
0.168336673346693	-1.30545350554503\\
0.152304609218437	-1.29865931856487\\
0.13627254509018	-1.2917388643662\\
0.133618009302535	-1.29058116232465\\
0.120240480961924	-1.28467124090096\\
0.104208416833667	-1.27747083164865\\
0.0977822830573675	-1.27454909819639\\
0.0881763527054105	-1.2701156811123\\
0.0721442885771539	-1.26261307365439\\
0.0635072186360268	-1.25851703406814\\
0.0561122244488974	-1.25494942629569\\
0.0400801603206409	-1.24712030396316\\
0.0307146649084245	-1.24248496993988\\
0.0240480961923843	-1.23912100138591\\
0.00801603206412782	-1.23093873401041\\
-0.000662997498562104	-1.22645290581162\\
-0.00801603206412826	-1.22256941906224\\
-0.0240480961923848	-1.21400479077449\\
-0.0306836309712991	-1.21042084168337\\
-0.0400801603206413	-1.20522266748881\\
-0.0561122244488979	-1.19624356755267\\
-0.0593960086430056	-1.19438877755511\\
-0.0721442885771544	-1.18699587554817\\
-0.0868459723648438	-1.17835671342685\\
-0.0881763527054109	-1.17755185713768\\
-0.104208416833667	-1.16778907856004\\
-0.113082025777746	-1.1623246492986\\
-0.120240480961924	-1.15777445022654\\
-0.13627254509018	-1.14748448552463\\
-0.138117715539481	-1.14629258517034\\
-0.152304609218437	-1.13680774609472\\
-0.162003670496862	-1.13026052104208\\
-0.168336673346694	-1.12582195227613\\
-0.18436873747495	-1.11449265935341\\
-0.184740867663911	-1.11422845691383\\
-0.200400801603207	-1.10265065749405\\
-0.206381986178503	-1.09819639278557\\
-0.216432865731463	-1.09037449375997\\
-0.226918621806142	-1.08216432865731\\
-0.232464929859719	-1.0776091362507\\
-0.246370394463127	-1.06613226452906\\
-0.248496993987976	-1.06428372268909\\
-0.264529058116232	-1.0502971862968\\
-0.264754258959078	-1.0501002004008\\
-0.280561122244489	-1.03548267855444\\
-0.282086876342295	-1.03406813627255\\
-0.296593186372745	-1.01978275351228\\
-0.298364453769031	-1.01803607214429\\
-0.312625250501002	-1.00302134746505\\
-0.313591315547883	-1.00200400801603\\
-0.327773155677305	-0.985971943887776\\
-0.328657314629258	-0.984891084092181\\
-0.340913467221137	-0.969939879759519\\
-0.344689378757515	-0.964920093587791\\
-0.35300011432646	-0.953907815631263\\
-0.360721442885771	-0.942662381514549\\
-0.36402270556034	-0.937875751503006\\
-0.373982303581655	-0.92184368737475\\
-0.376753507014028	-0.916821551990484\\
-0.382869989962245	-0.905811623246493\\
-0.390657046479397	-0.889779559118236\\
-0.392785571142285	-0.884645081355689\\
-0.397344394371589	-0.87374749498998\\
-0.402899660884825	-0.857715430861723\\
-0.407295135297882	-0.841683366733467\\
-0.408817635270541	-0.834052866146859\\
-0.410515100346529	-0.82565130260521\\
-0.412523236606446	-0.809619238476954\\
-0.413277810287527	-0.793587174348697\\
-0.412736999110873	-0.777555110220441\\
-0.410851566026232	-0.761523046092184\\
-0.408817635270541	-0.751545015188845\\
-0.407567400632433	-0.745490981963928\\
-0.40282214403213	-0.729458917835671\\
-0.396534012782737	-0.713426853707415\\
-0.392785571142285	-0.705755014038625\\
-0.3886205523287	-0.697394789579158\\
-0.3789807512068	-0.681362725450902\\
-0.376753507014028	-0.678192758609439\\
-0.367496050202917	-0.665330661322646\\
-0.360721442885771	-0.65715892520002\\
-0.354027390335937	-0.649298597194389\\
-0.344689378757515	-0.639598989253283\\
-0.338409018830419	-0.633266533066132\\
-0.328657314629258	-0.6244362507873\\
-0.320440281061835	-0.617234468937876\\
-0.312625250501002	-0.611005670505287\\
-0.299879974069355	-0.601202404809619\\
-0.296593186372745	-0.598878776512167\\
-0.280561122244489	-0.587909185462565\\
-0.276416879740852	-0.585170340681363\\
-0.264529058116232	-0.577873170647169\\
-0.249677818857906	-0.569138276553106\\
-0.248496993987976	-0.568488014789815\\
-0.232464929859719	-0.559913871365218\\
-0.219143213614718	-0.55310621242485\\
-0.216432865731463	-0.551798745464736\\
-0.200400801603207	-0.544304123687071\\
-0.18436873747495	-0.537161078774964\\
-0.184167711397594	-0.537074148296593\\
-0.168336673346694	-0.530557873970571\\
-0.152304609218437	-0.524263122847855\\
-0.143769036042608	-0.521042084168337\\
-0.13627254509018	-0.518331478642862\\
-0.120240480961924	-0.512750762896531\\
-0.104208416833667	-0.507429060606794\\
-0.0966268539659956	-0.50501002004008\\
-0.0881763527054109	-0.502408619852534\\
-0.0721442885771544	-0.497667189434588\\
-0.0561122244488979	-0.493148227494581\\
-0.0405863179577039	-0.488977955911824\\
-0.0400801603206413	-0.488845948616097\\
-0.0240480961923848	-0.484806640316839\\
-0.00801603206412826	-0.480960917046551\\
0.00801603206412782	-0.477302416284002\\
0.0240480961923843	-0.473825044593785\\
0.0282758869397849	-0.472945891783567\\
0.0400801603206409	-0.470544902342808\\
0.0561122244488974	-0.467438405923978\\
0.0721442885771539	-0.464492145980639\\
0.0881763527054105	-0.4617012799587\\
0.104208416833667	-0.459061168874064\\
0.11799143774163	-0.456913827655311\\
0.120240480961924	-0.456568738451119\\
0.13627254509018	-0.454225165232603\\
0.152304609218437	-0.452017194653508\\
0.168336673346693	-0.449941121766218\\
0.18436873747495	-0.447993399426268\\
0.200400801603206	-0.446170632452145\\
0.216432865731463	-0.444469572062522\\
0.232464929859719	-0.442887110578218\\
0.248496993987976	-0.441420276376838\\
0.254839904216782	-0.440881763527054\\
0.264529058116232	-0.44006428957776\\
0.280561122244489	-0.438816565736166\\
0.296593186372745	-0.437675711661675\\
0.312625250501002	-0.436639374707837\\
0.328657314629258	-0.435705311925049\\
0.344689378757515	-0.43487138658552\\
0.360721442885771	-0.434135564880284\\
0.376753507014028	-0.433495912781277\\
0.392785571142285	-0.432950593061899\\
0.408817635270541	-0.432497862469803\\
0.424849699398798	-0.432136069046025\\
0.440881763527054	-0.431863649584893\\
0.456913827655311	-0.431679127229451\\
0.472945891783567	-0.431581109197438\\
0.488977955911824	-0.43156828463315\\
0.50501002004008	-0.431639422580778\\
0.521042084168337	-0.431793370075078\\
0.537074148296593	-0.432029050345455\\
0.55310621242485	-0.432345461129834\\
0.569138276553106	-0.432741673094842\\
0.585170340681363	-0.43321682835911\\
0.601202404809619	-0.433770139116678\\
0.617234468937876	-0.434400886357689\\
0.633266533066132	-0.435108418683769\\
0.649298597194389	-0.43589215121566\\
0.665330661322646	-0.436751564590858\\
0.681362725450902	-0.43768620404918\\
0.697394789579159	-0.438695678604374\\
0.713426853707415	-0.439779660300001\\
0.728681178369629	-0.440881763527054\\
0.729458917835672	-0.440937169356576\\
0.745490981963928	-0.442153318763858\\
0.761523046092185	-0.443441555337079\\
0.777555110220441	-0.444801730147066\\
0.793587174348698	-0.446233754542128\\
0.809619238476954	-0.447737599774391\\
0.825651302605211	-0.449313296703352\\
0.841683366733467	-0.450960935575881\\
0.857715430861724	-0.452680665882014\\
0.87374749498998	-0.454472696286032\\
0.889779559118236	-0.456337294632449\\
0.894553608532723	-0.456913827655311\\
0.905811623246493	-0.458255574250193\\
0.921843687374749	-0.460236855780158\\
0.937875751503006	-0.462289463998749\\
0.953907815631262	-0.464413833001113\\
0.969939879759519	-0.466610456045862\\
0.985971943887775	-0.468879885902795\\
1.00200400801603	-0.471222735275674\\
1.01344179367746	-0.472945891783567\\
1.01803607214429	-0.473629836995794\\
1.03406813627254	-0.476085219993804\\
1.0501002004008	-0.478613477791274\\
1.06613226452906	-0.481215391197015\\
1.08216432865731	-0.483891802526196\\
1.09819639278557	-0.486643616374195\\
1.11143698686226	-0.488977955911824\\
1.11422845691383	-0.489464925942375\\
1.13026052104208	-0.492328947662649\\
1.14629258517034	-0.49526842201185\\
1.1623246492986	-0.498284428988867\\
1.17835671342685	-0.501378113820429\\
1.19438877755511	-0.504550688109075\\
1.19666158689287	-0.50501002004008\\
1.21042084168337	-0.507765282786676\\
1.22645290581162	-0.511051987562624\\
1.24248496993988	-0.514418253819137\\
1.25851703406814	-0.517865481478526\\
1.27295292300342	-0.521042084168337\\
1.27454909819639	-0.521390648220157\\
1.29058116232465	-0.524956972769979\\
1.30661322645291	-0.528605343491076\\
1.32264529058116	-0.532337359583706\\
1.33867735470942	-0.536154694988058\\
1.34247018477345	-0.537074148296593\\
1.35470941883768	-0.540022915431811\\
1.37074148296593	-0.543965466537279\\
1.38677354709419	-0.547994996146761\\
1.40280561122244	-0.552113395778651\\
1.40660625313145	-0.55310621242485\\
1.4188376753507	-0.556286697672394\\
1.43486973947896	-0.560538211609669\\
1.45090180360721	-0.564880733897725\\
1.46629442008748	-0.569138276553106\\
1.46693386773547	-0.569314633162043\\
1.48296593186373	-0.573799295456785\\
1.49899799599198	-0.578377461317298\\
1.51503006012024	-0.583051421052198\\
1.52217881393339	-0.585170340681363\\
1.5310621242485	-0.587799961662693\\
1.54709418837675	-0.592626781928237\\
1.56312625250501	-0.597552446528822\\
1.57479196546853	-0.601202404809619\\
1.57915831663327	-0.602569122994809\\
1.59519038076152	-0.607659213253254\\
1.61122244488978	-0.612851617745033\\
1.6245056348381	-0.617234468937876\\
1.62725450901804	-0.618143441761016\\
1.64328657314629	-0.623513156528112\\
1.65931863727455	-0.62898910754938\\
1.67162193337843	-0.633266533066132\\
1.67535070140281	-0.634568005590206\\
1.69138276553106	-0.64023559388867\\
1.70741482965932	-0.646013830364938\\
1.71638845287396	-0.649298597194389\\
1.72344689378758	-0.651897144457747\\
1.73947895791583	-0.657882929259136\\
1.75551102204409	-0.663984307527431\\
1.75900677238974	-0.665330661322646\\
1.77154308617235	-0.670195408726707\\
1.7875751503006	-0.676521993147994\\
1.79964009850105	-0.681362725450902\\
1.80360721442886	-0.682969645030969\\
1.81963927855711	-0.689538732193097\\
1.83567134268537	-0.696231235837756\\
1.83842916989687	-0.697394789579158\\
1.85170340681363	-0.703059455290105\\
1.86773547094188	-0.710015809157416\\
1.87549259334845	-0.713426853707415\\
1.88376753507014	-0.717114872131775\\
1.8997995991984	-0.724355846190772\\
1.91092891221336	-0.729458917835671\\
1.91583166332665	-0.731742211993973\\
1.93186372745491	-0.739290486337262\\
1.94482527305674	-0.745490981963928\\
1.94789579158317	-0.746986117132391\\
1.96392785571142	-0.754866500527064\\
1.97725758555169	-0.761523046092184\\
1.97995991983968	-0.762899720742124\\
1.99599198396794	-0.771139398329192\\
2.0082913930886	-0.777555110220441\\
2.01202404809619	-0.779545949295855\\
2.02805611222445	-0.788174763724199\\
2.03798262666264	-0.793587174348697\\
2.04408817635271	-0.796999099555937\\
2.06012024048096	-0.80604987049655\\
2.06637825368167	-0.809619238476954\\
2.07615230460922	-0.815346754165839\\
2.09218436873747	-0.824855641896282\\
2.09351683163443	-0.82565130260521\\
2.10821643286573	-0.834692117546505\\
2.11945116239955	-0.841683366733467\\
2.12424849699399	-0.844767115880768\\
2.14028056112224	-0.855156877354648\\
2.14419875898696	-0.857715430861723\\
2.1563126252505	-0.865908840341086\\
2.16779421212622	-0.87374749498998\\
2.17234468937876	-0.87697566374317\\
2.18837675350701	-0.88842965493899\\
2.19025591605488	-0.889779559118236\\
2.20440881763527	-0.900375765880688\\
2.21161885594914	-0.905811623246493\\
2.22044088176353	-0.912768516950856\\
2.23188259140286	-0.92184368737475\\
2.23647294589178	-0.925666535702521\\
2.25106575231567	-0.937875751503006\\
2.25250501002004	-0.939145312433786\\
2.2685370741483	-0.953331433158947\\
2.26918669964694	-0.953907815631263\\
2.28456913827655	-0.96835320190281\\
2.28625494830906	-0.969939879759519\\
2.30060120240481	-0.98430013334678\\
2.30226958638494	-0.985971943887776\\
2.31663326653307	-1.00136046242101\\
2.31723401058276	-1.00200400801603\\
2.33115658722833	-1.01803607214429\\
2.33266533066132	-1.01991837472451\\
2.34403425171651	-1.03406813627255\\
2.34869739478958	-1.04040575234506\\
2.3558560061553	-1.0501002004008\\
2.36472945891784	-1.06333908290441\\
2.36661051646348	-1.06613226452906\\
2.37630718441797	-1.08216432865731\\
2.38076152304609	-1.09048478399791\\
2.38491915052376	-1.09819639278557\\
2.39243602172105	-1.11422845691383\\
2.39679358717435	-1.12517666542078\\
2.39883608345157	-1.13026052104208\\
2.40410967157619	-1.14629258517034\\
2.40821618922675	-1.1623246492986\\
2.41113150882929	-1.17835671342685\\
2.41282502851576	-1.19438877755511\\
2.41282565130261	-1.1944120331923\\
}--cycle;


\addplot[area legend,solid,fill=mycolor5,draw=black,forget plot]
table[row sep=crcr] {%
x	y\\
2.09218436873747	-1.10567286848571\\
2.09508519042322	-1.11422845691383\\
2.09901152018938	-1.13026052104208\\
2.10140263568148	-1.14629258517034\\
2.10219798781542	-1.1623246492986\\
2.10132639338832	-1.17835671342685\\
2.09870521785296	-1.19438877755511\\
2.09423943173906	-1.21042084168337\\
2.09218436873747	-1.21562694155649\\
2.08779413892026	-1.22645290581162\\
2.07924103374919	-1.24248496993988\\
2.07615230460922	-1.24715413927238\\
2.06838886651756	-1.25851703406814\\
2.06012024048096	-1.26857940123754\\
2.05503689524475	-1.27454909819639\\
2.04408817635271	-1.28556497201737\\
2.03890414838713	-1.29058116232465\\
2.02805611222445	-1.29977881334611\\
2.01964486467362	-1.30661322645291\\
2.01202404809619	-1.31213560218618\\
1.99683452838171	-1.32264529058116\\
1.99599198396794	-1.32317270329132\\
1.97995991983968	-1.33284853174456\\
1.96980362260772	-1.33867735470942\\
1.96392785571142	-1.3417698197269\\
1.94789579158317	-1.34985024739518\\
1.93776462611943	-1.35470941883768\\
1.93186372745491	-1.35733464784757\\
1.91583166332665	-1.36416343331286\\
1.8997995991984	-1.37061977610875\\
1.89948525997869	-1.37074148296593\\
1.88376753507014	-1.37644877421776\\
1.86773547094188	-1.3819557461782\\
1.85290908677067	-1.38677354709419\\
1.85170340681363	-1.38714411090433\\
1.83567134268537	-1.39187004317821\\
1.81963927855711	-1.39633309927848\\
1.80360721442886	-1.40054305968273\\
1.79451829070245	-1.40280561122244\\
1.7875751503006	-1.40445601674553\\
1.77154308617235	-1.4080741663668\\
1.75551102204409	-1.41147732270831\\
1.73947895791583	-1.4146728099976\\
1.72344689378758	-1.41766761367379\\
1.71679748030774	-1.4188376753507\\
1.70741482965932	-1.42042899283545\\
1.69138276553106	-1.42298135701756\\
1.67535070140281	-1.42535920491414\\
1.65931863727455	-1.42756784264001\\
1.64328657314629	-1.42961233192818\\
1.62725450901804	-1.43149750035521\\
1.61122244488978	-1.43322795103238\\
1.59519038076152	-1.43480807179102\\
1.59450725372271	-1.43486973947896\\
1.57915831663327	-1.43621966723951\\
1.56312625250501	-1.43749173458122\\
1.54709418837675	-1.43862881786799\\
1.5310621242485	-1.43963425533461\\
1.51503006012024	-1.44051122207863\\
1.49899799599198	-1.44126273596127\\
1.48296593186373	-1.44189166320002\\
1.46693386773547	-1.44240072366762\\
1.45090180360721	-1.44279249591098\\
1.43486973947896	-1.44306942190289\\
1.4188376753507	-1.44323381153878\\
1.40280561122244	-1.44328784688966\\
1.38677354709419	-1.4432335862222\\
1.37074148296593	-1.44307296779589\\
1.35470941883768	-1.44280781344681\\
1.33867735470942	-1.44243983196684\\
1.32264529058116	-1.44197062228679\\
1.30661322645291	-1.4414016764712\\
1.29058116232465	-1.44073438253225\\
1.27454909819639	-1.43997002706964\\
1.25851703406814	-1.43910979774291\\
1.24248496993988	-1.43815478558235\\
1.22645290581162	-1.43710598714405\\
1.21042084168337	-1.43596430651438\\
1.1962021864091	-1.43486973947896\\
1.19438877755511	-1.43472981849443\\
1.17835671342685	-1.43339831946179\\
1.1623246492986	-1.4319768366409\\
1.14629258517034	-1.43046602771224\\
1.13026052104208	-1.42886646335446\\
1.11422845691383	-1.42717862866727\\
1.09819639278557	-1.42540292446332\\
1.08216432865731	-1.42353966843236\\
1.06613226452906	-1.42158909618104\\
1.0501002004008	-1.41955136215093\\
1.04471854719022	-1.4188376753507\\
1.03406813627254	-1.41742097103776\\
1.01803607214429	-1.41520184568288\\
1.00200400801603	-1.41289711100716\\
0.985971943887775	-1.41050669422552\\
0.969939879759519	-1.40803044336918\\
0.953907815631262	-1.40546812770269\\
0.937875751503006	-1.40281943802806\\
0.937794679407509	-1.40280561122244\\
0.921843687374749	-1.40007347421786\\
0.905811623246493	-1.39724221530196\\
0.889779559118236	-1.39432526218463\\
0.87374749498998	-1.39132207800237\\
0.857715430861724	-1.38823204728701\\
0.850351671068306	-1.38677354709419\\
0.841683366733467	-1.38504652776798\\
0.825651302605211	-1.38176765492821\\
0.809619238476954	-1.37840202353518\\
0.793587174348698	-1.37494878329855\\
0.777555110220441	-1.37140700379033\\
0.774611822588807	-1.37074148296593\\
0.761523046092185	-1.36775967857819\\
0.745490981963928	-1.36402049992909\\
0.729458917835672	-1.36019224946872\\
0.713426853707415	-1.35627374997196\\
0.707162379917998	-1.35470941883768\\
0.697394789579159	-1.35224757693566\\
0.681362725450902	-1.34812045864733\\
0.665330661322646	-1.3439020697555\\
0.649298597194389	-1.33959097235373\\
0.645965103109385	-1.33867735470942\\
0.633266533066132	-1.33515823589323\\
0.617234468937876	-1.33062529245488\\
0.601202404809619	-1.32599809596458\\
0.589809981190751	-1.32264529058116\\
0.585170340681363	-1.32126181422964\\
0.569138276553106	-1.31639756505913\\
0.55310621242485	-1.3114371565239\\
0.537814784687355	-1.30661322645291\\
0.537074148296593	-1.30637600654721\\
0.521042084168337	-1.30116144875821\\
0.50501002004008	-1.2958484433054\\
0.489409488711731	-1.29058116232465\\
0.488977955911824	-1.29043292066553\\
0.472945891783567	-1.28484660661955\\
0.456913827655311	-1.27915915019451\\
0.444139390209144	-1.27454909819639\\
0.440881763527054	-1.27335039776719\\
0.424849699398798	-1.26736821182332\\
0.408817635270541	-1.26128175329961\\
0.401641898828282	-1.25851703406814\\
0.392785571142285	-1.2550301242247\\
0.376753507014028	-1.2486250260597\\
0.3616352308178	-1.24248496993988\\
0.360721442885771	-1.24210476902355\\
0.344689378757515	-1.23535535479544\\
0.328657314629258	-1.22849708563115\\
0.323938616036272	-1.22645290581162\\
0.312625250501002	-1.22142023971341\\
0.296593186372745	-1.21418817009505\\
0.288350089730741	-1.21042084168337\\
0.280561122244489	-1.20675559574975\\
0.264529058116232	-1.19911806862703\\
0.254728910075435	-1.19438877755511\\
0.248496993987976	-1.19128361312176\\
0.232464929859719	-1.18320540614147\\
0.222958426228464	-1.17835671342685\\
0.216432865731463	-1.17491007964228\\
0.200400801603206	-1.1663519318892\\
0.192938077137183	-1.1623246492986\\
0.18436873747495	-1.15752100976468\\
0.168336673346693	-1.14843907553633\\
0.164581952704362	-1.14629258517034\\
0.152304609218437	-1.13897837162958\\
0.137822138954192	-1.13026052104208\\
0.13627254509018	-1.12928467588632\\
0.120240480961924	-1.11911462795315\\
0.112605196849132	-1.11422845691383\\
0.104208416833667	-1.1085863397909\\
0.0888686826186971	-1.09819639278557\\
0.0881763527054105	-1.09770180970072\\
0.0721442885771539	-1.08618339594694\\
0.0665872553649201	-1.08216432865731\\
0.0561122244488974	-1.07413998901423\\
0.0457171916629198	-1.06613226452906\\
0.0400801603206409	-1.06150836973367\\
0.0262337159092909	-1.0501002004008\\
0.0240480961923843	-1.04817141295368\\
0.00811870686361965	-1.03406813627255\\
0.00801603206412782	-1.03397012214328\\
-0.00801603206412826	-1.01863415828148\\
-0.00863999558139163	-1.01803607214429\\
-0.0240480961923848	-1.00201015477768\\
-0.024054000892589	-1.00200400801603\\
-0.0381237820841745	-0.985971943887776\\
-0.0400801603206413	-0.983504221357494\\
-0.0508492833704685	-0.969939879759519\\
-0.0561122244488979	-0.962512068696012\\
-0.0622246701725659	-0.953907815631263\\
-0.0721442885771544	-0.938027305240611\\
-0.0722393141280479	-0.937875751503006\\
-0.0808636975989025	-0.92184368737475\\
-0.0880890601349903	-0.905811623246493\\
-0.0881763527054109	-0.905567686765598\\
-0.0938670366246332	-0.889779559118236\\
-0.0981715551217002	-0.87374749498998\\
-0.100954086482767	-0.857715430861723\\
-0.102156079464795	-0.841683366733467\\
-0.101708480429144	-0.82565130260521\\
-0.0995309361207774	-0.809619238476954\\
-0.0955308721163357	-0.793587174348697\\
-0.089602432605782	-0.777555110220441\\
-0.0881763527054109	-0.774636278117961\\
-0.0815822672860885	-0.761523046092184\\
-0.0721442885771544	-0.746715082285994\\
-0.0713387069258798	-0.745490981963928\\
-0.0586302936601091	-0.729458917835671\\
-0.0561122244488979	-0.726755826954189\\
-0.0432202494941598	-0.713426853707415\\
-0.0400801603206413	-0.710600653217057\\
-0.0247928655396581	-0.697394789579158\\
-0.0240480961923848	-0.696823997434362\\
-0.00801603206412826	-0.684983262560185\\
-0.00288930400450903	-0.681362725450902\\
0.00801603206412782	-0.674411695459314\\
0.0229904532100606	-0.665330661322646\\
0.0240480961923843	-0.664744560799811\\
0.0400801603206409	-0.656181537372267\\
0.0536808991181782	-0.649298597194389\\
0.0561122244488974	-0.648160749693168\\
0.0721442885771539	-0.640945268542907\\
0.0881763527054105	-0.634123164957004\\
0.0902733750976727	-0.633266533066132\\
0.104208416833667	-0.627941834446976\\
0.120240480961924	-0.622134281935063\\
0.134526763000295	-0.617234468937876\\
0.13627254509018	-0.616669444625833\\
0.152304609218437	-0.611691804749125\\
0.168336673346693	-0.606988989029966\\
0.18436873747495	-0.602550581769417\\
0.189482960928257	-0.601202404809619\\
0.200400801603206	-0.598459744295848\\
0.216432865731463	-0.594645753504524\\
0.232464929859719	-0.591055323271024\\
0.248496993987976	-0.587680685329178\\
0.261173531237963	-0.585170340681363\\
0.264529058116232	-0.584531582846034\\
0.280561122244489	-0.581642070588381\\
0.296593186372745	-0.5789388692402\\
0.312625250501002	-0.576416096318473\\
0.328657314629258	-0.574068139485242\\
0.344689378757515	-0.571889644930021\\
0.360721442885771	-0.569875506363805\\
0.367053951674671	-0.569138276553106\\
0.376753507014028	-0.568041601713055\\
0.392785571142285	-0.566371903808447\\
0.408817635270541	-0.564848255554224\\
0.424849699398798	-0.563466762898583\\
0.440881763527054	-0.562223716440298\\
0.456913827655311	-0.561115584391578\\
0.472945891783567	-0.560139005905502\\
0.488977955911824	-0.559290784750153\\
0.50501002004008	-0.558567883312621\\
0.521042084168337	-0.557967416917027\\
0.537074148296593	-0.557486648441686\\
0.55310621242485	-0.55712298322137\\
0.569138276553106	-0.55687396422148\\
0.585170340681363	-0.556737267471728\\
0.601202404809619	-0.556710697747625\\
0.617234468937876	-0.556792184488818\\
0.633266533066132	-0.556979777943927\\
0.649298597194389	-0.557271645532194\\
0.665330661322646	-0.557666068412793\\
0.681362725450902	-0.558161438253262\\
0.697394789579159	-0.558756254188987\\
0.713426853707415	-0.559449119966202\\
0.729458917835672	-0.560238741261424\\
0.745490981963928	-0.561123923170692\\
0.761523046092185	-0.562103567862411\\
0.777555110220441	-0.563176672387984\\
0.793587174348698	-0.564342326644838\\
0.809619238476954	-0.565599711486781\\
0.825651302605211	-0.566948096976986\\
0.841683366733467	-0.568386840779253\\
0.849555646886845	-0.569138276553106\\
0.857715430861724	-0.569918996122569\\
0.87374749498998	-0.571543372022469\\
0.889779559118236	-0.573255668037112\\
0.905811623246493	-0.575055479087695\\
0.921843687374749	-0.576942483460597\\
0.937875751503006	-0.578916441743935\\
0.953907815631262	-0.580977195886924\\
0.969939879759519	-0.583124668379364\\
0.984618307736987	-0.585170340681363\\
0.985971943887775	-0.585359641337331\\
1.00200400801603	-0.587689181662701\\
1.01803607214429	-0.590104031806347\\
1.03406813627254	-0.592604341198635\\
1.0501002004008	-0.595190338172136\\
1.06613226452906	-0.597862329639255\\
1.08216432865731	-0.600620700882336\\
1.08544376929241	-0.601202404809619\\
1.09819639278557	-0.60347508722451\\
1.11422845691383	-0.606417387106125\\
1.13026052104208	-0.609445533127704\\
1.14629258517034	-0.612560139588003\\
1.1623246492986	-0.615761899641344\\
1.16950699116701	-0.617234468937876\\
1.17835671342685	-0.619060335045803\\
1.19438877755511	-0.622452857955618\\
1.21042084168337	-0.625932598263086\\
1.22645290581162	-0.629500486475553\\
1.24248496993988	-0.633157533845779\\
1.24295235074792	-0.633266533066132\\
1.25851703406814	-0.636925160665581\\
1.27454909819639	-0.640782118296343\\
1.29058116232465	-0.644728929240567\\
1.30661322645291	-0.648766855332829\\
1.3086823926057	-0.649298597194389\\
1.32264529058116	-0.652921738071333\\
1.33867735470942	-0.657171342897224\\
1.35470941883768	-0.661513256603292\\
1.3685127951711	-0.665330661322646\\
1.37074148296593	-0.665954262862323\\
1.38677354709419	-0.670521664023878\\
1.40280561122244	-0.675182910940616\\
1.4188376753507	-0.679939724531795\\
1.42355007250442	-0.681362725450902\\
1.43486973947896	-0.684827468175974\\
1.45090180360721	-0.689825527316975\\
1.46693386773547	-0.694921257121425\\
1.4745831368206	-0.697394789579158\\
1.48296593186373	-0.7001481343476\\
1.49899799599198	-0.705502792361159\\
1.51503006012024	-0.710957620927677\\
1.52217100956465	-0.713426853707415\\
1.5310621242485	-0.716556208928773\\
1.54709418837675	-0.722289762543329\\
1.56312625250501	-0.728126409281456\\
1.56673450451015	-0.729458917835671\\
1.57915831663327	-0.734139059242001\\
1.59519038076152	-0.740276548896056\\
1.60858947317622	-0.745490981963928\\
1.61122244488978	-0.746538781354645\\
1.62725450901804	-0.753000869615141\\
1.64328657314629	-0.759570347657242\\
1.64798987358909	-0.761523046092184\\
1.65931863727455	-0.766343684312459\\
1.67535070140281	-0.773265045683007\\
1.68514949430844	-0.777555110220441\\
1.69138276553106	-0.780359487606194\\
1.70741482965932	-0.787661988437245\\
1.72023842077742	-0.793587174348697\\
1.72344689378758	-0.795114757206133\\
1.73947895791583	-0.802830915525119\\
1.75338755355464	-0.809619238476954\\
1.75551102204409	-0.8106902012215\\
1.77154308617235	-0.818856224756142\\
1.78471033858131	-0.82565130260521\\
1.7875751503006	-0.827183575978418\\
1.80360721442886	-0.835839856169307\\
1.81430439927532	-0.841683366733467\\
1.81963927855711	-0.844713240840753\\
1.83567134268537	-0.853904923506763\\
1.84225293983876	-0.857715430861723\\
1.85170340681363	-0.863422674929471\\
1.86773547094188	-0.873200319218414\\
1.86862594787249	-0.87374749498998\\
1.88376753507014	-0.883486263001294\\
1.89346587809421	-0.889779559118236\\
1.8997995991984	-0.894099684471699\\
1.91583166332665	-0.905116775530308\\
1.91683653147148	-0.905811623246493\\
1.93186372745491	-0.916777501905554\\
1.93876143098378	-0.92184368737475\\
1.94789579158317	-0.928958783402094\\
1.95928287751743	-0.937875751503006\\
1.96392785571142	-0.941754360800132\\
1.97842273045091	-0.953907815631263\\
1.97995991983968	-0.955290524021375\\
1.99599198396794	-0.969753931609795\\
1.99619752050633	-0.969939879759519\\
2.01202404809619	-0.985389814844478\\
2.01261932871509	-0.985971943887776\\
2.02769733802308	-1.00200400801603\\
2.02805611222445	-1.00242304353819\\
2.04143132968385	-1.01803607214429\\
2.04408817635271	-1.0214789442239\\
2.05382003633854	-1.03406813627255\\
2.06012024048096	-1.0432315667924\\
2.06485642186074	-1.0501002004008\\
2.07452547747668	-1.06613226452906\\
2.07615230460922	-1.06929667460594\\
2.08280211319757	-1.08216432865731\\
2.08966838695467	-1.09819639278557\\
2.09218436873747	-1.10567286848571\\
}--cycle;


\addplot[area legend,solid,fill=mycolor6,draw=black,forget plot]
table[row sep=crcr] {%
x	y\\
1.77154308617235	-1.0721102266786\\
1.77502055746097	-1.08216432865731\\
1.77844177658726	-1.09819639278557\\
1.77965348116069	-1.11422845691383\\
1.77851466364042	-1.13026052104208\\
1.77486164939408	-1.14629258517034\\
1.77154308617235	-1.15476738934711\\
1.76845273793864	-1.1623246492986\\
1.75900479156252	-1.17835671342685\\
1.75551102204409	-1.18286817483177\\
1.74612625175431	-1.19438877755511\\
1.73947895791583	-1.20094231177603\\
1.72931682228573	-1.21042084168337\\
1.72344689378758	-1.21500326639011\\
1.70786381730301	-1.22645290581162\\
1.70741482965932	-1.2267372151051\\
1.69138276553106	-1.23639023781014\\
1.68054847498192	-1.24248496993988\\
1.67535070140281	-1.24506780189911\\
1.65931863727455	-1.25258707448201\\
1.6457486241366	-1.25851703406814\\
1.64328657314629	-1.25948558866325\\
1.62725450901804	-1.26545062185029\\
1.61122244488978	-1.27099117627141\\
1.60017980288532	-1.27454909819639\\
1.59519038076152	-1.27602094387987\\
1.57915831663327	-1.28045877760142\\
1.56312625250501	-1.28456485030714\\
1.54709418837675	-1.2883534857289\\
1.5369025309163	-1.29058116232465\\
1.5310621242485	-1.29176781608567\\
1.51503006012024	-1.29478686292834\\
1.49899799599198	-1.29754791511177\\
1.48296593186373	-1.30006078107955\\
1.46693386773547	-1.30233473797016\\
1.45090180360721	-1.30437856037785\\
1.43486973947896	-1.30620054723866\\
1.43079218635156	-1.30661322645291\\
1.4188376753507	-1.30775612502958\\
1.40280561122244	-1.30909814099147\\
1.38677354709419	-1.31025121001171\\
1.37074148296593	-1.31122118712108\\
1.35470941883768	-1.31201361086374\\
1.33867735470942	-1.31263371804054\\
1.32264529058116	-1.31308645755796\\
1.30661322645291	-1.31337650343841\\
1.29058116232465	-1.31350826704328\\
1.27454909819639	-1.31348590855676\\
1.25851703406814	-1.31331334777461\\
1.24248496993988	-1.31299427423937\\
1.22645290581162	-1.31253215676006\\
1.21042084168337	-1.31193025235219\\
1.19438877755511	-1.31119161463112\\
1.17835671342685	-1.31031910168938\\
1.1623246492986	-1.30931538348692\\
1.14629258517034	-1.3081829487805\\
1.13026052104208	-1.30692411161731\\
1.12667286435337	-1.30661322645291\\
1.11422845691383	-1.3055073067735\\
1.09819639278557	-1.30395632299345\\
1.08216432865731	-1.30228323236812\\
1.06613226452906	-1.3004899569642\\
1.0501002004008	-1.29857825799062\\
1.03406813627254	-1.29654974099085\\
1.01803607214429	-1.29440586067607\\
1.00200400801603	-1.29214792541706\\
0.991423923378463	-1.29058116232465\\
0.985971943887775	-1.28975246241379\\
0.969939879759519	-1.28719585542846\\
0.953907815631262	-1.28452823734294\\
0.937875751503006	-1.28175055008268\\
0.921843687374749	-1.27886359852082\\
0.905811623246493	-1.27586805388045\\
0.899010267466145	-1.27454909819639\\
0.889779559118236	-1.27270882354873\\
0.87374749498998	-1.26940078903611\\
0.857715430861724	-1.2659863857223\\
0.841683366733467	-1.26246591950334\\
0.825651302605211	-1.25883956927493\\
0.824268667090418	-1.25851703406814\\
0.809619238476954	-1.25499696806724\\
0.793587174348698	-1.25103949609307\\
0.777555110220441	-1.24697774658552\\
0.761523046092185	-1.24281152256217\\
0.760299271388649	-1.24248496993988\\
0.745490981963928	-1.23840511701579\\
0.729458917835672	-1.23388448285015\\
0.713426853707415	-1.22926058984881\\
0.703911932554443	-1.22645290581162\\
0.697394789579159	-1.22446167516543\\
0.681362725450902	-1.21945521317976\\
0.665330661322646	-1.21434659787932\\
0.653258472140329	-1.21042084168337\\
0.649298597194389	-1.20908356249466\\
0.633266533066132	-1.20356057867699\\
0.617234468937876	-1.19793646949036\\
0.607306911094619	-1.19438877755511\\
0.601202404809619	-1.19211633320152\\
0.585170340681363	-1.18604205489679\\
0.569138276553106	-1.17986763837586\\
0.565281983581636	-1.17835671342685\\
0.55310621242485	-1.1733712637193\\
0.537074148296593	-1.16670647960584\\
0.526693059748413	-1.1623246492986\\
0.521042084168337	-1.15982209446124\\
0.50501002004008	-1.15261850536373\\
0.491118841082587	-1.14629258517034\\
0.488977955911824	-1.14526537101029\\
0.472945891783567	-1.13746750971767\\
0.458302350895042	-1.13026052104208\\
0.456913827655311	-1.12953721446541\\
0.440881763527054	-1.12108138625004\\
0.428024147484063	-1.11422845691383\\
0.424849699398798	-1.11242879196951\\
0.408817635270541	-1.10324164428022\\
0.400097530084993	-1.09819639278557\\
0.392785571142285	-1.0936721411077\\
0.376753507014028	-1.08366896429747\\
0.374364530238479	-1.08216432865731\\
0.360721442885771	-1.072920221105\\
0.35076912993378	-1.06613226452906\\
0.344689378757515	-1.06163629636559\\
0.329173772437068	-1.0501002004008\\
0.328657314629258	-1.04968008210953\\
0.312625250501002	-1.03656602726647\\
0.309584948400861	-1.03406813627255\\
0.296593186372745	-1.02228570905816\\
0.291921424710149	-1.01803607214429\\
0.280561122244489	-1.00648694464938\\
0.276159540283867	-1.00200400801603\\
0.264529058116232	-0.988562615182317\\
0.262288887279839	-0.985971943887776\\
0.25032813632135	-0.969939879759519\\
0.248496993987976	-0.967006034148695\\
0.240308494554366	-0.953907815631263\\
0.232464929859719	-0.938375985932599\\
0.232211602760479	-0.937875751503006\\
0.226177792267307	-0.92184368737475\\
0.222210803334747	-0.905811623246493\\
0.220427864100186	-0.889779559118236\\
0.220966312387605	-0.87374749498998\\
0.223985813918821	-0.857715430861723\\
0.229670956278014	-0.841683366733467\\
0.232464929859719	-0.836354314380001\\
0.238339292119335	-0.82565130260521\\
0.248496993987976	-0.811920967572632\\
0.250287729137527	-0.809619238476954\\
0.264529058116232	-0.795058235060404\\
0.266049456822378	-0.793587174348697\\
0.280561122244489	-0.781911492668015\\
0.286312604721455	-0.777555110220441\\
0.296593186372745	-0.770875723601641\\
0.311969790840067	-0.761523046092184\\
0.312625250501002	-0.76117306316883\\
0.328657314629258	-0.753037518690777\\
0.344668570889552	-0.745490981963928\\
0.344689378757515	-0.745482195991458\\
0.360721442885771	-0.739054050253235\\
0.376753507014028	-0.733080169482493\\
0.387160288689072	-0.729458917835671\\
0.392785571142285	-0.727673706829862\\
0.408817635270541	-0.722895866406056\\
0.424849699398798	-0.718469647909262\\
0.440881763527054	-0.714379445735153\\
0.444885663223949	-0.713426853707415\\
0.456913827655311	-0.710775126655478\\
0.472945891783567	-0.707509607248497\\
0.488977955911824	-0.70451502881278\\
0.50501002004008	-0.701780792558993\\
0.521042084168337	-0.699296879187987\\
0.534609617092513	-0.697394789579158\\
0.537074148296593	-0.697070011128681\\
0.55310621242485	-0.695152393404444\\
0.569138276553106	-0.693444648606806\\
0.585170340681363	-0.691939751322376\\
0.601202404809619	-0.690631051737844\\
0.617234468937876	-0.689512257192057\\
0.633266533066132	-0.68857741486601\\
0.649298597194389	-0.687820895536904\\
0.665330661322646	-0.687237378327868\\
0.681362725450902	-0.686821836390084\\
0.697394789579159	-0.686569523458648\\
0.713426853707415	-0.686475961227841\\
0.729458917835672	-0.686536927495419\\
0.745490981963928	-0.686748445029145\\
0.761523046092185	-0.687106771112224\\
0.777555110220441	-0.687608387727322\\
0.793587174348698	-0.688249992341792\\
0.809619238476954	-0.68902848925935\\
0.825651302605211	-0.689940981505906\\
0.841683366733467	-0.690984763219548\\
0.857715430861724	-0.692157312516784\\
0.87374749498998	-0.693456284809092\\
0.889779559118236	-0.694879506545683\\
0.905811623246493	-0.696424969360046\\
0.915124642206304	-0.697394789579158\\
0.921843687374749	-0.698112452904158\\
0.937875751503006	-0.699950819891786\\
0.953907815631262	-0.701907501981076\\
0.969939879759519	-0.703980872840242\\
0.985971943887775	-0.706169458891952\\
1.00200400801603	-0.708471934687128\\
1.01803607214429	-0.710887118613187\\
1.03406813627254	-0.713413968920696\\
1.03414617146505	-0.713426853707415\\
1.0501002004008	-0.71613106092931\\
1.06613226452906	-0.718958361398809\\
1.08216432865731	-0.72189465660204\\
1.09819639278557	-0.724939266506081\\
1.11422845691383	-0.728091642892111\\
1.12094193506921	-0.729458917835671\\
1.13026052104208	-0.731410929021805\\
1.14629258517034	-0.734880172013118\\
1.1623246492986	-0.738455108671292\\
1.17835671342685	-0.742135558122638\\
1.19256202578	-0.745490981963928\\
1.19438877755511	-0.745935834838272\\
1.21042084168337	-0.749954189985494\\
1.22645290581162	-0.754076355544717\\
1.24248496993988	-0.758302532824733\\
1.25440198651449	-0.761523046092184\\
1.25851703406814	-0.76267228962181\\
1.27454909819639	-0.767260437995753\\
1.29058116232465	-0.771951273180969\\
1.30661322645291	-0.776745398649928\\
1.30926041812473	-0.777555110220441\\
1.32264529058116	-0.781796450310688\\
1.33867735470942	-0.786979843087664\\
1.35470941883768	-0.792265596763768\\
1.35863734437963	-0.793587174348697\\
1.37074148296593	-0.797818684092763\\
1.38677354709419	-0.803526346225895\\
1.40280561122244	-0.809335525678854\\
1.40357419000261	-0.809619238476954\\
1.4188376753507	-0.815491683569161\\
1.43486973947896	-0.821759472719033\\
1.44466227100284	-0.82565130260521\\
1.45090180360721	-0.828245614888258\\
1.46693386773547	-0.835016021213044\\
1.48249623129406	-0.841683366733467\\
1.48296593186373	-0.841894743674899\\
1.49899799599198	-0.849218094732429\\
1.51503006012024	-0.856635567814858\\
1.51733130937105	-0.857715430861723\\
1.5310621242485	-0.864508259000508\\
1.54709418837675	-0.872531431584777\\
1.54949309514644	-0.87374749498998\\
1.56312625250501	-0.881066999502288\\
1.57915831663327	-0.889760657390084\\
1.57919276026862	-0.889779559118236\\
1.59519038076152	-0.899124005606392\\
1.60654018682017	-0.905811623246493\\
1.61122244488978	-0.908767007107385\\
1.62725450901804	-0.918973795839684\\
1.63172384070709	-0.92184368737475\\
1.64328657314629	-0.929846092401464\\
1.65481584443196	-0.937875751503006\\
1.65931863727455	-0.941283166991659\\
1.67535070140281	-0.953479953008631\\
1.67590937500125	-0.953907815631263\\
1.69138276553106	-0.966885754290475\\
1.69501003454858	-0.969939879759519\\
1.70741482965932	-0.981502519971857\\
1.71219680323581	-0.985971943887776\\
1.72344689378758	-0.997764505355332\\
1.72748516732378	-1.00200400801603\\
1.73947895791583	-1.01635451712816\\
1.74088385040135	-1.01803607214429\\
1.75235639714701	-1.03406813627255\\
1.75551102204409	-1.03937627610607\\
1.76189650642811	-1.0501002004008\\
1.76948412858085	-1.06613226452906\\
1.77154308617235	-1.0721102266786\\
}--cycle;


\addplot[area legend,solid,fill=mycolor7,draw=black,forget plot]
table[row sep=crcr] {%
x	y\\
1.32264529058116	-1.01481420576371\\
1.32501866867429	-1.01803607214429\\
1.33221297968758	-1.03406813627255\\
1.33456415974334	-1.0501002004008\\
1.33129917681869	-1.06613226452906\\
1.32264529058116	-1.08030644330251\\
1.32136785213086	-1.08216432865731\\
1.30661322645291	-1.09486823063941\\
1.30217937548689	-1.09819639278557\\
1.29058116232465	-1.10439088990932\\
1.27454909819639	-1.1117791900016\\
1.26841161699314	-1.11422845691383\\
1.25851703406814	-1.11732761479046\\
1.24248496993988	-1.12164185456327\\
1.22645290581162	-1.12525292094194\\
1.21042084168337	-1.12821759016199\\
1.19664701077551	-1.13026052104208\\
1.19438877755511	-1.13054135348882\\
1.17835671342685	-1.13211475681373\\
1.1623246492986	-1.13325770308625\\
1.14629258517034	-1.13399691902496\\
1.13026052104208	-1.13435678574574\\
1.11422845691383	-1.13435956681737\\
1.09819639278557	-1.13402561002529\\
1.08216432865731	-1.13337352628247\\
1.06613226452906	-1.13242034862097\\
1.0501002004008	-1.13118167377513\\
1.04038318924896	-1.13026052104208\\
1.03406813627254	-1.12959829137712\\
1.01803607214429	-1.12761637972566\\
1.00200400801603	-1.12536880185992\\
0.985971943887775	-1.12286929308801\\
0.969939879759519	-1.12013034730399\\
0.953907815631262	-1.11716332871859\\
0.939152183744384	-1.11422845691383\\
0.937875751503006	-1.11394677810905\\
0.921843687374749	-1.11013281553955\\
0.905811623246493	-1.10611422380678\\
0.889779559118236	-1.10190027103868\\
0.876321630526467	-1.09819639278557\\
0.87374749498998	-1.09740433995364\\
0.857715430861724	-1.09221743843024\\
0.841683366733467	-1.08686445333345\\
0.828048484541989	-1.08216432865731\\
0.825651302605211	-1.08123008048192\\
0.809619238476954	-1.07473854376882\\
0.793587174348698	-1.06811973523632\\
0.788922409749965	-1.06613226452906\\
0.777555110220441	-1.06058486725849\\
0.761523046092185	-1.0526330281957\\
0.756536548698519	-1.0501002004008\\
0.745490981963928	-1.04354645448769\\
0.729638653580571	-1.03406813627255\\
0.729458917835672	-1.0339386239208\\
0.713426853707415	-1.02218039790515\\
0.707813049885961	-1.01803607214429\\
0.697394789579159	-1.00842482021371\\
0.690469707070471	-1.00200400801603\\
0.681362725450902	-0.990730668831577\\
0.677533156861799	-0.985971943887776\\
0.669131236808249	-0.969939879759519\\
0.665454501237288	-0.953907815631263\\
0.667252873669515	-0.937875751503006\\
0.675472453167688	-0.92184368737475\\
0.681362725450902	-0.915649275347754\\
0.691990287609928	-0.905811623246493\\
0.697394789579159	-0.902399702802681\\
0.713426853707415	-0.893509701941831\\
0.721202157774371	-0.889779559118236\\
0.729458917835672	-0.886736645588108\\
0.745490981963928	-0.881610149886467\\
0.761523046092185	-0.877283397449329\\
0.777279616725969	-0.87374749498998\\
0.777555110220441	-0.87369655231642\\
0.793587174348698	-0.871201470790451\\
0.809619238476954	-0.869211883795724\\
0.825651302605211	-0.867693680051094\\
0.841683366733467	-0.866615894378375\\
0.857715430861724	-0.865950379505564\\
0.87374749498998	-0.86567151825472\\
0.889779559118236	-0.865755970468029\\
0.905811623246493	-0.866182449910425\\
0.921843687374749	-0.866931527118981\\
0.937875751503006	-0.867985454777064\\
0.953907815631262	-0.869328012697879\\
0.969939879759519	-0.870944369925834\\
0.985971943887775	-0.872820961819977\\
0.992900689121321	-0.87374749498998\\
1.00200400801603	-0.875093003738543\\
1.01803607214429	-0.877734769080017\\
1.03406813627254	-0.880616461971991\\
1.0501002004008	-0.883726382388331\\
1.06613226452906	-0.88705391235819\\
1.07844706114796	-0.889779559118236\\
1.08216432865731	-0.890693575445782\\
1.09819639278557	-0.894894884608223\\
1.11422845691383	-0.899289306369214\\
1.13026052104208	-0.903868401068178\\
1.13674841043891	-0.905811623246493\\
1.14629258517034	-0.909012102273602\\
1.1623246492986	-0.914591117044196\\
1.17835671342685	-0.920324356264484\\
1.18244555095261	-0.92184368737475\\
1.19438877755511	-0.926869141129584\\
1.21042084168337	-0.933768746833001\\
1.2197340008397	-0.937875751503006\\
1.22645290581162	-0.941287928403988\\
1.24248496993988	-0.949601579932366\\
1.25064285701888	-0.953907815631263\\
1.25851703406814	-0.95879656286959\\
1.27454909819639	-0.968853063825775\\
1.2762479006438	-0.969939879759519\\
1.29058116232465	-0.981025344277367\\
1.29693929168541	-0.985971943887776\\
1.30661322645291	-0.995506108734241\\
1.31317532319271	-1.00200400801603\\
1.32264529058116	-1.01481420576371\\
}--cycle;

\end{axis}
\end{tikzpicture}%
    \caption{$p(\vec{y} \given \vec{\mu}, \mat{\Sigma}, \mat{A}, \vec{b})$}
    \label{transformed_pdf}
  \end{subfigure}
  \caption{Affine transformation example.
    (\subref{transformed_2d_pdf}) shows the joint density over
    $\vec{x} = [x_1, x_2]\trans$; this is the same density as in
    Figure \ref{2d_examples}(\subref{2d_example_3}).
    (\subref{transformed_pdf}) shows the density of $\vec{y} =
    \mat{A}\vec{x} + \vec{b}$. The values of $\mat{A}$ and $\vec{b}$
    are given in the text.  The density of the transformed vector is
    another Gaussian.}
  \label{transformed_example}
\end{figure}

Figure \ref{transformed_example} illustrates an affine transformation
of the vector $\vec{x}$ with the joint distribution shown in Figure
\ref{2d_examples}(\subref{2d_example_3}), for the values
\begin{equation*}
  \mat{A}
  =
  \begin{bmatrix}
    \nicefrac{1}{5} & -\nicefrac{3}{5} \\
    \nicefrac{1}{2} & \nicefrac{3}{10}
  \end{bmatrix};
  \qquad
  \vec{b}
  =
  \begin{bmatrix}
    1 \\
    -1
  \end{bmatrix}.
\end{equation*}
The density has been rotated and translated, but remains a Gaussian.

\section*{Selecting parameters}

The $d$-dimensional multivariate Gaussian distribution is specified by
the parameters $\vec{\mu}$ and $\mat{\Sigma}$.  Without any further
restrictions, specifying $\vec{\mu}$ requires $d$ parameters and
specifying $\mat{\Sigma}$ requires a further $\binom{d}{2} = \frac{d(d
  -1)}{2}$.  The number of parameters therefore grows quadratically in
the dimension, which can sometimes cause difficulty.  For this reason,
we sometimes restrict the covariance matrix $\vec{\Sigma}$ in some way
to reduce the number of parameters.

Common choices are to set $\mat{\Sigma} = \diag \vec{\tau}$, where
$\vec{\tau}$ is a vector of marginal variances, and $\mat{\Sigma} =
\sigma^2 \mat{I}$, a constant diagonal matrix.  Both of these options
assume independence between the variables in $\vec{x}$.  The former
case is more flexible, allowing a different scale parameter for each
entry, whereas the latter assumes an equal marginal variance of
$\sigma^2$ for each variable.  Geometrically, the densities are
axis-aligned, as in Figure \ref{2d_examples}(\subref{2d_example_1}),
and in the latter case, the isoprobability contours are spherical
(also as in Figure \ref{2d_examples}(\subref{2d_example_1})).

\end{document}
