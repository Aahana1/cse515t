\documentclass{article}

\usepackage[T1]{fontenc}
\usepackage[osf]{libertine}
\usepackage[scaled=0.8]{beramono}
\usepackage[margin=1.5in]{geometry}
\usepackage{url}
\usepackage{booktabs}
\usepackage{amsmath}
\usepackage{amssymb}
\usepackage{nicefrac}
\usepackage{microtype}
\usepackage{subcaption}
\usepackage{bm}

\usepackage{sectsty}
\sectionfont{\large}
\subsectionfont{\normalsize}

\usepackage{titlesec}
\titlespacing{\section}{0pt}{10pt plus 2pt minus 2pt}{0pt plus 2pt minus 0pt}
\titlespacing{\subsection}{0pt}{5pt plus 2pt minus 2pt}{0pt plus 2pt minus 0pt}

\usepackage{pgfplots}
\pgfplotsset{
  compat=newest,
  plot coordinates/math parser=false,
  tick label style={font=\footnotesize, /pgf/number format/fixed},
  label style={font=\small},
  legend style={font=\small},
  every axis/.append style={
    tick align=outside,
    clip mode=individual,
    scaled ticks=false,
    thick,
    tick style={semithick, black}
  }
}

\pgfkeys{/pgf/number format/.cd, set thousands separator={\,}}

\usepgfplotslibrary{external}
\tikzexternalize[prefix=tikz/]

\newlength\figurewidth
\newlength\figureheight

\setlength{\figurewidth}{8cm}
\setlength{\figureheight}{6cm}

\newlength\squarefigurewidth
\newlength\squarefigureheight

\setlength{\squarefigurewidth}{4cm}
\setlength{\squarefigureheight}{4cm}

\newlength\smallsquarefigurewidth
\newlength\smallsquarefigureheight

\setlength{\smallsquarefigurewidth}{3.25cm}
\setlength{\smallsquarefigureheight}{3.25cm}

\newlength\smallfigurewidth
\newlength\smallfigureheight

\setlength{\smallfigurewidth}{6.25cm}
\setlength{\smallfigureheight}{4cm}

\setlength{\parindent}{0pt}
\setlength{\parskip}{1ex}

\newcommand{\acro}[1]{\textsc{\MakeLowercase{#1}}}
\newcommand{\given}{\mid}
\newcommand{\mc}[1]{\mathcal{#1}}
\newcommand{\data}{\mc{D}}
\newcommand{\intd}[1]{\,\mathrm{d}{#1}}
\newcommand{\inv}{^{-1}}
\newcommand{\trans}{^\top}
\newcommand{\mat}[1]{\bm{\mathrm{#1}}}
\renewcommand{\vec}[1]{\bm{\mathrm{#1}}}
\newcommand{\R}{\mathbb{R}}
\renewcommand{\epsilon}{\varepsilon}

\DeclareMathOperator{\var}{var}
\DeclareMathOperator{\cov}{cov}
\DeclareMathOperator{\diag}{diag}
\DeclareMathOperator*{\argmin}{arg\,min}
\DeclareMathOperator*{\argmax}{arg\,max}

\begin{document}

Until now we have always worked with likelihoods and prior
distributions that were conjugate to each other, allowing the
computation of the posterior distribution to be done in closed form.
Unfortunately, there are numerous situations where this will not be
the case, forcing us to approximate the posterior and related
quantities (such as the model evidence or expectations under the
posterior distribution).  Logistic regression is a common linear
method for binary classification, and attempting to use the Bayesian
approach directly will be intractable.

\section*{Logistic Regression}

In linear regression, we supposed that were interested in the values
of a real-valued function $y(\vec{x})\colon \R^d \to \R$, where
$\vec{x}$ is a $d$-dimensional vector-valued input.  Here, we will
consider a similar setup, but with a twist: we restrict the output of
the function $y$ to the discrete space $y \in \{0, 1\}$.  In machine
learning, problems of this form fall under the category of
\emph{binary classification:} given a input $\vec{x}$, we wish to
\emph{classify} it into one of two categories, in this case denoted
arbitrarily by $0$ and $1$.

We again assume that we have made some observations of this mapping,
$\data = \bigl\{ (\vec{x}_i, y_i) \bigr\}_{i = 1}^N$, to serve as
training data.  Given these examples, the goal of binary
classification is to be able to predict the label at a new input
location $\vec{x}_\ast$.

As in linear regression, the problem is not yet well-posed without
some restrictions on $y$.  In linear regression, we assumed that the
relationship between $\vec{x}$ and $y$ was ``mostly'' linear:
\begin{equation*}
  y(\vec{x}) = \vec{x}\trans \vec{w} + \epsilon(\vec{x}),
\end{equation*}
where $\vec{w} \in \R^d$ is a vector of parameters, and
$\epsilon(\vec{x})$ is the residual.  This assumption is not very
desirable in the classification case, where the outputs are restricted
to $\{0, 1\}$ (note, for example, that $\vec{x}\trans \vec{w}$ is
unbounded as the norm of $\vec{x}$ increases, forcing the residuals to
grow ever larger).

In linear classification methods, we instead assume that the
class-conditional probability of belonging to the ``$1$'' class is
given by a nonlinear transformation of an underlying linear function
of $\vec{x}$:
\begin{equation*}
  \Pr(y = 1 \given \vec{x}, \vec{w})
  =
  \sigma(\vec{x}\trans \vec{w}),
\end{equation*}
where $\sigma$ is a so-called ``sigmoid'' (``s-shaped'') increasing
function mapping the real line to valid probabilities in $(0, 1)$.
The most-commonly used functions $\sigma$ are the \emph{logistic function}
\begin{equation*}
  \sigma(a) = \frac{\exp(a)}{1 + \exp(a)},
\end{equation*}
or the standard normal cumulative distribution function:
\begin{equation*}
  \sigma(a) = \Phi(a) = \int_{-\infty}^a \mc{N}(x; 0, 1) \intd x.
\end{equation*}
These two choices are compared in Figure \ref{sigmoids}.  The main
difference is that the logistic function has slightly heavier tails
than the normal \acro{CDF}.  Linear classification using the logistic
function is called \emph{logistic regression}; linear classification
using the normal \acro{CDF} is called \emph{probit regression.}
Logistic regression is more commonly encountered in practice.  Notice
that the linear assumption above combined with the logistic function
sigmoid implies that the \emph{log odds} are a linear function of the
input $\vec{x}$ (verify this!):
\begin{equation*}
  \log
  \frac{\Pr(y = 1 \given x, \vec{w})}
       {\Pr(y = 0 \given x, \vec{w})}
  =
  \vec{x}\trans \vec{w}.
\end{equation*}

\begin{figure}
  \centering
  % This file was created by matlab2tikz.
% Minimal pgfplots version: 1.3
%
\tikzsetnextfilename{sigmoids}
\definecolor{mycolor1}{rgb}{0.12157,0.47059,0.70588}%
\definecolor{mycolor2}{rgb}{0.89020,0.10196,0.10980}%
%
\begin{tikzpicture}

\begin{axis}[%
width=0.95092\figurewidth,
height=\figureheight,
at={(0\figurewidth,0\figureheight)},
scale only axis,
xmin=-5,
xmax=5,
xlabel={$a$},
ymin=0,
ymax=1,
axis x line*=bottom,
axis y line*=left,
legend style={at={(0.03,0.97)},anchor=north west,legend cell align=left,align=left,draw=none}
]
\addplot [color=mycolor1,solid]
  table[row sep=crcr]{%
-5	0.00669285092428486\\
-4.98998998998999	0.00675972771792696\\
-4.97997997997998	0.00682726816967834\\
-4.96996996996997	0.00689547877317896\\
-4.95995995995996	0.00696436608375462\\
-4.94994994994995	0.00703393671896556\\
-4.93993993993994	0.0071041973591593\\
-4.92992992992993	0.00717515474802748\\
-4.91991991991992	0.00724681569316699\\
-4.90990990990991	0.0073191870666451\\
-4.8998998998999	0.00739227580556891\\
-4.88988988988989	0.00746608891265891\\
-4.87987987987988	0.00754063345682672\\
-4.86986986986987	0.00761591657375712\\
-4.85985985985986	0.00769194546649423\\
-4.84984984984985	0.00776872740603197\\
-4.83983983983984	0.00784626973190873\\
-4.82982982982983	0.00792457985280634\\
-4.81981981981982	0.00800366524715327\\
-4.80980980980981	0.00808353346373211\\
-4.7997997997998	0.00816419212229134\\
-4.78978978978979	0.00824564891416144\\
-4.77977977977978	0.00832791160287522\\
-4.76976976976977	0.00841098802479249\\
-4.75975975975976	0.00849488608972908\\
-4.74974974974975	0.00857961378159016\\
-4.73973973973974	0.00866517915900787\\
-4.72972972972973	0.00875159035598327\\
-4.71971971971972	0.00883885558253271\\
-4.70970970970971	0.00892698312533846\\
-4.6996996996997	0.00901598134840366\\
-4.68968968968969	0.00910585869371176\\
-4.67967967967968	0.00919662368189011\\
-4.66966966966967	0.00928828491287808\\
-4.65965965965966	0.00938085106659939\\
-4.64964964964965	0.00947433090363887\\
-4.63963963963964	0.00956873326592356\\
-4.62962962962963	0.00966406707740808\\
-4.61961961961962	0.00976034134476445\\
-4.60960960960961	0.00985756515807615\\
-4.5995995995996	0.00995574769153659\\
-4.58958958958959	0.0100548982041518\\
-4.57957957957958	0.0101550260404477\\
-4.56956956956957	0.0102561406311811\\
-4.55955955955956	0.0103582514940558\\
-4.54954954954955	0.0104613682344423\\
-4.53953953953954	0.0105655005461023\\
-4.52952952952953	0.0106706582119169\\
-4.51951951951952	0.0107768511046198\\
-4.50950950950951	0.0108840891875339\\
-4.4994994994995	0.0109923825153129\\
-4.48948948948949	0.0111017412346865\\
-4.47947947947948	0.0112121755852104\\
-4.46946946946947	0.0113236959000196\\
-4.45945945945946	0.0114363126065871\\
-4.44944944944945	0.0115500362274853\\
-4.43943943943944	0.0116648773811527\\
-4.42942942942943	0.0117808467826642\\
-4.41941941941942	0.0118979552445052\\
-4.40940940940941	0.0120162136773501\\
-4.3993993993994	0.012135633090845\\
-4.38938938938939	0.0122562245943938\\
-4.37937937937938	0.0123779993979484\\
-4.36936936936937	0.0125009688128034\\
-4.35935935935936	0.012625144252394\\
-4.34934934934935	0.0127505372330978\\
-4.33933933933934	0.0128771593750407\\
-4.32932932932933	0.0130050224029067\\
-4.31931931931932	0.0131341381467507\\
-4.30930930930931	0.0132645185428156\\
-4.2992992992993	0.0133961756343531\\
-4.28928928928929	0.0135291215724476\\
-4.27927927927928	0.013663368616844\\
-4.26926926926927	0.013798929136779\\
-4.25925925925926	0.0139358156118155\\
-4.24924924924925	0.0140740406326809\\
-4.23923923923924	0.0142136169021082\\
-4.22922922922923	0.0143545572356807\\
-4.21921921921922	0.0144968745626801\\
-4.20920920920921	0.0146405819269372\\
-4.1991991991992	0.0147856924876858\\
-4.18918918918919	0.0149322195204203\\
-4.17917917917918	0.0150801764177555\\
-4.16916916916917	0.0152295766902892\\
-4.15915915915916	0.0153804339674685\\
-4.14914914914915	0.0155327619984579\\
-4.13913913913914	0.0156865746530102\\
-4.12912912912913	0.0158418859223406\\
-4.11911911911912	0.0159987099200028\\
-4.10910910910911	0.0161570608827671\\
-4.0990990990991	0.0163169531715018\\
-4.08908908908909	0.0164784012720558\\
-4.07907907907908	0.0166414197961442\\
-4.06906906906907	0.0168060234822353\\
-4.05905905905906	0.0169722271964398\\
-4.04904904904905	0.0171400459334015\\
-4.03903903903904	0.0173094948171906\\
-4.02902902902903	0.0174805891021974\\
-4.01901901901902	0.0176533441740282\\
-4.00900900900901	0.0178277755504028\\
-3.998998998999	0.0180038988820526\\
-3.98898898898899	0.0181817299536202\\
-3.97897897897898	0.0183612846845599\\
-3.96896896896897	0.0185425791300397\\
-3.95895895895896	0.0187256294818426\\
-3.94894894894895	0.0189104520692704\\
-3.93893893893894	0.0190970633600463\\
-3.92892892892893	0.0192854799612188\\
-3.91891891891892	0.0194757186200659\\
-3.90890890890891	0.0196677962249984\\
-3.8988988988989	0.0198617298064638\\
-3.88888888888889	0.0200575365378504\\
-3.87887887887888	0.0202552337363896\\
-3.86886886886887	0.0204548388640589\\
-3.85885885885886	0.0206563695284833\\
-3.84884884884885	0.0208598434838364\\
-3.83883883883884	0.0210652786317398\\
-3.82882882882883	0.0212726930221616\\
-3.81881881881882	0.0214821048543132\\
-3.80880880880881	0.0216935324775447\\
-3.7987987987988	0.0219069943922381\\
-3.78878878878879	0.0221225092506983\\
-3.77877877877878	0.0223400958580428\\
-3.76876876876877	0.0225597731730876\\
-3.75875875875876	0.0227815603092312\\
-3.74874874874875	0.0230054765353355\\
-3.73873873873874	0.0232315412766038\\
-3.72872872872873	0.023459774115455\\
-3.71871871871872	0.0236901947923941\\
-3.70870870870871	0.0239228232068801\\
-3.6986986986987	0.0241576794181877\\
-3.68868868868869	0.0243947836462669\\
-3.67867867867868	0.0246341562725965\\
-3.66866866866867	0.0248758178410331\\
-3.65865865865866	0.0251197890586557\\
-3.64864864864865	0.0253660907966038\\
-3.63863863863864	0.0256147440909107\\
-3.62862862862863	0.0258657701433307\\
-3.61861861861862	0.0261191903221597\\
-3.60860860860861	0.0263750261630498\\
-3.5985985985986	0.0266332993698164\\
-3.58858858858859	0.0268940318152395\\
-3.57857857857858	0.0271572455418557\\
-3.56856856856857	0.0274229627627442\\
-3.55855855855856	0.0276912058623039\\
-3.54854854854855	0.0279619973970216\\
-3.53853853853854	0.0282353600962329\\
-3.52852852852853	0.0285113168628729\\
-3.51851851851852	0.0287898907742174\\
-3.50850850850851	0.0290711050826154\\
-3.4984984984985	0.0293549832162101\\
-3.48848848848849	0.0296415487796508\\
-3.47847847847848	0.0299308255547923\\
-3.46846846846847	0.0302228375013849\\
-3.45845845845846	0.0305176087577509\\
-3.44844844844845	0.0308151636414504\\
-3.43843843843844	0.0311155266499343\\
-3.42842842842843	0.031418722461184\\
-3.41841841841842	0.0317247759343382\\
-3.40840840840841	0.032033712110306\\
-3.3983983983984	0.0323455562123661\\
-3.38838838838839	0.0326603336467506\\
-3.37837837837838	0.0329780700032145\\
-3.36836836836837	0.0332987910555901\\
-3.35835835835836	0.033622522762324\\
-3.34834834834835	0.0339492912669992\\
-3.33833833833834	0.0342791228988399\\
-3.32832832832833	0.0346120441731981\\
-3.31831831831832	0.0349480817920235\\
-3.30830830830831	0.0352872626443147\\
-3.2982982982983	0.0356296138065508\\
-3.28828828828829	0.0359751625431048\\
-3.27827827827828	0.0363239363066362\\
-3.26826826826827	0.0366759627384633\\
-3.25825825825826	0.0370312696689153\\
-3.24824824824825	0.0373898851176616\\
-3.23823823823824	0.03775183729402\\
-3.22822822822823	0.0381171545972417\\
-3.21821821821822	0.0384858656167737\\
-3.20820820820821	0.0388579991324966\\
-3.1981981981982	0.0392335841149386\\
-3.18818818818819	0.0396126497254641\\
-3.17817817817818	0.0399952253164371\\
-3.16816816816817	0.0403813404313583\\
-3.15815815815816	0.0407710248049748\\
-3.14814814814815	0.0411643083633631\\
-3.13813813813814	0.0415612212239836\\
-3.12812812812813	0.0419617936957066\\
-3.11811811811812	0.0423660562788081\\
-3.10810810810811	0.0427740396649366\\
-3.0980980980981	0.0431857747370486\\
-3.08808808808809	0.0436012925693123\\
-3.07807807807808	0.04402062442698\\
-3.06806806806807	0.0444438017662267\\
-3.05805805805806	0.0448708562339563\\
-3.04804804804805	0.0453018196675723\\
-3.03803803803804	0.0457367240947143\\
-3.02802802802803	0.0461756017329587\\
-3.01801801801802	0.0466184849894827\\
-3.00800800800801	0.0470654064606912\\
-2.997997997998	0.0475163989318058\\
-2.98798798798799	0.0479714953764146\\
-2.97797797797798	0.0484307289559834\\
-2.96796796796797	0.0488941330193256\\
-2.95795795795796	0.0493617411020319\\
-2.94794794794795	0.049833586925857\\
-2.93793793793794	0.0503097043980649\\
-2.92792792792793	0.0507901276107296\\
-2.91791791791792	0.051274890839992\\
-2.90790790790791	0.0517640285452717\\
-2.8978978978979	0.0522575753684322\\
-2.88788788788789	0.0527555661329002\\
-2.87787787787788	0.0532580358427361\\
-2.86786786786787	0.0537650196816568\\
-2.85785785785786	0.0542765530120083\\
-2.84784784784785	0.0547926713736892\\
-2.83783783783784	0.0553134104830214\\
-2.82782782782783	0.05583880623157\\
-2.81781781781782	0.0563688946849095\\
-2.80780780780781	0.0569037120813367\\
-2.7977977977978	0.0574432948305285\\
-2.78778778778779	0.0579876795121434\\
-2.77777777777778	0.0585369028743679\\
-2.76776776776777	0.059091001832404\\
-2.75775775775776	0.0596500134668986\\
-2.74774774774775	0.0602139750223142\\
-2.73773773773774	0.060782923905238\\
-2.72772772772773	0.0613568976826309\\
-2.71771771771772	0.0619359340800139\\
-2.70770770770771	0.0625200709795908\\
-2.6976976976977	0.0631093464183075\\
-2.68768768768769	0.0637037985858462\\
-2.67767767767768	0.0643034658225527\\
-2.66766766766767	0.064908386617298\\
-2.65765765765766	0.0655185996052709\\
-2.64764764764765	0.0661341435657019\\
-2.63763763763764	0.0667550574195177\\
-2.62762762762763	0.0673813802269242\\
-2.61761761761762	0.0680131511849179\\
-2.60760760760761	0.0686504096247249\\
-2.5975975975976	0.0692931950091651\\
-2.58758758758759	0.0699415469299428\\
-2.57757757757758	0.070595505104861\\
-2.56756756756757	0.0712551093749586\\
-2.55755755755756	0.0719203997015712\\
-2.54754754754755	0.0725914161633118\\
-2.53753753753754	0.0732681989529729\\
-2.52752752752753	0.0739507883743479\\
-2.51751751751752	0.0746392248389705\\
-2.50750750750751	0.0753335488627725\\
-2.4974974974975	0.0760338010626572\\
-2.48748748748749	0.0767400221529896\\
-2.47747747747748	0.077452252942\\
-2.46746746746747	0.0781705343281028\\
-2.45745745745746	0.0788949072961272\\
-2.44744744744745	0.0796254129134599\\
-2.43743743743744	0.0803620923260998\\
-2.42742742742743	0.0811049867546216\\
-2.41741741741742	0.0818541374900494\\
-2.40740740740741	0.0826095858896389\\
-2.3973973973974	0.0833713733725662\\
-2.38738738738739	0.0841395414155247\\
-2.37737737737738	0.0849141315482261\\
-2.36736736736737	0.0856951853488082\\
-2.35735735735736	0.0864827444391454\\
-2.34734734734735	0.0872768504800637\\
-2.33733733733734	0.0880775451664576\\
-2.32732732732733	0.0888848702223091\\
-2.31731731731732	0.089698867395608\\
-2.30730730730731	0.0905195784531722\\
-2.2972972972973	0.091347045175368\\
-2.28728728728729	0.0921813093507289\\
-2.27727727727728	0.0930224127704736\\
-2.26726726726727	0.0938703972229201\\
-2.25725725725726	0.0947253044877986\\
-2.24724724724725	0.0955871763304594\\
-2.23723723723724	0.0964560544959775\\
-2.22722722722723	0.0973319807031517\\
-2.21721721721722	0.0982149966383994\\
-2.20720720720721	0.0991051439495445\\
-2.1971971971972	0.1000024642395\\
-2.18718718718719	0.100906999059845\\
-2.17717717717718	0.101818789904288\\
-2.16716716716717	0.102737878202036\\
-2.15715715715716	0.10366430531104\\
-2.14714714714715	0.104598112511144\\
-2.13713713713714	0.105539340997123\\
-2.12712712712713	0.106488031871605\\
-2.11711711711712	0.107444226137897\\
-2.10710710710711	0.108407964692688\\
-2.0970970970971	0.109379288318657\\
-2.08708708708709	0.110358237676957\\
-2.07707707707708	0.111344853299604\\
-2.06706706706707	0.112339175581744\\
-2.05705705705706	0.113341244773819\\
-2.04704704704705	0.114351100973622\\
-2.03703703703704	0.115368784118239\\
-2.02702702702703	0.116394333975887\\
-2.01701701701702	0.117427790137642\\
-2.00700700700701	0.118469192009056\\
-1.996996996997	0.119518578801666\\
-1.98698698698699	0.120575989524402\\
-1.97697697697698	0.121641462974876\\
-1.96696696696697	0.122715037730577\\
-1.95695695695696	0.123796752139946\\
-1.94694694694695	0.124886644313358\\
-1.93693693693694	0.125984752113991\\
-1.92692692692693	0.12709111314859\\
-1.91691691691692	0.128205764758136\\
-1.90690690690691	0.129328744008398\\
-1.8968968968969	0.1304600876804\\
-1.88688688688689	0.131599832260768\\
-1.87687687687688	0.132748013931997\\
-1.86686686686687	0.1339046685626\\
-1.85685685685686	0.135069831697176\\
-1.84684684684685	0.136243538546365\\
-1.83683683683684	0.137425823976724\\
-1.82682682682683	0.138616722500496\\
-1.81681681681682	0.139816268265294\\
-1.80680680680681	0.141024495043689\\
-1.7967967967968	0.142241436222713\\
-1.78678678678679	0.143467124793272\\
-1.77677677677678	0.144701593339469\\
-1.76676676676677	0.145944874027852\\
-1.75675675675676	0.14719699859657\\
-1.74674674674675	0.148457998344451\\
-1.73673673673674	0.149727904120005\\
-1.72672672672673	0.151006746310349\\
-1.71671671671672	0.152294554830052\\
-1.70670670670671	0.153591359109916\\
-1.6966966966967	0.154897188085683\\
-1.68668668668669	0.156212070186675\\
-1.67667667667668	0.157536033324371\\
-1.66666666666667	0.158869104880915\\
-1.65665665665666	0.160211311697578\\
-1.64664664664665	0.161562680063147\\
-1.63663663663664	0.162923235702273\\
-1.62662662662663	0.16429300376376\\
-1.61661661661662	0.165672008808807\\
-1.60660660660661	0.167060274799212\\
-1.5965965965966	0.168457825085522\\
-1.58658658658659	0.169864682395157\\
-1.57657657657658	0.171280868820488\\
-1.56656656656657	0.17270640580689\\
-1.55655655655656	0.174141314140766\\
-1.54654654654655	0.175585613937542\\
-1.53653653653654	0.177039324629646\\
-1.52652652652653	0.178502464954468\\
-1.51651651651652	0.179975052942305\\
-1.50650650650651	0.181457105904298\\
-1.4964964964965	0.182948640420364\\
-1.48648648648649	0.184449672327126\\
-1.47647647647648	0.185960216705848\\
-1.46646646646647	0.18748028787037\\
-1.45645645645646	0.189009899355067\\
-1.44644644644645	0.190549063902811\\
-1.43643643643644	0.19209779345296\\
-1.42642642642643	0.193656099129376\\
-1.41641641641642	0.195223991228462\\
-1.40640640640641	0.196801479207244\\
-1.3963963963964	0.198388571671487\\
-1.38638638638639	0.199985276363855\\
-1.37637637637638	0.201591600152124\\
-1.36636636636637	0.203207549017445\\
-1.35635635635636	0.204833128042671\\
-1.34634634634635	0.206468341400741\\
-1.33633633633634	0.208113192343147\\
-1.32632632632633	0.20976768318846\\
-1.31631631631632	0.211431815310953\\
-1.30630630630631	0.213105589129293\\
-1.2962962962963	0.214789004095343\\
-1.28628628628629	0.216482058683045\\
-1.27627627627628	0.218184750377415\\
-1.26626626626627	0.219897075663645\\
-1.25625625625626	0.221619030016318\\
-1.24624624624625	0.223350607888742\\
-1.23623623623624	0.22509180270241\\
-1.22622622622623	0.226842606836599\\
-1.21621621621622	0.228603011618093\\
-1.20620620620621	0.230373007311059\\
-1.1961961961962	0.232152583107074\\
-1.18618618618619	0.233941727115297\\
-1.17617617617618	0.235740426352817\\
-1.16616616616617	0.237548666735154\\
-1.15615615615616	0.239366433066944\\
-1.14614614614615	0.241193709032798\\
-1.13613613613614	0.243030477188351\\
-1.12612612612613	0.244876718951499\\
-1.11611611611612	0.246732414593837\\
-1.10610610610611	0.248597543232297\\
-1.0960960960961	0.250472082821003\\
-1.08608608608609	0.252356010143333\\
-1.07607607607608	0.254249300804216\\
-1.06606606606607	0.256151929222642\\
-1.05605605605606	0.258063868624421\\
-1.04604604604605	0.259985091035171\\
-1.03603603603604	0.261915567273563\\
-1.02602602602603	0.26385526694481\\
-1.01601601601602	0.265804158434416\\
-1.00600600600601	0.267762208902195\\
-0.995995995995996	0.269729384276556\\
-0.985985985985986	0.27170564924906\\
-0.975975975975976	0.273690967269267\\
-0.965965965965966	0.275685300539864\\
-0.955955955955956	0.27768861001209\\
-0.945945945945946	0.279700855381458\\
-0.935935935935936	0.281721995083777\\
-0.925925925925926	0.283751986291492\\
-0.915915915915916	0.285790784910332\\
-0.905905905905906	0.287838345576278\\
-0.895895895895896	0.289894621652862\\
-0.885885885885886	0.291959565228788\\
-0.875875875875876	0.294033127115892\\
-0.865865865865866	0.296115256847444\\
-0.855855855855856	0.29820590267679\\
-0.845845845845846	0.300305011576345\\
-0.835835835835836	0.302412529236941\\
-0.825825825825826	0.304528400067527\\
-0.815815815815816	0.306652567195241\\
-0.805805805805806	0.308784972465835\\
-0.795795795795796	0.310925556444483\\
-0.785785785785786	0.313074258416952\\
-0.775775775775776	0.315231016391155\\
-0.765765765765765	0.317395767099089\\
-0.755755755755755	0.319568445999144\\
-0.745745745745745	0.321748987278812\\
-0.735735735735735	0.323937323857781\\
-0.725725725725725	0.326133387391421\\
-0.715715715715715	0.328337108274662\\
-0.705705705705705	0.330548415646281\\
-0.695695695695695	0.332767237393574\\
-0.685685685685685	0.334993500157442\\
-0.675675675675675	0.337227129337872\\
-0.665665665665665	0.339468049099831\\
-0.655655655655655	0.341716182379559\\
-0.645645645645645	0.343971450891278\\
-0.635635635635635	0.346233775134302\\
-0.625625625625625	0.348503074400561\\
-0.615615615615615	0.350779266782536\\
-0.605605605605605	0.353062269181603\\
-0.595595595595595	0.355351997316785\\
-0.585585585585585	0.357648365733922\\
-0.575575575575575	0.359951287815245\\
-0.565565565565565	0.362260675789358\\
-0.555555555555555	0.364576440741639\\
-0.545545545545545	0.366898492625039\\
-0.535535535535535	0.36922674027129\\
-0.525525525525525	0.371561091402523\\
-0.515515515515515	0.373901452643281\\
-0.505505505505505	0.376247729532944\\
-0.495495495495495	0.378599826538544\\
-0.485485485485485	0.380957647067982\\
-0.475475475475475	0.383321093483636\\
-0.465465465465465	0.385690067116365\\
-0.455455455455455	0.388064468279897\\
-0.445445445445445	0.3904441962856\\
-0.435435435435435	0.392829149457642\\
-0.425425425425425	0.395219225148518\\
-0.415415415415415	0.39761431975496\\
-0.405405405405405	0.400014328734204\\
-0.395395395395395	0.402419146620637\\
-0.385385385385385	0.404828667042784\\
-0.375375375375375	0.407242782740667\\
-0.365365365365365	0.409661385583502\\
-0.355355355355355	0.412084366587739\\
-0.345345345345345	0.414511615935451\\
-0.335335335335335	0.416943022993039\\
-0.325325325325325	0.419378476330273\\
-0.315315315315315	0.421817863739649\\
-0.305305305305305	0.424261072256057\\
-0.295295295295295	0.426707988176757\\
-0.285285285285285	0.429158497081653\\
-0.275275275275275	0.431612483853853\\
-0.265265265265265	0.434069832700521\\
-0.255255255255255	0.436530427174001\\
-0.245245245245245	0.438994150193206\\
-0.235235235235235	0.441460884065275\\
-0.225225225225225	0.443930510507473\\
-0.215215215215215	0.44640291066934\\
-0.205205205205205	0.448877965155076\\
-0.195195195195195	0.45135555404615\\
-0.185185185185185	0.453835556924122\\
-0.175175175175175	0.456317852893685\\
-0.165165165165165	0.458802320605892\\
-0.155155155155155	0.461288838281586\\
-0.145145145145145	0.463777283734999\\
-0.135135135135135	0.46626753439753\\
-0.125125125125125	0.468759467341681\\
-0.115115115115115	0.471252959305143\\
-0.105105105105105	0.473747886715025\\
-0.0950950950950951	0.476244125712212\\
-0.0850850850850851	0.47874155217585\\
-0.075075075075075	0.481240041747932\\
-0.065065065065065	0.483739469857994\\
-0.055055055055055	0.486239711747894\\
-0.045045045045045	0.488740642496679\\
-0.035035035035035	0.491242137045508\\
-0.025025025025025	0.49374407022265\\
-0.015015015015015	0.496246316768516\\
-0.005005005005005	0.498748751360737\\
0.005005005005005	0.501251248639263\\
0.015015015015015	0.503753683231484\\
0.025025025025025	0.50625592977735\\
0.035035035035035	0.508757862954492\\
0.045045045045045	0.511259357503321\\
0.055055055055055	0.513760288252106\\
0.065065065065065	0.516260530142006\\
0.075075075075075	0.518759958252068\\
0.0850850850850851	0.52125844782415\\
0.0950950950950951	0.523755874287788\\
0.105105105105105	0.526252113284975\\
0.115115115115115	0.528747040694857\\
0.125125125125125	0.531240532658319\\
0.135135135135135	0.53373246560247\\
0.145145145145145	0.536222716265001\\
0.155155155155155	0.538711161718414\\
0.165165165165165	0.541197679394108\\
0.175175175175175	0.543682147106315\\
0.185185185185185	0.546164443075878\\
0.195195195195195	0.54864444595385\\
0.205205205205205	0.551122034844924\\
0.215215215215215	0.55359708933066\\
0.225225225225225	0.556069489492527\\
0.235235235235235	0.558539115934725\\
0.245245245245245	0.561005849806794\\
0.255255255255255	0.563469572825999\\
0.265265265265265	0.565930167299479\\
0.275275275275275	0.568387516146147\\
0.285285285285285	0.570841502918347\\
0.295295295295295	0.573292011823243\\
0.305305305305305	0.575738927743943\\
0.315315315315315	0.578182136260351\\
0.325325325325325	0.580621523669727\\
0.335335335335335	0.583056977006961\\
0.345345345345345	0.585488384064549\\
0.355355355355355	0.587915633412261\\
0.365365365365365	0.590338614416498\\
0.375375375375375	0.592757217259333\\
0.385385385385385	0.595171332957216\\
0.395395395395395	0.597580853379363\\
0.405405405405405	0.599985671265796\\
0.415415415415415	0.60238568024504\\
0.425425425425425	0.604780774851482\\
0.435435435435435	0.607170850542358\\
0.445445445445445	0.6095558037144\\
0.455455455455455	0.611935531720103\\
0.465465465465465	0.614309932883635\\
0.475475475475475	0.616678906516364\\
0.485485485485485	0.619042352932018\\
0.495495495495495	0.621400173461456\\
0.505505505505505	0.623752270467056\\
0.515515515515515	0.626098547356719\\
0.525525525525525	0.628438908597477\\
0.535535535535535	0.63077325972871\\
0.545545545545545	0.633101507374961\\
0.555555555555555	0.635423559258361\\
0.565565565565565	0.637739324210642\\
0.575575575575575	0.640048712184755\\
0.585585585585585	0.642351634266078\\
0.595595595595595	0.644648002683215\\
0.605605605605605	0.646937730818397\\
0.615615615615615	0.649220733217464\\
0.625625625625625	0.651496925599439\\
0.635635635635635	0.653766224865698\\
0.645645645645645	0.656028549108722\\
0.655655655655655	0.658283817620441\\
0.665665665665665	0.660531950900169\\
0.675675675675675	0.662772870662128\\
0.685685685685685	0.665006499842558\\
0.695695695695695	0.667232762606426\\
0.705705705705705	0.669451584353719\\
0.715715715715715	0.671662891725338\\
0.725725725725725	0.673866612608579\\
0.735735735735735	0.676062676142219\\
0.745745745745745	0.678251012721188\\
0.755755755755755	0.680431554000857\\
0.765765765765765	0.682604232900911\\
0.775775775775776	0.684768983608845\\
0.785785785785786	0.686925741583048\\
0.795795795795796	0.689074443555517\\
0.805805805805806	0.691215027534165\\
0.815815815815816	0.693347432804759\\
0.825825825825826	0.695471599932473\\
0.835835835835836	0.697587470763059\\
0.845845845845846	0.699694988423655\\
0.855855855855856	0.701794097323211\\
0.865865865865866	0.703884743152556\\
0.875875875875876	0.705966872884108\\
0.885885885885886	0.708040434771212\\
0.895895895895896	0.710105378347138\\
0.905905905905906	0.712161654423722\\
0.915915915915916	0.714209215089668\\
0.925925925925926	0.716248013708508\\
0.935935935935936	0.718278004916223\\
0.945945945945946	0.720299144618542\\
0.955955955955956	0.72231138998791\\
0.965965965965966	0.724314699460136\\
0.975975975975976	0.726309032730733\\
0.985985985985986	0.72829435075094\\
0.995995995995996	0.730270615723444\\
1.00600600600601	0.732237791097805\\
1.01601601601602	0.734195841565584\\
1.02602602602603	0.73614473305519\\
1.03603603603604	0.738084432726436\\
1.04604604604605	0.740014908964829\\
1.05605605605606	0.741936131375579\\
1.06606606606607	0.743848070777358\\
1.07607607607608	0.745750699195784\\
1.08608608608609	0.747643989856667\\
1.0960960960961	0.749527917178997\\
1.10610610610611	0.751402456767703\\
1.11611611611612	0.753267585406163\\
1.12612612612613	0.755123281048501\\
1.13613613613614	0.756969522811649\\
1.14614614614615	0.758806290967202\\
1.15615615615616	0.760633566933056\\
1.16616616616617	0.762451333264846\\
1.17617617617618	0.764259573647183\\
1.18618618618619	0.766058272884703\\
1.1961961961962	0.767847416892926\\
1.20620620620621	0.769626992688941\\
1.21621621621622	0.771396988381907\\
1.22622622622623	0.773157393163401\\
1.23623623623624	0.774908197297589\\
1.24624624624625	0.776649392111258\\
1.25625625625626	0.778380969983682\\
1.26626626626627	0.780102924336355\\
1.27627627627628	0.781815249622585\\
1.28628628628629	0.783517941316955\\
1.2962962962963	0.785210995904657\\
1.30630630630631	0.786894410870707\\
1.31631631631632	0.788568184689047\\
1.32632632632633	0.79023231681154\\
1.33633633633634	0.791886807656853\\
1.34634634634635	0.793531658599259\\
1.35635635635636	0.79516687195733\\
1.36636636636637	0.796792450982555\\
1.37637637637638	0.798408399847876\\
1.38638638638639	0.800014723636145\\
1.3963963963964	0.801611428328513\\
1.40640640640641	0.803198520792756\\
1.41641641641642	0.804776008771538\\
1.42642642642643	0.806343900870624\\
1.43643643643644	0.80790220654704\\
1.44644644644645	0.809450936097189\\
1.45645645645646	0.810990100644933\\
1.46646646646647	0.81251971212963\\
1.47647647647648	0.814039783294152\\
1.48648648648649	0.815550327672874\\
1.4964964964965	0.817051359579636\\
1.50650650650651	0.818542894095702\\
1.51651651651652	0.820024947057695\\
1.52652652652653	0.821497535045532\\
1.53653653653654	0.822960675370354\\
1.54654654654655	0.824414386062458\\
1.55655655655656	0.825858685859234\\
1.56656656656657	0.82729359419311\\
1.57657657657658	0.828719131179512\\
1.58658658658659	0.830135317604843\\
1.5965965965966	0.831542174914478\\
1.60660660660661	0.832939725200788\\
1.61661661661662	0.834327991191193\\
1.62662662662663	0.83570699623624\\
1.63663663663664	0.837076764297727\\
1.64664664664665	0.838437319936853\\
1.65665665665666	0.839788688302422\\
1.66666666666667	0.841130895119085\\
1.67667667667668	0.842463966675629\\
1.68668668668669	0.843787929813325\\
1.6966966966967	0.845102811914317\\
1.70670670670671	0.846408640890084\\
1.71671671671672	0.847705445169948\\
1.72672672672673	0.848993253689651\\
1.73673673673674	0.850272095879995\\
1.74674674674675	0.851542001655549\\
1.75675675675676	0.85280300140343\\
1.76676676676677	0.854055125972148\\
1.77677677677678	0.855298406660531\\
1.78678678678679	0.856532875206728\\
1.7967967967968	0.857758563777287\\
1.80680680680681	0.858975504956311\\
1.81681681681682	0.860183731734706\\
1.82682682682683	0.861383277499504\\
1.83683683683684	0.862574176023276\\
1.84684684684685	0.863756461453635\\
1.85685685685686	0.864930168302824\\
1.86686686686687	0.8660953314374\\
1.87687687687688	0.867251986068003\\
1.88688688688689	0.868400167739232\\
1.8968968968969	0.8695399123196\\
1.90690690690691	0.870671255991602\\
1.91691691691692	0.871794235241864\\
1.92692692692693	0.87290888685141\\
1.93693693693694	0.874015247886009\\
1.94694694694695	0.875113355686642\\
1.95695695695696	0.876203247860054\\
1.96696696696697	0.877284962269423\\
1.97697697697698	0.878358537025124\\
1.98698698698699	0.879424010475598\\
1.996996996997	0.880481421198334\\
2.00700700700701	0.881530807990944\\
2.01701701701702	0.882572209862358\\
2.02702702702703	0.883605666024113\\
2.03703703703704	0.884631215881761\\
2.04704704704705	0.885648899026378\\
2.05705705705706	0.886658755226181\\
2.06706706706707	0.887660824418256\\
2.07707707707708	0.888655146700396\\
2.08708708708709	0.889641762323043\\
2.0970970970971	0.890620711681343\\
2.10710710710711	0.891592035307312\\
2.11711711711712	0.892555773862103\\
2.12712712712713	0.893511968128395\\
2.13713713713714	0.894460659002877\\
2.14714714714715	0.895401887488856\\
2.15715715715716	0.89633569468896\\
2.16716716716717	0.897262121797964\\
2.17717717717718	0.898181210095712\\
2.18718718718719	0.899093000940156\\
2.1971971971972	0.8999975357605\\
2.20720720720721	0.900894856050455\\
2.21721721721722	0.901785003361601\\
2.22722722722723	0.902668019296848\\
2.23723723723724	0.903543945504023\\
2.24724724724725	0.904412823669541\\
2.25725725725726	0.905274695512201\\
2.26726726726727	0.90612960277708\\
2.27727727727728	0.906977587229526\\
2.28728728728729	0.907818690649271\\
2.2972972972973	0.908652954824632\\
2.30730730730731	0.909480421546828\\
2.31731731731732	0.910301132604392\\
2.32732732732733	0.911115129777691\\
2.33733733733734	0.911922454833542\\
2.34734734734735	0.912723149519936\\
2.35735735735736	0.913517255560855\\
2.36736736736737	0.914304814651192\\
2.37737737737738	0.915085868451774\\
2.38738738738739	0.915860458584475\\
2.3973973973974	0.916628626627434\\
2.40740740740741	0.917390414110361\\
2.41741741741742	0.918145862509951\\
2.42742742742743	0.918895013245378\\
2.43743743743744	0.9196379076739\\
2.44744744744745	0.92037458708654\\
2.45745745745746	0.921105092703873\\
2.46746746746747	0.921829465671897\\
2.47747747747748	0.922547747058\\
2.48748748748749	0.92325997784701\\
2.4974974974975	0.923966198937343\\
2.50750750750751	0.924666451137228\\
2.51751751751752	0.92536077516103\\
2.52752752752753	0.926049211625652\\
2.53753753753754	0.926731801047027\\
2.54754754754755	0.927408583836688\\
2.55755755755756	0.928079600298429\\
2.56756756756757	0.928744890625041\\
2.57757757757758	0.929404494895139\\
2.58758758758759	0.930058453070057\\
2.5975975975976	0.930706804990835\\
2.60760760760761	0.931349590375275\\
2.61761761761762	0.931986848815082\\
2.62762762762763	0.932618619773076\\
2.63763763763764	0.933244942580482\\
2.64764764764765	0.933865856434298\\
2.65765765765766	0.934481400394729\\
2.66766766766767	0.935091613382702\\
2.67767767767768	0.935696534177447\\
2.68768768768769	0.936296201414154\\
2.6976976976977	0.936890653581693\\
2.70770770770771	0.937479929020409\\
2.71771771771772	0.938064065919986\\
2.72772772772773	0.938643102317369\\
2.73773773773774	0.939217076094762\\
2.74774774774775	0.939786024977686\\
2.75775775775776	0.940349986533102\\
2.76776776776777	0.940908998167596\\
2.77777777777778	0.941463097125632\\
2.78778778778779	0.942012320487857\\
2.7977977977978	0.942556705169472\\
2.80780780780781	0.943096287918663\\
2.81781781781782	0.943631105315091\\
2.82782782782783	0.94416119376843\\
2.83783783783784	0.944686589516979\\
2.84784784784785	0.945207328626311\\
2.85785785785786	0.945723446987992\\
2.86786786786787	0.946234980318343\\
2.87787787787788	0.946741964157264\\
2.88788788788789	0.9472444338671\\
2.8978978978979	0.947742424631568\\
2.90790790790791	0.948235971454728\\
2.91791791791792	0.948725109160008\\
2.92792792792793	0.94920987238927\\
2.93793793793794	0.949690295601935\\
2.94794794794795	0.950166413074143\\
2.95795795795796	0.950638258897968\\
2.96796796796797	0.951105866980674\\
2.97797797797798	0.951569271044017\\
2.98798798798799	0.952028504623585\\
2.997997997998	0.952483601068194\\
3.00800800800801	0.952934593539309\\
3.01801801801802	0.953381515010517\\
3.02802802802803	0.953824398267041\\
3.03803803803804	0.954263275905286\\
3.04804804804805	0.954698180332428\\
3.05805805805806	0.955129143766044\\
3.06806806806807	0.955556198233773\\
3.07807807807808	0.95597937557302\\
3.08808808808809	0.956398707430688\\
3.0980980980981	0.956814225262951\\
3.10810810810811	0.957225960335063\\
3.11811811811812	0.957633943721192\\
3.12812812812813	0.958038206304293\\
3.13813813813814	0.958438778776016\\
3.14814814814815	0.958835691636637\\
3.15815815815816	0.959228975195025\\
3.16816816816817	0.959618659568642\\
3.17817817817818	0.960004774683563\\
3.18818818818819	0.960387350274536\\
3.1981981981982	0.960766415885061\\
3.20820820820821	0.961142000867503\\
3.21821821821822	0.961514134383226\\
3.22822822822823	0.961882845402758\\
3.23823823823824	0.96224816270598\\
3.24824824824825	0.962610114882338\\
3.25825825825826	0.962968730331085\\
3.26826826826827	0.963324037261537\\
3.27827827827828	0.963676063693364\\
3.28828828828829	0.964024837456895\\
3.2982982982983	0.964370386193449\\
3.30830830830831	0.964712737355685\\
3.31831831831832	0.965051918207977\\
3.32832832832833	0.965387955826802\\
3.33833833833834	0.96572087710116\\
3.34834834834835	0.966050708733001\\
3.35835835835836	0.966377477237676\\
3.36836836836837	0.96670120894441\\
3.37837837837838	0.967021929996785\\
3.38838838838839	0.96733966635325\\
3.3983983983984	0.967654443787634\\
3.40840840840841	0.967966287889694\\
3.41841841841842	0.968275224065662\\
3.42842842842843	0.968581277538816\\
3.43843843843844	0.968884473350066\\
3.44844844844845	0.96918483635855\\
3.45845845845846	0.969482391242249\\
3.46846846846847	0.969777162498615\\
3.47847847847848	0.970069174445208\\
3.48848848848849	0.970358451220349\\
3.4984984984985	0.97064501678379\\
3.50850850850851	0.970928894917385\\
3.51851851851852	0.971210109225783\\
3.52852852852853	0.971488683137127\\
3.53853853853854	0.971764639903767\\
3.54854854854855	0.972038002602978\\
3.55855855855856	0.972308794137696\\
3.56856856856857	0.972577037237256\\
3.57857857857858	0.972842754458144\\
3.58858858858859	0.973105968184761\\
3.5985985985986	0.973366700630184\\
3.60860860860861	0.97362497383695\\
3.61861861861862	0.97388080967784\\
3.62862862862863	0.974134229856669\\
3.63863863863864	0.974385255909089\\
3.64864864864865	0.974633909203396\\
3.65865865865866	0.974880210941344\\
3.66866866866867	0.975124182158967\\
3.67867867867868	0.975365843727404\\
3.68868868868869	0.975605216353733\\
3.6986986986987	0.975842320581812\\
3.70870870870871	0.97607717679312\\
3.71871871871872	0.976309805207606\\
3.72872872872873	0.976540225884545\\
3.73873873873874	0.976768458723396\\
3.74874874874875	0.976994523464665\\
3.75875875875876	0.977218439690769\\
3.76876876876877	0.977440226826912\\
3.77877877877878	0.977659904141957\\
3.78878878878879	0.977877490749302\\
3.7987987987988	0.978093005607762\\
3.80880880880881	0.978306467522455\\
3.81881881881882	0.978517895145687\\
3.82882882882883	0.978727306977838\\
3.83883883883884	0.97893472136826\\
3.84884884884885	0.979140156516164\\
3.85885885885886	0.979343630471517\\
3.86886886886887	0.979545161135941\\
3.87887887887888	0.97974476626361\\
3.88888888888889	0.97994246346215\\
3.8988988988989	0.980138270193536\\
3.90890890890891	0.980332203775002\\
3.91891891891892	0.980524281379934\\
3.92892892892893	0.980714520038781\\
3.93893893893894	0.980902936639954\\
3.94894894894895	0.98108954793073\\
3.95895895895896	0.981274370518157\\
3.96896896896897	0.98145742086996\\
3.97897897897898	0.98163871531544\\
3.98898898898899	0.98181827004638\\
3.998998998999	0.981996101117947\\
4.00900900900901	0.982172224449597\\
4.01901901901902	0.982346655825972\\
4.02902902902903	0.982519410897803\\
4.03903903903904	0.982690505182809\\
4.04904904904905	0.982859954066598\\
4.05905905905906	0.98302777280356\\
4.06906906906907	0.983193976517765\\
4.07907907907908	0.983358580203856\\
4.08908908908909	0.983521598727944\\
4.0990990990991	0.983683046828498\\
4.10910910910911	0.983842939117233\\
4.11911911911912	0.984001290079997\\
4.12912912912913	0.984158114077659\\
4.13913913913914	0.98431342534699\\
4.14914914914915	0.984467238001542\\
4.15915915915916	0.984619566032531\\
4.16916916916917	0.984770423309711\\
4.17917917917918	0.984919823582245\\
4.18918918918919	0.98506778047958\\
4.1991991991992	0.985214307512314\\
4.20920920920921	0.985359418073063\\
4.21921921921922	0.98550312543732\\
4.22922922922923	0.985645442764319\\
4.23923923923924	0.985786383097892\\
4.24924924924925	0.985925959367319\\
4.25925925925926	0.986064184388184\\
4.26926926926927	0.986201070863221\\
4.27927927927928	0.986336631383156\\
4.28928928928929	0.986470878427552\\
4.2992992992993	0.986603824365647\\
4.30930930930931	0.986735481457184\\
4.31931931931932	0.986865861853249\\
4.32932932932933	0.986994977597093\\
4.33933933933934	0.987122840624959\\
4.34934934934935	0.987249462766902\\
4.35935935935936	0.987374855747606\\
4.36936936936937	0.987499031187197\\
4.37937937937938	0.987622000602052\\
4.38938938938939	0.987743775405606\\
4.3993993993994	0.987864366909155\\
4.40940940940941	0.98798378632265\\
4.41941941941942	0.988102044755495\\
4.42942942942943	0.988219153217336\\
4.43943943943944	0.988335122618847\\
4.44944944944945	0.988449963772515\\
4.45945945945946	0.988563687393413\\
4.46946946946947	0.98867630409998\\
4.47947947947948	0.98878782441479\\
4.48948948948949	0.988898258765313\\
4.4994994994995	0.989007617484687\\
4.50950950950951	0.989115910812466\\
4.51951951951952	0.98922314889538\\
4.52952952952953	0.989329341788083\\
4.53953953953954	0.989434499453898\\
4.54954954954955	0.989538631765558\\
4.55955955955956	0.989641748505944\\
4.56956956956957	0.989743859368819\\
4.57957957957958	0.989844973959552\\
4.58958958958959	0.989945101795848\\
4.5995995995996	0.990044252308463\\
4.60960960960961	0.990142434841924\\
4.61961961961962	0.990239658655236\\
4.62962962962963	0.990335932922592\\
4.63963963963964	0.990431266734076\\
4.64964964964965	0.990525669096361\\
4.65965965965966	0.990619148933401\\
4.66966966966967	0.990711715087122\\
4.67967967967968	0.99080337631811\\
4.68968968968969	0.990894141306288\\
4.6996996996997	0.990984018651596\\
4.70970970970971	0.991073016874662\\
4.71971971971972	0.991161144417467\\
4.72972972972973	0.991248409644017\\
4.73973973973974	0.991334820840992\\
4.74974974974975	0.99142038621841\\
4.75975975975976	0.991505113910271\\
4.76976976976977	0.991589011975208\\
4.77977977977978	0.991672088397125\\
4.78978978978979	0.991754351085839\\
4.7997997997998	0.991835807877709\\
4.80980980980981	0.991916466536268\\
4.81981981981982	0.991996334752847\\
4.82982982982983	0.992075420147194\\
4.83983983983984	0.992153730268091\\
4.84984984984985	0.992231272593968\\
4.85985985985986	0.992308054533506\\
4.86986986986987	0.992384083426243\\
4.87987987987988	0.992459366543173\\
4.88988988988989	0.992533911087341\\
4.8998998998999	0.992607724194431\\
4.90990990990991	0.992680812933355\\
4.91991991991992	0.992753184306833\\
4.92992992992993	0.992824845251973\\
4.93993993993994	0.992895802640841\\
4.94994994994995	0.992966063281034\\
4.95995995995996	0.993035633916245\\
4.96996996996997	0.993104521226821\\
4.97997997997998	0.993172731830322\\
4.98998998998999	0.993240272282073\\
5	0.993307149075715\\
};
\addlegendentry{logistic};

\addplot [color=mycolor2,solid]
  table[row sep=crcr]{%
-5	0.000864306524830867\\
-4.98998998998999	0.000882962095556502\\
-4.97997997997998	0.000901987211456678\\
-4.96996996996997	0.000921388429369194\\
-4.95995995995996	0.000941172405862063\\
-4.94994994994995	0.000961345898409526\\
-4.93993993993994	0.000981915766575248\\
-4.92992992992993	0.00100288897320257\\
-4.91991991991992	0.00102427258561174\\
-4.90990990990991	0.00104607377680403\\
-4.8998998998999	0.00106829982667256\\
-4.88988988988989	0.00109095812321989\\
-4.87987987987988	0.00111405616378201\\
-4.86986986986987	0.00113760155625893\\
-4.85985985985986	0.00116160202035141\\
-4.84984984984985	0.00118606538880403\\
-4.83983983983984	0.00121099960865419\\
-4.82982982982983	0.00123641274248717\\
-4.81981981981982	0.00126231296969684\\
-4.80980980980981	0.00128870858775215\\
-4.7997997997998	0.00131560801346906\\
-4.78978978978979	0.00134301978428787\\
-4.77977977977978	0.00137095255955575\\
-4.76976976976977	0.00139941512181431\\
-4.75975975975976	0.00142841637809214\\
-4.74974974974975	0.00145796536120195\\
-4.73973973973974	0.00148807123104246\\
-4.72972972972973	0.0015187432759045\\
-4.71971971971972	0.00154999091378149\\
-4.70970970970971	0.00158182369368384\\
-4.6996996996997	0.00161425129695726\\
-4.68968968968969	0.00164728353860472\\
-4.67967967967968	0.0016809303686119\\
-4.66966966966967	0.00171520187327586\\
-4.65965965965966	0.00175010827653683\\
-4.64964964964965	0.00178565994131286\\
-4.63963963963964	0.00182186737083705\\
-4.62962962962963	0.00185874120999733\\
-4.61961961961962	0.00189629224667831\\
-4.60960960960961	0.00193453141310524\\
-4.5995995995996	0.00197346978718968\\
-4.58958958958959	0.00201311859387666\\
-4.57957957957958	0.00205348920649313\\
-4.56956956956957	0.00209459314809753\\
-4.55955955955956	0.00213644209282999\\
-4.54954954954955	0.00217904786726325\\
-4.53953953953954	0.00222242245175367\\
-4.52952952952953	0.00226657798179241\\
-4.51951951951952	0.00231152674935622\\
-4.50950950950951	0.00235728120425785\\
-4.4994994994995	0.00240385395549547\\
-4.48948948948949	0.00245125777260118\\
-4.47947947947948	0.00249950558698805\\
-4.46946946946947	0.00254861049329546\\
-4.45945945945946	0.00259858575073258\\
-4.44944944944945	0.0026494447844195\\
-4.43943943943944	0.00270120118672583\\
-4.42942942942943	0.00275386871860635\\
-4.41941941941942	0.00280746131093359\\
-4.40940940940941	0.00286199306582675\\
-4.3993993993994	0.00291747825797686\\
-4.38938938938939	0.00297393133596771\\
-4.37937937937938	0.00303136692359235\\
-4.36936936936937	0.00308979982116461\\
-4.35935935935936	0.00314924500682556\\
-4.34934934934935	0.00320971763784433\\
-4.33933933933934	0.00327123305191304\\
-4.32932932932933	0.00333380676843554\\
-4.31931931931932	0.00339745448980944\\
-4.30930930930931	0.00346219210270116\\
-4.2992992992993	0.00352803567931371\\
-4.28928928928929	0.0035950014786466\\
-4.27927927927928	0.00366310594774767\\
-4.26926926926927	0.00373236572295648\\
-4.25925925925926	0.00380279763113866\\
-4.24924924924925	0.00387441869091107\\
-4.23923923923924	0.00394724611385725\\
-4.22922922922923	0.00402129730573266\\
-4.21921921921922	0.00409658986765955\\
-4.20920920920921	0.0041731415973108\\
-4.1991991991992	0.00425097049008246\\
-4.18918918918919	0.00433009474025448\\
-4.17917917917918	0.00441053274213926\\
-4.16916916916917	0.00449230309121765\\
-4.15915915915916	0.00457542458526171\\
-4.14914914914915	0.00465991622544413\\
-4.13913913913914	0.00474579721743366\\
-4.12912912912913	0.00483308697247607\\
-4.11911911911912	0.00492180510846041\\
-4.10910910910911	0.00501197145096978\\
-4.0990990990991	0.00510360603431656\\
-4.08908908908909	0.0051967291025612\\
-4.07907907907908	0.00529136111051452\\
-4.06906906906907	0.00538752272472266\\
-4.05905905905906	0.00548523482443456\\
-4.04904904904905	0.00558451850255125\\
-4.03903903903904	0.00568539506655653\\
-4.02902902902903	0.00578788603942866\\
-4.01901901901902	0.00589201316053236\\
-4.00900900900901	0.00599779838649093\\
-3.998998998999	0.00610526389203763\\
-3.98898898898899	0.00621443207084617\\
-3.97897897897898	0.0063253255363396\\
-3.96896896896897	0.00643796712247714\\
-3.95895895895896	0.0065523798845185\\
-3.94894894894895	0.00666858709976511\\
-3.93893893893894	0.0067866122682778\\
-3.92892892892893	0.00690647911357043\\
-3.91891891891892	0.00702821158327886\\
-3.90890890890891	0.00715183384980487\\
-3.8988988988989	0.00727737031093442\\
-3.88888888888889	0.00740484559042978\\
-3.87887887887888	0.00753428453859499\\
-3.86886886886887	0.00766571223281407\\
-3.85885885885886	0.00779915397806159\\
-3.84884884884885	0.00793463530738491\\
-3.83883883883884	0.00807218198235772\\
-3.82882882882883	0.00821181999350419\\
-3.81881881881882	0.00835357556069335\\
-3.80880880880881	0.008497475133503\\
-3.7987987987988	0.00864354539155285\\
-3.78878878878879	0.00879181324480607\\
-3.77877877877878	0.00894230583383899\\
-3.76876876876877	0.0090950505300782\\
-3.75875875875876	0.00925007493600464\\
-3.74874874874875	0.00940740688532412\\
-3.73873873873874	0.00956707444310368\\
-3.72872872872873	0.00972910590587334\\
-3.71871871871872	0.00989352980169264\\
-3.70870870870871	0.0100603748901814\\
-3.6986986986987	0.0102296701625144\\
-3.68868868868869	0.0104014448413791\\
-3.67867867867868	0.0105757283808959\\
-3.66866866866867	0.0107525504665011\\
-3.65865865865866	0.0109319410147906\\
-3.64864864864865	0.0111139301733257\\
-3.63863863863864	0.0112985483203987\\
-3.62862862862863	0.0114858260647586\\
-3.61861861861862	0.0116757942452969\\
-3.60860860860861	0.0118684839306914\\
-3.5985985985986	0.0120639264190096\\
-3.58858858858859	0.0122621532372691\\
-3.57857857857858	0.0124631961409561\\
-3.56856856856857	0.0126670871135002\\
-3.55855855855856	0.0128738583657061\\
-3.54854854854855	0.0130835423351412\\
-3.53853853853854	0.0132961716854783\\
-3.52852852852853	0.0135117793057935\\
-3.51851851851852	0.013730398309818\\
-3.50850850850851	0.0139520620351449\\
-3.4984984984985	0.0141768040423879\\
-3.48848848848849	0.0144046581142945\\
-3.47847847847848	0.0146356582548106\\
-3.46846846846847	0.0148698386880975\\
-3.45845845845846	0.0151072338575003\\
-3.44844844844845	0.0153478784244677\\
-3.43843843843844	0.0155918072674219\\
-3.42842842842843	0.015839055480579\\
-3.41841841841842	0.016089658372719\\
-3.40840840840841	0.0163436514659051\\
-3.3983983983984	0.0166010704941518\\
-3.38838838838839	0.0168619514020414\\
-3.37837837837838	0.0171263303432884\\
-3.36836836836837	0.0173942436792515\\
-3.35835835835836	0.0176657279773931\\
-3.34834834834835	0.0179408200096847\\
-3.33833833833834	0.0182195567509598\\
-3.32832832832833	0.0185019753772115\\
-3.31831831831832	0.0187881132638368\\
-3.30830830830831	0.0190780079838256\\
-3.2982982982983	0.0193716973058943\\
-3.28828828828829	0.0196692191925644\\
-3.27827827827828	0.019970611798185\\
-3.26826826826827	0.0202759134668991\\
-3.25825825825826	0.0205851627305533\\
-3.24824824824825	0.0208983983065512\\
-3.23823823823824	0.0212156590956486\\
-3.22822822822823	0.0215369841796923\\
-3.21821821821822	0.0218624128193005\\
-3.20820820820821	0.0221919844514854\\
-3.1981981981982	0.0225257386872177\\
-3.18818818818819	0.022863715308932\\
-3.17817817817818	0.0232059542679738\\
-3.16816816816817	0.0235524956819876\\
-3.15815815815816	0.0239033798322451\\
-3.14814814814815	0.0242586471609146\\
-3.13813813813814	0.02461833826827\\
-3.12812812812813	0.0249824939098407\\
-3.11811811811812	0.0253511549935008\\
-3.10810810810811	0.0257243625764981\\
-3.0980980980981	0.0261021578624231\\
-3.08808808808809	0.0264845821981168\\
-3.07807807807808	0.0268716770705186\\
-3.06806806806807	0.0272634841034528\\
-3.05805805805806	0.0276600450543542\\
-3.04804804804805	0.0280614018109331\\
-3.03803803803804	0.0284675963877786\\
-3.02802802802803	0.0288786709229016\\
-3.01801801801802	0.0292946676742154\\
-3.00800800800801	0.0297156290159563\\
-2.997997997998	0.030141597435042\\
-2.98798798798799	0.0305726155273689\\
-2.97797797797798	0.0310087259940488\\
-2.96796796796797	0.031449971637583\\
-2.95795795795796	0.0318963953579765\\
-2.94794794794795	0.0323480401487901\\
-2.93793793793794	0.0328049490931319\\
-2.92792792792793	0.0332671653595872\\
-2.91791791791792	0.0337347321980881\\
-2.90790790790791	0.0342076929357216\\
-2.8978978978979	0.0346860909724774\\
-2.88788788788789	0.0351699697769352\\
-2.87787787787788	0.0356593728818907\\
-2.86786786786787	0.0361543438799227\\
-2.85785785785786	0.0366549264188988\\
-2.84784784784785	0.037161164197422\\
-2.83783783783784	0.0376731009602175\\
-2.82782782782783	0.0381907804934602\\
-2.81781781781782	0.0387142466200422\\
-2.80780780780781	0.0392435431947831\\
-2.7977977977978	0.0397787140995797\\
-2.78778778778779	0.0403198032384989\\
-2.77777777777778	0.040866854532812\\
-2.76776776776777	0.0414199119159715\\
-2.75775775775776	0.0419790193285306\\
-2.74774774774775	0.0425442207130057\\
-2.73773773773774	0.0431155600086823\\
-2.72772772772773	0.0436930811463655\\
-2.71771771771772	0.044276828043074\\
-2.70770770770771	0.0448668445966789\\
-2.6976976976977	0.0454631746804887\\
-2.68768768768769	0.0460658621377788\\
-2.67767767767768	0.0466749507762678\\
-2.66766766766767	0.0472904843625408\\
-2.65765765765766	0.0479125066164196\\
-2.64764764764765	0.0485410612052806\\
-2.63763763763764	0.0491761917383213\\
-2.62762762762763	0.0498179417607763\\
-2.61761761761762	0.0504663547480819\\
-2.60760760760761	0.0511214740999915\\
-2.5975975975976	0.0517833431346421\\
-2.58758758758759	0.0524520050825718\\
-2.57757757757758	0.0531275030806898\\
-2.56756756756757	0.0538098801661995\\
-2.55755755755756	0.054499179270475\\
-2.54754754754755	0.0551954432128916\\
-2.53753753753754	0.0558987146946127\\
-2.52752752752753	0.056609036292331\\
-2.51751751751752	0.0573264504519677\\
-2.50750750750751	0.0580509994823284\\
-2.4974974974975	0.0587827255487179\\
-2.48748748748749	0.0595216706665137\\
-2.47747747747748	0.0602678766947\\
-2.46746746746747	0.0610213853293624\\
-2.45745745745746	0.0617822380971446\\
-2.44744744744745	0.0625504763486675\\
-2.43743743743744	0.0633261412519124\\
-2.42742742742743	0.0641092737855687\\
-2.41741741741742	0.0648999147323474\\
-2.40740740740741	0.0656981046722608\\
-2.3973973973974	0.0665038839758704\\
-2.38738738738739	0.0673172927975027\\
-2.37737737737738	0.0681383710684355\\
-2.36736736736737	0.0689671584900546\\
-2.35735735735736	0.0698036945269821\\
-2.34734734734735	0.0706480184001778\\
-2.33733733733734	0.0715001690800152\\
-2.32732732732733	0.0723601852793313\\
-2.31731731731732	0.0732281054464543\\
-2.30730730730731	0.0741039677582076\\
-2.2972972972973	0.0749878101128928\\
-2.28728728728729	0.0758796701232528\\
-2.27727727727728	0.0767795851094155\\
-2.26726726726727	0.0776875920918201\\
-2.25725725725726	0.0786037277841264\\
-2.24724724724725	0.0795280285861096\\
-2.23723723723724	0.0804605305765407\\
-2.22722722722723	0.0814012695060536\\
-2.21721721721722	0.0823502807900021\\
-2.20720720720721	0.0833075995013057\\
-2.1971971971972	0.0842732603632872\\
-2.18718718718719	0.0852472977425022\\
-2.17717717717718	0.0862297456415638\\
-2.16716716716717	0.087220637691961\\
-2.15715715715716	0.0882200071468751\\
-2.14714714714715	0.089227886873993\\
-2.13713713713714	0.0902443093483212\\
-2.12712712712713	0.0912693066449995\\
-2.11711711711712	0.0923029104321176\\
-2.10710710710711	0.0933451519635357\\
-2.0970970970971	0.0943960620717091\\
-2.08708708708709	0.0954556711605212\\
-2.07707707707708	0.0965240091981231\\
-2.06706706706707	0.0976011057097833\\
-2.05705705705706	0.0986869897707491\\
-2.04704704704705	0.0997816899991196\\
-2.03703703703704	0.100885234548733\\
-2.02702702702703	0.101997651102071\\
-2.01701701701702	0.103118966863178\\
-2.00700700700701	0.104249208550599\\
-1.996996996997	0.105388402390343\\
-1.98698698698699	0.106536574108858\\
-1.97697697697698	0.107693748926041\\
-1.96696696696697	0.108859951548264\\
-1.95695695695696	0.110035206161431\\
-1.94694694694695	0.111219536424062\\
-1.93693693693694	0.112412965460404\\
-1.92692692692693	0.11361551585358\\
-1.91691691691692	0.114827209638762\\
-1.90690690690691	0.116048068296386\\
-1.8968968968969	0.117278112745401\\
-1.88688688688689	0.118517363336548\\
-1.87687687687688	0.119765839845695\\
-1.86686686686687	0.121023561467195\\
-1.85685685685686	0.122290546807294\\
-1.84684684684685	0.123566813877589\\
-1.83683683683684	0.124852380088522\\
-1.82682682682683	0.126147262242926\\
-1.81681681681682	0.127451476529619\\
-1.80680680680681	0.12876503851705\\
-1.7967967967968	0.130087963146993\\
-1.78678678678679	0.131420264728304\\
-1.77677677677678	0.132761956930719\\
-1.76676676676677	0.134113052778726\\
-1.75675675675676	0.135473564645479\\
-1.74674674674675	0.136843504246786\\
-1.73673673673674	0.13822288263515\\
-1.72672672672673	0.139611710193879\\
-1.71671671671672	0.141009996631256\\
-1.70670670670671	0.14241775097478\\
-1.6966966966967	0.143834981565474\\
-1.68668668668669	0.145261696052261\\
-1.67667667667668	0.146697901386416\\
-1.66666666666667	0.148143603816085\\
-1.65665665665666	0.14959880888088\\
-1.64664664664665	0.151063521406559\\
-1.63663663663664	0.152537745499767\\
-1.62662662662663	0.154021484542876\\
-1.61661661661662	0.155514741188884\\
-1.60660660660661	0.157017517356419\\
-1.5965965965966	0.158529814224808\\
-1.58658658658659	0.160051632229241\\
-1.57657657657658	0.161582971056022\\
-1.56656656656657	0.163123829637903\\
-1.55655655655656	0.164674206149514\\
-1.54654654654655	0.166234098002879\\
-1.53653653653654	0.167803501843027\\
-1.52652652652653	0.169382413543699\\
-1.51651651651652	0.170970828203146\\
-1.50650650650651	0.172568740140028\\
-1.4964964964965	0.174176142889413\\
-1.48648648648649	0.175793029198866\\
-1.47647647647648	0.177419391024652\\
-1.46646646646647	0.179055219528031\\
-1.45645645645646	0.180700505071667\\
-1.44644644644645	0.182355237216128\\
-1.43643643643644	0.184019404716509\\
-1.42642642642643	0.185692995519149\\
-1.41641641641642	0.187375996758465\\
-1.40640640640641	0.18906839475389\\
-1.3963963963964	0.190770175006932\\
-1.38638638638639	0.192481322198335\\
-1.37637637637638	0.19420182018536\\
-1.36636636636637	0.195931651999179\\
-1.35635635635636	0.197670799842386\\
-1.34634634634635	0.199419245086625\\
-1.33633633633634	0.201176968270336\\
-1.32632632632633	0.202943949096619\\
-1.31631631631632	0.204720166431225\\
-1.30630630630631	0.206505598300658\\
-1.2962962962963	0.208300221890409\\
-1.28628628628629	0.210104013543307\\
-1.27627627627628	0.211916948758\\
-1.26626626626627	0.213739002187557\\
-1.25625625625626	0.215570147638199\\
-1.24624624624625	0.217410358068155\\
-1.23623623623624	0.219259605586648\\
-1.22622622622623	0.22111786145301\\
-1.21621621621622	0.222985096075924\\
-1.20620620620621	0.224861279012801\\
-1.1961961961962	0.226746378969285\\
-1.18618618618619	0.22864036379889\\
-1.17617617617618	0.23054320050277\\
-1.16616616616617	0.232454855229627\\
-1.15615615615616	0.234375293275743\\
-1.14614614614615	0.236304479085157\\
-1.13613613613614	0.238242376249972\\
-1.12612612612613	0.240188947510797\\
-1.11611611611612	0.242144154757329\\
-1.10610610610611	0.24410795902907\\
-1.0960960960961	0.246080320516177\\
-1.08608608608609	0.248061198560459\\
-1.07607607607608	0.250050551656505\\
-1.06606606606607	0.252048337452951\\
-1.05605605605606	0.25405451275389\\
-1.04604604604605	0.256069033520415\\
-1.03603603603604	0.258091854872309\\
-1.02602602602603	0.260122931089866\\
-1.01601601601602	0.262162215615863\\
-1.00600600600601	0.264209661057658\\
-0.995995995995996	0.266265219189441\\
-0.985985985985986	0.26832884095462\\
-0.975975975975976	0.270400476468345\\
-0.965965965965966	0.272480075020179\\
-0.955955955955956	0.274567585076901\\
-0.945945945945946	0.276662954285461\\
-0.935935935935936	0.278766129476061\\
-0.925925925925926	0.280877056665392\\
-0.915915915915916	0.282995681059999\\
-0.905905905905906	0.28512194705979\\
-0.895895895895896	0.287255798261689\\
-0.885885885885886	0.289397177463424\\
-0.875875875875876	0.291546026667456\\
-0.865865865865866	0.293702287085048\\
-0.855855855855856	0.29586589914047\\
-0.845845845845846	0.298036802475348\\
-0.835835835835836	0.300214935953145\\
-0.825825825825826	0.302400237663786\\
-0.815815815815816	0.304592644928414\\
-0.805805805805806	0.306792094304287\\
-0.795795795795796	0.308998521589814\\
-0.785785785785786	0.311211861829715\\
-0.775775775775776	0.313432049320333\\
-0.765765765765765	0.315659017615067\\
-0.755755755755755	0.317892699529946\\
-0.745745745745745	0.320133027149332\\
-0.735735735735735	0.322379931831758\\
-0.725725725725725	0.324633344215901\\
-0.715715715715715	0.326893194226678\\
-0.705705705705705	0.329159411081478\\
-0.695695695695695	0.331431923296524\\
-0.685685685685685	0.333710658693357\\
-0.675675675675675	0.335995544405457\\
-0.665665665665665	0.33828650688498\\
-0.655655655655655	0.340583471909631\\
-0.645645645645645	0.342886364589654\\
-0.635635635635635	0.345195109374946\\
-0.625625625625625	0.347509630062302\\
-0.615615615615615	0.349829849802766\\
-0.605605605605605	0.35215569110912\\
-0.595595595595595	0.354487075863473\\
-0.585585585585585	0.356823925324989\\
-0.575575575575575	0.359166160137711\\
-0.565565565565565	0.361513700338514\\
-0.555555555555555	0.363866465365167\\
-0.545545545545545	0.366224374064508\\
-0.535535535535535	0.368587344700729\\
-0.525525525525525	0.370955294963776\\
-0.515515515515515	0.373328141977851\\
-0.505505505505505	0.375705802310026\\
-0.495495495495495	0.378088191978957\\
-0.485485485485485	0.38047522646371\\
-0.475475475475475	0.382866820712684\\
-0.465465465465465	0.385262889152636\\
-0.455455455455455	0.387663345697803\\
-0.445445445445445	0.390068103759128\\
-0.435435435435435	0.39247707625358\\
-0.425425425425425	0.394890175613562\\
-0.415415415415415	0.39730731379642\\
-0.405405405405405	0.399728402294041\\
-0.395395395395395	0.402153352142537\\
-0.385385385385385	0.404582073932022\\
-0.375375375375375	0.40701447781647\\
-0.365365365365365	0.409450473523659\\
-0.355355355355355	0.411889970365197\\
-0.345345345345345	0.414332877246632\\
-0.335335335335335	0.41677910267763\\
-0.325325325325325	0.419228554782245\\
-0.315315315315315	0.421681141309249\\
-0.305305305305305	0.424136769642544\\
-0.295295295295295	0.426595346811639\\
-0.285285285285285	0.429056779502205\\
-0.275275275275275	0.431520974066681\\
-0.265265265265265	0.433987836534964\\
-0.255255255255255	0.436457272625144\\
-0.245245245245245	0.438929187754312\\
-0.235235235235235	0.441403487049423\\
-0.225225225225225	0.443880075358212\\
-0.215215215215215	0.44635885726017\\
-0.205205205205205	0.448839737077569\\
-0.195195195195195	0.451322618886537\\
-0.185185185185185	0.453807406528185\\
-0.175175175175175	0.456294003619775\\
-0.165165165165165	0.458782313565936\\
-0.155155155155155	0.461272239569918\\
-0.145145145145145	0.463763684644886\\
-0.135135135135135	0.466256551625258\\
-0.125125125125125	0.468750743178066\\
-0.115115115115115	0.471246161814361\\
-0.105105105105105	0.473742709900641\\
-0.0950950950950951	0.476240289670313\\
-0.0850850850850851	0.478738803235175\\
-0.075075075075075	0.481238152596929\\
-0.065065065065065	0.483738239658711\\
-0.055055055055055	0.486238966236639\\
-0.045045045045045	0.488740234071385\\
-0.035035035035035	0.491241944839753\\
-0.025025025025025	0.493744000166276\\
-0.015015015015015	0.496246301634821\\
-0.005005005005005	0.498748750800202\\
0.005005005005005	0.501251249199798\\
0.015015015015015	0.503753698365179\\
0.025025025025025	0.506255999833724\\
0.035035035035035	0.508758055160247\\
0.045045045045045	0.511259765928615\\
0.055055055055055	0.513761033763361\\
0.065065065065065	0.516261760341289\\
0.075075075075075	0.518761847403071\\
0.0850850850850851	0.521261196764825\\
0.0950950950950951	0.523759710329687\\
0.105105105105105	0.526257290099359\\
0.115115115115115	0.528753838185639\\
0.125125125125125	0.531249256821934\\
0.135135135135135	0.533743448374742\\
0.145145145145145	0.536236315355114\\
0.155155155155155	0.538727760430082\\
0.165165165165165	0.541217686434064\\
0.175175175175175	0.543705996380225\\
0.185185185185185	0.546192593471815\\
0.195195195195195	0.548677381113463\\
0.205205205205205	0.551160262922431\\
0.215215215215215	0.55364114273983\\
0.225225225225225	0.556119924641788\\
0.235235235235235	0.558596512950577\\
0.245245245245245	0.561070812245688\\
0.255255255255255	0.563542727374856\\
0.265265265265265	0.566012163465036\\
0.275275275275275	0.568479025933319\\
0.285285285285285	0.570943220497795\\
0.295295295295295	0.573404653188361\\
0.305305305305305	0.575863230357456\\
0.315315315315315	0.578318858690751\\
0.325325325325325	0.580771445217755\\
0.335335335335335	0.58322089732237\\
0.345345345345345	0.585667122753368\\
0.355355355355355	0.588110029634803\\
0.365365365365365	0.590549526476341\\
0.375375375375375	0.59298552218353\\
0.385385385385385	0.595417926067978\\
0.395395395395395	0.597846647857463\\
0.405405405405405	0.600271597705959\\
0.415415415415415	0.60269268620358\\
0.425425425425425	0.605109824386438\\
0.435435435435435	0.60752292374642\\
0.445445445445445	0.609931896240872\\
0.455455455455455	0.612336654302198\\
0.465465465465465	0.614737110847364\\
0.475475475475475	0.617133179287316\\
0.485485485485485	0.61952477353629\\
0.495495495495495	0.621911808021043\\
0.505505505505505	0.624294197689974\\
0.515515515515515	0.626671858022149\\
0.525525525525525	0.629044705036224\\
0.535535535535535	0.631412655299271\\
0.545545545545545	0.633775625935492\\
0.555555555555555	0.636133534634833\\
0.565565565565565	0.638486299661486\\
0.575575575575575	0.640833839862289\\
0.585585585585585	0.643176074675011\\
0.595595595595595	0.645512924136527\\
0.605605605605605	0.647844308890881\\
0.615615615615615	0.650170150197233\\
0.625625625625625	0.652490369937698\\
0.635635635635635	0.654804890625054\\
0.645645645645645	0.657113635410346\\
0.655655655655655	0.659416528090369\\
0.665665665665665	0.66171349311502\\
0.675675675675675	0.664004455594543\\
0.685685685685685	0.666289341306643\\
0.695695695695695	0.668568076703476\\
0.705705705705705	0.670840588918522\\
0.715715715715715	0.673106805773322\\
0.725725725725725	0.675366655784099\\
0.735735735735735	0.677620068168242\\
0.745745745745745	0.679866972850668\\
0.755755755755755	0.682107300470054\\
0.765765765765765	0.684340982384933\\
0.775775775775776	0.686567950679667\\
0.785785785785786	0.688788138170285\\
0.795795795795796	0.691001478410186\\
0.805805805805806	0.693207905695713\\
0.815815815815816	0.695407355071586\\
0.825825825825826	0.697599762336214\\
0.835835835835836	0.699785064046855\\
0.845845845845846	0.701963197524652\\
0.855855855855856	0.70413410085953\\
0.865865865865866	0.706297712914952\\
0.875875875875876	0.708453973332544\\
0.885885885885886	0.710602822536576\\
0.895895895895896	0.712744201738311\\
0.905905905905906	0.71487805294021\\
0.915915915915916	0.717004318940001\\
0.925925925925926	0.719122943334608\\
0.935935935935936	0.721233870523939\\
0.945945945945946	0.723337045714539\\
0.955955955955956	0.725432414923099\\
0.965965965965966	0.727519924979821\\
0.975975975975976	0.729599523531655\\
0.985985985985986	0.73167115904538\\
0.995995995995996	0.733734780810559\\
1.00600600600601	0.735790338942342\\
1.01601601601602	0.737837784384137\\
1.02602602602603	0.739877068910133\\
1.03603603603604	0.741908145127691\\
1.04604604604605	0.743930966479585\\
1.05605605605606	0.74594548724611\\
1.06606606606607	0.747951662547048\\
1.07607607607608	0.749949448343495\\
1.08608608608609	0.751938801439541\\
1.0960960960961	0.753919679483823\\
1.10610610610611	0.75589204097093\\
1.11611611611612	0.75785584524267\\
1.12612612612613	0.759811052489203\\
1.13613613613614	0.761757623750028\\
1.14614614614615	0.763695520914843\\
1.15615615615616	0.765624706724257\\
1.16616616616617	0.767545144770373\\
1.17617617617618	0.76945679949723\\
1.18618618618619	0.77135963620111\\
1.1961961961962	0.773253621030715\\
1.20620620620621	0.775138720987198\\
1.21621621621622	0.777014903924076\\
1.22622622622623	0.77888213854699\\
1.23623623623624	0.780740394413352\\
1.24624624624625	0.782589641931845\\
1.25625625625626	0.784429852361801\\
1.26626626626627	0.786260997812443\\
1.27627627627628	0.788083051242\\
1.28628628628629	0.789895986456693\\
1.2962962962963	0.791699778109591\\
1.30630630630631	0.793494401699342\\
1.31631631631632	0.795279833568775\\
1.32632632632633	0.797056050903381\\
1.33633633633634	0.798823031729664\\
1.34634634634635	0.800580754913375\\
1.35635635635636	0.802329200157614\\
1.36636636636637	0.804068348000821\\
1.37637637637638	0.80579817981464\\
1.38638638638639	0.807518677801665\\
1.3963963963964	0.809229824993068\\
1.40640640640641	0.81093160524611\\
1.41641641641642	0.812624003241535\\
1.42642642642643	0.814307004480851\\
1.43643643643644	0.815980595283491\\
1.44644644644645	0.817644762783872\\
1.45645645645646	0.819299494928333\\
1.46646646646647	0.820944780471969\\
1.47647647647648	0.822580608975348\\
1.48648648648649	0.824206970801134\\
1.4964964964965	0.825823857110587\\
1.50650650650651	0.827431259859972\\
1.51651651651652	0.829029171796854\\
1.52652652652653	0.830617586456301\\
1.53653653653654	0.832196498156973\\
1.54654654654655	0.833765901997121\\
1.55655655655656	0.835325793850486\\
1.56656656656657	0.836876170362097\\
1.57657657657658	0.838417028943978\\
1.58658658658659	0.839948367770759\\
1.5965965965966	0.841470185775192\\
1.60660660660661	0.842982482643581\\
1.61661661661662	0.844485258811116\\
1.62662662662663	0.845978515457124\\
1.63663663663664	0.847462254500233\\
1.64664664664665	0.848936478593441\\
1.65665665665666	0.85040119111912\\
1.66666666666667	0.851856396183916\\
1.67667667667668	0.853302098613584\\
1.68668668668669	0.854738303947739\\
1.6966966966967	0.856165018434526\\
1.70670670670671	0.85758224902522\\
1.71671671671672	0.858990003368744\\
1.72672672672673	0.860388289806121\\
1.73673673673674	0.86177711736485\\
1.74674674674675	0.863156495753214\\
1.75675675675676	0.864526435354521\\
1.76676676676677	0.865886947221274\\
1.77677677677678	0.867238043069281\\
1.78678678678679	0.868579735271696\\
1.7967967967968	0.869912036853007\\
1.80680680680681	0.871234961482951\\
1.81681681681682	0.872548523470381\\
1.82682682682683	0.873852737757074\\
1.83683683683684	0.875147619911478\\
1.84684684684685	0.876433186122411\\
1.85685685685686	0.877709453192706\\
1.86686686686687	0.878976438532805\\
1.87687687687688	0.880234160154304\\
1.88688688688689	0.881482636663452\\
1.8968968968969	0.882721887254599\\
1.90690690690691	0.883951931703614\\
1.91691691691692	0.885172790361238\\
1.92692692692693	0.88638448414642\\
1.93693693693694	0.887587034539596\\
1.94694694694695	0.888780463575938\\
1.95695695695696	0.889964793838569\\
1.96696696696697	0.891140048451736\\
1.97697697697698	0.892306251073959\\
1.98698698698699	0.893463425891142\\
1.996996996997	0.894611597609657\\
2.00700700700701	0.895750791449401\\
2.01701701701702	0.896881033136822\\
2.02702702702703	0.898002348897929\\
2.03703703703704	0.899114765451267\\
2.04704704704705	0.90021831000088\\
2.05705705705706	0.901313010229251\\
2.06706706706707	0.902398894290217\\
2.07707707707708	0.903475990801877\\
2.08708708708709	0.904544328839479\\
2.0970970970971	0.905603937928291\\
2.10710710710711	0.906654848036464\\
2.11711711711712	0.907697089567882\\
2.12712712712713	0.908730693355001\\
2.13713713713714	0.909755690651679\\
2.14714714714715	0.910772113126007\\
2.15715715715716	0.911779992853125\\
2.16716716716717	0.912779362308039\\
2.17717717717718	0.913770254358436\\
2.18718718718719	0.914752702257498\\
2.1971971971972	0.915726739636713\\
2.20720720720721	0.916692400498694\\
2.21721721721722	0.917649719209998\\
2.22722722722723	0.918598730493946\\
2.23723723723724	0.919539469423459\\
2.24724724724725	0.92047197141389\\
2.25725725725726	0.921396272215874\\
2.26726726726727	0.92231240790818\\
2.27727727727728	0.923220414890584\\
2.28728728728729	0.924120329876747\\
2.2972972972973	0.925012189887107\\
2.30730730730731	0.925896032241792\\
2.31731731731732	0.926771894553546\\
2.32732732732733	0.927639814720669\\
2.33733733733734	0.928499830919985\\
2.34734734734735	0.929351981599822\\
2.35735735735736	0.930196305473018\\
2.36736736736737	0.931032841509945\\
2.37737737737738	0.931861628931564\\
2.38738738738739	0.932682707202497\\
2.3973973973974	0.93349611602413\\
2.40740740740741	0.934301895327739\\
2.41741741741742	0.935100085267653\\
2.42742742742743	0.935890726214431\\
2.43743743743744	0.936673858748088\\
2.44744744744745	0.937449523651332\\
2.45745745745746	0.938217761902855\\
2.46746746746747	0.938978614670638\\
2.47747747747748	0.9397321233053\\
2.48748748748749	0.940478329333486\\
2.4974974974975	0.941217274451282\\
2.50750750750751	0.941949000517672\\
2.51751751751752	0.942673549548032\\
2.52752752752753	0.943390963707669\\
2.53753753753754	0.944101285305387\\
2.54754754754755	0.944804556787108\\
2.55755755755756	0.945500820729525\\
2.56756756756757	0.946190119833801\\
2.57757757757758	0.94687249691931\\
2.58758758758759	0.947547994917428\\
2.5975975975976	0.948216656865358\\
2.60760760760761	0.948878525900009\\
2.61761761761762	0.949533645251918\\
2.62762762762763	0.950182058239224\\
2.63763763763764	0.950823808261679\\
2.64764764764765	0.951458938794719\\
2.65765765765766	0.95208749338358\\
2.66766766766767	0.952709515637459\\
2.67767767767768	0.953325049223732\\
2.68768768768769	0.953934137862221\\
2.6976976976977	0.954536825319511\\
2.70770770770771	0.955133155403321\\
2.71771771771772	0.955723171956926\\
2.72772772772773	0.956306918853635\\
2.73773773773774	0.956884439991318\\
2.74774774774775	0.957455779286994\\
2.75775775775776	0.958020980671469\\
2.76776776776777	0.958580088084029\\
2.77777777777778	0.959133145467188\\
2.78778778778779	0.959680196761501\\
2.7977977977978	0.96022128590042\\
2.80780780780781	0.960756456805217\\
2.81781781781782	0.961285753379958\\
2.82782782782783	0.96180921950654\\
2.83783783783784	0.962326899039783\\
2.84784784784785	0.962838835802578\\
2.85785785785786	0.963345073581101\\
2.86786786786787	0.963845656120077\\
2.87787787787788	0.964340627118109\\
2.88788788788789	0.964830030223065\\
2.8978978978979	0.965313909027523\\
2.90790790790791	0.965792307064278\\
2.91791791791792	0.966265267801912\\
2.92792792792793	0.966732834640413\\
2.93793793793794	0.967195050906868\\
2.94794794794795	0.96765195985121\\
2.95795795795796	0.968103604642023\\
2.96796796796797	0.968550028362417\\
2.97797797797798	0.968991274005951\\
2.98798798798799	0.969427384472631\\
2.997997997998	0.969858402564958\\
3.00800800800801	0.970284370984044\\
3.01801801801802	0.970705332325785\\
3.02802802802803	0.971121329077098\\
3.03803803803804	0.971532403612221\\
3.04804804804805	0.971938598189067\\
3.05805805805806	0.972339954945646\\
3.06806806806807	0.972736515896547\\
3.07807807807808	0.973128322929481\\
3.08808808808809	0.973515417801883\\
3.0980980980981	0.973897842137577\\
3.10810810810811	0.974275637423502\\
3.11811811811812	0.974648845006499\\
3.12812812812813	0.975017506090159\\
3.13813813813814	0.97538166173173\\
3.14814814814815	0.975741352839085\\
3.15815815815816	0.976096620167755\\
3.16816816816817	0.976447504318012\\
3.17817817817818	0.976794045732026\\
3.18818818818819	0.977136284691068\\
3.1981981981982	0.977474261312782\\
3.20820820820821	0.977808015548515\\
3.21821821821822	0.9781375871807\\
3.22822822822823	0.978463015820308\\
3.23823823823824	0.978784340904351\\
3.24824824824825	0.979101601693449\\
3.25825825825826	0.979414837269447\\
3.26826826826827	0.979724086533101\\
3.27827827827828	0.980029388201815\\
3.28828828828829	0.980330780807436\\
3.2982982982983	0.980628302694106\\
3.30830830830831	0.980921992016174\\
3.31831831831832	0.981211886736163\\
3.32832832832833	0.981498024622789\\
3.33833833833834	0.98178044324904\\
3.34834834834835	0.982059179990315\\
3.35835835835836	0.982334272022607\\
3.36836836836837	0.982605756320749\\
3.37837837837838	0.982873669656712\\
3.38838838838839	0.983138048597959\\
3.3983983983984	0.983398929505848\\
3.40840840840841	0.983656348534095\\
3.41841841841842	0.983910341627281\\
3.42842842842843	0.984160944519421\\
3.43843843843844	0.984408192732578\\
3.44844844844845	0.984652121575532\\
3.45845845845846	0.9848927661425\\
3.46846846846847	0.985130161311903\\
3.47847847847848	0.985364341745189\\
3.48848848848849	0.985595341885706\\
3.4984984984985	0.985823195957612\\
3.50850850850851	0.986047937964855\\
3.51851851851852	0.986269601690182\\
3.52852852852853	0.986488220694206\\
3.53853853853854	0.986703828314522\\
3.54854854854855	0.986916457664859\\
3.55855855855856	0.987126141634294\\
3.56856856856857	0.9873329128865\\
3.57857857857858	0.987536803859044\\
3.58858858858859	0.987737846762731\\
3.5985985985986	0.98793607358099\\
3.60860860860861	0.988131516069309\\
3.61861861861862	0.988324205754703\\
3.62862862862863	0.988514173935241\\
3.63863863863864	0.988701451679601\\
3.64864864864865	0.988886069826674\\
3.65865865865866	0.989068058985209\\
3.66866866866867	0.989247449533499\\
3.67867867867868	0.989424271619104\\
3.68868868868869	0.989598555158621\\
3.6986986986987	0.989770329837486\\
3.70870870870871	0.989939625109819\\
3.71871871871872	0.990106470198307\\
3.72872872872873	0.990270894094127\\
3.73873873873874	0.990432925556896\\
3.74874874874875	0.990592593114676\\
3.75875875875876	0.990749925063995\\
3.76876876876877	0.990904949469922\\
3.77877877877878	0.991057694166161\\
3.78878878878879	0.991208186755194\\
3.7987987987988	0.991356454608447\\
3.80880880880881	0.991502524866497\\
3.81881881881882	0.991646424439307\\
3.82882882882883	0.991788180006496\\
3.83883883883884	0.991927818017642\\
3.84884884884885	0.992065364692615\\
3.85885885885886	0.992200846021938\\
3.86886886886887	0.992334287767186\\
3.87887887887888	0.992465715461405\\
3.88888888888889	0.99259515440957\\
3.8988988988989	0.992722629689066\\
3.90890890890891	0.992848166150195\\
3.91891891891892	0.992971788416721\\
3.92892892892893	0.99309352088643\\
3.93893893893894	0.993213387731722\\
3.94894894894895	0.993331412900235\\
3.95895895895896	0.993447620115481\\
3.96896896896897	0.993562032877523\\
3.97897897897898	0.99367467446366\\
3.98898898898899	0.993785567929154\\
3.998998998999	0.993894736107962\\
4.00900900900901	0.994002201613509\\
4.01901901901902	0.994107986839468\\
4.02902902902903	0.994212113960571\\
4.03903903903904	0.994314604933443\\
4.04904904904905	0.994415481497449\\
4.05905905905906	0.994514765175565\\
4.06906906906907	0.994612477275277\\
4.07907907907908	0.994708638889485\\
4.08908908908909	0.994803270897439\\
4.0990990990991	0.994896393965683\\
4.10910910910911	0.99498802854903\\
4.11911911911912	0.99507819489154\\
4.12912912912913	0.995166913027524\\
4.13913913913914	0.995254202782566\\
4.14914914914915	0.995340083774556\\
4.15915915915916	0.995424575414738\\
4.16916916916917	0.995507696908782\\
4.17917917917918	0.995589467257861\\
4.18918918918919	0.995669905259746\\
4.1991991991992	0.995749029509918\\
4.20920920920921	0.995826858402689\\
4.21921921921922	0.99590341013234\\
4.22922922922923	0.995978702694267\\
4.23923923923924	0.996052753886143\\
4.24924924924925	0.996125581309089\\
4.25925925925926	0.996197202368861\\
4.26926926926927	0.996267634277044\\
4.27927927927928	0.996336894052252\\
4.28928928928929	0.996404998521353\\
4.2992992992993	0.996471964320686\\
4.30930930930931	0.996537807897299\\
4.31931931931932	0.996602545510191\\
4.32932932932933	0.996666193231564\\
4.33933933933934	0.996728766948087\\
4.34934934934935	0.996790282362156\\
4.35935935935936	0.996850754993174\\
4.36936936936937	0.996910200178835\\
4.37937937937938	0.996968633076408\\
4.38938938938939	0.997026068664032\\
4.3993993993994	0.997082521742023\\
4.40940940940941	0.997138006934173\\
4.41941941941942	0.997192538689066\\
4.42942942942943	0.997246131281394\\
4.43943943943944	0.997298798813274\\
4.44944944944945	0.99735055521558\\
4.45945945945946	0.997401414249267\\
4.46946946946947	0.997451389506705\\
4.47947947947948	0.997500494413012\\
4.48948948948949	0.997548742227399\\
4.4994994994995	0.997596146044505\\
4.50950950950951	0.997642718795742\\
4.51951951951952	0.997688473250644\\
4.52952952952953	0.997733422018208\\
4.53953953953954	0.997777577548246\\
4.54954954954955	0.997820952132737\\
4.55955955955956	0.99786355790717\\
4.56956956956957	0.997905406851902\\
4.57957957957958	0.997946510793507\\
4.58958958958959	0.997986881406123\\
4.5995995995996	0.99802653021281\\
4.60960960960961	0.998065468586895\\
4.61961961961962	0.998103707753322\\
4.62962962962963	0.998141258790003\\
4.63963963963964	0.998178132629163\\
4.64964964964965	0.998214340058687\\
4.65965965965966	0.998249891723463\\
4.66966966966967	0.998284798126724\\
4.67967967967968	0.998319069631388\\
4.68968968968969	0.998352716461395\\
4.6996996996997	0.998385748703043\\
4.70970970970971	0.998418176306316\\
4.71971971971972	0.998450009086219\\
4.72972972972973	0.998481256724095\\
4.73973973973974	0.998511928768958\\
4.74974974974975	0.998542034638798\\
4.75975975975976	0.998571583621908\\
4.76976976976977	0.998600584878186\\
4.77977977977978	0.998629047440444\\
4.78978978978979	0.998656980215712\\
4.7997997997998	0.998684391986531\\
4.80980980980981	0.998711291412248\\
4.81981981981982	0.998737687030303\\
4.82982982982983	0.998763587257513\\
4.83983983983984	0.998789000391346\\
4.84984984984985	0.998813934611196\\
4.85985985985986	0.998838397979649\\
4.86986986986987	0.998862398443741\\
4.87987987987988	0.998885943836218\\
4.88988988988989	0.99890904187678\\
4.8998998998999	0.998931700173327\\
4.90990990990991	0.998953926223196\\
4.91991991991992	0.998975727414388\\
4.92992992992993	0.998997111026797\\
4.93993993993994	0.999018084233425\\
4.94994994994995	0.99903865410159\\
4.95995995995996	0.999058827594138\\
4.96996996996997	0.999078611570631\\
4.97997997997998	0.999098012788543\\
4.98998998998999	0.999117037904444\\
5	0.999135693475169\\
};
\addlegendentry{normal \acro{CDF}};

\end{axis}
\end{tikzpicture}%

  \caption{A comparison of the two sigmoid functions described in the
    text.  The normal \acro{CDF} curve in this example uses the
    transformation $\Phi(\sqrt{\frac{\pi}{8}} a)$, which ensures the
    slopes of the two curves are equal at the origin.}
  \label{sigmoids}
\end{figure}

With the choice of the sigmoid function, and an assumption that our
training labels $\vec{y}$ are generated independently given $\vec{w}$,
we have defined our likelihood $\Pr(\vec{y} \given \mat{X}, \vec{w})$:
\begin{equation}
  \label{likelihood}
  \Pr(\vec{y} \given \mat{X}, \vec{w})
  =
  \prod_{i = 1}^N
  \sigma(\vec{x}_i\trans \vec{w})^{y_i}
  \bigl(1 - \sigma(\vec{x}_i\trans \vec{w})\bigr)^{1 - y_i}.
\end{equation}
To verify this equation, notice that each $y_i$ will either be $0$ or
$1$, so exactly one of $y_i$ or $1 - y_i$ will be nonzero, which
picks out the correct contribution to the likelihood.

The traditional approach to logistic regression is to maximize the
likelihood of the training data as a function of the parameters
$\vec{w}$:
\begin{equation*}
  \hat{\vec{w}}
  =
  \argmax_{\vec{w}}
  \Pr(\vec{y} \given \mat{X}, \vec{w});
\end{equation*}
$\hat{\vec{w}}$ is therefore a maximum-likelihood estimator
(\acro{MLE}).  Unlike in linear regression, where there was a
closed-form expression for the maximum-likelihood estimator, there is
no such solution for logistic regression.  Things aren't too bad,
though, because it turns out that for logistic regression the negative
log-likelihood is convex and positive definite, which means there is a
unique global minimum (and therefore a unique \acro{MLE}).  There are
numerous off-the-shelf methods available for finding $\hat{\vec{w}}$:
steepest descent, Newton's method, etc.

\subsection*{Bayesian logistic regression}

A Bayesian approach to logistic regression requires us to select a
prior distribution for the parameters $\vec{w}$ and derive the
posterior distribution $p(\vec{w} \given \data)$.  For the former, we
will consider a multivariate Gaussian prior, identical to the one we used
for linear regression:
\begin{equation*}
  p(\vec{w})
  =
  \mc{N}(\vec{w}; \vec{\mu}, \mat{\Sigma}).
\end{equation*}

Now we apply Bayes' theorem to write down the desired posterior:
\begin{equation*}
  p(\vec{w} \given \data)
  =
  \frac{p(\vec{y} \given \mat{X}, \vec{w}) p(\vec{w})}
       {\int p(\vec{y} \given \mat{X}, \vec{w}) p(\vec{w}) \intd{\vec{w}}}.
  =
  \frac{p(\vec{y} \given \mat{X}, \vec{w}) p(\vec{w})}
       {p(\vec{y} \given \mat{X})}.
\end{equation*}
Unfortunately, the product of the Gaussian prior on $\vec{w}$ and the
likelihood \eqref{likelihood} (for either choice of sigmoid) does not
result in a posterior distribution in a nice parameteric family that
we know.  Likewise, the integral in the normalization constant (the
evidence) $p(\vec{y} \given \mat{X})$ is intractable as well.

How can we proceed?  There are two main approaches to continuing
Bayesian inference in such a situation.  The first is to use a
deterministic method to find an approximation to the posterior (that
will typically live inside a chosen parametric family).  The second is
to forgo a closed-form expression for the posterior and instead derive
an algorithm to draw \emph{samples} from the posterior distribution,
which we may use to, for example, make Monte Carlo estimates to
expectations.  Here we will consider the \emph{Laplace approximation,}
which is an example of the first type of approach.

\section*{Laplace Approximation to the Posterior}

Suppose we have an arbitrary parameter prior $p(\vec{\theta})$ and an
arbitrary likelihood $p(\data \given \vec{\theta})$, and wish to
approximate the posterior
\begin{equation*}
  p(\vec{\theta} \given \data)
  =
  \frac{1}{Z}
  p(\data \given \vec{\theta})
  p(\vec{\theta}),
\end{equation*}
where the normalization constant $Z$ is the unknown evidence.  We
define the following function:
\begin{equation*}
  \Psi(\vec{\theta})
  =
  \log p(\data \given \vec{\theta})
  +
  \log p(\vec{\theta}),
\end{equation*}
$\Psi$ is therefore the logarithm of the \emph{unnormalized} posterior
distribution.  The Laplace approximation is based on a Taylor
expansion to $\Psi$ around its maximum.  First, we find the maximum of
$\Psi$:
\begin{equation*}
  \hat{\vec{\theta}}
  =
  \argmax_{\vec{\theta}}
  \Psi(\vec{\theta}).
\end{equation*}
Notice that the point $\hat{\vec{\theta}}$ is a \emph{maximum a
  posteriori} (\acro{MAP}) approximation to the parameters.  Finding
$\hat{\vec{\theta}}$ can be done in a variety of ways, but in practice
it is usually fairly easy to find the gradient and Hessian of $\Psi$
with respect to $\vec{\theta}$ and use off-the-shelf optimization
routines.  This is another example of the mantra \emph{optimization is
  easier than integration.}

Once we have found $\hat{\vec{\theta}}$, we make a second-order Taylor
expansion to $\Psi$ around this point:
\begin{equation*}
  \Psi(\vec{\theta})
  \approx
  \Psi(\hat{\vec{\theta}})
  -
  \frac{1}{2}
  (\vec{\theta} - \hat{\vec{\theta}})\trans
  \mat{H}
  (\vec{\theta} - \hat{\vec{\theta}}),
\end{equation*}
where $\mat{H}$ is the Hessian of the negative log posterior evaluated
at $\hat{\vec{\theta}}$:
\begin{equation*}
  \mat{H}
  =
  -\nabla\nabla \Psi(\vec{\theta}) \bigr\rvert_{\vec{\theta} = \hat{\vec{\theta}}}.
\end{equation*}
Notice that the first-order term in the Taylor expansion vanishes
because we expand around a maximum, where the gradient is zero.
Exponentiating, we may derive an approximation to the (unnormalized)
posterior distribution:
\begin{equation}
  \label{almost_there}
  p(\vec{\theta} \given \data)
  \overset{\propto}{\sim}
  \exp\bigl(\Psi(\hat{\vec{\theta}})\bigr)
  \exp\biggl(-\frac{1}{2}
  (\vec{\theta} - \hat{\vec{\theta}})\trans
  \mat{H}
  (\vec{\theta} - \hat{\vec{\theta}})
  \biggr),
\end{equation}
which we recognize as being proportional to a Gaussian distribution!
The Laplace approximation therefore results in a normal approximation
to the posterior distribution:
\begin{equation*}
  p(\vec{\theta} \given \data)
  \approx
  \mc{N}(\vec{\theta}; \hat{\vec{\theta}}, \mat{H}\inv).
\end{equation*}
The approximation is a Gaussian centered on the mode of the posterior,
$\hat{\vec{\theta}}$, with covariance compelling the log of the
approximation to posterior to match the curvature of the true log
posterior at that point.

We note that the Laplace approximation also gives an approximation to
the normalizing constant $Z$.  In this case, it's simply a question of
which normalizing constant we had to use to get \eqref{almost_there}
to normalize.  A fairly straightforward calculation gives
\begin{equation*}
  Z
  =
  \int \Psi(\vec{\theta}) \intd{\vec{\theta}}
  \approx
  \int
  \exp\bigl(\Psi(\hat{\vec{\theta}})\bigr)
  \exp\biggl(-\frac{1}{2}
  (\vec{\theta} - \hat{\vec{\theta}})\trans
  \mat{H}
  (\vec{\theta} - \hat{\vec{\theta}})
  \biggr)
  =
  \exp\bigl(\Psi(\hat{\vec{\theta}})\bigr)
  \sqrt{
    \frac{(2\pi)^d}
         {\det \mat{H}}
   },
\end{equation*}
where $d$ is the dimension of $\vec{\theta}$.

\section*{Making Predictions}

Suppose we have obtained a Gaussian approximation to the posterior
distribution $p(\vec{w} \given \data)$; for example, applying the
Laplace approximation above gives $p(\vec{w} \given \data) \approx
\mc{N}(\vec{w}; \hat{\vec{w}}, \mat{H}\inv)$, where $\hat{\vec{w}}$ is
the \acro{MAP} approximation to the parameters and $\mat{H}$ is the
Hessian of the negative log posterior evaluated at $\hat{\vec{w}}$.

Suppose now that we are given a test input $\vec{x}_\ast$ and wish to
predict the binary label $y_\ast$.  In the Bayesian approach, we
marginalize the unknown parameters $\vec{w}$ to find the
\emph{predictive distribution}:
\begin{equation*}
  \Pr(y_\ast = 1 \given \vec{x}_\ast, \data)
  =
  \int
  \Pr(y_\ast = 1 \given \vec{x}_\ast, \vec{w})
  p(\vec{w} \given \data)
  \intd{\vec{w}}
  =
  \int
  \sigma(\vec{x}_\ast\trans \vec{w})
  p(\vec{w} \given \data)
  \intd{\vec{w}}.
\end{equation*}
Unfortunately, even with our Gaussian approximation to $p(\vec{w}
\given \data)$, this integral cannot be evaluated if we use the
logistic function in the role of the sigmoid $\sigma$.  We can,
however, compute the integral when using the normal \acro{CDF} for
$\sigma$:
\begin{equation*}
  \Pr(y_\ast = 1 \given \vec{x}_\ast, \data)
  =
  \int
  \Phi(\vec{x}_\ast\trans \vec{w})
  \mc{N}(\vec{w}; \hat{\vec{w}}, \mat{H})
  \intd{\vec{w}}.
\end{equation*}
To proceed, we define the scalar value $a = \vec{x}_\ast\trans
\vec{w}$ and rewrite this as
\begin{equation*}
  \Pr(y_\ast = 1 \given \vec{x}_\ast, \data)
  =
  \int_{-\infty}^\infty
  \Phi(a)
  p(a \given \data)
  \intd{\data}.
\end{equation*}
Notice that $a$ is a linear transformation of the Gaussian-distributed
$\vec{w}$; therefore, $a$ has a Gaussian distribution:
\begin{equation*}
  p(a \given \data)
  =
  \mc{N}(a; \mu_{a \given \data}, \sigma^2_{a \given \data}),
\end{equation*}
where
\begin{equation*}
  \mu_{a \given \data} = \vec{x}_\ast\trans \hat{\vec{w}};
  \qquad
  \sigma^2_{a \given \data} = \vec{x}_\ast\trans \mat{H} \vec{x}_\ast.
\end{equation*}
Now we may finally compute the integral:
\begin{equation*}
  \Pr(y_\ast = 1 \given \vec{x}_\ast, \data)
  =
  \int_{-\infty}^\infty
  \Phi(a)
  \,
  \mc{N}(a; \mu_{a \given \data}, \sigma^2_{a \given \data})
  \intd{\data}
  =
  \Phi
  \Biggl(
  \frac{\mu_{a \given \data}}{\sqrt{1 + \sigma^2_{a \given \data}}}
  \Biggr).
\end{equation*}
Notice that $\Phi(\mu_{a \given \data})$ would be the estimate we
would make using the \acro{MAP} $\hat{\vec{w}}$ as a plug-in
estimator.  The $\sqrt{1 + \sigma^2_{a \given \data}}$ term has the
effect of making our prediction less confident (that is, closer to
\nicefrac{1}{2}) according to our uncertainty in the value of $a =
\vec{x}_\ast\trans \vec{w}$.  This procedure is sometimes called
\emph{moderation,} because we force our predictions to be more
moderate than we would have using a plug-in point estimate of
$\vec{w}$.

We also note that if we only want to make point predictions of
$y_\ast$ using the 0--1 loss function, we only need to know which
class is more probable (this was a general result from our discussion
of Bayesian decision theory).  In this case, the moderation has no
effect on our ultimate predictions (you can check that we never change
which side of \nicefrac{1}{2} the final probability is), and we may
instead simply find $\hat{\vec{w}}$.  This is similar to the result we
had in linear regression, where we could simply find the \acro{MAP}
estimator for $\vec{w}$ if we only ultimately cared about point
predictions under a squared loss.

With loss functions different from the 0--1 loss, however, the
uncertainty in $\vec{w}$ can indeed be important.

\end{document}
