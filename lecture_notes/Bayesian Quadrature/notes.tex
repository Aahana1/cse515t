\documentclass{article}

\usepackage[T1]{fontenc}
\usepackage[osf]{libertine}
\usepackage[scaled=0.8]{beramono}
\usepackage[margin=1.5in]{geometry}
\usepackage{url}
\usepackage{booktabs}
\usepackage{amsmath}
\usepackage{amssymb}
\usepackage{nicefrac}
\usepackage{microtype}
\usepackage{subcaption}
\usepackage{bm}

\usepackage{amsthm}
\newtheorem{defn}{Definition}

\usepackage{sectsty}
\sectionfont{\large}
\subsectionfont{\normalsize}

\usepackage{titlesec}
\titlespacing{\section}{0pt}{10pt plus 2pt minus 2pt}{0pt plus 2pt minus 0pt}
\titlespacing{\subsection}{0pt}{5pt plus 2pt minus 2pt}{0pt plus 2pt minus 0pt}

\usepackage{pgfplots}
\pgfplotsset{
  compat=newest,
  plot coordinates/math parser=false,
  tick label style={font=\footnotesize, /pgf/number format/fixed},
  label style={font=\small},
  legend style={font=\small},
  every axis/.append style={
    tick align=outside,
    clip mode=individual,
    scaled ticks=false,
    thick,
    tick style={semithick, black}
  }
}

\pgfkeys{/pgf/number format/.cd, set thousands separator={\,}}

\usepgfplotslibrary{external}
\tikzexternalize[prefix=tikz/]

\newlength\figurewidth
\newlength\figureheight

\setlength{\figurewidth}{12cm}
\setlength{\figureheight}{6cm}

\newlength\squarefigurewidth
\newlength\squarefigureheight

\setlength{\squarefigurewidth}{4cm}
\setlength{\squarefigureheight}{4cm}

\newlength\smallsquarefigurewidth
\newlength\smallsquarefigureheight

\setlength{\smallsquarefigurewidth}{3.25cm}
\setlength{\smallsquarefigureheight}{3.25cm}

\newlength\smallfigurewidth
\newlength\smallfigureheight

\setlength{\smallfigurewidth}{6.25cm}
\setlength{\smallfigureheight}{4cm}

\setlength{\parindent}{0pt}
\setlength{\parskip}{1ex}

\newcommand{\acro}[1]{\textsc{\MakeLowercase{#1}}}
\newcommand{\given}{\mid}
\newcommand{\mc}[1]{\mathcal{#1}}
\newcommand{\data}{\mc{D}}
\newcommand{\intd}[1]{\,\mathrm{d}{#1}}
\newcommand{\inv}{^{-1}}
\newcommand{\trans}{^\top}
\newcommand{\mat}[1]{\bm{\mathrm{#1}}}
\renewcommand{\vec}[1]{\bm{\mathrm{#1}}}
\newcommand{\R}{\mathbb{R}}
\renewcommand{\epsilon}{\varepsilon}
\newcommand{\Exp}{\mathbb{E}}

\DeclareMathOperator{\var}{var}
\DeclareMathOperator{\cov}{cov}
\DeclareMathOperator{\diag}{diag}
\DeclareMathOperator*{\argmin}{arg\,min}
\DeclareMathOperator*{\argmax}{arg\,max}

\begin{document}

\section*{Conditioning on Outputs of Linear Operators}

Suppose we have a function $f\colon \mc{X} \to \R$ with a Gaussian
process prior distribution:
\begin{equation*}
  p(f) = \mc{GP}(f; \mu, K).
\end{equation*}
We have discussed how to perform inference about $f$ when given
(noisy) observations of the function at a set of points $\mat{X}$:
$\data = (\mat{X}, \vec{y})$.  Here we are going to expand the types
of observations we may use during \acro{GP} inference.

\subsection*{Functionals and linear functionals}

Specifically, we are going to consider so-called \emph{linear
  functionals} of $f$.  A \emph{functional} is a function $L[f]$ that
takes as an input a function $f$ and returns a scalar.  (Functionals
are sometimes called ``functions of functions.'')  A very simple
example of a functional is the \emph{point-evaluation functional.}
Let $x \in \mc{X}$ be an arbitrary fixed point in the domain.  We
define a corresponding functional $L_x$ by
\begin{equation*}
  f \mapsto L_x[f] = f(x).
\end{equation*}
So, given a function $f$, the point-evaluation functional $L_x$ simply
evaluates $f$ at $x$ and returns the result.  This is a functional we
are very accustomed to using.

A functional is said to be $\emph{linear}$ when it satisfies a simple
linearity property.  Specifically, let $a \in \R$ be an arbitrary
scalar constant and let $f$ and $g$ be two arbitrary functions.  A
functional $L$ is linear if the following equality always holds:
\begin{equation*}
  L[af + g] = aL[f] + L[g].
\end{equation*}
It is easy to see that the point-evaluation functional $L_x$ is
linear:
\begin{equation*}
  L_x[af + g]
  =
  (af + g)(x)
  =
  af(x) + g(x)
  =
  aL_x[f] + L_x[g].
\end{equation*}

There are several other quite-common linear functionals that we are
familiar with.  The two we will discuss here are integration against
an arbitrary function $p(x)$:
\begin{equation*}
  f \mapsto I_p[f] = \int_{\mc{X}} f(x) p(x) \intd{x},
\end{equation*}
and (partial) differentiation at a point $x$:
\begin{equation*}
  f \mapsto D_{x, i}[f] = \frac{\partial f(z)}{\partial z_i} \biggr\rvert_{z = x}.
\end{equation*}

\subsection*{Conditioning on linear functionals}

It turns out that we can once again exploit the closure of the
Gaussian distribution to linear transformations to condition a
\acro{GP} on $f$ on the observation of any linear functional of $f$!
This will allow us to both perform inference about $f$ given
observations of, for example, derivatives of $f$, and also to perform
inference about linear functionals of $f$ directly.  This will provide
us with a Bayesian mechanism for estimating integrals (a task
traditionally called \emph{quadrature}).

Suppose we have an unknown function $f\colon \mc{X} \to \R$ with the
Gaussian process prior above:
\begin{equation*}
  p(f) = \mc{GP}(f; \mu, K),
\end{equation*}
and let $L$ be a linear functional.  We will write $\ell = L[f]$.
Just as Gaussian distributions are closed under linear
transformations, so are Gaussian processes closed under the evaluation
of linear functionals!  The prior distribution for $\ell$ is a
Gaussian distribution:
\begin{equation*}
  p(\ell)
  =
  \mc{N}\bigl(\ell; L[\mu], L^2[K]\bigr)
\end{equation*}
where
\begin{equation*}
  L^2[K]
  =
  L\Bigl[L\bigl[K(\cdot, x')\bigr]\Bigr]
  =
  L\Bigl[L\bigl[K(x, \cdot)\bigr]\Bigr].
\end{equation*}
This result is essentially equivalent to the result for linear
transformations of Gaussian-distributed vectors we have been using
thus far, written with different notation.  Notice also that if we
consider the point-evaluation functional $L_x$, we recover a basic
result:
\begin{equation*}
  p\bigl(f(x) \given x\bigr)
  =
  \mc{N}\bigl(f(x); L_x[\mu], L_x^2[K]\bigr);
  =
  \mc{N}\bigl(f(x); \mu(x), K(x, x)\bigr).
\end{equation*}
Considering the integration functional, we obtain a perhaps
more-interesting result:
\begin{equation*}
  p\biggl(\int f(x) p(x) \intd{x} \biggr)
  =
  \mc{N}\biggl(\int f(x) p(x) \intd{x}; \int \mu(x) p(x) \intd{x}, \iint K(x, x') p(x) p(x') \intd{x} \intd{x'} \biggr).
\end{equation*}
Therefore a Gaussian process distribution on $f$ implies a Gaussian
distribution on its integral against an arbitrary function $p(x)$!
Further, the problem of estimating the integral of the (perhaps quite
complicated) function $f$ has been reduced to the perhaps-simpler
problem of integrating the mean and covariance functions $\mu$ and
$K$.  This is the main idea behind \emph{Bayesian quadrature,} also
called \emph{Bayesian Monte Carlo.}

Given an observation of $L[f] = \ell$, we may condition our prior on
this observation in a manner equivalent to that used to derive the
posterior distribution of $f$.  Let $\mat{X}$ be an arbitrary set of
input locations.  As before, we write the joint distribution between
$\ell$ and $\vec{f} = f(\mat{X})$:
\begin{equation*}
  p\Biggl(
  \begin{bmatrix}
    \ell
    \\
    \vec{f}
  \end{bmatrix}
  \given
  \vec{X}
  \Biggr)
  =
  \mc{N}
  \Biggl(
  \begin{bmatrix}
    \ell
    \\
    \vec{f}
  \end{bmatrix}
  ;
  \begin{bmatrix}
    L[\mu]
    \\
    \vec{\mu}
  \end{bmatrix}
  ,
  \begin{bmatrix}
    L^2[K] & \text{?}
    \\
    \text{?} & \mat{K}
  \end{bmatrix}
  \Biggr),
\end{equation*}
where we have defined:
\begin{equation*}
  \vec{\mu} = \mu(\mat{X})
  \qquad
  \mat{K} = K(\mat{X}, \mat{X}).
\end{equation*}
To fill in the missing observations, we need to know the covariance
between $\ell$ and the $i$th function value $f_i = f(\vec{x}_i)$.  Here
we can exploit the linearity of covariance:
\begin{equation*}
  \cov(\ell, f_i)
  =
  \cov\bigl(L[f], L_{\vec{x}_i}[f]\bigr)
  =
  L\Bigl[L_{\vec{x}_i}\bigl[
      \cov(f, f)
  \bigr]\Bigr]
  =
  L\Bigl[L_{\vec{x}_i}\bigl[
      K
  \bigr]\Bigr]
  =
  L\bigl[K(\cdot, \vec{x}_i)\bigr].
\end{equation*}
Now we have the general result
\begin{equation*}
  p\Biggl(
  \begin{bmatrix}
    \ell
    \\
    \vec{f}
  \end{bmatrix}
  \given
  \vec{X}
  \Biggr)
  =
  \mc{N}
  \Biggl(
  \begin{bmatrix}
    \ell
    \\
    \vec{f}
  \end{bmatrix}
  ;
  \begin{bmatrix}
    L[\mu]
    \\
    \vec{\mu}
  \end{bmatrix}
  ,
  \begin{bmatrix}
    L^2[K] & L\bigl[K(\cdot, \mat{X})\bigr]
    \\
    L\bigl[K(\mat{X}, \cdot)\bigr] & \mat{K}
  \end{bmatrix}
  \Biggr).
\end{equation*}
Finally, we may condition this joint distribution on the observed
value $\ell = L[f]$ to find the posterior of $\vec{f}$, which will be
an updated multivariate Gaussian distribution.  Because the set of
points $\mat{X}$ was arbitrary, we may conclude that the posterior
distribution is also a Gaussian process.  The posterior mean and
covariance functions are
\begin{align*}
  \mu_{f \given \ell}(\vec{x})
  &=
  \mu(\vec{x})
  +
  \frac{L\bigl[K(\cdot, \mat{X})\bigr]}{L^2[K]}
  \bigl(\ell - L[\mu]);
  \\
  K_{f \given \ell}(\vec{x}, \vec{x}')
  &=
  K(\vec{x}, \vec{x}')
  -
  \frac{L\bigl[K(\cdot, \mat{X})\bigr]L\bigl[K(\mat{X}, \cdot)\bigr]}{L^2[K]}.
\end{align*}
We can easily extend this result to include multiple observations of
functionals and also to incorporate Gaussian noise on each of these
observations.

An example is shown in Figure \ref{example}, where we condition a
Gaussian process prior on the integral observation $\int_{0}^{10} f(x)
\intd{x} = 5$.  Notice that the posterior samples all have integral
exactly equal to 5.

\begin{figure}
  \centering
  % This file was created by matlab2tikz.
% Minimal pgfplots version: 1.3
%
\tikzsetnextfilename{samples_example_1}
\definecolor{mycolor1}{rgb}{0.98824,0.57255,0.44706}%
\definecolor{mycolor2}{rgb}{0.98431,0.41569,0.29020}%
\definecolor{mycolor3}{rgb}{0.93725,0.23137,0.17255}%
\definecolor{mycolor4}{rgb}{0.65098,0.80784,0.89020}%
\definecolor{mycolor5}{rgb}{0.12157,0.47059,0.70588}%
%
\begin{tikzpicture}

\begin{axis}[%
width=0.95092\figurewidth,
height=\figureheight,
at={(0\figurewidth,0\figureheight)},
scale only axis,
xmin=0,
xmax=10,
xlabel={$x$},
ymin=-3,
ymax=3,
axis x line*=bottom,
axis y line*=left,
legend style={legend cell align=left,align=left,draw=white!15!black},
legend style={legend columns=-1, draw=none}, reverse legend
]
\addplot [color=mycolor1,solid,forget plot]
  table[row sep=crcr]{%
0	-0.356707751289252\\
0.01001001001001	-0.370042961505003\\
0.02002002002002	-0.383428680433551\\
0.03003003003003	-0.396860604294263\\
0.04004004004004	-0.410334475729796\\
0.0500500500500501	-0.423846086702714\\
0.0600600600600601	-0.437391281939299\\
0.0700700700700701	-0.450965963364064\\
0.0800800800800801	-0.464566091593718\\
0.0900900900900901	-0.478187692304172\\
0.1001001001001	-0.491826855863158\\
0.11011011011011	-0.505479743815741\\
0.12012012012012	-0.519142589596456\\
0.13013013013013	-0.532811701310166\\
0.14014014014014	-0.546483467137489\\
0.15015015015015	-0.560154354734848\\
0.16016016016016	-0.57382091633006\\
0.17017017017017	-0.587479789483759\\
0.18018018018018	-0.60112770002143\\
0.19019019019019	-0.614761465024442\\
0.2002002002002	-0.628377993098458\\
0.21021021021021	-0.641974288772261\\
0.22022022022022	-0.655547451837662\\
0.23023023023023	-0.66909468051711\\
0.24024024024024	-0.682613273234231\\
0.25025025025025	-0.696100628249111\\
0.26026026026026	-0.70955424757447\\
0.27027027027027	-0.722971736413654\\
0.28028028028028	-0.736350804728642\\
0.29029029029029	-0.749689267839638\\
0.3003003003003	-0.76298504807034\\
0.31031031031031	-0.776236174680745\\
0.32032032032032	-0.789440783835343\\
0.33033033033033	-0.802597121370997\\
0.34034034034034	-0.815703539654821\\
0.35035035035035	-0.828758500818559\\
0.36036036036036	-0.841760575359889\\
0.37037037037037	-0.85470844186264\\
0.38038038038038	-0.867600887215754\\
0.39039039039039	-0.880436806258112\\
0.4004004004004	-0.893215201041646\\
0.41041041041041	-0.90593518100492\\
0.42042042042042	-0.918595961056998\\
0.43043043043043	-0.931196861700559\\
0.44044044044044	-0.943737307239326\\
0.45045045045045	-0.956216825917655\\
0.46046046046046	-0.968635047181311\\
0.47047047047047	-0.980991701902679\\
0.48048048048048	-0.993286619936162\\
0.49049049049049	-1.00551972963598\\
0.500500500500501	-1.01769105443869\\
0.510510510510511	-1.02980071297867\\
0.520520520520521	-1.04184891682831\\
0.530530530530531	-1.05383596741236\\
0.540540540540541	-1.06576225656718\\
0.550550550550551	-1.07762826094085\\
0.560560560560561	-1.08943454305942\\
0.570570570570571	-1.1011817471379\\
0.580580580580581	-1.11287059728401\\
0.590590590590591	-1.12450189543097\\
0.600600600600601	-1.1360765179195\\
0.610610610610611	-1.14759541427313\\
0.620620620620621	-1.15905960266512\\
0.630630630630631	-1.17047016974792\\
0.640640640640641	-1.18182826517252\\
0.650650650650651	-1.19313510038869\\
0.660660660660661	-1.20439194611792\\
0.670670670670671	-1.21560012773666\\
0.680680680680681	-1.22676102371965\\
0.690690690690691	-1.23787606278924\\
0.700700700700701	-1.2489467182355\\
0.710710710710711	-1.25997450910981\\
0.720720720720721	-1.27096099275808\\
0.730730730730731	-1.28190776568388\\
0.740740740740741	-1.29281645610387\\
0.750750750750751	-1.30368872426206\\
0.760760760760761	-1.31452625805764\\
0.770770770770771	-1.32533076756353\\
0.780780780780781	-1.33610398590198\\
0.790790790790791	-1.34684766320365\\
0.800800800800801	-1.35756356261924\\
0.810810810810811	-1.36825345998258\\
0.820820820820821	-1.37891913765145\\
0.830830830830831	-1.38956238277605\\
0.840840840840841	-1.40018498315019\\
0.850850850850851	-1.41078872472715\\
0.860860860860861	-1.42137538811579\\
0.870870870870871	-1.43194674524203\\
0.880880880880881	-1.44250455539564\\
0.890890890890891	-1.45305056327183\\
0.900900900900901	-1.46358649444052\\
0.910910910910911	-1.47411405383092\\
0.920920920920921	-1.48463492186808\\
0.930930930930931	-1.49515075020397\\
0.940940940940941	-1.50566316048684\\
0.950950950950951	-1.51617374027425\\
0.960960960960961	-1.52668404066322\\
0.970970970970971	-1.53719557223126\\
0.980980980980981	-1.54770980455295\\
0.990990990990991	-1.55822816060295\\
1.001001001001	-1.56875201607183\\
1.01101101101101	-1.57928269425517\\
1.02102102102102	-1.589821467124\\
1.03103103103103	-1.60036954950907\\
1.04104104104104	-1.6109280969887\\
1.05105105105105	-1.62149820626017\\
1.06106106106106	-1.63208090766055\\
1.07107107107107	-1.64267716881978\\
1.08108108108108	-1.65328788710589\\
1.09109109109109	-1.66391389098738\\
1.1011011011011	-1.67455593666758\\
1.11111111111111	-1.68521470546659\\
1.12112112112112	-1.69589080298331\\
1.13113113113113	-1.70658475677354\\
1.14114114114114	-1.71729701539823\\
1.15115115115115	-1.72802794410645\\
1.16116116116116	-1.73877782676386\\
1.17117117117117	-1.74954686203778\\
1.18118118118118	-1.76033516275783\\
1.19119119119119	-1.77114275464956\\
1.2012012012012	-1.78196957471005\\
1.21121121121121	-1.79281547043606\\
1.22122122122122	-1.80368019798368\\
1.23123123123123	-1.81456342295279\\
1.24124124124124	-1.82546471718175\\
1.25125125125125	-1.83638356060067\\
1.26126126126126	-1.8473193377865\\
1.27127127127127	-1.8582713402475\\
1.28128128128128	-1.86923876260244\\
1.29129129129129	-1.88022070639199\\
1.3013013013013	-1.89121617588773\\
1.31131131131131	-1.90222408075171\\
1.32132132132132	-1.91324323350523\\
1.33133133133133	-1.92427235204171\\
1.34134134134134	-1.93531005803491\\
1.35135135135135	-1.94635487779478\\
1.36136136136136	-1.95740524249767\\
1.37137137137137	-1.96845948929387\\
1.38138138138138	-1.97951585967681\\
1.39139139139139	-1.9905725035717\\
1.4014014014014	-2.00162747687836\\
1.41141141141141	-2.01267874386191\\
1.42142142142142	-2.02372417862767\\
1.43143143143143	-2.03476156392144\\
1.44144144144144	-2.04578859485573\\
1.45145145145145	-2.05680287836839\\
1.46146146146146	-2.06780193660505\\
1.47147147147147	-2.07878320453719\\
1.48148148148148	-2.08974403616392\\
1.49149149149149	-2.10068170215031\\
1.5015015015015	-2.11159339476561\\
1.51151151151151	-2.12247622733823\\
1.52152152152152	-2.13332723664402\\
1.53153153153153	-2.14414338726771\\
1.54154154154154	-2.15492156851213\\
1.55155155155155	-2.1656586025714\\
1.56156156156156	-2.17635124277117\\
1.57157157157157	-2.1869961772217\\
1.58158158158158	-2.19759003110459\\
1.59159159159159	-2.20812936935259\\
1.6016016016016	-2.21861069939957\\
1.61161161161161	-2.22903047314376\\
1.62162162162162	-2.23938508996733\\
1.63163163163163	-2.24967090002152\\
1.64164164164164	-2.25988420732664\\
1.65165165165165	-2.27002127162375\\
1.66166166166166	-2.28007831112129\\
1.67167167167167	-2.29005150850499\\
1.68168168168168	-2.29993701081712\\
1.69169169169169	-2.30973093391033\\
1.7017017017017	-2.3194293659728\\
1.71171171171171	-2.32902837122671\\
1.72172172172172	-2.33852399192387\\
1.73173173173173	-2.34791225288754\\
1.74174174174174	-2.35718916475447\\
1.75175175175175	-2.36635072808142\\
1.76176176176176	-2.37539293563414\\
1.77177177177177	-2.38431177682276\\
1.78178178178178	-2.39310324135833\\
1.79179179179179	-2.4017633233815\\
1.8018018018018	-2.41028802420031\\
1.81181181181181	-2.41867335585548\\
1.82182182182182	-2.42691534676223\\
1.83183183183183	-2.43501004335235\\
1.84184184184184	-2.44295351517035\\
1.85185185185185	-2.45074185870631\\
1.86186186186186	-2.45837119959716\\
1.87187187187187	-2.46583769883944\\
1.88188188188188	-2.47313755509431\\
1.89189189189189	-2.48026700881739\\
1.9019019019019	-2.48722234582237\\
1.91191191191191	-2.49399990214239\\
1.92192192192192	-2.50059606625112\\
1.93193193193193	-2.50700728499184\\
1.94194194194194	-2.51323006433663\\
1.95195195195195	-2.5192609756215\\
1.96196196196196	-2.52509665878864\\
1.97197197197197	-2.53073382522603\\
1.98198198198198	-2.53616926179519\\
1.99199199199199	-2.54139983461587\\
2.002002002002	-2.54642249223129\\
2.01201201201201	-2.55123426940987\\
2.02202202202202	-2.5558322913907\\
2.03203203203203	-2.56021377424954\\
2.04204204204204	-2.56437603186435\\
2.05205205205205	-2.56831647757804\\
2.06206206206206	-2.57203262595111\\
2.07207207207207	-2.57552209866635\\
2.08208208208208	-2.57878262554644\\
2.09209209209209	-2.5818120477808\\
2.1021021021021	-2.5846083211063\\
2.11211211211211	-2.58716951811416\\
2.12212212212212	-2.5894938322836\\
2.13213213213213	-2.59157957828166\\
2.14214214214214	-2.59342519585177\\
2.15215215215215	-2.59502925145033\\
2.16216216216216	-2.59639044183269\\
2.17217217217217	-2.597507594005\\
2.18218218218218	-2.59837966944447\\
2.19219219219219	-2.59900576376174\\
2.2022022022022	-2.59938510990081\\
2.21221221221221	-2.59951708018687\\
2.22222222222222	-2.59940118595539\\
2.23223223223223	-2.59903708012874\\
2.24224224224224	-2.59842455873379\\
2.25225225225225	-2.5975635605101\\
2.26226226226226	-2.59645416997607\\
2.27227227227227	-2.59509661582306\\
2.28228228228228	-2.59349127386529\\
2.29229229229229	-2.59163866522604\\
2.3023023023023	-2.58953945907393\\
2.31231231231231	-2.58719447002692\\
2.32232232232232	-2.58460466133422\\
2.33233233233233	-2.58177114213477\\
2.34234234234234	-2.57869516826733\\
2.35235235235235	-2.57537814234467\\
2.36236236236236	-2.57182161200428\\
2.37237237237237	-2.5680272699418\\
2.38238238238238	-2.56399695305011\\
2.39239239239239	-2.55973264024517\\
2.4024024024024	-2.55523645329987\\
2.41241241241241	-2.55051065350563\\
2.42242242242242	-2.54555764108269\\
2.43243243243243	-2.54037995309227\\
2.44244244244244	-2.5349802623287\\
2.45245245245245	-2.52936137396406\\
2.46246246246246	-2.5235262254588\\
2.47247247247247	-2.51747788220612\\
2.48248248248248	-2.51121953553261\\
2.49249249249249	-2.50475450174881\\
2.5025025025025	-2.49808621654942\\
2.51251251251251	-2.49121823540091\\
2.52252252252252	-2.4841542277248\\
2.53253253253253	-2.47689797510144\\
2.54254254254254	-2.46945336746275\\
2.55255255255255	-2.4618244010854\\
2.56256256256256	-2.45401517232918\\
2.57257257257257	-2.44602987655627\\
2.58258258258258	-2.4378728027615\\
2.59259259259259	-2.42954833068931\\
2.6026026026026	-2.42106092635233\\
2.61261261261261	-2.4124151376852\\
2.62262262262262	-2.40361558998249\\
2.63263263263263	-2.39466698326521\\
2.64264264264264	-2.38557408605669\\
2.65265265265265	-2.37634173203502\\
2.66266266266266	-2.36697481422066\\
2.67267267267267	-2.35747828207181\\
2.68268268268268	-2.34785713491606\\
2.69269269269269	-2.33811641896552\\
2.7027027027027	-2.32826122062851\\
2.71271271271271	-2.31829666325972\\
2.72272272272272	-2.30822790095517\\
2.73273273273273	-2.29806011474398\\
2.74274274274274	-2.2877985064238\\
2.75275275275275	-2.27744829407865\\
2.76276276276276	-2.26701470758832\\
2.77277277277277	-2.25650298161042\\
2.78278278278278	-2.24591835305925\\
2.79279279279279	-2.23526605414151\\
2.8028028028028	-2.22455130854304\\
2.81281281281281	-2.21377932491344\\
2.82282282282282	-2.20295529245262\\
2.83283283283283	-2.19208437773843\\
2.84284284284284	-2.18117171528455\\
2.85285285285285	-2.17022240763039\\
2.86286286286286	-2.15924151665486\\
2.87287287287287	-2.14823406147711\\
2.88288288288288	-2.13720501054663\\
2.89289289289289	-2.12615927956196\\
2.9029029029029	-2.11510172657107\\
2.91291291291291	-2.10403714524423\\
2.92292292292292	-2.09297026190763\\
2.93293293293293	-2.08190573178912\\
2.94294294294294	-2.07084813301585\\
2.95295295295295	-2.0598019635048\\
2.96296296296296	-2.04877163605725\\
2.97297297297297	-2.03776147432296\\
2.98298298298298	-2.02677571076604\\
2.99299299299299	-2.01581847885818\\
3.003003003003	-2.00489381377367\\
3.01301301301301	-1.99400564549036\\
3.02302302302302	-1.98315779786208\\
3.03303303303303	-1.9723539834063\\
3.04304304304304	-1.96159780032349\\
3.05305305305305	-1.95089273148108\\
3.06306306306306	-1.94024213863446\\
3.07307307307307	-1.92964926163084\\
3.08308308308308	-1.91911721567818\\
3.09309309309309	-1.90864898810131\\
3.1031031031031	-1.89824743635864\\
3.11311311311311	-1.88791528679629\\
3.12312312312312	-1.87765513182319\\
3.13313313313313	-1.8674694280501\\
3.14314314314314	-1.85736049533829\\
3.15315315315315	-1.84733051455024\\
3.16316316316316	-1.83738152733685\\
3.17317317317317	-1.82751543407222\\
3.18318318318318	-1.81773399326599\\
3.19319319319319	-1.80803882103112\\
3.2032032032032	-1.79843139030511\\
3.21321321321321	-1.78891303018373\\
3.22322322322322	-1.7794849260914\\
3.23323323323323	-1.77014811971577\\
3.24324324324324	-1.76090350852359\\
3.25325325325325	-1.7517518466841\\
3.26326326326326	-1.74269374485668\\
3.27327327327327	-1.7337296719042\\
3.28328328328328	-1.72485995416032\\
3.29329329329329	-1.71608477729274\\
3.3033033033033	-1.70740418741818\\
3.31331331331331	-1.6988180912583\\
3.32332332332332	-1.69032625959027\\
3.33333333333333	-1.68192832593355\\
3.34334334334334	-1.67362379164284\\
3.35335335335335	-1.66541202438878\\
3.36336336336336	-1.65729226205657\\
3.37337337337337	-1.64926361483975\\
3.38338338338338	-1.64132506614522\\
3.39339339339339	-1.63347547713614\\
3.4034034034034	-1.62571358710008\\
3.41341341341341	-1.61803801627441\\
3.42342342342342	-1.61044727063346\\
3.43343343343343	-1.60293974204657\\
3.44344344344344	-1.59551371320609\\
3.45345345345345	-1.58816735961467\\
3.46346346346346	-1.58089875287886\\
3.47347347347347	-1.5737058641831\\
3.48348348348348	-1.56658656795244\\
3.49349349349349	-1.55953864344122\\
3.5035035035035	-1.55255978095343\\
3.51351351351351	-1.54564758367532\\
3.52352352352352	-1.53879957081319\\
3.53353353353353	-1.53201318301088\\
3.54354354354354	-1.52528578532085\\
3.55355355355355	-1.51861466925483\\
3.56356356356356	-1.51199705969778\\
3.57357357357357	-1.50543011672934\\
3.58358358358358	-1.49891094029007\\
3.59359359359359	-1.49243657301136\\
3.6036036036036	-1.48600400553054\\
3.61361361361361	-1.4796101804498\\
3.62362362362362	-1.47325199486237\\
3.63363363363363	-1.46692630659283\\
3.64364364364364	-1.46062993549644\\
3.65365365365365	-1.45435966962937\\
3.66366366366366	-1.4481122682596\\
3.67367367367367	-1.44188446692166\\
3.68368368368368	-1.43567297902504\\
3.69369369369369	-1.42947450293786\\
3.7037037037037	-1.42328572300307\\
3.71371371371371	-1.41710331550462\\
3.72372372372372	-1.41092395223734\\
3.73373373373373	-1.4047443026497\\
3.74374374374374	-1.39856104003269\\
3.75375375375375	-1.39237084307163\\
3.76376376376376	-1.38617040113196\\
3.77377377377377	-1.37995641847852\\
3.78378378378378	-1.37372561466789\\
3.79379379379379	-1.3674747313613\\
3.8038038038038	-1.36120053467425\\
3.81381381381381	-1.35489981900626\\
3.82382382382382	-1.34856940828873\\
3.83383383383383	-1.3422061631222\\
3.84384384384384	-1.33580698039948\\
3.85385385385385	-1.32936879759016\\
3.86386386386386	-1.32288859754564\\
3.87387387387387	-1.31636340828036\\
3.88388388388388	-1.30979030929145\\
3.89389389389389	-1.30316643162674\\
3.9039039039039	-1.29648896265538\\
3.91391391391391	-1.28975514634536\\
3.92392392392392	-1.28296228890207\\
3.93393393393393	-1.2761077591718\\
3.94394394394394	-1.2691889912656\\
3.95395395395395	-1.26220348765578\\
3.96396396396396	-1.25514882098283\\
3.97397397397397	-1.24802263661692\\
3.98398398398398	-1.24082265319823\\
3.99399399399399	-1.23354666678778\\
4.004004004004	-1.22619255220679\\
4.01401401401401	-1.21875826309265\\
4.02402402402402	-1.21124183633197\\
4.03403403403403	-1.20364139110378\\
4.04404404404404	-1.19595513295354\\
4.05405405405405	-1.18818135300499\\
4.06406406406406	-1.18031843143811\\
4.07407407407407	-1.17236483633626\\
4.08408408408408	-1.1643191282702\\
4.09409409409409	-1.1561799577616\\
4.1041041041041	-1.14794606892603\\
4.11411411411411	-1.13961629930685\\
4.12412412412412	-1.13118958187833\\
4.13413413413413	-1.12266494426352\\
4.14414414414414	-1.11404151093959\\
4.15415415415415	-1.10531850389225\\
4.16416416416416	-1.09649524078032\\
4.17417417417417	-1.08757113979056\\
4.18418418418418	-1.07854571724933\\
4.19419419419419	-1.06941858746203\\
4.2042042042042	-1.06018946470475\\
4.21421421421421	-1.05085816400815\\
4.22422422422422	-1.04142459883895\\
4.23423423423423	-1.03188878371909\\
4.24424424424424	-1.02225083342225\\
4.25425425425425	-1.01251096261431\\
4.26426426426426	-1.00266948592381\\
4.27427427427427	-0.992726819241655\\
4.28428428428428	-0.982683478035733\\
4.29429429429429	-0.972540077095034\\
4.3043043043043	-0.96229733136553\\
4.31431431431431	-0.951956055747436\\
4.32432432432432	-0.941517162931745\\
4.33433433433433	-0.930981665366503\\
4.34434434434434	-0.920350672995536\\
4.35435435435435	-0.909625394883566\\
4.36436436436436	-0.898807136177351\\
4.37437437437437	-0.887897298007659\\
4.38438438438438	-0.876897379827498\\
4.39439439439439	-0.865808975860552\\
4.4044044044044	-0.85463377353217\\
4.41441441441441	-0.843373556369914\\
4.42442442442442	-0.832030200388548\\
4.43443443443443	-0.820605674104022\\
4.44444444444444	-0.809102037487369\\
4.45445445445445	-0.797521442414852\\
4.46446446446446	-0.785866129363336\\
4.47447447447447	-0.77413842831741\\
4.48448448448448	-0.762340757966936\\
4.49449449449449	-0.750475622567509\\
4.5045045045045	-0.738545612489424\\
4.51451451451451	-0.726553404479488\\
4.52452452452452	-0.714501756823946\\
4.53453453453453	-0.702393511152497\\
4.54454454454454	-0.690231590951727\\
4.55455455455455	-0.678018998290946\\
4.56456456456456	-0.665758814771413\\
4.57457457457457	-0.653454200855695\\
4.58458458458458	-0.641108389569753\\
4.59459459459459	-0.628724691568717\\
4.6046046046046	-0.616306490148958\\
4.61461461461461	-0.603857238592243\\
4.62462462462462	-0.591380462989789\\
4.63463463463463	-0.578879755850426\\
4.64464464464464	-0.566358778704153\\
4.65465465465465	-0.553821256611609\\
4.66466466466466	-0.541270979946076\\
4.67467467467467	-0.528711799854327\\
4.68468468468468	-0.516147628759237\\
4.69469469469469	-0.503582437457165\\
4.7047047047047	-0.491020252792668\\
4.71471471471471	-0.478465156342855\\
4.72472472472472	-0.465921283358268\\
4.73473473473473	-0.453392819448612\\
4.74474474474474	-0.44088399856256\\
4.75475475475475	-0.428399102745113\\
4.76476476476476	-0.415942457160913\\
4.77477477477477	-0.40351843065958\\
4.78478478478478	-0.391131431429374\\
4.79479479479479	-0.378785906564218\\
4.8048048048048	-0.366486338939128\\
4.81481481481481	-0.35423724406316\\
4.82482482482482	-0.342043168745632\\
4.83483483483483	-0.329908689812268\\
4.84484484484484	-0.317838407366967\\
4.85485485485485	-0.305836948274148\\
4.86486486486486	-0.293908958560969\\
4.87487487487487	-0.282059103986257\\
4.88488488488488	-0.270292064279372\\
4.89489489489489	-0.25861253408095\\
4.9049049049049	-0.247025217902955\\
4.91491491491491	-0.235534827060806\\
4.92492492492492	-0.224146078812898\\
4.93493493493493	-0.212863691960692\\
4.94494494494494	-0.201692383800837\\
4.95495495495495	-0.19063686882072\\
4.96496496496496	-0.179701853014284\\
4.97497497497497	-0.168892033657563\\
4.98498498498498	-0.158212095673772\\
4.99499499499499	-0.147666706513192\\
5.00500500500501	-0.137260514879639\\
5.01501501501502	-0.126998147463549\\
5.02502502502503	-0.116884206522886\\
5.03503503503504	-0.106923263407479\\
5.04504504504505	-0.097119859931821\\
5.05505505505506	-0.0874785019747294\\
5.06506506506507	-0.07800365667218\\
5.07507507507508	-0.0686997507515888\\
5.08508508508509	-0.0595711643838633\\
5.0950950950951	-0.050622231989006\\
5.10510510510511	-0.0418572343613255\\
5.11511511511512	-0.0332803991806635\\
5.12512512512513	-0.0248958956302975\\
5.13513513513514	-0.0167078309934486\\
5.14514514514515	-0.00872024969707665\\
5.15515515515516	-0.00093712631911706\\
5.16516516516517	0.00663763375684993\\
5.17517517517518	0.0140002012450929\\
5.18518518518519	0.0211468233411162\\
5.1951951951952	0.0280738283162865\\
5.20520520520521	0.0347776280862014\\
5.21521521521522	0.0412547221730852\\
5.22522522522523	0.0475017005688291\\
5.23523523523524	0.0535152458844047\\
5.24524524524525	0.0592921380579634\\
5.25525525525526	0.0648292552241461\\
5.26526526526527	0.0701235788974823\\
5.27527527527528	0.0751721957483529\\
5.28528528528529	0.0799723002769107\\
5.2952952952953	0.0845211982096429\\
5.30530530530531	0.0888163082585092\\
5.31531531531532	0.0928551664927888\\
5.32532532532533	0.0966354275589964\\
5.33533533533534	0.10015486742719\\
5.34534534534535	0.103411385455555\\
5.35535535535536	0.106403008735757\\
5.36536536536537	0.109127891427854\\
5.37537537537538	0.111584319054982\\
5.38538538538539	0.113770709553494\\
5.3953953953954	0.11568561745457\\
5.40540540540541	0.117327731616362\\
5.41541541541542	0.118695880827481\\
5.42542542542543	0.119789033808804\\
5.43543543543544	0.120606301811328\\
5.44544544544545	0.121146938106427\\
5.45545545545546	0.12141034221569\\
5.46546546546547	0.121396057834254\\
5.47547547547548	0.121103777786723\\
5.48548548548549	0.120533341671397\\
5.4954954954955	0.119684738209664\\
5.50550550550551	0.118558105985821\\
5.51551551551552	0.117153734872634\\
5.52552552552553	0.115472063523851\\
5.53553553553554	0.11351368437918\\
5.54554554554555	0.111279339121755\\
5.55555555555556	0.108769922483369\\
5.56556556556557	0.105986480414003\\
5.57557557557558	0.102930209538865\\
5.58558558558559	0.0996024582413316\\
5.5955955955956	0.0960047253203122\\
5.60560560560561	0.0921386594667449\\
5.61561561561562	0.0880060582642681\\
5.62562562562563	0.0836088679236139\\
5.63563563563564	0.078949182783466\\
5.64564564564565	0.0740292416302074\\
5.65565565565566	0.0688514295311896\\
5.66566566566567	0.0634182739238634\\
5.67567567567568	0.0577324448953494\\
5.68568568568569	0.0517967515736401\\
5.6956956956957	0.0456141424536826\\
5.70570570570571	0.0391877006798141\\
5.71571571571572	0.0325206450671049\\
5.72572572572573	0.025616325125648\\
5.73573573573574	0.0184782194000982\\
5.74574574574575	0.011109933810823\\
5.75575575575576	0.00351519874301727\\
5.76576576576577	-0.00430213567095924\\
5.77577577577578	-0.0123380995989742\\
5.78578578578579	-0.0205886083339015\\
5.7957957957958	-0.0290494652036077\\
5.80580580580581	-0.037716364615964\\
5.81581581581582	-0.0465848957219833\\
5.82582582582583	-0.0556505458989271\\
5.83583583583584	-0.0649087034911705\\
5.84584584584585	-0.0743546628697897\\
5.85585585585586	-0.083983628105534\\
5.86586586586587	-0.093790715536223\\
5.87587587587588	-0.10377095875814\\
5.88588588588589	-0.113919313618259\\
5.8958958958959	-0.124230660216678\\
5.90590590590591	-0.134699808956636\\
5.91591591591592	-0.145321504220517\\
5.92592592592593	-0.156090428724123\\
5.93593593593594	-0.167001207763795\\
5.94594594594595	-0.1780484143385\\
5.95595595595596	-0.189226573429695\\
5.96596596596597	-0.200530166098587\\
5.97597597597598	-0.211953634596406\\
5.98598598598599	-0.22349138761632\\
5.995995995996	-0.235137804057981\\
6.00600600600601	-0.24688723839048\\
6.01601601601602	-0.258734025196713\\
6.02602602602603	-0.27067248323177\\
6.03603603603604	-0.282696921601692\\
6.04604604604605	-0.294801644515833\\
6.05605605605606	-0.306980954143067\\
6.06606606606607	-0.319229157900798\\
6.07607607607608	-0.331540570787884\\
6.08608608608609	-0.343909523169227\\
6.0960960960961	-0.356330362296928\\
6.10610610610611	-0.368797459172536\\
6.11611611611612	-0.381305212366705\\
6.12612612612613	-0.39384805323775\\
6.13613613613614	-0.40642044983515\\
6.14614614614615	-0.4190169109511\\
6.15615615615616	-0.43163199315907\\
6.16616616616617	-0.444260301825361\\
6.17617617617618	-0.456896497807462\\
6.18618618618619	-0.469535301044763\\
6.1961961961962	-0.482171494981835\\
6.20620620620621	-0.494799929191858\\
6.21621621621622	-0.50741552641499\\
6.22622622622623	-0.520013283626705\\
6.23623623623624	-0.532588277097518\\
6.24624624624625	-0.545135666420861\\
6.25625625625626	-0.557650697588289\\
6.26626626626627	-0.570128706803953\\
6.27627627627628	-0.582565124309892\\
6.28628628628629	-0.594955475497006\\
6.2962962962963	-0.607295387919278\\
6.30630630630631	-0.61958059080342\\
6.31631631631632	-0.631806919874883\\
6.32632632632633	-0.643970319522138\\
6.33633633633634	-0.656066845517524\\
6.34634634634635	-0.668092668420191\\
6.35635635635636	-0.680044073921511\\
6.36636636636637	-0.691917466569013\\
6.37637637637638	-0.703709372593248\\
6.38638638638639	-0.715416439810662\\
6.3963963963964	-0.727035440434347\\
6.40640640640641	-0.738563273618602\\
6.41641641641642	-0.74999696532506\\
6.42642642642643	-0.761333670618442\\
6.43643643643644	-0.772570674536629\\
6.44644644644645	-0.783705393068441\\
6.45645645645646	-0.794735374812472\\
6.46646646646647	-0.805658299502507\\
6.47647647647648	-0.816471981566981\\
6.48648648648649	-0.827174368300721\\
6.4964964964965	-0.837763541051569\\
6.50650650650651	-0.848237714444959\\
6.51651651651652	-0.858595237463152\\
6.52652652652653	-0.868834591666091\\
6.53653653653654	-0.878954393207182\\
6.54654654654655	-0.888953388207179\\
6.55655655655656	-0.898830457425256\\
6.56656656656657	-0.908584610594942\\
6.57657657657658	-0.918214987526118\\
6.58658658658659	-0.927720857325817\\
6.5965965965966	-0.937101615538947\\
6.60660660660661	-0.946356783729399\\
6.61661661661662	-0.955486008134962\\
6.62662662662663	-0.964489056663565\\
6.63663663663664	-0.973365818323377\\
6.64664664664665	-0.982116299832105\\
6.65665665665666	-0.990740624967195\\
6.66666666666667	-0.999239030181274\\
6.67667667667668	-1.00761186496844\\
6.68668668668669	-1.01585958613205\\
6.6966966966967	-1.02398275755119\\
6.70670670670671	-1.03198204643032\\
6.71671671671672	-1.03985821960258\\
6.72672672672673	-1.04761214239184\\
6.73673673673674	-1.05524477371706\\
6.74674674674675	-1.06275716313486\\
6.75675675675676	-1.07015044878405\\
6.76676676676677	-1.07742585257696\\
6.77677677677678	-1.08458467625328\\
6.78678678678679	-1.0916283002117\\
6.7967967967968	-1.09855817726461\\
6.80680680680681	-1.10537582973453\\
6.81681681681682	-1.1120828462581\\
6.82682682682683	-1.1186808767919\\
6.83683683683684	-1.12517162968179\\
6.84684684684685	-1.13155686703412\\
6.85685685685686	-1.13783840087927\\
6.86686686686687	-1.14401808871704\\
6.87687687687688	-1.15009782963086\\
6.88688688688689	-1.15607956091144\\
6.8968968968969	-1.1619652523879\\
6.90690690690691	-1.1677569028946\\
6.91691691691692	-1.17345653600765\\
6.92692692692693	-1.17906619645737\\
6.93693693693694	-1.18458794420627\\
6.94694694694695	-1.19002385200037\\
6.95695695695696	-1.19537599985506\\
6.96696696696697	-1.20064647162692\\
6.97697697697698	-1.20583734980703\\
6.98698698698699	-1.21095071223643\\
6.996996996997	-1.21598862770593\\
7.00700700700701	-1.22095315147824\\
7.01701701701702	-1.22584632107863\\
7.02702702702703	-1.230670153183\\
7.03703703703704	-1.23542663771084\\
7.04704704704705	-1.24011773610052\\
7.05705705705706	-1.24474537588107\\
7.06706706706707	-1.24931144614599\\
7.07707707707708	-1.25381779660647\\
7.08708708708709	-1.25826623025884\\
7.0970970970971	-1.26265850276293\\
7.10710710710711	-1.26699631603039\\
7.11711711711712	-1.27128131786691\\
7.12712712712713	-1.27551509549619\\
7.13713713713714	-1.27969917465715\\
7.14714714714715	-1.28383501429992\\
7.15715715715716	-1.28792400532753\\
7.16716716716717	-1.29196746613678\\
7.17717717717718	-1.29596664022626\\
7.18718718718719	-1.29992269409547\\
7.1971971971972	-1.30383671283792\\
7.20720720720721	-1.30770969919247\\
7.21721721721722	-1.31154256959218\\
7.22722722722723	-1.31533615406207\\
7.23723723723724	-1.31909119086006\\
7.24724724724725	-1.32280832776362\\
7.25725725725726	-1.32648811662194\\
7.26726726726727	-1.33013101431133\\
7.27727727727728	-1.33373737985266\\
7.28728728728729	-1.33730747342049\\
7.2972972972973	-1.3408414532785\\
7.30730730730731	-1.34433937688207\\
7.31731731731732	-1.34780119819085\\
7.32732732732733	-1.35122676679962\\
7.33733733733734	-1.35461582738567\\
7.34734734734735	-1.35796801889834\\
7.35735735735736	-1.36128287461763\\
7.36736736736737	-1.36455982003532\\
7.37737737737738	-1.36779817402987\\
7.38738738738739	-1.370997148912\\
7.3973973973974	-1.37415584962941\\
7.40740740740741	-1.37727327327039\\
7.41741741741742	-1.38034831251291\\
7.42742742742743	-1.383379751188\\
7.43743743743744	-1.3863662698951\\
7.44744744744745	-1.38930644326818\\
7.45745745745746	-1.39219874269384\\
7.46746746746747	-1.3950415361557\\
7.47747747747748	-1.39783309057324\\
7.48748748748749	-1.40057157212016\\
7.4974974974975	-1.40325504936248\\
7.50750750750751	-1.40588149171071\\
7.51751751751752	-1.40844877531222\\
7.52752752752753	-1.41095468174584\\
7.53753753753754	-1.41339690157747\\
7.54754754754755	-1.41577303524252\\
7.55755755755756	-1.41808059696835\\
7.56756756756757	-1.42031701676375\\
7.57757757757758	-1.42247964185921\\
7.58758758758759	-1.42456573969448\\
7.5975975975976	-1.42657250302662\\
7.60760760760761	-1.42849704940498\\
7.61761761761762	-1.43033642600095\\
7.62762762762763	-1.43208761257836\\
7.63763763763764	-1.43374752610904\\
7.64764764764765	-1.43531301906929\\
7.65765765765766	-1.4367808906244\\
7.66766766766767	-1.4381478831923\\
7.67767767767768	-1.4394106903396\\
7.68768768768769	-1.44056595860613\\
7.6976976976977	-1.44161029120478\\
7.70770770770771	-1.44254025328629\\
7.71771771771772	-1.44335237473453\\
7.72772772772773	-1.44404315504075\\
7.73773773773774	-1.44460906542883\\
7.74774774774775	-1.44504655633156\\
7.75775775775776	-1.44535205872087\\
7.76776776776777	-1.44552198932366\\
7.77777777777778	-1.4455527554628\\
7.78778778778779	-1.44544075980164\\
7.7977977977978	-1.44518240195153\\
7.80780780780781	-1.44477408655642\\
7.81781781781782	-1.44421222589327\\
7.82782782782783	-1.44349324442892\\
7.83783783783784	-1.4426135827703\\
7.84784784784785	-1.44156970470907\\
7.85785785785786	-1.44035809837062\\
7.86786786786787	-1.4389752834198\\
7.87787787787788	-1.43741781425357\\
7.88788788788789	-1.43568228455621\\
7.8978978978979	-1.43376533239986\\
7.90790790790791	-1.43166364557537\\
7.91791791791792	-1.42937396370234\\
7.92792792792793	-1.42689308434537\\
7.93793793793794	-1.42421786757395\\
7.94794794794795	-1.42134524053852\\
7.95795795795796	-1.4182721991188\\
7.96796796796797	-1.41499581683414\\
7.97797797797798	-1.4115132462875\\
7.98798798798799	-1.40782172180797\\
7.997997997998	-1.40391856858028\\
8.00800800800801	-1.39980120167891\\
8.01801801801802	-1.39546713175645\\
8.02802802802803	-1.39091397307602\\
8.03803803803804	-1.38613943871112\\
8.04804804804805	-1.3811413535033\\
8.05805805805806	-1.37591765126326\\
8.06806806806807	-1.3704663814162\\
8.07807807807808	-1.36478571219593\\
8.08808808808809	-1.35887393386151\\
8.0980980980981	-1.35272946149829\\
8.10810810810811	-1.34635083786323\\
8.11811811811812	-1.33973673882828\\
8.12812812812813	-1.33288597425118\\
8.13813813813814	-1.32579749059515\\
8.14814814814815	-1.31847037480794\\
8.15815815815816	-1.31090385621659\\
8.16816816816817	-1.30309731050319\\
8.17817817817818	-1.29505025941457\\
8.18818818818819	-1.28676237538078\\
8.1981981981982	-1.2782334821801\\
8.20820820820821	-1.26946355682599\\
8.21821821821822	-1.26045273318252\\
8.22822822822823	-1.25120130136918\\
8.23823823823824	-1.24170970843204\\
8.24824824824825	-1.23197856388174\\
8.25825825825826	-1.22200863769688\\
8.26826826826827	-1.21180086101179\\
8.27827827827828	-1.20135632910552\\
8.28828828828829	-1.1906762992873\\
8.2982982982983	-1.17976219472596\\
8.30830830830831	-1.16861560195402\\
8.31831831831832	-1.15723827201689\\
8.32832832832833	-1.14563212276874\\
8.33833833833834	-1.13379923420746\\
8.34834834834835	-1.12174185257027\\
8.35835835835836	-1.10946238674106\\
8.36836836836837	-1.09696340938005\\
8.37837837837838	-1.08424765762614\\
8.38838838838839	-1.07131802486796\\
8.3983983983984	-1.058177571407\\
8.40840840840841	-1.04482951243144\\
8.41841841841842	-1.03127722231861\\
8.42842842842843	-1.01752423174903\\
8.43843843843844	-1.00357422453553\\
8.44844844844845	-0.989431037861103\\
8.45845845845846	-0.975098659522315\\
8.46846846846847	-0.960581224766853\\
8.47847847847848	-0.945883013690486\\
8.48848848848849	-0.931008450984956\\
8.4984984984985	-0.915962100894197\\
8.50850850850851	-0.900748665998046\\
8.51851851851852	-0.885372981425713\\
8.52852852852853	-0.869840014887617\\
8.53853853853854	-0.85415486200731\\
8.54854854854855	-0.838322743176584\\
8.55855855855856	-0.822348998450908\\
8.56856856856857	-0.806239087536828\\
8.57857857857858	-0.789998580262449\\
8.58858858858859	-0.773633158440483\\
8.5985985985986	-0.757148607970502\\
8.60860860860861	-0.740550816853698\\
8.61861861861862	-0.723845769679141\\
8.62862862862863	-0.707039541914084\\
8.63863863863864	-0.690138297812697\\
8.64864864864865	-0.673148285667854\\
8.65865865865866	-0.656075830581697\\
8.66866866866867	-0.638927331791747\\
8.67867867867868	-0.621709258991694\\
8.68868868868869	-0.604428143054073\\
8.6986986986987	-0.587090573711186\\
8.70870870870871	-0.569703195003131\\
8.71871871871872	-0.552272699846776\\
8.72872872872873	-0.534805822822871\\
8.73873873873874	-0.517309336829792\\
8.74874874874875	-0.499790045339322\\
8.75875875875876	-0.482254781295598\\
8.76876876876877	-0.46471039773005\\
8.77877877877878	-0.447163763220522\\
8.78878878878879	-0.429621756466461\\
8.7987987987988	-0.412091261595209\\
8.80880880880881	-0.39457916206613\\
8.81881881881882	-0.377092333668302\\
8.82882882882883	-0.359637642374158\\
8.83883883883884	-0.342221934985181\\
8.84884884884885	-0.324852036737184\\
8.85885885885886	-0.307534740291767\\
8.86886886886887	-0.29027681057343\\
8.87887887887888	-0.273084964959547\\
8.88888888888889	-0.255965883972388\\
8.8988988988989	-0.238926191257963\\
8.90890890890891	-0.221972456251482\\
8.91891891891892	-0.205111189113475\\
8.92892892892893	-0.188348829362607\\
8.93893893893894	-0.171691748916512\\
8.94894894894895	-0.155146239591913\\
8.95895895895896	-0.138718512177032\\
8.96896896896897	-0.122414690136959\\
8.97897897897898	-0.106240805112469\\
8.98898898898899	-0.0902027906281859\\
8.998998998999	-0.0743064806973816\\
9.00900900900901	-0.0585576009311922\\
9.01901901901902	-0.0429617661762677\\
9.02902902902903	-0.0275244778915076\\
9.03903903903904	-0.0122511145075461\\
9.04904904904905	0.00285306467752061\\
9.05905905905906	0.0177829332867869\\
9.06906906906907	0.0325334950025396\\
9.07907907907908	0.0470998874879031\\
9.08908908908909	0.0614773904607079\\
9.0990990990991	0.07566142462947\\
9.10910910910911	0.0896475581190194\\
9.11911911911912	0.103431510276187\\
9.12912912912913	0.117009149407967\\
9.13913913913914	0.130376502479091\\
9.14914914914915	0.143529754829413\\
9.15915915915916	0.156465252776789\\
9.16916916916917	0.169179504313485\\
9.17917917917918	0.18166918521806\\
9.18918918918919	0.193931140448981\\
9.1991991991992	0.205962382532001\\
9.20920920920921	0.217760098087226\\
9.21921921921922	0.229321644252175\\
9.22922922922923	0.240644556281834\\
9.23923923923924	0.251726544681204\\
9.24924924924925	0.262565494500049\\
9.25925925925926	0.273159474284747\\
9.26926926926927	0.28350672657155\\
9.27927927927928	0.293605675390796\\
9.28928928928929	0.303454924703755\\
9.2992992992993	0.313053258453587\\
9.30930930930931	0.322399640280198\\
9.31931931931932	0.331493212554232\\
9.32932932932933	0.340333300368709\\
9.33933933933934	0.348919403404385\\
9.34934934934935	0.357251202285009\\
9.35935935935936	0.365328555382665\\
9.36936936936937	0.373151495479142\\
9.37937937937938	0.38072023171691\\
9.38938938938939	0.388035144972295\\
9.3993993993994	0.395096789240498\\
9.40940940940941	0.401905890067993\\
9.41941941941942	0.408463337609859\\
9.42942942942943	0.414770193161087\\
9.43943943943944	0.420827676200873\\
9.44944944944945	0.426637172416929\\
9.45945945945946	0.432200222918768\\
9.46946946946947	0.437518528022984\\
9.47947947947948	0.442593939145674\\
9.48948948948949	0.447428458481778\\
9.4994994994995	0.452024234709495\\
9.50950950950951	0.456383562194537\\
9.51951951951952	0.4605088734796\\
9.52952952952953	0.464402739411414\\
9.53953953953954	0.468067861922008\\
9.54954954954955	0.47150707439954\\
9.55955955955956	0.474723334928812\\
9.56956956956957	0.477719722602391\\
9.57957957957958	0.480499432545475\\
9.58958958958959	0.483065775164395\\
9.5995995995996	0.485422167815331\\
9.60960960960961	0.487572131194047\\
9.61961961961962	0.489519288184946\\
9.62962962962963	0.491267355539979\\
9.63963963963964	0.492820138824573\\
9.64964964964965	0.494181531155379\\
9.65965965965966	0.495355505668299\\
9.66966966966967	0.496346112893453\\
9.67967967967968	0.497157471429163\\
9.68968968968969	0.49779376928162\\
9.6996996996997	0.498259253135797\\
9.70970970970971	0.498558227756298\\
9.71971971971972	0.498695047686944\\
9.72972972972973	0.498674114396258\\
9.73973973973974	0.498499869151948\\
9.74974974974975	0.498176790354067\\
9.75975975975976	0.497709387276237\\
9.76976976976977	0.497102193511224\\
9.77977977977978	0.496359764268385\\
9.78978978978979	0.495486670890475\\
9.7997997997998	0.49448749393863\\
9.80980980980981	0.49336682108953\\
9.81981981981982	0.492129238927981\\
9.82982982982983	0.490779331292313\\
9.83983983983984	0.489321671948674\\
9.84984984984985	0.487760820224191\\
9.85985985985986	0.486101318340136\\
9.86986986986987	0.484347682695225\\
9.87987987987988	0.482504402115706\\
9.88988988988989	0.480575934627778\\
9.8998998998999	0.478566699493361\\
9.90990990990991	0.47648107277645\\
9.91991991991992	0.474323387262061\\
9.92992992992993	0.47209792217168\\
9.93993993993994	0.469808905605181\\
9.94994994994995	0.467460510168682\\
9.95995995995996	0.465056833086888\\
9.96996996996997	0.46260192678467\\
9.97997997997998	0.460099746936997\\
9.98998998998999	0.457554198352168\\
10	0.454969101914374\\
};
\addplot [color=mycolor2,solid,forget plot]
  table[row sep=crcr]{%
0	0.735205803646842\\
0.01001001001001	0.741095337313144\\
0.02002002002002	0.746985474444168\\
0.03003003003003	0.752871688070385\\
0.04004004004004	0.758749429178734\\
0.0500500500500501	0.764614131911058\\
0.0600600600600601	0.770461219280191\\
0.0700700700700701	0.776286108277684\\
0.0800800800800801	0.782084215844557\\
0.0900900900900901	0.787850963894684\\
0.1001001001001	0.793581785653054\\
0.11011011011011	0.799272130564681\\
0.12012012012012	0.804917470693359\\
0.13013013013013	0.810513306210502\\
0.14014014014014	0.816055170769745\\
0.15015015015015	0.82153863791296\\
0.16016016016016	0.826959326214641\\
0.17017017017017	0.832312905459143\\
0.18018018018018	0.837595102260767\\
0.19019019019019	0.842801705459185\\
0.2002002002002	0.847928572362792\\
0.21021021021021	0.852971633663606\\
0.22022022022022	0.857926899651913\\
0.23023023023023	0.862790465423\\
0.24024024024024	0.867558516254612\\
0.25025025025025	0.872227333327599\\
0.26026026026026	0.876793298609631\\
0.27027027027027	0.881252900243479\\
0.28028028028028	0.885602737689249\\
0.29029029029029	0.889839526764083\\
0.3003003003003	0.893960104242285\\
0.31031031031031	0.897961432968014\\
0.32032032032032	0.901840606431915\\
0.33033033033033	0.905594852860874\\
0.34034034034034	0.90922154035523\\
0.35035035035035	0.912718180274048\\
0.36036036036036	0.916082431724568\\
0.37037037037037	0.919312105489342\\
0.38038038038038	0.922405167569716\\
0.39039039039039	0.925359742850808\\
0.4004004004004	0.928174118291332\\
0.41041041041041	0.930846746171654\\
0.42042042042042	0.93337624728109\\
0.43043043043043	0.9357614133496\\
0.44044044044044	0.938001209954507\\
0.45045045045045	0.940094778603481\\
0.46046046046046	0.942041439177839\\
0.47047047047047	0.943840691642139\\
0.48048048048048	0.945492217855181\\
0.49049049049049	0.946995882967687\\
0.500500500500501	0.948351737119779\\
0.510510510510511	0.949560015883102\\
0.520520520520521	0.950621141390429\\
0.530530530530531	0.951535723109864\\
0.540540540540541	0.952304557571218\\
0.550550550550551	0.952928629313019\\
0.560560560560561	0.953409109986958\\
0.570570570570571	0.953747358395584\\
0.580580580580581	0.953944919968249\\
0.590590590590591	0.954003525419505\\
0.600600600600601	0.953925090277493\\
0.610610610610611	0.953711713046781\\
0.620620620620621	0.953365674117451\\
0.630630630630631	0.952889433349792\\
0.640640640640641	0.952285628718368\\
0.650650650650651	0.951557073618003\\
0.660660660660661	0.950706754346117\\
0.670670670670671	0.949737827642956\\
0.680680680680681	0.948653617490842\\
0.690690690690691	0.947457611997161\\
0.700700700700701	0.946153460335627\\
0.710710710710711	0.944744968405211\\
0.720720720720721	0.943236095881133\\
0.730730730730731	0.941630951276027\\
0.740740740740741	0.939933788715546\\
0.750750750750751	0.938149002674431\\
0.760760760760761	0.93628112364023\\
0.770770770770771	0.934334813894017\\
0.780780780780781	0.932314861421217\\
0.790790790790791	0.930226175573677\\
0.800800800800801	0.928073781818267\\
0.810810810810811	0.925862815657314\\
0.820820820820821	0.923598517598789\\
0.830830830830831	0.92128622704884\\
0.840840840840841	0.918931376680477\\
0.850850850850851	0.916539486172859\\
0.860860860860861	0.914116156031761\\
0.870870870870871	0.911667061535134\\
0.880880880880881	0.909197946317707\\
0.890890890890891	0.906714615838911\\
0.900900900900901	0.904222930948419\\
0.910910910910911	0.901728800986438\\
0.920920920920921	0.899238177174737\\
0.930930930930931	0.896757046216475\\
0.940940940940941	0.894291422985698\\
0.950950950950951	0.891847343876458\\
0.960960960960961	0.889430859961404\\
0.970970970970971	0.88704802994928\\
0.980980980980981	0.884704913094419\\
0.990990990990991	0.882407562641823\\
1.001001001001	0.880162018454481\\
1.01101101101101	0.877974300597294\\
1.02102102102102	0.875850401618286\\
1.03103103103103	0.873796280310537\\
1.04104104104104	0.871817854740526\\
1.05105105105105	0.869920994929334\\
1.06106106106106	0.868111517029474\\
1.07107107107107	0.866395175449875\\
1.08108108108108	0.864777657261389\\
1.09109109109109	0.863264575040376\\
1.1011011011011	0.861861460442797\\
1.11111111111111	0.860573758207116\\
1.12112112112112	0.859406819379336\\
1.13113113113113	0.858365895565104\\
1.14114114114114	0.857456132394646\\
1.15115115115115	0.856682564459441\\
1.16116116116116	0.856050108529257\\
1.17117117117117	0.855563558568106\\
1.18118118118118	0.855227580028452\\
1.19119119119119	0.855046704424721\\
1.2012012012012	0.855025324370623\\
1.21121121121121	0.855167688385005\\
1.22122122122122	0.855477896226967\\
1.23123123123123	0.855959893893564\\
1.24124124124124	0.856617469626498\\
1.25125125125125	0.857454249027889\\
1.26126126126126	0.858473691607348\\
1.27127127127127	0.859679086258864\\
1.28128128128128	0.861073548163058\\
1.29129129129129	0.862660014565736\\
1.3013013013013	0.864441242259359\\
1.31131131131131	0.866419803769824\\
1.32132132132132	0.868598085270117\\
1.33133133133133	0.870978283238762\\
1.34134134134134	0.873562402295963\\
1.35135135135135	0.876352253197043\\
1.36136136136136	0.879349450534682\\
1.37137137137137	0.882555411056037\\
1.38138138138138	0.885971352451637\\
1.39139139139139	0.889598291452411\\
1.4014014014014	0.89343704327139\\
1.41141141141141	0.897488220580153\\
1.42142142142142	0.901752232689483\\
1.43143143143143	0.906229285536228\\
1.44144144144144	0.91091938119755\\
1.45145145145145	0.915822318052273\\
1.46146146146146	0.920937690833315\\
1.47147147147147	0.926264891724584\\
1.48148148148148	0.931803110277872\\
1.49149149149149	0.937551335070696\\
1.5015015015015	0.943508354155686\\
1.51151151151151	0.949672756995834\\
1.52152152152152	0.956042935852246\\
1.53153153153153	0.962617086965744\\
1.54154154154154	0.96939321382767\\
1.55155155155155	0.976369127898086\\
1.56156156156156	0.98354245188675\\
1.57157157157157	0.990910621930691\\
1.58158158158158	0.998470890481304\\
1.59159159159159	1.00622032897933\\
1.6016016016016	1.01415583107653\\
1.61161161161161	1.02227411598648\\
1.62162162162162	1.03057173171481\\
1.63163163163163	1.03904505864985\\
1.64164164164164	1.04769031320355\\
1.65165165165165	1.05650355190388\\
1.66166166166166	1.06548067537943\\
1.67167167167167	1.07461743199006\\
1.68168168168168	1.08390942272458\\
1.69169169169169	1.09335210532864\\
1.7017017017017	1.10294079865984\\
1.71171171171171	1.11267068734171\\
1.72172172172172	1.12253682679095\\
1.73173173173173	1.13253414756843\\
1.74174174174174	1.14265746047468\\
1.75175175175175	1.15290146140969\\
1.76176176176176	1.16326073654198\\
1.77177177177177	1.17372976742024\\
1.78178178178178	1.18430293595573\\
1.79179179179179	1.19497452966432\\
1.8018018018018	1.20573874712113\\
1.81181181181181	1.21658970319075\\
1.82182182182182	1.22752143408418\\
1.83183183183183	1.23852790315253\\
1.84184184184184	1.24960300588285\\
1.85185185185185	1.26074057541383\\
1.86186186186186	1.27193438820252\\
1.87187187187187	1.28317816890596\\
1.88188188188188	1.29446559611204\\
1.89189189189189	1.30579030756139\\
1.9019019019019	1.3171459057528\\
1.91191191191191	1.32852596277612\\
1.92192192192192	1.33992402613851\\
1.93193193193193	1.3513336234734\\
1.94194194194194	1.36274826839869\\
1.95195195195195	1.37416146512381\\
1.96196196196196	1.38556671366778\\
1.97197197197197	1.39695751508233\\
1.98198198198198	1.40832737636098\\
1.99199199199199	1.41966981529144\\
2.002002002002	1.43097836526796\\
2.01201201201201	1.44224658015966\\
2.02202202202202	1.4534680386966\\
2.03203203203203	1.46463634967907\\
2.04204204204204	1.47574515574453\\
2.05205205205205	1.48678813807341\\
2.06206206206206	1.49775902104411\\
2.07207207207207	1.50865157565094\\
2.08208208208208	1.5194596242552\\
2.09209209209209	1.53017704433508\\
2.1021021021021	1.5407977724238\\
2.11211211211211	1.55131580779318\\
2.12212212212212	1.56172521585826\\
2.13213213213213	1.57202013229772\\
2.14214214214214	1.58219476607604\\
2.15215215215215	1.59224340279986\\
2.16216216216216	1.6021604076782\\
2.17217217217217	1.61194022901978\\
2.18218218218218	1.62157740064239\\
2.19219219219219	1.6310665451825\\
2.2022022022022	1.64040237638976\\
2.21221221221221	1.6495797016052\\
2.22222222222222	1.65859342453707\\
2.23223223223223	1.66743854739079\\
2.24224224224224	1.67611017277511\\
2.25225225225225	1.68460350626746\\
2.26226226226226	1.69291385771077\\
2.27227227227227	1.70103664358955\\
2.28228228228228	1.70896738811579\\
2.29229229229229	1.71670172537275\\
2.3023023023023	1.72423540014005\\
2.31231231231231	1.73156426983053\\
2.32232232232232	1.73868430495723\\
2.33233233233233	1.74559159081033\\
2.34234234234234	1.75228232811665\\
2.35235235235235	1.75875283389335\\
2.36236236236236	1.76499954219816\\
2.37237237237237	1.77101900480116\\
2.38238238238238	1.77680789158639\\
2.39239239239239	1.78236299120116\\
2.4024024024024	1.78768121095137\\
2.41241241241241	1.79275957740276\\
2.42242242242242	1.7975952363107\\
2.43243243243243	1.80218545270818\\
2.44244244244244	1.80652761067797\\
2.45245245245245	1.81061921348998\\
2.46246246246246	1.81445788293907\\
2.47247247247247	1.81804135938706\\
2.48248248248248	1.82136750130249\\
2.49249249249249	1.82443428448626\\
2.5025025025025	1.82723980200309\\
2.51251251251251	1.82978226291214\\
2.52252252252252	1.83205999223618\\
2.53253253253253	1.83407142976898\\
2.54254254254254	1.83581512950918\\
2.55255255255255	1.83728975837816\\
2.56256256256256	1.83849409595757\\
2.57257257257257	1.83942703283205\\
2.58258258258258	1.84008756998122\\
2.59259259259259	1.84047481747171\\
2.6026026026026	1.84058799340425\\
2.61261261261261	1.84042642281826\\
2.62262262262262	1.83998953649187\\
2.63263263263263	1.83927686946883\\
2.64264264264264	1.83828806005084\\
2.65265265265265	1.83702284831494\\
2.66266266266266	1.83548107502508\\
2.67267267267267	1.83366267986606\\
2.68268268268268	1.83156770055189\\
2.69269269269269	1.82919627094901\\
2.7027027027027	1.82654862005469\\
2.71271271271271	1.82362507027644\\
2.72272272272272	1.8204260361872\\
2.73273273273273	1.8169520228111\\
2.74274274274274	1.81320362439514\\
2.75275275275275	1.809181522825\\
2.76276276276276	1.80488648603722\\
2.77277277277277	1.80031936680798\\
2.78278278278278	1.7954811008402\\
2.79279279279279	1.79037270556218\\
2.8028028028028	1.784995278419\\
2.81281281281281	1.7793499955483\\
2.82282282282282	1.77343811029582\\
2.83283283283283	1.76726095115099\\
2.84284284284284	1.76081992130424\\
2.85285285285285	1.75411649600163\\
2.86286286286286	1.74715222179014\\
2.87287287287287	1.73992871456496\\
2.88288288288288	1.73244765865658\\
2.89289289289289	1.72471080472857\\
2.9029029029029	1.71671996843526\\
2.91291291291291	1.70847702933514\\
2.92292292292292	1.69998392911228\\
2.93293293293293	1.6912426699456\\
2.94294294294294	1.68225531340963\\
2.95295295295295	1.67302397878825\\
2.96296296296296	1.66355084184516\\
2.97297297297297	1.65383813325018\\
2.98298298298298	1.64388813686279\\
2.99299299299299	1.6337031891889\\
3.003003003003	1.62328567690731\\
3.01301301301301	1.61263803627824\\
3.02302302302302	1.60176275115275\\
3.03303303303303	1.59066235207538\\
3.04304304304304	1.57933941472529\\
3.05305305305305	1.56779655820818\\
3.06306306306306	1.55603644432408\\
3.07307307307307	1.54406177573651\\
3.08308308308308	1.53187529465282\\
3.09309309309309	1.5194797816624\\
3.1031031031031	1.50687805434776\\
3.11311311311311	1.49407296576209\\
3.12312312312312	1.48106740323203\\
3.13313313313313	1.46786428711937\\
3.14314314314314	1.45446656934747\\
3.15315315315315	1.44087723219598\\
3.16316316316316	1.42709928681074\\
3.17317317317317	1.41313577216683\\
3.18318318318318	1.39898975354068\\
3.19319319319319	1.38466432131372\\
3.2032032032032	1.37016258965439\\
3.21321321321321	1.35548769527645\\
3.22322322322322	1.34064279613014\\
3.23323323323323	1.325631070056\\
3.24324324324324	1.31045571373816\\
3.25325325325325	1.29511994120336\\
3.26326326326326	1.27962698278232\\
3.27327327327327	1.26398008357472\\
3.28328328328328	1.24818250260736\\
3.29329329329329	1.2322375113282\\
3.3033033033033	1.21614839242602\\
3.31331331331331	1.19991843887193\\
3.32332332332332	1.18355095223674\\
3.33333333333333	1.16704924217131\\
3.34334334334334	1.15041662447939\\
3.35335335335335	1.1336564206923\\
3.36336336336336	1.11677195650015\\
3.37337337337337	1.09976656069101\\
3.38338338338338	1.08264356422384\\
3.39339339339339	1.06540629879828\\
3.4034034034034	1.04805809612349\\
3.41341341341341	1.0306022869836\\
3.42342342342342	1.01304219952453\\
3.43343343343343	0.995381159111659\\
3.44344344344344	0.977622486676538\\
3.45345345345345	0.959769498174367\\
3.46346346346346	0.941825503533427\\
3.47347347347347	0.923793805687436\\
3.48348348348348	0.90567769967781\\
3.49349349349349	0.887480472254389\\
3.5035035035035	0.869205400336865\\
3.51351351351351	0.850855750723322\\
3.52352352352352	0.832434779428193\\
3.53353353353353	0.813945730615012\\
3.54354354354354	0.79539183604824\\
3.55355355355355	0.776776314888073\\
3.56356356356356	0.758102372420286\\
3.57357357357357	0.739373199950838\\
3.58358358358358	0.720591974114827\\
3.59359359359359	0.701761856741096\\
3.6036036036036	0.682885993970734\\
3.61361361361361	0.663967516072334\\
3.62362362362362	0.64500953732708\\
3.63363363363363	0.626015155289139\\
3.64364364364364	0.606987451110612\\
3.65365365365365	0.587929488994589\\
3.66366366366366	0.568844316124698\\
3.67367367367367	0.549734962428335\\
3.68368368368368	0.530604441077636\\
3.69369369369369	0.511455747800913\\
3.7037037037037	0.492291861556674\\
3.71371371371371	0.473115744257016\\
3.72372372372372	0.453930340972013\\
3.73373373373373	0.434738580496306\\
3.74374374374374	0.415543375149546\\
3.75375375375375	0.396347621581783\\
3.76376376376376	0.377154200750132\\
3.77377377377377	0.357965978295301\\
3.78378378378378	0.338785805590078\\
3.79379379379379	0.319616519756158\\
3.8038038038038	0.300460944272782\\
3.81381381381381	0.281321889642044\\
3.82382382382382	0.262202154476183\\
3.83383383383383	0.243104525334021\\
3.84384384384384	0.224031778345662\\
3.85385385385385	0.204986679694718\\
3.86386386386386	0.185971986074101\\
3.87387387387387	0.16699044633354\\
3.88388388388388	0.148044801616128\\
3.89389389389389	0.129137786861165\\
3.9039039039039	0.110272131466109\\
3.91391391391391	0.0914505607444892\\
3.92392392392392	0.0726757962913134\\
3.93393393393393	0.0539505576681409\\
3.94394394394394	0.0352775632188777\\
3.95395395395395	0.0166595312701907\\
3.96396396396396	-0.0019008188155807\\
3.97397397397397	-0.0204007654875626\\
3.98398398398398	-0.0388375834572758\\
3.99399399399399	-0.0572085430475107\\
4.004004004004	-0.0755109089081711\\
4.01401401401401	-0.0937419383692225\\
4.02402402402402	-0.111898880801921\\
4.03403403403403	-0.129978975956845\\
4.04404404404404	-0.147979453289782\\
4.05405405405405	-0.165897530205883\\
4.06406406406406	-0.183730411440045\\
4.07407407407407	-0.201475287285368\\
4.08408408408408	-0.219129333260503\\
4.09409409409409	-0.236689708128716\\
4.1041041041041	-0.254153553412662\\
4.11411411411411	-0.271517992035552\\
4.12412412412412	-0.288780127550965\\
4.13413413413413	-0.305937042843334\\
4.14414414414414	-0.322985799438169\\
4.15415415415415	-0.339923436527086\\
4.16416416416416	-0.356746969739737\\
4.17417417417417	-0.373453391096699\\
4.18418418418418	-0.390039667738211\\
4.19419419419419	-0.406502741042189\\
4.2042042042042	-0.422839526373299\\
4.21421421421421	-0.439046912591731\\
4.22422422422422	-0.455121761046239\\
4.23423423423423	-0.471060905567648\\
4.24424424424424	-0.486861152007883\\
4.25425425425425	-0.50251927784507\\
4.26426426426426	-0.518032031917396\\
4.27427427427427	-0.533396134696481\\
4.28428428428428	-0.548608277851743\\
4.29429429429429	-0.563665124267603\\
4.3043043043043	-0.578563308457219\\
4.31431431431431	-0.593299436718062\\
4.32432432432432	-0.607870087179622\\
4.33433433433433	-0.622271810605622\\
4.34434434434434	-0.636501130542998\\
4.35435435435435	-0.650554544485286\\
4.36436436436436	-0.664428523849093\\
4.37437437437437	-0.678119515067997\\
4.38438438438438	-0.691623941045897\\
4.39439439439439	-0.704938201591807\\
4.4044044044044	-0.718058674339896\\
4.41441441441441	-0.730981716768597\\
4.42442442442442	-0.743703666925728\\
4.43443443443443	-0.756220845070932\\
4.44444444444444	-0.768529555146506\\
4.45445445445445	-0.780626086796564\\
4.46446446446446	-0.792506716529283\\
4.47447447447447	-0.804167710078898\\
4.48448448448448	-0.815605324343869\\
4.49449449449449	-0.826815809109729\\
4.5045045045045	-0.837795409489461\\
4.51451451451451	-0.848540368591147\\
4.52452452452452	-0.85904692906771\\
4.53453453453453	-0.869311336263564\\
4.54454454454454	-0.879329840723046\\
4.55455455455455	-0.889098700435239\\
4.56456456456456	-0.898614184201896\\
4.57457457457457	-0.907872574488919\\
4.58458458458458	-0.916870169382847\\
4.59459459459459	-0.9256032870901\\
4.6046046046046	-0.934068268202603\\
4.61461461461461	-0.942261478682367\\
4.62462462462462	-0.95017931413227\\
4.63463463463463	-0.957818201981731\\
4.64464464464464	-0.965174606007473\\
4.65465465465465	-0.972245028724878\\
4.66466466466466	-0.979026015975505\\
4.67467467467467	-0.985514159751876\\
4.68468468468468	-0.991706102325651\\
4.69469469469469	-0.997598539882239\\
4.7047047047047	-1.0031882259608\\
4.71471471471471	-1.00847197556578\\
4.72472472472472	-1.01344666912056\\
4.73473473473473	-1.01810925594878\\
4.74474474474474	-1.02245675826632\\
4.75475475475475	-1.02648627552857\\
4.76476476476476	-1.03019498756436\\
4.77477477477477	-1.03358015928084\\
4.78478478478478	-1.03663914399449\\
4.79479479479479	-1.03936938788836\\
4.8048048048048	-1.04176843367128\\
4.81481481481481	-1.04383392437941\\
4.82482482482482	-1.04556360747494\\
4.83483483483483	-1.04695533908594\\
4.84484484484484	-1.04800708673028\\
4.85485485485485	-1.0487169347293\\
4.86486486486486	-1.04908308666365\\
4.87487487487487	-1.04910386996918\\
4.88488488488488	-1.04877773879791\\
4.89489489489489	-1.04810327871059\\
4.9049049049049	-1.04707920949952\\
4.91491491491491	-1.04570438877514\\
4.92492492492492	-1.04397781584034\\
4.93493493493493	-1.0418986347971\\
4.94494494494494	-1.0394661375949\\
4.95495495495495	-1.03667976787374\\
4.96496496496496	-1.03353912325032\\
4.97497497497497	-1.03004395911973\\
4.98498498498498	-1.02619419144006\\
4.99499499499499	-1.02198989896989\\
5.00500500500501	-1.01743132649712\\
5.01501501501502	-1.0125188874083\\
5.02502502502503	-1.00725316627618\\
5.03503503503504	-1.00163492048736\\
5.04504504504505	-0.995665083573163\\
5.05505505505506	-0.989344766474679\\
5.06506506506507	-0.982675259681059\\
5.07507507507508	-0.975658035399623\\
5.08508508508509	-0.968294748544408\\
5.0950950950951	-0.960587239289538\\
5.10510510510511	-0.952537533293945\\
5.11511511511512	-0.944147843911326\\
5.12512512512513	-0.935420572718331\\
5.13513513513514	-0.926358310425351\\
5.14514514514515	-0.91696383825125\\
5.15515515515516	-0.90724012751167\\
5.16516516516517	-0.89719034111126\\
5.17517517517518	-0.886817832894864\\
5.18518518518519	-0.876126148402696\\
5.1951951951952	-0.865119024529859\\
5.20520520520521	-0.853800389367865\\
5.21521521521522	-0.842174361672454\\
5.22522522522523	-0.830245250275745\\
5.23523523523524	-0.818017553587847\\
5.24524524524525	-0.805495958230025\\
5.25525525525526	-0.792685338528333\\
5.26526526526527	-0.779590754582838\\
5.27527527527528	-0.76621745115468\\
5.28528528528529	-0.752570856024582\\
5.2952952952953	-0.73865657810315\\
5.30530530530531	-0.724480405657845\\
5.31531531531532	-0.710048303702134\\
5.32532532532533	-0.695366412084898\\
5.33533533533534	-0.680441042869661\\
5.34534534534535	-0.665278677848022\\
5.35535535535536	-0.649885965117374\\
5.36536536536537	-0.634269716909515\\
5.37537537537538	-0.618436905845432\\
5.38538538538539	-0.602394661928075\\
5.3953953953954	-0.586150268486751\\
5.40540540540541	-0.569711159606019\\
5.41541541541542	-0.553084915396763\\
5.42542542542543	-0.536279258475928\\
5.43543543543544	-0.519302049619376\\
5.44544544544545	-0.502161284094818\\
5.45545545545546	-0.484865086519892\\
5.46546546546547	-0.467421707219928\\
5.47547547547548	-0.44983951665996\\
5.48548548548549	-0.432127001483108\\
5.4954954954955	-0.414292759324397\\
5.50550550550551	-0.396345493758928\\
5.51551551551552	-0.378294009013809\\
5.52552552552553	-0.360147205490039\\
5.53553553553554	-0.341914073250007\\
5.54554554554555	-0.323603687826629\\
5.55555555555556	-0.305225203695085\\
5.56556556556557	-0.286787849173887\\
5.57557557557558	-0.268300920739387\\
5.58558558558559	-0.249773776942893\\
5.5955955955956	-0.231215832747569\\
5.60560560560561	-0.212636553560393\\
5.61561561561562	-0.194045449340236\\
5.62562562562563	-0.17545206833056\\
5.63563563563564	-0.156865990870509\\
5.64564564564565	-0.138296823841872\\
5.65565565565566	-0.119754193693807\\
5.66566566566567	-0.101247740812663\\
5.67567567567568	-0.0827871127969811\\
5.68568568568569	-0.0643819586525399\\
5.6956956956957	-0.0460419219083992\\
5.70570570570571	-0.0277766349974405\\
5.71571571571572	-0.00959571213847105\\
5.72572572572573	0.00849125633115774\\
5.73573573573574	0.0264747104380645\\
5.74574574574575	0.0443451271111651\\
5.75575575575576	0.0620930264298982\\
5.76576576576577	0.0797089773820829\\
5.77577577577578	0.0971836048369716\\
5.78578578578579	0.114507595289923\\
5.7957957957958	0.131671703096069\\
5.80580580580581	0.148666756646174\\
5.81581581581582	0.165483664322415\\
5.82582582582583	0.182113420551117\\
5.83583583583584	0.198547111932743\\
5.84584584584585	0.214775922753256\\
5.85585585585586	0.2307911408397\\
5.86586586586587	0.246584163580762\\
5.87587587587588	0.262146503311498\\
5.88588588588589	0.277469792606179\\
5.8958958958959	0.292545790262279\\
5.90590590590591	0.307366386163878\\
5.91591591591592	0.321923606675525\\
5.92592592592593	0.336209619764468\\
5.93593593593594	0.350216740036525\\
5.94594594594595	0.363937433457855\\
5.95595595595596	0.377364322162864\\
5.96596596596597	0.390490189230866\\
5.97597597597598	0.403307983009882\\
5.98598598598599	0.415810821333397\\
5.995995995996	0.427991996008911\\
6.00600600600601	0.439844976623249\\
6.01601601601602	0.451363414585574\\
6.02602602602603	0.462541147040888\\
6.03603603603604	0.473372200111692\\
6.04604604604605	0.483850792324718\\
6.05605605605606	0.493971338411615\\
6.06606606606607	0.503728451762725\\
6.07607607607608	0.513116947981377\\
6.08608608608609	0.522131846916541\\
6.0960960960961	0.530768376058151\\
6.10610610610611	0.539021972381993\\
6.11611611611612	0.546888284864224\\
6.12612612612613	0.554363176368742\\
6.13613613613614	0.561442725705935\\
6.14614614614615	0.568123229507974\\
6.15615615615616	0.57440120301548\\
6.16616616616617	0.580273382383674\\
6.17617617617618	0.585736725169861\\
6.18618618618619	0.590788411526387\\
6.1961961961962	0.595425844901226\\
6.20620620620621	0.599646653061797\\
6.21621621621622	0.603448687739403\\
6.22622622622623	0.606830025556431\\
6.23623623623624	0.609788967842363\\
6.24624624624625	0.61232404038025\\
6.25625625625626	0.614433993264691\\
6.26626626626627	0.616117800314572\\
6.27627627627628	0.617374658354193\\
6.28628628628629	0.618203986861985\\
6.2962962962963	0.618605426005882\\
6.30630630630631	0.618578836351072\\
6.31631631631632	0.618124296966911\\
6.32632632632633	0.617242104120542\\
6.33633633633634	0.615932769503713\\
6.34634634634635	0.614197018141454\\
6.35635635635636	0.612035786917173\\
6.36636636636637	0.609450221970365\\
6.37637637637638	0.606441676180119\\
6.38638638638639	0.603011707203929\\
6.3963963963964	0.599162074556178\\
6.40640640640641	0.594894736606898\\
6.41641641641642	0.590211848058984\\
6.42642642642643	0.585115756644108\\
6.43643643643644	0.579609000012474\\
6.44644644644645	0.573694302415674\\
6.45645645645646	0.567374571066481\\
6.46646646646647	0.560652893105051\\
6.47647647647648	0.553532531177032\\
6.48648648648649	0.546016920201177\\
6.4964964964965	0.538109663258351\\
6.50650650650651	0.529814527746514\\
6.51651651651652	0.521135441055268\\
6.52652652652653	0.512076486787298\\
6.53653653653654	0.502641899847927\\
6.54654654654655	0.49283606314037\\
6.55655655655656	0.482663501813843\\
6.56656656656657	0.472128880004521\\
6.57657657657658	0.461236995724366\\
6.58658658658659	0.449992776192967\\
6.5965965965966	0.438401273570154\\
6.60660660660661	0.426467660024586\\
6.61661661661662	0.414197222868232\\
6.62662662662663	0.401595360202988\\
6.63663663663664	0.388667575700475\\
6.64664664664665	0.375419474211099\\
6.65665665665666	0.36185675650442\\
6.66666666666667	0.347985214963681\\
6.67667667667668	0.333810728047467\\
6.68668668668669	0.319339256225143\\
6.6966966966967	0.304576836560498\\
6.70670670670671	0.28952957808629\\
6.71671671671672	0.274203657226436\\
6.72672672672673	0.258605312510874\\
6.73673673673674	0.242740840289556\\
6.74674674674675	0.22661658991766\\
6.75675675675676	0.210238958766147\\
6.76676676676677	0.193614387912224\\
6.77677677677678	0.176749357681313\\
6.78678678678679	0.159650382446833\\
6.7967967967968	0.142324006855996\\
6.80680680680681	0.12477680115303\\
6.81681681681682	0.107015356688091\\
6.82682682682683	0.0890462818819719\\
6.83683683683684	0.0708761978161219\\
6.84684684684685	0.0525117341993558\\
6.85685685685686	0.0339595253243836\\
6.86686686686687	0.0152262061622961\\
6.87687687687688	-0.00368159157396247\\
6.88688688688689	-0.0227572433720926\\
6.8968968968969	-0.0419941352623176\\
6.90690690690691	-0.0613856675691491\\
6.91691691691692	-0.0809252583772872\\
6.92692692692693	-0.100606347063096\\
6.93693693693694	-0.120422397168528\\
6.94694694694695	-0.14036690002953\\
6.95695695695696	-0.160433377453824\\
6.96696696696697	-0.180615384917926\\
6.97697697697698	-0.200906514120856\\
6.98698698698699	-0.221300395956662\\
6.996996996997	-0.241790702966782\\
7.00700700700701	-0.262371151882793\\
7.01701701701702	-0.283035505838091\\
7.02702702702703	-0.303777576855356\\
7.03703703703704	-0.324591227603133\\
7.04704704704705	-0.345470373829091\\
7.05705705705706	-0.366408985960627\\
7.06706706706707	-0.387401090674782\\
7.07707707707708	-0.408440773270195\\
7.08708708708709	-0.429522178269004\\
7.0970970970971	-0.450639511625634\\
7.10710710710711	-0.471787041306623\\
7.11711711711712	-0.492959099257207\\
7.12712712712713	-0.514150081712342\\
7.13713713713714	-0.535354450757962\\
7.14714714714715	-0.55656673469876\\
7.15715715715716	-0.577781529216981\\
7.16716716716717	-0.598993497732976\\
7.17717717717718	-0.620197371964331\\
7.18718718718719	-0.641387952704947\\
7.1971971971972	-0.6625601096154\\
7.20720720720721	-0.683708781925533\\
7.21721721721722	-0.704828978223339\\
7.22722722722723	-0.725915777115549\\
7.23723723723724	-0.74696432636049\\
7.24724724724725	-0.767969843718735\\
7.25725725725726	-0.788927615823039\\
7.26726726726727	-0.809832998685785\\
7.27727727727728	-0.830681417060221\\
7.28728728728729	-0.851468364197381\\
7.2972972972973	-0.872189401043191\\
7.30730730730731	-0.892840156237531\\
7.31731731731732	-0.91341632528889\\
7.32732732732733	-0.933913669885116\\
7.33733733733734	-0.954328017401725\\
7.34734734734735	-0.974655260100176\\
7.35735735735736	-0.99489135463317\\
7.36736736736737	-1.0150323207562\\
7.37737737737738	-1.03507424101137\\
7.38738738738739	-1.05501325978771\\
7.3973973973974	-1.07484558240286\\
7.40740740740741	-1.09456747401132\\
7.41741741741742	-1.11417525946612\\
7.42742742742743	-1.13366532131269\\
7.43743743743744	-1.15303410005948\\
7.44744744744745	-1.17227809243245\\
7.45745745745746	-1.19139385093553\\
7.46746746746747	-1.21037798269746\\
7.47747747747748	-1.22922714880998\\
7.48748748748749	-1.24793806328893\\
7.4974974974975	-1.26650749257904\\
7.50750750750751	-1.28493225400968\\
7.51751751751752	-1.30320921598231\\
7.52752752752753	-1.32133529642798\\
7.53753753753754	-1.3393074624285\\
7.54754754754755	-1.35712272919535\\
7.55755755755756	-1.37477815989598\\
7.56756756756757	-1.39227086478693\\
7.57757757757758	-1.40959800042965\\
7.58758758758759	-1.42675676929362\\
7.5975975975976	-1.44374441972606\\
7.60760760760761	-1.46055824475552\\
7.61761761761762	-1.47719558210704\\
7.62762762762763	-1.49365381385873\\
7.63763763763764	-1.50993036651962\\
7.64764764764765	-1.52602270951859\\
7.65765765765766	-1.54192835702636\\
7.66766766766767	-1.55764486605386\\
7.67767767767768	-1.57316983756861\\
7.68768768768769	-1.58850091596819\\
7.6976976976977	-1.60363578917476\\
7.70770770770771	-1.61857218913647\\
7.71771771771772	-1.63330789173073\\
7.72772772772773	-1.64784071728387\\
7.73773773773774	-1.6621685303836\\
7.74774774774775	-1.67628924113661\\
7.75775775775776	-1.69020080488184\\
7.76776776776777	-1.70390122298626\\
7.77777777777778	-1.71738854354285\\
7.78778778778779	-1.73066086212068\\
7.7977977977978	-1.74371632175779\\
7.80780780780781	-1.75655311467292\\
7.81781781781782	-1.76916948257135\\
7.82782782782783	-1.78156371756743\\
7.83783783783784	-1.79373416295503\\
7.84784784784785	-1.80567921481827\\
7.85785785785786	-1.81739732220749\\
7.86786786786787	-1.82888698887338\\
7.87787787787788	-1.84014677407424\\
7.88788788788789	-1.85117529367943\\
7.8978978978979	-1.86197122155845\\
7.90790790790791	-1.872533290986\\
7.91791791791792	-1.88286029532025\\
7.92792792792793	-1.89295108972035\\
7.93793793793794	-1.90280459249763\\
7.94794794794795	-1.91241978643055\\
7.95795795795796	-1.92179571960399\\
7.96796796796797	-1.93093150776745\\
7.97797797797798	-1.93982633514538\\
7.98798798798799	-1.94847945538814\\
7.997997997998	-1.95689019424581\\
8.00800800800801	-1.96505794974413\\
8.01801801801802	-1.97298219396503\\
8.02802802802803	-1.98066247549876\\
8.03803803803804	-1.9880984187177\\
8.04804804804805	-1.99528972736341\\
8.05805805805806	-2.00223618442869\\
8.06806806806807	-2.0089376541393\\
8.07807807807808	-2.01539408327227\\
8.08808808808809	-2.02160550243917\\
8.0980980980981	-2.02757202718721\\
8.10810810810811	-2.03329385910874\\
8.11811811811812	-2.03877128776671\\
8.12812812812813	-2.04400469123536\\
8.13813813813814	-2.04899453711798\\
8.14814814814815	-2.05374138401962\\
8.15815815815816	-2.05824588229543\\
8.16816816816817	-2.06250877545388\\
8.17817817817818	-2.06653090031911\\
8.18818818818819	-2.0703131885399\\
8.1981981981982	-2.07385666703743\\
8.20820820820821	-2.07716245863016\\
8.21821821821822	-2.08023178319788\\
8.22822822822823	-2.08306595763881\\
8.23823823823824	-2.08566639624779\\
8.24824824824825	-2.08803461205761\\
8.25825825825826	-2.09017221645963\\
8.26826826826827	-2.092080919362\\
8.27827827827828	-2.09376252986246\\
8.28828828828829	-2.09521895564135\\
8.2982982982983	-2.09645220375727\\
8.30830830830831	-2.09746437986017\\
8.31831831831832	-2.09825768809715\\
8.32832832832833	-2.09883443163791\\
8.33833833833834	-2.0991970109078\\
8.34834834834835	-2.09934792429402\\
8.35835835835836	-2.09928976684301\\
8.36836836836837	-2.09902522992187\\
8.37837837837838	-2.09855710083031\\
8.38838838838839	-2.09788826021792\\
8.3983983983984	-2.09702168389242\\
8.40840840840841	-2.095960439198\\
8.41841841841842	-2.09470768519566\\
8.42842842842843	-2.09326667110268\\
8.43843843843844	-2.09164073462639\\
8.44844844844845	-2.08983330095632\\
8.45845845845846	-2.08784788104939\\
8.46846846846847	-2.08568806981792\\
8.47847847847848	-2.08335754425479\\
8.48848848848849	-2.08086006211955\\
8.4984984984985	-2.07819945954144\\
8.50850850850851	-2.07537964934153\\
8.51851851851852	-2.07240461810706\\
8.52852852852853	-2.06927842495822\\
8.53853853853854	-2.06600519874693\\
8.54854854854855	-2.06258913582212\\
8.55855855855856	-2.05903449721669\\
8.56856856856857	-2.05534560691863\\
8.57857857857858	-2.05152684791468\\
8.58858858858859	-2.04758266088883\\
8.5985985985986	-2.04351754050158\\
8.60860860860861	-2.03933603315549\\
8.61861861861862	-2.03504273367964\\
8.62862862862863	-2.03064228215071\\
8.63863863863864	-2.02613936138785\\
8.64864864864865	-2.02153869382076\\
8.65865865865866	-2.01684503776455\\
8.66866866866867	-2.01206318473101\\
8.67867867867868	-2.00719795646179\\
8.68868868868869	-2.0022542005752\\
8.6986986986987	-1.99723678792901\\
8.70870870870871	-1.99215060938032\\
8.71871871871872	-1.98700057228072\\
8.72872872872873	-1.98179159657874\\
8.73873873873874	-1.97652861185455\\
8.74874874874875	-1.97121655323013\\
8.75875875875876	-1.96586035886983\\
8.76876876876877	-1.96046496553405\\
8.77877877877878	-1.9550353052721\\
8.78878878878879	-1.9495763018483\\
8.7987987987988	-1.9440928674295\\
8.80880880880881	-1.93858989883145\\
8.81881881881882	-1.93307227371598\\
8.82882882882883	-1.92754484776062\\
8.83883883883884	-1.92201245025319\\
8.84884884884885	-1.91647988131971\\
8.85885885885886	-1.9109519071256\\
8.86886886886887	-1.90543325884041\\
8.87887887887888	-1.89992862571539\\
8.88888888888889	-1.89444265554918\\
8.8988988988989	-1.8889799478139\\
8.90890890890891	-1.88354505221616\\
8.91891891891892	-1.87814246532581\\
8.92892892892893	-1.87277662597689\\
8.93893893893894	-1.86745191390565\\
8.94894894894895	-1.86217264478407\\
8.95895895895896	-1.85694306822149\\
8.96896896896897	-1.85176736416054\\
8.97897897897898	-1.84664964012453\\
8.98898898898899	-1.84159392775617\\
8.998998998999	-1.83660418085084\\
9.00900900900901	-1.83168427143555\\
9.01901901901902	-1.82683798752444\\
9.02902902902903	-1.82206903086054\\
9.03903903903904	-1.81738101307035\\
9.04904904904905	-1.81277745497498\\
9.05905905905906	-1.80826178189274\\
9.06906906906907	-1.80383732289139\\
9.07907907907908	-1.79950730840287\\
9.08908908908909	-1.79527486704136\\
9.0990990990991	-1.79114302452201\\
9.10910910910911	-1.7871147010163\\
9.11911911911912	-1.78319270901804\\
9.12912912912913	-1.77937975277776\\
9.13913913913914	-1.77567842498817\\
9.14914914914915	-1.77209120594048\\
9.15915915915916	-1.76862046186291\\
9.16916916916917	-1.76526844400263\\
9.17917917917918	-1.76203728632908\\
9.18918918918919	-1.75892900452548\\
9.1991991991992	-1.7559454957335\\
9.20920920920921	-1.75308853627744\\
9.21921921921922	-1.7503597821539\\
9.22922922922923	-1.74776076663049\\
9.23923923923924	-1.74529290056981\\
9.24924924924925	-1.74295747241269\\
9.25925925925926	-1.74075564546514\\
9.26926926926927	-1.73868846025211\\
9.27927927927928	-1.736756832458\\
9.28928928928929	-1.73496155327805\\
9.2992992992993	-1.73330328940273\\
9.30930930930931	-1.73178258315077\\
9.31931931931932	-1.73039985279308\\
9.32932932932933	-1.72915539188418\\
9.33933933933934	-1.72804937138711\\
9.34934934934935	-1.72708183853973\\
9.35935935935936	-1.7262527179691\\
9.36936936936937	-1.72556181302131\\
9.37937937937938	-1.72500880571239\\
9.38938938938939	-1.72459325853258\\
9.3993993993994	-1.72431461468003\\
9.40940940940941	-1.72417219923978\\
9.41941941941942	-1.72416522152386\\
9.42942942942943	-1.72429277439088\\
9.43943943943944	-1.7245538382809\\
9.44944944944945	-1.72494728021273\\
9.45945945945946	-1.72547185736053\\
9.46946946946947	-1.72612621728335\\
9.47947947947948	-1.72690890093949\\
9.48948948948949	-1.72781834402731\\
9.4994994994995	-1.72885287916888\\
9.50950950950951	-1.73001073747081\\
9.51951951951952	-1.73129005165858\\
9.52952952952953	-1.73268885745328\\
9.53953953953954	-1.73420509670707\\
9.54954954954955	-1.73583661888818\\
9.55955955955956	-1.73758118425894\\
9.56956956956957	-1.73943646630843\\
9.57957957957958	-1.74140005459912\\
9.58958958958959	-1.74346945671306\\
9.5995995995996	-1.74564210188385\\
9.60960960960961	-1.74791534372674\\
9.61961961961962	-1.75028646227716\\
9.62962962962963	-1.75275266785869\\
9.63963963963964	-1.75531110421385\\
9.64964964964965	-1.75795885069968\\
9.65965965965966	-1.76069292607273\\
9.66966966966967	-1.76351029111243\\
9.67967967967968	-1.76640785292066\\
9.68968968968969	-1.76938246658913\\
9.6996996996997	-1.77243093988266\\
9.70970970970971	-1.77555003553682\\
9.71971971971972	-1.77873647527893\\
9.72972972972973	-1.78198694277738\\
9.73973973973974	-1.78529808741622\\
9.74974974974975	-1.78866652721339\\
9.75975975975976	-1.79208885246262\\
9.76976976976977	-1.79556162952913\\
9.77977977977978	-1.79908140374038\\
9.78978978978979	-1.80264470298138\\
9.7997997997998	-1.806248041457\\
9.80980980980981	-1.80988792261252\\
9.81981981981982	-1.81356084324422\\
9.82982982982983	-1.81726329615302\\
9.83983983983984	-1.82099177419143\\
9.84984984984985	-1.82474277352821\\
9.85985985985986	-1.82851279652454\\
9.86986986986987	-1.83229835606337\\
9.87987987987988	-1.8360959781682\\
9.88988988988989	-1.83990220493988\\
9.8998998998999	-1.84371359879547\\
9.90990990990991	-1.8475267456533\\
9.91991991991992	-1.85133825683159\\
9.92992992992993	-1.85514477386538\\
9.93993993993994	-1.85894297064724\\
9.94994994994995	-1.86272955520046\\
9.95995995995996	-1.86650127707846\\
9.96996996996997	-1.87025492272447\\
9.97997997997998	-1.87398732967411\\
9.98998998998999	-1.87769537620427\\
10	-1.8813759925314\\
};
\addplot [color=mycolor3,solid]
  table[row sep=crcr]{%
0	0.533357814069089\\
0.01001001001001	0.51739401835354\\
0.02002002002002	0.501272238550059\\
0.03003003003003	0.485000841063131\\
0.04004004004004	0.468588224071048\\
0.0500500500500501	0.452042812537083\\
0.0600600600600601	0.435373051024441\\
0.0700700700700701	0.418587395776147\\
0.0800800800800801	0.401694310953685\\
0.0900900900900901	0.38470225661428\\
0.1001001001001	0.367619688206744\\
0.11011011011011	0.350455044317976\\
0.12012012012012	0.333216743731448\\
0.13013013013013	0.315913179261183\\
0.14014014014014	0.298552706961338\\
0.15015015015015	0.281143645221805\\
0.16016016016016	0.26369426437529\\
0.17017017017017	0.246212783249478\\
0.18018018018018	0.22870736213217\\
0.19019019019019	0.211186095544697\\
0.2002002002002	0.193657009537659\\
0.21021021021021	0.176128051674713\\
0.22022022022022	0.158607089782691\\
0.23023023023023	0.141101904126943\\
0.24024024024024	0.123620181541914\\
0.25025025025025	0.10616951346969\\
0.26026026026026	0.088757386482004\\
0.27027027027027	0.0713911803245425\\
0.28028028028028	0.0540781625445982\\
0.29029029029029	0.0368254838945623\\
0.3003003003003	0.019640173270889\\
0.31031031031031	0.00252913407337344\\
0.32032032032032	-0.0145008588569727\\
0.33033033033033	-0.0314431667534804\\
0.34034034034034	-0.048291285493429\\
0.35035035035035	-0.0650388537819327\\
0.36036036036036	-0.0816796546418752\\
0.37037037037037	-0.0982076180864329\\
0.38038038038038	-0.114616825173862\\
0.39039039039039	-0.13090151052033\\
0.4004004004004	-0.147056064853029\\
0.41041041041041	-0.163075038585636\\
0.42042042042042	-0.178953142437949\\
0.43043043043043	-0.194685250714839\\
0.44044044044044	-0.210266402371702\\
0.45045045045045	-0.225691804301577\\
0.46046046046046	-0.24095683067356\\
0.47047047047047	-0.256057026536542\\
0.48048048048048	-0.270988107499764\\
0.49049049049049	-0.285745962331773\\
0.500500500500501	-0.300326650739219\\
0.510510510510511	-0.314726407055109\\
0.520520520520521	-0.328941639944772\\
0.530530530530531	-0.342968930764497\\
0.540540540540541	-0.356805037612775\\
0.550550550550551	-0.370446889823046\\
0.560560560560561	-0.383891592586588\\
0.570570570570571	-0.397136423580001\\
0.580580580580581	-0.410178833278906\\
0.590590590590591	-0.42301644438325\\
0.600600600600601	-0.435647049686465\\
0.610610610610611	-0.448068612866128\\
0.620620620620621	-0.460279264226513\\
0.630630630630631	-0.472277303302817\\
0.640640640640641	-0.484061192743021\\
0.650650650650651	-0.495629559236845\\
0.660660660660661	-0.506981191976031\\
0.670670670670671	-0.518115038252658\\
0.680680680680681	-0.529030202889998\\
0.690690690690691	-0.539725946906101\\
0.700700700700701	-0.550201680089766\\
0.710710710710711	-0.560456965883656\\
0.720720720720721	-0.570491511296008\\
0.730730730730731	-0.580305170574456\\
0.740740740740741	-0.58989793565327\\
0.750750750750751	-0.599269938412162\\
0.760760760760761	-0.608421445898154\\
0.770770770770771	-0.617352853336611\\
0.780780780780781	-0.62606468751629\\
0.790790790790791	-0.634557598851631\\
0.800800800800801	-0.642832356576663\\
0.810810810810811	-0.650889850199117\\
0.820820820820821	-0.658731080914619\\
0.830830830830831	-0.666357160390527\\
0.840840840840841	-0.673769305749134\\
0.850850850850851	-0.680968836653508\\
0.860860860860861	-0.687957171266518\\
0.870870870870871	-0.694735822290712\\
0.880880880880881	-0.701306390974848\\
0.890890890890891	-0.707670566416509\\
0.900900900900901	-0.713830118068529\\
0.910910910910911	-0.719786895088463\\
0.920920920920921	-0.725542820780648\\
0.930930930930931	-0.731099886630218\\
0.940940940940941	-0.736460151211177\\
0.950950950950951	-0.741625734216859\\
0.960960960960961	-0.746598813685867\\
0.970970970970971	-0.751381619494894\\
0.980980980980981	-0.755976433864276\\
0.990990990990991	-0.760385582402147\\
1.001001001001	-0.764611433759735\\
1.01101101101101	-0.768656391566688\\
1.02102102102102	-0.772522896853003\\
1.03103103103103	-0.77621341873819\\
1.04104104104104	-0.779730451264905\\
1.05105105105105	-0.783076514697135\\
1.06106106106106	-0.78625414289783\\
1.07107107107107	-0.789265890084822\\
1.08108108108108	-0.792114318196097\\
1.09109109109109	-0.794801999588354\\
1.1011011011011	-0.797331511047542\\
1.11111111111111	-0.799705430265307\\
1.12112112112112	-0.80192633347203\\
1.13113113113113	-0.803996792536651\\
1.14114114114114	-0.805919372457432\\
1.15115115115115	-0.807696624660484\\
1.16116116116116	-0.809331089700197\\
1.17117117117117	-0.810825290791737\\
1.18118118118118	-0.812181732474362\\
1.19119119119119	-0.813402898121754\\
1.2012012012012	-0.814491246997527\\
1.21121121121121	-0.815449212494828\\
1.22122122122122	-0.816279198581838\\
1.23123123123123	-0.816983580758808\\
1.24124124124124	-0.817564699907631\\
1.25125125125125	-0.818024864989639\\
1.26126126126126	-0.818366346621596\\
1.27127127127127	-0.818591380261015\\
1.28128128128128	-0.818702158637668\\
1.29129129129129	-0.818700838066381\\
1.3013013013013	-0.818589530042698\\
1.31131131131131	-0.818370304729661\\
1.32132132132132	-0.818045186257332\\
1.33133133133133	-0.817616155293998\\
1.34134134134134	-0.817085145963739\\
1.35135135135135	-0.816454045375174\\
1.36136136136136	-0.815724693380377\\
1.37137137137137	-0.814898882908582\\
1.38138138138138	-0.813978355950929\\
1.39139139139139	-0.812964809128577\\
1.4014014014014	-0.811859887655847\\
1.41141141141141	-0.810665188299682\\
1.42142142142142	-0.809382260191639\\
1.43143143143143	-0.808012600418595\\
1.44144144144144	-0.806557659564458\\
1.45145145145145	-0.805018838066386\\
1.46146146146146	-0.80339749048389\\
1.47147147147147	-0.801694919032273\\
1.48148148148148	-0.799912382252808\\
1.49149149149149	-0.798051088581341\\
1.5015015015015	-0.796112202739146\\
1.51151151151151	-0.794096841917038\\
1.52152152152152	-0.792006077682729\\
1.53153153153153	-0.789840940311405\\
1.54154154154154	-0.787602411586698\\
1.55155155155155	-0.785291435392762\\
1.56156156156156	-0.782908911716841\\
1.57157157157157	-0.780455700107532\\
1.58158158158158	-0.777932620550004\\
1.59159159159159	-0.775340454180427\\
1.6016016016016	-0.772679945252429\\
1.61161161161161	-0.76995180120651\\
1.62162162162162	-0.76715669341056\\
1.63163163163163	-0.764295259904171\\
1.64164164164164	-0.761368106669854\\
1.65165165165165	-0.758375806864295\\
1.66166166166166	-0.755318901375121\\
1.67167167167167	-0.75219790549788\\
1.68168168168168	-0.74901330400312\\
1.69169169169169	-0.745765554766268\\
1.7017017017017	-0.742455090146452\\
1.71171171171171	-0.739082318841966\\
1.72172172172172	-0.735647625245216\\
1.73173173173173	-0.732151371525396\\
1.74174174174174	-0.728593898975518\\
1.75175175175175	-0.724975530166228\\
1.76176176176176	-0.7212965673201\\
1.77177177177177	-0.71755729586521\\
1.78178178178178	-0.713757983998031\\
1.79179179179179	-0.709898885810536\\
1.8018018018018	-0.705980240177313\\
1.81181181181181	-0.702002271008337\\
1.82182182182182	-0.6979651920472\\
1.83183183183183	-0.693869203829705\\
1.84184184184184	-0.689714496682046\\
1.85185185185185	-0.685501251778271\\
1.86186186186186	-0.681229639125501\\
1.87187187187187	-0.676899822314889\\
1.88188188188188	-0.672511956743345\\
1.89189189189189	-0.66806619065925\\
1.9019019019019	-0.663562665690054\\
1.91191191191191	-0.659001518779402\\
1.92192192192192	-0.654382880449686\\
1.93193193193193	-0.649706878771218\\
1.94194194194194	-0.644973634870903\\
1.95195195195195	-0.640183267314944\\
1.96196196196196	-0.635335892439426\\
1.97197197197197	-0.630431622227229\\
1.98198198198198	-0.625470566320839\\
1.99199199199199	-0.620452832194559\\
2.002002002002	-0.615378524496787\\
2.01201201201201	-0.610247746307942\\
2.02202202202202	-0.605060600507875\\
2.03203203203203	-0.599817184487424\\
2.04204204204204	-0.594517596976598\\
2.05205205205205	-0.589161935040499\\
2.06206206206206	-0.583750292011778\\
2.07207207207207	-0.578282761788412\\
2.08208208208208	-0.572759436092432\\
2.09209209209209	-0.567180404428934\\
2.1021021021021	-0.561545754919069\\
2.11211211211211	-0.555855573252507\\
2.12212212212212	-0.550109944861671\\
2.13213213213213	-0.544308951273217\\
2.14214214214214	-0.538452672481991\\
2.15215215215215	-0.532541185090931\\
2.16216216216216	-0.5265745648879\\
2.17217217217217	-0.520552882924878\\
2.18218218218218	-0.514476209099542\\
2.19219219219219	-0.508344608296651\\
2.2022022022022	-0.502158142547781\\
2.21221221221221	-0.495916871667144\\
2.22222222222222	-0.489620849894898\\
2.23223223223223	-0.483270127974143\\
2.24224224224224	-0.476864752972563\\
2.25225225225225	-0.470404766435164\\
2.26226226226226	-0.463890207268287\\
2.27227227227227	-0.457321107612422\\
2.28228228228228	-0.450697496794215\\
2.29229229229229	-0.444019397308558\\
2.3023023023023	-0.437286828637452\\
2.31231231231231	-0.43049980265416\\
2.32232232232232	-0.423658328684652\\
2.33233233233233	-0.416762409286881\\
2.34234234234234	-0.409812041849614\\
2.35235235235235	-0.402807219786193\\
2.36236236236236	-0.395747930199658\\
2.37237237237237	-0.388634155615339\\
2.38238238238238	-0.381465874097548\\
2.39239239239239	-0.374243057548056\\
2.4024024024024	-0.366965675141314\\
2.41241241241241	-0.359633690479067\\
2.42242242242242	-0.352247063428756\\
2.43243243243243	-0.344805749491589\\
2.44244244244244	-0.337309701474215\\
2.45245245245245	-0.329758867762969\\
2.46246246246246	-0.322153195819428\\
2.47247247247247	-0.31449262945506\\
2.48248248248248	-0.306777109804603\\
2.49249249249249	-0.299006578938185\\
2.5025025025025	-0.291180975093536\\
2.51251251251251	-0.283300239062701\\
2.52252252252252	-0.275364310118581\\
2.53253253253253	-0.267373128693727\\
2.54254254254254	-0.259326636377285\\
2.55255255255255	-0.251224778902906\\
2.56256256256256	-0.243067502513287\\
2.57257257257257	-0.234854758564828\\
2.58258258258258	-0.22658650216147\\
2.59259259259259	-0.218262694426086\\
2.6026026026026	-0.209883302676311\\
2.61261261261261	-0.201448300788644\\
2.62262262262262	-0.192957669341315\\
2.63263263263263	-0.184411399769766\\
2.64264264264264	-0.175809491846152\\
2.65265265265265	-0.167151956707913\\
2.66266266266266	-0.158438815703536\\
2.67267267267267	-0.149670104036116\\
2.68268268268268	-0.140845869102543\\
2.69269269269269	-0.131966174093219\\
2.7027027027027	-0.123031095898664\\
2.71271271271271	-0.114040728999687\\
2.72272272272272	-0.104995184241217\\
2.73273273273273	-0.0958945918215236\\
2.74274274274274	-0.0867391000090269\\
2.75275275275275	-0.0775288775786779\\
2.76276276276276	-0.068264115464678\\
2.77277277277277	-0.0589450241545237\\
2.78278278278278	-0.049571839471503\\
2.79279279279279	-0.0401448198056389\\
2.8028028028028	-0.0306642496351603\\
2.81281281281281	-0.0211304373328795\\
2.82282282282282	-0.0115437178200584\\
2.83283283283283	-0.0019044558570996\\
2.84284284284284	0.00778696026044936\\
2.85285285285285	0.0175301094181437\\
2.86286286286286	0.0273245439154549\\
2.87287287287287	0.0371697832128437\\
2.88288288288288	0.0470653191207067\\
2.89289289289289	0.0570106106040893\\
2.9029029029029	0.0670050834231011\\
2.91291291291291	0.0770481333763779\\
2.92292292292292	0.087139123030935\\
2.93293293293293	0.0972773802868457\\
2.94294294294294	0.107462200768197\\
2.95295295295295	0.117692845589162\\
2.96296296296296	0.127968542101135\\
2.97297297297297	0.138288484172332\\
2.98298298298298	0.148651828530055\\
2.99299299299299	0.159057701173973\\
3.003003003003	0.169505190502789\\
3.01301301301301	0.179993352516721\\
3.02302302302302	0.190521206962675\\
3.03303303303303	0.201087740337014\\
3.04304304304304	0.211691906321287\\
3.05305305305305	0.222332622249943\\
3.06306306306306	0.233008774518816\\
3.07307307307307	0.243719215697024\\
3.08308308308308	0.25446276529612\\
3.09309309309309	0.265238211780842\\
3.1031031031031	0.276044312508379\\
3.11311311311311	0.286879793021248\\
3.12312312312312	0.297743349468578\\
3.13313313313313	0.308633649181923\\
3.14314314314314	0.319549330651807\\
3.15315315315315	0.330489005817482\\
3.16316316316316	0.34145125923701\\
3.17317317317317	0.352434650358474\\
3.18318318318318	0.363437714696332\\
3.19319319319319	0.3744589640373\\
3.2032032032032	0.385496888122596\\
3.21321321321321	0.396549956370739\\
3.22322322322322	0.407616618382176\\
3.23323323323323	0.418695305198805\\
3.24324324324324	0.429784431679595\\
3.25325325325325	0.440882396886349\\
3.26326326326326	0.451987586584263\\
3.27327327327327	0.463098372780353\\
3.28328328328328	0.474213117704353\\
3.29329329329329	0.485330174026605\\
3.3033033033033	0.496447886167978\\
3.31331331331331	0.507564593731944\\
3.32332332332332	0.518678629830108\\
3.33333333333333	0.529788327439321\\
3.34334334334334	0.540892015286655\\
3.35335335335335	0.551988025351965\\
3.36336336336336	0.563074691295453\\
3.37337337337337	0.574150350005481\\
3.38338338338338	0.585213345939824\\
3.39339339339339	0.596262028908084\\
3.4034034034034	0.607294759561322\\
3.41341341341341	0.618309910553111\\
3.42342342342342	0.629305865324043\\
3.43343343343343	0.640281023616191\\
3.44344344344344	0.651233800594144\\
3.45345345345345	0.662162629774805\\
3.46346346346346	0.673065964724647\\
3.47347347347347	0.68394228043719\\
3.48348348348348	0.694790074293232\\
3.49349349349349	0.705607870498468\\
3.5035035035035	0.716394217620343\\
3.51351351351351	0.727147692605505\\
3.52352352352352	0.737866903321543\\
3.53353353353353	0.748550486989681\\
3.54354354354354	0.759197113131279\\
3.55355355355355	0.769805487781356\\
3.56356356356356	0.780374349586271\\
3.57357357357357	0.790902474723372\\
3.58358358358358	0.801388676895519\\
3.59359359359359	0.811831810070134\\
3.6036036036036	0.822230768028633\\
3.61361361361361	0.832584484737023\\
3.62362362362362	0.84289193841367\\
3.63363363363363	0.853152148346606\\
3.64364364364364	0.863364180772335\\
3.65365365365365	0.873527145475434\\
3.66366366366366	0.883640198926347\\
3.67367367367367	0.893702542718878\\
3.68368368368368	0.903713428343941\\
3.69369369369369	0.913672152259133\\
3.7037037037037	0.923578061729429\\
3.71371371371371	0.933430551266724\\
3.72372372372372	0.943229063772897\\
3.73373373373373	0.952973093579386\\
3.74374374374374	0.962662182551558\\
3.75375375375375	0.972295923769974\\
3.76376376376376	0.981873959069099\\
3.77377377377377	0.991395977781193\\
3.78378378378378	1.00086172255285\\
3.79379379379379	1.0102709825779\\
3.8038038038038	1.01962359625189\\
3.81381381381381	1.02891944930485\\
3.82382382382382	1.0381584783516\\
3.83383383383383	1.04734066365517\\
3.84384384384384	1.05646603443432\\
3.85385385385385	1.06553466637421\\
3.86386386386386	1.07454667801575\\
3.87387387387387	1.08350223488051\\
3.88388388388388	1.09240154309561\\
3.89389389389389	1.10124485308453\\
3.9039039039039	1.11003245472341\\
3.91391391391391	1.11876468055805\\
3.92392392392392	1.12744189938209\\
3.93393393393393	1.13606451779391\\
3.94394394394394	1.1446329797247\\
3.95395395395395	1.15314776236546\\
3.96396396396396	1.16160937641566\\
3.97397397397397	1.170018363315\\
3.98398398398398	1.17837529621228\\
3.99399399399399	1.18668077468194\\
4.004004004004	1.19493542418933\\
4.01401401401401	1.2031398968729\\
4.02402402402402	1.2112948651534\\
4.03403403403403	1.21940102455266\\
4.04404404404404	1.22745908736732\\
4.05405405405405	1.23546978429991\\
4.06406406406406	1.24343385912567\\
4.07407407407407	1.25135207073694\\
4.08408408408408	1.25922518539645\\
4.09409409409409	1.2670539808363\\
4.1041041041041	1.2748392398516\\
4.11411411411411	1.28258174986306\\
4.12412412412412	1.2902822990313\\
4.13413413413413	1.29794167696835\\
4.14414414414414	1.30556066988662\\
4.15415415415415	1.31314005805041\\
4.16416416416416	1.32068061867468\\
4.17417417417417	1.32818311538322\\
4.18418418418418	1.33564830144932\\
4.19419419419419	1.3430769183589\\
4.2042042042042	1.35046969070857\\
4.21421421421421	1.35782732318697\\
4.22422422422422	1.36515050298352\\
4.23423423423423	1.37243989272409\\
4.24424424424424	1.37969613054361\\
4.25425425425425	1.38691982841733\\
4.26426426426426	1.39411156951129\\
4.27427427427427	1.40127190432845\\
4.28428428428428	1.40840135098865\\
4.29429429429429	1.41550039346488\\
4.3043043043043	1.42256947767767\\
4.31431431431431	1.42960900941024\\
4.32432432432432	1.43661935589805\\
4.33433433433433	1.44360083949808\\
4.34434434434434	1.45055373924198\\
4.35435435435435	1.45747828585065\\
4.36436436436436	1.46437466362382\\
4.37437437437437	1.4712430088649\\
4.38438438438438	1.47808340269768\\
4.39439439439439	1.48489587485057\\
4.4044044044044	1.49168040366701\\
4.41441441441441	1.49843690852266\\
4.42442442442442	1.50516525333193\\
4.43443443443443	1.51186524431271\\
4.44444444444444	1.51853662860096\\
4.45445445445445	1.52517909162163\\
4.46446446446446	1.53179225947304\\
4.47447447447447	1.53837569545653\\
4.48448448448448	1.54492889832362\\
4.49449449449449	1.55145130590569\\
4.5045045045045	1.55794229086913\\
4.51451451451451	1.56440115850603\\
4.52452452452452	1.570827152133\\
4.53453453453453	1.57721944783386\\
4.54454454454454	1.58357715441812\\
4.55455455455455	1.58989931670157\\
4.56456456456456	1.59618491197589\\
4.57457457457457	1.60243284819329\\
4.58458458458458	1.60864197271264\\
4.59459459459459	1.61481106163956\\
4.6046046046046	1.62093882580938\\
4.61461461461461	1.62702391303261\\
4.62462462462462	1.63306490153155\\
4.63463463463463	1.63906030808536\\
4.64464464464464	1.64500858214039\\
4.65465465465465	1.65090811215104\\
4.66466466466466	1.65675722143094\\
4.67467467467467	1.66255417322574\\
4.68468468468468	1.66829716756485\\
4.69469469469469	1.67398434488926\\
4.7047047047047	1.6796137870691\\
4.71471471471471	1.68518351833018\\
4.72472472472472	1.69069150490625\\
4.73473473473473	1.69613565795722\\
4.74474474474474	1.70151383577706\\
4.75475475475475	1.70682384208282\\
4.76476476476476	1.71206343184988\\
4.77477477477477	1.71723030907057\\
4.78478478478478	1.72232213139566\\
4.79479479479479	1.72733650942957\\
4.8048048048048	1.73227100961552\\
4.81481481481481	1.73712315748911\\
4.82482482482482	1.74189043794461\\
4.83483483483483	1.74657029559826\\
4.84484484484484	1.75116014320569\\
4.85485485485485	1.75565735486093\\
4.86486486486486	1.76005927557259\\
4.87487487487487	1.76436321882095\\
4.88488488488488	1.76856647341416\\
4.89489489489489	1.772666300707\\
4.9049049049049	1.77665993971093\\
4.91491491491491	1.78054461052048\\
4.92492492492492	1.78431751347213\\
4.93493493493493	1.78797583430424\\
4.94494494494494	1.79151674620858\\
4.95495495495495	1.79493741057803\\
4.96496496496496	1.79823498315304\\
4.97497497497497	1.80140661262618\\
4.98498498498498	1.80444944415004\\
4.99499499499499	1.80736062479653\\
5.00500500500501	1.8101373036012\\
5.01501501501502	1.81277663443798\\
5.02502502502503	1.81527577715807\\
5.03503503503504	1.8176319057646\\
5.04504504504505	1.81984220375619\\
5.05505505505506	1.82190387151553\\
5.06506506506507	1.82381412859097\\
5.07507507507508	1.82557021362927\\
5.08508508508509	1.82716939156568\\
5.0950950950951	1.82860894986876\\
5.10510510510511	1.82988620821002\\
5.11511511511512	1.83099851525602\\
5.12512512512513	1.83194325433045\\
5.13513513513514	1.8327178461935\\
5.14514514514515	1.83331974809559\\
5.15515515515516	1.83374646165318\\
5.16516516516517	1.83399552956151\\
5.17517517517518	1.83406454366436\\
5.18518518518519	1.83395114269105\\
5.1951951951952	1.83365301709319\\
5.20520520520521	1.83316791001722\\
5.21521521521522	1.83249362125316\\
5.22522522522523	1.83162800885939\\
5.23523523523524	1.8305689897477\\
5.24524524524525	1.82931454470131\\
5.25525525525526	1.82786271707912\\
5.26526526526527	1.8262116181868\\
5.27527527527528	1.82435942779543\\
5.28528528528529	1.82230439533984\\
5.2952952952953	1.82004484338039\\
5.30530530530531	1.81757916750579\\
5.31531531531532	1.8149058407643\\
5.32532532532533	1.81202341359054\\
5.33533533533534	1.80893051538976\\
5.34534534534535	1.80562585596404\\
5.35535535535536	1.802108229806\\
5.36536536536537	1.79837651319682\\
5.37537537537538	1.79442966880933\\
5.38538538538539	1.79026674548424\\
5.3953953953954	1.78588688286967\\
5.40540540540541	1.78128930614549\\
5.41541541541542	1.77647333308904\\
5.42542542542543	1.77143837259982\\
5.43543543543544	1.76618392699877\\
5.44544544544545	1.76070959038422\\
5.45545545545546	1.75501505338872\\
5.46546546546547	1.74910009918995\\
5.47547547547548	1.74296461012017\\
5.48548548548549	1.73660856289146\\
5.4954954954955	1.73003203182956\\
5.50550550550551	1.7232351890397\\
5.51551551551552	1.71621830624776\\
5.52552552552553	1.70898175041997\\
5.53553553553554	1.70152599138466\\
5.54554554554555	1.69385159456845\\
5.55555555555556	1.68595922687578\\
5.56556556556557	1.67784965413127\\
5.57557557557558	1.66952374034814\\
5.58558558558559	1.66098245001595\\
5.5955955955956	1.65222684657696\\
5.60560560560561	1.64325809213613\\
5.61561561561562	1.63407744714953\\
5.62562562562563	1.62468627071966\\
5.63563563563564	1.6150860211701\\
5.64564564564565	1.60527825162202\\
5.65565565565566	1.59526461434639\\
5.66566566566567	1.58504685631259\\
5.67567567567568	1.57462682135919\\
5.68568568568569	1.56400644661967\\
5.6956956956957	1.55318776505707\\
5.70570570570571	1.54217290038504\\
5.71571571571572	1.53096407099622\\
5.72572572572573	1.51956358483096\\
5.73573573573574	1.50797383944847\\
5.74574574574575	1.49619732219429\\
5.75575575575576	1.4842366087172\\
5.76576576576577	1.47209435867501\\
5.77577577577578	1.45977331888092\\
5.78578578578579	1.4472763191925\\
5.7957957957958	1.43460627143822\\
5.80580580580581	1.42176616823203\\
5.81581581581582	1.40875908114657\\
5.82582582582583	1.3955881589813\\
5.83583583583584	1.38225662753707\\
5.84584584584585	1.36876778575936\\
5.85585585585586	1.35512500419441\\
5.86586586586587	1.34133172531155\\
5.87587587587588	1.32739145990516\\
5.88588588588589	1.31330778399461\\
5.8958958958959	1.29908434008348\\
5.90590590590591	1.28472483229994\\
5.91591591591592	1.27023302542969\\
5.92592592592593	1.25561274284361\\
5.93593593593594	1.2408678646217\\
5.94594594594595	1.22600232442778\\
5.95595595595596	1.21102010778316\\
5.96596596596597	1.19592525044725\\
5.97597597597598	1.18072183559518\\
5.98598598598599	1.16541399052555\\
5.995995995996	1.15000588570321\\
6.00600600600601	1.13450173159916\\
6.01601601601602	1.11890577646254\\
6.02602602602603	1.10322230504563\\
6.03603603603604	1.08745563399012\\
6.04604604604605	1.07161010939794\\
6.05605605605606	1.05569010740875\\
6.06606606606607	1.03970002757601\\
6.07607607607608	1.023644294428\\
6.08608608608609	1.00752734984831\\
6.0960960960961	0.991353655666952\\
6.10610610610611	0.975127687918914\\
6.11611611611612	0.958853935102645\\
6.12612612612613	0.942536895111509\\
6.13613613613614	0.926181073396719\\
6.14614614614615	0.909790981142379\\
6.15615615615616	0.893371128827012\\
6.16616616616617	0.876926028700987\\
6.17617617617618	0.860460188668969\\
6.18618618618619	0.843978110594058\\
6.1961961961962	0.82748428752218\\
6.20620620620621	0.810983203238844\\
6.21621621621622	0.79447932512498\\
6.22622622622623	0.777977106034951\\
6.23623623623624	0.761480979867087\\
6.24624624624625	0.744995358426091\\
6.25625625625626	0.728524630099999\\
6.26626626626627	0.712073156654712\\
6.27627627627628	0.695645270083726\\
6.28628628628629	0.67924527391843\\
6.2962962962963	0.662877434473683\\
6.30630630630631	0.646545984140809\\
6.31631631631632	0.630255116185543\\
6.32632632632633	0.614008983193524\\
6.33633633633634	0.59781169480593\\
6.34634634634635	0.581667314167722\\
6.35635635635636	0.565579858841533\\
6.36636636636637	0.5495532963964\\
6.37637637637638	0.533591541032748\\
6.38638638638639	0.517698454944806\\
6.3963963963964	0.501877844485649\\
6.40640640640641	0.48613345698771\\
6.41641641641642	0.470468981354759\\
6.42642642642643	0.454888044729545\\
6.43643643643644	0.439394211184172\\
6.44644644644645	0.42399098010492\\
6.45645645645646	0.40868178321847\\
6.46646646646647	0.393469986605005\\
6.47647647647648	0.37835888424547\\
6.48648648648649	0.363351700130259\\
6.4964964964965	0.348451585346857\\
6.50650650650651	0.333661618173359\\
6.51651651651652	0.318984800919594\\
6.52652652652653	0.30442406115494\\
6.53653653653654	0.289982246785198\\
6.54654654654655	0.275662131580772\\
6.55655655655656	0.261466405586321\\
6.56656656656657	0.247397681788352\\
6.57657657657658	0.23345849215013\\
6.58658658658659	0.219651286115865\\
6.5965965965966	0.205978432476553\\
6.60660660660661	0.192442217118689\\
6.61661661661662	0.179044842216721\\
6.62662662662663	0.165788427669159\\
6.63663663663664	0.15267500889901\\
6.64664664664665	0.139706538600412\\
6.65665665665666	0.126884884215042\\
6.66666666666667	0.11421183131949\\
6.67667667667668	0.10168907906052\\
6.68668668668669	0.0893182454020419\\
6.6966966966967	0.0771008634196997\\
6.70670670670671	0.065038383078954\\
6.71671671671672	0.0531321728629078\\
6.72672672672673	0.0413835172578771\\
6.73673673673674	0.0297936200098401\\
6.74674674674675	0.0183636043544706\\
6.75675675675676	0.00709451169672161\\
6.76676676676677	-0.00401269547421807\\
6.77677677677678	-0.0149571320202557\\
6.78678678678679	-0.0257379930980799\\
6.7967967967968	-0.0363545491803155\\
6.80680680680681	-0.0468061460953222\\
6.81681681681682	-0.0570922052273345\\
6.82682682682683	-0.0672122202208992\\
6.83683683683684	-0.07716575763772\\
6.84684684684685	-0.0869524543936503\\
6.85685685685686	-0.0965720168499224\\
6.86686686686687	-0.106024218681387\\
6.87687687687688	-0.115308899903606\\
6.88688688688689	-0.124425966495762\\
6.8968968968969	-0.133375386439739\\
6.90690690690691	-0.142157189166269\\
6.91691691691692	-0.150771463821522\\
6.92692692692693	-0.15921835863881\\
6.93693693693694	-0.167498076375209\\
6.94694694694695	-0.175610875395922\\
6.95695695695696	-0.183557066042461\\
6.96696696696697	-0.19133700968892\\
6.97697697697698	-0.198951115571571\\
6.98698698698699	-0.20639984026576\\
6.996996996997	-0.213683685238118\\
7.00700700700701	-0.220803194608838\\
7.01701701701702	-0.22775895283191\\
7.02702702702703	-0.234551584473556\\
7.03703703703704	-0.241181748947829\\
7.04704704704705	-0.247650142389992\\
7.05705705705706	-0.253957493133963\\
7.06706706706707	-0.2601045587138\\
7.07707707707708	-0.266092128403338\\
7.08708708708709	-0.271921015666871\\
7.0970970970971	-0.277592060810858\\
7.10710710710711	-0.283106124697785\\
7.11711711711712	-0.288464091873851\\
7.12712712712713	-0.293666863714005\\
7.13713713713714	-0.298715360490751\\
7.14714714714715	-0.303610516091566\\
7.15715715715716	-0.308353279518234\\
7.16716716716717	-0.3129446104929\\
7.17717717717718	-0.317385478861534\\
7.18718718718719	-0.32167686393092\\
7.1971971971972	-0.325819750180077\\
7.20720720720721	-0.329815128268213\\
7.21721721721722	-0.333663991208182\\
7.22722722722723	-0.337367336318143\\
7.23723723723724	-0.340926159071913\\
7.24724724724725	-0.344341456841835\\
7.25725725725726	-0.347614222324619\\
7.26726726726727	-0.350745446584556\\
7.27727727727728	-0.353736116060174\\
7.28728728728729	-0.356587212487797\\
7.2972972972973	-0.359299709045835\\
7.30730730730731	-0.361874573529045\\
7.31731731731732	-0.364312764572004\\
7.32732732732733	-0.366615231585316\\
7.33733733733734	-0.368782914081643\\
7.34734734734735	-0.370816741127246\\
7.35735735735736	-0.372717631662608\\
7.36736736736737	-0.374486491227955\\
7.37737737737738	-0.376124214111771\\
7.38738738738739	-0.377631683053489\\
7.3973973973974	-0.379009767720169\\
7.40740740740741	-0.380259323588963\\
7.41741741741742	-0.381381196779817\\
7.42742742742743	-0.382376216093661\\
7.43743743743744	-0.383245201316793\\
7.44744744744745	-0.383988957581762\\
7.45745745745746	-0.384608278522261\\
7.46746746746747	-0.385103944741348\\
7.47747747747748	-0.385476725856018\\
7.48748748748749	-0.385727379439108\\
7.4974974974975	-0.385856654242393\\
7.50750750750751	-0.385865285477643\\
7.51751751751752	-0.385754002401793\\
7.52752752752753	-0.385523523690166\\
7.53753753753754	-0.385174560852617\\
7.54754754754755	-0.384707816964607\\
7.55755755755756	-0.384123990185223\\
7.56756756756757	-0.38342377409499\\
7.57757757757758	-0.382607856965826\\
7.58758758758759	-0.381676923397483\\
7.5975975975976	-0.380631659102065\\
7.60760760760761	-0.379472746748244\\
7.61761761761762	-0.378200869902688\\
7.62762762762763	-0.376816714316935\\
7.63763763763764	-0.375320971221298\\
7.64764764764765	-0.373714330793958\\
7.65765765765766	-0.371997495857254\\
7.66766766766767	-0.370171172156487\\
7.67767767767768	-0.368236076646004\\
7.68768768768769	-0.366192936000284\\
7.6976976976977	-0.364042487710125\\
7.70770770770771	-0.361785484113767\\
7.71771771771772	-0.35942269174598\\
7.72772772772773	-0.356954894476687\\
7.73773773773774	-0.354382891610632\\
7.74774774774775	-0.351707504878791\\
7.75775775775776	-0.348929575277237\\
7.76776776776777	-0.346049966098417\\
7.77777777777778	-0.343069565093122\\
7.78778778778779	-0.339989286802528\\
7.7977977977978	-0.336810069228281\\
7.80780780780781	-0.333532881366087\\
7.81781781781782	-0.330158721346611\\
7.82782782782783	-0.326688618055057\\
7.83783783783784	-0.323123631404362\\
7.84784784784785	-0.319464857752691\\
7.85785785785786	-0.315713425630322\\
7.86786786786787	-0.311870501115133\\
7.87787787787788	-0.307937286944385\\
7.88788788788789	-0.303915023470836\\
7.8978978978979	-0.299804990590688\\
7.90790790790791	-0.295608510170031\\
7.91791791791792	-0.291326942709452\\
7.92792792792793	-0.286961691302763\\
7.93793793793794	-0.282514202164916\\
7.94794794794795	-0.277985965859613\\
7.95795795795796	-0.273378513150125\\
7.96796796796797	-0.268693423421991\\
7.97797797797798	-0.263932320029384\\
7.98798798798799	-0.259096868579589\\
7.997997997998	-0.254188784859566\\
8.00800800800801	-0.249209827028564\\
8.01801801801802	-0.244161798386875\\
8.02802802802803	-0.239046553958568\\
8.03803803803804	-0.233865986381272\\
8.04804804804805	-0.228622040520289\\
8.05805805805806	-0.22331670287189\\
8.06806806806807	-0.217952005785103\\
8.07807807807808	-0.21253002686766\\
8.08808808808809	-0.20705288775621\\
8.0980980980981	-0.201522752912947\\
8.10810810810811	-0.1959418282046\\
8.11811811811812	-0.190312363895067\\
8.12812812812813	-0.18463665040847\\
8.13813813813814	-0.178917016765292\\
8.14814814814815	-0.173155831366596\\
8.15815815815816	-0.167355499958541\\
8.16816816816817	-0.16151846620673\\
8.17817817817818	-0.155647206684544\\
8.18818818818819	-0.149744232893631\\
8.1981981981982	-0.143812087567407\\
8.20820820820821	-0.137853343085579\\
8.21821821821822	-0.13187060259757\\
8.22822822822823	-0.125866494779113\\
8.23823823823824	-0.119843670483406\\
8.24824824824825	-0.113804807855315\\
8.25825825825826	-0.107752604925784\\
8.26826826826827	-0.101689777220017\\
8.27827827827828	-0.095619058928988\\
8.28828828828829	-0.0895431962150913\\
8.2982982982983	-0.0834649502830093\\
8.30830830830831	-0.0773870903214964\\
8.31831831831832	-0.0713123927836202\\
8.32832832832833	-0.0652436424347369\\
8.33833833833834	-0.0591836227249709\\
8.34834834834835	-0.0531351202240261\\
8.35835835835836	-0.0471009169963738\\
8.36836836836837	-0.0410837908857919\\
8.37837837837838	-0.0350865153496414\\
8.38838838838839	-0.0291118449536168\\
8.3983983983984	-0.0231625317789936\\
8.40840840840841	-0.0172413054601505\\
8.41841841841842	-0.0113508797027532\\
8.42842842842843	-0.00549394753617179\\
8.43843843843844	0.000326823595735908\\
8.44844844844845	0.00610879113601178\\
8.45845845845846	0.0118493410191254\\
8.46846846846847	0.0175458919528338\\
8.47847847847848	0.0231958983353978\\
8.48848848848849	0.0287968494048059\\
8.4984984984985	0.0343462750994737\\
8.50850850850851	0.0398417465522717\\
8.51851851851852	0.0452808828810485\\
8.52852852852853	0.0506613486962261\\
8.53853853853854	0.0559808585738275\\
8.54854854854855	0.0612371792693979\\
8.55855855855856	0.0664281346138451\\
8.56856856856857	0.0715516017169539\\
8.57857857857858	0.0766055225570862\\
8.58858858858859	0.0815878968753589\\
8.5985985985986	0.0864967903867886\\
8.60860860860861	0.0913303334751513\\
8.61861861861862	0.0960867250095732\\
8.62862862862863	0.100764236512524\\
8.63863863863864	0.105361210054094\\
8.64864864864865	0.109876060246896\\
8.65865865865866	0.114307280108673\\
8.66866866866867	0.11865343910674\\
8.67867867867868	0.122913182916855\\
8.68868868868869	0.12708524158496\\
8.6986986986987	0.131168426826107\\
8.70870870870871	0.135161631991068\\
8.71871871871872	0.139063833823961\\
8.72872872872873	0.142874096686193\\
8.73873873873874	0.146591570375922\\
8.74874874874875	0.150215494837893\\
8.75875875875876	0.153745193821823\\
8.76876876876877	0.157180082271523\\
8.77877877877878	0.160519665080715\\
8.78878878878879	0.163763537526766\\
8.7987987987988	0.166911384287911\\
8.80880880880881	0.169962980474458\\
8.81881881881882	0.172918194287637\\
8.82882882882883	0.175776981444282\\
8.83883883883884	0.178539391167546\\
8.84884884884885	0.181205561007328\\
8.85885885885886	0.183775725427078\\
8.86886886886887	0.186250198482067\\
8.87887887887888	0.188629396718725\\
8.88888888888889	0.190913812170001\\
8.8988988988989	0.193104035594229\\
8.90890890890891	0.195200742324129\\
8.91891891891892	0.197204691007985\\
8.92892892892893	0.199116732671134\\
8.93893893893894	0.200937795838175\\
8.94894894894895	0.202668897475727\\
8.95895895895896	0.20431113497678\\
8.96896896896897	0.205865686806292\\
8.97897897897898	0.207333810176257\\
8.98898898898899	0.208716841985638\\
8.998998998999	0.210016191847563\\
9.00900900900901	0.211233346766413\\
9.01901901901902	0.212369866412766\\
9.02902902902903	0.21342737767006\\
9.03903903903904	0.214407581719021\\
9.04904904904905	0.215312238522567\\
9.05905905905906	0.216143180782898\\
9.06906906906907	0.216902300315431\\
9.07907907907908	0.217591546122485\\
9.08908908908909	0.218212929214414\\
9.0990990990991	0.218768512765672\\
9.10910910910911	0.219260414388549\\
9.11911911911912	0.219690804760278\\
9.12912912912913	0.220061896160508\\
9.13913913913914	0.220375950934991\\
9.14914914914915	0.220635273942388\\
9.15915915915916	0.220842209687918\\
9.16916916916917	0.22099913672021\\
9.17917917917918	0.22110847122666\\
9.18918918918919	0.221172662862263\\
9.1991991991992	0.22119418645162\\
9.20920920920921	0.221175546394528\\
9.21921921921922	0.221119265770319\\
9.22922922922923	0.221027893273189\\
9.23923923923924	0.220903993746241\\
9.24924924924925	0.220750142407224\\
9.25925925925926	0.220568934920397\\
9.26926926926927	0.220362967934066\\
9.27927927927928	0.220134847490661\\
9.28928928928929	0.219887183117153\\
9.2992992992993	0.219622584658276\\
9.30930930930931	0.219343658926088\\
9.31931931931932	0.219053005642816\\
9.32932932932933	0.21875322153523\\
9.33933933933934	0.218446885576007\\
9.34934934934935	0.218136567137491\\
9.35935935935936	0.217824819898821\\
9.36936936936937	0.217514175434549\\
9.37937937937938	0.217207145405797\\
9.38938938938939	0.216906214035233\\
9.3993993993994	0.216613840105606\\
9.40940940940941	0.216332454861932\\
9.41941941941942	0.216064451147282\\
9.42942942942943	0.215812194931586\\
9.43943943943944	0.215578005581681\\
9.44944944944945	0.215364170441615\\
9.45945945945946	0.215172929003423\\
9.46946946946947	0.215006481184571\\
9.47947947947948	0.214866976507564\\
9.48948948948949	0.214756516665138\\
9.4994994994995	0.21467715099268\\
9.50950950950951	0.214630878852789\\
9.51951951951952	0.214619640738423\\
9.52952952952953	0.214645322632418\\
9.53953953953954	0.214709747946469\\
9.54954954954955	0.214814682737681\\
9.55955955955956	0.214961829354933\\
9.56956956956957	0.215152825260053\\
9.57957957957958	0.215389239858853\\
9.58958958958959	0.215672578914078\\
9.5995995995996	0.216004276296384\\
9.60960960960961	0.216385694125983\\
9.61961961961962	0.21681812660761\\
9.62962962962963	0.217302792984144\\
9.63963963963964	0.217840835395601\\
9.64964964964965	0.218433323693398\\
9.65965965965966	0.219081249673293\\
9.66966966966967	0.219785529768888\\
9.67967967967968	0.220546997224793\\
9.68968968968969	0.221366411000719\\
9.6996996996997	0.222244446034181\\
9.70970970970971	0.223181699594708\\
9.71971971971972	0.224178685391667\\
9.72972972972973	0.225235836713496\\
9.73973973973974	0.226353502484067\\
9.74974974974975	0.227531951002269\\
9.75975975975976	0.228771367569615\\
9.76976976976977	0.230071851948557\\
9.77977977977978	0.231433422038759\\
9.78978978978979	0.232856013205648\\
9.7997997997998	0.234339474983169\\
9.80980980980981	0.235883576124522\\
9.81981981981982	0.237487999421283\\
9.82982982982983	0.239152347118629\\
9.83983983983984	0.240876137373434\\
9.84984984984985	0.242658805430219\\
9.85985985985986	0.244499707300199\\
9.86986986986987	0.246398114040724\\
9.87987987987988	0.248353216828506\\
9.88988988988989	0.250364129865998\\
9.8998998998999	0.252429885605554\\
9.90990990990991	0.254549435465561\\
9.91991991991992	0.256721657095517\\
9.92992992992993	0.258945346638306\\
9.93993993993994	0.261219229218254\\
9.94994994994995	0.26354196027691\\
9.95995995995996	0.265912100236628\\
9.96996996996997	0.268328170659049\\
9.97997997997998	0.270788581596844\\
9.98998998998999	0.27329170645962\\
10	0.27583583975148\\
};
\addlegendentry{samples};


\addplot[area legend,solid,fill=mycolor4,opacity=3.000000e-01,draw=none]
table[row sep=crcr] {%
x	y\\
0	-2\\
0.01001001001001	-2\\
0.02002002002002	-2\\
0.03003003003003	-2\\
0.04004004004004	-2\\
0.0500500500500501	-2\\
0.0600600600600601	-2\\
0.0700700700700701	-2\\
0.0800800800800801	-2\\
0.0900900900900901	-2\\
0.1001001001001	-2\\
0.11011011011011	-2\\
0.12012012012012	-2\\
0.13013013013013	-2\\
0.14014014014014	-2\\
0.15015015015015	-2\\
0.16016016016016	-2\\
0.17017017017017	-2\\
0.18018018018018	-2\\
0.19019019019019	-2\\
0.2002002002002	-2\\
0.21021021021021	-2\\
0.22022022022022	-2\\
0.23023023023023	-2\\
0.24024024024024	-2\\
0.25025025025025	-2\\
0.26026026026026	-2\\
0.27027027027027	-2\\
0.28028028028028	-2\\
0.29029029029029	-2\\
0.3003003003003	-2\\
0.31031031031031	-2\\
0.32032032032032	-2\\
0.33033033033033	-2\\
0.34034034034034	-2\\
0.35035035035035	-2\\
0.36036036036036	-2\\
0.37037037037037	-2\\
0.38038038038038	-2\\
0.39039039039039	-2\\
0.4004004004004	-2\\
0.41041041041041	-2\\
0.42042042042042	-2\\
0.43043043043043	-2\\
0.44044044044044	-2\\
0.45045045045045	-2\\
0.46046046046046	-2\\
0.47047047047047	-2\\
0.48048048048048	-2\\
0.49049049049049	-2\\
0.500500500500501	-2\\
0.510510510510511	-2\\
0.520520520520521	-2\\
0.530530530530531	-2\\
0.540540540540541	-2\\
0.550550550550551	-2\\
0.560560560560561	-2\\
0.570570570570571	-2\\
0.580580580580581	-2\\
0.590590590590591	-2\\
0.600600600600601	-2\\
0.610610610610611	-2\\
0.620620620620621	-2\\
0.630630630630631	-2\\
0.640640640640641	-2\\
0.650650650650651	-2\\
0.660660660660661	-2\\
0.670670670670671	-2\\
0.680680680680681	-2\\
0.690690690690691	-2\\
0.700700700700701	-2\\
0.710710710710711	-2\\
0.720720720720721	-2\\
0.730730730730731	-2\\
0.740740740740741	-2\\
0.750750750750751	-2\\
0.760760760760761	-2\\
0.770770770770771	-2\\
0.780780780780781	-2\\
0.790790790790791	-2\\
0.800800800800801	-2\\
0.810810810810811	-2\\
0.820820820820821	-2\\
0.830830830830831	-2\\
0.840840840840841	-2\\
0.850850850850851	-2\\
0.860860860860861	-2\\
0.870870870870871	-2\\
0.880880880880881	-2\\
0.890890890890891	-2\\
0.900900900900901	-2\\
0.910910910910911	-2\\
0.920920920920921	-2\\
0.930930930930931	-2\\
0.940940940940941	-2\\
0.950950950950951	-2\\
0.960960960960961	-2\\
0.970970970970971	-2\\
0.980980980980981	-2\\
0.990990990990991	-2\\
1.001001001001	-2\\
1.01101101101101	-2\\
1.02102102102102	-2\\
1.03103103103103	-2\\
1.04104104104104	-2\\
1.05105105105105	-2\\
1.06106106106106	-2\\
1.07107107107107	-2\\
1.08108108108108	-2\\
1.09109109109109	-2\\
1.1011011011011	-2\\
1.11111111111111	-2\\
1.12112112112112	-2\\
1.13113113113113	-2\\
1.14114114114114	-2\\
1.15115115115115	-2\\
1.16116116116116	-2\\
1.17117117117117	-2\\
1.18118118118118	-2\\
1.19119119119119	-2\\
1.2012012012012	-2\\
1.21121121121121	-2\\
1.22122122122122	-2\\
1.23123123123123	-2\\
1.24124124124124	-2\\
1.25125125125125	-2\\
1.26126126126126	-2\\
1.27127127127127	-2\\
1.28128128128128	-2\\
1.29129129129129	-2\\
1.3013013013013	-2\\
1.31131131131131	-2\\
1.32132132132132	-2\\
1.33133133133133	-2\\
1.34134134134134	-2\\
1.35135135135135	-2\\
1.36136136136136	-2\\
1.37137137137137	-2\\
1.38138138138138	-2\\
1.39139139139139	-2\\
1.4014014014014	-2\\
1.41141141141141	-2\\
1.42142142142142	-2\\
1.43143143143143	-2\\
1.44144144144144	-2\\
1.45145145145145	-2\\
1.46146146146146	-2\\
1.47147147147147	-2\\
1.48148148148148	-2\\
1.49149149149149	-2\\
1.5015015015015	-2\\
1.51151151151151	-2\\
1.52152152152152	-2\\
1.53153153153153	-2\\
1.54154154154154	-2\\
1.55155155155155	-2\\
1.56156156156156	-2\\
1.57157157157157	-2\\
1.58158158158158	-2\\
1.59159159159159	-2\\
1.6016016016016	-2\\
1.61161161161161	-2\\
1.62162162162162	-2\\
1.63163163163163	-2\\
1.64164164164164	-2\\
1.65165165165165	-2\\
1.66166166166166	-2\\
1.67167167167167	-2\\
1.68168168168168	-2\\
1.69169169169169	-2\\
1.7017017017017	-2\\
1.71171171171171	-2\\
1.72172172172172	-2\\
1.73173173173173	-2\\
1.74174174174174	-2\\
1.75175175175175	-2\\
1.76176176176176	-2\\
1.77177177177177	-2\\
1.78178178178178	-2\\
1.79179179179179	-2\\
1.8018018018018	-2\\
1.81181181181181	-2\\
1.82182182182182	-2\\
1.83183183183183	-2\\
1.84184184184184	-2\\
1.85185185185185	-2\\
1.86186186186186	-2\\
1.87187187187187	-2\\
1.88188188188188	-2\\
1.89189189189189	-2\\
1.9019019019019	-2\\
1.91191191191191	-2\\
1.92192192192192	-2\\
1.93193193193193	-2\\
1.94194194194194	-2\\
1.95195195195195	-2\\
1.96196196196196	-2\\
1.97197197197197	-2\\
1.98198198198198	-2\\
1.99199199199199	-2\\
2.002002002002	-2\\
2.01201201201201	-2\\
2.02202202202202	-2\\
2.03203203203203	-2\\
2.04204204204204	-2\\
2.05205205205205	-2\\
2.06206206206206	-2\\
2.07207207207207	-2\\
2.08208208208208	-2\\
2.09209209209209	-2\\
2.1021021021021	-2\\
2.11211211211211	-2\\
2.12212212212212	-2\\
2.13213213213213	-2\\
2.14214214214214	-2\\
2.15215215215215	-2\\
2.16216216216216	-2\\
2.17217217217217	-2\\
2.18218218218218	-2\\
2.19219219219219	-2\\
2.2022022022022	-2\\
2.21221221221221	-2\\
2.22222222222222	-2\\
2.23223223223223	-2\\
2.24224224224224	-2\\
2.25225225225225	-2\\
2.26226226226226	-2\\
2.27227227227227	-2\\
2.28228228228228	-2\\
2.29229229229229	-2\\
2.3023023023023	-2\\
2.31231231231231	-2\\
2.32232232232232	-2\\
2.33233233233233	-2\\
2.34234234234234	-2\\
2.35235235235235	-2\\
2.36236236236236	-2\\
2.37237237237237	-2\\
2.38238238238238	-2\\
2.39239239239239	-2\\
2.4024024024024	-2\\
2.41241241241241	-2\\
2.42242242242242	-2\\
2.43243243243243	-2\\
2.44244244244244	-2\\
2.45245245245245	-2\\
2.46246246246246	-2\\
2.47247247247247	-2\\
2.48248248248248	-2\\
2.49249249249249	-2\\
2.5025025025025	-2\\
2.51251251251251	-2\\
2.52252252252252	-2\\
2.53253253253253	-2\\
2.54254254254254	-2\\
2.55255255255255	-2\\
2.56256256256256	-2\\
2.57257257257257	-2\\
2.58258258258258	-2\\
2.59259259259259	-2\\
2.6026026026026	-2\\
2.61261261261261	-2\\
2.62262262262262	-2\\
2.63263263263263	-2\\
2.64264264264264	-2\\
2.65265265265265	-2\\
2.66266266266266	-2\\
2.67267267267267	-2\\
2.68268268268268	-2\\
2.69269269269269	-2\\
2.7027027027027	-2\\
2.71271271271271	-2\\
2.72272272272272	-2\\
2.73273273273273	-2\\
2.74274274274274	-2\\
2.75275275275275	-2\\
2.76276276276276	-2\\
2.77277277277277	-2\\
2.78278278278278	-2\\
2.79279279279279	-2\\
2.8028028028028	-2\\
2.81281281281281	-2\\
2.82282282282282	-2\\
2.83283283283283	-2\\
2.84284284284284	-2\\
2.85285285285285	-2\\
2.86286286286286	-2\\
2.87287287287287	-2\\
2.88288288288288	-2\\
2.89289289289289	-2\\
2.9029029029029	-2\\
2.91291291291291	-2\\
2.92292292292292	-2\\
2.93293293293293	-2\\
2.94294294294294	-2\\
2.95295295295295	-2\\
2.96296296296296	-2\\
2.97297297297297	-2\\
2.98298298298298	-2\\
2.99299299299299	-2\\
3.003003003003	-2\\
3.01301301301301	-2\\
3.02302302302302	-2\\
3.03303303303303	-2\\
3.04304304304304	-2\\
3.05305305305305	-2\\
3.06306306306306	-2\\
3.07307307307307	-2\\
3.08308308308308	-2\\
3.09309309309309	-2\\
3.1031031031031	-2\\
3.11311311311311	-2\\
3.12312312312312	-2\\
3.13313313313313	-2\\
3.14314314314314	-2\\
3.15315315315315	-2\\
3.16316316316316	-2\\
3.17317317317317	-2\\
3.18318318318318	-2\\
3.19319319319319	-2\\
3.2032032032032	-2\\
3.21321321321321	-2\\
3.22322322322322	-2\\
3.23323323323323	-2\\
3.24324324324324	-2\\
3.25325325325325	-2\\
3.26326326326326	-2\\
3.27327327327327	-2\\
3.28328328328328	-2\\
3.29329329329329	-2\\
3.3033033033033	-2\\
3.31331331331331	-2\\
3.32332332332332	-2\\
3.33333333333333	-2\\
3.34334334334334	-2\\
3.35335335335335	-2\\
3.36336336336336	-2\\
3.37337337337337	-2\\
3.38338338338338	-2\\
3.39339339339339	-2\\
3.4034034034034	-2\\
3.41341341341341	-2\\
3.42342342342342	-2\\
3.43343343343343	-2\\
3.44344344344344	-2\\
3.45345345345345	-2\\
3.46346346346346	-2\\
3.47347347347347	-2\\
3.48348348348348	-2\\
3.49349349349349	-2\\
3.5035035035035	-2\\
3.51351351351351	-2\\
3.52352352352352	-2\\
3.53353353353353	-2\\
3.54354354354354	-2\\
3.55355355355355	-2\\
3.56356356356356	-2\\
3.57357357357357	-2\\
3.58358358358358	-2\\
3.59359359359359	-2\\
3.6036036036036	-2\\
3.61361361361361	-2\\
3.62362362362362	-2\\
3.63363363363363	-2\\
3.64364364364364	-2\\
3.65365365365365	-2\\
3.66366366366366	-2\\
3.67367367367367	-2\\
3.68368368368368	-2\\
3.69369369369369	-2\\
3.7037037037037	-2\\
3.71371371371371	-2\\
3.72372372372372	-2\\
3.73373373373373	-2\\
3.74374374374374	-2\\
3.75375375375375	-2\\
3.76376376376376	-2\\
3.77377377377377	-2\\
3.78378378378378	-2\\
3.79379379379379	-2\\
3.8038038038038	-2\\
3.81381381381381	-2\\
3.82382382382382	-2\\
3.83383383383383	-2\\
3.84384384384384	-2\\
3.85385385385385	-2\\
3.86386386386386	-2\\
3.87387387387387	-2\\
3.88388388388388	-2\\
3.89389389389389	-2\\
3.9039039039039	-2\\
3.91391391391391	-2\\
3.92392392392392	-2\\
3.93393393393393	-2\\
3.94394394394394	-2\\
3.95395395395395	-2\\
3.96396396396396	-2\\
3.97397397397397	-2\\
3.98398398398398	-2\\
3.99399399399399	-2\\
4.004004004004	-2\\
4.01401401401401	-2\\
4.02402402402402	-2\\
4.03403403403403	-2\\
4.04404404404404	-2\\
4.05405405405405	-2\\
4.06406406406406	-2\\
4.07407407407407	-2\\
4.08408408408408	-2\\
4.09409409409409	-2\\
4.1041041041041	-2\\
4.11411411411411	-2\\
4.12412412412412	-2\\
4.13413413413413	-2\\
4.14414414414414	-2\\
4.15415415415415	-2\\
4.16416416416416	-2\\
4.17417417417417	-2\\
4.18418418418418	-2\\
4.19419419419419	-2\\
4.2042042042042	-2\\
4.21421421421421	-2\\
4.22422422422422	-2\\
4.23423423423423	-2\\
4.24424424424424	-2\\
4.25425425425425	-2\\
4.26426426426426	-2\\
4.27427427427427	-2\\
4.28428428428428	-2\\
4.29429429429429	-2\\
4.3043043043043	-2\\
4.31431431431431	-2\\
4.32432432432432	-2\\
4.33433433433433	-2\\
4.34434434434434	-2\\
4.35435435435435	-2\\
4.36436436436436	-2\\
4.37437437437437	-2\\
4.38438438438438	-2\\
4.39439439439439	-2\\
4.4044044044044	-2\\
4.41441441441441	-2\\
4.42442442442442	-2\\
4.43443443443443	-2\\
4.44444444444444	-2\\
4.45445445445445	-2\\
4.46446446446446	-2\\
4.47447447447447	-2\\
4.48448448448448	-2\\
4.49449449449449	-2\\
4.5045045045045	-2\\
4.51451451451451	-2\\
4.52452452452452	-2\\
4.53453453453453	-2\\
4.54454454454454	-2\\
4.55455455455455	-2\\
4.56456456456456	-2\\
4.57457457457457	-2\\
4.58458458458458	-2\\
4.59459459459459	-2\\
4.6046046046046	-2\\
4.61461461461461	-2\\
4.62462462462462	-2\\
4.63463463463463	-2\\
4.64464464464464	-2\\
4.65465465465465	-2\\
4.66466466466466	-2\\
4.67467467467467	-2\\
4.68468468468468	-2\\
4.69469469469469	-2\\
4.7047047047047	-2\\
4.71471471471471	-2\\
4.72472472472472	-2\\
4.73473473473473	-2\\
4.74474474474474	-2\\
4.75475475475475	-2\\
4.76476476476476	-2\\
4.77477477477477	-2\\
4.78478478478478	-2\\
4.79479479479479	-2\\
4.8048048048048	-2\\
4.81481481481481	-2\\
4.82482482482482	-2\\
4.83483483483483	-2\\
4.84484484484484	-2\\
4.85485485485485	-2\\
4.86486486486486	-2\\
4.87487487487487	-2\\
4.88488488488488	-2\\
4.89489489489489	-2\\
4.9049049049049	-2\\
4.91491491491491	-2\\
4.92492492492492	-2\\
4.93493493493493	-2\\
4.94494494494494	-2\\
4.95495495495495	-2\\
4.96496496496496	-2\\
4.97497497497497	-2\\
4.98498498498498	-2\\
4.99499499499499	-2\\
5.00500500500501	-2\\
5.01501501501502	-2\\
5.02502502502503	-2\\
5.03503503503504	-2\\
5.04504504504505	-2\\
5.05505505505506	-2\\
5.06506506506507	-2\\
5.07507507507508	-2\\
5.08508508508509	-2\\
5.0950950950951	-2\\
5.10510510510511	-2\\
5.11511511511512	-2\\
5.12512512512513	-2\\
5.13513513513514	-2\\
5.14514514514515	-2\\
5.15515515515516	-2\\
5.16516516516517	-2\\
5.17517517517518	-2\\
5.18518518518519	-2\\
5.1951951951952	-2\\
5.20520520520521	-2\\
5.21521521521522	-2\\
5.22522522522523	-2\\
5.23523523523524	-2\\
5.24524524524525	-2\\
5.25525525525526	-2\\
5.26526526526527	-2\\
5.27527527527528	-2\\
5.28528528528529	-2\\
5.2952952952953	-2\\
5.30530530530531	-2\\
5.31531531531532	-2\\
5.32532532532533	-2\\
5.33533533533534	-2\\
5.34534534534535	-2\\
5.35535535535536	-2\\
5.36536536536537	-2\\
5.37537537537538	-2\\
5.38538538538539	-2\\
5.3953953953954	-2\\
5.40540540540541	-2\\
5.41541541541542	-2\\
5.42542542542543	-2\\
5.43543543543544	-2\\
5.44544544544545	-2\\
5.45545545545546	-2\\
5.46546546546547	-2\\
5.47547547547548	-2\\
5.48548548548549	-2\\
5.4954954954955	-2\\
5.50550550550551	-2\\
5.51551551551552	-2\\
5.52552552552553	-2\\
5.53553553553554	-2\\
5.54554554554555	-2\\
5.55555555555556	-2\\
5.56556556556557	-2\\
5.57557557557558	-2\\
5.58558558558559	-2\\
5.5955955955956	-2\\
5.60560560560561	-2\\
5.61561561561562	-2\\
5.62562562562563	-2\\
5.63563563563564	-2\\
5.64564564564565	-2\\
5.65565565565566	-2\\
5.66566566566567	-2\\
5.67567567567568	-2\\
5.68568568568569	-2\\
5.6956956956957	-2\\
5.70570570570571	-2\\
5.71571571571572	-2\\
5.72572572572573	-2\\
5.73573573573574	-2\\
5.74574574574575	-2\\
5.75575575575576	-2\\
5.76576576576577	-2\\
5.77577577577578	-2\\
5.78578578578579	-2\\
5.7957957957958	-2\\
5.80580580580581	-2\\
5.81581581581582	-2\\
5.82582582582583	-2\\
5.83583583583584	-2\\
5.84584584584585	-2\\
5.85585585585586	-2\\
5.86586586586587	-2\\
5.87587587587588	-2\\
5.88588588588589	-2\\
5.8958958958959	-2\\
5.90590590590591	-2\\
5.91591591591592	-2\\
5.92592592592593	-2\\
5.93593593593594	-2\\
5.94594594594595	-2\\
5.95595595595596	-2\\
5.96596596596597	-2\\
5.97597597597598	-2\\
5.98598598598599	-2\\
5.995995995996	-2\\
6.00600600600601	-2\\
6.01601601601602	-2\\
6.02602602602603	-2\\
6.03603603603604	-2\\
6.04604604604605	-2\\
6.05605605605606	-2\\
6.06606606606607	-2\\
6.07607607607608	-2\\
6.08608608608609	-2\\
6.0960960960961	-2\\
6.10610610610611	-2\\
6.11611611611612	-2\\
6.12612612612613	-2\\
6.13613613613614	-2\\
6.14614614614615	-2\\
6.15615615615616	-2\\
6.16616616616617	-2\\
6.17617617617618	-2\\
6.18618618618619	-2\\
6.1961961961962	-2\\
6.20620620620621	-2\\
6.21621621621622	-2\\
6.22622622622623	-2\\
6.23623623623624	-2\\
6.24624624624625	-2\\
6.25625625625626	-2\\
6.26626626626627	-2\\
6.27627627627628	-2\\
6.28628628628629	-2\\
6.2962962962963	-2\\
6.30630630630631	-2\\
6.31631631631632	-2\\
6.32632632632633	-2\\
6.33633633633634	-2\\
6.34634634634635	-2\\
6.35635635635636	-2\\
6.36636636636637	-2\\
6.37637637637638	-2\\
6.38638638638639	-2\\
6.3963963963964	-2\\
6.40640640640641	-2\\
6.41641641641642	-2\\
6.42642642642643	-2\\
6.43643643643644	-2\\
6.44644644644645	-2\\
6.45645645645646	-2\\
6.46646646646647	-2\\
6.47647647647648	-2\\
6.48648648648649	-2\\
6.4964964964965	-2\\
6.50650650650651	-2\\
6.51651651651652	-2\\
6.52652652652653	-2\\
6.53653653653654	-2\\
6.54654654654655	-2\\
6.55655655655656	-2\\
6.56656656656657	-2\\
6.57657657657658	-2\\
6.58658658658659	-2\\
6.5965965965966	-2\\
6.60660660660661	-2\\
6.61661661661662	-2\\
6.62662662662663	-2\\
6.63663663663664	-2\\
6.64664664664665	-2\\
6.65665665665666	-2\\
6.66666666666667	-2\\
6.67667667667668	-2\\
6.68668668668669	-2\\
6.6966966966967	-2\\
6.70670670670671	-2\\
6.71671671671672	-2\\
6.72672672672673	-2\\
6.73673673673674	-2\\
6.74674674674675	-2\\
6.75675675675676	-2\\
6.76676676676677	-2\\
6.77677677677678	-2\\
6.78678678678679	-2\\
6.7967967967968	-2\\
6.80680680680681	-2\\
6.81681681681682	-2\\
6.82682682682683	-2\\
6.83683683683684	-2\\
6.84684684684685	-2\\
6.85685685685686	-2\\
6.86686686686687	-2\\
6.87687687687688	-2\\
6.88688688688689	-2\\
6.8968968968969	-2\\
6.90690690690691	-2\\
6.91691691691692	-2\\
6.92692692692693	-2\\
6.93693693693694	-2\\
6.94694694694695	-2\\
6.95695695695696	-2\\
6.96696696696697	-2\\
6.97697697697698	-2\\
6.98698698698699	-2\\
6.996996996997	-2\\
7.00700700700701	-2\\
7.01701701701702	-2\\
7.02702702702703	-2\\
7.03703703703704	-2\\
7.04704704704705	-2\\
7.05705705705706	-2\\
7.06706706706707	-2\\
7.07707707707708	-2\\
7.08708708708709	-2\\
7.0970970970971	-2\\
7.10710710710711	-2\\
7.11711711711712	-2\\
7.12712712712713	-2\\
7.13713713713714	-2\\
7.14714714714715	-2\\
7.15715715715716	-2\\
7.16716716716717	-2\\
7.17717717717718	-2\\
7.18718718718719	-2\\
7.1971971971972	-2\\
7.20720720720721	-2\\
7.21721721721722	-2\\
7.22722722722723	-2\\
7.23723723723724	-2\\
7.24724724724725	-2\\
7.25725725725726	-2\\
7.26726726726727	-2\\
7.27727727727728	-2\\
7.28728728728729	-2\\
7.2972972972973	-2\\
7.30730730730731	-2\\
7.31731731731732	-2\\
7.32732732732733	-2\\
7.33733733733734	-2\\
7.34734734734735	-2\\
7.35735735735736	-2\\
7.36736736736737	-2\\
7.37737737737738	-2\\
7.38738738738739	-2\\
7.3973973973974	-2\\
7.40740740740741	-2\\
7.41741741741742	-2\\
7.42742742742743	-2\\
7.43743743743744	-2\\
7.44744744744745	-2\\
7.45745745745746	-2\\
7.46746746746747	-2\\
7.47747747747748	-2\\
7.48748748748749	-2\\
7.4974974974975	-2\\
7.50750750750751	-2\\
7.51751751751752	-2\\
7.52752752752753	-2\\
7.53753753753754	-2\\
7.54754754754755	-2\\
7.55755755755756	-2\\
7.56756756756757	-2\\
7.57757757757758	-2\\
7.58758758758759	-2\\
7.5975975975976	-2\\
7.60760760760761	-2\\
7.61761761761762	-2\\
7.62762762762763	-2\\
7.63763763763764	-2\\
7.64764764764765	-2\\
7.65765765765766	-2\\
7.66766766766767	-2\\
7.67767767767768	-2\\
7.68768768768769	-2\\
7.6976976976977	-2\\
7.70770770770771	-2\\
7.71771771771772	-2\\
7.72772772772773	-2\\
7.73773773773774	-2\\
7.74774774774775	-2\\
7.75775775775776	-2\\
7.76776776776777	-2\\
7.77777777777778	-2\\
7.78778778778779	-2\\
7.7977977977978	-2\\
7.80780780780781	-2\\
7.81781781781782	-2\\
7.82782782782783	-2\\
7.83783783783784	-2\\
7.84784784784785	-2\\
7.85785785785786	-2\\
7.86786786786787	-2\\
7.87787787787788	-2\\
7.88788788788789	-2\\
7.8978978978979	-2\\
7.90790790790791	-2\\
7.91791791791792	-2\\
7.92792792792793	-2\\
7.93793793793794	-2\\
7.94794794794795	-2\\
7.95795795795796	-2\\
7.96796796796797	-2\\
7.97797797797798	-2\\
7.98798798798799	-2\\
7.997997997998	-2\\
8.00800800800801	-2\\
8.01801801801802	-2\\
8.02802802802803	-2\\
8.03803803803804	-2\\
8.04804804804805	-2\\
8.05805805805806	-2\\
8.06806806806807	-2\\
8.07807807807808	-2\\
8.08808808808809	-2\\
8.0980980980981	-2\\
8.10810810810811	-2\\
8.11811811811812	-2\\
8.12812812812813	-2\\
8.13813813813814	-2\\
8.14814814814815	-2\\
8.15815815815816	-2\\
8.16816816816817	-2\\
8.17817817817818	-2\\
8.18818818818819	-2\\
8.1981981981982	-2\\
8.20820820820821	-2\\
8.21821821821822	-2\\
8.22822822822823	-2\\
8.23823823823824	-2\\
8.24824824824825	-2\\
8.25825825825826	-2\\
8.26826826826827	-2\\
8.27827827827828	-2\\
8.28828828828829	-2\\
8.2982982982983	-2\\
8.30830830830831	-2\\
8.31831831831832	-2\\
8.32832832832833	-2\\
8.33833833833834	-2\\
8.34834834834835	-2\\
8.35835835835836	-2\\
8.36836836836837	-2\\
8.37837837837838	-2\\
8.38838838838839	-2\\
8.3983983983984	-2\\
8.40840840840841	-2\\
8.41841841841842	-2\\
8.42842842842843	-2\\
8.43843843843844	-2\\
8.44844844844845	-2\\
8.45845845845846	-2\\
8.46846846846847	-2\\
8.47847847847848	-2\\
8.48848848848849	-2\\
8.4984984984985	-2\\
8.50850850850851	-2\\
8.51851851851852	-2\\
8.52852852852853	-2\\
8.53853853853854	-2\\
8.54854854854855	-2\\
8.55855855855856	-2\\
8.56856856856857	-2\\
8.57857857857858	-2\\
8.58858858858859	-2\\
8.5985985985986	-2\\
8.60860860860861	-2\\
8.61861861861862	-2\\
8.62862862862863	-2\\
8.63863863863864	-2\\
8.64864864864865	-2\\
8.65865865865866	-2\\
8.66866866866867	-2\\
8.67867867867868	-2\\
8.68868868868869	-2\\
8.6986986986987	-2\\
8.70870870870871	-2\\
8.71871871871872	-2\\
8.72872872872873	-2\\
8.73873873873874	-2\\
8.74874874874875	-2\\
8.75875875875876	-2\\
8.76876876876877	-2\\
8.77877877877878	-2\\
8.78878878878879	-2\\
8.7987987987988	-2\\
8.80880880880881	-2\\
8.81881881881882	-2\\
8.82882882882883	-2\\
8.83883883883884	-2\\
8.84884884884885	-2\\
8.85885885885886	-2\\
8.86886886886887	-2\\
8.87887887887888	-2\\
8.88888888888889	-2\\
8.8988988988989	-2\\
8.90890890890891	-2\\
8.91891891891892	-2\\
8.92892892892893	-2\\
8.93893893893894	-2\\
8.94894894894895	-2\\
8.95895895895896	-2\\
8.96896896896897	-2\\
8.97897897897898	-2\\
8.98898898898899	-2\\
8.998998998999	-2\\
9.00900900900901	-2\\
9.01901901901902	-2\\
9.02902902902903	-2\\
9.03903903903904	-2\\
9.04904904904905	-2\\
9.05905905905906	-2\\
9.06906906906907	-2\\
9.07907907907908	-2\\
9.08908908908909	-2\\
9.0990990990991	-2\\
9.10910910910911	-2\\
9.11911911911912	-2\\
9.12912912912913	-2\\
9.13913913913914	-2\\
9.14914914914915	-2\\
9.15915915915916	-2\\
9.16916916916917	-2\\
9.17917917917918	-2\\
9.18918918918919	-2\\
9.1991991991992	-2\\
9.20920920920921	-2\\
9.21921921921922	-2\\
9.22922922922923	-2\\
9.23923923923924	-2\\
9.24924924924925	-2\\
9.25925925925926	-2\\
9.26926926926927	-2\\
9.27927927927928	-2\\
9.28928928928929	-2\\
9.2992992992993	-2\\
9.30930930930931	-2\\
9.31931931931932	-2\\
9.32932932932933	-2\\
9.33933933933934	-2\\
9.34934934934935	-2\\
9.35935935935936	-2\\
9.36936936936937	-2\\
9.37937937937938	-2\\
9.38938938938939	-2\\
9.3993993993994	-2\\
9.40940940940941	-2\\
9.41941941941942	-2\\
9.42942942942943	-2\\
9.43943943943944	-2\\
9.44944944944945	-2\\
9.45945945945946	-2\\
9.46946946946947	-2\\
9.47947947947948	-2\\
9.48948948948949	-2\\
9.4994994994995	-2\\
9.50950950950951	-2\\
9.51951951951952	-2\\
9.52952952952953	-2\\
9.53953953953954	-2\\
9.54954954954955	-2\\
9.55955955955956	-2\\
9.56956956956957	-2\\
9.57957957957958	-2\\
9.58958958958959	-2\\
9.5995995995996	-2\\
9.60960960960961	-2\\
9.61961961961962	-2\\
9.62962962962963	-2\\
9.63963963963964	-2\\
9.64964964964965	-2\\
9.65965965965966	-2\\
9.66966966966967	-2\\
9.67967967967968	-2\\
9.68968968968969	-2\\
9.6996996996997	-2\\
9.70970970970971	-2\\
9.71971971971972	-2\\
9.72972972972973	-2\\
9.73973973973974	-2\\
9.74974974974975	-2\\
9.75975975975976	-2\\
9.76976976976977	-2\\
9.77977977977978	-2\\
9.78978978978979	-2\\
9.7997997997998	-2\\
9.80980980980981	-2\\
9.81981981981982	-2\\
9.82982982982983	-2\\
9.83983983983984	-2\\
9.84984984984985	-2\\
9.85985985985986	-2\\
9.86986986986987	-2\\
9.87987987987988	-2\\
9.88988988988989	-2\\
9.8998998998999	-2\\
9.90990990990991	-2\\
9.91991991991992	-2\\
9.92992992992993	-2\\
9.93993993993994	-2\\
9.94994994994995	-2\\
9.95995995995996	-2\\
9.96996996996997	-2\\
9.97997997997998	-2\\
9.98998998998999	-2\\
10	-2\\
10	2\\
9.98998998998999	2\\
9.97997997997998	2\\
9.96996996996997	2\\
9.95995995995996	2\\
9.94994994994995	2\\
9.93993993993994	2\\
9.92992992992993	2\\
9.91991991991992	2\\
9.90990990990991	2\\
9.8998998998999	2\\
9.88988988988989	2\\
9.87987987987988	2\\
9.86986986986987	2\\
9.85985985985986	2\\
9.84984984984985	2\\
9.83983983983984	2\\
9.82982982982983	2\\
9.81981981981982	2\\
9.80980980980981	2\\
9.7997997997998	2\\
9.78978978978979	2\\
9.77977977977978	2\\
9.76976976976977	2\\
9.75975975975976	2\\
9.74974974974975	2\\
9.73973973973974	2\\
9.72972972972973	2\\
9.71971971971972	2\\
9.70970970970971	2\\
9.6996996996997	2\\
9.68968968968969	2\\
9.67967967967968	2\\
9.66966966966967	2\\
9.65965965965966	2\\
9.64964964964965	2\\
9.63963963963964	2\\
9.62962962962963	2\\
9.61961961961962	2\\
9.60960960960961	2\\
9.5995995995996	2\\
9.58958958958959	2\\
9.57957957957958	2\\
9.56956956956957	2\\
9.55955955955956	2\\
9.54954954954955	2\\
9.53953953953954	2\\
9.52952952952953	2\\
9.51951951951952	2\\
9.50950950950951	2\\
9.4994994994995	2\\
9.48948948948949	2\\
9.47947947947948	2\\
9.46946946946947	2\\
9.45945945945946	2\\
9.44944944944945	2\\
9.43943943943944	2\\
9.42942942942943	2\\
9.41941941941942	2\\
9.40940940940941	2\\
9.3993993993994	2\\
9.38938938938939	2\\
9.37937937937938	2\\
9.36936936936937	2\\
9.35935935935936	2\\
9.34934934934935	2\\
9.33933933933934	2\\
9.32932932932933	2\\
9.31931931931932	2\\
9.30930930930931	2\\
9.2992992992993	2\\
9.28928928928929	2\\
9.27927927927928	2\\
9.26926926926927	2\\
9.25925925925926	2\\
9.24924924924925	2\\
9.23923923923924	2\\
9.22922922922923	2\\
9.21921921921922	2\\
9.20920920920921	2\\
9.1991991991992	2\\
9.18918918918919	2\\
9.17917917917918	2\\
9.16916916916917	2\\
9.15915915915916	2\\
9.14914914914915	2\\
9.13913913913914	2\\
9.12912912912913	2\\
9.11911911911912	2\\
9.10910910910911	2\\
9.0990990990991	2\\
9.08908908908909	2\\
9.07907907907908	2\\
9.06906906906907	2\\
9.05905905905906	2\\
9.04904904904905	2\\
9.03903903903904	2\\
9.02902902902903	2\\
9.01901901901902	2\\
9.00900900900901	2\\
8.998998998999	2\\
8.98898898898899	2\\
8.97897897897898	2\\
8.96896896896897	2\\
8.95895895895896	2\\
8.94894894894895	2\\
8.93893893893894	2\\
8.92892892892893	2\\
8.91891891891892	2\\
8.90890890890891	2\\
8.8988988988989	2\\
8.88888888888889	2\\
8.87887887887888	2\\
8.86886886886887	2\\
8.85885885885886	2\\
8.84884884884885	2\\
8.83883883883884	2\\
8.82882882882883	2\\
8.81881881881882	2\\
8.80880880880881	2\\
8.7987987987988	2\\
8.78878878878879	2\\
8.77877877877878	2\\
8.76876876876877	2\\
8.75875875875876	2\\
8.74874874874875	2\\
8.73873873873874	2\\
8.72872872872873	2\\
8.71871871871872	2\\
8.70870870870871	2\\
8.6986986986987	2\\
8.68868868868869	2\\
8.67867867867868	2\\
8.66866866866867	2\\
8.65865865865866	2\\
8.64864864864865	2\\
8.63863863863864	2\\
8.62862862862863	2\\
8.61861861861862	2\\
8.60860860860861	2\\
8.5985985985986	2\\
8.58858858858859	2\\
8.57857857857858	2\\
8.56856856856857	2\\
8.55855855855856	2\\
8.54854854854855	2\\
8.53853853853854	2\\
8.52852852852853	2\\
8.51851851851852	2\\
8.50850850850851	2\\
8.4984984984985	2\\
8.48848848848849	2\\
8.47847847847848	2\\
8.46846846846847	2\\
8.45845845845846	2\\
8.44844844844845	2\\
8.43843843843844	2\\
8.42842842842843	2\\
8.41841841841842	2\\
8.40840840840841	2\\
8.3983983983984	2\\
8.38838838838839	2\\
8.37837837837838	2\\
8.36836836836837	2\\
8.35835835835836	2\\
8.34834834834835	2\\
8.33833833833834	2\\
8.32832832832833	2\\
8.31831831831832	2\\
8.30830830830831	2\\
8.2982982982983	2\\
8.28828828828829	2\\
8.27827827827828	2\\
8.26826826826827	2\\
8.25825825825826	2\\
8.24824824824825	2\\
8.23823823823824	2\\
8.22822822822823	2\\
8.21821821821822	2\\
8.20820820820821	2\\
8.1981981981982	2\\
8.18818818818819	2\\
8.17817817817818	2\\
8.16816816816817	2\\
8.15815815815816	2\\
8.14814814814815	2\\
8.13813813813814	2\\
8.12812812812813	2\\
8.11811811811812	2\\
8.10810810810811	2\\
8.0980980980981	2\\
8.08808808808809	2\\
8.07807807807808	2\\
8.06806806806807	2\\
8.05805805805806	2\\
8.04804804804805	2\\
8.03803803803804	2\\
8.02802802802803	2\\
8.01801801801802	2\\
8.00800800800801	2\\
7.997997997998	2\\
7.98798798798799	2\\
7.97797797797798	2\\
7.96796796796797	2\\
7.95795795795796	2\\
7.94794794794795	2\\
7.93793793793794	2\\
7.92792792792793	2\\
7.91791791791792	2\\
7.90790790790791	2\\
7.8978978978979	2\\
7.88788788788789	2\\
7.87787787787788	2\\
7.86786786786787	2\\
7.85785785785786	2\\
7.84784784784785	2\\
7.83783783783784	2\\
7.82782782782783	2\\
7.81781781781782	2\\
7.80780780780781	2\\
7.7977977977978	2\\
7.78778778778779	2\\
7.77777777777778	2\\
7.76776776776777	2\\
7.75775775775776	2\\
7.74774774774775	2\\
7.73773773773774	2\\
7.72772772772773	2\\
7.71771771771772	2\\
7.70770770770771	2\\
7.6976976976977	2\\
7.68768768768769	2\\
7.67767767767768	2\\
7.66766766766767	2\\
7.65765765765766	2\\
7.64764764764765	2\\
7.63763763763764	2\\
7.62762762762763	2\\
7.61761761761762	2\\
7.60760760760761	2\\
7.5975975975976	2\\
7.58758758758759	2\\
7.57757757757758	2\\
7.56756756756757	2\\
7.55755755755756	2\\
7.54754754754755	2\\
7.53753753753754	2\\
7.52752752752753	2\\
7.51751751751752	2\\
7.50750750750751	2\\
7.4974974974975	2\\
7.48748748748749	2\\
7.47747747747748	2\\
7.46746746746747	2\\
7.45745745745746	2\\
7.44744744744745	2\\
7.43743743743744	2\\
7.42742742742743	2\\
7.41741741741742	2\\
7.40740740740741	2\\
7.3973973973974	2\\
7.38738738738739	2\\
7.37737737737738	2\\
7.36736736736737	2\\
7.35735735735736	2\\
7.34734734734735	2\\
7.33733733733734	2\\
7.32732732732733	2\\
7.31731731731732	2\\
7.30730730730731	2\\
7.2972972972973	2\\
7.28728728728729	2\\
7.27727727727728	2\\
7.26726726726727	2\\
7.25725725725726	2\\
7.24724724724725	2\\
7.23723723723724	2\\
7.22722722722723	2\\
7.21721721721722	2\\
7.20720720720721	2\\
7.1971971971972	2\\
7.18718718718719	2\\
7.17717717717718	2\\
7.16716716716717	2\\
7.15715715715716	2\\
7.14714714714715	2\\
7.13713713713714	2\\
7.12712712712713	2\\
7.11711711711712	2\\
7.10710710710711	2\\
7.0970970970971	2\\
7.08708708708709	2\\
7.07707707707708	2\\
7.06706706706707	2\\
7.05705705705706	2\\
7.04704704704705	2\\
7.03703703703704	2\\
7.02702702702703	2\\
7.01701701701702	2\\
7.00700700700701	2\\
6.996996996997	2\\
6.98698698698699	2\\
6.97697697697698	2\\
6.96696696696697	2\\
6.95695695695696	2\\
6.94694694694695	2\\
6.93693693693694	2\\
6.92692692692693	2\\
6.91691691691692	2\\
6.90690690690691	2\\
6.8968968968969	2\\
6.88688688688689	2\\
6.87687687687688	2\\
6.86686686686687	2\\
6.85685685685686	2\\
6.84684684684685	2\\
6.83683683683684	2\\
6.82682682682683	2\\
6.81681681681682	2\\
6.80680680680681	2\\
6.7967967967968	2\\
6.78678678678679	2\\
6.77677677677678	2\\
6.76676676676677	2\\
6.75675675675676	2\\
6.74674674674675	2\\
6.73673673673674	2\\
6.72672672672673	2\\
6.71671671671672	2\\
6.70670670670671	2\\
6.6966966966967	2\\
6.68668668668669	2\\
6.67667667667668	2\\
6.66666666666667	2\\
6.65665665665666	2\\
6.64664664664665	2\\
6.63663663663664	2\\
6.62662662662663	2\\
6.61661661661662	2\\
6.60660660660661	2\\
6.5965965965966	2\\
6.58658658658659	2\\
6.57657657657658	2\\
6.56656656656657	2\\
6.55655655655656	2\\
6.54654654654655	2\\
6.53653653653654	2\\
6.52652652652653	2\\
6.51651651651652	2\\
6.50650650650651	2\\
6.4964964964965	2\\
6.48648648648649	2\\
6.47647647647648	2\\
6.46646646646647	2\\
6.45645645645646	2\\
6.44644644644645	2\\
6.43643643643644	2\\
6.42642642642643	2\\
6.41641641641642	2\\
6.40640640640641	2\\
6.3963963963964	2\\
6.38638638638639	2\\
6.37637637637638	2\\
6.36636636636637	2\\
6.35635635635636	2\\
6.34634634634635	2\\
6.33633633633634	2\\
6.32632632632633	2\\
6.31631631631632	2\\
6.30630630630631	2\\
6.2962962962963	2\\
6.28628628628629	2\\
6.27627627627628	2\\
6.26626626626627	2\\
6.25625625625626	2\\
6.24624624624625	2\\
6.23623623623624	2\\
6.22622622622623	2\\
6.21621621621622	2\\
6.20620620620621	2\\
6.1961961961962	2\\
6.18618618618619	2\\
6.17617617617618	2\\
6.16616616616617	2\\
6.15615615615616	2\\
6.14614614614615	2\\
6.13613613613614	2\\
6.12612612612613	2\\
6.11611611611612	2\\
6.10610610610611	2\\
6.0960960960961	2\\
6.08608608608609	2\\
6.07607607607608	2\\
6.06606606606607	2\\
6.05605605605606	2\\
6.04604604604605	2\\
6.03603603603604	2\\
6.02602602602603	2\\
6.01601601601602	2\\
6.00600600600601	2\\
5.995995995996	2\\
5.98598598598599	2\\
5.97597597597598	2\\
5.96596596596597	2\\
5.95595595595596	2\\
5.94594594594595	2\\
5.93593593593594	2\\
5.92592592592593	2\\
5.91591591591592	2\\
5.90590590590591	2\\
5.8958958958959	2\\
5.88588588588589	2\\
5.87587587587588	2\\
5.86586586586587	2\\
5.85585585585586	2\\
5.84584584584585	2\\
5.83583583583584	2\\
5.82582582582583	2\\
5.81581581581582	2\\
5.80580580580581	2\\
5.7957957957958	2\\
5.78578578578579	2\\
5.77577577577578	2\\
5.76576576576577	2\\
5.75575575575576	2\\
5.74574574574575	2\\
5.73573573573574	2\\
5.72572572572573	2\\
5.71571571571572	2\\
5.70570570570571	2\\
5.6956956956957	2\\
5.68568568568569	2\\
5.67567567567568	2\\
5.66566566566567	2\\
5.65565565565566	2\\
5.64564564564565	2\\
5.63563563563564	2\\
5.62562562562563	2\\
5.61561561561562	2\\
5.60560560560561	2\\
5.5955955955956	2\\
5.58558558558559	2\\
5.57557557557558	2\\
5.56556556556557	2\\
5.55555555555556	2\\
5.54554554554555	2\\
5.53553553553554	2\\
5.52552552552553	2\\
5.51551551551552	2\\
5.50550550550551	2\\
5.4954954954955	2\\
5.48548548548549	2\\
5.47547547547548	2\\
5.46546546546547	2\\
5.45545545545546	2\\
5.44544544544545	2\\
5.43543543543544	2\\
5.42542542542543	2\\
5.41541541541542	2\\
5.40540540540541	2\\
5.3953953953954	2\\
5.38538538538539	2\\
5.37537537537538	2\\
5.36536536536537	2\\
5.35535535535536	2\\
5.34534534534535	2\\
5.33533533533534	2\\
5.32532532532533	2\\
5.31531531531532	2\\
5.30530530530531	2\\
5.2952952952953	2\\
5.28528528528529	2\\
5.27527527527528	2\\
5.26526526526527	2\\
5.25525525525526	2\\
5.24524524524525	2\\
5.23523523523524	2\\
5.22522522522523	2\\
5.21521521521522	2\\
5.20520520520521	2\\
5.1951951951952	2\\
5.18518518518519	2\\
5.17517517517518	2\\
5.16516516516517	2\\
5.15515515515516	2\\
5.14514514514515	2\\
5.13513513513514	2\\
5.12512512512513	2\\
5.11511511511512	2\\
5.10510510510511	2\\
5.0950950950951	2\\
5.08508508508509	2\\
5.07507507507508	2\\
5.06506506506507	2\\
5.05505505505506	2\\
5.04504504504505	2\\
5.03503503503504	2\\
5.02502502502503	2\\
5.01501501501502	2\\
5.00500500500501	2\\
4.99499499499499	2\\
4.98498498498498	2\\
4.97497497497497	2\\
4.96496496496496	2\\
4.95495495495495	2\\
4.94494494494494	2\\
4.93493493493493	2\\
4.92492492492492	2\\
4.91491491491491	2\\
4.9049049049049	2\\
4.89489489489489	2\\
4.88488488488488	2\\
4.87487487487487	2\\
4.86486486486486	2\\
4.85485485485485	2\\
4.84484484484484	2\\
4.83483483483483	2\\
4.82482482482482	2\\
4.81481481481481	2\\
4.8048048048048	2\\
4.79479479479479	2\\
4.78478478478478	2\\
4.77477477477477	2\\
4.76476476476476	2\\
4.75475475475475	2\\
4.74474474474474	2\\
4.73473473473473	2\\
4.72472472472472	2\\
4.71471471471471	2\\
4.7047047047047	2\\
4.69469469469469	2\\
4.68468468468468	2\\
4.67467467467467	2\\
4.66466466466466	2\\
4.65465465465465	2\\
4.64464464464464	2\\
4.63463463463463	2\\
4.62462462462462	2\\
4.61461461461461	2\\
4.6046046046046	2\\
4.59459459459459	2\\
4.58458458458458	2\\
4.57457457457457	2\\
4.56456456456456	2\\
4.55455455455455	2\\
4.54454454454454	2\\
4.53453453453453	2\\
4.52452452452452	2\\
4.51451451451451	2\\
4.5045045045045	2\\
4.49449449449449	2\\
4.48448448448448	2\\
4.47447447447447	2\\
4.46446446446446	2\\
4.45445445445445	2\\
4.44444444444444	2\\
4.43443443443443	2\\
4.42442442442442	2\\
4.41441441441441	2\\
4.4044044044044	2\\
4.39439439439439	2\\
4.38438438438438	2\\
4.37437437437437	2\\
4.36436436436436	2\\
4.35435435435435	2\\
4.34434434434434	2\\
4.33433433433433	2\\
4.32432432432432	2\\
4.31431431431431	2\\
4.3043043043043	2\\
4.29429429429429	2\\
4.28428428428428	2\\
4.27427427427427	2\\
4.26426426426426	2\\
4.25425425425425	2\\
4.24424424424424	2\\
4.23423423423423	2\\
4.22422422422422	2\\
4.21421421421421	2\\
4.2042042042042	2\\
4.19419419419419	2\\
4.18418418418418	2\\
4.17417417417417	2\\
4.16416416416416	2\\
4.15415415415415	2\\
4.14414414414414	2\\
4.13413413413413	2\\
4.12412412412412	2\\
4.11411411411411	2\\
4.1041041041041	2\\
4.09409409409409	2\\
4.08408408408408	2\\
4.07407407407407	2\\
4.06406406406406	2\\
4.05405405405405	2\\
4.04404404404404	2\\
4.03403403403403	2\\
4.02402402402402	2\\
4.01401401401401	2\\
4.004004004004	2\\
3.99399399399399	2\\
3.98398398398398	2\\
3.97397397397397	2\\
3.96396396396396	2\\
3.95395395395395	2\\
3.94394394394394	2\\
3.93393393393393	2\\
3.92392392392392	2\\
3.91391391391391	2\\
3.9039039039039	2\\
3.89389389389389	2\\
3.88388388388388	2\\
3.87387387387387	2\\
3.86386386386386	2\\
3.85385385385385	2\\
3.84384384384384	2\\
3.83383383383383	2\\
3.82382382382382	2\\
3.81381381381381	2\\
3.8038038038038	2\\
3.79379379379379	2\\
3.78378378378378	2\\
3.77377377377377	2\\
3.76376376376376	2\\
3.75375375375375	2\\
3.74374374374374	2\\
3.73373373373373	2\\
3.72372372372372	2\\
3.71371371371371	2\\
3.7037037037037	2\\
3.69369369369369	2\\
3.68368368368368	2\\
3.67367367367367	2\\
3.66366366366366	2\\
3.65365365365365	2\\
3.64364364364364	2\\
3.63363363363363	2\\
3.62362362362362	2\\
3.61361361361361	2\\
3.6036036036036	2\\
3.59359359359359	2\\
3.58358358358358	2\\
3.57357357357357	2\\
3.56356356356356	2\\
3.55355355355355	2\\
3.54354354354354	2\\
3.53353353353353	2\\
3.52352352352352	2\\
3.51351351351351	2\\
3.5035035035035	2\\
3.49349349349349	2\\
3.48348348348348	2\\
3.47347347347347	2\\
3.46346346346346	2\\
3.45345345345345	2\\
3.44344344344344	2\\
3.43343343343343	2\\
3.42342342342342	2\\
3.41341341341341	2\\
3.4034034034034	2\\
3.39339339339339	2\\
3.38338338338338	2\\
3.37337337337337	2\\
3.36336336336336	2\\
3.35335335335335	2\\
3.34334334334334	2\\
3.33333333333333	2\\
3.32332332332332	2\\
3.31331331331331	2\\
3.3033033033033	2\\
3.29329329329329	2\\
3.28328328328328	2\\
3.27327327327327	2\\
3.26326326326326	2\\
3.25325325325325	2\\
3.24324324324324	2\\
3.23323323323323	2\\
3.22322322322322	2\\
3.21321321321321	2\\
3.2032032032032	2\\
3.19319319319319	2\\
3.18318318318318	2\\
3.17317317317317	2\\
3.16316316316316	2\\
3.15315315315315	2\\
3.14314314314314	2\\
3.13313313313313	2\\
3.12312312312312	2\\
3.11311311311311	2\\
3.1031031031031	2\\
3.09309309309309	2\\
3.08308308308308	2\\
3.07307307307307	2\\
3.06306306306306	2\\
3.05305305305305	2\\
3.04304304304304	2\\
3.03303303303303	2\\
3.02302302302302	2\\
3.01301301301301	2\\
3.003003003003	2\\
2.99299299299299	2\\
2.98298298298298	2\\
2.97297297297297	2\\
2.96296296296296	2\\
2.95295295295295	2\\
2.94294294294294	2\\
2.93293293293293	2\\
2.92292292292292	2\\
2.91291291291291	2\\
2.9029029029029	2\\
2.89289289289289	2\\
2.88288288288288	2\\
2.87287287287287	2\\
2.86286286286286	2\\
2.85285285285285	2\\
2.84284284284284	2\\
2.83283283283283	2\\
2.82282282282282	2\\
2.81281281281281	2\\
2.8028028028028	2\\
2.79279279279279	2\\
2.78278278278278	2\\
2.77277277277277	2\\
2.76276276276276	2\\
2.75275275275275	2\\
2.74274274274274	2\\
2.73273273273273	2\\
2.72272272272272	2\\
2.71271271271271	2\\
2.7027027027027	2\\
2.69269269269269	2\\
2.68268268268268	2\\
2.67267267267267	2\\
2.66266266266266	2\\
2.65265265265265	2\\
2.64264264264264	2\\
2.63263263263263	2\\
2.62262262262262	2\\
2.61261261261261	2\\
2.6026026026026	2\\
2.59259259259259	2\\
2.58258258258258	2\\
2.57257257257257	2\\
2.56256256256256	2\\
2.55255255255255	2\\
2.54254254254254	2\\
2.53253253253253	2\\
2.52252252252252	2\\
2.51251251251251	2\\
2.5025025025025	2\\
2.49249249249249	2\\
2.48248248248248	2\\
2.47247247247247	2\\
2.46246246246246	2\\
2.45245245245245	2\\
2.44244244244244	2\\
2.43243243243243	2\\
2.42242242242242	2\\
2.41241241241241	2\\
2.4024024024024	2\\
2.39239239239239	2\\
2.38238238238238	2\\
2.37237237237237	2\\
2.36236236236236	2\\
2.35235235235235	2\\
2.34234234234234	2\\
2.33233233233233	2\\
2.32232232232232	2\\
2.31231231231231	2\\
2.3023023023023	2\\
2.29229229229229	2\\
2.28228228228228	2\\
2.27227227227227	2\\
2.26226226226226	2\\
2.25225225225225	2\\
2.24224224224224	2\\
2.23223223223223	2\\
2.22222222222222	2\\
2.21221221221221	2\\
2.2022022022022	2\\
2.19219219219219	2\\
2.18218218218218	2\\
2.17217217217217	2\\
2.16216216216216	2\\
2.15215215215215	2\\
2.14214214214214	2\\
2.13213213213213	2\\
2.12212212212212	2\\
2.11211211211211	2\\
2.1021021021021	2\\
2.09209209209209	2\\
2.08208208208208	2\\
2.07207207207207	2\\
2.06206206206206	2\\
2.05205205205205	2\\
2.04204204204204	2\\
2.03203203203203	2\\
2.02202202202202	2\\
2.01201201201201	2\\
2.002002002002	2\\
1.99199199199199	2\\
1.98198198198198	2\\
1.97197197197197	2\\
1.96196196196196	2\\
1.95195195195195	2\\
1.94194194194194	2\\
1.93193193193193	2\\
1.92192192192192	2\\
1.91191191191191	2\\
1.9019019019019	2\\
1.89189189189189	2\\
1.88188188188188	2\\
1.87187187187187	2\\
1.86186186186186	2\\
1.85185185185185	2\\
1.84184184184184	2\\
1.83183183183183	2\\
1.82182182182182	2\\
1.81181181181181	2\\
1.8018018018018	2\\
1.79179179179179	2\\
1.78178178178178	2\\
1.77177177177177	2\\
1.76176176176176	2\\
1.75175175175175	2\\
1.74174174174174	2\\
1.73173173173173	2\\
1.72172172172172	2\\
1.71171171171171	2\\
1.7017017017017	2\\
1.69169169169169	2\\
1.68168168168168	2\\
1.67167167167167	2\\
1.66166166166166	2\\
1.65165165165165	2\\
1.64164164164164	2\\
1.63163163163163	2\\
1.62162162162162	2\\
1.61161161161161	2\\
1.6016016016016	2\\
1.59159159159159	2\\
1.58158158158158	2\\
1.57157157157157	2\\
1.56156156156156	2\\
1.55155155155155	2\\
1.54154154154154	2\\
1.53153153153153	2\\
1.52152152152152	2\\
1.51151151151151	2\\
1.5015015015015	2\\
1.49149149149149	2\\
1.48148148148148	2\\
1.47147147147147	2\\
1.46146146146146	2\\
1.45145145145145	2\\
1.44144144144144	2\\
1.43143143143143	2\\
1.42142142142142	2\\
1.41141141141141	2\\
1.4014014014014	2\\
1.39139139139139	2\\
1.38138138138138	2\\
1.37137137137137	2\\
1.36136136136136	2\\
1.35135135135135	2\\
1.34134134134134	2\\
1.33133133133133	2\\
1.32132132132132	2\\
1.31131131131131	2\\
1.3013013013013	2\\
1.29129129129129	2\\
1.28128128128128	2\\
1.27127127127127	2\\
1.26126126126126	2\\
1.25125125125125	2\\
1.24124124124124	2\\
1.23123123123123	2\\
1.22122122122122	2\\
1.21121121121121	2\\
1.2012012012012	2\\
1.19119119119119	2\\
1.18118118118118	2\\
1.17117117117117	2\\
1.16116116116116	2\\
1.15115115115115	2\\
1.14114114114114	2\\
1.13113113113113	2\\
1.12112112112112	2\\
1.11111111111111	2\\
1.1011011011011	2\\
1.09109109109109	2\\
1.08108108108108	2\\
1.07107107107107	2\\
1.06106106106106	2\\
1.05105105105105	2\\
1.04104104104104	2\\
1.03103103103103	2\\
1.02102102102102	2\\
1.01101101101101	2\\
1.001001001001	2\\
0.990990990990991	2\\
0.980980980980981	2\\
0.970970970970971	2\\
0.960960960960961	2\\
0.950950950950951	2\\
0.940940940940941	2\\
0.930930930930931	2\\
0.920920920920921	2\\
0.910910910910911	2\\
0.900900900900901	2\\
0.890890890890891	2\\
0.880880880880881	2\\
0.870870870870871	2\\
0.860860860860861	2\\
0.850850850850851	2\\
0.840840840840841	2\\
0.830830830830831	2\\
0.820820820820821	2\\
0.810810810810811	2\\
0.800800800800801	2\\
0.790790790790791	2\\
0.780780780780781	2\\
0.770770770770771	2\\
0.760760760760761	2\\
0.750750750750751	2\\
0.740740740740741	2\\
0.730730730730731	2\\
0.720720720720721	2\\
0.710710710710711	2\\
0.700700700700701	2\\
0.690690690690691	2\\
0.680680680680681	2\\
0.670670670670671	2\\
0.660660660660661	2\\
0.650650650650651	2\\
0.640640640640641	2\\
0.630630630630631	2\\
0.620620620620621	2\\
0.610610610610611	2\\
0.600600600600601	2\\
0.590590590590591	2\\
0.580580580580581	2\\
0.570570570570571	2\\
0.560560560560561	2\\
0.550550550550551	2\\
0.540540540540541	2\\
0.530530530530531	2\\
0.520520520520521	2\\
0.510510510510511	2\\
0.500500500500501	2\\
0.49049049049049	2\\
0.48048048048048	2\\
0.47047047047047	2\\
0.46046046046046	2\\
0.45045045045045	2\\
0.44044044044044	2\\
0.43043043043043	2\\
0.42042042042042	2\\
0.41041041041041	2\\
0.4004004004004	2\\
0.39039039039039	2\\
0.38038038038038	2\\
0.37037037037037	2\\
0.36036036036036	2\\
0.35035035035035	2\\
0.34034034034034	2\\
0.33033033033033	2\\
0.32032032032032	2\\
0.31031031031031	2\\
0.3003003003003	2\\
0.29029029029029	2\\
0.28028028028028	2\\
0.27027027027027	2\\
0.26026026026026	2\\
0.25025025025025	2\\
0.24024024024024	2\\
0.23023023023023	2\\
0.22022022022022	2\\
0.21021021021021	2\\
0.2002002002002	2\\
0.19019019019019	2\\
0.18018018018018	2\\
0.17017017017017	2\\
0.16016016016016	2\\
0.15015015015015	2\\
0.14014014014014	2\\
0.13013013013013	2\\
0.12012012012012	2\\
0.11011011011011	2\\
0.1001001001001	2\\
0.0900900900900901	2\\
0.0800800800800801	2\\
0.0700700700700701	2\\
0.0600600600600601	2\\
0.0500500500500501	2\\
0.04004004004004	2\\
0.03003003003003	2\\
0.02002002002002	2\\
0.01001001001001	2\\
0	2\\
}--cycle;

\addlegendentry{$\pm 2\sigma$};

\addplot [color=mycolor5,solid]
  table[row sep=crcr]{%
0	0\\
0.01001001001001	0\\
0.02002002002002	0\\
0.03003003003003	0\\
0.04004004004004	0\\
0.0500500500500501	0\\
0.0600600600600601	0\\
0.0700700700700701	0\\
0.0800800800800801	0\\
0.0900900900900901	0\\
0.1001001001001	0\\
0.11011011011011	0\\
0.12012012012012	0\\
0.13013013013013	0\\
0.14014014014014	0\\
0.15015015015015	0\\
0.16016016016016	0\\
0.17017017017017	0\\
0.18018018018018	0\\
0.19019019019019	0\\
0.2002002002002	0\\
0.21021021021021	0\\
0.22022022022022	0\\
0.23023023023023	0\\
0.24024024024024	0\\
0.25025025025025	0\\
0.26026026026026	0\\
0.27027027027027	0\\
0.28028028028028	0\\
0.29029029029029	0\\
0.3003003003003	0\\
0.31031031031031	0\\
0.32032032032032	0\\
0.33033033033033	0\\
0.34034034034034	0\\
0.35035035035035	0\\
0.36036036036036	0\\
0.37037037037037	0\\
0.38038038038038	0\\
0.39039039039039	0\\
0.4004004004004	0\\
0.41041041041041	0\\
0.42042042042042	0\\
0.43043043043043	0\\
0.44044044044044	0\\
0.45045045045045	0\\
0.46046046046046	0\\
0.47047047047047	0\\
0.48048048048048	0\\
0.49049049049049	0\\
0.500500500500501	0\\
0.510510510510511	0\\
0.520520520520521	0\\
0.530530530530531	0\\
0.540540540540541	0\\
0.550550550550551	0\\
0.560560560560561	0\\
0.570570570570571	0\\
0.580580580580581	0\\
0.590590590590591	0\\
0.600600600600601	0\\
0.610610610610611	0\\
0.620620620620621	0\\
0.630630630630631	0\\
0.640640640640641	0\\
0.650650650650651	0\\
0.660660660660661	0\\
0.670670670670671	0\\
0.680680680680681	0\\
0.690690690690691	0\\
0.700700700700701	0\\
0.710710710710711	0\\
0.720720720720721	0\\
0.730730730730731	0\\
0.740740740740741	0\\
0.750750750750751	0\\
0.760760760760761	0\\
0.770770770770771	0\\
0.780780780780781	0\\
0.790790790790791	0\\
0.800800800800801	0\\
0.810810810810811	0\\
0.820820820820821	0\\
0.830830830830831	0\\
0.840840840840841	0\\
0.850850850850851	0\\
0.860860860860861	0\\
0.870870870870871	0\\
0.880880880880881	0\\
0.890890890890891	0\\
0.900900900900901	0\\
0.910910910910911	0\\
0.920920920920921	0\\
0.930930930930931	0\\
0.940940940940941	0\\
0.950950950950951	0\\
0.960960960960961	0\\
0.970970970970971	0\\
0.980980980980981	0\\
0.990990990990991	0\\
1.001001001001	0\\
1.01101101101101	0\\
1.02102102102102	0\\
1.03103103103103	0\\
1.04104104104104	0\\
1.05105105105105	0\\
1.06106106106106	0\\
1.07107107107107	0\\
1.08108108108108	0\\
1.09109109109109	0\\
1.1011011011011	0\\
1.11111111111111	0\\
1.12112112112112	0\\
1.13113113113113	0\\
1.14114114114114	0\\
1.15115115115115	0\\
1.16116116116116	0\\
1.17117117117117	0\\
1.18118118118118	0\\
1.19119119119119	0\\
1.2012012012012	0\\
1.21121121121121	0\\
1.22122122122122	0\\
1.23123123123123	0\\
1.24124124124124	0\\
1.25125125125125	0\\
1.26126126126126	0\\
1.27127127127127	0\\
1.28128128128128	0\\
1.29129129129129	0\\
1.3013013013013	0\\
1.31131131131131	0\\
1.32132132132132	0\\
1.33133133133133	0\\
1.34134134134134	0\\
1.35135135135135	0\\
1.36136136136136	0\\
1.37137137137137	0\\
1.38138138138138	0\\
1.39139139139139	0\\
1.4014014014014	0\\
1.41141141141141	0\\
1.42142142142142	0\\
1.43143143143143	0\\
1.44144144144144	0\\
1.45145145145145	0\\
1.46146146146146	0\\
1.47147147147147	0\\
1.48148148148148	0\\
1.49149149149149	0\\
1.5015015015015	0\\
1.51151151151151	0\\
1.52152152152152	0\\
1.53153153153153	0\\
1.54154154154154	0\\
1.55155155155155	0\\
1.56156156156156	0\\
1.57157157157157	0\\
1.58158158158158	0\\
1.59159159159159	0\\
1.6016016016016	0\\
1.61161161161161	0\\
1.62162162162162	0\\
1.63163163163163	0\\
1.64164164164164	0\\
1.65165165165165	0\\
1.66166166166166	0\\
1.67167167167167	0\\
1.68168168168168	0\\
1.69169169169169	0\\
1.7017017017017	0\\
1.71171171171171	0\\
1.72172172172172	0\\
1.73173173173173	0\\
1.74174174174174	0\\
1.75175175175175	0\\
1.76176176176176	0\\
1.77177177177177	0\\
1.78178178178178	0\\
1.79179179179179	0\\
1.8018018018018	0\\
1.81181181181181	0\\
1.82182182182182	0\\
1.83183183183183	0\\
1.84184184184184	0\\
1.85185185185185	0\\
1.86186186186186	0\\
1.87187187187187	0\\
1.88188188188188	0\\
1.89189189189189	0\\
1.9019019019019	0\\
1.91191191191191	0\\
1.92192192192192	0\\
1.93193193193193	0\\
1.94194194194194	0\\
1.95195195195195	0\\
1.96196196196196	0\\
1.97197197197197	0\\
1.98198198198198	0\\
1.99199199199199	0\\
2.002002002002	0\\
2.01201201201201	0\\
2.02202202202202	0\\
2.03203203203203	0\\
2.04204204204204	0\\
2.05205205205205	0\\
2.06206206206206	0\\
2.07207207207207	0\\
2.08208208208208	0\\
2.09209209209209	0\\
2.1021021021021	0\\
2.11211211211211	0\\
2.12212212212212	0\\
2.13213213213213	0\\
2.14214214214214	0\\
2.15215215215215	0\\
2.16216216216216	0\\
2.17217217217217	0\\
2.18218218218218	0\\
2.19219219219219	0\\
2.2022022022022	0\\
2.21221221221221	0\\
2.22222222222222	0\\
2.23223223223223	0\\
2.24224224224224	0\\
2.25225225225225	0\\
2.26226226226226	0\\
2.27227227227227	0\\
2.28228228228228	0\\
2.29229229229229	0\\
2.3023023023023	0\\
2.31231231231231	0\\
2.32232232232232	0\\
2.33233233233233	0\\
2.34234234234234	0\\
2.35235235235235	0\\
2.36236236236236	0\\
2.37237237237237	0\\
2.38238238238238	0\\
2.39239239239239	0\\
2.4024024024024	0\\
2.41241241241241	0\\
2.42242242242242	0\\
2.43243243243243	0\\
2.44244244244244	0\\
2.45245245245245	0\\
2.46246246246246	0\\
2.47247247247247	0\\
2.48248248248248	0\\
2.49249249249249	0\\
2.5025025025025	0\\
2.51251251251251	0\\
2.52252252252252	0\\
2.53253253253253	0\\
2.54254254254254	0\\
2.55255255255255	0\\
2.56256256256256	0\\
2.57257257257257	0\\
2.58258258258258	0\\
2.59259259259259	0\\
2.6026026026026	0\\
2.61261261261261	0\\
2.62262262262262	0\\
2.63263263263263	0\\
2.64264264264264	0\\
2.65265265265265	0\\
2.66266266266266	0\\
2.67267267267267	0\\
2.68268268268268	0\\
2.69269269269269	0\\
2.7027027027027	0\\
2.71271271271271	0\\
2.72272272272272	0\\
2.73273273273273	0\\
2.74274274274274	0\\
2.75275275275275	0\\
2.76276276276276	0\\
2.77277277277277	0\\
2.78278278278278	0\\
2.79279279279279	0\\
2.8028028028028	0\\
2.81281281281281	0\\
2.82282282282282	0\\
2.83283283283283	0\\
2.84284284284284	0\\
2.85285285285285	0\\
2.86286286286286	0\\
2.87287287287287	0\\
2.88288288288288	0\\
2.89289289289289	0\\
2.9029029029029	0\\
2.91291291291291	0\\
2.92292292292292	0\\
2.93293293293293	0\\
2.94294294294294	0\\
2.95295295295295	0\\
2.96296296296296	0\\
2.97297297297297	0\\
2.98298298298298	0\\
2.99299299299299	0\\
3.003003003003	0\\
3.01301301301301	0\\
3.02302302302302	0\\
3.03303303303303	0\\
3.04304304304304	0\\
3.05305305305305	0\\
3.06306306306306	0\\
3.07307307307307	0\\
3.08308308308308	0\\
3.09309309309309	0\\
3.1031031031031	0\\
3.11311311311311	0\\
3.12312312312312	0\\
3.13313313313313	0\\
3.14314314314314	0\\
3.15315315315315	0\\
3.16316316316316	0\\
3.17317317317317	0\\
3.18318318318318	0\\
3.19319319319319	0\\
3.2032032032032	0\\
3.21321321321321	0\\
3.22322322322322	0\\
3.23323323323323	0\\
3.24324324324324	0\\
3.25325325325325	0\\
3.26326326326326	0\\
3.27327327327327	0\\
3.28328328328328	0\\
3.29329329329329	0\\
3.3033033033033	0\\
3.31331331331331	0\\
3.32332332332332	0\\
3.33333333333333	0\\
3.34334334334334	0\\
3.35335335335335	0\\
3.36336336336336	0\\
3.37337337337337	0\\
3.38338338338338	0\\
3.39339339339339	0\\
3.4034034034034	0\\
3.41341341341341	0\\
3.42342342342342	0\\
3.43343343343343	0\\
3.44344344344344	0\\
3.45345345345345	0\\
3.46346346346346	0\\
3.47347347347347	0\\
3.48348348348348	0\\
3.49349349349349	0\\
3.5035035035035	0\\
3.51351351351351	0\\
3.52352352352352	0\\
3.53353353353353	0\\
3.54354354354354	0\\
3.55355355355355	0\\
3.56356356356356	0\\
3.57357357357357	0\\
3.58358358358358	0\\
3.59359359359359	0\\
3.6036036036036	0\\
3.61361361361361	0\\
3.62362362362362	0\\
3.63363363363363	0\\
3.64364364364364	0\\
3.65365365365365	0\\
3.66366366366366	0\\
3.67367367367367	0\\
3.68368368368368	0\\
3.69369369369369	0\\
3.7037037037037	0\\
3.71371371371371	0\\
3.72372372372372	0\\
3.73373373373373	0\\
3.74374374374374	0\\
3.75375375375375	0\\
3.76376376376376	0\\
3.77377377377377	0\\
3.78378378378378	0\\
3.79379379379379	0\\
3.8038038038038	0\\
3.81381381381381	0\\
3.82382382382382	0\\
3.83383383383383	0\\
3.84384384384384	0\\
3.85385385385385	0\\
3.86386386386386	0\\
3.87387387387387	0\\
3.88388388388388	0\\
3.89389389389389	0\\
3.9039039039039	0\\
3.91391391391391	0\\
3.92392392392392	0\\
3.93393393393393	0\\
3.94394394394394	0\\
3.95395395395395	0\\
3.96396396396396	0\\
3.97397397397397	0\\
3.98398398398398	0\\
3.99399399399399	0\\
4.004004004004	0\\
4.01401401401401	0\\
4.02402402402402	0\\
4.03403403403403	0\\
4.04404404404404	0\\
4.05405405405405	0\\
4.06406406406406	0\\
4.07407407407407	0\\
4.08408408408408	0\\
4.09409409409409	0\\
4.1041041041041	0\\
4.11411411411411	0\\
4.12412412412412	0\\
4.13413413413413	0\\
4.14414414414414	0\\
4.15415415415415	0\\
4.16416416416416	0\\
4.17417417417417	0\\
4.18418418418418	0\\
4.19419419419419	0\\
4.2042042042042	0\\
4.21421421421421	0\\
4.22422422422422	0\\
4.23423423423423	0\\
4.24424424424424	0\\
4.25425425425425	0\\
4.26426426426426	0\\
4.27427427427427	0\\
4.28428428428428	0\\
4.29429429429429	0\\
4.3043043043043	0\\
4.31431431431431	0\\
4.32432432432432	0\\
4.33433433433433	0\\
4.34434434434434	0\\
4.35435435435435	0\\
4.36436436436436	0\\
4.37437437437437	0\\
4.38438438438438	0\\
4.39439439439439	0\\
4.4044044044044	0\\
4.41441441441441	0\\
4.42442442442442	0\\
4.43443443443443	0\\
4.44444444444444	0\\
4.45445445445445	0\\
4.46446446446446	0\\
4.47447447447447	0\\
4.48448448448448	0\\
4.49449449449449	0\\
4.5045045045045	0\\
4.51451451451451	0\\
4.52452452452452	0\\
4.53453453453453	0\\
4.54454454454454	0\\
4.55455455455455	0\\
4.56456456456456	0\\
4.57457457457457	0\\
4.58458458458458	0\\
4.59459459459459	0\\
4.6046046046046	0\\
4.61461461461461	0\\
4.62462462462462	0\\
4.63463463463463	0\\
4.64464464464464	0\\
4.65465465465465	0\\
4.66466466466466	0\\
4.67467467467467	0\\
4.68468468468468	0\\
4.69469469469469	0\\
4.7047047047047	0\\
4.71471471471471	0\\
4.72472472472472	0\\
4.73473473473473	0\\
4.74474474474474	0\\
4.75475475475475	0\\
4.76476476476476	0\\
4.77477477477477	0\\
4.78478478478478	0\\
4.79479479479479	0\\
4.8048048048048	0\\
4.81481481481481	0\\
4.82482482482482	0\\
4.83483483483483	0\\
4.84484484484484	0\\
4.85485485485485	0\\
4.86486486486486	0\\
4.87487487487487	0\\
4.88488488488488	0\\
4.89489489489489	0\\
4.9049049049049	0\\
4.91491491491491	0\\
4.92492492492492	0\\
4.93493493493493	0\\
4.94494494494494	0\\
4.95495495495495	0\\
4.96496496496496	0\\
4.97497497497497	0\\
4.98498498498498	0\\
4.99499499499499	0\\
5.00500500500501	0\\
5.01501501501502	0\\
5.02502502502503	0\\
5.03503503503504	0\\
5.04504504504505	0\\
5.05505505505506	0\\
5.06506506506507	0\\
5.07507507507508	0\\
5.08508508508509	0\\
5.0950950950951	0\\
5.10510510510511	0\\
5.11511511511512	0\\
5.12512512512513	0\\
5.13513513513514	0\\
5.14514514514515	0\\
5.15515515515516	0\\
5.16516516516517	0\\
5.17517517517518	0\\
5.18518518518519	0\\
5.1951951951952	0\\
5.20520520520521	0\\
5.21521521521522	0\\
5.22522522522523	0\\
5.23523523523524	0\\
5.24524524524525	0\\
5.25525525525526	0\\
5.26526526526527	0\\
5.27527527527528	0\\
5.28528528528529	0\\
5.2952952952953	0\\
5.30530530530531	0\\
5.31531531531532	0\\
5.32532532532533	0\\
5.33533533533534	0\\
5.34534534534535	0\\
5.35535535535536	0\\
5.36536536536537	0\\
5.37537537537538	0\\
5.38538538538539	0\\
5.3953953953954	0\\
5.40540540540541	0\\
5.41541541541542	0\\
5.42542542542543	0\\
5.43543543543544	0\\
5.44544544544545	0\\
5.45545545545546	0\\
5.46546546546547	0\\
5.47547547547548	0\\
5.48548548548549	0\\
5.4954954954955	0\\
5.50550550550551	0\\
5.51551551551552	0\\
5.52552552552553	0\\
5.53553553553554	0\\
5.54554554554555	0\\
5.55555555555556	0\\
5.56556556556557	0\\
5.57557557557558	0\\
5.58558558558559	0\\
5.5955955955956	0\\
5.60560560560561	0\\
5.61561561561562	0\\
5.62562562562563	0\\
5.63563563563564	0\\
5.64564564564565	0\\
5.65565565565566	0\\
5.66566566566567	0\\
5.67567567567568	0\\
5.68568568568569	0\\
5.6956956956957	0\\
5.70570570570571	0\\
5.71571571571572	0\\
5.72572572572573	0\\
5.73573573573574	0\\
5.74574574574575	0\\
5.75575575575576	0\\
5.76576576576577	0\\
5.77577577577578	0\\
5.78578578578579	0\\
5.7957957957958	0\\
5.80580580580581	0\\
5.81581581581582	0\\
5.82582582582583	0\\
5.83583583583584	0\\
5.84584584584585	0\\
5.85585585585586	0\\
5.86586586586587	0\\
5.87587587587588	0\\
5.88588588588589	0\\
5.8958958958959	0\\
5.90590590590591	0\\
5.91591591591592	0\\
5.92592592592593	0\\
5.93593593593594	0\\
5.94594594594595	0\\
5.95595595595596	0\\
5.96596596596597	0\\
5.97597597597598	0\\
5.98598598598599	0\\
5.995995995996	0\\
6.00600600600601	0\\
6.01601601601602	0\\
6.02602602602603	0\\
6.03603603603604	0\\
6.04604604604605	0\\
6.05605605605606	0\\
6.06606606606607	0\\
6.07607607607608	0\\
6.08608608608609	0\\
6.0960960960961	0\\
6.10610610610611	0\\
6.11611611611612	0\\
6.12612612612613	0\\
6.13613613613614	0\\
6.14614614614615	0\\
6.15615615615616	0\\
6.16616616616617	0\\
6.17617617617618	0\\
6.18618618618619	0\\
6.1961961961962	0\\
6.20620620620621	0\\
6.21621621621622	0\\
6.22622622622623	0\\
6.23623623623624	0\\
6.24624624624625	0\\
6.25625625625626	0\\
6.26626626626627	0\\
6.27627627627628	0\\
6.28628628628629	0\\
6.2962962962963	0\\
6.30630630630631	0\\
6.31631631631632	0\\
6.32632632632633	0\\
6.33633633633634	0\\
6.34634634634635	0\\
6.35635635635636	0\\
6.36636636636637	0\\
6.37637637637638	0\\
6.38638638638639	0\\
6.3963963963964	0\\
6.40640640640641	0\\
6.41641641641642	0\\
6.42642642642643	0\\
6.43643643643644	0\\
6.44644644644645	0\\
6.45645645645646	0\\
6.46646646646647	0\\
6.47647647647648	0\\
6.48648648648649	0\\
6.4964964964965	0\\
6.50650650650651	0\\
6.51651651651652	0\\
6.52652652652653	0\\
6.53653653653654	0\\
6.54654654654655	0\\
6.55655655655656	0\\
6.56656656656657	0\\
6.57657657657658	0\\
6.58658658658659	0\\
6.5965965965966	0\\
6.60660660660661	0\\
6.61661661661662	0\\
6.62662662662663	0\\
6.63663663663664	0\\
6.64664664664665	0\\
6.65665665665666	0\\
6.66666666666667	0\\
6.67667667667668	0\\
6.68668668668669	0\\
6.6966966966967	0\\
6.70670670670671	0\\
6.71671671671672	0\\
6.72672672672673	0\\
6.73673673673674	0\\
6.74674674674675	0\\
6.75675675675676	0\\
6.76676676676677	0\\
6.77677677677678	0\\
6.78678678678679	0\\
6.7967967967968	0\\
6.80680680680681	0\\
6.81681681681682	0\\
6.82682682682683	0\\
6.83683683683684	0\\
6.84684684684685	0\\
6.85685685685686	0\\
6.86686686686687	0\\
6.87687687687688	0\\
6.88688688688689	0\\
6.8968968968969	0\\
6.90690690690691	0\\
6.91691691691692	0\\
6.92692692692693	0\\
6.93693693693694	0\\
6.94694694694695	0\\
6.95695695695696	0\\
6.96696696696697	0\\
6.97697697697698	0\\
6.98698698698699	0\\
6.996996996997	0\\
7.00700700700701	0\\
7.01701701701702	0\\
7.02702702702703	0\\
7.03703703703704	0\\
7.04704704704705	0\\
7.05705705705706	0\\
7.06706706706707	0\\
7.07707707707708	0\\
7.08708708708709	0\\
7.0970970970971	0\\
7.10710710710711	0\\
7.11711711711712	0\\
7.12712712712713	0\\
7.13713713713714	0\\
7.14714714714715	0\\
7.15715715715716	0\\
7.16716716716717	0\\
7.17717717717718	0\\
7.18718718718719	0\\
7.1971971971972	0\\
7.20720720720721	0\\
7.21721721721722	0\\
7.22722722722723	0\\
7.23723723723724	0\\
7.24724724724725	0\\
7.25725725725726	0\\
7.26726726726727	0\\
7.27727727727728	0\\
7.28728728728729	0\\
7.2972972972973	0\\
7.30730730730731	0\\
7.31731731731732	0\\
7.32732732732733	0\\
7.33733733733734	0\\
7.34734734734735	0\\
7.35735735735736	0\\
7.36736736736737	0\\
7.37737737737738	0\\
7.38738738738739	0\\
7.3973973973974	0\\
7.40740740740741	0\\
7.41741741741742	0\\
7.42742742742743	0\\
7.43743743743744	0\\
7.44744744744745	0\\
7.45745745745746	0\\
7.46746746746747	0\\
7.47747747747748	0\\
7.48748748748749	0\\
7.4974974974975	0\\
7.50750750750751	0\\
7.51751751751752	0\\
7.52752752752753	0\\
7.53753753753754	0\\
7.54754754754755	0\\
7.55755755755756	0\\
7.56756756756757	0\\
7.57757757757758	0\\
7.58758758758759	0\\
7.5975975975976	0\\
7.60760760760761	0\\
7.61761761761762	0\\
7.62762762762763	0\\
7.63763763763764	0\\
7.64764764764765	0\\
7.65765765765766	0\\
7.66766766766767	0\\
7.67767767767768	0\\
7.68768768768769	0\\
7.6976976976977	0\\
7.70770770770771	0\\
7.71771771771772	0\\
7.72772772772773	0\\
7.73773773773774	0\\
7.74774774774775	0\\
7.75775775775776	0\\
7.76776776776777	0\\
7.77777777777778	0\\
7.78778778778779	0\\
7.7977977977978	0\\
7.80780780780781	0\\
7.81781781781782	0\\
7.82782782782783	0\\
7.83783783783784	0\\
7.84784784784785	0\\
7.85785785785786	0\\
7.86786786786787	0\\
7.87787787787788	0\\
7.88788788788789	0\\
7.8978978978979	0\\
7.90790790790791	0\\
7.91791791791792	0\\
7.92792792792793	0\\
7.93793793793794	0\\
7.94794794794795	0\\
7.95795795795796	0\\
7.96796796796797	0\\
7.97797797797798	0\\
7.98798798798799	0\\
7.997997997998	0\\
8.00800800800801	0\\
8.01801801801802	0\\
8.02802802802803	0\\
8.03803803803804	0\\
8.04804804804805	0\\
8.05805805805806	0\\
8.06806806806807	0\\
8.07807807807808	0\\
8.08808808808809	0\\
8.0980980980981	0\\
8.10810810810811	0\\
8.11811811811812	0\\
8.12812812812813	0\\
8.13813813813814	0\\
8.14814814814815	0\\
8.15815815815816	0\\
8.16816816816817	0\\
8.17817817817818	0\\
8.18818818818819	0\\
8.1981981981982	0\\
8.20820820820821	0\\
8.21821821821822	0\\
8.22822822822823	0\\
8.23823823823824	0\\
8.24824824824825	0\\
8.25825825825826	0\\
8.26826826826827	0\\
8.27827827827828	0\\
8.28828828828829	0\\
8.2982982982983	0\\
8.30830830830831	0\\
8.31831831831832	0\\
8.32832832832833	0\\
8.33833833833834	0\\
8.34834834834835	0\\
8.35835835835836	0\\
8.36836836836837	0\\
8.37837837837838	0\\
8.38838838838839	0\\
8.3983983983984	0\\
8.40840840840841	0\\
8.41841841841842	0\\
8.42842842842843	0\\
8.43843843843844	0\\
8.44844844844845	0\\
8.45845845845846	0\\
8.46846846846847	0\\
8.47847847847848	0\\
8.48848848848849	0\\
8.4984984984985	0\\
8.50850850850851	0\\
8.51851851851852	0\\
8.52852852852853	0\\
8.53853853853854	0\\
8.54854854854855	0\\
8.55855855855856	0\\
8.56856856856857	0\\
8.57857857857858	0\\
8.58858858858859	0\\
8.5985985985986	0\\
8.60860860860861	0\\
8.61861861861862	0\\
8.62862862862863	0\\
8.63863863863864	0\\
8.64864864864865	0\\
8.65865865865866	0\\
8.66866866866867	0\\
8.67867867867868	0\\
8.68868868868869	0\\
8.6986986986987	0\\
8.70870870870871	0\\
8.71871871871872	0\\
8.72872872872873	0\\
8.73873873873874	0\\
8.74874874874875	0\\
8.75875875875876	0\\
8.76876876876877	0\\
8.77877877877878	0\\
8.78878878878879	0\\
8.7987987987988	0\\
8.80880880880881	0\\
8.81881881881882	0\\
8.82882882882883	0\\
8.83883883883884	0\\
8.84884884884885	0\\
8.85885885885886	0\\
8.86886886886887	0\\
8.87887887887888	0\\
8.88888888888889	0\\
8.8988988988989	0\\
8.90890890890891	0\\
8.91891891891892	0\\
8.92892892892893	0\\
8.93893893893894	0\\
8.94894894894895	0\\
8.95895895895896	0\\
8.96896896896897	0\\
8.97897897897898	0\\
8.98898898898899	0\\
8.998998998999	0\\
9.00900900900901	0\\
9.01901901901902	0\\
9.02902902902903	0\\
9.03903903903904	0\\
9.04904904904905	0\\
9.05905905905906	0\\
9.06906906906907	0\\
9.07907907907908	0\\
9.08908908908909	0\\
9.0990990990991	0\\
9.10910910910911	0\\
9.11911911911912	0\\
9.12912912912913	0\\
9.13913913913914	0\\
9.14914914914915	0\\
9.15915915915916	0\\
9.16916916916917	0\\
9.17917917917918	0\\
9.18918918918919	0\\
9.1991991991992	0\\
9.20920920920921	0\\
9.21921921921922	0\\
9.22922922922923	0\\
9.23923923923924	0\\
9.24924924924925	0\\
9.25925925925926	0\\
9.26926926926927	0\\
9.27927927927928	0\\
9.28928928928929	0\\
9.2992992992993	0\\
9.30930930930931	0\\
9.31931931931932	0\\
9.32932932932933	0\\
9.33933933933934	0\\
9.34934934934935	0\\
9.35935935935936	0\\
9.36936936936937	0\\
9.37937937937938	0\\
9.38938938938939	0\\
9.3993993993994	0\\
9.40940940940941	0\\
9.41941941941942	0\\
9.42942942942943	0\\
9.43943943943944	0\\
9.44944944944945	0\\
9.45945945945946	0\\
9.46946946946947	0\\
9.47947947947948	0\\
9.48948948948949	0\\
9.4994994994995	0\\
9.50950950950951	0\\
9.51951951951952	0\\
9.52952952952953	0\\
9.53953953953954	0\\
9.54954954954955	0\\
9.55955955955956	0\\
9.56956956956957	0\\
9.57957957957958	0\\
9.58958958958959	0\\
9.5995995995996	0\\
9.60960960960961	0\\
9.61961961961962	0\\
9.62962962962963	0\\
9.63963963963964	0\\
9.64964964964965	0\\
9.65965965965966	0\\
9.66966966966967	0\\
9.67967967967968	0\\
9.68968968968969	0\\
9.6996996996997	0\\
9.70970970970971	0\\
9.71971971971972	0\\
9.72972972972973	0\\
9.73973973973974	0\\
9.74974974974975	0\\
9.75975975975976	0\\
9.76976976976977	0\\
9.77977977977978	0\\
9.78978978978979	0\\
9.7997997997998	0\\
9.80980980980981	0\\
9.81981981981982	0\\
9.82982982982983	0\\
9.83983983983984	0\\
9.84984984984985	0\\
9.85985985985986	0\\
9.86986986986987	0\\
9.87987987987988	0\\
9.88988988988989	0\\
9.8998998998999	0\\
9.90990990990991	0\\
9.91991991991992	0\\
9.92992992992993	0\\
9.93993993993994	0\\
9.94994994994995	0\\
9.95995995995996	0\\
9.96996996996997	0\\
9.97997997997998	0\\
9.98998998998999	0\\
10	0\\
};
\addlegendentry{$\mu(x)$};

\end{axis}
\end{tikzpicture}%

  % This file was created by matlab2tikz.
% Minimal pgfplots version: 1.3
%
\tikzsetnextfilename{integral_posterior}
\definecolor{mycolor1}{rgb}{0.98824,0.57255,0.44706}%
\definecolor{mycolor2}{rgb}{0.98431,0.41569,0.29020}%
\definecolor{mycolor3}{rgb}{0.93725,0.23137,0.17255}%
\definecolor{mycolor4}{rgb}{0.65098,0.80784,0.89020}%
\definecolor{mycolor5}{rgb}{0.12157,0.47059,0.70588}%
%
\begin{tikzpicture}

\begin{axis}[%
width=0.95092\figurewidth,
height=\figureheight,
at={(0\figurewidth,0\figureheight)},
scale only axis,
xmin=0,
xmax=10,
xlabel={$x$},
ymin=-3,
ymax=3,
axis x line*=bottom,
axis y line*=left,
legend style={at={(0.97,0.03)},anchor=south east,legend cell align=left,align=left,draw=white!15!black},
legend style={legend columns=-1, draw=none}, reverse legend
]
\addplot [color=mycolor1,solid,forget plot]
  table[row sep=crcr]{%
0	1.20643230516917\\
0.01001001001001	1.21972279507876\\
0.02002002002002	1.23236310076783\\
0.03003003003003	1.2445659275789\\
0.04004004004004	1.25571375373616\\
0.0500500500500501	1.26823925926391\\
0.0600600600600601	1.27990139652695\\
0.0700700700700701	1.29235195888017\\
0.0800800800800801	1.30271146294462\\
0.0900900900900901	1.31313993571175\\
0.1001001001001	1.32248957937175\\
0.11011011011011	1.33302270222101\\
0.12012012012012	1.34462617001244\\
0.13013013013013	1.35347540009337\\
0.14014014014014	1.36276571781584\\
0.15015015015015	1.3708350707077\\
0.16016016016016	1.38159059712907\\
0.17017017017017	1.38991244448442\\
0.18018018018018	1.39844070622015\\
0.19019019019019	1.40861871602976\\
0.2002002002002	1.41427235154206\\
0.21021021021021	1.42325448443791\\
0.22022022022022	1.43043186901818\\
0.23023023023023	1.43722144501879\\
0.24024024024024	1.44513571788847\\
0.25025025025025	1.45209732466089\\
0.26026026026026	1.45996529368248\\
0.27027027027027	1.46494077437075\\
0.28028028028028	1.47292661673696\\
0.29029029029029	1.47658350923209\\
0.3003003003003	1.48063375993257\\
0.31031031031031	1.48793571690163\\
0.32032032032032	1.49378570944368\\
0.33033033033033	1.49825349635653\\
0.34034034034034	1.50373254967829\\
0.35035035035035	1.50937185704879\\
0.36036036036036	1.51332974612252\\
0.37037037037037	1.5158364189609\\
0.38038038038038	1.52046107251004\\
0.39039039039039	1.52360699559582\\
0.4004004004004	1.52776423197952\\
0.41041041041041	1.52986913251809\\
0.42042042042042	1.53299588941736\\
0.43043043043043	1.53787843422475\\
0.44044044044044	1.53885005995118\\
0.45045045045045	1.53979142695803\\
0.46046046046046	1.54380898341659\\
0.47047047047047	1.54413013679041\\
0.48048048048048	1.54539101500067\\
0.49049049049049	1.54886991979995\\
0.500500500500501	1.54760154416417\\
0.510510510510511	1.54846433646409\\
0.520520520520521	1.54743765850215\\
0.530530530530531	1.54931241936635\\
0.540540540540541	1.54807930753491\\
0.550550550550551	1.54750618913736\\
0.560560560560561	1.5469606740479\\
0.570570570570571	1.54933110446169\\
0.580580580580581	1.54636384316443\\
0.590590590590591	1.54416007088553\\
0.600600600600601	1.54242380463717\\
0.610610610610611	1.54450318801108\\
0.620620620620621	1.54196899727677\\
0.630630630630631	1.53822727392399\\
0.640640640640641	1.53772274611013\\
0.650650650650651	1.53482223930671\\
0.660660660660661	1.52848991132832\\
0.670670670670671	1.52593483697279\\
0.680680680680681	1.52385034171303\\
0.690690690690691	1.51958513116154\\
0.700700700700701	1.5165363453842\\
0.710710710710711	1.51078278676088\\
0.720720720720721	1.50802025307501\\
0.730730730730731	1.50457581649668\\
0.740740740740741	1.49890640676079\\
0.750750750750751	1.49553902653079\\
0.760760760760761	1.48827745565362\\
0.770770770770771	1.48231015781905\\
0.780780780780781	1.47638833388968\\
0.790790790790791	1.47130825799697\\
0.800800800800801	1.46640791677049\\
0.810810810810811	1.46141598738907\\
0.820820820820821	1.453991011617\\
0.830830830830831	1.44615325378017\\
0.840840840840841	1.44009169667903\\
0.850850850850851	1.43393763955995\\
0.860860860860861	1.42628126832617\\
0.870870870870871	1.41677586910976\\
0.880880880880881	1.41040392124192\\
0.890890890890891	1.40087499694564\\
0.900900900900901	1.39458587313928\\
0.910910910910911	1.38674465848406\\
0.920920920920921	1.37431044395001\\
0.930930930930931	1.36854227153177\\
0.940940940940941	1.35937651897986\\
0.950950950950951	1.35038276683148\\
0.960960960960961	1.34076675771353\\
0.970970970970971	1.33115711659359\\
0.980980980980981	1.32265432105216\\
0.990990990990991	1.31191748826648\\
1.001001001001	1.2998126251529\\
1.01101101101101	1.29117600567961\\
1.02102102102102	1.28183149264451\\
1.03103103103103	1.26794882683988\\
1.04104104104104	1.25939903335611\\
1.05105105105105	1.24661921463377\\
1.06106106106106	1.23700472802839\\
1.07107107107107	1.22573534431318\\
1.08108108108108	1.21358096591505\\
1.09109109109109	1.20271062919504\\
1.1011011011011	1.19014018642104\\
1.11111111111111	1.18051202912733\\
1.12112112112112	1.16727038762924\\
1.13113113113113	1.15503055337419\\
1.14114114114114	1.1429438377542\\
1.15115115115115	1.12949605096111\\
1.16116116116116	1.11846447554631\\
1.17117117117117	1.10430040520232\\
1.18118118118118	1.09166578398525\\
1.19119119119119	1.07927082186051\\
1.2012012012012	1.06597546932225\\
1.21121121121121	1.05270829624103\\
1.22122122122122	1.04071211841971\\
1.23123123123123	1.02552701593049\\
1.24124124124124	1.0107251243676\\
1.25125125125125	0.996817440952707\\
1.26126126126126	0.983266765089126\\
1.27127127127127	0.970277657928756\\
1.28128128128128	0.956530125236912\\
1.29129129129129	0.941632574881696\\
1.3013013013013	0.926228305376641\\
1.31131131131131	0.914627445930996\\
1.32132132132132	0.899296530316205\\
1.33133133133133	0.886299341887398\\
1.34134134134134	0.870054159484569\\
1.35135135135135	0.854607416729526\\
1.36136136136136	0.840325288606986\\
1.37137137137137	0.823671872609237\\
1.38138138138138	0.813463450837536\\
1.39139139139139	0.796748887954319\\
1.4014014014014	0.780381994761984\\
1.41141141141141	0.768154772564393\\
1.42142142142142	0.752012037687173\\
1.43143143143143	0.737346067124678\\
1.44144144144144	0.72281473589267\\
1.45145145145145	0.706597326900251\\
1.46146146146146	0.689527454084419\\
1.47147147147147	0.67610369278594\\
1.48148148148148	0.662413959401859\\
1.49149149149149	0.645908096723613\\
1.5015015015015	0.632071831345277\\
1.51151151151151	0.614331970088204\\
1.52152152152152	0.599795158206251\\
1.53153153153153	0.585254310614175\\
1.54154154154154	0.571409478706399\\
1.55155155155155	0.55447014402917\\
1.56156156156156	0.539286606261481\\
1.57157157157157	0.523102293861594\\
1.58158158158158	0.506224180474015\\
1.59159159159159	0.493215462844804\\
1.6016016016016	0.477443098082091\\
1.61161161161161	0.462500813655977\\
1.62162162162162	0.448450158710466\\
1.63163163163163	0.43175461973997\\
1.64164164164164	0.418762004294925\\
1.65165165165165	0.402581584949565\\
1.66166166166166	0.388532051771219\\
1.67167167167167	0.372115330128856\\
1.68168168168168	0.356630474209792\\
1.69169169169169	0.342802004365792\\
1.7017017017017	0.328684311675309\\
1.71171171171171	0.313074421793718\\
1.72172172172172	0.29871893479333\\
1.73173173173173	0.283729570837875\\
1.74174174174174	0.267189446611415\\
1.75175175175175	0.253386559274354\\
1.76176176176176	0.240977794565941\\
1.77177177177177	0.224869188871377\\
1.78178178178178	0.212134507467077\\
1.79179179179179	0.195819657166791\\
1.8018018018018	0.181575114312542\\
1.81181181181181	0.167852500561063\\
1.82182182182182	0.153526083374521\\
1.83183183183183	0.140633395177896\\
1.84184184184184	0.126181162311623\\
1.85185185185185	0.113694158401365\\
1.86186186186186	0.0981075133628896\\
1.87187187187187	0.0837887209636222\\
1.88188188188188	0.072018633767188\\
1.89189189189189	0.057056774174395\\
1.9019019019019	0.0448000054100252\\
1.91191191191191	0.0317479482905595\\
1.92192192192192	0.0187958019768623\\
1.93193193193193	0.00444678486067884\\
1.94194194194194	-0.00880710202466861\\
1.95195195195195	-0.018653385974633\\
1.96196196196196	-0.0315335215084223\\
1.97197197197197	-0.0465048466343227\\
1.98198198198198	-0.0592482370357387\\
1.99199199199199	-0.0698150595077832\\
2.002002002002	-0.0818341830633333\\
2.01201201201201	-0.0959278751008716\\
2.02202202202202	-0.105687720236113\\
2.03203203203203	-0.118815081085766\\
2.04204204204204	-0.130773353925576\\
2.05205205205205	-0.141636408394086\\
2.06206206206206	-0.153579119295997\\
2.07207207207207	-0.165781089043291\\
2.08208208208208	-0.176329168421213\\
2.09209209209209	-0.186484129886508\\
2.1021021021021	-0.19864578666484\\
2.11211211211211	-0.209161769273729\\
2.12212212212212	-0.219490834130988\\
2.13213213213213	-0.230098913112571\\
2.14214214214214	-0.239360667198707\\
2.15215215215215	-0.251499595249895\\
2.16216216216216	-0.260625888507096\\
2.17217217217217	-0.271846404648445\\
2.18218218218218	-0.280297146772087\\
2.19219219219219	-0.287777979269046\\
2.2022022022022	-0.298734952562246\\
2.21221221221221	-0.309519588352955\\
2.22222222222222	-0.319475011377982\\
2.23223223223223	-0.32625625804213\\
2.24224224224224	-0.335311204922937\\
2.25225225225225	-0.340833932108266\\
2.26226226226226	-0.352294342032963\\
2.27227227227227	-0.360021762712958\\
2.28228228228228	-0.368786012309139\\
2.29229229229229	-0.376052994583479\\
2.3023023023023	-0.38413011919611\\
2.31231231231231	-0.390433357442609\\
2.32232232232232	-0.400378715547861\\
2.33233233233233	-0.407596932065512\\
2.34234234234234	-0.411755391905163\\
2.35235235235235	-0.420456710492256\\
2.36236236236236	-0.427236659534526\\
2.37237237237237	-0.434166928412232\\
2.38238238238238	-0.442567349455988\\
2.39239239239239	-0.446793444863174\\
2.4024024024024	-0.450555332518711\\
2.41241241241241	-0.458246359551668\\
2.42242242242242	-0.464409597806763\\
2.43243243243243	-0.468314462452165\\
2.44244244244244	-0.473787935303565\\
2.45245245245245	-0.478362078322677\\
2.46246246246246	-0.48347616346226\\
2.47247247247247	-0.490282070740245\\
2.48248248248248	-0.493395415945266\\
2.49249249249249	-0.495071878236634\\
2.5025025025025	-0.500834143574123\\
2.51251251251251	-0.505314675758496\\
2.52252252252252	-0.507955626129196\\
2.53253253253253	-0.512961923714617\\
2.54254254254254	-0.516387167673775\\
2.55255255255255	-0.517888301042896\\
2.56256256256256	-0.521043535784825\\
2.57257257257257	-0.524038330629142\\
2.58258258258258	-0.527416040048325\\
2.59259259259259	-0.52731275548008\\
2.6026026026026	-0.531741489418508\\
2.61261261261261	-0.532154563487134\\
2.62262262262262	-0.536511911040322\\
2.63263263263263	-0.536806979655008\\
2.64264264264264	-0.536809993124452\\
2.65265265265265	-0.536897306102758\\
2.66266266266266	-0.538504720557538\\
2.67267267267267	-0.53926386127864\\
2.68268268268268	-0.540468717377762\\
2.69269269269269	-0.540512722482547\\
2.7027027027027	-0.53837350642664\\
2.71271271271271	-0.540146080278743\\
2.72272272272272	-0.54158014136099\\
2.73273273273273	-0.538609216289805\\
2.74274274274274	-0.538090295463294\\
2.75275275275275	-0.536002589342652\\
2.76276276276276	-0.536026989108799\\
2.77277277277277	-0.532160014699739\\
2.78278278278278	-0.529448865106548\\
2.79279279279279	-0.52891428583917\\
2.8028028028028	-0.525616454623106\\
2.81281281281281	-0.524325407140834\\
2.82282282282282	-0.523365437050867\\
2.83283283283283	-0.519951361080332\\
2.84284284284284	-0.517285305339955\\
2.85285285285285	-0.514542262347477\\
2.86286286286286	-0.509645424453764\\
2.87287287287287	-0.505290547785744\\
2.88288288288288	-0.502503356213202\\
2.89289289289289	-0.49813599807866\\
2.9029029029029	-0.494076997687747\\
2.91291291291291	-0.489317857663623\\
2.92292292292292	-0.484040082085575\\
2.93293293293293	-0.479793779776206\\
2.94294294294294	-0.475067082639463\\
2.95295295295295	-0.469146749970439\\
2.96296296296296	-0.461524696553206\\
2.97297297297297	-0.457746403783811\\
2.98298298298298	-0.452461029279586\\
2.99299299299299	-0.445693588861792\\
3.003003003003	-0.441119908295954\\
3.01301301301301	-0.433031531065475\\
3.02302302302302	-0.426473136172844\\
3.03303303303303	-0.419397761407055\\
3.04304304304304	-0.410855766042944\\
3.05305305305305	-0.406012133010989\\
3.06306306306306	-0.394868929305907\\
3.07307307307307	-0.390008066813038\\
3.08308308308308	-0.381619322595885\\
3.09309309309309	-0.373102159696763\\
3.1031031031031	-0.364842637203041\\
3.11311311311311	-0.358495157449201\\
3.12312312312312	-0.350001412563292\\
3.13313313313313	-0.338336225399901\\
3.14314314314314	-0.329186084741459\\
3.15315315315315	-0.322755293177407\\
3.16316316316316	-0.308895179540223\\
3.17317317317317	-0.303528842548138\\
3.18318318318318	-0.29132235862758\\
3.19319319319319	-0.281322539631244\\
3.2032032032032	-0.273430900280495\\
3.21321321321321	-0.261972466399693\\
3.22322322322322	-0.251746015283388\\
3.23323323323323	-0.242803992473671\\
3.24324324324324	-0.231285708472455\\
3.25325325325325	-0.220103818759543\\
3.26326326326326	-0.210022579267115\\
3.27327327327327	-0.19953697044657\\
3.28328328328328	-0.187238536299481\\
3.29329329329329	-0.177259597091475\\
3.3033033033033	-0.164521680881797\\
3.31331331331331	-0.153908293893277\\
3.32332332332332	-0.141768620341433\\
3.33333333333333	-0.132643542556985\\
3.34334334334334	-0.118830301555146\\
3.35335335335335	-0.107457832221858\\
3.36336336336336	-0.0949741890345208\\
3.37337337337337	-0.0801335930585646\\
3.38338338338338	-0.0705040167368024\\
3.39339339339339	-0.0584340066142672\\
3.4034034034034	-0.045639576744283\\
3.41341341341341	-0.0329790746313514\\
3.42342342342342	-0.0205455590370016\\
3.43343343343343	-0.00905299511716773\\
3.44344344344344	0.00497653584945645\\
3.45345345345345	0.0182921289829415\\
3.46346346346346	0.03206147968396\\
3.47347347347347	0.0444153211401233\\
3.48348348348348	0.0598832931315496\\
3.49349349349349	0.0701394336196947\\
3.5035035035035	0.0865901170013704\\
3.51351351351351	0.097796632557262\\
3.52352352352352	0.113057537990668\\
3.53353353353353	0.12596333705844\\
3.54354354354354	0.1405817724339\\
3.55355355355355	0.15182640314763\\
3.56356356356356	0.165197116853262\\
3.57357357357357	0.179364890641575\\
3.58358358358358	0.192249102711849\\
3.59359359359359	0.210192714190836\\
3.6036036036036	0.221204693545032\\
3.61361361361361	0.236318813195358\\
3.62362362362362	0.247688054285742\\
3.63363363363363	0.262097959848201\\
3.64364364364364	0.277834865635782\\
3.65365365365365	0.290354132471225\\
3.66366366366366	0.305273210347013\\
3.67367367367367	0.31911940038072\\
3.68368368368368	0.334306024035326\\
3.69369369369369	0.346559926022103\\
3.7037037037037	0.362014144269006\\
3.71371371371371	0.375444331130144\\
3.72372372372372	0.390088078565126\\
3.73373373373373	0.405192427869307\\
3.74374374374374	0.417548196196028\\
3.75375375375375	0.433267689491667\\
3.76376376376376	0.447788427098688\\
3.77377377377377	0.461766066310058\\
3.78378378378378	0.475708108642713\\
3.79379379379379	0.489635686009009\\
3.8038038038038	0.502607926367306\\
3.81381381381381	0.517669508207118\\
3.82382382382382	0.531888717149086\\
3.83383383383383	0.545729368444362\\
3.84384384384384	0.558640820753677\\
3.85385385385385	0.573511025927154\\
3.86386386386386	0.588204695527722\\
3.87387387387387	0.603254743738113\\
3.88388388388388	0.614773475709822\\
3.89389389389389	0.632687902050511\\
3.9039039039039	0.64572494567914\\
3.91391391391391	0.656364247137309\\
3.92392392392392	0.672778222572649\\
3.93393393393393	0.683788120620521\\
3.94394394394394	0.701085861158728\\
3.95395395395395	0.714399640787918\\
3.96396396396396	0.726857055293146\\
3.97397397397397	0.740503030719322\\
3.98398398398398	0.754153329375328\\
3.99399399399399	0.767557386292989\\
4.004004004004	0.781861196933272\\
4.01401401401401	0.795911884654411\\
4.02402402402402	0.807649234198919\\
4.03403403403403	0.823473147629168\\
4.04404404404404	0.835426589376191\\
4.05405405405405	0.850293639007435\\
4.06406406406406	0.862196914943113\\
4.07407407407407	0.875978546544369\\
4.08408408408408	0.888747541600355\\
4.09409409409409	0.901990226385194\\
4.1041041041041	0.915392128262889\\
4.11411411411411	0.929306306213879\\
4.12412412412412	0.939222691957258\\
4.13413413413413	0.953904306490461\\
4.14414414414414	0.96763879840319\\
4.15415415415415	0.978649205976357\\
4.16416416416416	0.993520376106895\\
4.17417417417417	1.00732453078667\\
4.18418418418418	1.01685982257613\\
4.19419419419419	1.03120308175254\\
4.2042042042042	1.04357829109264\\
4.21421421421421	1.05573053663961\\
4.22422422422422	1.06817913703545\\
4.23423423423423	1.08008684363618\\
4.24424424424424	1.09319317930781\\
4.25425425425425	1.10771464281328\\
4.26426426426426	1.11861783081376\\
4.27427427427427	1.13021172567333\\
4.28428428428428	1.14257495780929\\
4.29429429429429	1.15282505452455\\
4.3043043043043	1.16590244527847\\
4.31431431431431	1.17919432657826\\
4.32432432432432	1.18752201531651\\
4.33433433433433	1.20072921947731\\
4.34434434434434	1.2128248028354\\
4.35435435435435	1.22422140911078\\
4.36436436436436	1.23797605179139\\
4.37437437437437	1.247648721999\\
4.38438438438438	1.25861189543224\\
4.39439439439439	1.26936460974872\\
4.4044044044044	1.28380714652047\\
4.41441441441441	1.29108296849151\\
4.42442442442442	1.30396913936719\\
4.43443443443443	1.31499839685002\\
4.44444444444444	1.32667256234282\\
4.45445445445445	1.33832531580226\\
4.46446446446446	1.34734223967782\\
4.47447447447447	1.35958448591364\\
4.48448448448448	1.36912874897197\\
4.49449449449449	1.38091404947971\\
4.5045045045045	1.3895033996488\\
4.51451451451451	1.40067148575866\\
4.52452452452452	1.41150437863152\\
4.53453453453453	1.42101036692538\\
4.54454454454454	1.43330768599522\\
4.55455455455455	1.44086498900294\\
4.56456456456456	1.45204300240698\\
4.57457457457457	1.46240254076135\\
4.58458458458458	1.47241860258723\\
4.59459459459459	1.48331714312891\\
4.6046046046046	1.49285199240411\\
4.61461461461461	1.50029868745025\\
4.62462462462462	1.5117841940845\\
4.63463463463463	1.52106069003613\\
4.64464464464464	1.53033593072619\\
4.65465465465465	1.53907998535201\\
4.66466466466466	1.54786595910462\\
4.67467467467467	1.55635119031032\\
4.68468468468468	1.5678150291725\\
4.69469469469469	1.57549105864598\\
4.7047047047047	1.58503891326369\\
4.71471471471471	1.59327092476916\\
4.72472472472472	1.6005222602755\\
4.73473473473473	1.60966010473179\\
4.74474474474474	1.61987812536134\\
4.75475475475475	1.62723617504489\\
4.76476476476476	1.63439192356087\\
4.77477477477477	1.64484373407472\\
4.78478478478478	1.65166374752043\\
4.79479479479479	1.66052938763697\\
4.8048048048048	1.66763233996164\\
4.81481481481481	1.6753471564775\\
4.82482482482482	1.68352511204201\\
4.83483483483483	1.69061973915393\\
4.84484484484484	1.69735209858873\\
4.85485485485485	1.7033334904332\\
4.86486486486486	1.71332422726207\\
4.87487487487487	1.72082227680229\\
4.88488488488488	1.72620081976026\\
4.89489489489489	1.73553382048836\\
4.9049049049049	1.74164516380152\\
4.91491491491491	1.74610579246487\\
4.92492492492492	1.75314714372528\\
4.93493493493493	1.76058152872956\\
4.94494494494494	1.76690092700864\\
4.95495495495495	1.77125284993518\\
4.96496496496496	1.77920524646867\\
4.97497497497497	1.78452240378985\\
4.98498498498498	1.78821555335538\\
4.99499499499499	1.79443135659965\\
5.00500500500501	1.80122533841881\\
5.01501501501502	1.80700461641363\\
5.02502502502503	1.81094151157052\\
5.03503503503504	1.81839160157631\\
5.04504504504505	1.82230025634382\\
5.05505505505506	1.82742663245264\\
5.06506506506507	1.83226945410015\\
5.07507507507508	1.83610056841729\\
5.08508508508509	1.83967806442867\\
5.0950950950951	1.84554587676919\\
5.10510510510511	1.84983356986342\\
5.11511511511512	1.8524563328202\\
5.12512512512513	1.85747600609553\\
5.13513513513514	1.86023813496252\\
5.14514514514515	1.86434295086807\\
5.15515515515516	1.86831509367093\\
5.16516516516517	1.87055282454951\\
5.17517517517518	1.87430638229347\\
5.18518518518519	1.87684425510548\\
5.1951951951952	1.87911575439735\\
5.20520520520521	1.88192764825767\\
5.21521521521522	1.88430260214061\\
5.22522522522523	1.88714129814692\\
5.23523523523524	1.8894296076659\\
5.24524524524525	1.8903524715602\\
5.25525525525526	1.89496521637786\\
5.26526526526527	1.89412337811718\\
5.27527527527528	1.89718026648599\\
5.28528528528529	1.89788556122489\\
5.2952952952953	1.8998021975627\\
5.30530530530531	1.90038122626429\\
5.31531531531532	1.90144317926224\\
5.32532532532533	1.90269534757005\\
5.33533533533534	1.90259434000219\\
5.34534534534535	1.90364839266353\\
5.35535535535536	1.90533311026536\\
5.36536536536537	1.90437343354162\\
5.37537537537538	1.90494659315286\\
5.38538538538539	1.90573015344213\\
5.3953953953954	1.9031608966981\\
5.40540540540541	1.90216461153922\\
5.41541541541542	1.90284980158328\\
5.42542542542543	1.90274505910095\\
5.43543543543544	1.90253496620949\\
5.44544544544545	1.8995434980513\\
5.45545545545546	1.89906686962946\\
5.46546546546547	1.89890431497785\\
5.47547547547548	1.89590876896012\\
5.48548548548549	1.89546547475895\\
5.4954954954955	1.89419963688615\\
5.50550550550551	1.88897418701011\\
5.51551551551552	1.88935843291769\\
5.52552552552553	1.88471319656481\\
5.53553553553554	1.88499965270739\\
5.54554554554555	1.88344051047294\\
5.55555555555556	1.87888329403457\\
5.56556556556557	1.87646095249093\\
5.57557557557558	1.87478292056633\\
5.58558558558559	1.86968935960425\\
5.5955955955956	1.86732462156637\\
5.60560560560561	1.86558101486198\\
5.61561561561562	1.85757790101627\\
5.62562562562563	1.85378018774673\\
5.63563563563564	1.85076498506359\\
5.64564564564565	1.8475401478851\\
5.65565565565566	1.84342397554416\\
5.66566566566567	1.8382629445939\\
5.67567567567568	1.83199901529455\\
5.68568568568569	1.82834096922147\\
5.6956956956957	1.82264217171662\\
5.70570570570571	1.81789322368189\\
5.71571571571572	1.81079932123001\\
5.72572572572573	1.80793051775137\\
5.73573573573574	1.80224167999765\\
5.74574574574575	1.79486936545134\\
5.75575575575576	1.78820831168914\\
5.76576576576577	1.78223272838132\\
5.77577577577578	1.77783094261793\\
5.78578578578579	1.77145542636284\\
5.7957957957958	1.76173863490283\\
5.80580580580581	1.75603422664511\\
5.81581581581582	1.75064726038244\\
5.82582582582583	1.74058077330298\\
5.83583583583584	1.73610555992712\\
5.84584584584585	1.72680544534367\\
5.85585585585586	1.71853618827561\\
5.86586586586587	1.71069640680846\\
5.87587587587588	1.70247544203478\\
5.88588588588589	1.69532065098884\\
5.8958958958959	1.68887953077377\\
5.90590590590591	1.6795073640642\\
5.91591591591592	1.66913028928446\\
5.92592592592593	1.66179445446173\\
5.93593593593594	1.65337438709173\\
5.94594594594595	1.64473981425657\\
5.95595595595596	1.63412859668447\\
5.96596596596597	1.6229026290844\\
5.97597597597598	1.61686607664804\\
5.98598598598599	1.60616681361911\\
5.995995995996	1.59664745711362\\
6.00600600600601	1.58495076244701\\
6.01601601601602	1.57563770925721\\
6.02602602602603	1.56534382975381\\
6.03603603603604	1.55176092532133\\
6.04604604604605	1.54526760839699\\
6.05605605605606	1.53481762362831\\
6.06606606606607	1.52224942498936\\
6.07607607607608	1.51202038427164\\
6.08608608608609	1.50137208888713\\
6.0960960960961	1.48933434814763\\
6.10610610610611	1.4794829667585\\
6.11611611611612	1.46648753704787\\
6.12612612612613	1.45595208312503\\
6.13613613613614	1.44179271998896\\
6.14614614614615	1.43145275519437\\
6.15615615615616	1.41947958376756\\
6.16616616616617	1.40795186044853\\
6.17617617617618	1.39443146937405\\
6.18618618618619	1.382892608174\\
6.1961961961962	1.36847310692092\\
6.20620620620621	1.35429649542985\\
6.21621621621622	1.34292974945751\\
6.22622622622623	1.33287479883569\\
6.23623623623624	1.31854226667059\\
6.24624624624625	1.30576168520198\\
6.25625625625626	1.29051406261123\\
6.26626626626627	1.27558532843234\\
6.27627627627628	1.26476682729335\\
6.28628628628629	1.25039235967586\\
6.2962962962963	1.23626872744636\\
6.30630630630631	1.22278750582176\\
6.31631631631632	1.20860965745508\\
6.32632632632633	1.19563515721705\\
6.33633633633634	1.18098884371639\\
6.34634634634635	1.1679484891955\\
6.35635635635636	1.1514781185365\\
6.36636636636637	1.13727016853707\\
6.37637637637638	1.12615294133906\\
6.38638638638639	1.10940956449599\\
6.3963963963964	1.09515931671298\\
6.40640640640641	1.08095215173334\\
6.41641641641642	1.06516959785244\\
6.42642642642643	1.04905022036535\\
6.43643643643644	1.03678981580515\\
6.44644644644645	1.02098015728225\\
6.45645645645646	1.00728398834269\\
6.46646646646647	0.989688551336395\\
6.47647647647648	0.977129495445634\\
6.48648648648649	0.961449983806022\\
6.4964964964965	0.945260184960243\\
6.50650650650651	0.929858849297378\\
6.51651651651652	0.917259238966461\\
6.52652652652653	0.898807272873977\\
6.53653653653654	0.883762339735405\\
6.54654654654655	0.870309307581791\\
6.55655655655656	0.852243098260184\\
6.56656656656657	0.839238293479207\\
6.57657657657658	0.8222070717015\\
6.58658658658659	0.809787095276475\\
6.5965965965966	0.793455467874826\\
6.60660660660661	0.777445280222622\\
6.61661661661662	0.761823815040373\\
6.62662662662663	0.747119684751804\\
6.63663663663664	0.729605075761726\\
6.64664664664665	0.716221567206485\\
6.65665665665666	0.698178140665249\\
6.66666666666667	0.684337927119858\\
6.67667667667668	0.668800725985371\\
6.68668668668669	0.654357245169381\\
6.6966966966967	0.637785838986274\\
6.70670670670671	0.621727106279402\\
6.71671671671672	0.605143690917599\\
6.72672672672673	0.592224896014807\\
6.73673673673674	0.577267648952668\\
6.74674674674675	0.559655050403242\\
6.75675675675676	0.544988378934602\\
6.76676676676677	0.529909398596195\\
6.77677677677678	0.514522941502104\\
6.78678678678679	0.498203725324892\\
6.7967967967968	0.48500651836717\\
6.80680680680681	0.468410923244457\\
6.81681681681682	0.45352046318766\\
6.82682682682683	0.438732491185665\\
6.83683683683684	0.427275619917113\\
6.84684684684685	0.410330500617297\\
6.85685685685686	0.394866947719723\\
6.86686686686687	0.379203204719253\\
6.87687687687688	0.365292111835764\\
6.88688688688689	0.350726082514485\\
6.8968968968969	0.335597405902542\\
6.90690690690691	0.323187791606744\\
6.91691691691692	0.30770875095394\\
6.92692692692693	0.292917102726201\\
6.93693693693694	0.281043501504445\\
6.94694694694695	0.265312428287912\\
6.95695695695696	0.250996827499523\\
6.96696696696697	0.236425383481134\\
6.97697697697698	0.222683378861601\\
6.98698698698699	0.209556187221374\\
6.996996996997	0.195329015615887\\
7.00700700700701	0.179374422993638\\
7.01701701701702	0.168019898889636\\
7.02702702702703	0.153618447507967\\
7.03703703703704	0.143811778248233\\
7.04704704704705	0.126051161243203\\
7.05705705705706	0.113690526315343\\
7.06706706706707	0.1019977344327\\
7.07707707707708	0.0881087763172058\\
7.08708708708709	0.0775556285560191\\
7.0970970970971	0.0654109104110234\\
7.10710710710711	0.0528013261287665\\
7.11711711711712	0.038592892935283\\
7.12712712712713	0.0287286218021622\\
7.13713713713714	0.01431798839968\\
7.14714714714715	0.00397515717501906\\
7.15715715715716	-0.00763775527560495\\
7.16716716716717	-0.0226244946953711\\
7.17717717717718	-0.0317723608656074\\
7.18718718718719	-0.0432576772780499\\
7.1971971971972	-0.0561378017601994\\
7.20720720720721	-0.065711192396708\\
7.21721721721722	-0.0767237562088248\\
7.22722722722723	-0.0862754481719905\\
7.23723723723724	-0.0977552602742997\\
7.24724724724725	-0.110193092539505\\
7.25725725725726	-0.120776934705559\\
7.26726726726727	-0.131234171767909\\
7.27727727727728	-0.140363229675902\\
7.28728728728729	-0.149127078437382\\
7.2972972972973	-0.160511701481705\\
7.30730730730731	-0.170413318125952\\
7.31731731731732	-0.178306052151266\\
7.32732732732733	-0.1885337776734\\
7.33733733733734	-0.19761638813541\\
7.34734734734735	-0.208472761675276\\
7.35735735735736	-0.216186431829786\\
7.36736736736737	-0.226233168357469\\
7.37737737737738	-0.234054584385813\\
7.38738738738739	-0.242726847078678\\
7.3973973973974	-0.251550650733848\\
7.40740740740741	-0.261221620962297\\
7.41741741741742	-0.269661476763151\\
7.42742742742743	-0.276238161459253\\
7.43743743743744	-0.282163181872695\\
7.44744744744745	-0.29067020843413\\
7.45745745745746	-0.29980817580039\\
7.46746746746747	-0.307110982218119\\
7.47747747747748	-0.31129814328126\\
7.48748748748749	-0.317238732703687\\
7.4974974974975	-0.325988838779329\\
7.50750750750751	-0.331200863465414\\
7.51751751751752	-0.338502329493684\\
7.52752752752753	-0.344566913784178\\
7.53753753753754	-0.351782681776068\\
7.54754754754755	-0.357504424507772\\
7.55755755755756	-0.363862635437391\\
7.56756756756757	-0.367450949988194\\
7.57757757757758	-0.372835396083541\\
7.58758758758759	-0.377349873610343\\
7.5975975975976	-0.384279067048814\\
7.60760760760761	-0.389451395430052\\
7.61761761761762	-0.394454361062093\\
7.62762762762763	-0.398210856363597\\
7.63763763763764	-0.401208504347044\\
7.64764764764765	-0.406436340287026\\
7.65765765765766	-0.40928111996168\\
7.66766766766767	-0.415553012159167\\
7.67767767767768	-0.416150836768966\\
7.68768768768769	-0.421466494604727\\
7.6976976976977	-0.424191093367996\\
7.70770770770771	-0.427597730740309\\
7.71771771771772	-0.431022584821147\\
7.72772772772773	-0.433087575650138\\
7.73773773773774	-0.438355896996451\\
7.74774774774775	-0.438789758071568\\
7.75775775775776	-0.439839791122005\\
7.76776776776777	-0.444403451361283\\
7.77777777777778	-0.445043586694432\\
7.78778778778779	-0.44648706593858\\
7.7977977977978	-0.447454561499184\\
7.80780780780781	-0.44882940575291\\
7.81781781781782	-0.450655912549163\\
7.82782782782783	-0.451835902276856\\
7.83783783783784	-0.453672907909398\\
7.84784784784785	-0.452721642737278\\
7.85785785785786	-0.453440767489452\\
7.86786786786787	-0.454230438241646\\
7.87787787787788	-0.453637605991515\\
7.88788788788789	-0.455685550673836\\
7.8978978978979	-0.455620885031379\\
7.90790790790791	-0.454247093386337\\
7.91791791791792	-0.454090669423103\\
7.92792792792793	-0.452548879774394\\
7.93793793793794	-0.452328576650409\\
7.94794794794795	-0.451660673671772\\
7.95795795795796	-0.451592433375375\\
7.96796796796797	-0.451892014426526\\
7.97797797797798	-0.448390597005442\\
7.98798798798799	-0.448307838730963\\
7.997997997998	-0.445401193879018\\
8.00800800800801	-0.444181951930564\\
8.01801801801802	-0.440747730585375\\
8.02802802802803	-0.439863028965829\\
8.03803803803804	-0.439118704465882\\
8.04804804804805	-0.434250524644013\\
8.05805805805806	-0.432465874167719\\
8.06806806806807	-0.430030928546454\\
8.07807807807808	-0.428777831786685\\
8.08808808808809	-0.426009594323831\\
8.0980980980981	-0.423254166100764\\
8.10810810810811	-0.419224735627976\\
8.11811811811812	-0.417427188929977\\
8.12812812812813	-0.41227435116248\\
8.13813813813814	-0.410091109721621\\
8.14814814814815	-0.404985107943874\\
8.15815815815816	-0.402745660322831\\
8.16816816816817	-0.399450096916202\\
8.17817817817818	-0.396486969253714\\
8.18818818818819	-0.391044056084768\\
8.1981981981982	-0.390456660808233\\
8.20820820820821	-0.383160781666722\\
8.21821821821822	-0.379415483463427\\
8.22822822822823	-0.376258648816321\\
8.23823823823824	-0.371593831921902\\
8.24824824824825	-0.368862464237185\\
8.25825825825826	-0.363381910957448\\
8.26826826826827	-0.358767835577986\\
8.27827827827828	-0.355588653375932\\
8.28828828828829	-0.348407522674656\\
8.2982982982983	-0.344396789964921\\
8.30830830830831	-0.342823468595303\\
8.31831831831832	-0.336182861253154\\
8.32832832832833	-0.331432371219622\\
8.33833833833834	-0.32660116556045\\
8.34834834834835	-0.322097434166688\\
8.35835835835836	-0.317379632208702\\
8.36836836836837	-0.312929130758243\\
8.37837837837838	-0.308712871339404\\
8.38838838838839	-0.304837629551981\\
8.3983983983984	-0.297708725852072\\
8.40840840840841	-0.29335233542677\\
8.41841841841842	-0.290092158104863\\
8.42842842842843	-0.284275759627917\\
8.43843843843844	-0.280057106158681\\
8.44844844844845	-0.274546389492393\\
8.45845845845846	-0.269657618398687\\
8.46846846846847	-0.26796699547691\\
8.47847847847848	-0.261617443906564\\
8.48848848848849	-0.258060177625959\\
8.4984984984985	-0.251573104563367\\
8.50850850850851	-0.248049971023585\\
8.51851851851852	-0.245012792556102\\
8.52852852852853	-0.238610619502168\\
8.53853853853854	-0.233769857781968\\
8.54854854854855	-0.232187691808121\\
8.55855855855856	-0.228214259053749\\
8.56856856856857	-0.221529468982629\\
8.57857857857858	-0.216646809059862\\
8.58858858858859	-0.211035242863008\\
8.5985985985986	-0.210463677784026\\
8.60860860860861	-0.206042175992329\\
8.61861861861862	-0.201155151061319\\
8.62862862862863	-0.199310948980487\\
8.63863863863864	-0.19524536946087\\
8.64864864864865	-0.189645185188636\\
8.65865865865866	-0.185704167646406\\
8.66866866866867	-0.185673724135017\\
8.67867867867868	-0.182503010044617\\
8.68868868868869	-0.176659223925618\\
8.6986986986987	-0.175361331754772\\
8.70870870870871	-0.172844079560779\\
8.71871871871872	-0.169601159416219\\
8.72872872872873	-0.166398320084176\\
8.73873873873874	-0.164935756090268\\
8.74874874874875	-0.160676057664259\\
8.75875875875876	-0.158369308355045\\
8.76876876876877	-0.155309912807233\\
8.77877877877878	-0.154466442033354\\
8.78878878878879	-0.152044917485465\\
8.7987987987988	-0.149356085210351\\
8.80880880880881	-0.149353639134151\\
8.81881881881882	-0.146450300893019\\
8.82882882882883	-0.145945654871228\\
8.83883883883884	-0.146088540870935\\
8.84884884884885	-0.143970847660518\\
8.85885885885886	-0.141591580411949\\
8.86886886886887	-0.143341503094744\\
8.87887887887888	-0.141592881667271\\
8.88888888888889	-0.140471875206379\\
8.8988988988989	-0.139520714591658\\
8.90890890890891	-0.140749640841646\\
8.91891891891892	-0.13831955591812\\
8.92892892892893	-0.140053998725163\\
8.93893893893894	-0.138682421080186\\
8.94894894894895	-0.139767445420541\\
8.95895895895896	-0.140557151144578\\
8.96896896896897	-0.143160697923873\\
8.97897897897898	-0.140423277405336\\
8.98898898898899	-0.142727283086611\\
8.998998998999	-0.14269262271132\\
9.00900900900901	-0.144140265177014\\
9.01901901901902	-0.144837662281032\\
9.02902902902903	-0.147898043025076\\
9.03903903903904	-0.148821564793924\\
9.04904904904905	-0.150310561869623\\
9.05905905905906	-0.152350824338299\\
9.06906906906907	-0.154416326840916\\
9.07907907907908	-0.155537824034863\\
9.08908908908909	-0.156993139663393\\
9.0990990990991	-0.158013180731718\\
9.10910910910911	-0.162797253330377\\
9.11911911911912	-0.165245523479412\\
9.12912912912913	-0.170152281974108\\
9.13913913913914	-0.170105973040549\\
9.14914914914915	-0.174543597772112\\
9.15915915915916	-0.176030156914322\\
9.16916916916917	-0.181741514912679\\
9.17917917917918	-0.181589056420221\\
9.18918918918919	-0.187749787567712\\
9.1991991991992	-0.190439392751619\\
9.20920920920921	-0.193003485146315\\
9.21921921921922	-0.197107304588286\\
9.22922922922923	-0.200777433297782\\
9.23923923923924	-0.203948676746369\\
9.24924924924925	-0.209800489717741\\
9.25925925925926	-0.214637910052946\\
9.26926926926927	-0.218810206993016\\
9.27927927927928	-0.223091410324032\\
9.28928928928929	-0.228156960468161\\
9.2992992992993	-0.23300964249808\\
9.30930930930931	-0.236607278453004\\
9.31931931931932	-0.241016061559454\\
9.32932932932933	-0.247067940390783\\
9.33933933933934	-0.252368788526582\\
9.34934934934935	-0.256187415552533\\
9.35935935935936	-0.260908213274217\\
9.36936936936937	-0.267510596593122\\
9.37937937937938	-0.271524316273098\\
9.38938938938939	-0.27669410170114\\
9.3993993993994	-0.282825877399763\\
9.40940940940941	-0.28615718112717\\
9.41941941941942	-0.292179783682789\\
9.42942942942943	-0.297630492968965\\
9.43943943943944	-0.303097840907701\\
9.44944944944945	-0.308164653023438\\
9.45945945945946	-0.315226478096547\\
9.46946946946947	-0.320042050580619\\
9.47947947947948	-0.326053444635401\\
9.48948948948949	-0.332162189405456\\
9.4994994994995	-0.33753066762408\\
9.50950950950951	-0.343207673856187\\
9.51951951951952	-0.348373017933136\\
9.52952952952953	-0.353149952562511\\
9.53953953953954	-0.358819566326442\\
9.54954954954955	-0.366796346417887\\
9.55955955955956	-0.371901699770797\\
9.56956956956957	-0.376634735508057\\
9.57957957957958	-0.382630315090607\\
9.58958958958959	-0.389629018264714\\
9.5995995995996	-0.39329123789296\\
9.60960960960961	-0.400730507958621\\
9.61961961961962	-0.405306215564665\\
9.62962962962963	-0.411343186768558\\
9.63963963963964	-0.41466470097931\\
9.64964964964965	-0.422744231452005\\
9.65965965965966	-0.426580004632216\\
9.66966966966967	-0.432584825383284\\
9.67967967967968	-0.437356176514187\\
9.68968968968969	-0.441907243556466\\
9.6996996996997	-0.448583291764961\\
9.70970970970971	-0.451881075650535\\
9.71971971971972	-0.458127198718117\\
9.72972972972973	-0.462440889628246\\
9.73973973973974	-0.468762076806049\\
9.74974974974975	-0.47287241880835\\
9.75975975975976	-0.476875238024373\\
9.76976976976977	-0.480357608284187\\
9.77977977977978	-0.486170785803625\\
9.78978978978979	-0.488909395671598\\
9.7997997997998	-0.494943439650543\\
9.80980980980981	-0.49869628599801\\
9.81981981981982	-0.503888276669647\\
9.82982982982983	-0.507638136498574\\
9.83983983983984	-0.511075719963318\\
9.84984984984985	-0.513295082696111\\
9.85985985985986	-0.518040968071362\\
9.86986986986987	-0.522535569513609\\
9.87987987987988	-0.524195164013686\\
9.88988988988989	-0.528795056731085\\
9.8998998998999	-0.529583088098568\\
9.90990990990991	-0.532858691306703\\
9.91991991991992	-0.536620751630178\\
9.92992992992993	-0.539137705743934\\
9.93993993993994	-0.540572541936027\\
9.94994994994995	-0.543294539127177\\
9.95995995995996	-0.543772661700596\\
9.96996996996997	-0.547679019844261\\
9.97997997997998	-0.550036999015391\\
9.98998998998999	-0.550329717022123\\
10	-0.550831366997929\\
};
\addplot [color=mycolor2,solid,forget plot]
  table[row sep=crcr]{%
0	0.508646786766459\\
0.01001001001001	0.513695267177838\\
0.02002002002002	0.519392412848691\\
0.03003003003003	0.527057745442516\\
0.04004004004004	0.530930847076517\\
0.0500500500500501	0.537651494572885\\
0.0600600600600601	0.541323067071195\\
0.0700700700700701	0.547944065550976\\
0.0800800800800801	0.55385422620262\\
0.0900900900900901	0.558159541404143\\
0.1001001001001	0.562382526202586\\
0.11011011011011	0.569230886526841\\
0.12012012012012	0.574786022915527\\
0.13013013013013	0.581516883269085\\
0.14014014014014	0.587157858735652\\
0.15015015015015	0.591350723799698\\
0.16016016016016	0.596917887127544\\
0.17017017017017	0.603259200009887\\
0.18018018018018	0.605899466367246\\
0.19019019019019	0.612030620248602\\
0.2002002002002	0.616982557187997\\
0.21021021021021	0.621140111410628\\
0.22022022022022	0.625746353302021\\
0.23023023023023	0.629776012695521\\
0.24024024024024	0.637152618117597\\
0.25025025025025	0.641537801546707\\
0.26026026026026	0.643873016872007\\
0.27027027027027	0.651070882904525\\
0.28028028028028	0.654825043257907\\
0.29029029029029	0.661767323521246\\
0.3003003003003	0.664470423557003\\
0.31031031031031	0.668069761976025\\
0.32032032032032	0.675003247236046\\
0.33033033033033	0.679543365188691\\
0.34034034034034	0.681855582160564\\
0.35035035035035	0.689156453818696\\
0.36036036036036	0.691342923694538\\
0.37037037037037	0.697145538518496\\
0.38038038038038	0.702143570965277\\
0.39039039039039	0.703989383057522\\
0.4004004004004	0.708593055270635\\
0.41041041041041	0.713329222470412\\
0.42042042042042	0.719013046516158\\
0.43043043043043	0.722546709838768\\
0.44044044044044	0.727234079767577\\
0.45045045045045	0.733990764569275\\
0.46046046046046	0.736572220304792\\
0.47047047047047	0.73877417365436\\
0.48048048048048	0.746577685849449\\
0.49049049049049	0.750810041170601\\
0.500500500500501	0.753960840640876\\
0.510510510510511	0.758227814887875\\
0.520520520520521	0.762877164063785\\
0.530530530530531	0.768513036338332\\
0.540540540540541	0.773124518508959\\
0.550550550550551	0.776375866422388\\
0.560560560560561	0.78018893976951\\
0.570570570570571	0.785032684068527\\
0.580580580580581	0.788733959967404\\
0.590590590590591	0.793742136535039\\
0.600600600600601	0.796579565958043\\
0.610610610610611	0.801517462101476\\
0.620620620620621	0.806411563888478\\
0.630630630630631	0.810914869571325\\
0.640640640640641	0.815840676450442\\
0.650650650650651	0.819171416018402\\
0.660660660660661	0.821965104179984\\
0.670670670670671	0.824754939124815\\
0.680680680680681	0.831777465665413\\
0.690690690690691	0.834977625802477\\
0.700700700700701	0.840870504627785\\
0.710710710710711	0.843315788406725\\
0.720720720720721	0.845284332995556\\
0.730730730730731	0.850854664258683\\
0.740740740740741	0.85548652352264\\
0.750750750750751	0.85861977262865\\
0.760760760760761	0.862780812077248\\
0.770770770770771	0.867002776357956\\
0.780780780780781	0.871936721658627\\
0.790790790790791	0.875668079855584\\
0.800800800800801	0.879739996043031\\
0.810810810810811	0.883122672991609\\
0.820820820820821	0.888233862105455\\
0.830830830830831	0.891803705948466\\
0.840840840840841	0.894704391987457\\
0.850850850850851	0.900827333843728\\
0.860860860860861	0.90162863894849\\
0.870870870870871	0.905505814385624\\
0.880880880880881	0.910068755364367\\
0.890890890890891	0.912993839225454\\
0.900900900900901	0.916294329118917\\
0.910910910910911	0.919831989608862\\
0.920920920920921	0.924910285217037\\
0.930930930930931	0.92625000988716\\
0.940940940940941	0.930045073682316\\
0.950950950950951	0.933858446552961\\
0.960960960960961	0.938114947071391\\
0.970970970970971	0.94046407209019\\
0.980980980980981	0.944211382179807\\
0.990990990990991	0.946185896202189\\
1.001001001001	0.948906851608447\\
1.01101101101101	0.955262256249618\\
1.02102102102102	0.956564635782521\\
1.03103103103103	0.95910002051393\\
1.04104104104104	0.961655747237164\\
1.05105105105105	0.964340596422726\\
1.06106106106106	0.968288470002999\\
1.07107107107107	0.970349077690788\\
1.08108108108108	0.972079283866317\\
1.09109109109109	0.977489174574084\\
1.1011011011011	0.979007457475605\\
1.11111111111111	0.981956212817703\\
1.12112112112112	0.983440105912875\\
1.13113113113113	0.986142808984748\\
1.14114114114114	0.987603243483089\\
1.15115115115115	0.992191560121688\\
1.16116116116116	0.994123869863973\\
1.17117117117117	0.995900903777046\\
1.18118118118118	0.997512227946611\\
1.19119119119119	0.999041431727313\\
1.2012012012012	1.00318869074198\\
1.21121121121121	1.00472621097357\\
1.22122122122122	1.00649111403826\\
1.23123123123123	1.00666729563946\\
1.24124124124124	1.01021969268502\\
1.25125125125125	1.01133246556532\\
1.26126126126126	1.01199967940121\\
1.27127127127127	1.01481982586189\\
1.28128128128128	1.01654922248683\\
1.29129129129129	1.01803583888517\\
1.3013013013013	1.02129516490538\\
1.31131131131131	1.02146371816116\\
1.32132132132132	1.02323620022125\\
1.33133133133133	1.02372481473204\\
1.34134134134134	1.02552406004374\\
1.35135135135135	1.02755783726114\\
1.36136136136136	1.02804777462778\\
1.37137137137137	1.02830935298452\\
1.38138138138138	1.02714363511374\\
1.39139139139139	1.03268272721741\\
1.4014014014014	1.02959100094631\\
1.41141141141141	1.03241585957775\\
1.42142142142142	1.03058480593378\\
1.43143143143143	1.03359976948573\\
1.44144144144144	1.0328171062719\\
1.45145145145145	1.03214259967361\\
1.46146146146146	1.03446033437047\\
1.47147147147147	1.03515247998572\\
1.48148148148148	1.03563874987371\\
1.49149149149149	1.03632614805196\\
1.5015015015015	1.03661184544686\\
1.51151151151151	1.03675218932726\\
1.52152152152152	1.03541616708122\\
1.53153153153153	1.03607502045927\\
1.54154154154154	1.03623636226225\\
1.55155155155155	1.0336232424646\\
1.56156156156156	1.03335598154147\\
1.57157157157157	1.03405584920533\\
1.58158158158158	1.03461007911868\\
1.59159159159159	1.03433176381222\\
1.6016016016016	1.03208438854313\\
1.61161161161161	1.03378099921222\\
1.62162162162162	1.03379665310653\\
1.63163163163163	1.03367852470122\\
1.64164164164164	1.03067095026239\\
1.65165165165165	1.03002503964182\\
1.66166166166166	1.02876963101021\\
1.67167167167167	1.02922641509397\\
1.68168168168168	1.02761360428568\\
1.69169169169169	1.026777765455\\
1.7017017017017	1.02604258826506\\
1.71171171171171	1.02503728925517\\
1.72172172172172	1.02377763985276\\
1.73173173173173	1.02134497305921\\
1.74174174174174	1.02220439569514\\
1.75175175175175	1.02011213379164\\
1.76176176176176	1.01907805237412\\
1.77177177177177	1.01885694517849\\
1.78178178178178	1.01630357624633\\
1.79179179179179	1.01350932442461\\
1.8018018018018	1.01077695808246\\
1.81181181181181	1.01199079902946\\
1.82182182182182	1.00812532733736\\
1.83183183183183	1.00795876986755\\
1.84184184184184	1.00390482988571\\
1.85185185185185	1.00301111469408\\
1.86186186186186	1.00080009186201\\
1.87187187187187	0.998465935914355\\
1.88188188188188	0.999040615514525\\
1.89189189189189	0.994739262525888\\
1.9019019019019	0.993566286510825\\
1.91191191191191	0.991284393106264\\
1.92192192192192	0.989195035964302\\
1.93193193193193	0.984658541787603\\
1.94194194194194	0.984481837565677\\
1.95195195195195	0.98103326040541\\
1.96196196196196	0.979249400349169\\
1.97197197197197	0.975545568724409\\
1.98198198198198	0.971856797746253\\
1.99199199199199	0.970841839564046\\
2.002002002002	0.965407081358867\\
2.01201201201201	0.96429469240332\\
2.02202202202202	0.96095964606072\\
2.03203203203203	0.958482800693008\\
2.04204204204204	0.954506660584562\\
2.05205205205205	0.950248545339879\\
2.06206206206206	0.949023316442048\\
2.07207207207207	0.944466806975024\\
2.08208208208208	0.940294648908781\\
2.09209209209209	0.939938929978597\\
2.1021021021021	0.93398149439459\\
2.11211211211211	0.929436529579796\\
2.12212212212212	0.92907148033545\\
2.13213213213213	0.924115596659312\\
2.14214214214214	0.919143394064204\\
2.15215215215215	0.918374785141021\\
2.16216216216216	0.912645787382891\\
2.17217217217217	0.910142159010331\\
2.18218218218218	0.904611973290034\\
2.19219219219219	0.902288704640319\\
2.2022022022022	0.896467913831143\\
2.21221221221221	0.893696523747683\\
2.22222222222222	0.889942932823947\\
2.23223223223223	0.886305631660508\\
2.24224224224224	0.881692140709011\\
2.25225225225225	0.875788934028256\\
2.26226226226226	0.872491558520041\\
2.27227227227227	0.865065956882104\\
2.28228228228228	0.860123546349697\\
2.29229229229229	0.858909728068773\\
2.3023023023023	0.853199157004057\\
2.31231231231231	0.846157583232197\\
2.32232232232232	0.84275011333241\\
2.33233233233233	0.838704959905113\\
2.34234234234234	0.831186951300205\\
2.35235235235235	0.827956265114644\\
2.36236236236236	0.822364479547548\\
2.37237237237237	0.81720155851668\\
2.38238238238238	0.811991835382218\\
2.39239239239239	0.80662744956138\\
2.4024024024024	0.804622443449414\\
2.41241241241241	0.794795924682203\\
2.42242242242242	0.791157262146607\\
2.43243243243243	0.786559054722816\\
2.44244244244244	0.777727492647705\\
2.45245245245245	0.771818570561497\\
2.46246246246246	0.767545615681861\\
2.47247247247247	0.758983348917473\\
2.48248248248248	0.751988510098536\\
2.49249249249249	0.748889326670208\\
2.5025025025025	0.742934073100492\\
2.51251251251251	0.735748188924739\\
2.52252252252252	0.728702147991591\\
2.53253253253253	0.722466935947928\\
2.54254254254254	0.714870000305379\\
2.55255255255255	0.708893367803728\\
2.56256256256256	0.703499265489495\\
2.57257257257257	0.694733891396708\\
2.58258258258258	0.688810942334098\\
2.59259259259259	0.683043769661284\\
2.6026026026026	0.675998168339392\\
2.61261261261261	0.667296135617348\\
2.62262262262262	0.660565144076776\\
2.63263263263263	0.652782633711606\\
2.64264264264264	0.645585872231524\\
2.65265265265265	0.637418056549688\\
2.66266266266266	0.628359305280704\\
2.67267267267267	0.622695011440391\\
2.68268268268268	0.616506520744936\\
2.69269269269269	0.608563818874016\\
2.7027027027027	0.599634406564702\\
2.71271271271271	0.593882114949435\\
2.72272272272272	0.585773819899444\\
2.73273273273273	0.577854400150474\\
2.74274274274274	0.569247931895428\\
2.75275275275275	0.560500519643444\\
2.76276276276276	0.552532358257377\\
2.77277277277277	0.543947131723214\\
2.78278278278278	0.536985864466072\\
2.79279279279279	0.528629945040821\\
2.8028028028028	0.520182201083417\\
2.81281281281281	0.51010580297913\\
2.82282282282282	0.502167250488637\\
2.83283283283283	0.495097747961589\\
2.84284284284284	0.484527118074206\\
2.85285285285285	0.476880301422918\\
2.86286286286286	0.46910137558706\\
2.87287287287287	0.459335980663535\\
2.88288288288288	0.45050574241265\\
2.89289289289289	0.442621394478489\\
2.9029029029029	0.434273882316401\\
2.91291291291291	0.425935882059431\\
2.92292292292292	0.417532183256147\\
2.93293293293293	0.405711842753703\\
2.94294294294294	0.397192547853731\\
2.95295295295295	0.388782083769227\\
2.96296296296296	0.380635022950102\\
2.97297297297297	0.37025239952497\\
2.98298298298298	0.359229920914787\\
2.99299299299299	0.35351200612993\\
3.003003003003	0.343232759816181\\
3.01301301301301	0.333509413926713\\
3.02302302302302	0.324996144374734\\
3.03303303303303	0.315646658214706\\
3.04304304304304	0.305274066488421\\
3.05305305305305	0.295385606601987\\
3.06306306306306	0.288856600269782\\
3.07307307307307	0.276335176355591\\
3.08308308308308	0.268272211616337\\
3.09309309309309	0.260902188305473\\
3.1031031031031	0.251650472154912\\
3.11311311311311	0.240836809966854\\
3.12312312312312	0.23181484431304\\
3.13313313313313	0.222229131338504\\
3.14314314314314	0.213526251070864\\
3.15315315315315	0.205162072919222\\
3.16316316316316	0.196015128295144\\
3.17317317317317	0.185532887789807\\
3.18318318318318	0.175555898501469\\
3.19319319319319	0.167599640453455\\
3.2032032032032	0.159477317149777\\
3.21321321321321	0.149738660978908\\
3.22322322322322	0.141591333727851\\
3.23323323323323	0.131429304921208\\
3.24324324324324	0.120800003644274\\
3.25325325325325	0.111916564077191\\
3.26326326326326	0.102895846210131\\
3.27327327327327	0.0945717773374807\\
3.28328328328328	0.0859874568373716\\
3.29329329329329	0.0753506051022363\\
3.3033033033033	0.0698166882252577\\
3.31331331331331	0.0593686345132658\\
3.32332332332332	0.0507814591507864\\
3.33333333333333	0.0415954518508008\\
3.34334334334334	0.0336037674413847\\
3.35335335335335	0.0247070061402812\\
3.36336336336336	0.0151762310192232\\
3.37337337337337	0.00853052005840216\\
3.38338338338338	-0.000747711897653436\\
3.39339339339339	-0.00774520139235824\\
3.4034034034034	-0.0159945344039394\\
3.41341341341341	-0.0264426118492579\\
3.42342342342342	-0.0338856172145181\\
3.43343343343343	-0.0406957578689809\\
3.44344344344344	-0.049084208614543\\
3.45345345345345	-0.0562230469905999\\
3.46346346346346	-0.0645244850825447\\
3.47347347347347	-0.0720384627905301\\
3.48348348348348	-0.0786400790487152\\
3.49349349349349	-0.0866387572172566\\
3.5035035035035	-0.0932567374085895\\
3.51351351351351	-0.101111114557121\\
3.52352352352352	-0.108432888527968\\
3.53353353353353	-0.115822197527017\\
3.54354354354354	-0.123254245921596\\
3.55355355355355	-0.1282721694296\\
3.56356356356356	-0.135711357840038\\
3.57357357357357	-0.142587316407579\\
3.58358358358358	-0.147729573064436\\
3.59359359359359	-0.155010614042819\\
3.6036036036036	-0.162335970274409\\
3.61361361361361	-0.168662882169312\\
3.62362362362362	-0.174244391588553\\
3.63363363363363	-0.180623994403591\\
3.64364364364364	-0.187318240780251\\
3.65365365365365	-0.192088794271528\\
3.66366366366366	-0.195978906066402\\
3.67367367367367	-0.205560249091604\\
3.68368368368368	-0.20972107771852\\
3.69369369369369	-0.215031243058195\\
3.7037037037037	-0.219425724197229\\
3.71371371371371	-0.226432204350193\\
3.72372372372372	-0.23011604096935\\
3.73373373373373	-0.236029723256588\\
3.74374374374374	-0.240188542498437\\
3.75375375375375	-0.245622802437932\\
3.76376376376376	-0.252199827988978\\
3.77377377377377	-0.254555972051359\\
3.78378378378378	-0.258959632142093\\
3.79379379379379	-0.264858698647042\\
3.8038038038038	-0.268765144543115\\
3.81381381381381	-0.272921762656664\\
3.82382382382382	-0.278247582371722\\
3.83383383383383	-0.282601620658267\\
3.84384384384384	-0.285827066059612\\
3.85385385385385	-0.290316435743149\\
3.86386386386386	-0.297281889776611\\
3.87387387387387	-0.298363690615883\\
3.88388388388388	-0.301467131212926\\
3.89389389389389	-0.306457157886231\\
3.9039039039039	-0.308751202446301\\
3.91391391391391	-0.31409054279773\\
3.92392392392392	-0.316853675372369\\
3.93393393393393	-0.32235544294034\\
3.94394394394394	-0.32276502823958\\
3.95395395395395	-0.328467313853757\\
3.96396396396396	-0.33155826636048\\
3.97397397397397	-0.332953866942864\\
3.98398398398398	-0.33801506579035\\
3.99399399399399	-0.342990390207159\\
4.004004004004	-0.345043044450808\\
4.01401401401401	-0.347738384359469\\
4.02402402402402	-0.349523381777828\\
4.03403403403403	-0.354916201390033\\
4.04404404404404	-0.355715703378699\\
4.05405405405405	-0.358228858152332\\
4.06406406406406	-0.365561704167545\\
4.07407407407407	-0.364116452242456\\
4.08408408408408	-0.370014509392026\\
4.09409409409409	-0.372328155148565\\
4.1041041041041	-0.375007249941767\\
4.11411411411411	-0.380317050029357\\
4.12412412412412	-0.381039327651707\\
4.13413413413413	-0.38463297397486\\
4.14414414414414	-0.385998522259571\\
4.15415415415415	-0.390861523378343\\
4.16416416416416	-0.391395357770102\\
4.17417417417417	-0.395116204281119\\
4.18418418418418	-0.397797647990692\\
4.19419419419419	-0.400251040186758\\
4.2042042042042	-0.40380583929199\\
4.21421421421421	-0.405686577451153\\
4.22422422422422	-0.409907701341699\\
4.23423423423423	-0.413257940983234\\
4.24424424424424	-0.414638915096086\\
4.25425425425425	-0.419114121279353\\
4.26426426426426	-0.42116596732354\\
4.27427427427427	-0.42515649529973\\
4.28428428428428	-0.426326088706917\\
4.29429429429429	-0.431410750355614\\
4.3043043043043	-0.433256205819764\\
4.31431431431431	-0.437534014174287\\
4.32432432432432	-0.443355793394338\\
4.33433433433433	-0.44207746426876\\
4.34434434434434	-0.4490394524096\\
4.35435435435435	-0.451115705859145\\
4.36436436436436	-0.454275495084495\\
4.37437437437437	-0.458529194859474\\
4.38438438438438	-0.462922461786401\\
4.39439439439439	-0.464023632283442\\
4.4044044044044	-0.469457308484602\\
4.41441441441441	-0.475029174394251\\
4.42442442442442	-0.475068507244729\\
4.43443443443443	-0.480810078924341\\
4.44444444444444	-0.484452687957785\\
4.45445445445445	-0.488000745160612\\
4.46446446446446	-0.491207686888321\\
4.47447447447447	-0.495346523673287\\
4.48448448448448	-0.500862949989429\\
4.49449449449449	-0.504059202621683\\
4.5045045045045	-0.507121514868368\\
4.51451451451451	-0.512866291695823\\
4.52452452452452	-0.517323103210512\\
4.53453453453453	-0.522239319678845\\
4.54454454454454	-0.524393799648514\\
4.55455455455455	-0.530756302076123\\
4.56456456456456	-0.534551587572004\\
4.57457457457457	-0.537024985566855\\
4.58458458458458	-0.541868357730742\\
4.59459459459459	-0.550159601633065\\
4.6046046046046	-0.551831833581163\\
4.61461461461461	-0.557174140064532\\
4.62462462462462	-0.56362575272377\\
4.63463463463463	-0.566191180496374\\
4.64464464464464	-0.571633129401202\\
4.65465465465465	-0.578694513299121\\
4.66466466466466	-0.583364472560607\\
4.67467467467467	-0.586100885681047\\
4.68468468468468	-0.593065708556627\\
4.69469469469469	-0.595889733446352\\
4.7047047047047	-0.60233716589835\\
4.71471471471471	-0.605902069600549\\
4.72472472472472	-0.612919817473742\\
4.73473473473473	-0.616357614084965\\
4.74474474474474	-0.623195976181392\\
4.75475475475475	-0.628389079902913\\
4.76476476476476	-0.634357599272339\\
4.77477477477477	-0.640086729587037\\
4.78478478478478	-0.644670353408413\\
4.79479479479479	-0.647477279270576\\
4.8048048048048	-0.654043331634114\\
4.81481481481481	-0.65960023071245\\
4.82482482482482	-0.664377974297304\\
4.83483483483483	-0.66865243715683\\
4.84484484484484	-0.67409325176979\\
4.85485485485485	-0.678529941359506\\
4.86486486486486	-0.683420746793738\\
4.87487487487487	-0.68890024677578\\
4.88488488488488	-0.694141019525428\\
4.89489489489489	-0.698392296221073\\
4.9049049049049	-0.705221728923117\\
4.91491491491491	-0.708468445350697\\
4.92492492492492	-0.714463479075283\\
4.93493493493493	-0.719849350452468\\
4.94494494494494	-0.723351611910697\\
4.95495495495495	-0.728385879334717\\
4.96496496496496	-0.733615699728164\\
4.97497497497497	-0.739246047040172\\
4.98498498498498	-0.742512096701943\\
4.99499499499499	-0.747627877384457\\
5.00500500500501	-0.751290286101709\\
5.01501501501502	-0.757932201661723\\
5.02502502502503	-0.760259275619262\\
5.03503503503504	-0.761921197185306\\
5.04504504504505	-0.769267890350712\\
5.05505505505506	-0.772364494283334\\
5.06506506506507	-0.775592747474666\\
5.07507507507508	-0.781068842801915\\
5.08508508508509	-0.781868773596753\\
5.0950950950951	-0.787575945910646\\
5.10510510510511	-0.790193717492431\\
5.11511511511512	-0.794596520415553\\
5.12512512512513	-0.797297506172612\\
5.13513513513514	-0.799322908830046\\
5.14514514514515	-0.802230729933009\\
5.15515515515516	-0.803691461798237\\
5.16516516516517	-0.805312022876326\\
5.17517517517518	-0.809480723441012\\
5.18518518518519	-0.813345989884401\\
5.1951951951952	-0.813326349680279\\
5.20520520520521	-0.815201801140455\\
5.21521521521522	-0.819409406512946\\
5.22522522522523	-0.820030760775787\\
5.23523523523524	-0.821851836607049\\
5.24524524524525	-0.822613287992068\\
5.25525525525526	-0.821730425035343\\
5.26526526526527	-0.824224116059699\\
5.27527527527528	-0.823662574365294\\
5.28528528528529	-0.826892999705746\\
5.2952952952953	-0.825335088607044\\
5.30530530530531	-0.827473091597744\\
5.31531531531532	-0.826896407307887\\
5.32532532532533	-0.82500725427317\\
5.33533533533534	-0.826900822593876\\
5.34534534534535	-0.824822159675279\\
5.35535535535536	-0.825678518526462\\
5.36536536536537	-0.823139041622281\\
5.37537537537538	-0.824287731236097\\
5.38538538538539	-0.822501024084742\\
5.3953953953954	-0.82012996475193\\
5.40540540540541	-0.817154346147142\\
5.41541541541542	-0.815157016949769\\
5.42542542542543	-0.814459465758043\\
5.43543543543544	-0.811307288057963\\
5.44544544544545	-0.80768747597825\\
5.45545545545546	-0.805163055895751\\
5.46546546546547	-0.803505180100681\\
5.47547547547548	-0.800040613331303\\
5.48548548548549	-0.796792265996983\\
5.4954954954955	-0.791169878555444\\
5.50550550550551	-0.786287542739376\\
5.51551551551552	-0.784098744798578\\
5.52552552552553	-0.781737307206961\\
5.53553553553554	-0.776287361361984\\
5.54554554554555	-0.773044166788363\\
5.55555555555556	-0.767815915554355\\
5.56556556556557	-0.760328924308208\\
5.57557557557558	-0.755657370728785\\
5.58558558558559	-0.750336842663498\\
5.5955955955956	-0.744519903355696\\
5.60560560560561	-0.739123934900827\\
5.61561561561562	-0.733307124369319\\
5.62562562562563	-0.726712241663986\\
5.63563563563564	-0.718873796415519\\
5.64564564564565	-0.711218542224028\\
5.65565565565566	-0.704671324350557\\
5.66566566566567	-0.699535862008331\\
5.67567567567568	-0.690252990022112\\
5.68568568568569	-0.681954705023268\\
5.6956956956957	-0.675197903775984\\
5.70570570570571	-0.66712416626513\\
5.71571571571572	-0.657778293998586\\
5.72572572572573	-0.649316351102262\\
5.73573573573574	-0.641478686783956\\
5.74574574574575	-0.63123454893341\\
5.75575575575576	-0.623332648704861\\
5.76576576576577	-0.615332162995476\\
5.77577577577578	-0.605736266750749\\
5.78578578578579	-0.594592608615794\\
5.7957957957958	-0.584884957348241\\
5.80580580580581	-0.574781516664104\\
5.81581581581582	-0.565241725552483\\
5.82582582582583	-0.553690065511271\\
5.83583583583584	-0.542569416695519\\
5.84584584584585	-0.532220106324315\\
5.85585585585586	-0.520719788998868\\
5.86586586586587	-0.510014462051502\\
5.87587587587588	-0.499467577526728\\
5.88588588588589	-0.487253758644552\\
5.8958958958959	-0.478512207230538\\
5.90590590590591	-0.462916460199693\\
5.91591591591592	-0.451915481809164\\
5.92592592592593	-0.440207434554829\\
5.93593593593594	-0.42725294605029\\
5.94594594594595	-0.41494804754839\\
5.95595595595596	-0.403133642383241\\
5.96596596596597	-0.389781645370299\\
5.97597597597598	-0.37709353252285\\
5.98598598598599	-0.3634974998578\\
5.995995995996	-0.350431169813339\\
6.00600600600601	-0.337127860821048\\
6.01601601601602	-0.323017916632256\\
6.02602602602603	-0.311466847479795\\
6.03603603603604	-0.296714763726247\\
6.04604604604605	-0.282834073995844\\
6.05605605605606	-0.269303524079825\\
6.06606606606607	-0.255651962445918\\
6.07607607607608	-0.240015400179474\\
6.08608608608609	-0.226355968383479\\
6.0960960960961	-0.21251400776754\\
6.10610610610611	-0.198613146465662\\
6.11611611611612	-0.185169389117327\\
6.12612612612613	-0.170173871797387\\
6.13613613613614	-0.15285989865835\\
6.14614614614615	-0.139087258472729\\
6.15615615615616	-0.123268405237435\\
6.16616616616617	-0.10950747284059\\
6.17617617617618	-0.0931840281227809\\
6.18618618618619	-0.0775060127865832\\
6.1961961961962	-0.0626349112270989\\
6.20620620620621	-0.0466778269553978\\
6.21621621621622	-0.0337474709273877\\
6.22622622622623	-0.0156689025135851\\
6.23623623623624	-0.000623256415670626\\
6.24624624624625	0.0156058139987361\\
6.25625625625626	0.0324442602105929\\
6.26626626626627	0.0483612087529238\\
6.27627627627628	0.0624675677694771\\
6.28628628628629	0.0809549661675539\\
6.2962962962963	0.0955483701861757\\
6.30630630630631	0.1127515404863\\
6.31631631631632	0.128169154545637\\
6.32632632632633	0.145567410201737\\
6.33633633633634	0.161629814917491\\
6.34634634634635	0.178599911505344\\
6.35635635635636	0.196174174677801\\
6.36636636636637	0.214475327510895\\
6.37637637637638	0.228707674254728\\
6.38638638638639	0.248629775762565\\
6.3963963963964	0.26199507674667\\
6.40640640640641	0.279049609745116\\
6.41641641641642	0.296335119183721\\
6.42642642642643	0.312733245515796\\
6.43643643643644	0.331790531646317\\
6.44644644644645	0.34776710602186\\
6.45645645645646	0.36535697037176\\
6.46646646646647	0.380505905111946\\
6.47647647647648	0.400782664288402\\
6.48648648648649	0.417833348046778\\
6.4964964964965	0.431837151243652\\
6.50650650650651	0.450213336998933\\
6.51651651651652	0.468905946476785\\
6.52652652652653	0.486064866620621\\
6.53653653653654	0.503497836009964\\
6.54654654654655	0.521024814661238\\
6.55655655655656	0.538403428940342\\
6.56656656656657	0.55577405982374\\
6.57657657657658	0.57592995055413\\
6.58658658658659	0.591273289892552\\
6.5965965965966	0.609199872820411\\
6.60660660660661	0.628499951670675\\
6.61661661661662	0.644943734828229\\
6.62662662662663	0.663172962893492\\
6.63663663663664	0.679366655457863\\
6.64664664664665	0.697262399556578\\
6.65665665665666	0.71625975299336\\
6.66666666666667	0.734469466464146\\
6.67667667667668	0.751395047397525\\
6.68668668668669	0.76859884273225\\
6.6966966966967	0.783221694896254\\
6.70670670670671	0.803055387394379\\
6.71671671671672	0.822537951042781\\
6.72672672672673	0.83761112235721\\
6.73673673673674	0.856925762374832\\
6.74674674674675	0.876234152429514\\
6.75675675675676	0.894162517308533\\
6.76676676676677	0.911327439881066\\
6.77677677677678	0.929410936306895\\
6.78678678678679	0.94724988597042\\
6.7967967967968	0.96680584274379\\
6.80680680680681	0.983538055728408\\
6.81681681681682	0.999939301426697\\
6.82682682682683	1.01711486196664\\
6.83683683683684	1.03502162159164\\
6.84684684684685	1.05452142352195\\
6.85685685685686	1.06841446340626\\
6.86686686686687	1.08795122612326\\
6.87687687687688	1.10502954556114\\
6.88688688688689	1.1231022169592\\
6.8968968968969	1.14134604469087\\
6.90690690690691	1.15608350303168\\
6.91691691691692	1.17556366078062\\
6.92692692692693	1.19247665450746\\
6.93693693693694	1.2085843337324\\
6.94694694694695	1.22623764167574\\
6.95695695695696	1.24188351816379\\
6.96696696696697	1.26011355657447\\
6.97697697697698	1.27700206892686\\
6.98698698698699	1.29280602216713\\
6.996996996997	1.3121961999122\\
7.00700700700701	1.32540171328953\\
7.01701701701702	1.34476484786337\\
7.02702702702703	1.35970513518216\\
7.03703703703704	1.37822234103215\\
7.04704704704705	1.39347999814577\\
7.05705705705706	1.40919757514657\\
7.06706706706707	1.42720346233805\\
7.07707707707708	1.43977016352771\\
7.08708708708709	1.45808572805031\\
7.0970970970971	1.47385180176727\\
7.10710710710711	1.48898001268187\\
7.11711711711712	1.5039457599119\\
7.12712712712713	1.51871933326947\\
7.13713713713714	1.53431451793384\\
7.14714714714715	1.55035424461267\\
7.15715715715716	1.56376147636561\\
7.16716716716717	1.57943912556547\\
7.17717717717718	1.59509864929577\\
7.18718718718719	1.60962443787928\\
7.1971971971972	1.62556971575188\\
7.20720720720721	1.63821360861761\\
7.21721721721722	1.65256240590499\\
7.22722722722723	1.66581722110937\\
7.23723723723724	1.68142243274904\\
7.24724724724725	1.69333547663748\\
7.25725725725726	1.70610270301597\\
7.26726726726727	1.71960809804283\\
7.27727727727728	1.7302210265748\\
7.28728728728729	1.74447807014922\\
7.2972972972973	1.75641396327718\\
7.30730730730731	1.77004851456897\\
7.31731731731732	1.78324377476712\\
7.32732732732733	1.79462971413386\\
7.33733733733734	1.8078969651259\\
7.34734734734735	1.81719797805632\\
7.35735735735736	1.82911966623339\\
7.36736736736737	1.83992259110022\\
7.37737737737738	1.85055032916529\\
7.38738738738739	1.86077616255447\\
7.3973973973974	1.87508450526556\\
7.40740740740741	1.88246423701156\\
7.41741741741742	1.89349189350148\\
7.42742742742743	1.90285130085274\\
7.43743743743744	1.91275801889236\\
7.44744744744745	1.92160110434182\\
7.45745745745746	1.92915527767228\\
7.46746746746747	1.93835927764237\\
7.47747747747748	1.94494730265371\\
7.48748748748749	1.95499917203682\\
7.4974974974975	1.96344663678973\\
7.50750750750751	1.9704709379149\\
7.51751751751752	1.97719235088126\\
7.52752752752753	1.98380667937663\\
7.53753753753754	1.99143378008434\\
7.54754754754755	1.99989017090354\\
7.55755755755756	2.00445939355758\\
7.56756756756757	2.01053335833837\\
7.57757757757758	2.01575933420196\\
7.58758758758759	2.02140975865399\\
7.5975975975976	2.02921613674566\\
7.60760760760761	2.03048965158483\\
7.61761761761762	2.03685844227222\\
7.62762762762763	2.04118193194583\\
7.63763763763764	2.04247669266287\\
7.64764764764765	2.04816987785463\\
7.65765765765766	2.05325831308975\\
7.66766766766767	2.05499549961872\\
7.67767767767768	2.0571957317991\\
7.68768768768769	2.05930881995476\\
7.6976976976977	2.06335616634558\\
7.70770770770771	2.06543531031344\\
7.71771771771772	2.06633633083812\\
7.72772772772773	2.06833053084099\\
7.73773773773774	2.06882397539762\\
7.74774774774775	2.07014193346306\\
7.75775775775776	2.06867590654224\\
7.76776776776777	2.07083290886936\\
7.77777777777778	2.07044081904102\\
7.78778778778779	2.06886536989822\\
7.7977977977978	2.07006279212405\\
7.80780780780781	2.06913425508894\\
7.81781781781782	2.06850751846284\\
7.82782782782783	2.06641038837396\\
7.83783783783784	2.06389201788334\\
7.84784784784785	2.06211277732138\\
7.85785785785786	2.05872588278003\\
7.86786786786787	2.05620398806303\\
7.87787787787788	2.05665045517726\\
7.88788788788789	2.04920620968307\\
7.8978978978979	2.04541084798396\\
7.90790790790791	2.04423747384787\\
7.91791791791792	2.03833084144149\\
7.92792792792793	2.03359960512805\\
7.93793793793794	2.02952854852978\\
7.94794794794795	2.02626948772918\\
7.95795795795796	2.01807598515233\\
7.96796796796797	2.01575392130876\\
7.97797797797798	2.00895687570492\\
7.98798798798799	2.00108680956362\\
7.997997997998	1.99607983491918\\
8.00800800800801	1.99082323930971\\
8.01801801801802	1.98240272828903\\
8.02802802802803	1.97543308575065\\
8.03803803803804	1.96860570099885\\
8.04804804804805	1.96314088752329\\
8.05805805805806	1.95321828645683\\
8.06806806806807	1.94646489475145\\
8.07807807807808	1.93752649480503\\
8.08808808808809	1.92826918470726\\
8.0980980980981	1.9198909553875\\
8.10810810810811	1.9138950622353\\
8.11811811811812	1.90443366799538\\
8.12812812812813	1.89603988243881\\
8.13813813813814	1.8872560628777\\
8.14814814814815	1.87566895823468\\
8.15815815815816	1.86432519702133\\
8.16816816816817	1.85379348950846\\
8.17817817817818	1.84361063852016\\
8.18818818818819	1.83453667310338\\
8.1981981981982	1.8237940186993\\
8.20820820820821	1.81321122561544\\
8.21821821821822	1.80330940469035\\
8.22822822822823	1.79102068898357\\
8.23823823823824	1.77956144249609\\
8.24824824824825	1.76727688710625\\
8.25825825825826	1.75666463055899\\
8.26826826826827	1.74340530001561\\
8.27827827827828	1.73164836374825\\
8.28828828828829	1.71965742892061\\
8.2982982982983	1.70712044454671\\
8.30830830830831	1.69577835974826\\
8.31831831831832	1.68366619888803\\
8.32832832832833	1.67013252254463\\
8.33833833833834	1.65816615623138\\
8.34834834834835	1.64390419364892\\
8.35835835835836	1.63005110319519\\
8.36836836836837	1.61625693395508\\
8.37837837837838	1.60353800368467\\
8.38838838838839	1.59110536922638\\
8.3983983983984	1.57766404399692\\
8.40840840840841	1.56387544643804\\
8.41841841841842	1.54745888688055\\
8.42842842842843	1.53594319760536\\
8.43843843843844	1.52135935448159\\
8.44844844844845	1.50686867282141\\
8.45845845845846	1.491528459454\\
8.46846846846847	1.47858209781815\\
8.47847847847848	1.46498054106478\\
8.48848848848849	1.44810637410133\\
8.4984984984985	1.43374516223571\\
8.50850850850851	1.42121294813633\\
8.51851851851852	1.40626759391277\\
8.52852852852853	1.39058274739032\\
8.53853853853854	1.37520304302596\\
8.54854854854855	1.36036969081058\\
8.55855855855856	1.34495743893359\\
8.56856856856857	1.33111897822985\\
8.57857857857858	1.31643849930271\\
8.58858858858859	1.30100324921691\\
8.5985985985986	1.28465411352636\\
8.60860860860861	1.26951719363808\\
8.61861861861862	1.25595382730455\\
8.62862862862863	1.23891539978467\\
8.63863863863864	1.22578951186134\\
8.64864864864865	1.20904885048394\\
8.65865865865866	1.19484335148667\\
8.66866866866867	1.17650427784081\\
8.67867867867868	1.16184009206533\\
8.68868868868869	1.1481952539611\\
8.6986986986987	1.13331665435777\\
8.70870870870871	1.1165307556658\\
8.71871871871872	1.10055484107255\\
8.72872872872873	1.08618196435494\\
8.73873873873874	1.07174338272725\\
8.74874874874875	1.05665389049691\\
8.75875875875876	1.04112063075713\\
8.76876876876877	1.02615934507165\\
8.77877877877878	1.01087382426245\\
8.78878878878879	0.994251352156899\\
8.7987987987988	0.9803244081812\\
8.80880880880881	0.96244090244424\\
8.81881881881882	0.948401504172814\\
8.82882882882883	0.932641542727327\\
8.83883883883884	0.917946659773652\\
8.84884884884885	0.903855628286376\\
8.85885885885886	0.887524747343964\\
8.86886886886887	0.871761209405885\\
8.87887887887888	0.859231262878713\\
8.88888888888889	0.842732118083392\\
8.8988988988989	0.826252983828243\\
8.90890890890891	0.810757442001132\\
8.91891891891892	0.797546772319886\\
8.92892892892893	0.782643255455319\\
8.93893893893894	0.769275815835636\\
8.94894894894895	0.753263802229833\\
8.95895895895896	0.737756879033841\\
8.96896896896897	0.723731339779948\\
8.97897897897898	0.709764124265466\\
8.98898898898899	0.693444528706864\\
8.998998998999	0.679493022264713\\
9.00900900900901	0.665040873945759\\
9.01901901901902	0.651635312646954\\
9.02902902902903	0.636694054990923\\
9.03903903903904	0.622476327969705\\
9.04904904904905	0.609200327570881\\
9.05905905905906	0.595714502567305\\
9.06906906906907	0.582380760842901\\
9.07907907907908	0.566001653601048\\
9.08908908908909	0.552053250662742\\
9.0990990990991	0.538191479189257\\
9.10910910910911	0.523679949645207\\
9.11911911911912	0.512508951125775\\
9.12912912912913	0.498125144907683\\
9.13913913913914	0.484336810342955\\
9.14914914914915	0.472633683416831\\
9.15915915915916	0.456162276880883\\
9.16916916916917	0.444380906243334\\
9.17917917917918	0.431947455978286\\
9.18918918918919	0.420013930320464\\
9.1991991991992	0.404675420659685\\
9.20920920920921	0.393129357734367\\
9.21921921921922	0.379962588849887\\
9.22922922922923	0.366531786537533\\
9.23923923923924	0.354479624414828\\
9.24924924924925	0.343366240085283\\
9.25925925925926	0.329202303143815\\
9.26926926926927	0.317230249079279\\
9.27927927927928	0.303706278597353\\
9.28928928928929	0.2921453804184\\
9.2992992992993	0.281221671146008\\
9.30930930930931	0.267561010139252\\
9.31931931931932	0.255631747369617\\
9.32932932932933	0.244628218040894\\
9.33933933933934	0.233742385512064\\
9.34934934934935	0.221083626717739\\
9.35935935935936	0.208914897177312\\
9.36936936936937	0.198695407946762\\
9.37937937937938	0.186275308338632\\
9.38938938938939	0.177274881868197\\
9.3993993993994	0.165295262046012\\
9.40940940940941	0.154480799413384\\
9.41941941941942	0.142714809287953\\
9.42942942942943	0.130865196519363\\
9.43943943943944	0.123947114562673\\
9.44944944944945	0.112611001825245\\
9.45945945945946	0.10034374444293\\
9.46946946946947	0.0898755255741061\\
9.47947947947948	0.0810744551425799\\
9.48948948948949	0.0694315949631697\\
9.4994994994995	0.0603015401539712\\
9.50950950950951	0.0507540724386359\\
9.51951951951952	0.0401136569249105\\
9.52952952952953	0.0289641297299644\\
9.53953953953954	0.0197899087583193\\
9.54954954954955	0.0110847781261884\\
9.55955955955956	0.00171067910391065\\
9.56956956956957	-0.00743874817606321\\
9.57957957957958	-0.0172845941069045\\
9.58958958958959	-0.0280313046196549\\
9.5995995995996	-0.0362176659118886\\
9.60960960960961	-0.0442530755591528\\
9.61961961961962	-0.0524508320922119\\
9.62962962962963	-0.0615511906393671\\
9.63963963963964	-0.0699006140772687\\
9.64964964964965	-0.0778209081209539\\
9.65965965965966	-0.0880866727280879\\
9.66966966966967	-0.0968893584858295\\
9.67967967967968	-0.103318176442602\\
9.68968968968969	-0.112417907478087\\
9.6996996996997	-0.121004001522094\\
9.70970970970971	-0.129262296659419\\
9.71971971971972	-0.135229478775989\\
9.72972972972973	-0.14266112707714\\
9.73973973973974	-0.149799146945939\\
9.74974974974975	-0.158489938787471\\
9.75975975975976	-0.166456270171705\\
9.76976976976977	-0.174861209277377\\
9.77977977977978	-0.180557495844453\\
9.78978978978979	-0.188424754584212\\
9.7997997997998	-0.193021115673556\\
9.80980980980981	-0.201948890131413\\
9.81981981981982	-0.206537598511844\\
9.82982982982983	-0.213791928467233\\
9.83983983983984	-0.219609289077699\\
9.84984984984985	-0.229341378485621\\
9.85985985985986	-0.232136847082079\\
9.86986986986987	-0.240292774478967\\
9.87987987987988	-0.245590816632834\\
9.88988988988989	-0.249389444010862\\
9.8998998998999	-0.2558050356908\\
9.90990990990991	-0.262458026795471\\
9.91991991991992	-0.269081498940792\\
9.92992992992993	-0.27246259573567\\
9.93993993993994	-0.277654370572922\\
9.94994994994995	-0.284361373426438\\
9.95995995995996	-0.287899831573963\\
9.96996996996997	-0.292707200553769\\
9.97997997997998	-0.297054896958645\\
9.98998998998999	-0.300518871481894\\
10	-0.304819537647068\\
};
\addplot [color=mycolor3,solid]
  table[row sep=crcr]{%
0	1.7800304221382\\
0.01001001001001	1.79406265274188\\
0.02002002002002	1.80893129526262\\
0.03003003003003	1.82346464478479\\
0.04004004004004	1.83529170494044\\
0.0500500500500501	1.84880040279664\\
0.0600600600600601	1.86028872207858\\
0.0700700700700701	1.8741149189711\\
0.0800800800800801	1.88650689394505\\
0.0900900900900901	1.89741485692287\\
0.1001001001001	1.90919085246062\\
0.11011011011011	1.91918574309119\\
0.12012012012012	1.92763325836091\\
0.13013013013013	1.93829116998733\\
0.14014014014014	1.94801337891536\\
0.15015015015015	1.95701051768414\\
0.16016016016016	1.967077216274\\
0.17017017017017	1.97402659033509\\
0.18018018018018	1.98306678138073\\
0.19019019019019	1.99121219816283\\
0.2002002002002	1.99872188109225\\
0.21021021021021	2.00456909772885\\
0.22022022022022	2.01178250014356\\
0.23023023023023	2.01654168335136\\
0.24024024024024	2.02295300898662\\
0.25025025025025	2.02823911298911\\
0.26026026026026	2.03443031787381\\
0.27027027027027	2.03755127865561\\
0.28028028028028	2.04387610244779\\
0.29029029029029	2.04640386625029\\
0.3003003003003	2.05121995816554\\
0.31031031031031	2.05433220831155\\
0.32032032032032	2.05781722281248\\
0.33033033033033	2.06213720804876\\
0.34034034034034	2.06272834386393\\
0.35035035035035	2.06392292156706\\
0.36036036036036	2.06538733665377\\
0.37037037037037	2.06832291738863\\
0.38038038038038	2.06959112662323\\
0.39039039039039	2.07110146807311\\
0.4004004004004	2.07168649830286\\
0.41041041041041	2.07057531691145\\
0.42042042042042	2.07251375628478\\
0.43043043043043	2.07193928373947\\
0.44044044044044	2.0702456347433\\
0.45045045045045	2.06997379579449\\
0.46046046046046	2.06878109613372\\
0.47047047047047	2.06744570267774\\
0.48048048048048	2.06763189798779\\
0.49049049049049	2.06332985303542\\
0.500500500500501	2.06310615403792\\
0.510510510510511	2.06017402384998\\
0.520520520520521	2.05732719769625\\
0.530530530530531	2.05569179282797\\
0.540540540540541	2.05111123805939\\
0.550550550550551	2.04887143576333\\
0.560560560560561	2.04627114687359\\
0.570570570570571	2.0441624582101\\
0.580580580580581	2.03733063771101\\
0.590590590590591	2.03382421398681\\
0.600600600600601	2.03068164957308\\
0.610610610610611	2.02757527399534\\
0.620620620620621	2.0215125060845\\
0.630630630630631	2.01634332606468\\
0.640640640640641	2.01276764003989\\
0.650650650650651	2.00765515053343\\
0.660660660660661	2.00479812523867\\
0.670670670670671	1.99598660807671\\
0.680680680680681	1.99167538002035\\
0.690690690690691	1.98466976206862\\
0.700700700700701	1.97975194545626\\
0.710710710710711	1.97277911229479\\
0.720720720720721	1.96820998011647\\
0.730730730730731	1.95944046908841\\
0.740740740740741	1.95332724124408\\
0.750750750750751	1.94818930887639\\
0.760760760760761	1.94122914169792\\
0.770770770770771	1.93457979152906\\
0.780780780780781	1.92729569566231\\
0.790790790790791	1.9197491650283\\
0.800800800800801	1.91526573166809\\
0.810810810810811	1.90613193819857\\
0.820820820820821	1.8991969132776\\
0.830830830830831	1.89250207073264\\
0.840840840840841	1.8857856547938\\
0.850850850850851	1.87522322490363\\
0.860860860860861	1.86633095494432\\
0.870870870870871	1.86028027934403\\
0.880880880880881	1.853282750646\\
0.890890890890891	1.84482710845856\\
0.900900900900901	1.8368727230285\\
0.910910910910911	1.82848955124305\\
0.920920920920921	1.81807121308308\\
0.930930930930931	1.81165303960211\\
0.940940940940941	1.80468959157053\\
0.950950950950951	1.79528283575286\\
0.960960960960961	1.78767417025223\\
0.970970970970971	1.77667064368978\\
0.980980980980981	1.76844746524962\\
0.990990990990991	1.76128330968415\\
1.001001001001	1.75071147923538\\
1.01101101101101	1.74461533142237\\
1.02102102102102	1.7347879285308\\
1.03103103103103	1.72451591250157\\
1.04104104104104	1.71767748671086\\
1.05105105105105	1.70795026602107\\
1.06106106106106	1.6988175824146\\
1.07107107107107	1.69151291823224\\
1.08108108108108	1.68128471209119\\
1.09109109109109	1.67258710607869\\
1.1011011011011	1.66289865254193\\
1.11111111111111	1.65485238272243\\
1.12112112112112	1.64343050123014\\
1.13113113113113	1.6362592837329\\
1.14114114114114	1.62692668447548\\
1.15115115115115	1.61849266323369\\
1.16116116116116	1.60877798452848\\
1.17117117117117	1.59829456143968\\
1.18118118118118	1.59058283886154\\
1.19119119119119	1.58343671596122\\
1.2012012012012	1.57203245226354\\
1.21121121121121	1.56298478797601\\
1.22122122122122	1.55382198242034\\
1.23123123123123	1.54367734044939\\
1.24124124124124	1.53608659858374\\
1.25125125125125	1.52887133330193\\
1.26126126126126	1.51782685451427\\
1.27127127127127	1.51018499583744\\
1.28128128128128	1.50009048510512\\
1.29129129129129	1.49171323214813\\
1.3013013013013	1.48424848740251\\
1.31131131131131	1.47168069825357\\
1.32132132132132	1.46494343826549\\
1.33133133133133	1.45802689424435\\
1.34134134134134	1.44679891563274\\
1.35135135135135	1.43814990149305\\
1.36136136136136	1.42995482905822\\
1.37137137137137	1.42118111716668\\
1.38138138138138	1.41268715319635\\
1.39139139139139	1.40414802289417\\
1.4014014014014	1.39563183391415\\
1.41141141141141	1.38757706507222\\
1.42142142142142	1.38080131396749\\
1.43143143143143	1.37006984014539\\
1.44144144144144	1.36244694275284\\
1.45145145145145	1.35403727115542\\
1.46146146146146	1.34516327786395\\
1.47147147147147	1.33750320004106\\
1.48148148148148	1.32866713826578\\
1.49149149149149	1.32041803817655\\
1.5015015015015	1.31254129535607\\
1.51151151151151	1.30444437721022\\
1.52152152152152	1.29579683982667\\
1.53153153153153	1.29008948799356\\
1.54154154154154	1.28240292857307\\
1.55155155155155	1.27338279430806\\
1.56156156156156	1.26703146052746\\
1.57157157157157	1.25802594695879\\
1.58158158158158	1.25161424828962\\
1.59159159159159	1.24443938278945\\
1.6016016016016	1.23505374792109\\
1.61161161161161	1.2276721219745\\
1.62162162162162	1.21949891623357\\
1.63163163163163	1.21352742156621\\
1.64164164164164	1.20579273751797\\
1.65165165165165	1.1988085406552\\
1.66166166166166	1.19337104312452\\
1.67167167167167	1.18566598325547\\
1.68168168168168	1.17711355053913\\
1.69169169169169	1.17046447781777\\
1.7017017017017	1.16393532274938\\
1.71171171171171	1.15742736433093\\
1.72172172172172	1.15030683840259\\
1.73173173173173	1.14339804299206\\
1.74174174174174	1.13604104123975\\
1.75175175175175	1.12988268030196\\
1.76176176176176	1.12437435736734\\
1.77177177177177	1.11875569424648\\
1.78178178178178	1.11158860857094\\
1.79179179179179	1.10479215805833\\
1.8018018018018	1.09760050975125\\
1.81181181181181	1.09324251191207\\
1.82182182182182	1.089100308532\\
1.83183183183183	1.07903588025808\\
1.84184184184184	1.07212806626505\\
1.85185185185185	1.06933819604464\\
1.86186186186186	1.06083688661781\\
1.87187187187187	1.05892101303486\\
1.88188188188188	1.05121195715455\\
1.89189189189189	1.04485509407292\\
1.9019019019019	1.03999285151722\\
1.91191191191191	1.03057308732022\\
1.92192192192192	1.02730456050066\\
1.93193193193193	1.02236266443291\\
1.94194194194194	1.016884818696\\
1.95195195195195	1.01075678082185\\
1.96196196196196	1.00651682482106\\
1.97197197197197	0.999949868086392\\
1.98198198198198	0.994039551422532\\
1.99199199199199	0.989158813153056\\
2.002002002002	0.985020258069799\\
2.01201201201201	0.978463128722687\\
2.02202202202202	0.97185886544775\\
2.03203203203203	0.966611080840966\\
2.04204204204204	0.962103177640335\\
2.05205205205205	0.957228988494782\\
2.06206206206206	0.950896556335187\\
2.07207207207207	0.945531341277261\\
2.08208208208208	0.939697751335081\\
2.09209209209209	0.933024496374274\\
2.1021021021021	0.929041872729578\\
2.11211211211211	0.923912769983969\\
2.12212212212212	0.91747513782721\\
2.13213213213213	0.913280543416332\\
2.14214214214214	0.908498961505781\\
2.15215215215215	0.903829852024531\\
2.16216216216216	0.898903345204826\\
2.17217217217217	0.891564466242153\\
2.18218218218218	0.889952692588407\\
2.19219219219219	0.881891896452359\\
2.2022022022022	0.875793569378055\\
2.21221221221221	0.868618943259347\\
2.22222222222222	0.863747210217972\\
2.23223223223223	0.85913432511535\\
2.24224224224224	0.854471984451115\\
2.25225225225225	0.847059367630173\\
2.26226226226226	0.842412179147726\\
2.27227227227227	0.838265578121363\\
2.28228228228228	0.830596543747098\\
2.29229229229229	0.823338428809977\\
2.3023023023023	0.821427160668893\\
2.31231231231231	0.814102855156612\\
2.32232232232232	0.807328059780047\\
2.33233233233233	0.802983281232651\\
2.34234234234234	0.797114043659874\\
2.35235235235235	0.789022123562687\\
2.36236236236236	0.783222299949738\\
2.37237237237237	0.777548289353799\\
2.38238238238238	0.771154198980623\\
2.39239239239239	0.763478495088622\\
2.4024024024024	0.760124680028431\\
2.41241241241241	0.753331823399134\\
2.42242242242242	0.745384844327466\\
2.43243243243243	0.739306297543692\\
2.44244244244244	0.730998688686549\\
2.45245245245245	0.726407323526151\\
2.46246246246246	0.718345874124584\\
2.47247247247247	0.712067622275918\\
2.48248248248248	0.706208818457494\\
2.49249249249249	0.697578243502887\\
2.5025025025025	0.689106852358438\\
2.51251251251251	0.681456399186191\\
2.52252252252252	0.675351917801918\\
2.53253253253253	0.668377831796761\\
2.54254254254254	0.660851823860487\\
2.55255255255255	0.655189248275589\\
2.56256256256256	0.644554598984535\\
2.57257257257257	0.636333525277344\\
2.58258258258258	0.629170489428702\\
2.59259259259259	0.623457238064842\\
2.6026026026026	0.614617722829893\\
2.61261261261261	0.605268122061223\\
2.62262262262262	0.597347081551767\\
2.63263263263263	0.588678972179722\\
2.64264264264264	0.579498588673047\\
2.65265265265265	0.574000137991334\\
2.66266266266266	0.561766361245875\\
2.67267267267267	0.554076922861793\\
2.68268268268268	0.545074110451551\\
2.69269269269269	0.535886454098532\\
2.7027027027027	0.529132959538488\\
2.71271271271271	0.519158618116676\\
2.72272272272272	0.511276362156967\\
2.73273273273273	0.499126031475502\\
2.74274274274274	0.490891698471947\\
2.75275275275275	0.478918932778368\\
2.76276276276276	0.47215050971155\\
2.77277277277277	0.462822193431023\\
2.78278278278278	0.451053127789792\\
2.79279279279279	0.44294259835605\\
2.8028028028028	0.433288349989964\\
2.81281281281281	0.421350168534633\\
2.82282282282282	0.413159880901373\\
2.83283283283283	0.40333045187576\\
2.84284284284284	0.390292191057101\\
2.85285285285285	0.381879825430203\\
2.86286286286286	0.37211315533141\\
2.87287287287287	0.360687834775165\\
2.88288288288288	0.349789508405999\\
2.89289289289289	0.339221617134528\\
2.9029029029029	0.328309692380611\\
2.91291291291291	0.319449335750076\\
2.92292292292292	0.307484188778959\\
2.93293293293293	0.297005236793679\\
2.94294294294294	0.285794506287609\\
2.95295295295295	0.274763265846431\\
2.96296296296296	0.263091427065638\\
2.97297297297297	0.250961305944532\\
2.98298298298298	0.240904269037575\\
2.99299299299299	0.228742460929599\\
3.003003003003	0.217862856158613\\
3.01301301301301	0.206850076287645\\
3.02302302302302	0.195305353477008\\
3.03303303303303	0.184781786432361\\
3.04304304304304	0.173722143045457\\
3.05305305305305	0.160572279186446\\
3.06306306306306	0.149526007881702\\
3.07307307307307	0.1383592192498\\
3.08308308308308	0.12574218322168\\
3.09309309309309	0.114369307617038\\
3.1031031031031	0.103618831474611\\
3.11311311311311	0.0922094950234707\\
3.12312312312312	0.0804282568188046\\
3.13313313313313	0.0676828524229535\\
3.14314314314314	0.0566822190030097\\
3.15315315315315	0.043398307897335\\
3.16316316316316	0.0329237844169241\\
3.17317317317317	0.0218664780874605\\
3.18318318318318	0.00770299676294095\\
3.19319319319319	-0.00354509572494288\\
3.2032032032032	-0.0145785484438203\\
3.21321321321321	-0.0245886103158598\\
3.22322322322322	-0.0354312350801897\\
3.23323323323323	-0.0508456728655964\\
3.24324324324324	-0.0606560060720893\\
3.25325325325325	-0.0739546538299836\\
3.26326326326326	-0.0829933089845978\\
3.27327327327327	-0.0945198023700368\\
3.28328328328328	-0.108889602650119\\
3.29329329329329	-0.119642746044176\\
3.3033033033033	-0.131595329222352\\
3.31331331331331	-0.142830186986289\\
3.32332332332332	-0.153645398423491\\
3.33333333333333	-0.16536746092459\\
3.34334334334334	-0.176717180433103\\
3.35335335335335	-0.187681910132363\\
3.36336336336336	-0.198871502677856\\
3.37337337337337	-0.209344082927699\\
3.38338338338338	-0.219894297907932\\
3.39339339339339	-0.231665951016985\\
3.4034034034034	-0.244140304029829\\
3.41341341341341	-0.253957956123383\\
3.42342342342342	-0.264296191953052\\
3.43343343343343	-0.276505628397352\\
3.44344344344344	-0.286576740668288\\
3.45345345345345	-0.298957587237471\\
3.46346346346346	-0.307524638604814\\
3.47347347347347	-0.318307220192478\\
3.48348348348348	-0.328430405951594\\
3.49349349349349	-0.339801342071256\\
3.5035035035035	-0.348219144142949\\
3.51351351351351	-0.359621690897496\\
3.52352352352352	-0.369387630685946\\
3.53353353353353	-0.380347963836941\\
3.54354354354354	-0.389411049813841\\
3.55355355355355	-0.398248690714821\\
3.56356356356356	-0.410188543708131\\
3.57357357357357	-0.418186037195735\\
3.58358358358358	-0.42634501348033\\
3.59359359359359	-0.434928070827035\\
3.6036036036036	-0.445882850654268\\
3.61361361361361	-0.456089065050921\\
3.62362362362362	-0.464420602534096\\
3.63363363363363	-0.471597498263759\\
3.64364364364364	-0.481941711413255\\
3.65365365365365	-0.491565143649904\\
3.66366366366366	-0.49981632485676\\
3.67367367367367	-0.510034368518634\\
3.68368368368368	-0.517620259612719\\
3.69369369369369	-0.524856557439674\\
3.7037037037037	-0.532654967556475\\
3.71371371371371	-0.541316800701496\\
3.72372372372372	-0.548177783393144\\
3.73373373373373	-0.558161740870795\\
3.74374374374374	-0.564905389757839\\
3.75375375375375	-0.570625276358154\\
3.76376376376376	-0.578670706857221\\
3.77377377377377	-0.586267650096455\\
3.78378378378378	-0.592747301059524\\
3.79379379379379	-0.599141692112974\\
3.8038038038038	-0.607893498795519\\
3.81381381381381	-0.612956841268971\\
3.82382382382382	-0.62114723075\\
3.83383383383383	-0.628901351644185\\
3.84384384384384	-0.633880816344666\\
3.85385385385385	-0.638687476298295\\
3.86386386386386	-0.64802726824784\\
3.87387387387387	-0.652503107014834\\
3.88388388388388	-0.657889076610149\\
3.89389389389389	-0.664012706392113\\
3.9039039039039	-0.669466283208307\\
3.91391391391391	-0.675674473282371\\
3.92392392392392	-0.678075557160545\\
3.93393393393393	-0.684999427949332\\
3.94394394394394	-0.691729827896387\\
3.95395395395395	-0.69571369927057\\
3.96396396396396	-0.699921943606992\\
3.97397397397397	-0.703912177117292\\
3.98398398398398	-0.710978482023123\\
3.99399399399399	-0.7144209825535\\
4.004004004004	-0.718561628463279\\
4.01401401401401	-0.721624050982303\\
4.02402402402402	-0.728063123841443\\
4.03403403403403	-0.731719165906574\\
4.04404404404404	-0.733095589581269\\
4.05405405405405	-0.737396306281405\\
4.06406406406406	-0.739366238801341\\
4.07407407407407	-0.742361829681743\\
4.08408408408408	-0.748169829357497\\
4.09409409409409	-0.750980661356571\\
4.1041041041041	-0.752067990845273\\
4.11411411411411	-0.755064532868116\\
4.12412412412412	-0.756826748803838\\
4.13413413413413	-0.75985847697764\\
4.14414414414414	-0.764283606489426\\
4.15415415415415	-0.764676937340538\\
4.16416416416416	-0.767918116699167\\
4.17417417417417	-0.769950047264624\\
4.18418418418418	-0.773501932098193\\
4.19419419419419	-0.773766179164184\\
4.2042042042042	-0.774185146119903\\
4.21421421421421	-0.77782642395221\\
4.22422422422422	-0.778079087025098\\
4.23423423423423	-0.778566789471576\\
4.24424424424424	-0.779784433884791\\
4.25425425425425	-0.781678108779833\\
4.26426426426426	-0.783604512036026\\
4.27427427427427	-0.784195934711444\\
4.28428428428428	-0.783319126144789\\
4.29429429429429	-0.784243224814958\\
4.3043043043043	-0.782974326541507\\
4.31431431431431	-0.785215503330596\\
4.32432432432432	-0.783868264662847\\
4.33433433433433	-0.784556312481846\\
4.34434434434434	-0.783772937406521\\
4.35435435435435	-0.784047333678572\\
4.36436436436436	-0.785411021415588\\
4.37437437437437	-0.78299486812793\\
4.38438438438438	-0.781964197554733\\
4.39439439439439	-0.780241411597973\\
4.4044044044044	-0.779437112097595\\
4.41441441441441	-0.779565121254972\\
4.42442442442442	-0.78015950873155\\
4.43443443443443	-0.776029706986702\\
4.44444444444444	-0.776741688543341\\
4.45445445445445	-0.775537887066472\\
4.46446446446446	-0.773660693290833\\
4.47447447447447	-0.771757404649069\\
4.48448448448448	-0.76988138333027\\
4.49449449449449	-0.768957043314771\\
4.5045045045045	-0.766536989479707\\
4.51451451451451	-0.762845954359583\\
4.52452452452452	-0.760606902661497\\
4.53453453453453	-0.758611195340518\\
4.54454454454454	-0.756664680697036\\
4.55455455455455	-0.752197866542028\\
4.56456456456456	-0.750251788199234\\
4.57457457457457	-0.748454086376975\\
4.58458458458458	-0.743844660700046\\
4.59459459459459	-0.741781141234377\\
4.6046046046046	-0.737349834364831\\
4.61461461461461	-0.733456945050047\\
4.62462462462462	-0.729210250418443\\
4.63463463463463	-0.728466140612958\\
4.64464464464464	-0.722392489707638\\
4.65465465465465	-0.718707287751762\\
4.66466466466466	-0.715784913731558\\
4.67467467467467	-0.709138242292225\\
4.68468468468468	-0.705754649283579\\
4.69469469469469	-0.701940311694446\\
4.7047047047047	-0.698104877102644\\
4.71471471471471	-0.69250764391459\\
4.72472472472472	-0.688889686863473\\
4.73473473473473	-0.682285117416641\\
4.74474474474474	-0.677994690250399\\
4.75475475475475	-0.673474016800406\\
4.76476476476476	-0.668075177045589\\
4.77477477477477	-0.664049047232963\\
4.78478478478478	-0.656734577656325\\
4.79479479479479	-0.651217969647623\\
4.8048048048048	-0.645557965370019\\
4.81481481481481	-0.637983245580243\\
4.82482482482482	-0.63562220005122\\
4.83483483483483	-0.62755210279471\\
4.84484484484484	-0.620398568832709\\
4.85485485485485	-0.614924527796014\\
4.86486486486486	-0.607211753192958\\
4.87487487487487	-0.603332802105272\\
4.88488488488488	-0.595026741626778\\
4.89489489489489	-0.588011938288181\\
4.9049049049049	-0.580297410654233\\
4.91491491491491	-0.574708282201526\\
4.92492492492492	-0.565929218981351\\
4.93493493493493	-0.559246784366033\\
4.94494494494494	-0.552080412961498\\
4.95495495495495	-0.543501914283163\\
4.96496496496496	-0.536078115647114\\
4.97497497497497	-0.529129295280569\\
4.98498498498498	-0.520785538892631\\
4.99499499499499	-0.512630922157023\\
5.00500500500501	-0.505322509461519\\
5.01501501501502	-0.495474684715728\\
5.02502502502503	-0.488751769550565\\
5.03503503503504	-0.48045415222376\\
5.04504504504505	-0.469632391625062\\
5.05505505505506	-0.459761605807012\\
5.06506506506507	-0.453826286026523\\
5.07507507507508	-0.445955630857268\\
5.08508508508509	-0.433921519891807\\
5.0950950950951	-0.426252067136319\\
5.10510510510511	-0.415171982107663\\
5.11511511511512	-0.407130394913865\\
5.12512512512513	-0.399229578271264\\
5.13513513513514	-0.389023614395014\\
5.14514514514515	-0.377712272175316\\
5.15515515515516	-0.366892571203697\\
5.16516516516517	-0.359462827358883\\
5.17517517517518	-0.347114437279291\\
5.18518518518519	-0.338256247286623\\
5.1951951951952	-0.32663295885064\\
5.20520520520521	-0.316941056768812\\
5.21521521521522	-0.305198070699714\\
5.22522522522523	-0.295635406087681\\
5.23523523523524	-0.284794335366617\\
5.24524524524525	-0.273377965010115\\
5.25525525525526	-0.262794353915438\\
5.26526526526527	-0.25174957125809\\
5.27527527527528	-0.241190451330052\\
5.28528528528529	-0.231122840585636\\
5.2952952952953	-0.220266047405252\\
5.30530530530531	-0.208099148094571\\
5.31531531531532	-0.198884781831631\\
5.32532532532533	-0.186510058152741\\
5.33533533533534	-0.174530741635674\\
5.34534534534535	-0.163740206808848\\
5.35535535535536	-0.15285549808505\\
5.36536536536537	-0.138594214539374\\
5.37537537537538	-0.127911647223574\\
5.38538538538539	-0.114852853056696\\
5.3953953953954	-0.104902083774471\\
5.40540540540541	-0.0947178489120101\\
5.41541541541542	-0.0830700140524939\\
5.42542542542543	-0.0676707564958275\\
5.43543543543544	-0.059085030246507\\
5.44544544544545	-0.046455079961506\\
5.45545545545546	-0.0355015410663115\\
5.46546546546547	-0.0239080018154033\\
5.47547547547548	-0.00823981428972154\\
5.48548548548549	0.00339389250258149\\
5.4954954954955	0.0145049321442687\\
5.50550550550551	0.0265940828030319\\
5.51551551551552	0.036581287022737\\
5.52552552552553	0.0497505346550444\\
5.53553553553554	0.0613695386091427\\
5.54554554554555	0.0760672372683501\\
5.55555555555556	0.0873438161584555\\
5.56556556556557	0.098430679941458\\
5.57557557557558	0.10863364120465\\
5.58558558558559	0.125021441609553\\
5.5955955955956	0.133492243450266\\
5.60560560560561	0.146324695484582\\
5.61561561561562	0.157806049764578\\
5.62562562562563	0.170825632480781\\
5.63563563563564	0.181761994808576\\
5.64564564564565	0.195114380326771\\
5.65565565565566	0.205573512050363\\
5.66566566566567	0.218665769619733\\
5.67567567567568	0.230305891123514\\
5.68568568568569	0.241297443318922\\
5.6956956956957	0.255226743166096\\
5.70570570570571	0.264658203118787\\
5.71571571571572	0.274574279888683\\
5.72572572572573	0.289171283881094\\
5.73573573573574	0.300097086492034\\
5.74574574574575	0.310646168623218\\
5.75575575575576	0.320273582929274\\
5.76576576576577	0.33118038479919\\
5.77577577577578	0.343261348899916\\
5.78578578578579	0.354948303888375\\
5.7957957957958	0.366039063215678\\
5.80580580580581	0.378256986586913\\
5.81581581581582	0.386913496619102\\
5.82582582582583	0.397311392348514\\
5.83583583583584	0.407267255923363\\
5.84584584584585	0.417789142791557\\
5.85585585585586	0.429312664653725\\
5.86586586586587	0.439520346382646\\
5.87587587587588	0.448181603472507\\
5.88588588588589	0.45749333490946\\
5.8958958958959	0.467735976581034\\
5.90590590590591	0.480008047202891\\
5.91591591591592	0.485219167973337\\
5.92592592592593	0.497096867314708\\
5.93593593593594	0.503952761562804\\
5.94594594594595	0.514194356897422\\
5.95595595595596	0.524132278816069\\
5.96596596596597	0.531729853068542\\
5.97597597597598	0.541008032752193\\
5.98598598598599	0.547116728805477\\
5.995995995996	0.557080632593775\\
6.00600600600601	0.564546291309545\\
6.01601601601602	0.572739891747815\\
6.02602602602603	0.581175358862649\\
6.03603603603604	0.588944063831407\\
6.04604604604605	0.596384280714844\\
6.05605605605606	0.601952513893177\\
6.06606606606607	0.608966862696311\\
6.07607607607608	0.616057687174109\\
6.08608608608609	0.623450984169881\\
6.0960960960961	0.627855262372703\\
6.10610610610611	0.635008253682235\\
6.11611611611612	0.640671302007005\\
6.12612612612613	0.645760302884946\\
6.13613613613614	0.653656400585521\\
6.14614614614615	0.657310143835112\\
6.15615615615616	0.662598322788402\\
6.16616616616617	0.668892771607356\\
6.17617617617618	0.672876690884794\\
6.18618618618619	0.67819368495921\\
6.1961961961962	0.685843831300846\\
6.20620620620621	0.688547246273922\\
6.21621621621622	0.693270824401231\\
6.22622622622623	0.6946275866072\\
6.23623623623624	0.697531584284649\\
6.24624624624625	0.702810732399729\\
6.25625625625626	0.704158665480772\\
6.26626626626627	0.707268058021449\\
6.27627627627628	0.712325511608926\\
6.28628628628629	0.714017329416003\\
6.2962962962963	0.717422483191767\\
6.30630630630631	0.719339835501566\\
6.31631631631632	0.720590696037114\\
6.32632632632633	0.723441740345068\\
6.33633633633634	0.725224953311053\\
6.34634634634635	0.726065937904816\\
6.35635635635636	0.727745496518764\\
6.36636636636637	0.728914400419331\\
6.37637637637638	0.729043638532759\\
6.38638638638639	0.732106200583791\\
6.3963963963964	0.730069033202501\\
6.40640640640641	0.73013973624957\\
6.41641641641642	0.731879388745763\\
6.42642642642643	0.730962386597859\\
6.43643643643644	0.731283859320177\\
6.44644644644645	0.729647944404647\\
6.45645645645646	0.730884649400882\\
6.46646646646647	0.727243867444284\\
6.47647647647648	0.7272283388481\\
6.48648648648649	0.726815514328251\\
6.4964964964965	0.725338618766787\\
6.50650650650651	0.72454216078985\\
6.51651651651652	0.721613512717407\\
6.52652652652653	0.721175122683347\\
6.53653653653654	0.720094416793791\\
6.54654654654655	0.716516129180635\\
6.55655655655656	0.711391750452972\\
6.56656656656657	0.709998802693501\\
6.57657657657658	0.707924257609877\\
6.58658658658659	0.704212253655723\\
6.5965965965966	0.699659022437956\\
6.60660660660661	0.697500911205349\\
6.61661661661662	0.694358056874413\\
6.62662662662663	0.691856935342618\\
6.63663663663664	0.687451136617667\\
6.64664664664665	0.685308767864978\\
6.65665665665666	0.680065226522816\\
6.66666666666667	0.67547894400491\\
6.67667667667668	0.672255061746461\\
6.68668668668669	0.667304731750419\\
6.6966966966967	0.665290771949741\\
6.70670670670671	0.659388028713096\\
6.71671671671672	0.654527539284511\\
6.72672672672673	0.651488533752859\\
6.73673673673674	0.643813963274896\\
6.74674674674675	0.640965799048534\\
6.75675675675676	0.634765430847423\\
6.76676676676677	0.628004619705539\\
6.77677677677678	0.62405027655194\\
6.78678678678679	0.618703316524032\\
6.7967967967968	0.614807432702611\\
6.80680680680681	0.607654239052352\\
6.81681681681682	0.601721409690015\\
6.82682682682683	0.599273810610744\\
6.83683683683684	0.593992941508688\\
6.84684684684685	0.586007594774405\\
6.85685685685686	0.580262514106843\\
6.86686686686687	0.575335052054146\\
6.87687687687688	0.568496849561836\\
6.88688688688689	0.565222958883819\\
6.8968968968969	0.558994549032367\\
6.90690690690691	0.554006992813331\\
6.91691691691692	0.546586155777451\\
6.92692692692693	0.54126423102408\\
6.93693693693694	0.53620491065047\\
6.94694694694695	0.531125160584347\\
6.95695695695696	0.524182358651285\\
6.96696696696697	0.518469789258767\\
6.97697697697698	0.511846943924801\\
6.98698698698699	0.506499353323982\\
6.996996996997	0.501254437945672\\
7.00700700700701	0.494690732104971\\
7.01701701701702	0.490542651355921\\
7.02702702702703	0.485068516450263\\
7.03703703703704	0.477719269636095\\
7.04704704704705	0.473868168979782\\
7.05705705705706	0.46771494327751\\
7.06706706706707	0.462175658767731\\
7.07707707707708	0.456852310819717\\
7.08708708708709	0.451365642587476\\
7.0970970970971	0.447613115906448\\
7.10710710710711	0.441492262269319\\
7.11711711711712	0.433816600124409\\
7.12712712712713	0.431313870733888\\
7.13713713713714	0.426213200097496\\
7.14714714714715	0.421719464835365\\
7.15715715715716	0.417045104219099\\
7.16716716716717	0.413385169259006\\
7.17717717717718	0.408555935786949\\
7.18718718718719	0.403964662661406\\
7.1971971971972	0.394632971092181\\
7.20720720720721	0.394535960659919\\
7.21721721721722	0.390425268173752\\
7.22722722722723	0.387427154479938\\
7.23723723723724	0.382333008938516\\
7.24724724724725	0.378898135314422\\
7.25725725725726	0.376212810327305\\
7.26726726726727	0.370969186779054\\
7.27727727727728	0.369090008314313\\
7.28728728728729	0.365503214869561\\
7.2972972972973	0.363713104553615\\
7.30730730730731	0.357086277950745\\
7.31731731731732	0.35660225105866\\
7.32732732732733	0.353017827129607\\
7.33733733733734	0.351627689769432\\
7.34734734734735	0.349373735346024\\
7.35735735735736	0.344482793284384\\
7.36736736736737	0.344054189649752\\
7.37737737737738	0.343521146613982\\
7.38738738738739	0.341109524643499\\
7.3973973973974	0.338165765409906\\
7.40740740740741	0.337454692398646\\
7.41741741741742	0.337431241254778\\
7.42742742742743	0.335247753758342\\
7.43743743743744	0.333628072963848\\
7.44744744744745	0.335278325849116\\
7.45745745745746	0.332851491243281\\
7.46746746746747	0.332365455333459\\
7.47747747747748	0.331197401979797\\
7.48748748748749	0.331718175164118\\
7.4974974974975	0.331958829122246\\
7.50750750750751	0.330734773984238\\
7.51751751751752	0.329720422400683\\
7.52752752752753	0.331563938037651\\
7.53753753753754	0.335009690840651\\
7.54754754754755	0.335593489704086\\
7.55755755755756	0.335936143301298\\
7.56756756756757	0.333532847113498\\
7.57757757757758	0.336995060830292\\
7.58758758758759	0.337605786382626\\
7.5975975975976	0.338101157597563\\
7.60760760760761	0.342480949783207\\
7.61761761761762	0.343324825020797\\
7.62762762762763	0.345521242693604\\
7.63763763763764	0.346749137141825\\
7.64764764764765	0.349183649500526\\
7.65765765765766	0.350419469094825\\
7.66766766766767	0.354221940200958\\
7.67767767767768	0.354754144807552\\
7.68768768768769	0.359054309084491\\
7.6976976976977	0.360469674489159\\
7.70770770770771	0.364695996110544\\
7.71771771771772	0.367730758255133\\
7.72772772772773	0.368639720834316\\
7.73773773773774	0.374951214206419\\
7.74774774774775	0.37674920181487\\
7.75775775775776	0.379541744676205\\
7.76776776776777	0.38616364906824\\
7.77777777777778	0.388862396204815\\
7.78778778778779	0.392153043502326\\
7.7977977977978	0.392755999691905\\
7.80780780780781	0.397835112467634\\
7.81781781781782	0.405714128317504\\
7.82782782782783	0.408945709686509\\
7.83783783783784	0.41340514428949\\
7.84784784784785	0.415154088772841\\
7.85785785785786	0.420562858152275\\
7.86786786786787	0.424772692561241\\
7.87787787787788	0.428296538046733\\
7.88788788788789	0.432188583788875\\
7.8978978978979	0.437469121416483\\
7.90790790790791	0.441121507477175\\
7.91791791791792	0.447410269043862\\
7.92792792792793	0.451529872468762\\
7.93793793793794	0.454418621126587\\
7.94794794794795	0.460582363954498\\
7.95795795795796	0.464590784026005\\
7.96796796796797	0.470080371477657\\
7.97797797797798	0.476307797446134\\
7.98798798798799	0.477237197244633\\
7.997997997998	0.482432592677189\\
8.00800800800801	0.488643776023836\\
8.01801801801802	0.494537110452572\\
8.02802802802803	0.497362973690905\\
8.03803803803804	0.501818376183945\\
8.04804804804805	0.507777885093061\\
8.05805805805806	0.511299467797622\\
8.06806806806807	0.514933476626073\\
8.07807807807808	0.520452245407286\\
8.08808808808809	0.52353554302607\\
8.0980980980981	0.52927202420196\\
8.10810810810811	0.532779475715935\\
8.11811811811812	0.537322732426788\\
8.12812812812813	0.540268744559154\\
8.13813813813814	0.54595667765502\\
8.14814814814815	0.551555034206561\\
8.15815815815816	0.555468495033579\\
8.16816816816817	0.558710113423837\\
8.17817817817818	0.563156117562366\\
8.18818818818819	0.565699853843715\\
8.1981981981982	0.569610562151877\\
8.20820820820821	0.572668271767674\\
8.21821821821822	0.579524307437516\\
8.22822822822823	0.581292017593218\\
8.23823823823824	0.586267419136707\\
8.24824824824825	0.588944103273618\\
8.25825825825826	0.592810591570409\\
8.26826826826827	0.594809589562607\\
8.27827827827828	0.597233679476395\\
8.28828828828829	0.599235968321411\\
8.2982982982983	0.603755076275616\\
8.30830830830831	0.604981375555977\\
8.31831831831832	0.609980225371003\\
8.32832832832833	0.612012599071307\\
8.33833833833834	0.614054243918685\\
8.34834834834835	0.61591785083283\\
8.35835835835836	0.618621038746202\\
8.36836836836837	0.620324486121829\\
8.37837837837838	0.621676547448121\\
8.38838838838839	0.623552572297717\\
8.3983983983984	0.626652126984633\\
8.40840840840841	0.625838052647772\\
8.41841841841842	0.630187035357069\\
8.42842842842843	0.630462782115693\\
8.43843843843844	0.630381393361803\\
8.44844844844845	0.632407714589213\\
8.45845845845846	0.632498546381592\\
8.46846846846847	0.634393934140947\\
8.47847847847848	0.635535461765806\\
8.48848848848849	0.633559830551196\\
8.4984984984985	0.634941047081825\\
8.50850850850851	0.635850762176357\\
8.51851851851852	0.635448052693353\\
8.52852852852853	0.63637639634166\\
8.53853853853854	0.634149553818605\\
8.54854854854855	0.635197514515685\\
8.55855855855856	0.634487172849106\\
8.56856856856857	0.634040137758032\\
8.57857857857858	0.6336738153273\\
8.58858858858859	0.632638565372359\\
8.5985985985986	0.633704083511722\\
8.60860860860861	0.631058869912205\\
8.61861861861862	0.630819482444308\\
8.62862862862863	0.629218907754782\\
8.63863863863864	0.62873188673681\\
8.64864864864865	0.626824389390443\\
8.65865865865866	0.624267925838501\\
8.66866866866867	0.624397211546597\\
8.67867867867868	0.62151637661822\\
8.68868868868869	0.620805143687664\\
8.6986986986987	0.618816798270959\\
8.70870870870871	0.616594356347748\\
8.71871871871872	0.614357196020027\\
8.72872872872873	0.612765717482092\\
8.73873873873874	0.610340784838263\\
8.74874874874875	0.606413243624074\\
8.75875875875876	0.604030864155073\\
8.76876876876877	0.604026643226342\\
8.77877877877878	0.601163857892727\\
8.78878878878879	0.598209306459151\\
8.7987987987988	0.596194361753081\\
8.80880880880881	0.593303614735109\\
8.81881881881882	0.58756616907071\\
8.82882882882883	0.587243462725308\\
8.83883883883884	0.583109096342571\\
8.84884884884885	0.579724726943339\\
8.85885885885886	0.577988729861687\\
8.86886886886887	0.572333116657941\\
8.87887887887888	0.56963428114499\\
8.88888888888889	0.568551522126973\\
8.8988988988989	0.563993711197936\\
8.90890890890891	0.56002707607635\\
8.91891891891892	0.557554723278004\\
8.92892892892893	0.555227725278975\\
8.93893893893894	0.549569712026745\\
8.94894894894895	0.547315672881672\\
8.95895895895896	0.544212208493381\\
8.96896896896897	0.539032440018336\\
8.97897897897898	0.534723501339709\\
8.98898898898899	0.53206428888239\\
8.998998998999	0.527792745221217\\
9.00900900900901	0.523963108760721\\
9.01901901901902	0.521703264433434\\
9.02902902902903	0.518512314546828\\
9.03903903903904	0.512888432017\\
9.04904904904905	0.508861172939461\\
9.05905905905906	0.509498910660107\\
9.06906906906907	0.503152219945626\\
9.07907907907908	0.498892841562205\\
9.08908908908909	0.496871264963737\\
9.0990990990991	0.491634845489595\\
9.10910910910911	0.489619637291422\\
9.11911911911912	0.484079966219833\\
9.12912912912913	0.482166754337293\\
9.13913913913914	0.478553368678151\\
9.14914914914915	0.475286439379942\\
9.15915915915916	0.469498630820069\\
9.16916916916917	0.467417665244931\\
9.17917917917918	0.463360844491349\\
9.18918918918919	0.460270056466819\\
9.1991991991992	0.457781051454416\\
9.20920920920921	0.453091131967302\\
9.21921921921922	0.450228025021743\\
9.22922922922923	0.447918910026897\\
9.23923923923924	0.446336028051384\\
9.24924924924925	0.442333433155624\\
9.25925925925926	0.437457732526481\\
9.26926926926927	0.43482860872158\\
9.27927927927928	0.43059622879754\\
9.28928928928929	0.427485859179129\\
9.2992992992993	0.425368042040633\\
9.30930930930931	0.422681293524314\\
9.31931931931932	0.419255860477655\\
9.32932932932933	0.416463499515183\\
9.33933933933934	0.414005956483999\\
9.34934934934935	0.409260720775314\\
9.35935935935936	0.407404882500821\\
9.36936936936937	0.404965184538885\\
9.37937937937938	0.402879688077777\\
9.38938938938939	0.401329830558034\\
9.3993993993994	0.397922383875067\\
9.40940940940941	0.39668269103173\\
9.41941941941942	0.393730220440324\\
9.42942942942943	0.393704959520828\\
9.43943943943944	0.39215904884249\\
9.44944944944945	0.388777511022133\\
9.45945945945946	0.386479229090415\\
9.46946946946947	0.384335136708156\\
9.47947947947948	0.383961743128796\\
9.48948948948949	0.38217335226514\\
9.4994994994995	0.378880974105779\\
9.50950950950951	0.379322748495884\\
9.51951951951952	0.376391355947207\\
9.52952952952953	0.375274645491633\\
9.53953953953954	0.376099604446013\\
9.54954954954955	0.37499939214823\\
9.55955955955956	0.372359371557007\\
9.56956956956957	0.370377667320373\\
9.57957957957958	0.372563891003401\\
9.58958958958959	0.36983015640827\\
9.5995995995996	0.368355741159876\\
9.60960960960961	0.368734961599187\\
9.61961961961962	0.366785862465464\\
9.62962962962963	0.367876902208864\\
9.63963963963964	0.366918658091671\\
9.64964964964965	0.366329688458814\\
9.65965965965966	0.365217138405498\\
9.66966966966967	0.365854449905274\\
9.67967967967968	0.368180423964875\\
9.68968968968969	0.366754387939092\\
9.6996996996997	0.366654758451203\\
9.70970970970971	0.365888519009148\\
9.71971971971972	0.367071363773786\\
9.72972972972973	0.366450896951965\\
9.73973973973974	0.366305265411795\\
9.74974974974975	0.366758302804393\\
9.75975975975976	0.368227759342812\\
9.76976976976977	0.368350123575031\\
9.77977977977978	0.368526020070046\\
9.78978978978979	0.370646072233176\\
9.7997997997998	0.37205685974673\\
9.80980980980981	0.372639565184625\\
9.81981981981982	0.373357622956122\\
9.82982982982983	0.374269247193067\\
9.83983983983984	0.376441085519223\\
9.84984984984985	0.375414046089437\\
9.85985985985986	0.378182435141321\\
9.86986986986987	0.380101632238549\\
9.87987987987988	0.382865724056415\\
9.88988988988989	0.383533599633488\\
9.8998998998999	0.383700504916315\\
9.90990990990991	0.38533222686271\\
9.91991991991992	0.388235818303222\\
9.92992992992993	0.390538592544724\\
9.93993993993994	0.392575276122741\\
9.94994994994995	0.393974076833765\\
9.95995995995996	0.395797110522233\\
9.96996996996997	0.397166987644074\\
9.97997997997998	0.401363003984083\\
9.98998998998999	0.401881479299676\\
10	0.40532872696674\\
};
\addlegendentry{samples};


\addplot[area legend,solid,fill=mycolor4,opacity=3.000000e-01,draw=none]
table[row sep=crcr] {%
x	y\\
0	-1.65902451685932\\
0.01001001001001	-1.65572306576642\\
0.02002002002002	-1.65241227267804\\
0.03003003003003	-1.64909245643542\\
0.04004004004004	-1.64576393939361\\
0.0500500500500501	-1.64242704733195\\
0.0600600600600601	-1.63908210936222\\
0.0700700700700701	-1.63572945783468\\
0.0800800800800801	-1.63236942824181\\
0.0900900900900901	-1.62900235912001\\
0.1001001001001	-1.62562859194916\\
0.11011011011011	-1.62224847105013\\
0.12012012012012	-1.61886234348033\\
0.13013013013013	-1.61547055892723\\
0.14014014014014	-1.61207346960013\\
0.15015015015015	-1.60867143011992\\
0.16016016016016	-1.60526479740721\\
0.17017017017017	-1.60185393056862\\
0.18018018018018	-1.59843919078146\\
0.19019019019019	-1.59502094117682\\
0.2002002002002	-1.59159954672106\\
0.21021021021021	-1.58817537409589\\
0.22022022022022	-1.58474879157699\\
0.23023023023023	-1.58132016891131\\
0.24024024024024	-1.57788987719309\\
0.25025025025025	-1.57445828873869\\
0.26026026026026	-1.57102577696026\\
0.27027027027027	-1.5675927162384\\
0.28028028028028	-1.56415948179378\\
0.29029029029029	-1.56072644955789\\
0.3003003003003	-1.55729399604295\\
0.31031031031031	-1.55386249821107\\
0.32032032032032	-1.55043233334272\\
0.33033033033033	-1.54700387890464\\
0.34034034034034	-1.5435775124172\\
0.35035035035035	-1.54015361132132\\
0.36036036036036	-1.53673255284511\\
0.37037037037037	-1.53331471387015\\
0.38038038038038	-1.52990047079766\\
0.39039039039039	-1.52649019941451\\
0.4004004004004	-1.52308427475927\\
0.41041041041041	-1.51968307098828\\
0.42042042042042	-1.51628696124191\\
0.43043043043043	-1.51289631751106\\
0.44044044044044	-1.50951151050392\\
0.45045045045045	-1.50613290951322\\
0.46046046046046	-1.50276088228398\\
0.47047047047047	-1.49939579488171\\
0.48048048048048	-1.49603801156145\\
0.49049049049049	-1.49268789463746\\
0.500500500500501	-1.48934580435375\\
0.510510510510511	-1.48601209875552\\
0.520520520520521	-1.48268713356169\\
0.530530530530531	-1.47937126203839\\
0.540540540540541	-1.4760648348737\\
0.550550550550551	-1.47276820005362\\
0.560560560560561	-1.46948170273936\\
0.570570570570571	-1.46620568514603\\
0.580580580580581	-1.46294048642287\\
0.590590590590591	-1.459686442535\\
0.600600600600601	-1.45644388614679\\
0.610610610610611	-1.45321314650704\\
0.620620620620621	-1.44999454933585\\
0.630630630630631	-1.44678841671344\\
0.640640640640641	-1.44359506697085\\
0.650650650650651	-1.44041481458266\\
0.660660660660661	-1.43724797006183\\
0.670670670670671	-1.43409483985662\\
0.680680680680681	-1.43095572624975\\
0.690690690690691	-1.42783092725983\\
0.700700700700701	-1.42472073654515\\
0.710710710710711	-1.42162544330977\\
0.720720720720721	-1.4185453322122\\
0.730730730730731	-1.41548068327643\\
0.740740740740741	-1.41243177180569\\
0.750750750750751	-1.40939886829863\\
0.760760760760761	-1.40638223836837\\
0.770770770770771	-1.40338214266405\\
0.780780780780781	-1.40039883679529\\
0.790790790790791	-1.39743257125938\\
0.800800800800801	-1.3944835913713\\
0.810810810810811	-1.3915521371966\\
0.820820820820821	-1.38863844348728\\
0.830830830830831	-1.38574273962048\\
0.840840840840841	-1.38286524954023\\
0.850850850850851	-1.38000619170216\\
0.860860860860861	-1.37716577902129\\
0.870870870870871	-1.37434421882278\\
0.880880880880881	-1.37154171279581\\
0.890890890890891	-1.36875845695053\\
0.900900900900901	-1.36599464157812\\
0.910910910910911	-1.36325045121395\\
0.920920920920921	-1.36052606460385\\
0.930930930930931	-1.35782165467352\\
0.940940940940941	-1.35513738850111\\
0.950950950950951	-1.35247342729289\\
0.960960960960961	-1.34982992636205\\
0.970970970970971	-1.34720703511068\\
0.980980980980981	-1.34460489701481\\
0.990990990990991	-1.34202364961265\\
1.001001001001	-1.33946342449585\\
1.01101101101101	-1.33692434730393\\
1.02102102102102	-1.33440653772171\\
1.03103103103103	-1.33191010947985\\
1.04104104104104	-1.32943517035842\\
1.05105105105105	-1.32698182219349\\
1.06106106106106	-1.32455016088666\\
1.07107107107107	-1.32214027641759\\
1.08108108108108	-1.31975225285948\\
1.09109109109109	-1.31738616839738\\
1.1011011011011	-1.3150420953494\\
1.11111111111111	-1.3127201001908\\
1.12112112112112	-1.3104202435807\\
1.13113113113113	-1.30814258039174\\
1.14114114114114	-1.30588715974228\\
1.15115115115115	-1.30365402503134\\
1.16116116116116	-1.30144321397608\\
1.17117117117117	-1.29925475865194\\
1.18118118118118	-1.29708868553521\\
1.19119119119119	-1.29494501554803\\
1.2012012012012	-1.29282376410589\\
1.21121121121121	-1.29072494116743\\
1.22122122122122	-1.28864855128647\\
1.23123123123123	-1.28659459366637\\
1.24124124124124	-1.28456306221653\\
1.25125125125125	-1.28255394561094\\
1.26126126126126	-1.28056722734891\\
1.27127127127127	-1.27860288581763\\
1.28128128128128	-1.27666089435678\\
1.29129129129129	-1.27474122132496\\
1.3013013013013	-1.27284383016787\\
1.31131131131131	-1.27096867948828\\
1.32132132132132	-1.26911572311761\\
1.33133133133133	-1.26728491018916\\
1.34134134134134	-1.26547618521278\\
1.35135135135135	-1.26368948815105\\
1.36136136136136	-1.26192475449689\\
1.37137137137137	-1.26018191535235\\
1.38138138138138	-1.2584608975088\\
1.39139139139139	-1.25676162352818\\
1.4014014014014	-1.25508401182544\\
1.41141141141141	-1.25342797675195\\
1.42142142142142	-1.25179342867992\\
1.43143143143143	-1.25018027408773\\
1.44144144144144	-1.24858841564603\\
1.45145145145145	-1.24701775230464\\
1.46146146146146	-1.24546817938017\\
1.47147147147147	-1.24393958864419\\
1.48148148148148	-1.24243186841203\\
1.49149149149149	-1.24094490363198\\
1.5015015015015	-1.23947857597504\\
1.51151151151151	-1.23803276392488\\
1.52152152152152	-1.23660734286817\\
1.53153153153153	-1.23520218518518\\
1.54154154154154	-1.23381716034041\\
1.55155155155155	-1.23245213497346\\
1.56156156156156	-1.23110697298982\\
1.57157157157157	-1.22978153565175\\
1.58158158158158	-1.22847568166897\\
1.59159159159159	-1.22718926728927\\
1.6016016016016	-1.22592214638894\\
1.61161161161161	-1.22467417056287\\
1.62162162162162	-1.22344518921442\\
1.63163163163163	-1.22223504964488\\
1.64164164164164	-1.22104359714249\\
1.65165165165165	-1.21987067507111\\
1.66166166166166	-1.2187161249582\\
1.67167167167167	-1.21757978658237\\
1.68168168168168	-1.21646149806027\\
1.69169169169169	-1.21536109593281\\
1.7017017017017	-1.21427841525068\\
1.71171171171171	-1.21321328965914\\
1.72172172172172	-1.21216555148202\\
1.73173173173173	-1.21113503180484\\
1.74174174174174	-1.21012156055717\\
1.75175175175175	-1.20912496659397\\
1.76176176176176	-1.20814507777605\\
1.77177177177177	-1.20718172104956\\
1.78178178178178	-1.20623472252445\\
1.79179179179179	-1.20530390755191\\
1.8018018018018	-1.20438910080073\\
1.81181181181181	-1.20349012633253\\
1.82182182182182	-1.20260680767597\\
1.83183183183183	-1.20173896789971\\
1.84184184184184	-1.20088642968421\\
1.85185185185185	-1.2000490153924\\
1.86186186186186	-1.19922654713908\\
1.87187187187187	-1.19841884685908\\
1.88188188188188	-1.19762573637419\\
1.89189189189189	-1.19684703745886\\
1.9019019019019	-1.19608257190449\\
1.91191191191191	-1.19533216158257\\
1.92192192192192	-1.19459562850639\\
1.93193193193193	-1.19387279489152\\
1.94194194194194	-1.19316348321488\\
1.95195195195195	-1.19246751627251\\
1.96196196196196	-1.19178471723598\\
1.97197197197197	-1.19111490970746\\
1.98198198198198	-1.19045791777341\\
1.99199199199199	-1.1898135660569\\
2.002002002002	-1.18918167976859\\
2.01201201201201	-1.18856208475633\\
2.02202202202202	-1.18795460755339\\
2.03203203203203	-1.18735907542531\\
2.04204204204204	-1.18677531641541\\
2.05205205205205	-1.1862031593889\\
2.06206206206206	-1.18564243407568\\
2.07207207207207	-1.18509297111177\\
2.08208208208208	-1.18455460207933\\
2.09209209209209	-1.18402715954543\\
2.1021021021021	-1.1835104770994\\
2.11211211211211	-1.18300438938896\\
2.12212212212212	-1.1825087321549\\
2.13213213213213	-1.18202334226456\\
2.14214214214214	-1.18154805774395\\
2.15215215215215	-1.18108271780861\\
2.16216216216216	-1.18062716289321\\
2.17217217217217	-1.18018123467984\\
2.18218218218218	-1.1797447761251\\
2.19219219219219	-1.17931763148594\\
2.2022022022022	-1.17889964634425\\
2.21221221221221	-1.17849066763033\\
2.22222222222222	-1.17809054364503\\
2.23223223223223	-1.17769912408089\\
2.24224224224224	-1.17731626004198\\
2.25225225225225	-1.17694180406266\\
2.26226226226226	-1.17657561012521\\
2.27227227227227	-1.17621753367639\\
2.28228228228228	-1.17586743164284\\
2.29229229229229	-1.17552516244545\\
2.3023023023023	-1.17519058601272\\
2.31231231231231	-1.17486356379303\\
2.32232232232232	-1.17454395876588\\
2.33233233233233	-1.17423163545226\\
2.34234234234234	-1.17392645992387\\
2.35235235235235	-1.17362829981157\\
2.36236236236236	-1.17333702431271\\
2.37237237237237	-1.17305250419774\\
2.38238238238238	-1.17277461181577\\
2.39239239239239	-1.17250322109933\\
2.4024024024024	-1.17223820756826\\
2.41241241241241	-1.1719794483328\\
2.42242242242242	-1.17172682209581\\
2.43243243243243	-1.17148020915425\\
2.44244244244244	-1.17123949139986\\
2.45245245245245	-1.17100455231908\\
2.46246246246246	-1.1707752769923\\
2.47247247247247	-1.17055155209231\\
2.48248248248248	-1.17033326588215\\
2.49249249249249	-1.17012030821222\\
2.5025025025025	-1.16991257051675\\
2.51251251251251	-1.16970994580973\\
2.52252252252252	-1.16951232868006\\
2.53253253253253	-1.16931961528629\\
2.54254254254254	-1.16913170335065\\
2.55255255255255	-1.16894849215261\\
2.56256256256256	-1.16876988252187\\
2.57257257257257	-1.16859577683086\\
2.58258258258258	-1.16842607898672\\
2.59259259259259	-1.16826069442282\\
2.6026026026026	-1.1680995300898\\
2.61261261261261	-1.16794249444623\\
2.62262262262262	-1.16778949744875\\
2.63263263263263	-1.16764045054189\\
2.64264264264264	-1.16749526664748\\
2.65265265265265	-1.16735386015367\\
2.66266266266266	-1.16721614690359\\
2.67267267267267	-1.16708204418373\\
2.68268268268268	-1.16695147071195\\
2.69269269269269	-1.16682434662516\\
2.7027027027027	-1.16670059346682\\
2.71271271271271	-1.16658013417401\\
2.72272272272272	-1.16646289306443\\
2.73273273273273	-1.16634879582294\\
2.74274274274274	-1.16623776948809\\
2.75275275275275	-1.16612974243828\\
2.76276276276276	-1.16602464437777\\
2.77277277277277	-1.16592240632253\\
2.78278278278278	-1.16582296058589\\
2.79279279279279	-1.16572624076402\\
2.8028028028028	-1.1656321817213\\
2.81281281281281	-1.16554071957552\\
2.82282282282282	-1.16545179168296\\
2.83283283283283	-1.16536533662343\\
2.84284284284284	-1.16528129418507\\
2.85285285285285	-1.16519960534923\\
2.86286286286286	-1.16512021227513\\
2.87287287287287	-1.16504305828456\\
2.88288288288288	-1.16496808784645\\
2.89289289289289	-1.16489524656147\\
2.9029029029029	-1.1648244811465\\
2.91291291291291	-1.16475573941918\\
2.92292292292292	-1.16468897028235\\
2.93293293293293	-1.16462412370859\\
2.94294294294294	-1.16456115072461\\
2.95295295295295	-1.16450000339582\\
2.96296296296296	-1.1644406348108\\
2.97297297297297	-1.16438299906584\\
2.98298298298298	-1.16432705124953\\
2.99299299299299	-1.16427274742733\\
3.003003003003	-1.16422004462627\\
3.01301301301301	-1.16416890081966\\
3.02302302302302	-1.16411927491185\\
3.03303303303303	-1.16407112672309\\
3.04304304304304	-1.16402441697445\\
3.05305305305305	-1.16397910727281\\
3.06306306306306	-1.16393516009599\\
3.07307307307307	-1.16389253877786\\
3.08308308308308	-1.16385120749368\\
3.09309309309309	-1.16381113124545\\
3.1031031031031	-1.16377227584737\\
3.11311311311311	-1.16373460791148\\
3.12312312312312	-1.16369809483335\\
3.13313313313313	-1.16366270477788\\
3.14314314314314	-1.16362840666528\\
3.15315315315315	-1.16359517015715\\
3.16316316316316	-1.16356296564263\\
3.17317317317317	-1.16353176422483\\
3.18318318318318	-1.1635015377072\\
3.19319319319319	-1.16347225858022\\
3.2032032032032	-1.1634439000081\\
3.21321321321321	-1.16341643581571\\
3.22322322322322	-1.16338984047558\\
3.23323323323323	-1.16336408909513\\
3.24324324324324	-1.163339157404\\
3.25325325325325	-1.16331502174151\\
3.26326326326326	-1.16329165904431\\
3.27327327327327	-1.16326904683422\\
3.28328328328328	-1.16324716320612\\
3.29329329329329	-1.16322598681609\\
3.3033033033033	-1.16320549686969\\
3.31331331331331	-1.16318567311036\\
3.32332332332332	-1.16316649580801\\
3.33333333333333	-1.16314794574777\\
3.34334334334334	-1.16313000421892\\
3.35335335335335	-1.16311265300389\\
3.36336336336336	-1.16309587436757\\
3.37337337337337	-1.16307965104666\\
3.38338338338338	-1.1630639662392\\
3.39339339339339	-1.16304880359432\\
3.4034034034034	-1.16303414720211\\
3.41341341341341	-1.16301998158365\\
3.42342342342342	-1.16300629168118\\
3.43343343343343	-1.16299306284851\\
3.44344344344344	-1.16298028084149\\
3.45345345345345	-1.16296793180871\\
3.46346346346346	-1.16295600228233\\
3.47347347347347	-1.16294447916907\\
3.48348348348348	-1.16293334974134\\
3.49349349349349	-1.16292260162861\\
3.5035035035035	-1.16291222280877\\
3.51351351351351	-1.16290220159984\\
3.52352352352352	-1.16289252665168\\
3.53353353353353	-1.16288318693793\\
3.54354354354354	-1.16287417174807\\
3.55355355355355	-1.16286547067966\\
3.56356356356356	-1.16285707363067\\
3.57357357357357	-1.16284897079202\\
3.58358358358358	-1.16284115264026\\
3.59359359359359	-1.16283360993032\\
3.6036036036036	-1.1628263336885\\
3.61361361361361	-1.16281931520553\\
3.62362362362362	-1.16281254602983\\
3.63363363363363	-1.16280601796082\\
3.64364364364364	-1.16279972304249\\
3.65365365365365	-1.16279365355696\\
3.66366366366366	-1.16278780201831\\
3.67367367367367	-1.16278216116646\\
3.68368368368368	-1.16277672396117\\
3.69369369369369	-1.16277148357625\\
3.7037037037037	-1.1627664333938\\
3.71371371371371	-1.16276156699863\\
3.72372372372372	-1.16275687817281\\
3.73373373373373	-1.16275236089031\\
3.74374374374374	-1.16274800931176\\
3.75375375375375	-1.16274381777935\\
3.76376376376376	-1.16273978081182\\
3.77377377377377	-1.16273589309963\\
3.78378378378378	-1.16273214950011\\
3.79379379379379	-1.16272854503288\\
3.8038038038038	-1.16272507487525\\
3.81381381381381	-1.16272173435778\\
3.82382382382382	-1.16271851895998\\
3.83383383383383	-1.16271542430604\\
3.84384384384384	-1.1627124461607\\
3.85385385385385	-1.16270958042523\\
3.86386386386386	-1.16270682313349\\
3.87387387387387	-1.16270417044809\\
3.88388388388388	-1.16270161865665\\
3.89389389389389	-1.16269916416813\\
3.9039039039039	-1.1626968035093\\
3.91391391391391	-1.16269453332126\\
3.92392392392392	-1.16269235035602\\
3.93393393393393	-1.16269025147325\\
3.94394394394394	-1.16268823363707\\
3.95395395395395	-1.16268629391287\\
3.96396396396396	-1.1626844294643\\
3.97397397397397	-1.16268263755029\\
3.98398398398398	-1.16268091552217\\
3.99399399399399	-1.16267926082082\\
4.004004004004	-1.16267767097397\\
4.01401401401401	-1.16267614359351\\
4.02402402402402	-1.16267467637289\\
4.03403403403403	-1.16267326708461\\
4.04404404404404	-1.16267191357774\\
4.05405405405405	-1.16267061377555\\
4.06406406406406	-1.16266936567317\\
4.07407407407407	-1.16266816733533\\
4.08408408408408	-1.16266701689418\\
4.09409409409409	-1.16266591254712\\
4.1041041041041	-1.16266485255475\\
4.11411411411411	-1.16266383523883\\
4.12412412412412	-1.16266285898034\\
4.13413413413413	-1.16266192221756\\
4.14414414414414	-1.16266102344421\\
4.15415415415415	-1.16266016120766\\
4.16416416416416	-1.16265933410721\\
4.17417417417417	-1.16265854079235\\
4.18418418418418	-1.16265777996112\\
4.19419419419419	-1.16265705035855\\
4.2042042042042	-1.16265635077507\\
4.21421421421421	-1.16265568004504\\
4.22422422422422	-1.16265503704523\\
4.23423423423423	-1.16265442069348\\
4.24424424424424	-1.16265382994727\\
4.25425425425425	-1.16265326380242\\
4.26426426426426	-1.16265272129177\\
4.27427427427427	-1.16265220148396\\
4.28428428428428	-1.16265170348219\\
4.29429429429429	-1.16265122642304\\
4.3043043043043	-1.16265076947537\\
4.31431431431431	-1.16265033183918\\
4.32432432432432	-1.16264991274454\\
4.33433433433433	-1.16264951145058\\
4.34434434434434	-1.16264912724444\\
4.35435435435435	-1.16264875944036\\
4.36436436436436	-1.16264840737869\\
4.37437437437437	-1.16264807042501\\
4.38438438438438	-1.16264774796922\\
4.39439439439439	-1.16264743942472\\
4.4044044044044	-1.16264714422758\\
4.41441441441441	-1.16264686183569\\
4.42442442442442	-1.16264659172809\\
4.43443443443443	-1.16264633340412\\
4.44444444444444	-1.16264608638276\\
4.45445445445445	-1.16264585020194\\
4.46446446446446	-1.16264562441781\\
4.47447447447447	-1.16264540860416\\
4.48448448448448	-1.16264520235174\\
4.49449449449449	-1.16264500526767\\
4.5045045045045	-1.16264481697486\\
4.51451451451451	-1.16264463711146\\
4.52452452452452	-1.16264446533027\\
4.53453453453453	-1.16264430129824\\
4.54454454454454	-1.16264414469599\\
4.55455455455455	-1.16264399521728\\
4.56456456456456	-1.16264385256853\\
4.57457457457457	-1.16264371646842\\
4.58458458458458	-1.1626435866474\\
4.59459459459459	-1.1626434628473\\
4.6046046046046	-1.16264334482091\\
4.61461461461461	-1.16264323233159\\
4.62462462462462	-1.16264312515288\\
4.63463463463463	-1.16264302306819\\
4.64464464464464	-1.16264292587039\\
4.65465465465465	-1.16264283336152\\
4.66466466466466	-1.16264274535243\\
4.67467467467467	-1.16264266166252\\
4.68468468468468	-1.16264258211941\\
4.69469469469469	-1.16264250655867\\
4.7047047047047	-1.16264243482355\\
4.71471471471471	-1.16264236676471\\
4.72472472472472	-1.16264230223999\\
4.73473473473473	-1.16264224111416\\
4.74474474474474	-1.16264218325869\\
4.75475475475475	-1.16264212855154\\
4.76476476476476	-1.16264207687695\\
4.77477477477477	-1.16264202812528\\
4.78478478478478	-1.16264198219274\\
4.79479479479479	-1.16264193898131\\
4.8048048048048	-1.16264189839849\\
4.81481481481481	-1.16264186035722\\
4.82482482482482	-1.16264182477564\\
4.83483483483483	-1.16264179157703\\
4.84484484484484	-1.16264176068964\\
4.85485485485485	-1.16264173204656\\
4.86486486486486	-1.16264170558562\\
4.87487487487487	-1.16264168124926\\
4.88488488488488	-1.16264165898446\\
4.89489489489489	-1.16264163874262\\
4.9049049049049	-1.16264162047951\\
4.91491491491491	-1.16264160415512\\
4.92492492492492	-1.16264158973369\\
4.93493493493493	-1.16264157718355\\
4.94494494494494	-1.16264156647713\\
4.95495495495495	-1.1626415575909\\
4.96496496496496	-1.16264155050529\\
4.97497497497497	-1.1626415452047\\
4.98498498498498	-1.16264154167745\\
4.99499499499499	-1.16264153991577\\
5.00500500500501	-1.16264153991577\\
5.01501501501502	-1.16264154167745\\
5.02502502502503	-1.1626415452047\\
5.03503503503504	-1.16264155050529\\
5.04504504504505	-1.1626415575909\\
5.05505505505506	-1.16264156647713\\
5.06506506506507	-1.16264157718355\\
5.07507507507508	-1.16264158973369\\
5.08508508508509	-1.16264160415512\\
5.0950950950951	-1.16264162047951\\
5.10510510510511	-1.16264163874262\\
5.11511511511512	-1.16264165898446\\
5.12512512512513	-1.16264168124926\\
5.13513513513514	-1.16264170558562\\
5.14514514514515	-1.16264173204656\\
5.15515515515516	-1.16264176068964\\
5.16516516516517	-1.16264179157703\\
5.17517517517518	-1.16264182477564\\
5.18518518518519	-1.16264186035722\\
5.1951951951952	-1.16264189839849\\
5.20520520520521	-1.16264193898131\\
5.21521521521522	-1.16264198219274\\
5.22522522522523	-1.16264202812528\\
5.23523523523524	-1.16264207687695\\
5.24524524524525	-1.16264212855154\\
5.25525525525526	-1.16264218325869\\
5.26526526526527	-1.16264224111416\\
5.27527527527528	-1.16264230223999\\
5.28528528528529	-1.16264236676471\\
5.2952952952953	-1.16264243482355\\
5.30530530530531	-1.16264250655867\\
5.31531531531532	-1.16264258211941\\
5.32532532532533	-1.16264266166252\\
5.33533533533534	-1.16264274535243\\
5.34534534534535	-1.16264283336152\\
5.35535535535536	-1.16264292587039\\
5.36536536536537	-1.16264302306819\\
5.37537537537538	-1.16264312515288\\
5.38538538538539	-1.16264323233159\\
5.3953953953954	-1.16264334482091\\
5.40540540540541	-1.1626434628473\\
5.41541541541542	-1.1626435866474\\
5.42542542542543	-1.16264371646842\\
5.43543543543544	-1.16264385256853\\
5.44544544544545	-1.16264399521728\\
5.45545545545546	-1.16264414469599\\
5.46546546546547	-1.16264430129824\\
5.47547547547548	-1.16264446533027\\
5.48548548548549	-1.16264463711146\\
5.4954954954955	-1.16264481697486\\
5.50550550550551	-1.16264500526767\\
5.51551551551552	-1.16264520235174\\
5.52552552552553	-1.16264540860416\\
5.53553553553554	-1.16264562441781\\
5.54554554554555	-1.16264585020194\\
5.55555555555556	-1.16264608638276\\
5.56556556556557	-1.16264633340412\\
5.57557557557558	-1.16264659172809\\
5.58558558558559	-1.16264686183569\\
5.5955955955956	-1.16264714422758\\
5.60560560560561	-1.16264743942472\\
5.61561561561562	-1.16264774796922\\
5.62562562562563	-1.16264807042501\\
5.63563563563564	-1.16264840737869\\
5.64564564564565	-1.16264875944036\\
5.65565565565566	-1.16264912724444\\
5.66566566566567	-1.16264951145058\\
5.67567567567568	-1.16264991274454\\
5.68568568568569	-1.16265033183918\\
5.6956956956957	-1.16265076947537\\
5.70570570570571	-1.16265122642304\\
5.71571571571572	-1.16265170348219\\
5.72572572572573	-1.16265220148396\\
5.73573573573574	-1.16265272129177\\
5.74574574574575	-1.16265326380242\\
5.75575575575576	-1.16265382994727\\
5.76576576576577	-1.16265442069348\\
5.77577577577578	-1.16265503704523\\
5.78578578578579	-1.16265568004504\\
5.7957957957958	-1.16265635077507\\
5.80580580580581	-1.16265705035855\\
5.81581581581582	-1.16265777996112\\
5.82582582582583	-1.16265854079235\\
5.83583583583584	-1.16265933410721\\
5.84584584584585	-1.16266016120766\\
5.85585585585586	-1.16266102344421\\
5.86586586586587	-1.16266192221756\\
5.87587587587588	-1.16266285898034\\
5.88588588588589	-1.16266383523883\\
5.8958958958959	-1.16266485255475\\
5.90590590590591	-1.16266591254712\\
5.91591591591592	-1.16266701689418\\
5.92592592592593	-1.16266816733533\\
5.93593593593594	-1.16266936567317\\
5.94594594594595	-1.16267061377555\\
5.95595595595596	-1.16267191357774\\
5.96596596596597	-1.16267326708461\\
5.97597597597598	-1.16267467637289\\
5.98598598598599	-1.16267614359351\\
5.995995995996	-1.16267767097397\\
6.00600600600601	-1.16267926082082\\
6.01601601601602	-1.16268091552217\\
6.02602602602603	-1.16268263755029\\
6.03603603603604	-1.1626844294643\\
6.04604604604605	-1.16268629391287\\
6.05605605605606	-1.16268823363707\\
6.06606606606607	-1.16269025147325\\
6.07607607607608	-1.16269235035602\\
6.08608608608609	-1.16269453332126\\
6.0960960960961	-1.1626968035093\\
6.10610610610611	-1.16269916416813\\
6.11611611611612	-1.16270161865665\\
6.12612612612613	-1.16270417044809\\
6.13613613613614	-1.16270682313349\\
6.14614614614615	-1.16270958042523\\
6.15615615615616	-1.1627124461607\\
6.16616616616617	-1.16271542430604\\
6.17617617617618	-1.16271851895998\\
6.18618618618619	-1.16272173435778\\
6.1961961961962	-1.16272507487525\\
6.20620620620621	-1.16272854503288\\
6.21621621621622	-1.16273214950011\\
6.22622622622623	-1.16273589309963\\
6.23623623623624	-1.16273978081182\\
6.24624624624625	-1.16274381777935\\
6.25625625625626	-1.16274800931176\\
6.26626626626627	-1.16275236089031\\
6.27627627627628	-1.16275687817281\\
6.28628628628629	-1.16276156699863\\
6.2962962962963	-1.1627664333938\\
6.30630630630631	-1.16277148357625\\
6.31631631631632	-1.16277672396117\\
6.32632632632633	-1.16278216116646\\
6.33633633633634	-1.16278780201831\\
6.34634634634635	-1.16279365355696\\
6.35635635635636	-1.16279972304249\\
6.36636636636637	-1.16280601796082\\
6.37637637637638	-1.16281254602983\\
6.38638638638639	-1.16281931520553\\
6.3963963963964	-1.1628263336885\\
6.40640640640641	-1.16283360993032\\
6.41641641641642	-1.16284115264026\\
6.42642642642643	-1.16284897079202\\
6.43643643643644	-1.16285707363067\\
6.44644644644645	-1.16286547067966\\
6.45645645645646	-1.16287417174807\\
6.46646646646647	-1.16288318693793\\
6.47647647647648	-1.16289252665168\\
6.48648648648649	-1.16290220159984\\
6.4964964964965	-1.16291222280877\\
6.50650650650651	-1.16292260162861\\
6.51651651651652	-1.16293334974134\\
6.52652652652653	-1.16294447916907\\
6.53653653653654	-1.16295600228233\\
6.54654654654655	-1.16296793180871\\
6.55655655655656	-1.16298028084149\\
6.56656656656657	-1.16299306284851\\
6.57657657657658	-1.16300629168118\\
6.58658658658659	-1.16301998158365\\
6.5965965965966	-1.16303414720211\\
6.60660660660661	-1.16304880359432\\
6.61661661661662	-1.1630639662392\\
6.62662662662663	-1.16307965104666\\
6.63663663663664	-1.16309587436757\\
6.64664664664665	-1.16311265300389\\
6.65665665665666	-1.16313000421892\\
6.66666666666667	-1.16314794574777\\
6.67667667667668	-1.16316649580801\\
6.68668668668669	-1.16318567311036\\
6.6966966966967	-1.16320549686969\\
6.70670670670671	-1.16322598681609\\
6.71671671671672	-1.16324716320612\\
6.72672672672673	-1.16326904683422\\
6.73673673673674	-1.16329165904431\\
6.74674674674675	-1.16331502174151\\
6.75675675675676	-1.163339157404\\
6.76676676676677	-1.16336408909513\\
6.77677677677678	-1.16338984047558\\
6.78678678678679	-1.16341643581571\\
6.7967967967968	-1.1634439000081\\
6.80680680680681	-1.16347225858022\\
6.81681681681682	-1.1635015377072\\
6.82682682682683	-1.16353176422483\\
6.83683683683684	-1.16356296564263\\
6.84684684684685	-1.16359517015715\\
6.85685685685686	-1.16362840666528\\
6.86686686686687	-1.16366270477788\\
6.87687687687688	-1.16369809483335\\
6.88688688688689	-1.16373460791148\\
6.8968968968969	-1.16377227584737\\
6.90690690690691	-1.16381113124545\\
6.91691691691692	-1.16385120749368\\
6.92692692692693	-1.16389253877786\\
6.93693693693694	-1.16393516009599\\
6.94694694694695	-1.16397910727281\\
6.95695695695696	-1.16402441697445\\
6.96696696696697	-1.16407112672309\\
6.97697697697698	-1.16411927491185\\
6.98698698698699	-1.16416890081966\\
6.996996996997	-1.16422004462627\\
7.00700700700701	-1.16427274742733\\
7.01701701701702	-1.16432705124953\\
7.02702702702703	-1.16438299906584\\
7.03703703703704	-1.1644406348108\\
7.04704704704705	-1.16450000339582\\
7.05705705705706	-1.16456115072461\\
7.06706706706707	-1.16462412370859\\
7.07707707707708	-1.16468897028235\\
7.08708708708709	-1.16475573941918\\
7.0970970970971	-1.1648244811465\\
7.10710710710711	-1.16489524656147\\
7.11711711711712	-1.16496808784645\\
7.12712712712713	-1.16504305828456\\
7.13713713713714	-1.16512021227513\\
7.14714714714715	-1.16519960534923\\
7.15715715715716	-1.16528129418507\\
7.16716716716717	-1.16536533662343\\
7.17717717717718	-1.16545179168296\\
7.18718718718719	-1.16554071957552\\
7.1971971971972	-1.1656321817213\\
7.20720720720721	-1.16572624076402\\
7.21721721721722	-1.16582296058589\\
7.22722722722723	-1.16592240632253\\
7.23723723723724	-1.16602464437777\\
7.24724724724725	-1.16612974243828\\
7.25725725725726	-1.16623776948809\\
7.26726726726727	-1.16634879582294\\
7.27727727727728	-1.16646289306443\\
7.28728728728729	-1.16658013417401\\
7.2972972972973	-1.16670059346682\\
7.30730730730731	-1.16682434662516\\
7.31731731731732	-1.16695147071195\\
7.32732732732733	-1.16708204418373\\
7.33733733733734	-1.16721614690359\\
7.34734734734735	-1.16735386015367\\
7.35735735735736	-1.16749526664748\\
7.36736736736737	-1.16764045054189\\
7.37737737737738	-1.16778949744875\\
7.38738738738739	-1.16794249444623\\
7.3973973973974	-1.1680995300898\\
7.40740740740741	-1.16826069442282\\
7.41741741741742	-1.16842607898672\\
7.42742742742743	-1.16859577683086\\
7.43743743743744	-1.16876988252187\\
7.44744744744745	-1.16894849215261\\
7.45745745745746	-1.16913170335065\\
7.46746746746747	-1.16931961528629\\
7.47747747747748	-1.16951232868006\\
7.48748748748749	-1.16970994580973\\
7.4974974974975	-1.16991257051675\\
7.50750750750751	-1.17012030821222\\
7.51751751751752	-1.17033326588215\\
7.52752752752753	-1.17055155209231\\
7.53753753753754	-1.1707752769923\\
7.54754754754755	-1.17100455231908\\
7.55755755755756	-1.17123949139986\\
7.56756756756757	-1.17148020915425\\
7.57757757757758	-1.17172682209581\\
7.58758758758759	-1.1719794483328\\
7.5975975975976	-1.17223820756826\\
7.60760760760761	-1.17250322109933\\
7.61761761761762	-1.17277461181577\\
7.62762762762763	-1.17305250419774\\
7.63763763763764	-1.17333702431271\\
7.64764764764765	-1.17362829981157\\
7.65765765765766	-1.17392645992387\\
7.66766766766767	-1.17423163545226\\
7.67767767767768	-1.17454395876588\\
7.68768768768769	-1.17486356379303\\
7.6976976976977	-1.17519058601272\\
7.70770770770771	-1.17552516244545\\
7.71771771771772	-1.17586743164284\\
7.72772772772773	-1.17621753367639\\
7.73773773773774	-1.17657561012521\\
7.74774774774775	-1.17694180406266\\
7.75775775775776	-1.17731626004198\\
7.76776776776777	-1.17769912408089\\
7.77777777777778	-1.17809054364503\\
7.78778778778779	-1.17849066763033\\
7.7977977977978	-1.17889964634425\\
7.80780780780781	-1.17931763148594\\
7.81781781781782	-1.1797447761251\\
7.82782782782783	-1.18018123467984\\
7.83783783783784	-1.18062716289321\\
7.84784784784785	-1.18108271780861\\
7.85785785785786	-1.18154805774395\\
7.86786786786787	-1.18202334226456\\
7.87787787787788	-1.1825087321549\\
7.88788788788789	-1.18300438938896\\
7.8978978978979	-1.1835104770994\\
7.90790790790791	-1.18402715954543\\
7.91791791791792	-1.18455460207933\\
7.92792792792793	-1.18509297111177\\
7.93793793793794	-1.18564243407568\\
7.94794794794795	-1.1862031593889\\
7.95795795795796	-1.18677531641541\\
7.96796796796797	-1.18735907542531\\
7.97797797797798	-1.18795460755339\\
7.98798798798799	-1.18856208475633\\
7.997997997998	-1.18918167976859\\
8.00800800800801	-1.1898135660569\\
8.01801801801802	-1.19045791777341\\
8.02802802802803	-1.19111490970746\\
8.03803803803804	-1.19178471723598\\
8.04804804804805	-1.19246751627251\\
8.05805805805806	-1.19316348321488\\
8.06806806806807	-1.19387279489152\\
8.07807807807808	-1.19459562850639\\
8.08808808808809	-1.19533216158257\\
8.0980980980981	-1.19608257190449\\
8.10810810810811	-1.19684703745886\\
8.11811811811812	-1.19762573637419\\
8.12812812812813	-1.19841884685908\\
8.13813813813814	-1.19922654713908\\
8.14814814814815	-1.2000490153924\\
8.15815815815816	-1.20088642968421\\
8.16816816816817	-1.20173896789971\\
8.17817817817818	-1.20260680767597\\
8.18818818818819	-1.20349012633253\\
8.1981981981982	-1.20438910080073\\
8.20820820820821	-1.20530390755191\\
8.21821821821822	-1.20623472252445\\
8.22822822822823	-1.20718172104956\\
8.23823823823824	-1.20814507777605\\
8.24824824824825	-1.20912496659397\\
8.25825825825826	-1.21012156055717\\
8.26826826826827	-1.21113503180484\\
8.27827827827828	-1.21216555148202\\
8.28828828828829	-1.21321328965914\\
8.2982982982983	-1.21427841525068\\
8.30830830830831	-1.21536109593281\\
8.31831831831832	-1.21646149806027\\
8.32832832832833	-1.21757978658237\\
8.33833833833834	-1.2187161249582\\
8.34834834834835	-1.21987067507111\\
8.35835835835836	-1.22104359714249\\
8.36836836836837	-1.22223504964488\\
8.37837837837838	-1.22344518921442\\
8.38838838838839	-1.22467417056287\\
8.3983983983984	-1.22592214638894\\
8.40840840840841	-1.22718926728927\\
8.41841841841842	-1.22847568166897\\
8.42842842842843	-1.22978153565175\\
8.43843843843844	-1.23110697298982\\
8.44844844844845	-1.23245213497346\\
8.45845845845846	-1.23381716034041\\
8.46846846846847	-1.23520218518518\\
8.47847847847848	-1.23660734286817\\
8.48848848848849	-1.23803276392488\\
8.4984984984985	-1.23947857597504\\
8.50850850850851	-1.24094490363198\\
8.51851851851852	-1.24243186841203\\
8.52852852852853	-1.24393958864419\\
8.53853853853854	-1.24546817938017\\
8.54854854854855	-1.24701775230464\\
8.55855855855856	-1.24858841564603\\
8.56856856856857	-1.25018027408773\\
8.57857857857858	-1.25179342867992\\
8.58858858858859	-1.25342797675195\\
8.5985985985986	-1.25508401182544\\
8.60860860860861	-1.25676162352818\\
8.61861861861862	-1.2584608975088\\
8.62862862862863	-1.26018191535235\\
8.63863863863864	-1.26192475449689\\
8.64864864864865	-1.26368948815105\\
8.65865865865866	-1.26547618521278\\
8.66866866866867	-1.26728491018916\\
8.67867867867868	-1.26911572311761\\
8.68868868868869	-1.27096867948828\\
8.6986986986987	-1.27284383016787\\
8.70870870870871	-1.27474122132496\\
8.71871871871872	-1.27666089435678\\
8.72872872872873	-1.27860288581763\\
8.73873873873874	-1.28056722734891\\
8.74874874874875	-1.28255394561094\\
8.75875875875876	-1.28456306221653\\
8.76876876876877	-1.28659459366637\\
8.77877877877878	-1.28864855128647\\
8.78878878878879	-1.29072494116743\\
8.7987987987988	-1.29282376410589\\
8.80880880880881	-1.29494501554803\\
8.81881881881882	-1.29708868553521\\
8.82882882882883	-1.29925475865194\\
8.83883883883884	-1.30144321397608\\
8.84884884884885	-1.30365402503134\\
8.85885885885886	-1.30588715974229\\
8.86886886886887	-1.30814258039174\\
8.87887887887888	-1.3104202435807\\
8.88888888888889	-1.3127201001908\\
8.8988988988989	-1.3150420953494\\
8.90890890890891	-1.31738616839738\\
8.91891891891892	-1.31975225285948\\
8.92892892892893	-1.32214027641759\\
8.93893893893894	-1.32455016088666\\
8.94894894894895	-1.32698182219349\\
8.95895895895896	-1.32943517035842\\
8.96896896896897	-1.33191010947985\\
8.97897897897898	-1.33440653772171\\
8.98898898898899	-1.33692434730393\\
8.998998998999	-1.33946342449585\\
9.00900900900901	-1.34202364961265\\
9.01901901901902	-1.34460489701481\\
9.02902902902903	-1.34720703511068\\
9.03903903903904	-1.34982992636205\\
9.04904904904905	-1.35247342729289\\
9.05905905905906	-1.35513738850111\\
9.06906906906907	-1.35782165467352\\
9.07907907907908	-1.36052606460385\\
9.08908908908909	-1.36325045121395\\
9.0990990990991	-1.36599464157812\\
9.10910910910911	-1.36875845695053\\
9.11911911911912	-1.37154171279581\\
9.12912912912913	-1.37434421882278\\
9.13913913913914	-1.37716577902129\\
9.14914914914915	-1.38000619170216\\
9.15915915915916	-1.38286524954023\\
9.16916916916917	-1.38574273962048\\
9.17917917917918	-1.38863844348728\\
9.18918918918919	-1.3915521371966\\
9.1991991991992	-1.3944835913713\\
9.20920920920921	-1.39743257125938\\
9.21921921921922	-1.40039883679529\\
9.22922922922923	-1.40338214266405\\
9.23923923923924	-1.40638223836837\\
9.24924924924925	-1.40939886829863\\
9.25925925925926	-1.41243177180569\\
9.26926926926927	-1.41548068327643\\
9.27927927927928	-1.4185453322122\\
9.28928928928929	-1.42162544330977\\
9.2992992992993	-1.42472073654515\\
9.30930930930931	-1.42783092725983\\
9.31931931931932	-1.43095572624975\\
9.32932932932933	-1.43409483985662\\
9.33933933933934	-1.43724797006183\\
9.34934934934935	-1.44041481458266\\
9.35935935935936	-1.44359506697085\\
9.36936936936937	-1.44678841671345\\
9.37937937937938	-1.44999454933585\\
9.38938938938939	-1.45321314650704\\
9.3993993993994	-1.45644388614679\\
9.40940940940941	-1.459686442535\\
9.41941941941942	-1.46294048642287\\
9.42942942942943	-1.46620568514603\\
9.43943943943944	-1.46948170273936\\
9.44944944944945	-1.47276820005362\\
9.45945945945946	-1.4760648348737\\
9.46946946946947	-1.47937126203839\\
9.47947947947948	-1.48268713356169\\
9.48948948948949	-1.48601209875552\\
9.4994994994995	-1.48934580435375\\
9.50950950950951	-1.49268789463746\\
9.51951951951952	-1.49603801156145\\
9.52952952952953	-1.49939579488171\\
9.53953953953954	-1.50276088228398\\
9.54954954954955	-1.50613290951322\\
9.55955955955956	-1.50951151050392\\
9.56956956956957	-1.51289631751106\\
9.57957957957958	-1.51628696124191\\
9.58958958958959	-1.51968307098828\\
9.5995995995996	-1.52308427475927\\
9.60960960960961	-1.52649019941451\\
9.61961961961962	-1.52990047079766\\
9.62962962962963	-1.53331471387015\\
9.63963963963964	-1.53673255284511\\
9.64964964964965	-1.54015361132132\\
9.65965965965966	-1.5435775124172\\
9.66966966966967	-1.54700387890464\\
9.67967967967968	-1.55043233334272\\
9.68968968968969	-1.55386249821107\\
9.6996996996997	-1.55729399604295\\
9.70970970970971	-1.56072644955789\\
9.71971971971972	-1.56415948179378\\
9.72972972972973	-1.5675927162384\\
9.73973973973974	-1.57102577696026\\
9.74974974974975	-1.57445828873869\\
9.75975975975976	-1.57788987719309\\
9.76976976976977	-1.58132016891131\\
9.77977977977978	-1.58474879157699\\
9.78978978978979	-1.58817537409589\\
9.7997997997998	-1.59159954672106\\
9.80980980980981	-1.59502094117682\\
9.81981981981982	-1.59843919078146\\
9.82982982982983	-1.60185393056862\\
9.83983983983984	-1.60526479740721\\
9.84984984984985	-1.60867143011992\\
9.85985985985986	-1.61207346960013\\
9.86986986986987	-1.61547055892723\\
9.87987987987988	-1.61886234348033\\
9.88988988988989	-1.62224847105013\\
9.8998998998999	-1.62562859194916\\
9.90990990990991	-1.62900235912001\\
9.91991991991992	-1.63236942824181\\
9.92992992992993	-1.63572945783468\\
9.93993993993994	-1.63908210936222\\
9.94994994994995	-1.64242704733195\\
9.95995995995996	-1.64576393939361\\
9.96996996996997	-1.64909245643542\\
9.97997997997998	-1.65241227267804\\
9.98998998998999	-1.65572306576642\\
10	-1.65902451685932\\
10	2.20237783960136\\
9.98998998998999	2.20341598798152\\
9.97997997997998	2.20444435956244\\
9.96996996996997	2.2054628385123\\
9.95995995995996	2.20647131277441\\
9.94994994994995	2.20746967410815\\
9.93993993993994	2.20845781812748\\
9.92992992992993	2.20943564433695\\
9.91991991991992	2.21040305616522\\
9.90990990990991	2.21135996099603\\
9.8998998998999	2.21230627019662\\
9.88988988988989	2.21324189914361\\
9.87987987987988	2.21416676724629\\
9.86986986986987	2.2150807979673\\
9.85985985985986	2.21598391884069\\
9.84984984984985	2.21687606148746\\
9.83983983983984	2.21775716162834\\
9.82982982982983	2.21862715909409\\
9.81981981981982	2.21948599783308\\
9.80980980980981	2.22033362591626\\
9.7997997997998	2.22116999553961\\
9.78978978978979	2.22199506302383\\
9.77977977977978	2.22280878881152\\
9.76976976976977	2.22361113746178\\
9.75975975975976	2.22440207764218\\
9.74974974974975	2.22518158211818\\
9.73973973973974	2.22594962774\\
9.72972972972973	2.22670619542704\\
9.71971971971972	2.22745127014963\\
9.70970970970971	2.22818484090851\\
9.6996996996997	2.22890690071165\\
9.68968968968969	2.22961744654883\\
9.67967967967968	2.23031647936369\\
9.66966966966967	2.23100400402346\\
9.65965965965966	2.23168002928645\\
9.64964964964965	2.23234456776709\\
9.63963963963964	2.23299763589886\\
9.62962962962963	2.23363925389497\\
9.61961961961962	2.23426944570681\\
9.60960960960961	2.23488823898039\\
9.5995995995996	2.23549566501065\\
9.58958958958959	2.2360917586937\\
9.57957957957958	2.23667655847716\\
9.56956956956957	2.23725010630857\\
9.55955955955956	2.23781244758184\\
9.54954954954955	2.23836363108202\\
9.53953953953954	2.2389037089282\\
9.52952952952953	2.23943273651472\\
9.51951951951952	2.23995077245083\\
9.50950950950951	2.24045787849862\\
9.4994994994995	2.24095411950958\\
9.48948948948949	2.24143956335956\\
9.47947947947948	2.24191428088241\\
9.46946946946947	2.24237834580232\\
9.45945945945946	2.24283183466477\\
9.44944944944945	2.24327482676643\\
9.43943943943944	2.24370740408378\\
9.42942942942943	2.24412965120079\\
9.41941941941942	2.2445416552355\\
9.40940940940941	2.24494350576573\\
9.3993993993994	2.24533529475391\\
9.38938938938939	2.24571711647111\\
9.37937937937938	2.24608906742035\\
9.36936936936937	2.24645124625926\\
9.35935935935936	2.24680375372217\\
9.34934934934935	2.24714669254163\\
9.33933933933934	2.24748016736959\\
9.32932932932933	2.24780428469807\\
9.31931931931932	2.24811915277967\\
9.30930930930931	2.24842488154771\\
9.2992992992993	2.24872158253629\\
9.28928928928929	2.24900936880024\\
9.27927927927928	2.249288354835\\
9.26926926926927	2.24955865649656\\
9.25925925925926	2.24982039092153\\
9.24924924924925	2.25007367644732\\
9.23923923923924	2.25031863253262\\
9.22922922922923	2.25055537967812\\
9.21921921921922	2.25078403934761\\
9.20920920920921	2.25100473388954\\
9.1991991991992	2.25121758645901\\
9.18918918918919	2.25142272094033\\
9.17917917917918	2.25162026187018\\
9.16916916916917	2.25181033436144\\
9.15915915915916	2.25199306402773\\
9.14914914914915	2.25216857690871\\
9.13913913913914	2.25233699939622\\
9.12912912912913	2.25249845816129\\
9.11911911911912	2.25265308008209\\
9.10910910910911	2.25280099217283\\
9.0990990990991	2.25294232151371\\
9.08908908908909	2.25307719518192\\
9.07907907907908	2.25320574018377\\
9.06906906906907	2.25332808338796\\
9.05905905905906	2.25344435146003\\
9.04904904904905	2.25355467079812\\
9.03903903903904	2.25365916746986\\
9.02902902902903	2.25375796715073\\
9.01901901901902	2.25385119506361\\
9.00900900900901	2.25393897591982\\
8.998998998999	2.25402143386148\\
8.98898898898899	2.25409869240529\\
8.97897897897898	2.25417087438787\\
8.96896896896897	2.25423810191241\\
8.95895895895896	2.25430049629697\\
8.94894894894895	2.25435817802418\\
8.93893893893894	2.25441126669256\\
8.92892892892893	2.25445988096932\\
8.91891891891892	2.25450413854479\\
8.90890890890891	2.25454415608837\\
8.8988988988989	2.25458004920609\\
8.88888888888889	2.25461193239976\\
8.87887887887888	2.25463991902772\\
8.86886886886887	2.25466412126723\\
8.85885885885886	2.25468465007839\\
8.84884884884885	2.2547016151697\\
8.83883883883884	2.25471512496533\\
8.82882882882883	2.2547252865738\\
8.81881881881882	2.25473220575845\\
8.80880880880881	2.25473598690938\\
8.7987987987988	2.25473673301702\\
8.78878878878879	2.25473454564732\\
8.77877877877878	2.25472952491844\\
8.76876876876877	2.25472176947901\\
8.75875875875876	2.25471137648801\\
8.74874874874875	2.25469844159606\\
8.73873873873874	2.2546830589283\\
8.72872872872873	2.25466532106876\\
8.71871871871872	2.25464531904617\\
8.70870870870871	2.25462314232125\\
8.6986986986987	2.25459887877545\\
8.68868868868869	2.25457261470105\\
8.67867867867868	2.25454443479272\\
8.66866866866867	2.25451442214035\\
8.65865865865866	2.25448265822335\\
8.64864864864865	2.25444922290611\\
8.63863863863864	2.25441419443489\\
8.62862862862863	2.25437764943585\\
8.61861861861862	2.25433966291439\\
8.60860860860861	2.25430030825565\\
8.5985985985986	2.2542596572262\\
8.58858858858859	2.25421777997683\\
8.57857857857858	2.25417474504651\\
8.56856856856857	2.2541306193674\\
8.55855855855856	2.25408546827087\\
8.54854854854855	2.25403935549456\\
8.53853853853854	2.25399234319048\\
8.52852852852853	2.25394449193393\\
8.51851851851852	2.25389586073346\\
8.50850850850851	2.25384650704163\\
8.4984984984985	2.25379648676663\\
8.48848848848849	2.25374585428479\\
8.47847847847848	2.25369466245373\\
8.46846846846847	2.2536429626264\\
8.45845845845846	2.25359080466576\\
8.44844844844845	2.25353823696013\\
8.43843843843844	2.25348530643924\\
8.42842842842843	2.25343205859086\\
8.41841841841842	2.25337853747801\\
8.40840840840841	2.25332478575676\\
8.3983983983984	2.2532708446945\\
8.38838838838839	2.25321675418872\\
8.37837837837838	2.25316255278625\\
8.36836836836837	2.25310827770291\\
8.35835835835836	2.25305396484357\\
8.34834834834835	2.25299964882256\\
8.33833833833834	2.25294536298443\\
8.32832832832833	2.25289113942501\\
8.31831831831832	2.25283700901273\\
8.30830830830831	2.25278300141022\\
8.2982982982983	2.25272914509607\\
8.28828828828829	2.25267546738692\\
8.27827827827828	2.25262199445953\\
8.26826826826827	2.25256875137317\\
8.25825825825826	2.25251576209201\\
8.24824824824825	2.25246304950763\\
8.23823823823824	2.25241063546167\\
8.22822822822823	2.25235854076844\\
8.21821821821822	2.25230678523758\\
8.20820820820821	2.25225538769676\\
8.1981981981982	2.25220436601431\\
8.18818818818819	2.25215373712187\\
8.17817817817818	2.25210351703691\\
8.16816816816817	2.25205372088527\\
8.15815815815816	2.25200436292351\\
8.14814814814815	2.25195545656124\\
8.13813813813814	2.25190701438321\\
8.12812812812813	2.25185904817141\\
8.11811811811812	2.25181156892684\\
8.10810810810811	2.25176458689128\\
8.0980980980981	2.25171811156873\\
8.08808808808809	2.2516721517467\\
8.07807807807808	2.25162671551736\\
8.06806806806807	2.25158181029834\\
8.05805805805806	2.25153744285339\\
8.04804804804805	2.25149361931275\\
8.03803803803804	2.25145034519329\\
8.02802802802803	2.25140762541837\\
8.01801801801802	2.25136546433742\\
8.00800800800801	2.25132386574527\\
7.997997997998	2.25128283290115\\
7.98798798798799	2.25124236854743\\
7.97797797797798	2.25120247492805\\
7.96796796796797	2.25116315380663\\
7.95795795795796	2.25112440648429\\
7.94794794794795	2.25108623381712\\
7.93793793793794	2.25104863623337\\
7.92792792792793	2.25101161375029\\
7.91791791791792	2.25097516599066\\
7.90790790790791	2.25093929219895\\
7.8978978978979	2.25090399125721\\
7.88788788788789	2.25086926170059\\
7.87787787787788	2.25083510173251\\
7.86786786786787	2.25080150923951\\
7.85785785785786	2.25076848180583\\
7.84784784784785	2.2507360167275\\
7.83783783783784	2.25070411102624\\
7.82782782782783	2.25067276146297\\
7.81781781781782	2.25064196455098\\
7.80780780780781	2.25061171656874\\
7.7977977977978	2.25058201357245\\
7.78778778778779	2.25055285140823\\
7.77777777777778	2.2505242257239\\
7.76776776776777	2.25049613198061\\
7.75775775775776	2.25046856546396\\
7.74774774774775	2.25044152129494\\
7.73773773773774	2.25041499444045\\
7.72772772772773	2.25038897972361\\
7.71771771771772	2.25036347183367\\
7.70770770770771	2.25033846533566\\
7.6976976976977	2.25031395467971\\
7.68768768768769	2.25028993421014\\
7.67767767767768	2.25026639817414\\
7.66766766766767	2.25024334073028\\
7.65765765765766	2.25022075595663\\
7.64764764764765	2.25019863785872\\
7.63763763763764	2.25017698037708\\
7.62762762762763	2.25015577739462\\
7.61761761761762	2.25013502274371\\
7.60760760760761	2.25011471021298\\
7.5975975975976	2.25009483355389\\
7.58758758758759	2.25007538648702\\
7.57757757757758	2.25005636270814\\
7.56756756756757	2.25003775589403\\
7.55755755755756	2.25001955970808\\
7.54754754754755	2.25000176780558\\
7.53753753753754	2.24998437383891\\
7.52752752752753	2.24996737146242\\
7.51751751751752	2.24995075433712\\
7.50750750750751	2.24993451613514\\
7.4974974974975	2.24991865054401\\
7.48748748748749	2.24990315127077\\
7.47747747747748	2.2498880120458\\
7.46746746746747	2.2498732266265\\
7.45745745745746	2.24985878880085\\
7.44744744744745	2.24984469239068\\
7.43743743743744	2.24983093125482\\
7.42742742742743	2.24981749929212\\
7.41741741741742	2.24980439044421\\
7.40740740740741	2.24979159869819\\
7.3973973973974	2.24977911808908\\
7.38738738738739	2.24976694270224\\
7.37737737737738	2.24975506667548\\
7.36736736736737	2.2497434842012\\
7.35735735735736	2.24973218952825\\
7.34734734734735	2.24972117696375\\
7.33733733733734	2.24971044087474\\
7.32732732732733	2.24969997568972\\
7.31731731731732	2.24968977590006\\
7.30730730730731	2.24967983606131\\
7.2972972972973	2.24967015079439\\
7.28728728728729	2.24966071478666\\
7.27727727727728	2.24965152279291\\
7.26726726726727	2.24964256963625\\
7.25725725725726	2.24963385020886\\
7.24724724724725	2.24962535947273\\
7.23723723723724	2.24961709246022\\
7.22722722722723	2.2496090442746\\
7.21721721721722	2.2496012100905\\
7.20720720720721	2.24959358515424\\
7.1971971971972	2.24958616478413\\
7.18718718718719	2.24957894437068\\
7.17717717717718	2.24957191937676\\
7.16716716716717	2.24956508533766\\
7.15715715715716	2.24955843786111\\
7.14714714714715	2.24955197262724\\
7.13713713713714	2.24954568538849\\
7.12712712712713	2.24953957196945\\
7.11711711711712	2.24953362826665\\
7.10710710710711	2.24952785024831\\
7.0970970970971	2.24952223395404\\
7.08708708708709	2.24951677549452\\
7.07707707707708	2.24951147105107\\
7.06706706706707	2.24950631687526\\
7.05705705705706	2.24950130928845\\
7.04704704704705	2.24949644468127\\
7.03703703703704	2.24949171951312\\
7.02702702702703	2.24948713031159\\
7.01701701701702	2.2494826736719\\
7.00700700700701	2.24947834625624\\
6.996996996997	2.24947414479317\\
6.98698698698699	2.24947006607696\\
6.97697697697698	2.24946610696686\\
6.96696696696697	2.24946226438646\\
6.95695695695696	2.2494585353229\\
6.94694694694695	2.24945491682619\\
6.93693693693694	2.24945140600842\\
6.92692692692693	2.24944800004298\\
6.91691691691692	2.24944469616382\\
6.90690690690691	2.24944149166463\\
6.8968968968969	2.24943838389802\\
6.88688688688689	2.24943537027475\\
6.87687687687688	2.24943244826287\\
6.86686686686687	2.24942961538691\\
6.85685685685686	2.24942686922703\\
6.84684684684685	2.24942420741821\\
6.83683683683684	2.2494216276494\\
6.82682682682683	2.24941912766264\\
6.81681681681682	2.24941670525227\\
6.80680680680681	2.24941435826402\\
6.7967967967968	2.24941208459421\\
6.78678678678679	2.24940988218889\\
6.77677677677678	2.24940774904296\\
6.76676676676677	2.24940568319937\\
6.75675675675676	2.24940368274826\\
6.74674674674675	2.24940174582609\\
6.73673673673674	2.24939987061484\\
6.72672672672673	2.24939805534118\\
6.71671671671672	2.24939629827559\\
6.70670670670671	2.24939459773161\\
6.6966966966967	2.24939295206496\\
6.68668668668669	2.24939135967274\\
6.67667667667668	2.24938981899267\\
6.66666666666667	2.24938832850222\\
6.65665665665666	2.24938688671788\\
6.64664664664665	2.24938549219433\\
6.63663663663664	2.2493841435237\\
6.62662662662663	2.24938283933476\\
6.61661661661662	2.2493815782922\\
6.60660660660661	2.24938035909585\\
6.5965965965966	2.24937918047995\\
6.58658658658659	2.24937804121241\\
6.57657657657658	2.24937694009409\\
6.56656656656657	2.24937587595807\\
6.55655655655656	2.24937484766895\\
6.54654654654655	2.24937385412217\\
6.53653653653654	2.24937289424328\\
6.52652652652653	2.24937196698733\\
6.51651651651652	2.24937107133811\\
6.50650650650651	2.24937020630759\\
6.4964964964965	2.24936937093518\\
6.48648648648649	2.24936856428718\\
6.47647647647648	2.24936778545608\\
6.46646646646647	2.24936703355997\\
6.45645645645646	2.24936630774194\\
6.44644644644645	2.24936560716948\\
6.43643643643644	2.24936493103389\\
6.42642642642643	2.24936427854967\\
6.41641641641642	2.24936364895402\\
6.40640640640641	2.24936304150623\\
6.3963963963964	2.24936245548716\\
6.38638638638639	2.24936189019866\\
6.37637637637638	2.24936134496313\\
6.36636636636637	2.2493608191229\\
6.35635635635636	2.24936031203979\\
6.34634634634635	2.2493598230946\\
6.33633633633634	2.24935935168661\\
6.32632632632633	2.24935889723312\\
6.31631631631632	2.24935845916895\\
6.30630630630631	2.24935803694603\\
6.2962962962963	2.24935763003289\\
6.28628628628629	2.24935723791428\\
6.27627627627628	2.2493568600907\\
6.26626626626627	2.249356496078\\
6.25625625625626	2.24935614540695\\
6.24624624624625	2.24935580762284\\
6.23623623623624	2.2493554822851\\
6.22622622622623	2.24935516896689\\
6.21621621621622	2.24935486725474\\
6.20620620620621	2.2493545767482\\
6.1961961961962	2.24935429705941\\
6.18618618618619	2.24935402781283\\
6.17617617617618	2.24935376864484\\
6.16616616616617	2.24935351920342\\
6.15615615615616	2.24935327914784\\
6.14614614614615	2.2493530481483\\
6.13613613613614	2.24935282588565\\
6.12612612612613	2.24935261205108\\
6.11611611611612	2.24935240634579\\
6.10610610610611	2.24935220848074\\
6.0960960960961	2.24935201817636\\
6.08608608608609	2.24935183516224\\
6.07607607607608	2.24935165917689\\
6.06606606606607	2.24935148996747\\
6.05605605605606	2.24935132728953\\
6.04604604604605	2.24935117090676\\
6.03603603603604	2.24935102059075\\
6.02602602602603	2.24935087612075\\
6.01601601601602	2.24935073728342\\
6.00600600600601	2.24935060387266\\
5.995995995996	2.24935047568932\\
5.98598598598599	2.24935035254103\\
5.97597597597598	2.24935023424199\\
5.96596596596597	2.24935012061275\\
5.95595595595596	2.24935001148002\\
5.94594594594595	2.24934990667647\\
5.93593593593594	2.24934980604059\\
5.92592592592593	2.24934970941643\\
5.91591591591592	2.24934961665348\\
5.90590590590591	2.2493495276065\\
5.8958958958959	2.24934944213531\\
5.88588588588589	2.24934936010468\\
5.87587587587588	2.24934928138412\\
5.86586586586587	2.24934920584779\\
5.85585585585586	2.24934913337427\\
5.84584584584585	2.24934906384649\\
5.83583583583584	2.24934899715157\\
5.82582582582583	2.24934893318064\\
5.81581581581582	2.24934887182878\\
5.80580580580581	2.24934881299481\\
5.7957957957958	2.24934875658124\\
5.78578578578579	2.24934870249412\\
5.77577577577578	2.24934865064289\\
5.76576576576577	2.24934860094033\\
5.75575575575576	2.24934855330239\\
5.74574574574575	2.24934850764813\\
5.73573573573574	2.24934846389956\\
5.72572572572573	2.24934842198162\\
5.71571571571572	2.24934838182199\\
5.70570570570571	2.24934834335108\\
5.6956956956957	2.24934830650186\\
5.68568568568569	2.24934827120986\\
5.67567567567568	2.24934823741298\\
5.66566566566567	2.2493482050515\\
5.65565565565566	2.24934817406794\\
5.64564564564565	2.24934814440702\\
5.63563563563564	2.24934811601554\\
5.62562562562563	2.24934808884236\\
5.61561561561562	2.24934806283828\\
5.60560560560561	2.249348037956\\
5.5955955955956	2.24934801415005\\
5.58558558558559	2.24934799137673\\
5.57557557557558	2.24934796959403\\
5.56556556556557	2.24934794876157\\
5.55555555555556	2.24934792884057\\
5.54554554554555	2.24934790979377\\
5.53553553553554	2.24934789158538\\
5.52552552552553	2.24934787418103\\
5.51551551551552	2.24934785754773\\
5.50550550550551	2.24934784165379\\
5.4954954954955	2.2493478264688\\
5.48548548548549	2.24934781196359\\
5.47547547547548	2.24934779811016\\
5.46546546546547	2.24934778488165\\
5.45545545545546	2.2493477722523\\
5.44544544544545	2.24934776019743\\
5.43543543543544	2.24934774869335\\
5.42542542542543	2.24934773771739\\
5.41541541541542	2.2493477272478\\
5.40540540540541	2.24934771726377\\
5.3953953953954	2.24934770774536\\
5.38538538538539	2.24934769867348\\
5.37537537537538	2.24934769002989\\
5.36536536536537	2.2493476817971\\
5.35535535535536	2.24934767395841\\
5.34534534534535	2.24934766649787\\
5.33533533533534	2.24934765940022\\
5.32532532532533	2.24934765265089\\
5.31531531531532	2.24934764623598\\
5.30530530530531	2.24934764014224\\
5.2952952952953	2.24934763435701\\
5.28528528528529	2.24934762886827\\
5.27527527527528	2.24934762366454\\
5.26526526526527	2.24934761873492\\
5.25525525525526	2.24934761406904\\
5.24524524524525	2.24934760965706\\
5.23523523523524	2.24934760548965\\
5.22522522522523	2.24934760155796\\
5.21521521521522	2.24934759785362\\
5.20520520520521	2.24934759436874\\
5.1951951951952	2.24934759109584\\
5.18518518518519	2.24934758802791\\
5.17517517517518	2.24934758515835\\
5.16516516516517	2.24934758248097\\
5.15515515515516	2.24934757998998\\
5.14514514514515	2.24934757767999\\
5.13513513513514	2.24934757554599\\
5.12512512512513	2.24934757358332\\
5.11511511511512	2.24934757178772\\
5.10510510510511	2.24934757015527\\
5.0950950950951	2.24934756868239\\
5.08508508508509	2.24934756736587\\
5.07507507507508	2.24934756620282\\
5.06506506506507	2.24934756519068\\
5.05505505505506	2.24934756432724\\
5.04504504504505	2.24934756361058\\
5.03503503503504	2.24934756303915\\
5.02502502502503	2.24934756261167\\
5.01501501501502	2.2493475623272\\
5.00500500500501	2.24934756218513\\
4.99499499499499	2.24934756218513\\
4.98498498498498	2.2493475623272\\
4.97497497497497	2.24934756261167\\
4.96496496496496	2.24934756303915\\
4.95495495495495	2.24934756361058\\
4.94494494494494	2.24934756432724\\
4.93493493493493	2.24934756519068\\
4.92492492492492	2.24934756620282\\
4.91491491491491	2.24934756736587\\
4.9049049049049	2.24934756868239\\
4.89489489489489	2.24934757015527\\
4.88488488488488	2.24934757178772\\
4.87487487487487	2.24934757358332\\
4.86486486486486	2.24934757554599\\
4.85485485485485	2.24934757767999\\
4.84484484484484	2.24934757998998\\
4.83483483483483	2.24934758248097\\
4.82482482482482	2.24934758515835\\
4.81481481481481	2.24934758802791\\
4.8048048048048	2.24934759109584\\
4.79479479479479	2.24934759436874\\
4.78478478478478	2.24934759785362\\
4.77477477477477	2.24934760155796\\
4.76476476476476	2.24934760548965\\
4.75475475475475	2.24934760965706\\
4.74474474474474	2.24934761406904\\
4.73473473473473	2.24934761873492\\
4.72472472472472	2.24934762366454\\
4.71471471471471	2.24934762886827\\
4.7047047047047	2.24934763435701\\
4.69469469469469	2.24934764014224\\
4.68468468468468	2.24934764623598\\
4.67467467467467	2.24934765265089\\
4.66466466466466	2.24934765940022\\
4.65465465465465	2.24934766649787\\
4.64464464464464	2.24934767395841\\
4.63463463463463	2.2493476817971\\
4.62462462462462	2.24934769002989\\
4.61461461461461	2.24934769867348\\
4.6046046046046	2.24934770774536\\
4.59459459459459	2.24934771726377\\
4.58458458458458	2.2493477272478\\
4.57457457457457	2.24934773771739\\
4.56456456456456	2.24934774869335\\
4.55455455455455	2.24934776019743\\
4.54454454454454	2.2493477722523\\
4.53453453453453	2.24934778488165\\
4.52452452452452	2.24934779811016\\
4.51451451451451	2.24934781196359\\
4.5045045045045	2.2493478264688\\
4.49449449449449	2.24934784165379\\
4.48448448448448	2.24934785754773\\
4.47447447447447	2.24934787418103\\
4.46446446446446	2.24934789158538\\
4.45445445445445	2.24934790979377\\
4.44444444444444	2.24934792884057\\
4.43443443443443	2.24934794876157\\
4.42442442442442	2.24934796959403\\
4.41441441441441	2.24934799137673\\
4.4044044044044	2.24934801415005\\
4.39439439439439	2.249348037956\\
4.38438438438438	2.24934806283828\\
4.37437437437437	2.24934808884236\\
4.36436436436436	2.24934811601554\\
4.35435435435435	2.24934814440702\\
4.34434434434434	2.24934817406794\\
4.33433433433433	2.2493482050515\\
4.32432432432432	2.24934823741298\\
4.31431431431431	2.24934827120986\\
4.3043043043043	2.24934830650186\\
4.29429429429429	2.24934834335108\\
4.28428428428428	2.24934838182199\\
4.27427427427427	2.24934842198162\\
4.26426426426426	2.24934846389956\\
4.25425425425425	2.24934850764813\\
4.24424424424424	2.24934855330239\\
4.23423423423423	2.24934860094033\\
4.22422422422422	2.24934865064289\\
4.21421421421421	2.24934870249412\\
4.2042042042042	2.24934875658124\\
4.19419419419419	2.24934881299481\\
4.18418418418418	2.24934887182878\\
4.17417417417417	2.24934893318064\\
4.16416416416416	2.24934899715157\\
4.15415415415415	2.24934906384649\\
4.14414414414414	2.24934913337427\\
4.13413413413413	2.24934920584779\\
4.12412412412412	2.24934928138412\\
4.11411411411411	2.24934936010468\\
4.1041041041041	2.24934944213531\\
4.09409409409409	2.2493495276065\\
4.08408408408408	2.24934961665348\\
4.07407407407407	2.24934970941643\\
4.06406406406406	2.24934980604059\\
4.05405405405405	2.24934990667647\\
4.04404404404404	2.24935001148002\\
4.03403403403403	2.24935012061275\\
4.02402402402402	2.24935023424199\\
4.01401401401401	2.24935035254103\\
4.004004004004	2.24935047568932\\
3.99399399399399	2.24935060387266\\
3.98398398398398	2.24935073728342\\
3.97397397397397	2.24935087612075\\
3.96396396396396	2.24935102059075\\
3.95395395395395	2.24935117090676\\
3.94394394394394	2.24935132728953\\
3.93393393393393	2.24935148996747\\
3.92392392392392	2.24935165917689\\
3.91391391391391	2.24935183516224\\
3.9039039039039	2.24935201817636\\
3.89389389389389	2.24935220848074\\
3.88388388388388	2.24935240634579\\
3.87387387387387	2.24935261205108\\
3.86386386386386	2.24935282588565\\
3.85385385385385	2.2493530481483\\
3.84384384384384	2.24935327914784\\
3.83383383383383	2.24935351920342\\
3.82382382382382	2.24935376864484\\
3.81381381381381	2.24935402781283\\
3.8038038038038	2.24935429705941\\
3.79379379379379	2.2493545767482\\
3.78378378378378	2.24935486725474\\
3.77377377377377	2.24935516896689\\
3.76376376376376	2.2493554822851\\
3.75375375375375	2.24935580762284\\
3.74374374374374	2.24935614540695\\
3.73373373373373	2.249356496078\\
3.72372372372372	2.2493568600907\\
3.71371371371371	2.24935723791428\\
3.7037037037037	2.24935763003289\\
3.69369369369369	2.24935803694603\\
3.68368368368368	2.24935845916895\\
3.67367367367367	2.24935889723312\\
3.66366366366366	2.24935935168661\\
3.65365365365365	2.2493598230946\\
3.64364364364364	2.24936031203979\\
3.63363363363363	2.2493608191229\\
3.62362362362362	2.24936134496313\\
3.61361361361361	2.24936189019866\\
3.6036036036036	2.24936245548716\\
3.59359359359359	2.24936304150623\\
3.58358358358358	2.24936364895402\\
3.57357357357357	2.24936427854967\\
3.56356356356356	2.24936493103389\\
3.55355355355355	2.24936560716948\\
3.54354354354354	2.24936630774194\\
3.53353353353353	2.24936703355997\\
3.52352352352352	2.24936778545608\\
3.51351351351351	2.24936856428718\\
3.5035035035035	2.24936937093518\\
3.49349349349349	2.24937020630759\\
3.48348348348348	2.24937107133811\\
3.47347347347347	2.24937196698733\\
3.46346346346346	2.24937289424328\\
3.45345345345345	2.24937385412217\\
3.44344344344344	2.24937484766895\\
3.43343343343343	2.24937587595807\\
3.42342342342342	2.24937694009409\\
3.41341341341341	2.24937804121241\\
3.4034034034034	2.24937918047995\\
3.39339339339339	2.24938035909585\\
3.38338338338338	2.2493815782922\\
3.37337337337337	2.24938283933476\\
3.36336336336336	2.2493841435237\\
3.35335335335335	2.24938549219433\\
3.34334334334334	2.24938688671788\\
3.33333333333333	2.24938832850222\\
3.32332332332332	2.24938981899267\\
3.31331331331331	2.24939135967274\\
3.3033033033033	2.24939295206496\\
3.29329329329329	2.24939459773161\\
3.28328328328328	2.24939629827559\\
3.27327327327327	2.24939805534118\\
3.26326326326326	2.24939987061484\\
3.25325325325325	2.24940174582609\\
3.24324324324324	2.24940368274826\\
3.23323323323323	2.24940568319937\\
3.22322322322322	2.24940774904296\\
3.21321321321321	2.24940988218889\\
3.2032032032032	2.24941208459421\\
3.19319319319319	2.24941435826402\\
3.18318318318318	2.24941670525227\\
3.17317317317317	2.24941912766264\\
3.16316316316316	2.2494216276494\\
3.15315315315315	2.24942420741821\\
3.14314314314314	2.24942686922703\\
3.13313313313313	2.24942961538691\\
3.12312312312312	2.24943244826287\\
3.11311311311311	2.24943537027475\\
3.1031031031031	2.24943838389802\\
3.09309309309309	2.24944149166463\\
3.08308308308308	2.24944469616382\\
3.07307307307307	2.24944800004298\\
3.06306306306306	2.24945140600842\\
3.05305305305305	2.24945491682619\\
3.04304304304304	2.2494585353229\\
3.03303303303303	2.24946226438646\\
3.02302302302302	2.24946610696686\\
3.01301301301301	2.24947006607696\\
3.003003003003	2.24947414479317\\
2.99299299299299	2.24947834625624\\
2.98298298298298	2.2494826736719\\
2.97297297297297	2.24948713031159\\
2.96296296296296	2.24949171951312\\
2.95295295295295	2.24949644468127\\
2.94294294294294	2.24950130928845\\
2.93293293293293	2.24950631687526\\
2.92292292292292	2.24951147105107\\
2.91291291291291	2.24951677549452\\
2.9029029029029	2.24952223395404\\
2.89289289289289	2.24952785024831\\
2.88288288288288	2.24953362826665\\
2.87287287287287	2.24953957196945\\
2.86286286286286	2.24954568538849\\
2.85285285285285	2.24955197262724\\
2.84284284284284	2.24955843786111\\
2.83283283283283	2.24956508533766\\
2.82282282282282	2.24957191937676\\
2.81281281281281	2.24957894437068\\
2.8028028028028	2.24958616478413\\
2.79279279279279	2.24959358515424\\
2.78278278278278	2.2496012100905\\
2.77277277277277	2.2496090442746\\
2.76276276276276	2.24961709246022\\
2.75275275275275	2.24962535947273\\
2.74274274274274	2.24963385020886\\
2.73273273273273	2.24964256963625\\
2.72272272272272	2.24965152279291\\
2.71271271271271	2.24966071478666\\
2.7027027027027	2.24967015079439\\
2.69269269269269	2.24967983606131\\
2.68268268268268	2.24968977590006\\
2.67267267267267	2.24969997568972\\
2.66266266266266	2.24971044087474\\
2.65265265265265	2.24972117696375\\
2.64264264264264	2.24973218952825\\
2.63263263263263	2.2497434842012\\
2.62262262262262	2.24975506667548\\
2.61261261261261	2.24976694270224\\
2.6026026026026	2.24977911808908\\
2.59259259259259	2.24979159869819\\
2.58258258258258	2.24980439044421\\
2.57257257257257	2.24981749929212\\
2.56256256256256	2.24983093125482\\
2.55255255255255	2.24984469239068\\
2.54254254254254	2.24985878880085\\
2.53253253253253	2.2498732266265\\
2.52252252252252	2.2498880120458\\
2.51251251251251	2.24990315127077\\
2.5025025025025	2.24991865054401\\
2.49249249249249	2.24993451613514\\
2.48248248248248	2.24995075433712\\
2.47247247247247	2.24996737146242\\
2.46246246246246	2.24998437383891\\
2.45245245245245	2.25000176780558\\
2.44244244244244	2.25001955970808\\
2.43243243243243	2.25003775589403\\
2.42242242242242	2.25005636270814\\
2.41241241241241	2.25007538648702\\
2.4024024024024	2.25009483355389\\
2.39239239239239	2.25011471021298\\
2.38238238238238	2.25013502274371\\
2.37237237237237	2.25015577739462\\
2.36236236236236	2.25017698037708\\
2.35235235235235	2.25019863785872\\
2.34234234234234	2.25022075595663\\
2.33233233233233	2.25024334073028\\
2.32232232232232	2.25026639817414\\
2.31231231231231	2.25028993421014\\
2.3023023023023	2.25031395467971\\
2.29229229229229	2.25033846533566\\
2.28228228228228	2.25036347183367\\
2.27227227227227	2.25038897972361\\
2.26226226226226	2.25041499444045\\
2.25225225225225	2.25044152129494\\
2.24224224224224	2.25046856546396\\
2.23223223223223	2.25049613198061\\
2.22222222222222	2.2505242257239\\
2.21221221221221	2.25055285140823\\
2.2022022022022	2.25058201357245\\
2.19219219219219	2.25061171656874\\
2.18218218218218	2.25064196455098\\
2.17217217217217	2.25067276146297\\
2.16216216216216	2.25070411102624\\
2.15215215215215	2.2507360167275\\
2.14214214214214	2.25076848180583\\
2.13213213213213	2.25080150923951\\
2.12212212212212	2.25083510173251\\
2.11211211211211	2.25086926170059\\
2.1021021021021	2.25090399125721\\
2.09209209209209	2.25093929219895\\
2.08208208208208	2.25097516599066\\
2.07207207207207	2.25101161375029\\
2.06206206206206	2.25104863623337\\
2.05205205205205	2.25108623381712\\
2.04204204204204	2.25112440648429\\
2.03203203203203	2.25116315380663\\
2.02202202202202	2.25120247492805\\
2.01201201201201	2.25124236854743\\
2.002002002002	2.25128283290115\\
1.99199199199199	2.25132386574527\\
1.98198198198198	2.25136546433742\\
1.97197197197197	2.25140762541837\\
1.96196196196196	2.25145034519329\\
1.95195195195195	2.25149361931275\\
1.94194194194194	2.25153744285339\\
1.93193193193193	2.25158181029834\\
1.92192192192192	2.25162671551736\\
1.91191191191191	2.2516721517467\\
1.9019019019019	2.25171811156873\\
1.89189189189189	2.25176458689128\\
1.88188188188188	2.25181156892684\\
1.87187187187187	2.25185904817141\\
1.86186186186186	2.25190701438321\\
1.85185185185185	2.25195545656124\\
1.84184184184184	2.25200436292351\\
1.83183183183183	2.25205372088527\\
1.82182182182182	2.25210351703691\\
1.81181181181181	2.25215373712187\\
1.8018018018018	2.25220436601431\\
1.79179179179179	2.25225538769676\\
1.78178178178178	2.25230678523758\\
1.77177177177177	2.25235854076844\\
1.76176176176176	2.25241063546167\\
1.75175175175175	2.25246304950763\\
1.74174174174174	2.25251576209201\\
1.73173173173173	2.25256875137317\\
1.72172172172172	2.25262199445953\\
1.71171171171171	2.25267546738692\\
1.7017017017017	2.25272914509607\\
1.69169169169169	2.25278300141022\\
1.68168168168168	2.25283700901273\\
1.67167167167167	2.25289113942501\\
1.66166166166166	2.25294536298443\\
1.65165165165165	2.25299964882256\\
1.64164164164164	2.25305396484357\\
1.63163163163163	2.25310827770291\\
1.62162162162162	2.25316255278625\\
1.61161161161161	2.25321675418872\\
1.6016016016016	2.2532708446945\\
1.59159159159159	2.25332478575676\\
1.58158158158158	2.25337853747801\\
1.57157157157157	2.25343205859086\\
1.56156156156156	2.25348530643924\\
1.55155155155155	2.25353823696013\\
1.54154154154154	2.25359080466576\\
1.53153153153153	2.2536429626264\\
1.52152152152152	2.25369466245373\\
1.51151151151151	2.25374585428479\\
1.5015015015015	2.25379648676663\\
1.49149149149149	2.25384650704163\\
1.48148148148148	2.25389586073346\\
1.47147147147147	2.25394449193393\\
1.46146146146146	2.25399234319048\\
1.45145145145145	2.25403935549456\\
1.44144144144144	2.25408546827087\\
1.43143143143143	2.2541306193674\\
1.42142142142142	2.25417474504651\\
1.41141141141141	2.25421777997683\\
1.4014014014014	2.2542596572262\\
1.39139139139139	2.25430030825565\\
1.38138138138138	2.25433966291439\\
1.37137137137137	2.25437764943585\\
1.36136136136136	2.25441419443489\\
1.35135135135135	2.25444922290611\\
1.34134134134134	2.25448265822335\\
1.33133133133133	2.25451442214035\\
1.32132132132132	2.25454443479272\\
1.31131131131131	2.25457261470105\\
1.3013013013013	2.25459887877545\\
1.29129129129129	2.25462314232125\\
1.28128128128128	2.25464531904617\\
1.27127127127127	2.25466532106876\\
1.26126126126126	2.2546830589283\\
1.25125125125125	2.25469844159606\\
1.24124124124124	2.25471137648801\\
1.23123123123123	2.25472176947901\\
1.22122122122122	2.25472952491844\\
1.21121121121121	2.25473454564732\\
1.2012012012012	2.25473673301702\\
1.19119119119119	2.25473598690938\\
1.18118118118118	2.25473220575845\\
1.17117117117117	2.2547252865738\\
1.16116116116116	2.25471512496533\\
1.15115115115115	2.2547016151697\\
1.14114114114114	2.25468465007838\\
1.13113113113113	2.25466412126723\\
1.12112112112112	2.25463991902772\\
1.11111111111111	2.25461193239976\\
1.1011011011011	2.25458004920609\\
1.09109109109109	2.25454415608837\\
1.08108108108108	2.25450413854479\\
1.07107107107107	2.25445988096932\\
1.06106106106106	2.25441126669256\\
1.05105105105105	2.25435817802418\\
1.04104104104104	2.25430049629697\\
1.03103103103103	2.25423810191241\\
1.02102102102102	2.25417087438787\\
1.01101101101101	2.25409869240529\\
1.001001001001	2.25402143386148\\
0.990990990990991	2.25393897591982\\
0.980980980980981	2.25385119506361\\
0.970970970970971	2.25375796715073\\
0.960960960960961	2.25365916746986\\
0.950950950950951	2.25355467079812\\
0.940940940940941	2.25344435146003\\
0.930930930930931	2.25332808338796\\
0.920920920920921	2.25320574018377\\
0.910910910910911	2.25307719518192\\
0.900900900900901	2.25294232151371\\
0.890890890890891	2.25280099217283\\
0.880880880880881	2.25265308008209\\
0.870870870870871	2.25249845816129\\
0.860860860860861	2.25233699939622\\
0.850850850850851	2.25216857690871\\
0.840840840840841	2.25199306402773\\
0.830830830830831	2.25181033436144\\
0.820820820820821	2.25162026187018\\
0.810810810810811	2.25142272094033\\
0.800800800800801	2.25121758645901\\
0.790790790790791	2.25100473388954\\
0.780780780780781	2.25078403934761\\
0.770770770770771	2.25055537967812\\
0.760760760760761	2.25031863253262\\
0.750750750750751	2.25007367644732\\
0.740740740740741	2.24982039092153\\
0.730730730730731	2.24955865649656\\
0.720720720720721	2.249288354835\\
0.710710710710711	2.24900936880024\\
0.700700700700701	2.24872158253629\\
0.690690690690691	2.24842488154771\\
0.680680680680681	2.24811915277967\\
0.670670670670671	2.24780428469807\\
0.660660660660661	2.24748016736959\\
0.650650650650651	2.24714669254163\\
0.640640640640641	2.24680375372217\\
0.630630630630631	2.24645124625926\\
0.620620620620621	2.24608906742035\\
0.610610610610611	2.24571711647111\\
0.600600600600601	2.24533529475391\\
0.590590590590591	2.24494350576573\\
0.580580580580581	2.2445416552355\\
0.570570570570571	2.24412965120079\\
0.560560560560561	2.24370740408378\\
0.550550550550551	2.24327482676643\\
0.540540540540541	2.24283183466477\\
0.530530530530531	2.24237834580232\\
0.520520520520521	2.24191428088241\\
0.510510510510511	2.24143956335956\\
0.500500500500501	2.24095411950958\\
0.49049049049049	2.24045787849862\\
0.48048048048048	2.23995077245083\\
0.47047047047047	2.23943273651472\\
0.46046046046046	2.2389037089282\\
0.45045045045045	2.23836363108202\\
0.44044044044044	2.23781244758184\\
0.43043043043043	2.23725010630857\\
0.42042042042042	2.23667655847716\\
0.41041041041041	2.2360917586937\\
0.4004004004004	2.23549566501065\\
0.39039039039039	2.23488823898039\\
0.38038038038038	2.23426944570681\\
0.37037037037037	2.23363925389497\\
0.36036036036036	2.23299763589886\\
0.35035035035035	2.23234456776709\\
0.34034034034034	2.23168002928645\\
0.33033033033033	2.23100400402346\\
0.32032032032032	2.23031647936369\\
0.31031031031031	2.22961744654883\\
0.3003003003003	2.22890690071165\\
0.29029029029029	2.22818484090851\\
0.28028028028028	2.22745127014963\\
0.27027027027027	2.22670619542704\\
0.26026026026026	2.22594962774\\
0.25025025025025	2.22518158211818\\
0.24024024024024	2.22440207764218\\
0.23023023023023	2.22361113746178\\
0.22022022022022	2.22280878881152\\
0.21021021021021	2.22199506302383\\
0.2002002002002	2.22116999553961\\
0.19019019019019	2.22033362591626\\
0.18018018018018	2.21948599783308\\
0.17017017017017	2.21862715909409\\
0.16016016016016	2.21775716162834\\
0.15015015015015	2.21687606148746\\
0.14014014014014	2.21598391884069\\
0.13013013013013	2.2150807979673\\
0.12012012012012	2.21416676724629\\
0.11011011011011	2.21324189914361\\
0.1001001001001	2.21230627019662\\
0.0900900900900901	2.21135996099603\\
0.0800800800800801	2.21040305616522\\
0.0700700700700701	2.20943564433695\\
0.0600600600600601	2.20845781812748\\
0.0500500500500501	2.20746967410815\\
0.04004004004004	2.20647131277441\\
0.03003003003003	2.2054628385123\\
0.02002002002002	2.20444435956244\\
0.01001001001001	2.20341598798152\\
0	2.20237783960136\\
}--cycle;

\addlegendentry{$\pm 2\sigma$};

\addplot [color=mycolor5,solid]
  table[row sep=crcr]{%
0	0.271676661371016\\
0.01001001001001	0.273846461107548\\
0.02002002002002	0.276016043442201\\
0.03003003003003	0.278185191038444\\
0.04004004004004	0.280353686690401\\
0.0500500500500501	0.282521313388101\\
0.0600600600600601	0.284687854382627\\
0.0700700700700701	0.286853093251135\\
0.0800800800800801	0.289016813961707\\
0.0900900900900901	0.291178800938009\\
0.1001001001001	0.293338839123727\\
0.11011011011011	0.295496714046737\\
0.12012012012012	0.297652211882983\\
0.13013013013013	0.299805119520032\\
0.14014014014014	0.301955224620282\\
0.15015015015015	0.304102315683767\\
0.16016016016016	0.306246182110565\\
0.17017017017017	0.308386614262737\\
0.18018018018018	0.310523403525807\\
0.19019019019019	0.312656342369723\\
0.2002002002002	0.314785224409276\\
0.21021021021021	0.316909844463968\\
0.22022022022022	0.319029998617265\\
0.23023023023023	0.321145484275239\\
0.24024024024024	0.323256100224548\\
0.25025025025025	0.325361646689746\\
0.26026026026026	0.327461925389872\\
0.27027027027027	0.329556739594317\\
0.28028028028028	0.331645894177926\\
0.29029029029029	0.333729195675309\\
0.3003003003003	0.335806452334352\\
0.31031031031031	0.337877474168883\\
0.32032032032032	0.339942073010483\\
0.33033033033033	0.34200006255941\\
0.34034034034034	0.344051258434626\\
0.35035035035035	0.346095478222884\\
0.36036036036036	0.348132541526876\\
0.37037037037037	0.350162270012406\\
0.38038038038038	0.352184487454573\\
0.39039039039039	0.354199019782942\\
0.4004004004004	0.356205695125692\\
0.41041041041041	0.35820434385271\\
0.42042042042042	0.360194798617624\\
0.43043043043043	0.362176894398754\\
0.44044044044044	0.364150468538962\\
0.45045045045045	0.366115360784399\\
0.46046046046046	0.36807141332211\\
0.47047047047047	0.370018470816507\\
0.48048048048048	0.371956380444687\\
0.49049049049049	0.37388499193058\\
0.500500500500501	0.375804157577918\\
0.510510510510511	0.377713732302018\\
0.520520520520521	0.37961357366036\\
0.530530530530531	0.381503541881962\\
0.540540540540541	0.383383499895533\\
0.550550550550551	0.385253313356401\\
0.560560560560561	0.38711285067221\\
0.570570570570571	0.388961983027379\\
0.580580580580581	0.390800584406311\\
0.590590590590591	0.392628531615365\\
0.600600600600601	0.39444570430356\\
0.610610610610611	0.396251984982036\\
0.620620620620621	0.39804725904225\\
0.630630630630631	0.39983141477291\\
0.640640640640641	0.401604343375659\\
0.650650650650651	0.403365938979485\\
0.660660660660661	0.405116098653878\\
0.670670670670671	0.406854722420727\\
0.680680680680681	0.408581713264962\\
0.690690690690691	0.410296977143937\\
0.700700700700701	0.412000422995571\\
0.710710710710711	0.413691962745234\\
0.720720720720721	0.4153715113114\\
0.730730730730731	0.417038986610063\\
0.740740740740741	0.41869430955792\\
0.750750750750751	0.420337404074343\\
0.760760760760761	0.421968197082126\\
0.770770770770771	0.423586618507035\\
0.780780780780781	0.425192601276159\\
0.790790790790791	0.42678608131508\\
0.800800800800801	0.428366997543857\\
0.810810810810811	0.429935291871862\\
0.820820820820821	0.431490909191447\\
0.830830830830831	0.433033797370479\\
0.840840840840841	0.434563907243752\\
0.850850850850851	0.436081192603274\\
0.860860860860861	0.437585610187463\\
0.870870870870871	0.439077119669255\\
0.880880880880881	0.440555683643141\\
0.890890890890891	0.44202126761115\\
0.900900900900901	0.443473839967791\\
0.910910910910911	0.444913371983982\\
0.920920920920921	0.446339837789961\\
0.930930930930931	0.447753214357219\\
0.940940940940941	0.449153481479461\\
0.950950950950951	0.450540621752613\\
0.960960960960961	0.451914620553905\\
0.970970970970971	0.453275466020026\\
0.980980980980981	0.454623149024402\\
0.990990990990991	0.455957663153588\\
1.001001001001	0.457279004682812\\
1.01101101101101	0.458587172550681\\
1.02102102102102	0.459882168333082\\
1.03103103103103	0.461163996216284\\
1.04104104104104	0.462432662969273\\
1.05105105105105	0.463688177915343\\
1.06106106106106	0.464930552902947\\
1.07107107107107	0.466159802275862\\
1.08108108108108	0.467375942842655\\
1.09109109109109	0.468578993845497\\
1.1011011011011	0.469768976928341\\
1.11111111111111	0.470945916104479\\
1.12112112112112	0.472109837723511\\
1.13113113113113	0.473260770437745\\
1.14114114114114	0.47439874516805\\
1.15115115115115	0.475523795069184\\
1.16116116116116	0.476635955494626\\
1.17117117117117	0.477735263960931\\
1.18118118118118	0.478821760111624\\
1.19119119119119	0.479895485680677\\
1.2012012012012	0.480956484455564\\
1.21121121121121	0.482004802239947\\
1.22122122122122	0.483040486815984\\
1.23123123123123	0.48406358790632\\
1.24124124124124	0.485074157135742\\
1.25125125125125	0.486072247992557\\
1.26126126126126	0.487057915789695\\
1.27127127127127	0.488031217625566\\
1.28128128128128	0.488992212344694\\
1.29129129129129	0.489940960498147\\
1.3013013013013	0.490877524303791\\
1.31131131131131	0.491801967606388\\
1.32132132132132	0.492714355837551\\
1.33133133133133	0.493614755975594\\
1.34134134134134	0.494503236505285\\
1.35135135135135	0.495379867377528\\
1.36136136136136	0.496244719969\\
1.37137137137137	0.497097867041749\\
1.38138138138138	0.497939382702796\\
1.39139139139139	0.498769342363736\\
1.4014014014014	0.499587822700379\\
1.41141141141141	0.50039490161244\\
1.42142142142142	0.501190658183295\\
1.43143143143143	0.501975172639836\\
1.44144144144144	0.50274852631242\\
1.45145145145145	0.503510801594962\\
1.46146146146146	0.504262081905156\\
1.47147147147147	0.50500245164487\\
1.48148148148148	0.505731996160718\\
1.49149149149149	0.506450801704821\\
1.5015015015015	0.507158955395795\\
1.51151151151151	0.507856545179956\\
1.52152152152152	0.508543659792778\\
1.53153153153153	0.509220388720612\\
1.54154154154154	0.509886822162675\\
1.55155155155155	0.510543050993336\\
1.56156156156156	0.511189166724708\\
1.57157157157157	0.511825261469552\\
1.58158158158158	0.512451427904521\\
1.59159159159159	0.513067759233746\\
1.6016016016016	0.513674349152782\\
1.61161161161161	0.514271291812926\\
1.62162162162162	0.514858681785914\\
1.63163163163163	0.515436614029016\\
1.64164164164164	0.516005183850536\\
1.65165165165165	0.516564486875723\\
1.66166166166166	0.517114619013117\\
1.67167167167167	0.517655676421322\\
1.68168168168168	0.518187755476231\\
1.69169169169169	0.518710952738702\\
1.7017017017017	0.519225364922695\\
1.71171171171171	0.519731088863887\\
1.72172172172172	0.520228221488758\\
1.73173173173173	0.520716859784166\\
1.74174174174174	0.521197100767417\\
1.75175175175175	0.52166904145683\\
1.76176176176176	0.522132778842813\\
1.77177177177177	0.522588409859442\\
1.78178178178178	0.523036031356566\\
1.79179179179179	0.523475740072423\\
1.8018018018018	0.523907632606793\\
1.81181181181181	0.524331805394668\\
1.82182182182182	0.524748354680467\\
1.83183183183183	0.525157376492779\\
1.84184184184184	0.525558966619651\\
1.85185185185185	0.525953220584416\\
1.86186186186186	0.526340233622065\\
1.87187187187187	0.526720100656164\\
1.88188188188188	0.527092916276324\\
1.89189189189189	0.527458774716212\\
1.9019019019019	0.527817769832118\\
1.91191191191191	0.528169995082067\\
1.92192192192192	0.528515543505484\\
1.93193193193193	0.528854507703407\\
1.94194194194194	0.529186979819251\\
1.95195195195195	0.529513051520118\\
1.96196196196196	0.529832813978654\\
1.97197197197197	0.530146357855454\\
1.98198198198198	0.530453773282006\\
1.99199199199199	0.530755149844187\\
2.002002002002	0.53105057656628\\
2.01201201201201	0.531340141895549\\
2.02202202202202	0.531623933687329\\
2.03203203203203	0.531902039190661\\
2.04204204204204	0.532174545034443\\
2.05205205205205	0.532441537214112\\
2.06206206206206	0.532703101078843\\
2.07207207207207	0.532959321319259\\
2.08208208208208	0.533210281955663\\
2.09209209209209	0.533456066326762\\
2.1021021021021	0.533696757078906\\
2.11211211211211	0.533932436155816\\
2.12212212212212	0.534163184788802\\
2.13213213213213	0.534389083487477\\
2.14214214214214	0.534610212030941\\
2.15215215215215	0.534826649459442\\
2.16216216216216	0.535038474066514\\
2.17217217217217	0.535245763391566\\
2.18218218218218	0.535448594212937\\
2.19219219219219	0.535647042541399\\
2.2022022022022	0.5358411836141\\
2.21221221221221	0.53603109188895\\
2.22222222222222	0.536216841039436\\
2.23223223223223	0.536398503949859\\
2.24224224224224	0.536576152710991\\
2.25225225225225	0.53674985861614\\
2.26226226226226	0.536919692157618\\
2.27227227227227	0.537085723023609\\
2.28228228228228	0.537248020095417\\
2.29229229229229	0.537406651445103\\
2.3023023023023	0.537561684333494\\
2.31231231231231	0.537713185208556\\
2.32232232232232	0.537861219704129\\
2.33233233233233	0.538005852639008\\
2.34234234234234	0.53814714801638\\
2.35235235235235	0.538285169023578\\
2.36236236236236	0.538419978032184\\
2.37237237237237	0.538551636598439\\
2.38238238238238	0.538680205463971\\
2.39239239239239	0.538805744556829\\
2.4024024024024	0.538928312992817\\
2.41241241241241	0.53904796907711\\
2.42242242242242	0.539164770306164\\
2.43243243243243	0.539278773369891\\
2.44244244244244	0.539390034154109\\
2.45245245245245	0.539498607743248\\
2.46246246246246	0.539604548423307\\
2.47247247247247	0.539707909685057\\
2.48248248248248	0.539808744227485\\
2.49249249249249	0.53990710396146\\
2.5025025025025	0.54000304001363\\
2.51251251251251	0.540096602730522\\
2.52252252252252	0.540187841682867\\
2.53253253253253	0.540276805670105\\
2.54254254254254	0.540363542725101\\
2.55255255255255	0.540448100119035\\
2.56256256256256	0.540530524366477\\
2.57257257257257	0.540610861230631\\
2.58258258258258	0.540689155728746\\
2.59259259259259	0.540765452137683\\
2.6026026026026	0.540839793999639\\
2.61261261261261	0.540912224128003\\
2.62262262262262	0.540982784613368\\
2.63263263263263	0.541051516829656\\
2.64264264264264	0.541118461440385\\
2.65265265265265	0.541183658405043\\
2.66266266266266	0.541247146985577\\
2.67267267267267	0.541308965752995\\
2.68268268268268	0.541369152594058\\
2.69269269269269	0.541427744718077\\
2.7027027027027	0.541484778663789\\
2.71271271271271	0.541540290306324\\
2.72272272272272	0.541594314864244\\
2.73273273273273	0.541646886906655\\
2.74274274274274	0.541698040360385\\
2.75275275275275	0.541747808517225\\
2.76276276276276	0.541796224041225\\
2.77277277277277	0.541843318976036\\
2.78278278278278	0.541889124752306\\
2.79279279279279	0.541933672195109\\
2.8028028028028	0.541976991531413\\
2.81281281281281	0.542019112397582\\
2.82282282282282	0.5420600638469\\
2.83283283283283	0.542099874357116\\
2.84284284284284	0.542138571838018\\
2.85285285285285	0.542176183639006\\
2.86286286286286	0.542212736556682\\
2.87287287287287	0.542248256842447\\
2.88288288288288	0.542282770210099\\
2.89289289289289	0.542316301843421\\
2.9029029029029	0.542348876403773\\
2.91291291291291	0.542380518037672\\
2.92292292292292	0.542411250384357\\
2.93293293293293	0.542441096583336\\
2.94294294294294	0.542470079281918\\
2.95295295295295	0.542498220642723\\
2.96296296296296	0.542525542351158\\
2.97297297297297	0.542552065622873\\
2.98298298298298	0.542577811211184\\
2.99299299299299	0.542602799414453\\
3.003003003003	0.542627050083449\\
3.01301301301301	0.542650582628647\\
3.02302302302302	0.542673416027505\\
3.03303303303303	0.542695568831684\\
3.04304304304304	0.542717059174228\\
3.05305305305305	0.542737904776689\\
3.06306306306306	0.542758122956213\\
3.07307307307307	0.542777730632558\\
3.08308308308308	0.542796744335068\\
3.09309309309309	0.54281518020959\\
3.1031031031031	0.542833054025327\\
3.11311311311311	0.542850381181634\\
3.12312312312312	0.54286717671476\\
3.13313313313313	0.542883455304512\\
3.14314314314314	0.542899231280872\\
3.15315315315315	0.542914518630533\\
3.16316316316316	0.542929331003382\\
3.17317317317317	0.542943681718906\\
3.18318318318318	0.542957583772532\\
3.19319319319319	0.542971049841897\\
3.2032032032032	0.542984092293054\\
3.21321321321321	0.54299672318659\\
3.22322322322322	0.543008954283691\\
3.23323323323323	0.543020797052119\\
3.24324324324324	0.543032262672128\\
3.25325325325325	0.54304336204229\\
3.26326326326326	0.543054105785265\\
3.27327327327327	0.54306450425348\\
3.28328328328328	0.543074567534739\\
3.29329329329329	0.543084305457761\\
3.3033033033033	0.543093727597633\\
3.31331331331331	0.543102843281191\\
3.32332332332332	0.54311166159233\\
3.33333333333333	0.543120191377224\\
3.34334334334334	0.543128441249482\\
3.35335335335335	0.543136419595221\\
3.36336336336336	0.543144134578062\\
3.37337337337337	0.54315159414405\\
3.38338338338338	0.543158806026501\\
3.39339339339339	0.543165777750766\\
3.4034034034034	0.54317251663892\\
3.41341341341341	0.543179029814384\\
3.42342342342342	0.543185324206456\\
3.43343343343343	0.54319140655478\\
3.44344344344344	0.54319728341373\\
3.45345345345345	0.543202961156727\\
3.46346346346346	0.543208445980476\\
3.47347347347347	0.54321374390913\\
3.48348348348348	0.543218860798384\\
3.49349349349349	0.54322380233949\\
3.5035035035035	0.543228574063207\\
3.51351351351351	0.543233181343671\\
3.52352352352352	0.543237629402198\\
3.53353353353353	0.543241923311018\\
3.54354354354354	0.543246067996934\\
3.55355355355355	0.543250068244913\\
3.56356356356356	0.54325392870161\\
3.57357357357357	0.543257653878825\\
3.58358358358358	0.543261248156882\\
3.59359359359359	0.543264715787958\\
3.6036036036036	0.54326806089933\\
3.61361361361361	0.543271287496567\\
3.62362362362362	0.543274399466651\\
3.63363363363363	0.543277400581037\\
3.64364364364364	0.543280294498651\\
3.65365365365365	0.543283084768821\\
3.66366366366366	0.54328577483415\\
3.67367367367367	0.543288368033331\\
3.68368368368368	0.543290867603891\\
3.69369369369369	0.543293276684889\\
3.7037037037037	0.543295598319547\\
3.71371371371371	0.543297835457826\\
3.72372372372372	0.543299990958945\\
3.73373373373373	0.543302067593845\\
3.74374374374374	0.543304068047594\\
3.75375375375375	0.543305994921747\\
3.76376376376376	0.543307850736636\\
3.77377377377377	0.543309637933629\\
3.78378378378378	0.543311358877315\\
3.79379379379379	0.543313015857657\\
3.8038038038038	0.543314611092081\\
3.81381381381381	0.543316146727523\\
3.82382382382382	0.543317624842427\\
3.83383383383383	0.54331904744869\\
3.84384384384384	0.543320416493569\\
3.85385385385385	0.543321733861535\\
3.86386386386386	0.54332300137608\\
3.87387387387387	0.543324220801492\\
3.88388388388388	0.54332539384457\\
3.89389389389389	0.543326522156307\\
3.9039039039039	0.543327607333529\\
3.91391391391391	0.543328650920493\\
3.92392392392392	0.543329654410438\\
3.93393393393393	0.543330619247109\\
3.94394394394394	0.54333154682623\\
3.95395395395395	0.543332438496946\\
3.96396396396396	0.543333295563225\\
3.97397397397397	0.543334119285227\\
3.98398398398398	0.543334910880627\\
3.99399399399399	0.543335671525921\\
4.004004004004	0.543336402357675\\
4.01401401401401	0.543337104473762\\
4.02402402402402	0.543337778934551\\
4.03403403403403	0.543338426764071\\
4.04404404404404	0.543339048951139\\
4.05405405405405	0.543339646450464\\
4.06406406406406	0.543340220183711\\
4.07407407407407	0.543340771040549\\
4.08408408408408	0.543341299879652\\
4.09409409409409	0.54334180752969\\
4.1041041041041	0.543342294790282\\
4.11411411411411	0.543342762432924\\
4.12412412412412	0.54334321120189\\
4.13413413413413	0.543343641815113\\
4.14414414414414	0.543344054965029\\
4.15415415415415	0.543344451319414\\
4.16416416416416	0.543344831522178\\
4.17417417417417	0.543345196194148\\
4.18418418418418	0.543345545933829\\
4.19419419419419	0.543345881318129\\
4.2042042042042	0.543346202903084\\
4.21421421421421	0.543346511224541\\
4.22422422422422	0.543346806798832\\
4.23423423423423	0.543347090123426\\
4.24424424424424	0.543347361677561\\
4.25425425425425	0.543347621922853\\
4.26426426426426	0.543347871303895\\
4.27427427427427	0.543348110248828\\
4.28428428428428	0.543348339169903\\
4.29429429429429	0.543348558464018\\
4.3043043043043	0.543348768513245\\
4.31431431431431	0.543348969685337\\
4.32432432432432	0.543349162334218\\
4.33433433433433	0.543349346800462\\
4.34434434434434	0.543349523411753\\
4.35435435435435	0.543349692483332\\
4.36436436436436	0.543349854318427\\
4.37437437437437	0.543350009208676\\
4.38438438438438	0.543350157434529\\
4.39439439439439	0.543350299265637\\
4.4044044044044	0.543350434961239\\
4.41441441441441	0.543350564770519\\
4.42442442442442	0.543350688932968\\
4.43443443443443	0.543350807678724\\
4.44444444444444	0.543350921228901\\
4.45445445445445	0.543351029795914\\
4.46446446446446	0.543351133583783\\
4.47447447447447	0.543351232788436\\
4.48448448448448	0.543351327597996\\
4.49449449449449	0.54335141819306\\
4.5045045045045	0.543351504746969\\
4.51451451451451	0.543351587426065\\
4.52452452452452	0.543351666389946\\
4.53453453453453	0.543351741791702\\
4.54454454454454	0.543351813778154\\
4.55455455455455	0.543351882490077\\
4.56456456456456	0.543351948062413\\
4.57457457457457	0.543352010624487\\
4.58458458458458	0.543352070300201\\
4.59459459459459	0.543352127208234\\
4.6046046046046	0.543352181462224\\
4.61461461461461	0.543352233170948\\
4.62462462462462	0.543352282438501\\
4.63463463463463	0.543352329364451\\
4.64464464464464	0.543352374044009\\
4.65465465465465	0.543352416568176\\
4.66466466466466	0.543352457023894\\
4.67467467467467	0.543352495494184\\
4.68468468468468	0.543352532058286\\
4.69469469469469	0.543352566791783\\
4.7047047047047	0.543352599766732\\
4.71471471471471	0.543352631051778\\
4.72472472472472	0.543352660712272\\
4.73473473473473	0.543352688810377\\
4.74474474474474	0.543352715405174\\
4.75475475475475	0.543352740552761\\
4.76476476476476	0.543352764306346\\
4.77477477477477	0.54335278671634\\
4.78478478478478	0.543352807830441\\
4.79479479479479	0.543352827693715\\
4.8048048048048	0.543352846348674\\
4.81481481481481	0.543352863835349\\
4.82482482482482	0.543352880191357\\
4.83483483483483	0.543352895451969\\
4.84484484484484	0.54335290965017\\
4.85485485485485	0.543352922816715\\
4.86486486486486	0.543352934980184\\
4.87487487487487	0.543352946167032\\
4.88488488488488	0.543352956401631\\
4.89489489489489	0.543352965706322\\
4.9049049049049	0.543352974101443\\
4.91491491491491	0.543352981605375\\
4.92492492492492	0.543352988234566\\
4.93493493493493	0.543352994003567\\
4.94494494494494	0.543352998925051\\
4.95495495495495	0.543353003009841\\
4.96496496496496	0.543353006266928\\
4.97497497497497	0.543353008703484\\
4.98498498498498	0.543353010324877\\
4.99499499499499	0.54335301113468\\
5.00500500500501	0.54335301113468\\
5.01501501501502	0.543353010324877\\
5.02502502502503	0.543353008703484\\
5.03503503503504	0.543353006266928\\
5.04504504504505	0.543353003009841\\
5.05505505505506	0.543352998925051\\
5.06506506506507	0.543352994003567\\
5.07507507507508	0.543352988234566\\
5.08508508508509	0.543352981605375\\
5.0950950950951	0.543352974101443\\
5.10510510510511	0.543352965706322\\
5.11511511511512	0.543352956401631\\
5.12512512512513	0.543352946167032\\
5.13513513513514	0.543352934980184\\
5.14514514514515	0.543352922816715\\
5.15515515515516	0.54335290965017\\
5.16516516516517	0.543352895451969\\
5.17517517517518	0.543352880191357\\
5.18518518518519	0.543352863835349\\
5.1951951951952	0.543352846348674\\
5.20520520520521	0.543352827693715\\
5.21521521521522	0.543352807830441\\
5.22522522522523	0.54335278671634\\
5.23523523523524	0.543352764306346\\
5.24524524524525	0.543352740552761\\
5.25525525525526	0.543352715405174\\
5.26526526526527	0.543352688810377\\
5.27527527527528	0.543352660712272\\
5.28528528528529	0.543352631051778\\
5.2952952952953	0.543352599766732\\
5.30530530530531	0.543352566791783\\
5.31531531531532	0.543352532058286\\
5.32532532532533	0.543352495494184\\
5.33533533533534	0.543352457023894\\
5.34534534534535	0.543352416568176\\
5.35535535535536	0.543352374044009\\
5.36536536536537	0.543352329364451\\
5.37537537537538	0.543352282438501\\
5.38538538538539	0.543352233170948\\
5.3953953953954	0.543352181462224\\
5.40540540540541	0.543352127208234\\
5.41541541541542	0.543352070300201\\
5.42542542542543	0.543352010624487\\
5.43543543543544	0.543351948062413\\
5.44544544544545	0.543351882490077\\
5.45545545545546	0.543351813778154\\
5.46546546546547	0.543351741791702\\
5.47547547547548	0.543351666389946\\
5.48548548548549	0.543351587426065\\
5.4954954954955	0.543351504746969\\
5.50550550550551	0.54335141819306\\
5.51551551551552	0.543351327597996\\
5.52552552552553	0.543351232788436\\
5.53553553553554	0.543351133583783\\
5.54554554554555	0.543351029795914\\
5.55555555555556	0.543350921228901\\
5.56556556556557	0.543350807678724\\
5.57557557557558	0.543350688932968\\
5.58558558558559	0.543350564770519\\
5.5955955955956	0.543350434961239\\
5.60560560560561	0.543350299265637\\
5.61561561561562	0.543350157434529\\
5.62562562562563	0.543350009208676\\
5.63563563563564	0.543349854318427\\
5.64564564564565	0.543349692483332\\
5.65565565565566	0.543349523411753\\
5.66566566566567	0.543349346800462\\
5.67567567567568	0.543349162334218\\
5.68568568568569	0.543348969685337\\
5.6956956956957	0.543348768513245\\
5.70570570570571	0.543348558464018\\
5.71571571571572	0.543348339169903\\
5.72572572572573	0.543348110248828\\
5.73573573573574	0.543347871303895\\
5.74574574574575	0.543347621922853\\
5.75575575575576	0.543347361677561\\
5.76576576576577	0.543347090123426\\
5.77577577577578	0.543346806798832\\
5.78578578578579	0.543346511224541\\
5.7957957957958	0.543346202903084\\
5.80580580580581	0.543345881318129\\
5.81581581581582	0.543345545933829\\
5.82582582582583	0.543345196194148\\
5.83583583583584	0.543344831522178\\
5.84584584584585	0.543344451319414\\
5.85585585585586	0.543344054965029\\
5.86586586586587	0.543343641815113\\
5.87587587587588	0.54334321120189\\
5.88588588588589	0.543342762432924\\
5.8958958958959	0.543342294790282\\
5.90590590590591	0.54334180752969\\
5.91591591591592	0.543341299879652\\
5.92592592592593	0.543340771040549\\
5.93593593593594	0.543340220183711\\
5.94594594594595	0.543339646450464\\
5.95595595595596	0.543339048951139\\
5.96596596596597	0.543338426764071\\
5.97597597597598	0.543337778934551\\
5.98598598598599	0.543337104473762\\
5.995995995996	0.543336402357675\\
6.00600600600601	0.543335671525921\\
6.01601601601602	0.543334910880627\\
6.02602602602603	0.543334119285227\\
6.03603603603604	0.543333295563225\\
6.04604604604605	0.543332438496946\\
6.05605605605606	0.54333154682623\\
6.06606606606607	0.543330619247109\\
6.07607607607608	0.543329654410438\\
6.08608608608609	0.543328650920493\\
6.0960960960961	0.543327607333529\\
6.10610610610611	0.543326522156307\\
6.11611611611612	0.54332539384457\\
6.12612612612613	0.543324220801492\\
6.13613613613614	0.54332300137608\\
6.14614614614615	0.543321733861535\\
6.15615615615616	0.543320416493569\\
6.16616616616617	0.54331904744869\\
6.17617617617618	0.543317624842427\\
6.18618618618619	0.543316146727523\\
6.1961961961962	0.543314611092081\\
6.20620620620621	0.543313015857657\\
6.21621621621622	0.543311358877315\\
6.22622622622623	0.543309637933629\\
6.23623623623624	0.543307850736636\\
6.24624624624625	0.543305994921747\\
6.25625625625626	0.543304068047594\\
6.26626626626627	0.543302067593845\\
6.27627627627628	0.543299990958945\\
6.28628628628629	0.543297835457826\\
6.2962962962963	0.543295598319547\\
6.30630630630631	0.543293276684889\\
6.31631631631632	0.543290867603891\\
6.32632632632633	0.543288368033331\\
6.33633633633634	0.54328577483415\\
6.34634634634635	0.543283084768821\\
6.35635635635636	0.543280294498651\\
6.36636636636637	0.543277400581037\\
6.37637637637638	0.543274399466651\\
6.38638638638639	0.543271287496567\\
6.3963963963964	0.54326806089933\\
6.40640640640641	0.543264715787958\\
6.41641641641642	0.543261248156882\\
6.42642642642643	0.543257653878825\\
6.43643643643644	0.54325392870161\\
6.44644644644645	0.543250068244913\\
6.45645645645646	0.543246067996934\\
6.46646646646647	0.543241923311018\\
6.47647647647648	0.543237629402198\\
6.48648648648649	0.543233181343671\\
6.4964964964965	0.543228574063207\\
6.50650650650651	0.54322380233949\\
6.51651651651652	0.543218860798384\\
6.52652652652653	0.54321374390913\\
6.53653653653654	0.543208445980476\\
6.54654654654655	0.543202961156727\\
6.55655655655656	0.54319728341373\\
6.56656656656657	0.54319140655478\\
6.57657657657658	0.543185324206456\\
6.58658658658659	0.543179029814384\\
6.5965965965966	0.54317251663892\\
6.60660660660661	0.543165777750766\\
6.61661661661662	0.543158806026501\\
6.62662662662663	0.54315159414405\\
6.63663663663664	0.543144134578062\\
6.64664664664665	0.543136419595221\\
6.65665665665666	0.543128441249482\\
6.66666666666667	0.543120191377224\\
6.67667667667668	0.54311166159233\\
6.68668668668669	0.543102843281191\\
6.6966966966967	0.543093727597633\\
6.70670670670671	0.543084305457761\\
6.71671671671672	0.543074567534739\\
6.72672672672673	0.54306450425348\\
6.73673673673674	0.543054105785265\\
6.74674674674675	0.54304336204229\\
6.75675675675676	0.543032262672128\\
6.76676676676677	0.543020797052119\\
6.77677677677678	0.543008954283691\\
6.78678678678679	0.54299672318659\\
6.7967967967968	0.542984092293054\\
6.80680680680681	0.542971049841897\\
6.81681681681682	0.542957583772532\\
6.82682682682683	0.542943681718906\\
6.83683683683684	0.542929331003382\\
6.84684684684685	0.542914518630533\\
6.85685685685686	0.542899231280872\\
6.86686686686687	0.542883455304512\\
6.87687687687688	0.54286717671476\\
6.88688688688689	0.542850381181634\\
6.8968968968969	0.542833054025327\\
6.90690690690691	0.54281518020959\\
6.91691691691692	0.542796744335068\\
6.92692692692693	0.542777730632558\\
6.93693693693694	0.542758122956213\\
6.94694694694695	0.542737904776689\\
6.95695695695696	0.542717059174228\\
6.96696696696697	0.542695568831684\\
6.97697697697698	0.542673416027505\\
6.98698698698699	0.542650582628647\\
6.996996996997	0.542627050083449\\
7.00700700700701	0.542602799414453\\
7.01701701701702	0.542577811211184\\
7.02702702702703	0.542552065622873\\
7.03703703703704	0.542525542351158\\
7.04704704704705	0.542498220642723\\
7.05705705705706	0.542470079281918\\
7.06706706706707	0.542441096583336\\
7.07707707707708	0.542411250384357\\
7.08708708708709	0.542380518037672\\
7.0970970970971	0.542348876403773\\
7.10710710710711	0.542316301843421\\
7.11711711711712	0.542282770210099\\
7.12712712712713	0.542248256842447\\
7.13713713713714	0.542212736556682\\
7.14714714714715	0.542176183639006\\
7.15715715715716	0.542138571838018\\
7.16716716716717	0.542099874357116\\
7.17717717717718	0.5420600638469\\
7.18718718718719	0.542019112397582\\
7.1971971971972	0.541976991531413\\
7.20720720720721	0.541933672195109\\
7.21721721721722	0.541889124752306\\
7.22722722722723	0.541843318976036\\
7.23723723723724	0.541796224041225\\
7.24724724724725	0.541747808517225\\
7.25725725725726	0.541698040360385\\
7.26726726726727	0.541646886906655\\
7.27727727727728	0.541594314864244\\
7.28728728728729	0.541540290306324\\
7.2972972972973	0.541484778663789\\
7.30730730730731	0.541427744718077\\
7.31731731731732	0.541369152594058\\
7.32732732732733	0.541308965752995\\
7.33733733733734	0.541247146985577\\
7.34734734734735	0.541183658405043\\
7.35735735735736	0.541118461440385\\
7.36736736736737	0.541051516829656\\
7.37737737737738	0.540982784613368\\
7.38738738738739	0.540912224128003\\
7.3973973973974	0.540839793999639\\
7.40740740740741	0.540765452137683\\
7.41741741741742	0.540689155728746\\
7.42742742742743	0.540610861230631\\
7.43743743743744	0.540530524366477\\
7.44744744744745	0.540448100119035\\
7.45745745745746	0.540363542725101\\
7.46746746746747	0.540276805670105\\
7.47747747747748	0.540187841682867\\
7.48748748748749	0.540096602730522\\
7.4974974974975	0.54000304001363\\
7.50750750750751	0.53990710396146\\
7.51751751751752	0.539808744227485\\
7.52752752752753	0.539707909685057\\
7.53753753753754	0.539604548423307\\
7.54754754754755	0.539498607743248\\
7.55755755755756	0.539390034154109\\
7.56756756756757	0.539278773369891\\
7.57757757757758	0.539164770306164\\
7.58758758758759	0.53904796907711\\
7.5975975975976	0.538928312992817\\
7.60760760760761	0.538805744556829\\
7.61761761761762	0.538680205463971\\
7.62762762762763	0.538551636598439\\
7.63763763763764	0.538419978032184\\
7.64764764764765	0.538285169023578\\
7.65765765765766	0.53814714801638\\
7.66766766766767	0.538005852639008\\
7.67767767767768	0.537861219704129\\
7.68768768768769	0.537713185208556\\
7.6976976976977	0.537561684333494\\
7.70770770770771	0.537406651445103\\
7.71771771771772	0.537248020095417\\
7.72772772772773	0.537085723023609\\
7.73773773773774	0.536919692157618\\
7.74774774774775	0.53674985861614\\
7.75775775775776	0.536576152710991\\
7.76776776776777	0.536398503949859\\
7.77777777777778	0.536216841039436\\
7.78778778778779	0.53603109188895\\
7.7977977977978	0.5358411836141\\
7.80780780780781	0.535647042541399\\
7.81781781781782	0.535448594212937\\
7.82782782782783	0.535245763391566\\
7.83783783783784	0.535038474066514\\
7.84784784784785	0.534826649459442\\
7.85785785785786	0.534610212030941\\
7.86786786786787	0.534389083487477\\
7.87787787787788	0.534163184788802\\
7.88788788788789	0.533932436155816\\
7.8978978978979	0.533696757078906\\
7.90790790790791	0.533456066326762\\
7.91791791791792	0.533210281955663\\
7.92792792792793	0.532959321319259\\
7.93793793793794	0.532703101078843\\
7.94794794794795	0.532441537214112\\
7.95795795795796	0.532174545034443\\
7.96796796796797	0.531902039190661\\
7.97797797797798	0.531623933687329\\
7.98798798798799	0.531340141895549\\
7.997997997998	0.53105057656628\\
8.00800800800801	0.530755149844187\\
8.01801801801802	0.530453773282006\\
8.02802802802803	0.530146357855453\\
8.03803803803804	0.529832813978654\\
8.04804804804805	0.529513051520118\\
8.05805805805806	0.529186979819251\\
8.06806806806807	0.528854507703407\\
8.07807807807808	0.528515543505484\\
8.08808808808809	0.528169995082067\\
8.0980980980981	0.527817769832118\\
8.10810810810811	0.527458774716212\\
8.11811811811812	0.527092916276324\\
8.12812812812813	0.526720100656164\\
8.13813813813814	0.526340233622065\\
8.14814814814815	0.525953220584416\\
8.15815815815816	0.525558966619651\\
8.16816816816817	0.525157376492779\\
8.17817817817818	0.524748354680467\\
8.18818818818819	0.524331805394668\\
8.1981981981982	0.523907632606793\\
8.20820820820821	0.523475740072423\\
8.21821821821822	0.523036031356566\\
8.22822822822823	0.522588409859442\\
8.23823823823824	0.522132778842813\\
8.24824824824825	0.52166904145683\\
8.25825825825826	0.521197100767417\\
8.26826826826827	0.520716859784166\\
8.27827827827828	0.520228221488758\\
8.28828828828829	0.519731088863887\\
8.2982982982983	0.519225364922695\\
8.30830830830831	0.518710952738702\\
8.31831831831832	0.518187755476231\\
8.32832832832833	0.517655676421322\\
8.33833833833834	0.517114619013117\\
8.34834834834835	0.516564486875723\\
8.35835835835836	0.516005183850536\\
8.36836836836837	0.515436614029016\\
8.37837837837838	0.514858681785914\\
8.38838838838839	0.514271291812926\\
8.3983983983984	0.513674349152782\\
8.40840840840841	0.513067759233746\\
8.41841841841842	0.512451427904521\\
8.42842842842843	0.511825261469552\\
8.43843843843844	0.511189166724708\\
8.44844844844845	0.510543050993336\\
8.45845845845846	0.509886822162674\\
8.46846846846847	0.509220388720612\\
8.47847847847848	0.508543659792778\\
8.48848848848849	0.507856545179956\\
8.4984984984985	0.507158955395795\\
8.50850850850851	0.506450801704821\\
8.51851851851852	0.505731996160718\\
8.52852852852853	0.50500245164487\\
8.53853853853854	0.504262081905156\\
8.54854854854855	0.503510801594962\\
8.55855855855856	0.50274852631242\\
8.56856856856857	0.501975172639836\\
8.57857857857858	0.501190658183295\\
8.58858858858859	0.50039490161244\\
8.5985985985986	0.499587822700379\\
8.60860860860861	0.498769342363736\\
8.61861861861862	0.497939382702796\\
8.62862862862863	0.497097867041749\\
8.63863863863864	0.496244719969\\
8.64864864864865	0.495379867377528\\
8.65865865865866	0.494503236505285\\
8.66866866866867	0.493614755975594\\
8.67867867867868	0.492714355837551\\
8.68868868868869	0.491801967606388\\
8.6986986986987	0.490877524303791\\
8.70870870870871	0.489940960498147\\
8.71871871871872	0.488992212344694\\
8.72872872872873	0.488031217625566\\
8.73873873873874	0.487057915789695\\
8.74874874874875	0.486072247992557\\
8.75875875875876	0.485074157135742\\
8.76876876876877	0.48406358790632\\
8.77877877877878	0.483040486815984\\
8.78878878878879	0.482004802239947\\
8.7987987987988	0.480956484455564\\
8.80880880880881	0.479895485680677\\
8.81881881881882	0.478821760111624\\
8.82882882882883	0.477735263960931\\
8.83883883883884	0.476635955494626\\
8.84884884884885	0.475523795069184\\
8.85885885885886	0.47439874516805\\
8.86886886886887	0.473260770437745\\
8.87887887887888	0.472109837723511\\
8.88888888888889	0.470945916104479\\
8.8988988988989	0.469768976928341\\
8.90890890890891	0.468578993845497\\
8.91891891891892	0.467375942842655\\
8.92892892892893	0.466159802275862\\
8.93893893893894	0.464930552902947\\
8.94894894894895	0.463688177915343\\
8.95895895895896	0.462432662969273\\
8.96896896896897	0.461163996216283\\
8.97897897897898	0.459882168333082\\
8.98898898898899	0.458587172550681\\
8.998998998999	0.457279004682812\\
9.00900900900901	0.455957663153588\\
9.01901901901902	0.454623149024402\\
9.02902902902903	0.453275466020026\\
9.03903903903904	0.451914620553905\\
9.04904904904905	0.450540621752613\\
9.05905905905906	0.449153481479461\\
9.06906906906907	0.447753214357219\\
9.07907907907908	0.446339837789961\\
9.08908908908909	0.444913371983982\\
9.0990990990991	0.443473839967791\\
9.10910910910911	0.44202126761115\\
9.11911911911912	0.440555683643141\\
9.12912912912913	0.439077119669255\\
9.13913913913914	0.437585610187463\\
9.14914914914915	0.436081192603274\\
9.15915915915916	0.434563907243752\\
9.16916916916917	0.433033797370479\\
9.17917917917918	0.431490909191447\\
9.18918918918919	0.429935291871862\\
9.1991991991992	0.428366997543857\\
9.20920920920921	0.426786081315079\\
9.21921921921922	0.425192601276159\\
9.22922922922923	0.423586618507035\\
9.23923923923924	0.421968197082126\\
9.24924924924925	0.420337404074343\\
9.25925925925926	0.41869430955792\\
9.26926926926927	0.417038986610063\\
9.27927927927928	0.4153715113114\\
9.28928928928929	0.413691962745234\\
9.2992992992993	0.412000422995571\\
9.30930930930931	0.410296977143937\\
9.31931931931932	0.408581713264962\\
9.32932932932933	0.406854722420727\\
9.33933933933934	0.405116098653878\\
9.34934934934935	0.403365938979485\\
9.35935935935936	0.401604343375659\\
9.36936936936937	0.39983141477291\\
9.37937937937938	0.39804725904225\\
9.38938938938939	0.396251984982036\\
9.3993993993994	0.39444570430356\\
9.40940940940941	0.392628531615365\\
9.41941941941942	0.390800584406311\\
9.42942942942943	0.388961983027379\\
9.43943943943944	0.38711285067221\\
9.44944944944945	0.385253313356401\\
9.45945945945946	0.383383499895533\\
9.46946946946947	0.381503541881962\\
9.47947947947948	0.37961357366036\\
9.48948948948949	0.377713732302018\\
9.4994994994995	0.375804157577918\\
9.50950950950951	0.37388499193058\\
9.51951951951952	0.371956380444687\\
9.52952952952953	0.370018470816507\\
9.53953953953954	0.36807141332211\\
9.54954954954955	0.366115360784399\\
9.55955955955956	0.364150468538962\\
9.56956956956957	0.362176894398753\\
9.57957957957958	0.360194798617624\\
9.58958958958959	0.35820434385271\\
9.5995995995996	0.356205695125692\\
9.60960960960961	0.354199019782942\\
9.61961961961962	0.352184487454573\\
9.62962962962963	0.350162270012406\\
9.63963963963964	0.348132541526876\\
9.64964964964965	0.346095478222884\\
9.65965965965966	0.344051258434626\\
9.66966966966967	0.34200006255941\\
9.67967967967968	0.339942073010482\\
9.68968968968969	0.337877474168883\\
9.6996996996997	0.335806452334352\\
9.70970970970971	0.333729195675309\\
9.71971971971972	0.331645894177926\\
9.72972972972973	0.329556739594317\\
9.73973973973974	0.327461925389872\\
9.74974974974975	0.325361646689746\\
9.75975975975976	0.323256100224548\\
9.76976976976977	0.321145484275239\\
9.77977977977978	0.319029998617265\\
9.78978978978979	0.316909844463968\\
9.7997997997998	0.314785224409276\\
9.80980980980981	0.312656342369723\\
9.81981981981982	0.310523403525807\\
9.82982982982983	0.308386614262737\\
9.83983983983984	0.306246182110565\\
9.84984984984985	0.304102315683767\\
9.85985985985986	0.301955224620282\\
9.86986986986987	0.299805119520032\\
9.87987987987988	0.297652211882983\\
9.88988988988989	0.295496714046737\\
9.8998998998999	0.293338839123727\\
9.90990990990991	0.291178800938009\\
9.91991991991992	0.289016813961707\\
9.92992992992993	0.286853093251135\\
9.93993993993994	0.284687854382627\\
9.94994994994995	0.282521313388101\\
9.95995995995996	0.280353686690401\\
9.96996996996997	0.278185191038444\\
9.97997997997998	0.276016043442201\\
9.98998998998999	0.273846461107548\\
10	0.271676661371016\\
};
\addlegendentry{$\mu(x)$};

\end{axis}
\end{tikzpicture}%
  \caption{Above: a Gaussian process prior on a function $f$ with mean
    zero and squared exponential covariance.  Below: the posterior
    distribution on $f$ after conditioning on the obesrvation
    $\int_{0}^{10} f(x) \intd{x} = 5$.  The posterior samples all have
    integral identically equal to 5.}
  \label{example}
\end{figure}

\section*{Bayesian Quadrature}

Above, we conditioned a Gaussian process on an integral observation.
In \emph{Bayesian quadrature,} we do the opposite: given (potentially
noisy) observations of a function $\data = (\mat{X}, \vec{y})$, we
perform inference about an integral of interest, for example the
expectation of $f$ under a distribution $p$:
\begin{equation*}
  I_p[f] = \int f(x) p(x) \intd{x}.
\end{equation*}
The traditional method for estimating integrals of this form is
\emph{Monte Carlo} estimation, where we sample some points $\{ x_i
\}_{i = 1}^N$ from the distribution $p(x)$ and estimate
\begin{equation*}
  \int f(x) p(x) \intd{x}
  \approx
  \sum_{i = 1}^N f(x_i).
\end{equation*}

In Bayesian quadrature, we place a Gaussian process prior on $f$,
which we condition on the observations $\data$.  Notice that the input
locations $\mat{X}$ do not need to be random samples from $p$, but
rather we are allowed to evaluate $f$ anywhere.  The result is the posterior
\begin{equation*}
  p(f \given \data)
  =
  \mc{GP}(f; \mu_{f \given \data}, K_{f \given \data}).
\end{equation*}
Following the above, we may also derive the posterior distribution
of the expectation $I_p[f]:$
\begin{equation*}
  p\bigl(I_p[f] \given \data \bigr)
  =
  \mc{N}
  \biggl(
  I_p[f];
  \int \mu_{f \given \data}(x) p(x) \intd{x},
  \iint K_{f \given \data}(x, x') p(x) p(x') \intd{x} \intd{x'}
  \biggr).
\end{equation*}
For some choices of the prior prior mean and covariance functions
$\mu$ and $K$ and the distribution $p$, we may compute the required
integrals exactly, giving a closed-form expression for the posterior
distribution of the integral of interest.

Why is this useful?  The main advantages to this approach are that we
may explicitly model the structure of $f$ via the covariance function
$K$, and that the posterior variance of the integral may be used to
derive an active sampling scheme, revealing the most-informative
points to evaluate the function so as to estimate the integral with
the highest precision.  Note that the posterior variance of the
integral only depends on where we sample the function, and not the
actual values we observe.  This property can be exploited to design
optimal quadrature rules.

\end{document}
