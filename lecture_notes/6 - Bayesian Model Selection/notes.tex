\documentclass{article}

\usepackage[T1]{fontenc}
\usepackage[osf]{libertine}
\usepackage[scaled=0.8]{beramono}
\usepackage[margin=1.5in]{geometry}
\usepackage{url}
\usepackage{booktabs}
\usepackage{amsmath}
\usepackage{amssymb}
\usepackage{nicefrac}
\usepackage{microtype}
\usepackage{subcaption}
\usepackage{bm}

\usepackage{sectsty}
\sectionfont{\large}
\subsectionfont{\normalsize}

\usepackage{titlesec}
\titlespacing{\section}{0pt}{10pt plus 2pt minus 2pt}{0pt plus 2pt minus 0pt}
\titlespacing{\subsection}{0pt}{5pt plus 2pt minus 2pt}{0pt plus 2pt minus 0pt}

\usepackage{pgfplots}
\pgfplotsset{
  compat=newest,
  plot coordinates/math parser=false,
  tick label style={font=\footnotesize, /pgf/number format/fixed},
  label style={font=\small},
  legend style={font=\small},
  every axis/.append style={
    tick align=outside,
    clip mode=individual,
    scaled ticks=false,
    thick,
    tick style={semithick, black}
  }
}

\pgfkeys{/pgf/number format/.cd, set thousands separator={\,}}

\usepgfplotslibrary{external}
\tikzexternalize[prefix=tikz/]

\newlength\figurewidth
\newlength\figureheight

\setlength{\figurewidth}{8cm}
\setlength{\figureheight}{6cm}

\newlength\squarefigurewidth
\newlength\squarefigureheight

\setlength{\squarefigurewidth}{4cm}
\setlength{\squarefigureheight}{4cm}

\newlength\smallsquarefigurewidth
\newlength\smallsquarefigureheight

\setlength{\smallsquarefigurewidth}{3.25cm}
\setlength{\smallsquarefigureheight}{3.25cm}

\newlength\smallfigurewidth
\newlength\smallfigureheight

\setlength{\smallfigurewidth}{6.25cm}
\setlength{\smallfigureheight}{4cm}

\setlength{\parindent}{0pt}
\setlength{\parskip}{1ex}

\newcommand{\acro}[1]{\textsc{\MakeLowercase{#1}}}
\newcommand{\given}{\mid}
\newcommand{\mc}[1]{\mathcal{#1}}
\newcommand{\data}{\mc{D}}
\newcommand{\model}{\mc{M}}
\newcommand{\intd}[1]{\,\mathrm{d}{#1}}
\newcommand{\inv}{^{-1}}
\newcommand{\trans}{^\top}
\newcommand{\mat}[1]{\bm{\mathrm{#1}}}
\renewcommand{\vec}[1]{\bm{\mathrm{#1}}}
\newcommand{\R}{\mathbb{R}}
\renewcommand{\epsilon}{\varepsilon}

\DeclareMathOperator{\var}{var}
\DeclareMathOperator{\cov}{cov}
\DeclareMathOperator{\diag}{diag}
\DeclareMathOperator*{\argmin}{arg\,min}
\DeclareMathOperator*{\argmax}{arg\,max}

\begin{document}

\section*{Bayesian model selection}

Consider the regression problem, where we want to predict the values
of an unknown function $y\colon \R^d \to \R$ given examples $\data =
\bigl\{ (\vec{x}_i, y_i) \bigr\}_{i = 1}^N$ to serve as training data.
In Bayesian linear regression, we made the following assumption about
$y(\vec{x})$:
\begin{equation}
  y(\vec{x}) = \phi(\vec{x})\trans \vec{w} + \epsilon(\vec{x}),
\end{equation}
where $\phi(\vec{x})$ is a now explicitly-written feature expansion of
$\vec{x}$.  We proceed in the normal Bayesian way: we place Gaussian
priors on our unknowns, the parameters $\vec{w}$ and the residuals
$\vec{\epsilon}$, then derive the posterior distribution over
$\vec{w}$ given $\data$, which we use to make predictions.

One question left unanswered is how to choose a good feature expansion
function $\phi(\vec{x})$.  For example, a purely linear model could
use $\phi(\vec{x}) = [1, \vec{x}]\trans$, whereas a quadratic model
could use $\phi(\vec{x}) = [1, \vec{x}, \vec{x}^2]\trans$, etc.  In
general, arbitrary feature expansions $\phi$ are allowed.  How can I
select between them?  Even more generally, how do I select whether I
should use linear regression or a completely different probabilistic
model to explain my data?  These are questions of \emph{model
  selection,} and naturally there is a Bayesian approach to it.

Before we continue our discussion of model selection, we will first
define the word \emph{model,} which is often used loosely without
explicit definition.  A model is a parametric family of probability
distributions, each of which can explain the observed data.  Another
way to explain the concept of a model is that if we have chosen a
likelihood $p(\data \given \theta)$ for our data, which depends on a
parameter $\theta$, then the model is the set of all likelihoods (each
one of which is a distribution over $\data$) for every possible value
of the parameter $\theta$.

In the case of linear regression, the weight vector $\vec{w}$ defines
the parametric family, and the model is the set of distributions
\begin{equation*}
  \bigl\{ p(\vec{y} \given \mat{X}, \vec{w}, \sigma^2) \bigr\}
  =
  \bigl\{ \mc{N}(\vec{y}; \mat{X}\vec{w}, \sigma^2 \mat{I}) \bigr\},
\end{equation*}
indexed by all possible $\vec{w}$.  Each one of these is a potential
explanation of the observed values $\vec{y}$ given $\vec{X}$.  In the
case of flipping a coin $n$ times with an unknown bias $\theta$ and
observing the number of heads $x$, the model is
\begin{equation*}
  \bigl\{ p(x \given n, \theta) \bigr\}
  =
  \bigl\{ \mathrm{Binomial}(x, n, \theta) \bigr\},
\end{equation*}
where there is one binomial distribution for every possible $\theta
\in (0, 1)$.  In the Bayesian method, we maintain a belief over which
elements in the model we consider plausible by reasoning about
$p(\theta \given \data)$ via Bayes' theorem.

Suppose now that I have at my disposal a finite set of models
$\{\model_i\}_i$ that I may use to explain my observed data $\data$,
and let us write $\theta_i$ for the parameters of model $\model_i$.
How do we know which model to prefer?  We work out the posterior
probability over the models via Bayes' theorem!  We have:
\begin{equation*}
  \Pr(\model_i \given \data)
  =
  \frac{p(\data \given \model_i)\Pr(\model_i)}
       {\sum_j p(\data \given \model_j)\Pr(\model_j)}.
\end{equation*}
Here $\Pr(\model_i)$ is a prior distribution over models that we have
selected; a common practice is to set this to a uniform distribution
over the models.  The value $p(\data \given \model_i)$ may also be
written in a more-familiar familiar form:
\begin{equation*}
  p(\data \given \model_i)
  =
  \int
  p(\data \given \theta_i, \model_i)
  p(\theta_i \given \model_i)
  \intd{\theta_i}.
\end{equation*}
This is exactly the denominator when applying Bayes' theorem to find
the posterior $p(\theta_i \given \data, \model_i)$!
\begin{equation*}
  p(\theta_i \given \data, \model_i)
  =
  \frac{p(\data \given \theta_i, \model_i) p(\theta_i \given \model_i)}
       {\int
         p(\data \given \theta_i, \model_i)
         p(\theta_i \given \model_i)
         \intd{\theta_i}.
       }
  =
  \frac{p(\data \given \theta_i, \model_i) p(\theta_i \given \model_i)}
       {p(\data \given \model_i)},
\end{equation*}
where we have simply conditioned on $\model_i$ to be explicit.  In the
context of model selection, the term $p(\data \given \model_i)$ is
known as the \emph{model evidence} or simply \emph{evidence.}  One
interpretation of the model evidence is the probability that your
model could have generated the observed data, under the chosen prior
belief over its parameters $\theta_i$.

Suppose now that we have exactly two models for the observed data that
we wish to compare: $\model_1$ and $\model_2$, with corresponding
parameter vectors $\theta_1$ and $\theta_2$ and prior probabilities
$\Pr(\model_1)$ and $\Pr(\model_2)$. In this case it is easiest to
compute the \emph{posterior odds}, the ratio of the models'
probabilities given the data:
\begin{equation*}
  \frac{\Pr(\model_1 \given \data)}
       {\Pr(\model_2 \given \data)}
  = \frac{\Pr(\model_1) p(\data \given \model_1)}
         {\Pr(\model_2) p(\data \given \model_2)}
  = \frac{\Pr(\model_1) \int p(\data \given \theta_1, \model_1)
          p(\theta_1 \given \model_1) \intd{\theta_1}}
         {\Pr(\model_2) \int p(\data \given \theta_2, \model_2)
          p(\theta_2 \given \model_2) \intd{\theta_2}},
\end{equation*}
which is simply the prior odds multiplied by the ratio of the evidence
for each model.  The latter quantity is also called the \emph{Bayes
  factor} in favor of $\model_1$.  Publishing Bayes factors allows
another practitioner to easily substitute their own model priors and
derive their own conclusions about the models being considered.

\subsection*{Example}

Wikipedia gives a truly excellent example of Bayesian model selection
in
practice.\footnote{\url{http://en.wikipedia.org/wiki/Bayes_factor#Example}}
Suppose I am presented with a coin and want to compare two models for
explaining its behavior.  The first model, $\model_1$, assumes that
the heads probability is fixed to $\nicefrac{1}{2}$.  Notice that this
model does not have any parameters.  The second model, $\model_2$,
assumes that the heads probability is fixed to an unknown value
$\theta \in (0, 1)$, with a uniform prior on $\theta$: $p(\theta
\given \model_2) = 1$ (this is equivalent to a beta prior on $\theta$
with $\alpha = \beta = 1$).  For simplicity, we choose a uniform model
prior: $\Pr(\model_1) = \Pr(\model_2) = \nicefrac{1}{2}$.

Suppose we flip the coin $n = 200$ times and observe $x = 115$ heads.
Which model should we prefer in light of this data?  We compute the
model evidence for each model.  The model evidence for $\model_1$ is
quite straightforward, as it has no parameters:
\begin{equation*}
  \Pr(x \given n, \model_1)
  =
  \mathrm{Binomial}(n, x, \nicefrac{1}{2})
  =
  \binom{200}{115}
  \frac{1}{2^{200}}
  \approx
  0.005956.
\end{equation*}
The model evidence for $\model_2$ requires integrating over the
parameter $\theta$:
\begin{align*}
  \Pr(x \given n, \model_2)
  &=
  \int
  \Pr(x \given n, \theta, \model_2)
  p(\theta \given \model2)
  \intd{\theta}
  \\
  &=
  \int_{0}^{1}
  \binom{200}{115}
  \theta^{115}
  (1 - \theta)^{200 - 115}
  \intd{\theta}
  \\
  &=
  \frac{1}{201}
  \approx
  0.004975.
\end{align*}
The Bayes factor in favor of $\model_1$ is approximately $1.2$, so the
data give very weak evidence in favor of the simpler model $\model_1$.

An interesting aside here is that a frequentist hypothesis test would
reject the null hypothesis $\theta = \frac{1}{2}$ at the $\alpha =
0.05$ level. The probability of generating at least 115 heads under
model $\model_1$ is approximately $0.02$ (similarly, the probability
of generating at least 115 tails is also $0.02$), so a two-sided test
would give a $p$-value of approximately $4\%$.

\subsection*{Occam's razor}

One spin on Bayesian decision theory is that it automatically gives a
preference towards simpler models, in line with Occam's razor.  One
way to see this is to consider the model evidence $p(\data \given
\model)$ as a probability distribution over datasets $\data$.  More
complex models can explain more datasets, so the support of this
distribution is wider in the sample space.  But note that the
distribution must normalize over the sample space as well, so we pay a
price for generality.  When moving from a simpler model to a more
complex model, the probability of some datasets that are well
explained by the simpler model must inevitably decrease to ``give up''
probability mass for the newly explained datasets in the widened
support of the more-complex model.  The model selection process then
drives us to select the model that is ``just complex enough'' to
explain the data at hand, an in-build Occam's razor.

In the coin flipping example above, model $\model_1$ can only explain
datasets with empirical heads probability reasonably near
$\frac{1}{2}$.  An observation of 200 heads, for example, would have
astronomically small probability under this model.  The second model
$\model_2$ can explain \emph{any} set of observations by selecting an
appropriate $\theta$.  The price for this generality, though, is that
datasets with a roughly equal number of heads and tails have a smaller
prior probability under the model than before.

\subsection*{Model selection for Bayesian linear regression}

A common application for model selection is for selecting between
feature expansion functions $\phi(\vec{x})$ in Bayesian linear
regression.  Here the model $\model_i$ could for example correspond
to order-$i$ polynomial regression with
\begin{equation*}
  \phi_i(\vec{x})
  =
  [1, \vec{x}, \vec{x}^2, \dotsc \vec{x}^i]\trans.
\end{equation*}
After selecting a set of these models to compare, as well as a prior
probability for each, the only remaining task is to compute the
evidence for each model in observed data $(\mat{X}, \vec{y})$.  In our
discussion of Bayesian linear regression, we have actually already
computed the desired quantity:
\begin{equation*}
  p(\vec{y} \given \mat{X}, \sigma^2, \model_i)
  =
  \mc{N}\bigl(\vec{y};
  \phi_i(\mat{X})\vec{\mu},
  \phi_i(\mat{X})\mat{\Sigma}\phi_i(\mat{X})\trans + \sigma^2\mat{I}\bigr),
\end{equation*}
where I have explicitly written the basis expansion in $\phi_i$.

Note that the model $\phi_i$ can also easily explain all datasets
well-explained by the models $\phi_j$ for $j < i$, by simply setting
the weights on higher-order terms to zero.  Again, however, the
simpler model will be preferred due the Occam's razor effect described
above.

\section*{Bayesian Model Averaging}

Note that a ``full Bayesian'' treatment of a problem would eschew
model selection entirely.  Instead, when making predictions, we should
theoretically use the sum rule to marginalize the unknown model, e.g.:
\begin{equation*}
  p(y_\ast \given \vec{x}_\ast, \data)
  =
  \sum_i
  p(y_\ast \given \vec{x}_\ast, \data, \model_i)
  \Pr(\model_i \given \data).
\end{equation*}
Such an approach is called \emph{Bayesian model averaging.}  Although
this is sometimes seen, model selection is still used widely in
practice.  The reason is that the computational overhead of using a
single model is much lower than having to continually retrain multiple
models, and that Bayesian model averaging uses a mixture distribution
for predictions, which can have annoying analytic properties (for
example, the predictive distribution could be multimodal).

\end{document}
