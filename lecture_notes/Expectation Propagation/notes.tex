\documentclass{article}

\usepackage[T1]{fontenc}
\usepackage[osf]{libertine}
\usepackage[scaled=0.8]{beramono}
\usepackage[margin=1.5in]{geometry}
\usepackage{url}
\usepackage{booktabs}
\usepackage{amsmath}
\usepackage{amssymb}
\usepackage{nicefrac}
\usepackage{microtype}
\usepackage{subcaption}
\usepackage{bm}

\usepackage{amsthm}
\newtheorem{defn}{Definition}

\usepackage{sectsty}
\sectionfont{\large}
\subsectionfont{\normalsize}

\usepackage{titlesec}
\titlespacing{\section}{0pt}{10pt plus 2pt minus 2pt}{0pt plus 2pt minus 0pt}
\titlespacing{\subsection}{0pt}{5pt plus 2pt minus 2pt}{0pt plus 2pt minus 0pt}

\usepackage{pgfplots}
\pgfplotsset{
  compat=newest,
  plot coordinates/math parser=false,
  tick label style={font=\footnotesize, /pgf/number format/fixed},
  label style={font=\small},
  legend style={font=\small},
  every axis/.append style={
    tick align=outside,
    clip mode=individual,
    scaled ticks=false,
    thick,
    tick style={semithick, black}
  }
}

\pgfkeys{/pgf/number format/.cd, set thousands separator={\,}}

\usepgfplotslibrary{external}
\tikzexternalize[prefix=tikz/]

\newlength\figurewidth
\newlength\figureheight

\setlength{\figurewidth}{12cm}
\setlength{\figureheight}{6cm}

\newlength\squarefigurewidth
\newlength\squarefigureheight

\setlength{\squarefigurewidth}{4cm}
\setlength{\squarefigureheight}{4cm}

\newlength\smallsquarefigurewidth
\newlength\smallsquarefigureheight

\setlength{\smallsquarefigurewidth}{3.25cm}
\setlength{\smallsquarefigureheight}{3.25cm}

\newlength\smallfigurewidth
\newlength\smallfigureheight

\setlength{\smallfigurewidth}{6.25cm}
\setlength{\smallfigureheight}{4cm}

\setlength{\parindent}{0pt}
\setlength{\parskip}{1ex}

\newcommand{\acro}[1]{\textsc{\MakeLowercase{#1}}}
\newcommand{\given}{\mid}
\newcommand{\mc}[1]{\mathcal{#1}}
\newcommand{\data}{\mc{D}}
\newcommand{\intd}[1]{\,\mathrm{d}{#1}}
\newcommand{\inv}{^{-1}}
\newcommand{\trans}{^\top}
\newcommand{\mat}[1]{\bm{\mathrm{#1}}}
\renewcommand{\vec}[1]{\bm{\mathrm{#1}}}
\newcommand{\R}{\mathbb{R}}
\renewcommand{\epsilon}{\varepsilon}
\newcommand{\Exp}{\mathbb{E}}

\DeclareMathOperator{\var}{var}
\DeclareMathOperator{\cov}{cov}
\DeclareMathOperator{\diag}{diag}
\DeclareMathOperator*{\argmin}{arg\,min}
\DeclareMathOperator*{\argmax}{arg\,max}

\begin{document}

\section*{Gaussian Process Classification}

Just as we could use the kernel trick to extend Bayesian linear
regression to Gaussian processes for general-purpose nonlinear
regression, we may also extend Bayesian linear classification in the
same way.  In Gaussian process classification, we assume there is a
latent function $f\colon \mc{X} \to \R$ that is commensurate with the
probability of a positive observation; higher latent function values
correspond to higher probabilities of positive observations.  In
Bayesian linear classification, we assumed a parametric (linear) form
for this latent function:
\[
  f(\vec{x}) = \vec{x}\trans \vec{w}.
\]
In Gaussian process classification, rather than choosing a parametric
form for $f$, we instead place a Gaussian process prior on $f$:
\[
  p(f) = \mc{GP}(f; \mu, K).
\]
Note that a Gaussian prior on the weight vector $\vec{w}$ above
induces a Gaussian process prior on $f$ with mean function
\[
  \mu(\vec{x}) = \vec{x}\trans \vec{\mu}
\]
and covariance function
\[
  K(\vec{x}, \vec{x}') = \vec{x}\trans \mat{\Sigma} \vec{x},
\]
where $p(\vec{w}) = \mc{N}(\vec{w}; \vec{\mu}, \mat{\Sigma})$. The
Gaussian process formalism allows us to model arbitrary nonlinear
classification boundaries by using any desired mean and covariance
function for $f$.

\subsection*{Likelihood}

Suppose we have made binary observations at a set of values $\mat{X}$,
and define $\vec{f} = f(\mat{X})$ to be the associated set of latent
function values.  As in Bayesian linear classification, we assume
the following likelihood for a given binary observation $y_i$
associated with $\vec{x}_i$:
\[
  p(y_i = 1 \given f_i)
  =
  \sigma(f_i),
\]
where $\sigma\colon \R \to (0, 1)$ is a monotonically increasing
sigmoid function such as the logisitic function or the standard normal
\acro{CDF}.  We again assume the observations are conditionally
independent given the latent function values:
\[
  p(\vec{y} \given \vec{f})
  =
  \prod p(y_i \given f_i).
\]

\subsection*{Inference}

Given our prior $p(f) = \mc{GP}(f; \mu, K)$ and a set of observations
$\data = (\mat{X}, \vec{y})$, we wish to find the posterior
distribution of the latent function values $\vec{f} = f(\mat{X})$.
Note that if we had a Gaussian posterior $\vec{f}$, this would induce
a Gaussian process posterior for the function $f$ given $\data$.
The posterior is
\[
  p(\vec{f} \given \data)
  =
  \frac{1}{Z}
  p(\vec{f} \given \mat{X})
  p(\vec{y} \given \vec{f})
  =
  \mc{N}\bigl(\vec{f}; \mu(\mat{X}), K(\mat{X}, \mat{X})\bigr)
  \prod_i p(y_i \given f_i).
\]
Unfortunately, the sigmoid likelihood coupled with the Gaussian prior
do not couple to form a tractable posterior.  Instead, we must
approximate this posterior in some way.  Previously we described the
Laplace approximation, which approximates the unnormalized log
posterior with a second-order Taylor expansion, resulting in a
Gaussian approximate posterior centered at the posterior mode
\[
  \hat{\vec{f}} = \argmax_{\vec{f}} p(\vec{f} \given \data).
\]

Here we will consider two more general-purpose approximation
techniques for approximating intractable posterior distributions.
These techniques are useful when the likelihood factorizes into
one-dimensional terms, as in \acro{GP} classification.

\section*{Assumed Density Filtering}

Consider a posterior distribution of the form
\[
  p(\vec{\theta} \given \data)
  =
  \frac{1}{Z}
  p_0(\vec{\theta})
  \prod_{i = 1}^N
  t_i(\vec{\theta}),
\]
where the $t_i$ are typically likelihood terms, for example our $p(y_i
\given f_i)$ above.  We assume the prior $p_0(\vec{\theta})$ has been
chosen to be some nice form, for example a Gaussian, and we will use
the Gaussian case to illustrate the idea below.

In \emph{assumed density filtering} (\acro{ADF}), we assume that the
posterior has the same form as the prior $p_0$, and we seek an
approximating distribution
\[
  q(\vec{\theta}) \approx p(\vec{\theta} \given \data)
\]
from the same family as $p_0$ that approximates the true posterior
``as well as possible.''

Mechanically, ensuring that our approximate distribution $q$ remains
in the same family as $p_0$ is achieved by selecting an (unnormalized)
member $\tilde{t}_i$ from the likelihood conjugate to the prior for
each of the likelihood terms $t_i$.  The result is
\[
  q(\vec{\theta})
  =
  p_0(\vec{\theta})
  \prod_{i = 1}^N \tilde{Z}_i \tilde{t}_i(\vec{\theta}; \tilde{\vec{\theta}}_i),
\]
and this product will belong to the desired family by conjugacy.  Here
the constants $\tilde{Z}_i$ and the local parameter vectors
$\{\tilde{\vec{\theta}}_i\}$ are free parameters, called \emph{site
  parameters}, we may choose for each of the approximating
distributions $\tilde{t}_i$ to try to improve the fit of the
approximating distribution.

For example, the Gaussian distribution is self-conjugate, so the
\acro{ADF} approximation will in this case take the form
\[
  q(\vec{\theta})
  =
  p_0(\vec{\theta})
  \prod_{i=1}^N t_i(\vec{\theta})
  \approx
  \mc{N}(\vec{\theta}; \vec{\mu}, \mat{\Sigma})
  \prod_{i=1}^N
  \tilde{Z}_i
  \mc{N}(\vec{\theta}; \tilde{\vec{\mu}}_i, \tilde{\mat{\Sigma}}_i),
\]
where the site parameters are the constants $\{\tilde{Z}_i\}$ (chosen
so that the approximation normalizes) as well as the local mean
vectors $\{\tilde{\vec{\mu}}_i\}$ and covariance matrices
$\tilde{\mat{\Sigma}}_i$.

Note that in the \acro{GP} classification case, the likelihood terms
in the product are in fact one dimensional, because the likelihood for
$y_i$ only depends on $f_i$:
\[
  q(\vec{f})
  =
  \mc{N}(\vec{f}; \vec{\mu}, \mat{\Sigma})
  \prod_{i=1}^N
  \tilde{Z}_i
  \mc{N}(f_i; \tilde{\mu}_i, \tilde{\sigma}^2_i).
\]

Consider just the first two terms in this product:
\[
  p(\vec{\theta} \given \data_1)
  \propto
  p_0(\vec{\theta})
  t_1(\vec{\theta}).
\]
This product will not have the same nice form of the prior, but has
only been warped ``slightly'' away from the prior via a single
likelihood term.  In many cases, this distribution will be at least
partially manageable.  Perhaps there will not be a nice closed
expression, but we might still be able to compute the normalizing
constant or the moments of the posterior.

In assumed density filtering, we will approximate this product with a
member of the desired family:
\[
  q_1(\vec{\theta})
  =
  \tilde{Z}_1
  p_0(\vec{\theta})
  \tilde{t}_1(\vec{\theta}; \tilde{\vec{\theta}}_1)
  \approx
  p(\vec{\theta} \given \data_1)
  \propto
  p_0(\vec{\theta})
  t_1(\vec{\theta}).
\]
This approximation is done by matching the moments between the
approximation $q_1(\vec{\theta})$ and the true posterior
$p(\vec{\theta} \given \data_1)$.

For example, consider $p_0(\vec{\theta}) = \mc{N}(\vec{\theta};
\vec{\mu}, \mat{\Sigma})$.  We select the site parameters
\[
  \tilde{Z}_1^{-1}
  \qquad
  \tilde{\vec{\mu}}_1 \qquad
  \tilde{\mat{\Sigma}}_1
\]
such that
\begin{align*}
  \int q_1(\vec{\theta}) \intd{\vec{\theta}} &= 1 \\
  \mathbb{E}\bigl[q_1(\vec{\theta})\bigr] &= \mspace{14.5mu} \mathbb{E}\bigl[ p(\vec{\theta} \given \data_1) \bigr] \\
  \cov\bigl[q_1(\vec{\theta})\bigr] &= \cov\bigl[ p(\vec{\theta} \given \data_1) \bigr].
\end{align*}
Now the approximate distribution is in the desired density family
(Gaussians) and matches the true posterior up to the second moment.

Now consider the first three terms of the product:
\[
  p(\vec{\theta} \given \data_1, \data_2)
  \propto
  p_0(\vec{\theta})
  t_1(\vec{\theta})
  t_2(\vec{\theta})
  \propto
  p(\vec{\theta} \given \data_1)
  t_2(\vec{\theta}).
\]
The idea in assumed density filtering is to substitute in our
approximation $q_1(\vec{\theta})$, giving:
\[
  p(\vec{\theta} \given \data_1, \data_2)
  \stackrel{\propto}{\sim}
  q_1(\vec{\theta})
  t_2(\vec{\theta}).
\]
Now we are in the same situation we were in before!  We have a nice
``prior'' distribution $q_1$ multiplied by a single likelihood term
$t_2$.  We proceed as before, replacing the true likelihood term $t_2$
with an approximate (conjugate) term $\tilde{Z}_2
\tilde{t}_2(\vec{\theta}; \tilde{\vec{\theta}}_2)$, where we again
choose the site parameters $(\tilde{Z}_2, \tilde{\vec{\theta}}_2)$ to
match the moments between our new approximation
\[
  q_2(\vec{\theta})
  =
  \tilde{Z}_2
  q_1(\vec{\theta})
  \tilde{t}_2(\vec{\theta}; \tilde{\vec{\theta}}_2)
  =
  \tilde{Z}_1
  \tilde{Z}_2
  p_0(\vec{\theta})
  \tilde{t}_1(\vec{\theta}; \tilde{\vec{\theta}}_1)
  \tilde{t}_2(\vec{\theta}; \tilde{\vec{\theta}}_2).
\]
and the ``less-approximate'' posterior
$q_1(\vec{\theta})t_2(\vec{\theta})$.  We proceed in this fashion
until we have processed all the local likelihood terms, resulting
in the final approximation
\[
  q(\vec{\theta})
  =
  p_0(\vec{\theta})
  \prod_{i = 1}^N \tilde{Z}_i \tilde{t}_i(\vec{\theta}; \tilde{\vec{\theta}}_i).
\]

\section*{Expectation Propagation}

One thing to note about assumed density filtering is that our final
approximation is dependent on the sequence in which we process the
likelihood terms $\{t_i\}$.  Note that we are always matching the
moments between
\[
  q_{i-1}(\vec{\theta})t_i(\vec{\theta}),
\]
the approximation using data up to the $i$th term as well as the true
$i$th likelihood term, and the new approximation
\[
  q_i(\vec{\theta})
  =
  \tilde{Z}_i
  q_{i-1}(\vec\theta)
  \tilde{t}_i(\vec{\theta}; \tilde{\vec{\theta}}_i).
\]
This moment matching at time $i$ therefore never considers data that
appears in future terms.  For this reason, we might accumulate errors
and/or wish to later ``revisit'' a particular term and update the site
parameters $(\tilde{Z}_i, \tilde{\vec{\theta}}_i)$ in light of future
data.  This idea leads to \emph{expectation propagation,} a refinement
of assumed density filtering that can address some of these issues.

The idea is simple.  Once we have processed each of the likelihood
terms, resulting in the approximation above
\[
  q(\vec{\theta})
  =
  p_0(\vec{\theta})
  \prod_{i = 1}^N \tilde{Z}_i \tilde{t}_i(\vec{\theta}; \tilde{\vec{\theta}}_i),
\]
we repeatedly revisit each term and update its site parameters.  The
mechanism for doing so is quite simple.  First, we select a site $1
\leq i \leq N$ to update, then form the so-called \emph{cavity
  distribution,} which is our approximation using all but the $i$th
term in the product:
\[
  q_{-i}(\vec{\theta})
  =
  p_0(\vec{\theta})
  \prod_{j \neq i} \tilde{Z}_j \tilde{t}_j(\vec{\theta}; \tilde{\vec{\theta}}_j),
\]
where we have divided by the old site term $\tilde{Z}_i
\tilde{t}_i(\vec{\theta}; \tilde{\vec{\theta}}_i)$.  Now we replace
the removed site term with the true likelihood term $t_i$, forming the
\emph{tilted distribution}
\[
  q_{-i}(\vec{\theta})
  t_i(\vec{\theta}).
\]
Finally, we select new site parameters to (re)match the moments
between the tilted distribution and our new approximation:
\[
  q_{\text{new}}(\vec{\theta})
  =
  \tilde{Z}_i
  q_{-i}(\vec{\theta})
  \tilde{t}_i(\vec{\theta}; \tilde{\vec{\theta}}_i).
\]
We proceed continually updating site parameters in this manner until
we reach convergence (i.e., none of the site parameters changes very
much) or we expend a chosen computational budget.

\subsection*{Theoretical motivation}

We conclude with one brief note about the theoretical motivation
behind the moment matching used in assumed density filtering and
expectation propagation.  The Kullback--Leibler (\acro{KL}) divergence
(also called \emph{relative entropy}) is a notion of ``distance''
between probability distributions, defined by
\[
d_{\text{\acro{KL}}}\bigl(p(\vec{\theta}) \parallel q(\vec{\theta})\bigr)
  =
  \int
  p(\vec{\theta})
  \frac{p(\vec{\theta})}{q(\vec{\theta})}
  \intd{\vec{\theta}}.
\]
Notice that \acro{KL} divergence is not a true distance, as it is not
symmetric, but it does satisfy $d_{\text{\acro{KL}}}(p \parallel q)
\geq 0$ with $d_{\text{\acro{KL}}}(p \parallel q) = 0$ if and only if
$p(\vec{\theta}) = q(\vec{\theta})$ almost everywhere.

A well-known result is that the \acro{KL} divergence between an
arbitrary probability distribution $p(\vec{\theta})$ and a
multivariate Gaussian distribution $q(\vec{\theta})$ (in the direction
$d_{\text{\acro{KL}}}(p \parallel q)$) is minimized when $q$ is chosen
to match the moments of $p$.  Therefore, in the case of Gaussian
approximations, these methods can be seen as iteratively building up
an approximate posterior in the desired family by minimizing the
\acro{KL} divergence at every step.

Minimizing \acro{KL} divergence in ``the other direction,''
$d_{\text{\acro{KL}}}(q \parallel p)$, gives rise to another family of
approximation techniques known as \emph{variational Bayesian
  inference.}

\end{document}
