\documentclass{article}

\usepackage[T1]{fontenc}
\usepackage[osf]{libertine}
\usepackage[scaled=0.8]{beramono}
\usepackage[margin=1.5in]{geometry}
\usepackage{url}
\usepackage{booktabs}
\usepackage{amsmath}
\usepackage{nicefrac}
\usepackage{microtype}

\usepackage{sectsty}
\sectionfont{\large}
\subsectionfont{\normalsize}

\usepackage{titlesec}
\titlespacing{\section}{0pt}{10pt plus 2pt minus 2pt}{0pt plus 2pt minus 0pt}
\titlespacing{\subsection}{0pt}{5pt plus 2pt minus 2pt}{0pt plus 2pt minus 0pt}

\usepackage{pgfplots}
\pgfplotsset{
  compat=newest,
  plot coordinates/math parser=false,
  tick label style={font=\footnotesize, /pgf/number format/fixed},
  label style={font=\small},
  legend style={font=\small},
  every axis/.append style={
    tick align=outside,
    clip mode=individual,
    scaled ticks=false,
    thick,
    tick style={semithick, black}
  }
}

\pgfkeys{/pgf/number format/.cd, set thousands separator={\,}}

\usepgfplotslibrary{external}
\tikzexternalize[prefix=tikz/]

\newlength\figurewidth
\newlength\figureheight

\setlength{\figurewidth}{8cm}
\setlength{\figureheight}{6cm}

\setlength{\parindent}{0pt}
\setlength{\parskip}{1ex}

\newcommand{\acro}[1]{\textsc{\MakeLowercase{#1}}}
\newcommand{\given}{\mid}
\newcommand{\mc}[1]{\mathcal{#1}}
\newcommand{\data}{\mc{D}}
\newcommand{\intd}[1]{\,\mathrm{d}{#1}}
\newcommand{\inv}{^{-1}}

\begin{document}

\section*{Coin flipping}

Suppose there is a coin that may be biased -- this coin has unknown
probability $\theta$ of giving a ``heads.''  If we repeatedly flip
this coin and observe the outcomes, how can we maintain our belief
about $\theta$?

Note that the coin-flipping problem can be seen as a simplification of
the survey problem we discussed last time, where we assume that people
always tell the truth, are sampled uniformly at random, and whose
opinions are generated independently (by flipping a coin!).

Before we select a prior for $\theta$, we write down the likelihood.
For a particular problem, it is almost always easier to derive an
appropriate likelihood than it is to identify an appropriate prior
distribution.

Suppose we flip the coin $n$ times and observe $x$ ``heads.''  Every
statistician, regardless of philosophy, would agree that the
probability of this observation, given the value of $\theta$, comes
from a binomial distribution:
\begin{equation*}
  \Pr(x \given n, \theta)
  =
  \binom{n}{x} \theta^x (1 - \theta)^{n - x}.
\end{equation*}

\subsection*{Classical method}

Before we continue with the Bayesian approach, we pause to discuss how
a classical statistician would proceed with this problem.  Recall that
in the frequentist approach, the value $\theta$ can only be considered
in terms of the frequency of success (``heads'') seen during an
infinite number of trials.  It is not valid in this framework to represent
a ``belief'' about $\theta$ in terms of probability.

Rather, the frequentist approach to reasoning about $\theta$ is to
construct an \emph{estimator} for $\theta$, which in theory can be any
function of the observed data: $\hat{\theta}(x, n)$.  Estimators are
then analyzed in terms of their behavior as the number of observations
goes to infinity (for example, we might prove that $\hat{\theta} \to
\theta$ as $n \to \infty$).  The classical estimator in this case is
the empirical frequency $\hat{\theta} = \nicefrac{x}{n}$.

\subsection*{Bayesian method}

An interesting thing to note about the frequentist approach is that it
ignores all prior information, opting instead to only look at the
observed data.  To a Bayesian, every such problem is different and
should be analyzed contextually given the known information.

With the likelihood decided, we must now choose a prior distribution
$p(\theta)$.  A convenient prior in this case is the \emph{beta
  distribution,} which has two parameters $\alpha$ and $\beta$:
\begin{equation*}
  p(\theta \given \alpha, \beta)
  =
  \mc{B}(\theta; \alpha, \beta)
  =
  \frac{1}{B(\alpha, \beta)}
  \theta^{\alpha - 1}(1 - \theta)^{\beta - 1}.
\end{equation*}
Here the normalizing constant $B(\alpha, \beta)$ is the \emph{beta
  function:}
\begin{equation*}
  B(\alpha, \beta)
  =
  \int_{0}^{1} \theta^{\alpha - 1}(1 - \theta)^{\beta - 1} \intd{\theta}.
\end{equation*}

The support of the beta distribution is $\theta \in (0, 1)$, and by
selecting various values of $\alpha$ and $\beta$, we can control its
shape to represent a variety of different prior beliefs.

Given our observations $\data = (x, n)$, we can now compute the
posterior distribution of $\theta$:
\begin{equation*}
  p(\theta \given x, n, \alpha, \beta)
  =
  \frac
      {     \Pr(x \given n, \theta) p(\theta \given \alpha, \beta)}
      {\int \Pr(x \given n, \theta) p(\theta \given \alpha, \beta) \intd{\theta}}.
\end{equation*}

First we handle the normalization constant $\Pr(x \given n, \alpha, \beta)$:
\begin{align*}
  \int \Pr(x \given n, \theta) p(\theta \given \alpha, \beta) \intd{\theta}
  &=
  \binom{n}{x}
  \frac{1}{B(\alpha, \beta)}
  \int_{0}^{1}
  \theta^{\alpha + x - 1}(1 - \theta)^{\beta + n - x - 1} \intd{\theta}
  \\
  &=
  \binom{n}{x}
  \frac{B(\alpha + x, \beta + n - x)}{B(\alpha, \beta)}.
\end{align*}

Now we apply Bayes theorem:
\begin{align*}
  p(\theta \given x, n, \alpha, \beta)
  &=
  \frac
      {     \Pr(x \given n, \theta) p(\theta \given \alpha, \beta)}
      {\int \Pr(x \given n, \theta) p(\theta \given \alpha, \beta) \intd{\theta}}
  \\
  &=
  \biggl[
    \binom{n}{x}
    \frac{B(\alpha + x, \beta + n - x)}{B(\alpha, \beta)}
  \biggr]\inv
  \biggl[
    \binom{n}{x} \theta^x (1 - \theta)^{n - x}
  \biggr]
  \biggl[
    \frac{\theta^{\alpha - 1}(1 - \theta)^{\beta - 1}}{B(\alpha, \beta)}
  \biggr]
  \\
  &=
  \frac{1}{B(\alpha + x, \beta + n - x)}
  \theta^{\alpha + x - 1}(1 - \theta)^{\beta + n - x - 1}
  \\
  &=
  \mc{B}(\alpha + x, \beta + n - x).
\end{align*}
The posterior is therefore another beta distribution with parameters
$(\alpha + x, \beta + n - x)$; we have added the number of successes
to the first parameter and the number of failures to the second.

The rather convenient fact that the posterior remains a beta
distribution is because the beta distribution satisfies a property
known as \emph{conjugacy} with the binomial likelihood.  This fact
also leads to a common interpretation of the parameters $\alpha$ and
$\beta$: they serve as ``pseudocounts,'' or fake observations we
pretend to have seen before seeing the data.

Figure \ref{coin_flipping} shows the relevant functions for the coin
flipping example for $(\alpha, \beta) = (3, 5)$ and $(x, n) = (5, 6)$.
Notice that the likelihood favors higher values of $\theta$, whereas
the prior had favored lower values of $\theta$.  The posterior, taking
into account both sources of information, lies in between these
extremes.  Notice also that the posterior has support over a narrower
range of plausible $\theta$ values than the prior; this is because we
can draw more confident conclusions from having access to more
information.

\begin{figure}
  \centering
  % This file was created by matlab2tikz.
% Minimal pgfplots version: 1.3
%
\tikzsetnextfilename{beta_example}
\definecolor{mycolor1}{rgb}{0.12157,0.47059,0.70588}%
\definecolor{mycolor2}{rgb}{0.20000,0.62745,0.17255}%
\definecolor{mycolor3}{rgb}{0.89020,0.10196,0.10980}%
%
\begin{tikzpicture}

\begin{axis}[%
width=0.95092\figurewidth,
height=\figureheight,
at={(0\figurewidth,0\figureheight)},
scale only axis,
every outer x axis line/.append style={black},
every x tick label/.append style={font=\color{black}},
xmin=0,
xmax=1,
xlabel={$\theta$},
every outer y axis line/.append style={black},
every y tick label/.append style={font=\color{black}},
ymin=0,
ymax=3.5,
axis x line*=bottom,
axis y line*=left,
legend style={at={(0.03,0.97)},anchor=north west,legend cell align=left,align=left,fill=none,draw=none}
]
\addplot [color=mycolor1,solid]
  table[row sep=crcr]{%
0	0\\
0.001001001001001	0.000104789685000631\\
0.002002002002002	0.000417481268408812\\
0.003003003003003	0.000935569882767654\\
0.004004004004004	0.00165656574011104\\
0.005005005005005	0.00257799408150101\\
0.00600600600600601	0.00369739512664119\\
0.00700700700700701	0.00501232402356629\\
0.00800800800800801	0.00652035079840764\\
0.00900900900900901	0.00821906030523481\\
0.01001001001001	0.0101060521759732\\
0.011011011011011	0.0121789407703979\\
0.012012012012012	0.0144353551262033\\
0.013013013013013	0.016872938909149\\
0.014014014014014	0.0194893503632815\\
0.015015015015015	0.0222822622612326\\
0.016016016016016	0.0252493618545929\\
0.017017017017017	0.0283883508243619\\
0.018018018018018	0.0316969452314745\\
0.019019019019019	0.0351728754674026\\
0.02002002002002	0.0388138862048337\\
0.021021021021021	0.0426177363484252\\
0.022022022022022	0.0465821989856342\\
0.023023023023023	0.0507050613376246\\
0.024024024024024	0.0549841247102489\\
0.025025025025025	0.0594172044451072\\
0.026026026026026	0.0640021298706812\\
0.027027027027027	0.0687367442535453\\
0.028028028028028	0.0736189047496529\\
0.029029029029029	0.0786464823556994\\
0.03003003003003	0.0838173618605605\\
0.031031031031031	0.0891294417968076\\
0.032032032032032	0.0945806343922984\\
0.033033033033033	0.100168865521844\\
0.034034034034034	0.10589207465895\\
0.035035035035035	0.11174821482764\\
0.036036036036036	0.117735252554346\\
0.037037037037037	0.123851167819883\\
0.038038038038038	0.130093954011492\\
0.039039039039039	0.136461617874968\\
0.04004004004004	0.142952179466857\\
0.041041041041041	0.149563672106732\\
0.042042042042042	0.156294142329543\\
0.043043043043043	0.163141649838046\\
0.044044044044044	0.170104267455305\\
0.045045045045045	0.177180081077275\\
0.046046046046046	0.184367189625452\\
0.047047047047047	0.191663704999612\\
0.048048048048048	0.199067752030611\\
0.049049049049049	0.206577468433275\\
0.0500500500500501	0.214191004759356\\
0.0510510510510511	0.221906524350569\\
0.0520520520520521	0.229722203291706\\
0.0530530530530531	0.237636230363821\\
0.0540540540540541	0.245646806997496\\
0.0550550550550551	0.253752147226179\\
0.0560560560560561	0.261950477639604\\
0.0570570570570571	0.270240037337281\\
0.0580580580580581	0.278619077882061\\
0.0590590590590591	0.287085863253787\\
0.0600600600600601	0.29563866980301\\
0.0610610610610611	0.304275786204786\\
0.0620620620620621	0.31299551341255\\
0.0630630630630631	0.321796164612064\\
0.0640640640640641	0.330676065175438\\
0.0650650650650651	0.339633552615239\\
0.0660660660660661	0.348666976538659\\
0.0670670670670671	0.357774698601772\\
0.0680680680680681	0.366955092463864\\
0.0690690690690691	0.376206543741835\\
0.0700700700700701	0.385527449964682\\
0.0710710710710711	0.394916220528054\\
0.0720720720720721	0.404371276648888\\
0.0730730730730731	0.413891051320114\\
0.0740740740740741	0.423473989265447\\
0.0750750750750751	0.433118546894239\\
0.0760760760760761	0.442823192256425\\
0.0770770770770771	0.45258640499753\\
0.0780780780780781	0.462406676313763\\
0.0790790790790791	0.472282508907179\\
0.0800800800800801	0.482212416940922\\
0.0810810810810811	0.492194925994544\\
0.0820820820820821	0.502228573019395\\
0.0830830830830831	0.512311906294097\\
0.0840840840840841	0.522443485380086\\
0.0850850850850851	0.532621881077237\\
0.0860860860860861	0.542845675379559\\
0.0870870870870871	0.553113461430969\\
0.0880880880880881	0.563423843481146\\
0.0890890890890891	0.573775436841452\\
0.0900900900900901	0.584166867840937\\
0.0910910910910911	0.594596773782417\\
0.0920920920920921	0.605063802898625\\
0.0930930930930931	0.615566614308447\\
0.0940940940940941	0.626103877973222\\
0.0950950950950951	0.63667427465313\\
0.0960960960960961	0.647276495863647\\
0.0970970970970971	0.657909243832079\\
0.0980980980980981	0.668571231454174\\
0.0990990990990991	0.67926118225081\\
0.1001001001001	0.689977830324755\\
0.101101101101101	0.700719920317506\\
0.102102102102102	0.711486207366205\\
0.103103103103103	0.72227545706063\\
0.104104104104104	0.733086445400258\\
0.105105105105105	0.743917958751415\\
0.106106106106106	0.754768793804486\\
0.107107107107107	0.765637757531222\\
0.108108108108108	0.776523667142096\\
0.109109109109109	0.787425350043763\\
0.11011011011011	0.798341643796576\\
0.111111111111111	0.809271396072188\\
0.112112112112112	0.82021346461123\\
0.113113113113113	0.831166717181052\\
0.114114114114114	0.842130031533565\\
0.115115115115115	0.853102295363132\\
0.116116116116116	0.864082406264552\\
0.117117117117117	0.875069271691121\\
0.118118118118118	0.886061808912754\\
0.119119119119119	0.897058944974199\\
0.12012012012012	0.908059616653319\\
0.121121121121121	0.919062770419451\\
0.122122122122122	0.930067362391845\\
0.123123123123123	0.941072358298169\\
0.124124124124124	0.952076733433107\\
0.125125125125125	0.963079472617013\\
0.126126126126126	0.974079570154657\\
0.127127127127127	0.98507602979404\\
0.128128128128128	0.996067864685286\\
0.129129129129129	1.00705409733961\\
0.13013013013013	1.01803375958836\\
0.131131131131131	1.02900589254215\\
0.132132132132132	1.03996954655002\\
0.133133133133133	1.05092378115877\\
0.134134134134134	1.06186766507225\\
0.135135135135135	1.07280027611081\\
0.136136136136136	1.0837207011708\\
0.137137137137137	1.09462803618415\\
0.138138138138138	1.10552138607801\\
0.139139139139139	1.11639986473449\\
0.14014014014014	1.12726259495047\\
0.141141141141141	1.13810870839744\\
0.142142142142142	1.14893734558153\\
0.143143143143143	1.15974765580349\\
0.144144144144144	1.17053879711878\\
0.145145145145145	1.18130993629783\\
0.146146146146146	1.19206024878623\\
0.147147147147147	1.20278891866511\\
0.148148148148148	1.21349513861153\\
0.149149149149149	1.22417810985896\\
0.15015015015015	1.2348370421579\\
0.151151151151151	1.24547115373646\\
0.152152152152152	1.2560796712611\\
0.153153153153153	1.26666182979743\\
0.154154154154154	1.27721687277108\\
0.155155155155155	1.28774405192864\\
0.156156156156156	1.29824262729866\\
0.157157157157157	1.3087118671528\\
0.158158158158158	1.31915104796696\\
0.159159159159159	1.32955945438253\\
0.16016016016016	1.33993637916776\\
0.161161161161161	1.35028112317911\\
0.162162162162162	1.36059299532275\\
0.163163163163163	1.37087131251612\\
0.164164164164164	1.38111539964955\\
0.165165165165165	1.39132458954797\\
0.166166166166166	1.4014982229327\\
0.167167167167167	1.41163564838328\\
0.168168168168168	1.42173622229945\\
0.169169169169169	1.43179930886311\\
0.17017017017017	1.44182428000046\\
0.171171171171171	1.45181051534412\\
0.172172172172172	1.46175740219537\\
0.173173173173173	1.4716643354865\\
0.174174174174174	1.48153071774316\\
0.175175175175175	1.49135595904685\\
0.176176176176176	1.50113947699748\\
0.177177177177177	1.51088069667593\\
0.178178178178178	1.52057905060681\\
0.179179179179179	1.53023397872118\\
0.18018018018018	1.53984492831944\\
0.181181181181181	1.54941135403423\\
0.182182182182182	1.55893271779341\\
0.183183183183183	1.56840848878319\\
0.184184184184184	1.57783814341122\\
0.185185185185185	1.58722116526986\\
0.186186186186186	1.59655704509947\\
0.187187187187187	1.60584528075178\\
0.188188188188188	1.61508537715337\\
0.189189189189189	1.62427684626915\\
0.19019019019019	1.63341920706606\\
0.191191191191191	1.64251198547663\\
0.192192192192192	1.65155471436286\\
0.193193193193193	1.66054693347999\\
0.194194194194194	1.6694881894404\\
0.195195195195195	1.67837803567764\\
0.196196196196196	1.68721603241048\\
0.197197197197197	1.69600174660704\\
0.198198198198198	1.70473475194901\\
0.199199199199199	1.71341462879595\\
0.2002002002002	1.72204096414964\\
0.201201201201201	1.73061335161854\\
0.202202202202202	1.73913139138233\\
0.203203203203203	1.74759469015644\\
0.204204204204204	1.7560028611568\\
0.205205205205205	1.76435552406454\\
0.206206206206206	1.77265230499084\\
0.207207207207207	1.78089283644179\\
0.208208208208208	1.78907675728342\\
0.209209209209209	1.79720371270673\\
0.21021021021021	1.80527335419279\\
0.211211211211211	1.81328533947801\\
0.212212212212212	1.82123933251935\\
0.213213213213213	1.82913500345971\\
0.214214214214214	1.83697202859339\\
0.215215215215215	1.84475009033155\\
0.216216216216216	1.8524688771678\\
0.217217217217217	1.86012808364391\\
0.218218218218218	1.8677274103155\\
0.219219219219219	1.87526656371787\\
0.22022022022022	1.88274525633186\\
0.221221221221221	1.89016320654989\\
0.222222222222222	1.89752013864192\\
0.223223223223223	1.9048157827216\\
0.224224224224224	1.91204987471248\\
0.225225225225225	1.91922215631426\\
0.226226226226226	1.92633237496912\\
0.227227227227227	1.93338028382819\\
0.228228228228228	1.94036564171799\\
0.229229229229229	1.94728821310706\\
0.23023023023023	1.95414776807256\\
0.231231231231231	1.96094408226703\\
0.232232232232232	1.96767693688518\\
0.233233233233233	1.97434611863074\\
0.234234234234234	1.98095141968349\\
0.235235235235235	1.98749263766621\\
0.236236236236236	1.99396957561182\\
0.237237237237237	2.00038204193057\\
0.238238238238238	2.00672985037728\\
0.239239239239239	2.01301282001869\\
0.24024024024024	2.01923077520085\\
0.241241241241241	2.02538354551663\\
0.242242242242242	2.03147096577327\\
0.243243243243243	2.03749287596003\\
0.244244244244244	2.04344912121587\\
0.245245245245245	2.04933955179729\\
0.246246246246246	2.05516402304617\\
0.247247247247247	2.06092239535769\\
0.248248248248248	2.0666145341484\\
0.249249249249249	2.07224030982428\\
0.25025025025025	2.07779959774893\\
0.251251251251251	2.08329227821178\\
0.252252252252252	2.08871823639646\\
0.253253253253253	2.09407736234918\\
0.254254254254254	2.0993695509472\\
0.255255255255255	2.10459470186737\\
0.256256256256256	2.10975271955479\\
0.257257257257257	2.1148435131915\\
0.258258258258258	2.11986699666523\\
0.259259259259259	2.1248230885383\\
0.26026026026026	2.12971171201654\\
0.261261261261261	2.13453279491826\\
0.262262262262262	2.13928626964338\\
0.263263263263263	2.14397207314258\\
0.264264264264264	2.1485901468865\\
0.265265265265265	2.15314043683512\\
0.266266266266266	2.15762289340708\\
0.267267267267267	2.16203747144916\\
0.268268268268268	2.16638413020587\\
0.269269269269269	2.170662833289\\
0.27027027027027	2.17487354864733\\
0.271271271271271	2.17901624853644\\
0.272272272272272	2.18309090948849\\
0.273273273273273	2.1870975122822\\
0.274274274274274	2.19103604191278\\
0.275275275275275	2.19490648756208\\
0.276276276276276	2.19870884256867\\
0.277277277277277	2.20244310439812\\
0.278278278278278	2.20610927461323\\
0.279279279279279	2.2097073588445\\
0.28028028028028	2.2132373667605\\
0.281281281281281	2.21669931203843\\
0.282282282282282	2.22009321233475\\
0.283283283283283	2.22341908925581\\
0.284284284284284	2.22667696832865\\
0.285285285285285	2.22986687897181\\
0.286286286286286	2.23298885446624\\
0.287287287287287	2.2360429319263\\
0.288288288288288	2.2390291522708\\
0.289289289289289	2.24194756019415\\
0.29029029029029	2.24479820413758\\
0.291291291291291	2.24758113626041\\
0.292292292292292	2.25029641241144\\
0.293293293293293	2.25294409210035\\
0.294294294294294	2.25552423846927\\
0.295295295295295	2.25803691826433\\
0.296296296296296	2.26048220180735\\
0.297297297297297	2.26286016296759\\
0.298298298298298	2.26517087913357\\
0.299299299299299	2.26741443118493\\
0.3003003003003	2.26959090346447\\
0.301301301301301	2.27170038375015\\
0.302302302302302	2.27374296322722\\
0.303303303303303	2.27571873646045\\
0.304304304304304	2.27762780136638\\
0.305305305305305	2.2794702591857\\
0.306306306306306	2.28124621445563\\
0.307307307307307	2.28295577498249\\
0.308308308308308	2.28459905181426\\
0.309309309309309	2.28617615921319\\
0.31031031031031	2.28768721462861\\
0.311311311311311	2.28913233866969\\
0.312312312312312	2.29051165507835\\
0.313313313313313	2.29182529070221\\
0.314314314314314	2.29307337546762\\
0.315315315315315	2.29425604235281\\
0.316316316316316	2.29537342736104\\
0.317317317317317	2.29642566949388\\
0.318318318318318	2.29741291072458\\
0.319319319319319	2.29833529597144\\
0.32032032032032	2.29919297307133\\
0.321321321321321	2.29998609275329\\
0.322322322322322	2.30071480861211\\
0.323323323323323	2.30137927708212\\
0.324324324324324	2.30197965741093\\
0.325325325325325	2.30251611163337\\
0.326326326326326	2.30298880454539\\
0.327327327327327	2.30339790367808\\
0.328328328328328	2.30374357927183\\
0.329329329329329	2.30402600425046\\
0.33033033033033	2.30424535419549\\
0.331331331331331	2.30440180732048\\
0.332332332332332	2.3044955444454\\
0.333333333333333	2.3045267489712\\
0.334334334334334	2.30449560685424\\
0.335335335335335	2.30440230658107\\
0.336336336336336	2.30424703914302\\
0.337337337337337	2.30402999801105\\
0.338338338338338	2.30375137911061\\
0.339339339339339	2.30341138079656\\
0.34034034034034	2.3030102038282\\
0.341341341341341	2.30254805134437\\
0.342342342342342	2.30202512883859\\
0.343343343343343	2.30144164413433\\
0.344344344344344	2.30079780736031\\
0.345345345345345	2.30009383092592\\
0.346346346346346	2.29932992949668\\
0.347347347347347	2.29850631996977\\
0.348348348348348	2.2976232214497\\
0.349349349349349	2.29668085522395\\
0.35035035035035	2.29567944473879\\
0.351351351351351	2.29461921557514\\
0.352352352352352	2.29350039542445\\
0.353353353353353	2.29232321406476\\
0.354354354354354	2.29108790333674\\
0.355355355355355	2.28979469711986\\
0.356356356356356	2.28844383130862\\
0.357357357357357	2.28703554378889\\
0.358358358358358	2.28557007441424\\
0.359359359359359	2.28404766498243\\
0.36036036036036	2.28246855921196\\
0.361361361361361	2.28083300271867\\
0.362362362362362	2.2791412429924\\
0.363363363363363	2.27739352937383\\
0.364364364364364	2.27559011303122\\
0.365365365365365	2.27373124693743\\
0.366366366366366	2.27181718584683\\
0.367367367367367	2.26984818627242\\
0.368368368368368	2.26782450646297\\
0.369369369369369	2.26574640638019\\
0.37037037037037	2.26361414767612\\
0.371371371371371	2.26142799367042\\
0.372372372372372	2.25918820932785\\
0.373373373373373	2.25689506123583\\
0.374374374374374	2.25454881758196\\
0.375375375375375	2.2521497481318\\
0.376376376376376	2.24969812420652\\
0.377377377377377	2.24719421866082\\
0.378378378378378	2.24463830586077\\
0.379379379379379	2.24203066166184\\
0.38038038038038	2.23937156338693\\
0.381381381381381	2.23666128980449\\
0.382382382382382	2.23390012110679\\
0.383383383383383	2.23108833888813\\
0.384384384384384	2.22822622612326\\
0.385385385385385	2.2253140671458\\
0.386386386386386	2.22235214762674\\
0.387387387387387	2.21934075455305\\
0.388388388388388	2.21628017620634\\
0.389389389389389	2.2131707021416\\
0.39039039039039	2.21001262316602\\
0.391391391391391	2.20680623131788\\
0.392392392392392	2.20355181984554\\
0.393393393393393	2.20024968318646\\
0.394394394394394	2.19690011694635\\
0.395395395395395	2.19350341787833\\
0.396396396396396	2.19005988386226\\
0.397397397397397	2.18656981388403\\
0.398398398398398	2.18303350801505\\
0.399399399399399	2.17945126739169\\
0.4004004004004	2.1758233941949\\
0.401401401401401	2.17215019162987\\
0.402402402402402	2.16843196390573\\
0.403403403403403	2.16466901621538\\
0.404404404404404	2.16086165471536\\
0.405405405405405	2.15701018650582\\
0.406406406406406	2.15311491961057\\
0.407407407407407	2.14917616295714\\
0.408408408408408	2.14519422635704\\
0.409409409409409	2.14116942048594\\
0.41041041041041	2.1371020568641\\
0.411411411411411	2.13299244783671\\
0.412412412412412	2.12884090655442\\
0.413413413413413	2.12464774695391\\
0.414414414414414	2.12041328373851\\
0.415415415415415	2.11613783235894\\
0.416416416416416	2.11182170899411\\
0.417417417417417	2.10746523053197\\
0.418418418418418	2.10306871455047\\
0.419419419419419	2.09863247929859\\
0.42042042042042	2.09415684367743\\
0.421421421421421	2.0896421272214\\
0.422422422422422	2.08508865007947\\
0.423423423423423	2.08049673299646\\
0.424424424424424	2.07586669729452\\
0.425425425425425	2.07119886485453\\
0.426426426426426	2.06649355809771\\
0.427427427427427	2.06175109996725\\
0.428428428428428	2.05697181390998\\
0.429429429429429	2.05215602385819\\
0.43043043043043	2.04730405421149\\
0.431431431431431	2.04241622981874\\
0.432432432432432	2.03749287596003\\
0.433433433433433	2.03253431832883\\
0.434434434434434	2.02754088301412\\
0.435435435435435	2.02251289648263\\
0.436436436436436	2.01745068556117\\
0.437437437437437	2.012354577419\\
0.438438438438438	2.00722489955034\\
0.439439439439439	2.00206197975687\\
0.44044044044044	1.99686614613038\\
0.441441441441441	1.99163772703547\\
0.442442442442442	1.9863770510923\\
0.443443443443443	1.98108444715945\\
0.444444444444444	1.97576024431687\\
0.445445445445445	1.97040477184883\\
0.446446446446446	1.96501835922705\\
0.447447447447447	1.95960133609383\\
0.448448448448448	1.95415403224527\\
0.449449449449449	1.94867677761459\\
0.45045045045045	1.94316990225552\\
0.451451451451451	1.93763373632575\\
0.452452452452452	1.93206861007043\\
0.453453453453453	1.92647485380585\\
0.454454454454454	1.92085279790307\\
0.455455455455455	1.91520277277169\\
0.456456456456456	1.90952510884372\\
0.457457457457457	1.90382013655745\\
0.458458458458458	1.89808818634149\\
0.459459459459459	1.89232958859878\\
0.46046046046046	1.88654467369078\\
0.461461461461461	1.88073377192167\\
0.462462462462462	1.87489721352264\\
0.463463463463463	1.86903532863625\\
0.464464464464464	1.86314844730093\\
0.465465465465465	1.85723689943543\\
0.466466466466466	1.85130101482347\\
0.467467467467467	1.8453411230984\\
0.468468468468468	1.83935755372794\\
0.469469469469469	1.83335063599903\\
0.47047047047047	1.8273206990027\\
0.471471471471471	1.82126807161909\\
0.472472472472472	1.81519308250248\\
0.473473473473473	1.8090960600664\\
0.474474474474474	1.8029773324689\\
0.475475475475475	1.79683722759778\\
0.476476476476476	1.79067607305595\\
0.477477477477477	1.7844941961469\\
0.478478478478478	1.77829192386018\\
0.479479479479479	1.77206958285702\\
0.48048048048048	1.76582749945595\\
0.481481481481481	1.75956599961857\\
0.482482482482482	1.75328540893536\\
0.483483483483483	1.74698605261158\\
0.484484484484485	1.7406682554532\\
0.485485485485485	1.73433234185301\\
0.486486486486487	1.72797863577667\\
0.487487487487487	1.72160746074896\\
0.488488488488488	1.71521913984001\\
0.48948948948949	1.70881399565169\\
0.49049049049049	1.70239235030398\\
0.491491491491492	1.69595452542155\\
0.492492492492492	1.68950084212024\\
0.493493493493493	1.68303162099379\\
0.494494494494495	1.67654718210054\\
0.495495495495495	1.67004784495021\\
0.496496496496497	1.66353392849084\\
0.497497497497497	1.65700575109567\\
0.498498498498498	1.65046363055025\\
0.4994994994995	1.64390788403949\\
0.500500500500501	1.63733882813486\\
0.501501501501502	1.6307567787817\\
0.502502502502503	1.62416205128645\\
0.503503503503503	1.61755496030419\\
0.504504504504504	1.61093581982602\\
0.505505505505506	1.6043049431667\\
0.506506506506507	1.59766264295224\\
0.507507507507508	1.59100923110766\\
0.508508508508508	1.58434501884475\\
0.509509509509509	1.57767031664995\\
0.510510510510511	1.5709854342723\\
0.511511511511512	1.56429068071147\\
0.512512512512513	1.55758636420582\\
0.513513513513513	1.55087279222062\\
0.514514514514514	1.54415027143626\\
0.515515515515516	1.53741910773662\\
0.516516516516517	1.53067960619743\\
0.517517517517518	1.52393207107475\\
0.518518518518518	1.51717680579356\\
0.519519519519519	1.51041411293636\\
0.520520520520521	1.50364429423187\\
0.521521521521522	1.49686765054384\\
0.522522522522523	1.49008448185988\\
0.523523523523523	1.48329508728041\\
0.524524524524524	1.47649976500768\\
0.525525525525526	1.46969881233481\\
0.526526526526527	1.46289252563502\\
0.527527527527528	1.45608120035082\\
0.528528528528528	1.44926513098331\\
0.529529529529529	1.44244461108164\\
0.530530530530531	1.4356199332324\\
0.531531531531532	1.42879138904923\\
0.532532532532533	1.42195926916236\\
0.533533533533533	1.41512386320838\\
0.534534534534535	1.40828545981998\\
0.535535535535536	1.40144434661578\\
0.536536536536537	1.39460081019027\\
0.537537537537538	1.38775513610383\\
0.538538538538539	1.38090760887276\\
0.53953953953954	1.37405851195945\\
0.540540540540541	1.36720812776262\\
0.541541541541542	1.36035673760762\\
0.542542542542543	1.35350462173679\\
0.543543543543544	1.34665205929994\\
0.544544544544545	1.33979932834484\\
0.545545545545546	1.33294670580789\\
0.546546546546547	1.32609446750475\\
0.547547547547548	1.31924288812111\\
0.548548548548549	1.31239224120356\\
0.54954954954955	1.30554279915045\\
0.550550550550551	1.29869483320291\\
0.551551551551552	1.2918486134359\\
0.552552552552553	1.28500440874938\\
0.553553553553554	1.27816248685947\\
0.554554554554555	1.27132311428979\\
0.555555555555556	1.2644865563628\\
0.556556556556557	1.25765307719125\\
0.557557557557558	1.25082293966974\\
0.558558558558559	1.24399640546622\\
0.55955955955956	1.23717373501378\\
0.560560560560561	1.23035518750231\\
0.561561561561562	1.22354102087034\\
0.562562562562563	1.21673149179699\\
0.563563563563564	1.20992685569389\\
0.564564564564565	1.20312736669724\\
0.565565565565566	1.19633327765998\\
0.566566566566567	1.18954484014395\\
0.567567567567568	1.18276230441217\\
0.568568568568569	1.17598591942124\\
0.56956956956957	1.16921593281373\\
0.570570570570571	1.1624525909107\\
0.571571571571572	1.15569613870429\\
0.572572572572573	1.14894681985039\\
0.573573573573574	1.14220487666134\\
0.574574574574575	1.13547055009879\\
0.575575575575576	1.12874407976656\\
0.576576576576577	1.12202570390361\\
0.577577577577578	1.11531565937708\\
0.578578578578579	1.10861418167541\\
0.57957957957958	1.10192150490154\\
0.580580580580581	1.09523786176617\\
0.581581581581582	1.08856348358109\\
0.582582582582583	1.08189860025264\\
0.583583583583584	1.07524344027516\\
0.584584584584585	1.06859823072463\\
0.585585585585586	1.06196319725223\\
0.586586586586587	1.05533856407815\\
0.587587587587588	1.04872455398534\\
0.588588588588589	1.04212138831339\\
0.58958958958959	1.03552928695252\\
0.590590590590591	1.02894846833757\\
0.591591591591592	1.02237914944213\\
0.592592592592593	1.01582154577271\\
0.593593593593594	1.009275871363\\
0.594594594594595	1.0027423387682\\
0.595595595595596	0.996221159059449\\
0.596596596596597	0.989712541818282\\
0.597597597597598	0.983216695131216\\
0.598598598598599	0.976733825584374\\
0.5995995995996	0.97026413825821\\
0.600600600600601	0.963807836722289\\
0.601601601601602	0.957365123030158\\
0.602602602602603	0.950936197714288\\
0.603603603603604	0.944521259781095\\
0.604604604604605	0.93812050670603\\
0.605605605605606	0.931734134428751\\
0.606606606606607	0.925362337348371\\
0.607607607607608	0.919005308318777\\
0.608608608608609	0.912663238644035\\
0.60960960960961	0.906336318073859\\
0.610610610610611	0.900024734799164\\
0.611611611611612	0.893728675447691\\
0.612612612612613	0.887448325079713\\
0.613613613613614	0.881183867183812\\
0.614614614614615	0.874935483672736\\
0.615615615615616	0.868703354879323\\
0.616616616616617	0.86248765955252\\
0.617617617617618	0.856288574853456\\
0.618618618618619	0.850106276351604\\
0.61961961961962	0.843940938021021\\
0.620620620620621	0.837792732236652\\
0.621621621621622	0.831661829770721\\
0.622622622622623	0.825548399789197\\
0.623623623623624	0.819452609848326\\
0.624624624624625	0.813374625891256\\
0.625625625625626	0.807314612244722\\
0.626626626626627	0.801272731615815\\
0.627627627627628	0.795249145088828\\
0.628628628628629	0.78924401212217\\
0.62962962962963	0.78325749054537\\
0.630630630630631	0.777289736556144\\
0.631631631631632	0.771340904717541\\
0.632632632632633	0.765411147955173\\
0.633633633633634	0.759500617554508\\
0.634634634634635	0.753609463158255\\
0.635635635635636	0.747737832763805\\
0.636636636636637	0.741885872720769\\
0.637637637637638	0.736053727728576\\
0.638638638638639	0.730241540834158\\
0.63963963963964	0.724449453429703\\
0.640640640640641	0.718677605250489\\
0.641641641641642	0.712926134372789\\
0.642642642642643	0.707195177211863\\
0.643643643643644	0.70148486852001\\
0.644644644644645	0.695795341384712\\
0.645645645645646	0.690126727226842\\
0.646646646646647	0.684479155798954\\
0.647647647647648	0.678852755183649\\
0.648648648648649	0.673247651792012\\
0.64964964964965	0.667663970362136\\
0.650650650650651	0.662101833957708\\
0.651651651651652	0.656561363966684\\
0.652652652652653	0.651042680100024\\
0.653653653653654	0.645545900390528\\
0.654654654654655	0.640071141191721\\
0.655655655655656	0.63461851717683\\
0.656656656656657	0.629188141337838\\
0.657657657657658	0.623780124984598\\
0.658658658658659	0.61839457774405\\
0.65965965965966	0.613031607559484\\
0.660660660660661	0.607691320689902\\
0.661661661661662	0.602373821709445\\
0.662662662662663	0.597079213506898\\
0.663663663663664	0.591807597285273\\
0.664664664664665	0.586559072561467\\
0.665665665665666	0.581333737165995\\
0.666666666666667	0.576131687242799\\
0.667667667667668	0.570953017249134\\
0.668668668668669	0.565797819955535\\
0.66966966966967	0.560666186445846\\
0.670670670670671	0.555558206117341\\
0.671671671671672	0.550473966680913\\
0.672672672672673	0.545413554161338\\
0.673673673673674	0.540377052897619\\
0.674674674674675	0.535364545543403\\
0.675675675675676	0.530376113067479\\
0.676676676676677	0.525411834754342\\
0.677677677677678	0.520471788204843\\
0.678678678678679	0.515556049336917\\
0.67967967967968	0.51066469238637\\
0.680680680680681	0.505797789907762\\
0.681681681681682	0.500955412775357\\
0.682682682682683	0.496137630184146\\
0.683683683683684	0.491344509650954\\
0.684684684684685	0.486576117015616\\
0.685685685685686	0.481832516442231\\
0.686686686686687	0.477113770420498\\
0.687687687687688	0.472419939767118\\
0.688688688688689	0.467751083627279\\
0.68968968968969	0.463107259476217\\
0.690690690690691	0.458488523120845\\
0.691691691691692	0.453894928701472\\
0.692692692692693	0.449326528693585\\
0.693693693693694	0.444783373909716\\
0.694694694694695	0.440265513501378\\
0.695695695695696	0.435772994961081\\
0.696696696696697	0.431305864124425\\
0.697697697697698	0.42686416517227\\
0.698698698698699	0.422447940632974\\
0.6996996996997	0.418057231384714\\
0.700700700700701	0.413692076657887\\
0.701701701701702	0.409352514037574\\
0.702702702702703	0.405038579466093\\
0.703703703703704	0.400750307245625\\
0.704704704704705	0.396487730040905\\
0.705705705705706	0.392250878882007\\
0.706706706706707	0.388039783167193\\
0.707707707707708	0.383854470665839\\
0.708708708708709	0.379694967521442\\
0.70970970970971	0.375561298254699\\
0.710710710710711	0.371453485766665\\
0.711711711711712	0.36737155134198\\
0.712712712712713	0.363315514652185\\
0.713713713713714	0.359285393759101\\
0.714714714714715	0.355281205118293\\
0.715715715715716	0.351302963582602\\
0.716716716716717	0.347350682405763\\
0.717717717717718	0.343424373246089\\
0.718718718718719	0.339524046170239\\
0.71971971971972	0.335649709657059\\
0.720720720720721	0.331801370601494\\
0.721721721721722	0.327979034318588\\
0.722722722722723	0.324182704547546\\
0.723723723723724	0.320412383455885\\
0.724724724724725	0.316668071643651\\
0.725725725725726	0.312949768147717\\
0.726726726726727	0.309257470446159\\
0.727727727727728	0.3055911744627\\
0.728728728728729	0.301950874571241\\
0.72972972972973	0.298336563600457\\
0.730730730730731	0.294748232838479\\
0.731731731731732	0.291185872037642\\
0.732732732732733	0.287649469419323\\
0.733733733733734	0.284139011678838\\
0.734734734734735	0.280654483990427\\
0.735735735735736	0.277195870012313\\
0.736736736736737	0.273763151891834\\
0.737737737737738	0.270356310270653\\
0.738738738738739	0.266975324290044\\
0.73973973973974	0.263620171596255\\
0.740740740740741	0.260290828345942\\
0.741741741741742	0.256987269211691\\
0.742742742742743	0.253709467387597\\
0.743743743743744	0.250457394594941\\
0.744744744744745	0.247231021087925\\
0.745745745745746	0.244030315659491\\
0.746746746746747	0.240855245647221\\
0.747747747747748	0.237705776939298\\
0.748748748748749	0.234581873980562\\
0.74974974974975	0.231483499778624\\
0.750750750750751	0.228410615910072\\
0.751751751751752	0.225363182526739\\
0.752752752752753	0.222341158362058\\
0.753753753753754	0.219344500737487\\
0.754754754754755	0.216373165569012\\
0.755755755755756	0.213427107373726\\
0.756756756756757	0.210506279276482\\
0.757757757757758	0.207610633016628\\
0.758758758758759	0.20474011895481\\
0.75975975975976	0.201894686079855\\
0.760760760760761	0.199074282015733\\
0.761761761761762	0.196278853028591\\
0.762762762762763	0.193508344033863\\
0.763763763763764	0.190762698603457\\
0.764764764764765	0.188041858973021\\
0.765765765765766	0.18534576604928\\
0.766766766766767	0.182674359417449\\
0.767767767767768	0.180027577348731\\
0.768768768768769	0.17740535680788\\
0.76976976976977	0.174807633460845\\
0.770770770770771	0.172234341682489\\
0.771771771771772	0.169685414564386\\
0.772772772772773	0.167160783922694\\
0.773773773773774	0.164660380306099\\
0.774774774774775	0.162184133003841\\
0.775775775775776	0.159731970053817\\
0.776776776776777	0.157303818250748\\
0.777777777777778	0.154899603154442\\
0.778778778778779	0.152519249098115\\
0.77977977977978	0.150162679196795\\
0.780780780780781	0.147829815355806\\
0.781781781781782	0.145520578279323\\
0.782782782782783	0.143234887479\\
0.783783783783784	0.140972661282686\\
0.784784784784785	0.138733816843203\\
0.785785785785786	0.136518270147207\\
0.786786786786787	0.134325936024131\\
0.787787787787788	0.132156728155186\\
0.788788788788789	0.130010559082462\\
0.78978978978979	0.127887340218085\\
0.790790790790791	0.125786981853457\\
0.791791791791792	0.123709393168579\\
0.792792792792793	0.121654482241439\\
0.793793793793794	0.119622156057482\\
0.794794794794795	0.117612320519153\\
0.795795795795796	0.115624880455522\\
0.796796796796797	0.11365973963198\\
0.797797797797798	0.111716800760009\\
0.798798798798799	0.109795965507033\\
0.7997997997998	0.107897134506345\\
0.800800800800801	0.106020207367107\\
0.801801801801802	0.104165082684424\\
0.802802802802803	0.102331658049503\\
0.803803803803804	0.10051983005988\\
0.804804804804805	0.098729494329723\\
0.805805805805806	0.0969605455002178\\
0.806806806806807	0.0952128772500231\\
0.807807807807808	0.0934863823058041\\
0.808808808808809	0.0917809524528432\\
0.80980980980981	0.090096478545725\\
0.810810810810811	0.0884328505190984\\
0.811811811811812	0.0867899573985148\\
0.812812812812813	0.0851676873113415\\
0.813813813813814	0.0835659274977516\\
0.814814814814815	0.0819845643217905\\
0.815815815815816	0.0804234832825177\\
0.816816816816817	0.0788825690252251\\
0.817817817817818	0.077361705352731\\
0.818818818818819	0.0758607752367505\\
0.81981981981982	0.0743796608293417\\
0.820820820820821	0.072918243474428\\
0.821821821821822	0.0714764037193971\\
0.822822822822823	0.0700540213267744\\
0.823823823823824	0.0686509752859747\\
0.824824824824825	0.0672671438251278\\
0.825825825825826	0.0659024044229814\\
0.826826826826827	0.0645566338208809\\
0.827827827827828	0.063229708034822\\
0.828828828828829	0.0619215023675835\\
0.82982982982983	0.0606318914209331\\
0.830830830830831	0.0593607491079103\\
0.831831831831832	0.0581079486651861\\
0.832832832832833	0.0568733626654975\\
0.833833833833834	0.0556568630301582\\
0.834834834834835	0.0544583210416467\\
0.835835835835836	0.0532776073562685\\
0.836836836836837	0.0521145920168965\\
0.837837837837838	0.0509691444657848\\
0.838838838838839	0.0498411335574612\\
0.83983983983984	0.0487304275716946\\
0.840840840840841	0.0476368942265374\\
0.841841841841842	0.0465604006914467\\
0.842842842842843	0.0455008136004782\\
0.843843843843844	0.0444579990655587\\
0.844844844844845	0.0434318226898339\\
0.845845845845846	0.0424221495810917\\
0.846846846846847	0.0414288443652624\\
0.847847847847848	0.040451771199994\\
0.848848848848849	0.0394907937883047\\
0.84984984984985	0.0385457753923105\\
0.850850850850851	0.0376165788470295\\
0.851851851851852	0.0367030665742616\\
0.852852852852853	0.0358051005965451\\
0.853853853853854	0.0349225425511881\\
0.854854854854855	0.0340552537043776\\
0.855855855855856	0.0332030949653637\\
0.856856856856857	0.0323659269007193\\
0.857857857857858	0.031543609748677\\
0.858858858858859	0.0307360034335414\\
0.85985985985986	0.0299429675801777\\
0.860860860860861	0.0291643615285765\\
0.861861861861862	0.0284000443484937\\
0.862862862862863	0.027649874854168\\
0.863863863863864	0.0269137116191131\\
0.864864864864865	0.0261914129909866\\
0.865865865865866	0.0254828371065353\\
0.866866866866867	0.0247878419066153\\
0.867867867867868	0.0241062851512895\\
0.868868868868869	0.0234380244350005\\
0.86986986986987	0.0227829172018196\\
0.870870870870871	0.0221408207607722\\
0.871871871871872	0.0215115923012383\\
0.872872872872873	0.0208950889084305\\
0.873873873873874	0.0202911675789471\\
0.874874874874875	0.0196996852364007\\
0.875875875875876	0.0191204987471248\\
0.876876876876877	0.0185534649359545\\
0.877877877877878	0.0179984406020839\\
0.878878878878879	0.0174552825350003\\
0.87987987987988	0.0169238475304932\\
0.880880880880881	0.0164039924067402\\
0.881881881881882	0.0158955740204689\\
0.882882882882883	0.0153984492831945\\
0.883883883883884	0.0149124751775334\\
0.884884884884885	0.0144375087735937\\
0.885885885885886	0.0139734072454406\\
0.886886886886887	0.0135200278876387\\
0.887887887887888	0.0130772281318701\\
0.888888888888889	0.012644865563628\\
0.88988988988989	0.0122227979389875\\
0.890890890890891	0.0118108832014517\\
0.891891891891892	0.0114089794988738\\
0.892892892892893	0.0110169452004552\\
0.893893893893894	0.0106346389138205\\
0.894894894894895	0.0102619195021675\\
0.895895895895896	0.00989864610149394\\
0.896896896896897	0.00954467813789988\\
0.897897897897898	0.00919987534496632\\
0.898898898898899	0.00886409778121004\\
0.8998998998999	0.00853720584761408\\
0.900900900900901	0.0082190603052348\\
0.901901901901902	0.00790952229288445\\
0.902902902902903	0.00760845334489019\\
0.903903903903904	0.00731571540892898\\
0.904904904904905	0.00703117086393866\\
0.905905905905906	0.00675468253810493\\
0.906906906906907	0.00648611372692445\\
0.907907907907908	0.00622532821134405\\
0.908908908908909	0.00597219027597588\\
0.90990990990991	0.00572656472738885\\
0.910910910910911	0.00548831691247572\\
0.911911911911912	0.00525731273689664\\
0.912912912912913	0.00503341868359857\\
0.913913913913914	0.00481650183141072\\
0.914914914914915	0.00460642987371622\\
0.915915915915916	0.00440307113719954\\
0.916916916916917	0.00420629460067032\\
0.917917917917918	0.00401596991396298\\
0.918918918918919	0.00383196741691253\\
0.91991991991992	0.00365415815840643\\
0.920920920920921	0.00348241391551241\\
0.921921921921922	0.0033166072126824\\
0.922922922922923	0.0031566113410326\\
0.923923923923924	0.00300230037769944\\
0.924924924924925	0.00285354920527175\\
0.925925925925926	0.00271023353129886\\
0.926926926926927	0.00257222990787484\\
0.927927927927928	0.00243941575129879\\
0.928928928928929	0.00231166936181108\\
0.92992992992993	0.00218886994340586\\
0.930930930930931	0.00207089762371936\\
0.931931931931932	0.00195763347399446\\
0.932932932932933	0.00184895952912118\\
0.933933933933934	0.00174475880775333\\
0.934934934934935	0.00164491533250116\\
0.935935935935936	0.0015493141502\\
0.936936936936937	0.0014578413522551\\
0.937937937937938	0.00137038409506241\\
0.938938938938939	0.00128683062050546\\
0.93993993993994	0.00120707027652833\\
0.940940940940941	0.00113099353778455\\
0.941941941941942	0.00105849202636223\\
0.942942942942943	0.000989458532585079\\
0.943943943943944	0.000923787035889606\\
0.944944944944945	0.000861372725778323\\
0.945945945945946	0.000802112022848966\\
0.946946946946947	0.000745902599899851\\
0.947947947947948	0.000692643403111226\\
0.948948948948949	0.000642234673302699\\
0.94994994994995	0.000594577967266735\\
0.950950950950951	0.000549576179178164\\
0.951951951951952	0.000507133562079796\\
0.952952952952953	0.000467155749444056\\
0.953953953953954	0.000429549776810686\\
0.954954954954955	0.000394224103500521\\
0.955955955955956	0.000361088634405274\\
0.956956956956957	0.00033005474185342\\
0.957957957957958	0.000301035287552113\\
0.958958958958959	0.000273944644605166\\
0.95995995995996	0.000248698719607094\\
0.960960960960961	0.00022521497481318\\
0.961961961961962	0.000203412450385638\\
0.962962962962963	0.000183211786715804\\
0.963963963963964	0.000164535246822389\\
0.964964964964965	0.000147306738825798\\
0.965965965965966	0.000131451838498479\\
0.966966966966967	0.000116897811891354\\
0.967967967967968	0.000103573638036288\\
0.968968968968969	9.14100317246215e-05\\
0.96996996996997	8.03394663617598e-05\\
0.970970970970971	7.02961968978035e-05\\
0.971971971971972	6.12162828342511e-05\\
0.972972972972973	5.30376113067475e-05\\
0.973973973973974	4.56999202438927e-05\\
0.974974974974975	3.91448216020981e-05\\
0.975975975975976	3.33158246765059e-05\\
0.976976976976977	2.81583594879606e-05\\
0.977977977977978	2.36198002460342e-05\\
0.978978978978979	1.96494888881104e-05\\
0.97997997997998	1.6198758694519e-05\\
0.980980980980981	1.32209579797297e-05\\
0.981981981981982	1.06714738595992e-05\\
0.982982982982983	8.50775609467408e-06\\
0.983983983983984	6.68934100954863e-06\\
0.984984984984985	5.17787548827721e-06\\
0.985985985985986	3.93714104584319e-06\\
0.986986986986987	2.93307797568207e-06\\
0.987987987987988	2.13380957326013e-06\\
0.988988988988989	1.5096664357084e-06\\
0.98998998998999	1.03321083751123e-06\\
0.990990990990991	6.79261182250817e-07\\
0.991991991991992	4.24916530406452e-07\\
0.992992992992993	2.49581203209082e-07\\
0.993993993993994	1.34989462551131e-07\\
0.994994994994995	6.52302669513308e-08\\
0.995995995995996	2.67721035749371e-08\\
0.996996996996997	8.48789630904074e-09\\
0.997997997997998	1.67998989308457e-09\\
0.998998998998999	1.05210104578518e-10\\
1	0\\
};
\addlegendentry{prior};

\addplot [color=mycolor2,solid]
  table[row sep=crcr]{%
0	0\\
0.001001001001001	8.04816038472103e-21\\
0.002002002002002	1.02913230025707e-18\\
0.003003003003003	1.75660535614622e-17\\
0.004004004004004	1.31464683812177e-16\\
0.005005005005005	6.26242439755503e-16\\
0.00600600600600601	2.24168240148205e-15\\
0.00700700700700701	6.5881584153633e-15\\
0.00800800800800801	1.67598313690524e-14\\
0.00900900900900901	3.81855316742258e-14\\
0.01001001001001	7.97558178405726e-14\\
0.011011011011011	1.55264375872399e-13\\
0.012012012012012	2.85201599067104e-13\\
0.013013013013013	4.9893786264261e-13\\
0.014014014014014	8.37333682468763e-13\\
0.015015015015015	1.35581362202552e-12\\
0.016016016016016	2.12794048655868e-12\\
0.017017017017017	3.24952601825166e-12\\
0.018018018018018	4.84331398108001e-12\\
0.019019019019019	7.06427119886241e-12\\
0.02002002002002	1.01055217848714e-11\\
0.021021021021021	1.42049594762125e-11\\
0.022022022022022	1.96525726610313e-11\\
0.023023023023023	2.67985163614521e-11\\
0.024024024024024	3.60619651100047e-11\\
0.025025025025025	4.79407803321384e-11\\
0.026026026026026	6.30220255222728e-11\\
0.027027027027027	8.19933621756863e-11\\
0.028028028028028	1.05655359113378e-10\\
0.029029029029029	1.34934747511913e-10\\
0.03003003003003	1.7089865362507e-10\\
0.031031031031031	2.14769840859002e-10\\
0.032032032032032	2.67942992537422e-10\\
0.033033033033033	3.32002066368238e-10\\
0.034034034034034	4.08738751297807e-10\\
0.035035035035035	5.00172057113964e-10\\
0.036036036036036	6.08569066833683e-10\\
0.037037037037037	7.36466881586207e-10\\
0.038038038038038	8.86695787377212e-10\\
0.039039039039039	1.06240367279453e-09\\
0.04004004004004	1.26708172639074e-09\\
0.041041041041041	1.5045914421527e-09\\
0.042042042042042	1.77919296114328e-09\\
0.043043043043043	2.09557477707479e-09\\
0.044044044044044	2.45888483324919e-09\\
0.045045045045045	2.87476303797438e-09\\
0.046046046046046	3.34937522524116e-09\\
0.047047047047047	3.88944858711979e-09\\
0.048048048048048	4.50230860401091e-09\\
0.049049049049049	5.19591749855888e-09\\
0.0500500500500501	5.97891423871201e-09\\
0.0510510510510511	6.86065611508835e-09\\
0.0520520520520521	7.85126191747979e-09\\
0.0530530530530531	8.9616567350038e-09\\
0.0540540540540541	1.02036184040854e-08\\
0.0550550550550551	1.1589825628127e-08\\
0.0560560560560561	1.31339077924008e-08\\
0.0570570570570571	1.48504964973686e-08\\
0.0580580580580581	1.67552788333143e-08\\
0.0590590590590591	1.88650524188453e-08\\
0.0600600600600601	2.11977822254947e-08\\
0.0610610610610611	2.37726592103326e-08\\
0.0620620620620621	2.66101607781665e-08\\
0.0630630630630631	2.97321130945913e-08\\
0.0640640640640641	3.31617552708158e-08\\
0.0650650650650651	3.69238054408751e-08\\
0.0660660660660661	4.10445287515118e-08\\
0.0670670670670671	4.55518072846768e-08\\
0.0680680680680681	5.04752119322871e-08\\
0.0690690690690691	5.5846076242542e-08\\
0.0700700700700701	6.16975722567797e-08\\
0.0710710710710711	6.80647883555325e-08\\
0.0720720720720721	7.49848091321058e-08\\
0.0730730730730731	8.249679731169e-08\\
0.0740740740740741	9.06420777336866e-08\\
0.0750750750750751	9.94642234146006e-08\\
0.0760760760760761	1.09009143708526e-07\\
0.0770770770770771	1.19325174581939e-07\\
0.0780780780780781	1.30463171019169e-07\\
0.0790790790790791	1.42476601574607e-07\\
0.0800800800800801	1.55421645087378e-07\\
0.0810810810810811	1.69357289573879e-07\\
0.0820820820820821	1.84345433313273e-07\\
0.0830830830830831	2.00450988140679e-07\\
0.0840840840840841	2.17741984962489e-07\\
0.0850850850850851	2.36289681507926e-07\\
0.0860860860860861	2.56168672330615e-07\\
0.0870870870870871	2.77457001073609e-07\\
0.0880880880880881	3.00236275011018e-07\\
0.0890890890890891	3.24591781879028e-07\\
0.0900900900900901	3.50612609008802e-07\\
0.0910910910910911	3.78391764773397e-07\\
0.0920920920920921	4.08026302360543e-07\\
0.0930930930930931	4.39617445882773e-07\\
0.0940940940940941	4.7327071883609e-07\\
0.0950950950950951	5.09096074918018e-07\\
0.0960960960960961	5.47208031215561e-07\\
0.0970970970970971	5.87725803773282e-07\\
0.0980980980980981	6.30773445551365e-07\\
0.0990990990990991	6.76479986783232e-07\\
0.1001001001001	7.24979577741907e-07\\
0.101101101101101	7.76411633924065e-07\\
0.102102102102102	8.3092098366032e-07\\
0.103103103103103	8.88658018160011e-07\\
0.104104104104104	9.49778843998399e-07\\
0.105105105105105	1.01444543805391e-06\\
0.106106106106106	1.08282580490262e-06\\
0.107107107107107	1.15509413667708e-06\\
0.108108108108108	1.2314309753959e-06\\
0.109109109109109	1.31202337777061e-06\\
0.11011011011011	1.39706508249558e-06\\
0.111111111111111	1.4867566800268e-06\\
0.112112112112112	1.58130578485474e-06\\
0.113113113113113	1.68092721027634e-06\\
0.114114114114114	1.78584314567082e-06\\
0.115115115115115	1.89628333628359e-06\\
0.116116116116116	2.01248526552234e-06\\
0.117117117117117	2.13469433976901e-06\\
0.118118118118118	2.26316407571102e-06\\
0.119119119119119	2.39815629019466e-06\\
0.12012012012012	2.53994129260369e-06\\
0.121121121121121	2.6887980797652e-06\\
0.122122122122122	2.84501453338495e-06\\
0.123123123123123	3.00888762001406e-06\\
0.124124124124124	3.18072359354819e-06\\
0.125125125125125	3.3608382002606e-06\\
0.126126126126126	3.54955688636967e-06\\
0.127127127127127	3.7472150081414e-06\\
0.128128128128128	3.95415804452695e-06\\
0.129129129129129	4.17074181233518e-06\\
0.13013013013013	4.39733268393944e-06\\
0.131131131131131	4.63430780751785e-06\\
0.132132132132132	4.88205532982612e-06\\
0.133133133133133	5.14097462150085e-06\\
0.134134134134134	5.4114765048921e-06\\
0.135135135135135	5.69398348442264e-06\\
0.136136136136136	5.98892997947173e-06\\
0.137137137137137	6.2967625597802e-06\\
0.138138138138138	6.61794018337425e-06\\
0.139139139139139	6.95293443700393e-06\\
0.14014014014014	7.30222977909305e-06\\
0.141141141141141	7.66632378519601e-06\\
0.142142142142142	8.04572739595743e-06\\
0.143143143143143	8.4409651675695e-06\\
0.144144144144144	8.85257552472236e-06\\
0.145145145145145	9.2811110160417e-06\\
0.146146146146146	9.72713857200814e-06\\
0.147147147147147	1.01912397653523e-05\\
0.148148148148148	1.0674011073919e-05\\
0.149149149149149	1.11760641459939e-05\\
0.15015015015015	1.1698026068086e-05\\
0.151151151151151	1.22405396351581e-05\\
0.152152152152152	1.28042636232974e-05\\
0.153153153153153	1.33898730648193e-05\\
0.154154154154154	1.39980595257944e-05\\
0.155155155155155	1.46295313859921e-05\\
0.156156156156156	1.52850141212295e-05\\
0.157157157157157	1.59652505881184e-05\\
0.158158158158158	1.66710013111993e-05\\
0.159159159159159	1.74030447724533e-05\\
0.16016016016016	1.81621777031814e-05\\
0.161161161161161	1.89492153782398e-05\\
0.162162162162162	1.97649919126221e-05\\
0.163163163163163	2.0610360560375e-05\\
0.164164164164164	2.14861940158372e-05\\
0.165165165165165	2.239338471719e-05\\
0.166166166166166	2.33328451523055e-05\\
0.167167167167167	2.4305508166882e-05\\
0.168168168168168	2.53123272748525e-05\\
0.169169169169169	2.63542769710527e-05\\
0.17017017017017	2.74323530461369e-05\\
0.171171171171171	2.8547572903725e-05\\
0.172172172172172	2.97009758797696e-05\\
0.173173173173173	3.08936235641261e-05\\
0.174174174174174	3.21266001243126e-05\\
0.175175175175175	3.34010126314434e-05\\
0.176176176176176	3.47179913883212e-05\\
0.177177177177177	3.60786902596725e-05\\
0.178178178178178	3.7484287004509e-05\\
0.179179179179179	3.89359836105994e-05\\
0.18018018018018	4.04350066310348e-05\\
0.181181181181181	4.19826075228696e-05\\
0.182182182182182	4.35800629878223e-05\\
0.183183183183183	4.52286753150168e-05\\
0.184184184184184	4.69297727257462e-05\\
0.185185185185185	4.86847097202422e-05\\
0.186186186186186	5.04948674264292e-05\\
0.187187187187187	5.23616539506467e-05\\
0.188188188188188	5.42865047303171e-05\\
0.189189189189189	5.62708828885424e-05\\
0.19019019019019	5.83162795906092e-05\\
0.191191191191191	6.04242144023786e-05\\
0.192192192192192	6.25962356505455e-05\\
0.193193193193193	6.48339207847414e-05\\
0.194194194194194	6.71388767414622e-05\\
0.195195195195195	6.95127403097987e-05\\
0.196196196196196	7.19571784989475e-05\\
0.197197197197197	7.44738889074795e-05\\
0.198198198198198	7.70646000943458e-05\\
0.199199199199199	7.97310719515943e-05\\
0.2002002002002	8.24750960787769e-05\\
0.201201201201201	8.52984961590209e-05\\
0.202202202202202	8.82031283367455e-05\\
0.203203203203203	9.11908815969918e-05\\
0.204204204204204	9.42636781463481e-05\\
0.205205205205205	9.74234737954438e-05\\
0.206206206206206	0.000100672258342984\\
0.207207207207207	0.000104012055961303\\
0.208208208208208	0.000107444925583408\\
0.209209209209209	0.000110972961291489\\
0.21021021021021	0.000114598292706868\\
0.211211211211211	0.000118323085381356\\
0.212212212212212	0.00012214954119\\
0.213213213213213	0.000126079898725183\\
0.214214214214214	0.000130116433692055\\
0.215215215215215	0.000134261459305262\\
0.216216216216216	0.000138517326686957\\
0.217217217217217	0.000142886425266039\\
0.218218218218218	0.000147371183178621\\
0.219219219219219	0.000151974067669675\\
0.22022022022022	0.000156697585495838\\
0.221221221221221	0.000161544283329333\\
0.222222222222222	0.000166516748163002\\
0.223223223223223	0.000171617607716381\\
0.224224224224224	0.00017684953084283\\
0.225225225225225	0.000182215227937646\\
0.226226226226226	0.000187717451347158\\
0.227227227227227	0.000193358995778756\\
0.228228228228228	0.000199142698711821\\
0.229229229229229	0.000205071440809535\\
0.23023023023023	0.000211148146331523\\
0.231231231231231	0.000217375783547308\\
0.232232232232232	0.000223757365150533\\
0.233233233233233	0.000230295948673927\\
0.234234234234234	0.000236994636904968\\
0.235235235235235	0.00024385657830222\\
0.236236236236236	0.000250884967412305\\
0.237237237237237	0.000258083045287471\\
0.238238238238238	0.000265454099903729\\
0.239239239239239	0.000273001466579506\\
0.24024024024024	0.000280728528394806\\
0.241241241241241	0.000288638716610808\\
0.242242242242242	0.000296735511089896\\
0.243243243243243	0.000305022440716061\\
0.244244244244244	0.000313503083815647\\
0.245245245245245	0.000322181068578404\\
0.246246246246246	0.000331060073478807\\
0.247247247247247	0.000340143827697603\\
0.248248248248248	0.000349436111543543\\
0.249249249249249	0.000358940756875272\\
0.25025025025025	0.000368661647523323\\
0.251251251251251	0.000378602719712182\\
0.252252252252252	0.000388767962482385\\
0.253253253253253	0.00039916141811261\\
0.254254254254254	0.000409787182541701\\
0.255255255255255	0.000420649405790623\\
0.256256256256256	0.000431752292384261\\
0.257257257257257	0.000443100101773062\\
0.258258258258258	0.000454697148754446\\
0.259259259259259	0.000466547803893961\\
0.26026026026026	0.000478656493946144\\
0.261261261261261	0.000491027702275013\\
0.262262262262262	0.000503665969274199\\
0.263263263263263	0.000516575892786621\\
0.264264264264264	0.000529762128523693\\
0.265265265265265	0.000543229390484013\\
0.266266266266266	0.000556982451371474\\
0.267267267267267	0.000571026143012776\\
0.268268268268268	0.00058536535677427\\
0.269269269269269	0.000600005043978109\\
0.27027027027027	0.000614950216317645\\
0.271271271271271	0.000630205946272023\\
0.272272272272272	0.000645777367519954\\
0.273273273273273	0.000661669675352575\\
0.274274274274274	0.000677888127085386\\
0.275275275275275	0.000694438042469211\\
0.276276276276276	0.000711324804100101\\
0.277277277277277	0.000728553857828185\\
0.278278278278278	0.000746130713165371\\
0.279279279279279	0.000764060943691885\\
0.28028028028028	0.000782350187461574\\
0.281281281281281	0.000801004147405922\\
0.282282282282282	0.000820028591736771\\
0.283283283283283	0.000839429354347626\\
0.284284284284284	0.000859212335213563\\
0.285285285285285	0.00087938350078964\\
0.286286286286286	0.000899948884407784\\
0.287287287287287	0.000920914586672104\\
0.288288288288288	0.000942286775852554\\
0.289289289289289	0.000964071688276924\\
0.29029029029029	0.000986275628721095\\
0.291291291291291	0.00100890497079749\\
0.292292292292292	0.00103196615734168\\
0.293293293293293	0.00105546570079711\\
0.294294294294294	0.00107941018359788\\
0.295295295295295	0.00110380625854946\\
0.296296296296296	0.00112866064920743\\
0.297297297297297	0.00115398015025404\\
0.298298298298298	0.00117977162787264\\
0.299299299299299	0.00120604202011982\\
0.3003003003003	0.0012327983372954\\
0.301301301301301	0.00126004766230992\\
0.302302302302302	0.00128779715104984\\
0.303303303303303	0.00131605403274029\\
0.304304304304304	0.00134482561030524\\
0.305305305305305	0.00137411926072518\\
0.306306306306306	0.00140394243539212\\
0.307307307307307	0.00143430266046196\\
0.308308308308308	0.00146520753720411\\
0.309309309309309	0.00149666474234822\\
0.31031031031031	0.00152868202842822\\
0.311311311311311	0.00156126722412331\\
0.312312312312312	0.00159442823459602\\
0.313313313313313	0.0016281730418272\\
0.314314314314314	0.00166250970494801\\
0.315315315315315	0.00169744636056862\\
0.316316316316316	0.00173299122310379\\
0.317317317317317	0.00176915258509513\\
0.318318318318318	0.00180593881753001\\
0.319319319319319	0.00184335837015712\\
0.32032032032032	0.00188141977179845\\
0.321321321321321	0.00192013163065791\\
0.322322322322322	0.0019595026346262\\
0.323323323323323	0.00199954155158209\\
0.324324324324324	0.00204025722969003\\
0.325325325325325	0.00208165859769388\\
0.326326326326326	0.00212375466520687\\
0.327327327327327	0.00216655452299761\\
0.328328328328328	0.00221006734327213\\
0.329329329329329	0.00225430237995191\\
0.33033033033033	0.00229926896894773\\
0.331331331331331	0.00234497652842947\\
0.332332332332332	0.00239143455909151\\
0.333333333333333	0.00243865264441396\\
0.334334334334334	0.00248664045091946\\
0.335335335335335	0.00253540772842555\\
0.336336336336336	0.00258496431029251\\
0.337337337337337	0.00263532011366667\\
0.338338338338338	0.00268648513971907\\
0.339339339339339	0.00273846947387932\\
0.34034034034034	0.00279128328606482\\
0.341341341341341	0.00284493683090506\\
0.342342342342342	0.00289944044796094\\
0.343343343343343	0.00295480456193924\\
0.344344344344344	0.00301103968290191\\
0.345345345345345	0.0030681564064703\\
0.346346346346346	0.00312616541402412\\
0.347347347347347	0.00318507747289523\\
0.348348348348348	0.00324490343655603\\
0.349349349349349	0.00330565424480238\\
0.35035035035035	0.00336734092393115\\
0.351351351351351	0.00342997458691213\\
0.352352352352352	0.00349356643355432\\
0.353353353353353	0.00355812775066657\\
0.354354354354354	0.00362366991221236\\
0.355355355355355	0.00369020437945876\\
0.356356356356356	0.0037577427011195\\
0.357357357357357	0.00382629651349201\\
0.358358358358358	0.00389587754058838\\
0.359359359359359	0.00396649759426015\\
0.36036036036036	0.00403816857431697\\
0.361361361361361	0.00411090246863884\\
0.362362362362362	0.00418471135328202\\
0.363363363363363	0.00425960739257856\\
0.364364364364364	0.00433560283922912\\
0.365365365365365	0.00441271003438946\\
0.366366366366366	0.00449094140774991\\
0.367367367367367	0.00457030947760835\\
0.368368368368368	0.00465082685093628\\
0.369369369369369	0.00473250622343786\\
0.37037037037037	0.00481536037960212\\
0.371371371371371	0.00489940219274798\\
0.372372372372372	0.00498464462506212\\
0.373373373373373	0.00507110072762973\\
0.374374374374374	0.0051587836404578\\
0.375375375375375	0.00524770659249111\\
0.376376376376376	0.00533788290162072\\
0.377377377377377	0.00542932597468486\\
0.378378378378378	0.0055220493074623\\
0.379379379379379	0.00561606648465787\\
0.38038038038038	0.00571139117988027\\
0.381381381381381	0.00580803715561197\\
0.382382382382382	0.0059060182631711\\
0.383383383383383	0.00600534844266542\\
0.384384384384384	0.00610604172293797\\
0.385385385385385	0.00620811222150474\\
0.386386386386386	0.00631157414448381\\
0.387387387387387	0.00641644178651626\\
0.388388388388388	0.0065227295306786\\
0.389389389389389	0.00663045184838656\\
0.39039039039039	0.00673962329929033\\
0.391391391391391	0.00685025853116109\\
0.392392392392392	0.00696237227976863\\
0.393393393393393	0.00707597936875022\\
0.394394394394394	0.00719109470947034\\
0.395395395395395	0.00730773330087151\\
0.396396396396396	0.00742591022931584\\
0.397397397397397	0.00754564066841742\\
0.398398398398398	0.00766693987886532\\
0.399399399399399	0.0077898232082372\\
0.4004004004004	0.00791430609080347\\
0.401401401401401	0.00804040404732175\\
0.402402402402402	0.00816813268482173\\
0.403403403403403	0.00829750769638026\\
0.404404404404404	0.00842854486088649\\
0.405405405405405	0.00856126004279721\\
0.406406406406406	0.00869566919188198\\
0.407407407407407	0.00883178834295826\\
0.408408408408408	0.00896963361561628\\
0.409409409409409	0.00910922121393353\\
0.41041041041041	0.00925056742617898\\
0.411411411411411	0.00939368862450663\\
0.412412412412412	0.00953860126463866\\
0.413413413413413	0.00968532188553774\\
0.414414414414414	0.00983386710906864\\
0.415415415415415	0.00998425363964901\\
0.416416416416416	0.010136498263889\\
0.417417417417417	0.0102906178502203\\
0.418418418418418	0.0104466293485133\\
0.419419419419419	0.0106045497896835\\
0.42042042042042	0.010764396285287\\
0.421421421421421	0.0109261860271031\\
0.422422422422422	0.0110899362867074\\
0.423423423423423	0.0112556644150313\\
0.424424424424424	0.0114233878419111\\
0.425425425425425	0.0115931240756248\\
0.426426426426426	0.0117648907024171\\
0.427427427427427	0.0119387053860122\\
0.428428428428428	0.0121145858671145\\
0.429429429429429	0.0122925499628977\\
0.43043043043043	0.0124726155664807\\
0.431431431431431	0.0126548006463915\\
0.432432432432432	0.0128391232460187\\
0.433433433433433	0.0130256014830504\\
0.434434434434434	0.0132142535489002\\
0.435435435435435	0.0134050977081205\\
0.436436436436436	0.013598152297803\\
0.437437437437437	0.0137934357269658\\
0.438438438438438	0.0139909664759283\\
0.439439439439439	0.0141907630956714\\
0.44044044044044	0.0143928442071864\\
0.441441441441441	0.0145972285008083\\
0.442442442442442	0.0148039347355375\\
0.443443443443443	0.0150129817383467\\
0.444444444444444	0.0152243884034745\\
0.445445445445445	0.0154381736917051\\
0.446446446446446	0.0156543566296344\\
0.447447447447447	0.0158729563089216\\
0.448448448448448	0.0160939918855266\\
0.449449449449449	0.016317482578934\\
0.45045045045045	0.0165434476713621\\
0.451451451451451	0.016771906506957\\
0.452452452452452	0.0170028784909736\\
0.453453453453453	0.0172363830889405\\
0.454454454454454	0.0174724398258109\\
0.455455455455455	0.0177110682850984\\
0.456456456456456	0.017952288107998\\
0.457457457457457	0.0181961189924918\\
0.458458458458458	0.0184425806924398\\
0.459459459459459	0.0186916930166552\\
0.46046046046046	0.0189434758279641\\
0.461461461461461	0.0191979490422503\\
0.462462462462462	0.019455132627484\\
0.463463463463463	0.0197150466027347\\
0.464464464464464	0.0199777110371689\\
0.465465465465465	0.0202431460490308\\
0.466466466466466	0.0205113718046084\\
0.467467467467467	0.0207824085171818\\
0.468468468468468	0.0210562764459567\\
0.469469469469469	0.0213329958949803\\
0.47047047047047	0.0216125872120418\\
0.471471471471471	0.0218950707875552\\
0.472472472472472	0.0221804670534265\\
0.473473473473473	0.0224687964819031\\
0.474474474474474	0.0227600795844069\\
0.475475475475475	0.02305433691035\\
0.476476476476476	0.0233515890459332\\
0.477477477477477	0.023651856612928\\
0.478478478478478	0.0239551602674396\\
0.479479479479479	0.0242615206986543\\
0.48048048048048	0.0245709586275675\\
0.481481481481481	0.0248834948056957\\
0.482482482482482	0.0251991500137689\\
0.483483483483483	0.0255179450604065\\
0.484484484484485	0.0258399007807741\\
0.485485485485485	0.0261650380352226\\
0.486486486486487	0.0264933777079093\\
0.487487487487487	0.0268249407053993\\
0.488488488488488	0.0271597479552501\\
0.48948948948949	0.0274978204045759\\
0.49049049049049	0.0278391790185944\\
0.491491491491492	0.0281838447791537\\
0.492492492492492	0.0285318386832412\\
0.493493493493493	0.0288831817414721\\
0.494494494494495	0.0292378949765599\\
0.495495495495495	0.0295959994217664\\
0.496496496496497	0.0299575161193335\\
0.497497497497497	0.0303224661188939\\
0.498498498498498	0.0306908704758632\\
0.4994994994995	0.0310627502498119\\
0.500500500500501	0.0314381265028166\\
0.501501501501502	0.0318170202977927\\
0.502502502502503	0.032199452696805\\
0.503503503503503	0.0325854447593597\\
0.504504504504504	0.0329750175406746\\
0.505505505505506	0.0333681920899297\\
0.506506506506507	0.0337649894484962\\
0.507507507507508	0.0341654306481453\\
0.508508508508508	0.034569536709236\\
0.509509509509509	0.0349773286388816\\
0.510510510510511	0.035388827429095\\
0.511511511511512	0.0358040540549125\\
0.512512512512513	0.036223029472497\\
0.513513513513513	0.0366457746172186\\
0.514514514514514	0.0370723104017138\\
0.515515515515516	0.0375026577139235\\
0.516516516516517	0.0379368374151078\\
0.517517517517518	0.0383748703378399\\
0.518518518518518	0.0388167772839776\\
0.519519519519519	0.0392625790226114\\
0.520520520520521	0.0397122962879918\\
0.521521521521522	0.0401659497774324\\
0.522522522522523	0.0406235601491913\\
0.523523523523523	0.0410851480203298\\
0.524524524524524	0.0415507339645472\\
0.525525525525526	0.042020338509993\\
0.526526526526527	0.0424939821370559\\
0.527527527527528	0.0429716852761299\\
0.528528528528528	0.0434534683053556\\
0.529529529529529	0.0439393515483395\\
0.530530530530531	0.0444293552718477\\
0.531531531531532	0.0449234996834781\\
0.532532532532533	0.0454218049293058\\
0.533533533533533	0.0459242910915071\\
0.534534534534535	0.0464309781859568\\
0.535535535535536	0.0469418861598027\\
0.536536536536537	0.0474570348890154\\
0.537537537537538	0.0479764441759123\\
0.538538538538539	0.0485001337466579\\
0.53953953953954	0.0490281232487391\\
0.540540540540541	0.0495604322484146\\
0.541541541541542	0.0500970802281401\\
0.542542542542543	0.0506380865839674\\
0.543543543543544	0.0511834706229183\\
0.544544544544545	0.0517332515603331\\
0.545545545545546	0.0522874485171926\\
0.546546546546547	0.0528460805174155\\
0.547547547547548	0.0534091664851284\\
0.548548548548549	0.0539767252419106\\
0.54954954954955	0.054548775504012\\
0.550550550550551	0.0551253358795449\\
0.551551551551552	0.0557064248656493\\
0.552552552552553	0.0562920608456312\\
0.553553553553554	0.0568822620860743\\
0.554554554554555	0.0574770467339244\\
0.555555555555556	0.058076432813547\\
0.556556556556557	0.0586804382237575\\
0.557557557557558	0.0592890807348245\\
0.558558558558559	0.0599023779854438\\
0.55955955955956	0.0605203474796877\\
0.560560560560561	0.0611430065839231\\
0.561561561561562	0.0617703725237046\\
0.562562562562563	0.0624024623806375\\
0.563563563563564	0.0630392930892132\\
0.564564564564565	0.0636808814336171\\
0.565565565565566	0.0643272440445058\\
0.566566566566567	0.0649783973957582\\
0.567567567567568	0.0656343578011958\\
0.568568568568569	0.0662951414112753\\
0.56956956956957	0.0669607642097511\\
0.570570570570571	0.0676312420103092\\
0.571571571571572	0.0683065904531717\\
0.572572572572573	0.0689868250016716\\
0.573573573573574	0.069671960938798\\
0.574574574574575	0.0703620133637113\\
0.575575575575576	0.0710569971882283\\
0.576576576576577	0.071756927133278\\
0.577577577577578	0.0724618177253262\\
0.578578578578579	0.0731716832927702\\
0.57957957957958	0.0738865379623032\\
0.580580580580581	0.0746063956552466\\
0.581581581581582	0.0753312700838535\\
0.582582582582583	0.0760611747475801\\
0.583583583583584	0.0767961229293252\\
0.584584584584585	0.0775361276916405\\
0.585585585585586	0.0782812018729071\\
0.586586586586587	0.0790313580834823\\
0.587587587587588	0.0797866087018135\\
0.588588588588589	0.0805469658705209\\
0.58958958958959	0.0813124414924477\\
0.590590590590591	0.0820830472266783\\
0.591591591591592	0.0828587944845241\\
0.592592592592593	0.0836396944254772\\
0.593593593593594	0.084425757953131\\
0.594594594594595	0.085216995711068\\
0.595595595595596	0.0860134180787147\\
0.596596596596597	0.0868150351671641\\
0.597597597597598	0.0876218568149639\\
0.598598598598599	0.0884338925838717\\
0.5995995995996	0.0892511517545773\\
0.600600600600601	0.09007364332239\\
0.601601601601602	0.0909013759928936\\
0.602602602602603	0.0917343581775655\\
0.603603603603604	0.0925725979893639\\
0.604604604604605	0.0934161032382785\\
0.605605605605606	0.0942648814268481\\
0.606606606606607	0.0951189397456433\\
0.607607607607608	0.095978285068714\\
0.608608608608609	0.0968429239490026\\
0.60960960960961	0.0977128626137212\\
0.610610610610611	0.0985881069596946\\
0.611611611611612	0.0994686625486671\\
0.612612612612613	0.100354534602574\\
0.613613613613614	0.101245727998778\\
0.614614614614615	0.102142247265268\\
0.615615615615616	0.103044096575824\\
0.616616616616617	0.103951279745143\\
0.617617617617618	0.104863800223934\\
0.618618618618619	0.105781661093968\\
0.61961961961962	0.106704865063099\\
0.620620620620621	0.107633414460247\\
0.621621621621622	0.108567311230336\\
0.622622622622623	0.109506556929207\\
0.623623623623624	0.110451152718488\\
0.624624624624625	0.11140109936042\\
0.625625625625626	0.112356397212659\\
0.626626626626627	0.113317046223029\\
0.627627627627628	0.114283045924239\\
0.628628628628629	0.115254395428571\\
0.62962962962963	0.116231093422513\\
0.630630630630631	0.11721313816137\\
0.631631631631632	0.118200527463827\\
0.632632632632633	0.119193258706477\\
0.633633633633634	0.120191328818304\\
0.634634634634635	0.121194734275138\\
0.635635635635636	0.122203471094061\\
0.636636636636637	0.123217534827775\\
0.637637637637638	0.124236920558935\\
0.638638638638639	0.125261622894441\\
0.63963963963964	0.126291635959683\\
0.640640640640641	0.127326953392758\\
0.641641641641642	0.128367568338632\\
0.642642642642643	0.12941347344328\\
0.643643643643644	0.130464660847766\\
0.644644644644645	0.131521122182296\\
0.645645645645646	0.132582848560224\\
0.646646646646647	0.133649830572019\\
0.647647647647648	0.13472205827919\\
0.648648648648649	0.135799521208168\\
0.64964964964965	0.13688220834415\\
0.650650650650651	0.137970108124898\\
0.651651651651652	0.139063208434496\\
0.652652652652653	0.14016149659707\\
0.653653653653654	0.141264959370458\\
0.654654654654655	0.142373582939843\\
0.655655655655656	0.143487352911339\\
0.656656656656657	0.144606254305543\\
0.657657657657658	0.145730271551031\\
0.658658658658659	0.146859388477822\\
0.65965965965966	0.147993588310792\\
0.660660660660661	0.149132853663049\\
0.661661661661662	0.150277166529262\\
0.662662662662663	0.151426508278943\\
0.663663663663664	0.152580859649693\\
0.664664664664665	0.153740200740396\\
0.665665665665666	0.154904511004374\\
0.666666666666667	0.156073769242494\\
0.667667667667668	0.15724795359623\\
0.668668668668669	0.158427041540686\\
0.66966966966967	0.159611009877564\\
0.670670670670671	0.160799834728098\\
0.671671671671672	0.161993491525931\\
0.672672672672673	0.16319195500996\\
0.673673673673674	0.164395199217118\\
0.674674674674675	0.165603197475127\\
0.675675675675676	0.166815922395194\\
0.676676676676677	0.168033345864667\\
0.677677677677678	0.169255439039638\\
0.678678678678679	0.170482172337505\\
0.67967967967968	0.171713515429484\\
0.680680680680681	0.172949437233078\\
0.681681681681682	0.174189905904493\\
0.682682682682683	0.175434888831008\\
0.683683683683684	0.176684352623305\\
0.684684684684685	0.17793826310774\\
0.685685685685686	0.179196585318571\\
0.686686686686687	0.180459283490143\\
0.687687687687688	0.181726321049015\\
0.688688688688689	0.182997660606045\\
0.68968968968969	0.184273263948426\\
0.690690690690691	0.185553092031671\\
0.691691691691692	0.18683710497155\\
0.692692692692693	0.188125262035975\\
0.693693693693694	0.189417521636844\\
0.694694694694695	0.190713841321825\\
0.695695695695696	0.192014177766097\\
0.696696696696697	0.193318486764038\\
0.697697697697698	0.194626723220866\\
0.698698698698699	0.195938841144225\\
0.6996996996997	0.197254793635723\\
0.700700700700701	0.198574532882423\\
0.701701701701702	0.199898010148277\\
0.702702702702703	0.201225175765511\\
0.703703703703704	0.20255597912596\\
0.704704704704705	0.203890368672351\\
0.705705705705706	0.205228291889535\\
0.706706706706707	0.206569695295665\\
0.707707707707708	0.207914524433323\\
0.708708708708709	0.209262723860599\\
0.70970970970971	0.210614237142108\\
0.710710710710711	0.211969006839967\\
0.711711711711712	0.213326974504705\\
0.712712712712713	0.214688080666137\\
0.713713713713714	0.216052264824165\\
0.714714714714715	0.217419465439546\\
0.715715715715716	0.218789619924589\\
0.716716716716717	0.220162664633812\\
0.717717717717718	0.221538534854533\\
0.718718718718719	0.22291716479742\\
0.71971971971972	0.224298487586975\\
0.720720720720721	0.225682435251968\\
0.721721721721722	0.22706893871582\\
0.722722722722723	0.228457927786926\\
0.723723723723724	0.229849331148922\\
0.724724724724725	0.231243076350906\\
0.725725725725726	0.232639089797591\\
0.726726726726727	0.23403729673941\\
0.727727727727728	0.235437621262567\\
0.728728728728729	0.236839986279024\\
0.72972972972973	0.238244313516438\\
0.730730730730731	0.239650523508041\\
0.731731731731732	0.241058535582462\\
0.732732732732733	0.242468267853492\\
0.733733733733734	0.243879637209793\\
0.734734734734735	0.245292559304549\\
0.735735735735736	0.246706948545061\\
0.736736736736737	0.24812271808228\\
0.737737737737738	0.249539779800292\\
0.738738738738739	0.250958044305734\\
0.73973973973974	0.252377420917154\\
0.740740740740741	0.253797817654324\\
0.741741741741742	0.255219141227475\\
0.742742742742743	0.256641297026493\\
0.743743743743744	0.258064189110038\\
0.744744744744745	0.259487720194621\\
0.745745745745746	0.260911791643607\\
0.746746746746747	0.262336303456168\\
0.747747747747748	0.26376115425617\\
0.748748748748749	0.265186241281005\\
0.74974974974975	0.266611460370362\\
0.750750750750751	0.268036705954931\\
0.751751751751752	0.269461871045057\\
0.752752752752753	0.270886847219327\\
0.753753753753754	0.272311524613092\\
0.754754754754755	0.273735791906941\\
0.755755755755756	0.275159536315096\\
0.756756756756757	0.276582643573764\\
0.757757757757758	0.278004997929409\\
0.758758758758759	0.279426482126977\\
0.75975975975976	0.280846977398052\\
0.760760760760761	0.282266363448946\\
0.761761761761762	0.283684518448734\\
0.762762762762763	0.285101319017225\\
0.763763763763764	0.28651664021286\\
0.764764764764765	0.287930355520562\\
0.765765765765766	0.289342336839513\\
0.766766766766767	0.290752454470866\\
0.767767767767768	0.2921605771054\\
0.768768768768769	0.293566571811106\\
0.76976976976977	0.29497030402071\\
0.770770770770771	0.296371637519132\\
0.771771771771772	0.297770434430879\\
0.772772772772773	0.299166555207376\\
0.773773773773774	0.300559858614226\\
0.774774774774775	0.301950201718412\\
0.775775775775776	0.303337439875429\\
0.776776776776777	0.30472142671635\\
0.777777777777778	0.306102014134828\\
0.778778778778779	0.307479052274031\\
0.77977977977978	0.308852389513511\\
0.780780780780781	0.310221872456004\\
0.781781781781782	0.31158734591417\\
0.782782782782783	0.312948652897254\\
0.783783783783784	0.314305634597694\\
0.784784784784785	0.315658130377649\\
0.785785785785786	0.317005977755469\\
0.786786786786787	0.318349012392089\\
0.787787787787788	0.319687068077363\\
0.788788788788789	0.321019976716322\\
0.78978978978979	0.322347568315369\\
0.790790790790791	0.323669670968402\\
0.791791791791792	0.324986110842869\\
0.792792792792793	0.326296712165759\\
0.793793793793794	0.327601297209509\\
0.794794794794795	0.328899686277862\\
0.795795795795796	0.330191697691637\\
0.796796796796797	0.331477147774437\\
0.797797797797798	0.332755850838289\\
0.798798798798799	0.334027619169211\\
0.7997997997998	0.335292263012703\\
0.800800800800801	0.336549590559175\\
0.801801801801802	0.337799407929303\\
0.802802802802803	0.339041519159306\\
0.803803803803804	0.340275726186163\\
0.804804804804805	0.341501828832748\\
0.805805805805806	0.342719624792897\\
0.806806806806807	0.343928909616406\\
0.807807807807808	0.345129476693951\\
0.808808808808809	0.346321117241938\\
0.80980980980981	0.347503620287281\\
0.810810810810811	0.348676772652104\\
0.811811811811812	0.349840358938375\\
0.812812812812813	0.350994161512457\\
0.813813813813814	0.352137960489596\\
0.814814814814815	0.35327153371833\\
0.815815815815816	0.354394656764821\\
0.816816816816817	0.355507102897116\\
0.817817817817818	0.356608643069336\\
0.818818818818819	0.357699045905783\\
0.81981981981982	0.358778077684978\\
0.820820820820821	0.359845502323621\\
0.821821821821822	0.360901081360474\\
0.822822822822823	0.361944573940174\\
0.823823823823824	0.362975736796961\\
0.824824824824825	0.363994324238341\\
0.825825825825826	0.365000088128658\\
0.826826826826827	0.365992777872607\\
0.827827827827828	0.366972140398653\\
0.828828828828829	0.367937920142384\\
0.82982982982983	0.368889859029783\\
0.830830830830831	0.369827696460423\\
0.831831831831832	0.370751169290582\\
0.832832832832833	0.371660011816282\\
0.833833833833834	0.372553955756249\\
0.834834834834835	0.373432730234799\\
0.835835835835836	0.374296061764634\\
0.836836836836837	0.375143674229571\\
0.837837837837838	0.375975288867185\\
0.838838838838839	0.376790624251376\\
0.83983983983984	0.377589396274852\\
0.840840840840841	0.378371318131537\\
0.841841841841842	0.379136100298897\\
0.842842842842843	0.379883450520181\\
0.843843843843844	0.380613073786593\\
0.844844844844845	0.381324672319368\\
0.845845845845846	0.382017945551781\\
0.846846846846847	0.382692590111064\\
0.847847847847848	0.383348299800249\\
0.848848848848849	0.383984765579926\\
0.84984984984985	0.384601675549915\\
0.850850850850851	0.385198714930865\\
0.851851851851852	0.385775566045762\\
0.852852852852853	0.386331908301361\\
0.853853853853854	0.386867418169529\\
0.854854854854855	0.387381769168506\\
0.855855855855856	0.387874631844089\\
0.856856856856857	0.388345673750725\\
0.857857857857858	0.388794559432521\\
0.858858858858859	0.389220950404173\\
0.85985985985986	0.38962450513181\\
0.860860860860861	0.390004879013752\\
0.861861861861862	0.390361724361181\\
0.862862862862863	0.390694690378734\\
0.863863863863864	0.391003423145001\\
0.864864864864865	0.391287565592949\\
0.865865865865866	0.391546757490249\\
0.866866866866867	0.391780635419522\\
0.867867867867868	0.391988832758501\\
0.868868868868869	0.392170979660103\\
0.86986986986987	0.392326703032415\\
0.870870870870871	0.392455626518592\\
0.871871871871872	0.392557370476672\\
0.872872872872873	0.392631551959299\\
0.873873873873874	0.392677784693358\\
0.874874874874875	0.392695679059526\\
0.875875875875876	0.392684842071733\\
0.876876876876877	0.392644877356532\\
0.877877877877878	0.392575385132385\\
0.878878878878879	0.392475962188855\\
0.87987987987988	0.392346201865714\\
0.880880880880881	0.392185694031954\\
0.881881881881882	0.391994025064721\\
0.882882882882883	0.391770777828145\\
0.883883883883884	0.391515531652085\\
0.884884884884885	0.391227862310793\\
0.885885885885886	0.39090734200147\\
0.886886886886887	0.390553539322745\\
0.887887887887888	0.390166019253057\\
0.888888888888889	0.389744343128946\\
0.88988988988989	0.389288068623253\\
0.890890890890891	0.38879674972323\\
0.891891891891892	0.388269936708553\\
0.892892892892893	0.387707176129247\\
0.893893893893894	0.387108010783515\\
0.894894894894895	0.386471979695482\\
0.895895895895896	0.38579861809283\\
0.896896896896897	0.38508745738436\\
0.897897897897898	0.384338025137441\\
0.898898898898899	0.38354984505538\\
0.8998998998999	0.382722436954689\\
0.900900900900901	0.381855316742259\\
0.901901901901902	0.380947996392444\\
0.902902902902903	0.379999983924048\\
0.903903903903904	0.379010783377209\\
0.904904904904905	0.3779798947902\\
0.905905905905906	0.376906814176129\\
0.906906906906907	0.375791033499539\\
0.907907907907908	0.374632040652918\\
0.908908908908909	0.373429319433111\\
0.90990990990991	0.372182349517632\\
0.910910910910911	0.370890606440884\\
0.911911911911912	0.369553561570278\\
0.912912912912913	0.368170682082257\\
0.913913913913914	0.36674143093822\\
0.914914914914915	0.365265266860352\\
0.915915915915916	0.36374164430735\\
0.916916916916917	0.362170013450057\\
0.917917917917918	0.360549820146991\\
0.918918918918919	0.35888050591978\\
0.91991991991992	0.357161507928496\\
0.920920920920921	0.355392258946889\\
0.921921921921922	0.353572187337521\\
0.922922922922923	0.351700717026802\\
0.923923923923924	0.349777267479923\\
0.924924924924925	0.347801253675692\\
0.925925925925926	0.345772086081265\\
0.926926926926927	0.343689170626779\\
0.927927927927928	0.341551908679883\\
0.928928928928929	0.339359697020169\\
0.92992992992993	0.337111927813496\\
0.930930930930931	0.334807988586218\\
0.931931931931932	0.332447262199305\\
0.932932932932933	0.330029126822367\\
0.933933933933934	0.327552955907569\\
0.934934934934935	0.325018118163445\\
0.935935935935936	0.322423977528613\\
0.936936936936937	0.319769893145376\\
0.937937937937938	0.317055219333234\\
0.938938938938939	0.314279305562276\\
0.93993993993994	0.31144149642648\\
0.940940940940941	0.308541131616899\\
0.941941941941942	0.305577545894752\\
0.942942942942943	0.302550069064399\\
0.943943943943944	0.29945802594622\\
0.944944944944945	0.296300736349379\\
0.945945945945946	0.293077515044492\\
0.946946946946947	0.28978767173618\\
0.947947947947948	0.286430511035523\\
0.948948948948949	0.283005332432397\\
0.94994994994995	0.279511430267717\\
0.950950950950951	0.275948093705563\\
0.951951951951952	0.2723146067052\\
0.952952952952953	0.268610247992992\\
0.953953953953954	0.264834291034209\\
0.954954954954955	0.260986004004723\\
0.955955955955956	0.257064649762595\\
0.956956956956957	0.253069485819554\\
0.957957957957958	0.24899976431237\\
0.958958958958959	0.24485473197411\\
0.95995995995996	0.240633630105293\\
0.960960960960961	0.236335694544927\\
0.961961961961962	0.23196015564144\\
0.962962962962963	0.227506238223504\\
0.963963963963964	0.222973161570735\\
0.964964964964965	0.218360139384303\\
0.965965965965966	0.213666379757406\\
0.966966966966967	0.208891085145653\\
0.967967967967968	0.204033452337323\\
0.968968968968969	0.199092672423517\\
0.96996996996997	0.194067930768196\\
0.970970970970971	0.188958406978106\\
0.971971971971972	0.183763274872589\\
0.972972972972973	0.178481702453283\\
0.973973973973974	0.173112851873711\\
0.974974974974975	0.167655879408746\\
0.975975975975976	0.162109935423971\\
0.976976976976977	0.156474164344924\\
0.977977977977978	0.150747704626225\\
0.978978978978979	0.144929688720588\\
0.97997997997998	0.139019243047721\\
0.980980980980981	0.133015487963107\\
0.981981981981982	0.126917537726672\\
0.982982982982983	0.120724500471335\\
0.983983983983984	0.114435478171444\\
0.984984984984985	0.108049566611091\\
0.985985985985986	0.101565855352314\\
0.986986986986987	0.0949834277031814\\
0.987987987987988	0.0883013606857575\\
0.988988988988989	0.0815187250039529\\
0.98998998998999	0.074634585011251\\
0.990990990990991	0.0676479986783234\\
0.991991991991992	0.060558017560522\\
0.992992992992993	0.0533636867652538\\
0.993993993993994	0.0460640449192386\\
0.994994994994995	0.0386581241356416\\
0.995995995995996	0.0311449499810917\\
0.996996996996997	0.0235235414425781\\
0.997997997997998	0.0157929108942253\\
0.998998998998999	0.00795206406395167\\
1	0\\
};
\addlegendentry{likelihood};

\addplot [color=mycolor3,solid]
  table[row sep=crcr]{%
0	0\\
0.001001001001001	3.01502698478937e-23\\
0.002002002002002	1.53597536260669e-20\\
0.003003003003003	5.8752517649496e-19\\
0.004004004004004	7.7856311117508e-18\\
0.005005005005005	5.77165625920614e-17\\
0.00600600600600601	2.96309784725133e-16\\
0.00700700700700701	1.1805359528958e-15\\
0.00800800800800801	3.90675927957949e-15\\
0.00900900900900901	1.12201084573405e-14\\
0.01001001001001	2.88150883175238e-14\\
0.011011011011011	6.76016640407232e-14\\
0.012012012012012	1.47182262552135e-13\\
0.013013013013013	3.00963093710925e-13\\
0.014014014014014	5.83407449932823e-13\\
0.015015015015015	1.08002876064384e-12\\
0.016016016016016	1.92081673176816e-12\\
0.017017017017017	3.29789047512997e-12\\
0.018018018018018	5.4882777233973e-12\\
0.019019019019019	8.88282863845331e-12\\
0.02002002002002	1.40223859703951e-11\\
0.021021021021021	2.1642450036253e-11\\
0.022022022022022	3.27276967973595e-11\\
0.023023023023023	4.85778298671653e-11\\
0.024024024024024	7.0886372231859e-11\\
0.025025025025025	1.01834130478531e-10\\
0.026026026026026	1.44199193073289e-10\\
0.027027027027027	2.01485454397315e-10\\
0.028028028028028	2.78071787524176e-10\\
0.029029029029029	3.79384120807349e-10\\
0.03003003003003	5.1209280595241e-10\\
0.031031031031031	6.84337798092235e-10\\
0.032032032032032	9.0598430119198e-10\\
0.033033033033033	1.18891166461975e-09\\
0.034034034034034	1.54733844867149e-09\\
0.035035035035035	1.99818670798725e-09\\
0.036036036036036	2.56148867190026e-09\\
0.037037037037037	3.260839129588e-09\\
0.038038038038038	4.12389695522211e-09\\
0.039039039039039	5.18293933392413e-09\\
0.04004004004004	6.47547237301908e-09\\
0.041041041041041	8.04490190463683e-09\\
0.042042042042042	9.94126840494823e-09\\
0.043043043043043	1.22220500720524e-08\\
0.044044044044044	1.4953038218576e-08\\
0.045045045045045	1.82092892462285e-08\\
0.046046046046046	2.20761575777181e-08\\
0.047047047047047	2.66504140264106e-08\\
0.048048048048048	3.20414541857601e-08\\
0.049049049049049	3.83726015186958e-08\\
0.0500500500500501	4.57825099217108e-08\\
0.0510510510510511	5.44266706291784e-08\\
0.0520520520520521	6.44790284103582e-08\\
0.0530530530530531	7.61337120944707e-08\\
0.0540540540540541	8.9606884538044e-08\\
0.0550550550550551	1.05138717223333e-07\\
0.0560560560560561	1.22995494746907e-07\\
0.0570570570570571	1.43471854523356e-07\\
0.0580580580580581	1.66893167090466e-07\\
0.0590590590590591	1.93618062458969e-07\\
0.0600600600600601	2.24041108002109e-07\\
0.0610610610610611	2.58595643427635e-07\\
0.0620620620620621	2.9775677841741e-07\\
0.0630630630630631	3.42044558557544e-07\\
0.0640640640640641	3.92027305214693e-07\\
0.0650650650650651	4.4832513504201e-07\\
0.0660660660660661	5.11613664820962e-07\\
0.0670670670670671	5.82627907362976e-07\\
0.0680680680680681	6.62166364207405e-07\\
0.0690690690690691	7.510953208598e-07\\
0.0700700700700701	8.50353350316752e-07\\
0.0710710710710711	9.60956030620614e-07\\
0.0720720720720721	1.08400088217931e-06\\
0.0730730730730731	1.22067253057312e-06\\
0.0740740740740741	1.37224810055165e-06\\
0.0750750750750751	1.54010284690033e-06\\
0.0760760760760761	1.72571602782682e-06\\
0.0770770770770771	1.93067702648328e-06\\
0.0780780780780781	2.15669172620072e-06\\
0.0790790790790791	2.405589144967e-06\\
0.0800800800800801	2.67932833463004e-06\\
0.0810810810810811	2.98000555025215e-06\\
0.0820820820820821	3.30986169498153e-06\\
0.0830830830830831	3.6712900457407e-06\\
0.0840840840840841	4.06684426496137e-06\\
0.0850850850850851	4.49924670351917e-06\\
0.0860860860860861	4.97139699994071e-06\\
0.0870870870870871	5.48638098086935e-06\\
0.0880880880880881	6.0474798676853e-06\\
0.0890890890890891	6.65817979407953e-06\\
0.0900900900900901	7.32218163927999e-06\\
0.0910910910910911	8.04341118152345e-06\\
0.0920920920920921	8.82602957625437e-06\\
0.0930930930930931	9.67444416341917e-06\\
0.0940940940940941	1.05933196081016e-05\\
0.0950950950950951	1.15875893786219e-05\\
0.0960960960960961	1.26624675660931e-05\\
0.0970970970970971	1.38234610492936e-05\\
0.0980980980980981	1.50763820085761e-05\\
0.0990990990990991	1.64273607923922e-05\\
0.1001001001001	1.78828591398642e-05\\
0.101101101101101	1.94496837626825e-05\\
0.102102102102102	2.11350002894559e-05\\
0.103103103103103	2.29463475754768e-05\\
0.104104104104104	2.48916523807077e-05\\
0.105105105105105	2.69792444186175e-05\\
0.106106106106106	2.92178717783352e-05\\
0.107107107107107	3.16167167224086e-05\\
0.108108108108108	3.41854118622727e-05\\
0.109109109109109	3.69340567133617e-05\\
0.11011011011011	3.98732346315989e-05\\
0.111111111111111	4.30140301328215e-05\\
0.112112112112112	4.6368046596496e-05\\
0.113113113113113	4.99474243548882e-05\\
0.114114114114114	5.3764859168655e-05\\
0.115115115115115	5.78336210896152e-05\\
0.116116116116116	6.21675737112606e-05\\
0.117117117117117	6.67811938073574e-05\\
0.118118118118118	7.16895913587732e-05\\
0.119119119119119	7.69085299684579e-05\\
0.12012012012012	8.24544476642903e-05\\
0.121121121121121	8.8344478089278e-05\\
0.122122122122122	9.4596472078387e-05\\
0.123123123123123	0.000101229019621043\\
0.124124124124124	0.000108261472208138\\
0.125125125125125	0.000115713965562126\\
0.126126126126126	0.000123607442748592\\
0.127127127127127	0.000131963677667413\\
0.128128128128128	0.000140805298921433\\
0.129129129129129	0.000150155814060315\\
0.13013013013013	0.000160039634197001\\
0.131131131131131	0.00017048209899399\\
0.132132132132132	0.000181509502016392\\
0.133133133133133	0.000193149116448465\\
0.134134134134134	0.000205429221170141\\
0.135135135135135	0.00021837912718976\\
0.136136136136136	0.00023202920442902\\
0.137137137137137	0.000246410908855901\\
0.138138138138138	0.000261556809961063\\
0.139139139139139	0.000277500618572998\\
0.14014014014014	0.000294277215006953\\
0.141141141141141	0.000311922677542404\\
0.142142142142142	0.000330474311223627\\
0.143143143143143	0.000349970676977621\\
0.144144144144144	0.000370451621043491\\
0.145145145145145	0.000391958304707016\\
0.146146146146146	0.00041453323433401\\
0.147147147147147	0.000438220291695792\\
0.148148148148148	0.000463064764579782\\
0.149149149149149	0.000489113377678115\\
0.15015015015015	0.0005164143237468\\
0.151151151151151	0.000545017295027821\\
0.152152152152152	0.000574973514926236\\
0.153153153153153	0.000606335769934157\\
0.154154154154154	0.000639158441793252\\
0.155155155155155	0.000673497539887123\\
0.156156156156156	0.000709410733854756\\
0.157157157157157	0.000746957386415912\\
0.158158158158158	0.000786198586399188\\
0.159159159159159	0.000827197181963216\\
0.16016016016016	0.000870017814001131\\
0.161161161161161	0.000914726949718456\\
0.162162162162162	0.00096139291637405\\
0.163163163163163	0.00101008593517374\\
0.164164164164164	0.00106087815530592\\
0.165165165165165	0.00111384368810837\\
0.166166166166166	0.00116905864135488\\
0.167167167167167	0.00122660115365078\\
0.168168168168168	0.00128655142892547\\
0.169169169169169	0.00134899177101047\\
0.17017017017017	0.00141400661829086\\
0.171171171171171	0.00148168257841821\\
0.172172172172172	0.00155210846307228\\
0.173173173173173	0.00162537532275938\\
0.174174174174174	0.00170157648163432\\
0.175175175175175	0.00178080757233315\\
0.176176176176176	0.00186316657080366\\
0.177177177177177	0.00194875383112013\\
0.178178178178178	0.00203767212026912\\
0.179179179179179	0.00213002665289236\\
0.18018018018018	0.00222592512597319\\
0.181181181181181	0.00232547775345231\\
0.182182182182182	0.0024287973007589\\
0.183183183183183	0.00253599911924255\\
0.184184184184184	0.00264720118049179\\
0.185185185185185	0.0027625241105243\\
0.186186186186186	0.00288209122383429\\
0.187187187187187	0.00300602855728189\\
0.188188188188188	0.0031344649038096\\
0.189189189189189	0.00326753184597062\\
0.19019019019019	0.00340536378925367\\
0.191191191191191	0.00354809799518876\\
0.192192192192192	0.00369587461421847\\
0.193193193193193	0.00384883671831897\\
0.194194194194194	0.00400713033335492\\
0.195195195195195	0.00417090447115233\\
0.196196196196196	0.00434031116127346\\
0.197197197197197	0.0045155054824774\\
0.198198198198198	0.00469664559385036\\
0.199199199199199	0.00488389276558911\\
0.2002002002002	0.00507741140942143\\
0.201201201201201	0.00527736910864679\\
0.202202202202202	0.00548393664778093\\
0.203203203203203	0.00569728804178773\\
0.204204204204204	0.00591760056488138\\
0.205205205205205	0.00614505477888261\\
0.206206206206206	0.00637983456111157\\
0.207207207207207	0.00662212713180113\\
0.208208208208208	0.00687212308101325\\
0.209209209209209	0.00713001639504169\\
0.21021021021021	0.00739600448228409\\
0.211211211211211	0.00767028819856648\\
0.212212212212212	0.00795307187190301\\
0.213213213213213	0.00824456332667423\\
0.214214214214214	0.00854497390720652\\
0.215215215215215	0.00885451850073606\\
0.216216216216216	0.00917341555973961\\
0.217217217217217	0.00950188712361624\\
0.218218218218218	0.00984015883970178\\
0.219219219219219	0.0101884599835997\\
0.22022022022022	0.0105470234788113\\
0.221221221221221	0.0109160859156484\\
0.222222222222222	0.0112958875694114\\
0.223223223223223	0.0116866724178164\\
0.224224224224224	0.012088688157653\\
0.225225225225225	0.0125021862206592\\
0.226226226226226	0.0129274217885928\\
0.227227227227227	0.0133646538074859\\
0.228228228228228	0.0138141450010641\\
0.229229229229229	0.0142761618833151\\
0.23023023023023	0.0147509747701889\\
0.231231231231231	0.0152388577904152\\
0.232232232232232	0.0157400888954201\\
0.233233233233233	0.0162549498683269\\
0.234234234234234	0.0167837263320251\\
0.235235235235235	0.0173267077562914\\
0.236236236236236	0.0178841874639463\\
0.237237237237237	0.0184564626360333\\
0.238238238238238	0.0190438343160009\\
0.239239239239239	0.019646607412877\\
0.24024024024024	0.0202650907034159\\
0.241241241241241	0.0208995968332067\\
0.242242242242242	0.0215504423167253\\
0.243243243243243	0.0222179475363171\\
0.244244244244244	0.0229024367400953\\
0.245245245245245	0.0236042380387398\\
0.246246246246246	0.0243236834011836\\
0.247247247247247	0.0250611086491721\\
0.248248248248248	0.0258168534506806\\
0.249249249249249	0.0265912613121789\\
0.25025025025025	0.0273846795697266\\
0.251251251251251	0.0281974593788875\\
0.252252252252252	0.0290299557034506\\
0.253253253253253	0.029882527302943\\
0.254254254254254	0.0307555367189244\\
0.255255255255255	0.0316493502600489\\
0.256256256256256	0.0325643379858836\\
0.257257257257257	0.0335008736894706\\
0.258258258258258	0.0344593348786232\\
0.259259259259259	0.035440102755941\\
0.26026026026026	0.036443562197538\\
0.261261261261261	0.0374701017304675\\
0.262262262262262	0.0385201135088383\\
0.263263263263263	0.0395939932886072\\
0.264264264264264	0.0406921404010416\\
0.265265265265265	0.0418149577248411\\
0.266266266266266	0.0429628516569067\\
0.267267267267267	0.0441362320817527\\
0.268268268268268	0.0453355123395486\\
0.269269269269269	0.0465611091927842\\
0.27027027027027	0.04781344279155\\
0.271271271271271	0.0490929366374239\\
0.272272272272272	0.0504000175459573\\
0.273273273273273	0.0517351156077535\\
0.274274274274274	0.0530986641481306\\
0.275275275275275	0.0544910996853644\\
0.276276276276276	0.0559128618875012\\
0.277277277277277	0.0573643935277386\\
0.278278278278278	0.0588461404383678\\
0.279279279279279	0.0603585514632682\\
0.28028028028028	0.0619020784089576\\
0.281281281281281	0.0634771759941832\\
0.282282282282282	0.0650843017980594\\
0.283283283283283	0.0667239162067404\\
0.284284284284284	0.0683964823586297\\
0.285285285285285	0.0701024660881222\\
0.286286286286286	0.0718423358678706\\
0.287287287287287	0.0736165627495863\\
0.288288288288288	0.075425620303359\\
0.289289289289289	0.0772699845555031\\
0.29029029029029	0.0791501339249254\\
0.291291291291291	0.0810665491580156\\
0.292292292292292	0.0830197132620566\\
0.293293293293293	0.0850101114371581\\
0.294294294294294	0.0870382310067117\\
0.295295295295295	0.0891045613463685\\
0.296296296296296	0.0912095938115408\\
0.297297297297297	0.093353821663432\\
0.298298298298298	0.0955377399935921\\
0.299299299299299	0.0977618456470051\\
0.3003003003003	0.10002663714371\\
0.301301301301301	0.102332614598962\\
0.302302302302302	0.104680279641925\\
0.303303303303303	0.107070135332925\\
0.304304304304304	0.109502686079241\\
0.305305305305305	0.111978437549453\\
0.306306306306306	0.114497896586355\\
0.307307307307307	0.117061571118434\\
0.308308308308308	0.119669970069921\\
0.309309309309309	0.122323603269419\\
0.31031031031031	0.125022981357122\\
0.311311311311311	0.127768615690634\\
0.312312312312312	0.130561018249382\\
0.313313313313313	0.133400701537643\\
0.314314314314314	0.136288178486203\\
0.315315315315315	0.13922396235263\\
0.316316316316316	0.142208566620203\\
0.317317317317317	0.145242504895484\\
0.318318318318318	0.148326290804553\\
0.319319319319319	0.151460437887919\\
0.32032032032032	0.154645459494113\\
0.321321321321321	0.157881868671987\\
0.322322322322322	0.161170178061711\\
0.323323323323323	0.164510899784504\\
0.324324324324324	0.167904545331095\\
0.325325325325325	0.171351625448943\\
0.326326326326326	0.174852650028215\\
0.327327327327327	0.178408127986551\\
0.328328328328328	0.182018567152619\\
0.329329329329329	0.185684474148491\\
0.33033033033033	0.189406354270836\\
0.331331331331331	0.193184711370972\\
0.332332332332332	0.197020047733764\\
0.333333333333333	0.200912863955421\\
0.334334334334334	0.204863658820176\\
0.335335335335335	0.208872929175886\\
0.336336336336336	0.212941169808577\\
0.337337337337337	0.217068873315934\\
0.338338338338338	0.221256529979769\\
0.339339339339339	0.225504627637488\\
0.34034034034034	0.229813651552571\\
0.341341341341341	0.234184084284088\\
0.342342342342342	0.238616405555281\\
0.343343343343343	0.243111092121221\\
0.344344344344344	0.247668617635568\\
0.345345345345345	0.252289452516464\\
0.346346346346346	0.256974063811573\\
0.347347347347347	0.261722915062288\\
0.348348348348348	0.266536466167152\\
0.349349349349349	0.271415173244479\\
0.35035035035035	0.276359488494249\\
0.351351351351351	0.281369860059251\\
0.352352352352352	0.286446731885543\\
0.353353353353353	0.291590543582232\\
0.354354354354354	0.296801730280592\\
0.355355355355355	0.302080722492589\\
0.356356356356356	0.307427945968784\\
0.357357357357357	0.312843821555684\\
0.358358358358358	0.318328765052559\\
0.359359359359359	0.323883187067736\\
0.36036036036036	0.329507492874427\\
0.361361361361361	0.335202082266089\\
0.362362362362362	0.340967349411389\\
0.363363363363363	0.346803682708745\\
0.364364364364364	0.352711464640531\\
0.365365365365365	0.358691071626947\\
0.366366366366366	0.364742873879577\\
0.367367367367367	0.370867235254693\\
0.368368368368368	0.377064513106326\\
0.369369369369369	0.383335058139112\\
0.37037037037037	0.389679214260993\\
0.371371371371371	0.396097318435765\\
0.372372372372372	0.402589700535514\\
0.373373373373373	0.409156683193005\\
0.374374374374374	0.415798581654002\\
0.375375375375375	0.422515703629605\\
0.376376376376376	0.429308349148613\\
0.377377377377377	0.43617681040994\\
0.378378378378378	0.44312137163515\\
0.379379379379379	0.450142308921103\\
0.38038038038038	0.457239890092799\\
0.381381381381381	0.464414374556399\\
0.382382382382382	0.471666013152506\\
0.383383383383383	0.47899504800972\\
0.384384384384384	0.486401712398503\\
0.385385385385385	0.493886230585399\\
0.386386386386386	0.501448817687639\\
0.387387387387387	0.50908967952817\\
0.388388388388388	0.516809012491152\\
0.389389389389389	0.524607003377947\\
0.39039039039039	0.53248382926365\\
0.391391391391391	0.540439657354194\\
0.392392392392392	0.548474644844063\\
0.393393393393393	0.556588938774675\\
0.394394394394394	0.564782675893423\\
0.395395395395395	0.573055982513476\\
0.396396396396396	0.581408974374335\\
0.397397397397397	0.589841756503187\\
0.398398398398398	0.598354423077118\\
0.399399399399399	0.606947057286203\\
0.4004004004004	0.615619731197519\\
0.401401401401401	0.624372505620131\\
0.402402402402402	0.63320542997106\\
0.403403403403403	0.642118542142325\\
0.404404404404404	0.651111868369025\\
0.405405405405405	0.660185423098586\\
0.406406406406406	0.669339208861145\\
0.407407407407407	0.678573216141128\\
0.408408408408408	0.687887423250098\\
0.409409409409409	0.697281796200857\\
0.41041041041041	0.706756288582883\\
0.411411411411411	0.71631084143913\\
0.412412412412412	0.725945383144211\\
0.413413413413413	0.735659829284062\\
0.414414414414414	0.745454082537033\\
0.415415415415415	0.755328032556551\\
0.416416416416416	0.765281555855317\\
0.417417417417417	0.775314515691104\\
0.418418418418418	0.785426761954224\\
0.419419419419419	0.795618131056634\\
0.42042042042042	0.805888445822801\\
0.421421421421421	0.816237515382308\\
0.422422422422422	0.826665135064269\\
0.423423423423423	0.837171086293581\\
0.424424424424424	0.847755136489055\\
0.425425425425425	0.858417038963474\\
0.426426426426426	0.869156532825601\\
0.427427427427427	0.879973342884186\\
0.428428428428428	0.890867179554019\\
0.429429429429429	0.901837738764045\\
0.43043043043043	0.912884701867615\\
0.431431431431431	0.924007735554851\\
0.432432432432432	0.935206491767252\\
0.433433433433433	0.946480607614496\\
0.434434434434434	0.957829705293513\\
0.435435435435435	0.969253392009882\\
0.436436436436436	0.980751259901546\\
0.437437437437437	0.992322885964927\\
0.438438438438438	1.00396783198344\\
0.439439439439439	1.0156856444585\\
0.44044044044044	1.02747585454296\\
0.441441441441441	1.03933797797712\\
0.442442442442442	1.05127151502729\\
0.443443443443443	1.06327595042695\\
0.444444444444444	1.07535075332054\\
0.445445445445445	1.08749537720991\\
0.446446446446446	1.09970925990352\\
0.447447447447447	1.1119918234683\\
0.448448448448448	1.12434247418437\\
0.449449449449449	1.1367606025025\\
0.45045045045045	1.14924558300444\\
0.451451451451451	1.16179677436613\\
0.452452452452452	1.17441351932375\\
0.453453453453453	1.18709514464285\\
0.454454454454454	1.19984096109027\\
0.455455455455455	1.21265026340919\\
0.456456456456456	1.22552233029721\\
0.457457457457457	1.23845642438737\\
0.458458458458458	1.25145179223241\\
0.459459459459459	1.26450766429208\\
0.46046046046046	1.27762325492358\\
0.461461461461461	1.29079776237524\\
0.462462462462462	1.30403036878339\\
0.463463463463463	1.31732024017241\\
0.464464464464464	1.33066652645812\\
0.465465465465465	1.34406836145441\\
0.466466466466466	1.35752486288323\\
0.467467467467467	1.37103513238785\\
0.468468468468468	1.38459825554958\\
0.469469469469469	1.39821330190781\\
0.47047047047047	1.41187932498344\\
0.471471471471471	1.4255953623059\\
0.472472472472472	1.43936043544342\\
0.473473473473473	1.45317355003695\\
0.474474474474474	1.46703369583749\\
0.475475475475475	1.48093984674701\\
0.476476476476476	1.49489096086287\\
0.477477477477477	1.5088859805258\\
0.478478478478478	1.52292383237152\\
0.479479479479479	1.5370034273859\\
0.48048048048048	1.55112366096377\\
0.481481481481481	1.5652834129714\\
0.482482482482482	1.57948154781254\\
0.483483483483483	1.59371691449823\\
0.484484484484485	1.60798834672025\\
0.485485485485485	1.62229466292825\\
0.486486486486487	1.63663466641064\\
0.487487487487487	1.65100714537916\\
0.488488488488488	1.6654108730572\\
0.48948948948949	1.67984460777189\\
0.49049049049049	1.69430709304993\\
0.491491491491492	1.70879705771718\\
0.492492492492492	1.72331321600205\\
0.493493493493493	1.73785426764264\\
0.494494494494495	1.75241889799773\\
0.495495495495495	1.7670057781615\\
0.496496496496497	1.78161356508215\\
0.497497497497497	1.79624090168419\\
0.498498498498498	1.81088641699468\\
0.4994994994995	1.82554872627324\\
0.500500500500501	1.84022643114586\\
0.501501501501502	1.85491811974255\\
0.502502502502503	1.86962236683879\\
0.503503503503503	1.88433773400088\\
0.504504504504504	1.89906276973499\\
0.505505505505506	1.91379600964016\\
0.506506506506507	1.92853597656502\\
0.507507507507508	1.94328118076835\\
0.508508508508508	1.95803012008349\\
0.509509509509509	1.97278128008653\\
0.510510510510511	1.98753313426826\\
0.511511511511512	2.00228414420996\\
0.512512512512513	2.01703275976297\\
0.513513513513513	2.03177741923203\\
0.514514514514514	2.04651654956233\\
0.515515515515516	2.06124856653044\\
0.516516516516517	2.07597187493885\\
0.517517517517518	2.0906848688143\\
0.518518518518518	2.10538593160986\\
0.519519519519519	2.12007343641062\\
0.520520520520521	2.13474574614317\\
0.521521521521522	2.14940121378864\\
0.522522522522523	2.16403818259955\\
0.523523523523523	2.17865498632009\\
0.524524524524524	2.19324994941018\\
0.525525525525526	2.20782138727305\\
0.526526526526527	2.22236760648643\\
0.527527527527528	2.23688690503723\\
0.528528528528528	2.25137757255984\\
0.529529529529529	2.26583789057782\\
0.530530530530531	2.2802661327492\\
0.531531531531532	2.2946605651151\\
0.532532532532533	2.30901944635186\\
0.533533533533533	2.32334102802658\\
0.534534534534535	2.33762355485592\\
0.535535535535536	2.35186526496839\\
0.536536536536537	2.3660643901698\\
0.537537537537538	2.38021915621207\\
0.538538538538539	2.39432778306529\\
0.53953953953954	2.40838848519293\\
0.540540540540541	2.42239947183023\\
0.541541541541542	2.43635894726581\\
0.542542542542543	2.45026511112625\\
0.543543543543544	2.46411615866383\\
0.544544544544545	2.47791028104723\\
0.545545545545546	2.49164566565527\\
0.546546546546547	2.50532049637351\\
0.547547547547548	2.51893295389386\\
0.548548548548549	2.53248121601688\\
0.54954954954955	2.54596345795709\\
0.550550550550551	2.55937785265089\\
0.551551551551552	2.57272257106723\\
0.552552552552553	2.58599578252102\\
0.553553553553554	2.59919565498911\\
0.554554554554555	2.61232035542879\\
0.555555555555556	2.62536805009897\\
0.556556556556557	2.63833690488368\\
0.557557557557558	2.65122508561801\\
0.558558558558559	2.66403075841657\\
0.55955955955956	2.67675209000403\\
0.560560560560561	2.68938724804812\\
0.561561561561562	2.70193440149471\\
0.562562562562563	2.7143917209051\\
0.563563563563564	2.72675737879534\\
0.564564564564565	2.73902954997767\\
0.565565565565566	2.75120641190384\\
0.566566566566567	2.76328614501034\\
0.567567567567568	2.77526693306559\\
0.568568568568569	2.78714696351876\\
0.56956956956957	2.79892442785041\\
0.570570570570571	2.81059752192477\\
0.571571571571572	2.82216444634358\\
0.572572572572573	2.83362340680152\\
0.573573573573574	2.844972614443\\
0.574574574574575	2.85621028622042\\
0.575575575575576	2.86733464525371\\
0.576576576576577	2.87834392119117\\
0.577577577577578	2.88923635057142\\
0.578578578578579	2.90001017718651\\
0.57957957957958	2.91066365244605\\
0.580580580580581	2.9211950357423\\
0.581581581581582	2.93160259481616\\
0.582582582582583	2.94188460612395\\
0.583583583583584	2.952039355205\\
0.584584584584585	2.9620651370498\\
0.585585585585586	2.97196025646877\\
0.586586586586587	2.98172302846162\\
0.587587587587588	2.99135177858702\\
0.588588588588589	3.0008448433327\\
0.58958958958959	3.01020057048578\\
0.590590590590591	3.01941731950326\\
0.591591591591592	3.02849346188275\\
0.592592592592593	3.03742738153303\\
0.593593593593594	3.0462174751447\\
0.594594594594595	3.05486215256065\\
0.595595595595596	3.06335983714625\\
0.596596596596597	3.07170896615931\\
0.597597597597598	3.07990799111957\\
0.598598598598599	3.08795537817776\\
0.5995995995996	3.09584960848405\\
0.600600600600601	3.10358917855585\\
0.601601601601602	3.11117260064493\\
0.602602602602603	3.11859840310356\\
0.603603603603604	3.12586513074989\\
0.604604604604605	3.13297134523217\\
0.605605605605606	3.13991562539201\\
0.606606606606607	3.14669656762629\\
0.607607607607608	3.15331278624794\\
0.608608608608609	3.1597629138452\\
0.60960960960961	3.16604560163952\\
0.610610610610611	3.17215951984183\\
0.611611611611612	3.1781033580072\\
0.612612612612613	3.1838758253877\\
0.613613613613614	3.18947565128348\\
0.614614614614615	3.19490158539185\\
0.615615615615616	3.20015239815439\\
0.616616616616617	3.20522688110188\\
0.617617617617618	3.21012384719705\\
0.618618618618619	3.21484213117497\\
0.61961961961962	3.21938058988111\\
0.620620620620621	3.2237381026068\\
0.621621621621622	3.22791357142211\\
0.622622622622623	3.2319059215061\\
0.623623623623624	3.23571410147417\\
0.624624624624625	3.23933708370258\\
0.625625625625626	3.24277386465004\\
0.626626626626627	3.24602346517598\\
0.627627627627628	3.24908493085599\\
0.628628628628629	3.25195733229367\\
0.62962962962963	3.25463976542925\\
0.630630630630631	3.25713135184472\\
0.631631631631632	3.25943123906539\\
0.632632632632633	3.26153860085775\\
0.633633633633634	3.2634526375236\\
0.634634634634635	3.26517257619032\\
0.635635635635636	3.26669767109711\\
0.636636636636637	3.26802720387734\\
0.637637637637638	3.2691604838365\\
0.638638638638639	3.27009684822621\\
0.63963963963964	3.27083566251355\\
0.640640640640641	3.27137632064628\\
0.641641641641642	3.27171824531322\\
0.642642642642643	3.2718608882003\\
0.643643643643644	3.27180373024158\\
0.644644644644645	3.27154628186567\\
0.645645645645646	3.27108808323723\\
0.646646646646647	3.27042870449327\\
0.647647647647648	3.26956774597453\\
0.648648648648649	3.26850483845166\\
0.64964964964965	3.26723964334597\\
0.650650650650651	3.26577185294495\\
0.651651651651652	3.26410119061226\\
0.652652652652653	3.26222741099208\\
0.653653653653654	3.26015030020801\\
0.654654654654655	3.25786967605596\\
0.655655655655656	3.25538538819145\\
0.656656656656657	3.25269731831084\\
0.657657657657658	3.24980538032656\\
0.658658658658659	3.24670952053633\\
0.65965965965966	3.24340971778614\\
0.660660660660661	3.2399059836269\\
0.661661661661662	3.23619836246485\\
0.662662662662663	3.2322869317055\\
0.663663663663664	3.22817180189096\\
0.664664664664665	3.22385311683078\\
0.665665665665666	3.21933105372606\\
0.666666666666667	3.21460582328674\\
0.667667667667668	3.20967766984218\\
0.668668668668669	3.20454687144466\\
0.66966966966967	3.19921373996595\\
0.670670670670671	3.19367862118686\\
0.671671671671672	3.18794189487954\\
0.672672672672673	3.18200397488256\\
0.673673673673674	3.1758653091688\\
0.674674674674675	3.1695263799058\\
0.675675675675676	3.16298770350882\\
0.676676676676677	3.1562498306863\\
0.677677677677678	3.14931334647774\\
0.678678678678679	3.142178870284\\
0.67967967967968	3.13484705588983\\
0.680680680680681	3.12731859147859\\
0.681681681681682	3.11959419963926\\
0.682682682682683	3.11167463736541\\
0.683683683683684	3.10356069604631\\
0.684684684684685	3.09525320144994\\
0.685685685685686	3.08675301369799\\
0.686686686686687	3.07806102723274\\
0.687687687687688	3.06917817077569\\
0.688688688688689	3.060105407278\\
0.68968968968969	3.05084373386268\\
0.690690690690691	3.04139418175833\\
0.691691691691692	3.03175781622466\\
0.692692692692693	3.02193573646949\\
0.693693693693694	3.01192907555726\\
0.694694694694695	3.0017390003091\\
0.695695695695696	2.99136671119434\\
0.696696696696697	2.98081344221339\\
0.697697697697698	2.97008046077204\\
0.698698698698699	2.95916906754716\\
0.6996996996997	2.9480805963435\\
0.700700700700701	2.9368164139421\\
0.701701701701702	2.92537791993967\\
0.702702702702703	2.91376654657937\\
0.703703703703704	2.90198375857271\\
0.704704704704705	2.8900310529127\\
0.705705705705706	2.87790995867815\\
0.706706706706707	2.86562203682907\\
0.707707707707708	2.85316887999331\\
0.708708708708709	2.84055211224422\\
0.70970970970971	2.82777338886947\\
0.710710710710711	2.81483439613102\\
0.711711711711712	2.80173685101611\\
0.712712712712713	2.78848250097943\\
0.713713713713714	2.77507312367628\\
0.714714714714715	2.76151052668694\\
0.715715715715716	2.74779654723213\\
0.716716716716717	2.73393305187951\\
0.717717717717718	2.71992193624132\\
0.718718718718719	2.70576512466324\\
0.71971971971972	2.69146456990431\\
0.720720720720721	2.67702225280797\\
0.721721721721722	2.66244018196446\\
0.722722722722723	2.64772039336424\\
0.723723723723724	2.63286495004272\\
0.724724724724725	2.61787594171628\\
0.725725725725726	2.6027554844095\\
0.726726726726727	2.58750572007381\\
0.727727727727728	2.57212881619738\\
0.728728728728729	2.55662696540651\\
0.72972972972973	2.54100238505841\\
0.730730730730731	2.5252573168255\\
0.731731731731732	2.50939402627113\\
0.732732732732733	2.49341480241708\\
0.733733733733734	2.47732195730248\\
0.734734734734735	2.46111782553456\\
0.735735735735736	2.4448047638311\\
0.736736736736737	2.42838515055471\\
0.737737737737738	2.41186138523895\\
0.738738738738739	2.39523588810643\\
0.73973973973974	2.37851109957891\\
0.740740740740741	2.3616894797795\\
0.741741741741742	2.344773508027\\
0.742742742742743	2.32776568232256\\
0.743743743743744	2.31066851882854\\
0.744744744744745	2.29348455133991\\
0.745745745745746	2.2762163307481\\
0.746746746746747	2.25886642449751\\
0.747747747747748	2.24143741603456\\
0.748748748748749	2.22393190424975\\
0.74974974974975	2.20635250291246\\
0.750750750750751	2.18870184009879\\
0.751751751751752	2.17098255761257\\
0.752752752752753	2.15319731039952\\
0.753753753753754	2.13534876595478\\
0.754754754754755	2.11743960372391\\
0.755755755755756	2.09947251449745\\
0.756756756756757	2.0814501997992\\
0.757757757757758	2.06337537126831\\
0.758758758758759	2.04525075003539\\
0.75975975975976	2.02707906609264\\
0.760760760760761	2.00886305765827\\
0.761761761761762	1.99060547053531\\
0.762762762762763	1.97230905746493\\
0.763763763763764	1.95397657747438\\
0.764764764764765	1.93561079521986\\
0.765765765765766	1.91721448032429\\
0.766766766766767	1.89879040671021\\
0.767767767767768	1.88034135192804\\
0.768768768768769	1.8618700964798\\
0.76976976976977	1.84337942313833\\
0.770770770770771	1.82487211626251\\
0.771771771771772	1.8063509611083\\
0.772772772772773	1.78781874313602\\
0.773773773773774	1.76927824731392\\
0.774774774774775	1.75073225741827\\
0.775775775775776	1.73218355533021\\
0.776776776776777	1.71363492032938\\
0.777777777777778	1.69508912838481\\
0.778778778778779	1.67654895144291\\
0.77977977977978	1.65801715671309\\
0.780780780780781	1.639496505951\\
0.781781781781782	1.62098975473969\\
0.782782782782783	1.60249965176883\\
0.783783783783784	1.58402893811235\\
0.784784784784785	1.56558034650448\\
0.785785785785786	1.54715660061464\\
0.786786786786787	1.52876041432131\\
0.787787787787788	1.51039449098497\\
0.788788788788789	1.49206152272062\\
0.78978978978979	1.47376418966991\\
0.790790790790791	1.45550515927304\\
0.791791791791792	1.43728708554102\\
0.792792792792793	1.41911260832803\\
0.793793793793794	1.4009843526046\\
0.794794794794795	1.38290492773149\\
0.795795795795796	1.3648769267348\\
0.796796796796797	1.34690292558228\\
0.797797797797798	1.32898548246136\\
0.798798798798799	1.31112713705899\\
0.7997997997998	1.29333040984356\\
0.800800800800801	1.27559780134925\\
0.801801801801802	1.25793179146296\\
0.802802802802803	1.24033483871419\\
0.803803803803804	1.22280937956803\\
0.804804804804805	1.20535782772165\\
0.805805805805806	1.18798257340438\\
0.806806806806807	1.17068598268197\\
0.807807807807808	1.15347039676485\\
0.808808808808809	1.13633813132117\\
0.80980980980981	1.11929147579444\\
0.810810810810811	1.10233269272645\\
0.811811811811812	1.0854640170854\\
0.812812812812813	1.06868765559983\\
0.813813813813814	1.0520057860984\\
0.814814814814815	1.03542055685597\\
0.815815815815816	1.01893408594617\\
0.816816816816817	1.00254846060081\\
0.817817817817818	0.986265736576375\\
0.818818818818819	0.970087937527947\\
0.81981981981982	0.954017054390823\\
0.820820820820821	0.938055044770114\\
0.821821821821822	0.922203832338652\\
0.822822822822823	0.906465306243468\\
0.823823823823824	0.890841320521173\\
0.824824824824825	0.875333693522523\\
0.825825825825826	0.859944207346461\\
0.826826826826827	0.844674607283966\\
0.827827827827828	0.829526601271956\\
0.828828828828829	0.814501859357625\\
0.82982982982983	0.799602013173432\\
0.830830830830831	0.784828655423082\\
0.831831831831832	0.77018333937884\\
0.832832832832833	0.755667578390382\\
0.833833833833834	0.741282845405572\\
0.834834834834835	0.727030572503424\\
0.835835835835836	0.712912150439551\\
0.836836836836837	0.698928928204419\\
0.837837837837838	0.685082212594661\\
0.838838838838839	0.671373267797804\\
0.83983983983984	0.657803314990662\\
0.840840840840841	0.644373531951705\\
0.841841841841842	0.631085052687701\\
0.842842842842843	0.6179389670749\\
0.843843843843844	0.604936320515088\\
0.844844844844845	0.592078113606758\\
0.845845845845846	0.579365301831708\\
0.846846846846847	0.566798795257351\\
0.847847847847848	0.554379458254993\\
0.848848848848849	0.542108109234382\\
0.84984984984985	0.529985520394804\\
0.850850850850851	0.518012417492974\\
0.851851851851852	0.506189479628012\\
0.852852852852853	0.494517339043779\\
0.853853853853854	0.482996580948796\\
0.854854854854855	0.471627743354055\\
0.855855855855856	0.460411316928948\\
0.856856856856857	0.449347744875549\\
0.857857857857858	0.438437422821549\\
0.858858858858859	0.427680698732033\\
0.85985985985986	0.417077872840364\\
0.860860860860861	0.406629197598401\\
0.861861861861862	0.396334877646278\\
0.862862862862863	0.386195069801965\\
0.863863863863864	0.376209883070828\\
0.864864864864865	0.366379378675405\\
0.865865865865866	0.356703570105599\\
0.866866866866867	0.347182423189472\\
0.867867867867868	0.337815856184862\\
0.868868868868869	0.328603739891983\\
0.86986986986987	0.319545897787201\\
0.870870870870871	0.310642106178142\\
0.871871871871872	0.301892094380316\\
0.872872872872873	0.293295544915395\\
0.873873873873874	0.284852093731298\\
0.874874874874875	0.276561330444229\\
0.875875875875876	0.268422798602797\\
0.876876876876877	0.260435995974317\\
0.877877877877878	0.25260037485345\\
0.878878878878879	0.244915342393242\\
0.87987987987988	0.237380260958677\\
0.880880880880881	0.22999444850283\\
0.881881881881882	0.222757178965672\\
0.882882882882883	0.215667682695619\\
0.883883883883884	0.208725146893853\\
0.884884884884885	0.201928716081462\\
0.885885885885886	0.195277492589438\\
0.886886886886887	0.188770537071527\\
0.887887887887888	0.182406869039961\\
0.888888888888889	0.176185467424038\\
0.88988988988989	0.170105271151556\\
0.890890890890891	0.164165179753044\\
0.891891891891892	0.158364053988743\\
0.892892892892893	0.152700716498298\\
0.893893893893894	0.147173952473047\\
0.894894894894895	0.14178251035085\\
0.895895895895896	0.136525102533332\\
0.896896896896897	0.131400406125415\\
0.897897897897898	0.126407063697021\\
0.898898898898899	0.121543684066764\\
0.8998998998999	0.11680884310748\\
0.900900900900901	0.112201084573405\\
0.901901901901902	0.107718920948771\\
0.902902902902903	0.103360834317639\\
0.903903903903904	0.0991252772546782\\
0.904904904904905	0.0950106737366732\\
0.905905905905906	0.091015420074445\\
0.906906906906907	0.0871378858649041\\
0.907907907907908	0.0833764149629124\\
0.908908908908909	0.079729326472607\\
0.90990990990991	0.0761949157578313\\
0.910910910910911	0.0727714554712814\\
0.911911911911912	0.0694571966019693\\
0.912912912912913	0.0662503695405715\\
0.913913913913914	0.0631491851622103\\
0.914914914914915	0.060151835926198\\
0.915915915915916	0.0572564969922371\\
0.916916916916917	0.0544613273525631\\
0.917917917917918	0.0517644709794738\\
0.918918918918919	0.049164057987676\\
0.91991991991992	0.0466582058108472\\
0.920920920920921	0.0442450203917876\\
0.921921921921922	0.0419225973855087\\
0.922922922922923	0.0396890233745755\\
0.923923923923924	0.0375423770959969\\
0.924924924924925	0.0354807306789216\\
0.925925925925926	0.0335021508923743\\
0.926926926926927	0.0316047004022357\\
0.927927927927928	0.0297864390366354\\
0.928928928928929	0.0280454250589018\\
0.92992992992993	0.0263797164471758\\
0.930930930930931	0.024787372179765\\
0.931931931931932	0.0232664535252853\\
0.932932932932933	0.0218150253365958\\
0.933933933933934	0.0204311573475069\\
0.934934934934935	0.0191129254712011\\
0.935935935935936	0.0178584130992712\\
0.936936936936937	0.0166657124002484\\
0.937937937937938	0.0155329256164506\\
0.938938938938939	0.014458166357947\\
0.93993993993994	0.0134395608923971\\
0.940940940940941	0.0124752494294802\\
0.941941941941942	0.0115633873985963\\
0.942942942942943	0.0107021467184766\\
0.943943943943944	0.00988971705729903\\
0.944944944944945	0.00912430708186812\\
0.945945945945946	0.00840414569436889\\
0.946946946946947	0.00772748325516815\\
0.947947947947948	0.00709259279008752\\
0.948948948948949	0.00649777118053068\\
0.94994994994995	0.00594134033480118\\
0.950950950950951	0.00542164833889969\\
0.951951951951952	0.0049370705850454\\
0.952952952952953	0.00448601087611532\\
0.953953953953954	0.00406690250414878\\
0.954954954954955	0.00367820930101429\\
0.955955955955956	0.00331842665928507\\
0.956956956956957	0.00298608252132013\\
0.957957957957958	0.00267973833449402\\
0.958958958958959	0.00239798997046683\\
0.95995995995996	0.00213946860633195\\
0.960960960960961	0.00190284156542448\\
0.961961961961962	0.00168681311551822\\
0.962962962962963	0.00149012522208261\\
0.963963963963964	0.00131155825421369\\
0.964964964964965	0.00114993164079538\\
0.965965965965966	0.00100410447438821\\
0.966966966966967	0.000872976060282659\\
0.967967967967968	0.000755486408093562\\
0.968968968968969	0.000650616663209847\\
0.96996996996997	0.000557389475351558\\
0.970970970970971	0.000474869301421807\\
0.971971971971972	0.000402162639777063\\
0.972972972972973	0.000338418192972995\\
0.973973973973974	0.000282826955976615\\
0.974974974974975	0.000234622226767436\\
0.975975975975976	0.000193079536181843\\
0.976976976976977	0.000157516493784743\\
0.977977977977978	0.0001272925464818\\
0.978978978978979	0.000101808646513591\\
0.97997997997998	8.05068253998833e-05\\
0.980980980980981	6.28696703282317e-05\\
0.981981981981982	4.84196994057522e-05\\
0.982982982982983	3.67186321166482e-05\\
0.983983983983984	2.73665512505938e-05\\
0.984984984984985	2.00009524884649e-05\\
0.985985985985986	1.42956777522046e-05\\
0.986986986986987	9.95972834467292e-06\\
0.987987987987988	6.73595382328313e-06\\
0.988988988988989	4.39961246797251e-06\\
0.98998998998999	2.7567991196023e-06\\
0.990990990990991	1.64273607923924e-06\\
0.991991991991992	9.19922671886428e-07\\
0.992992992992993	4.76138990131968e-07\\
0.993993993993994	2.22299243830205e-07\\
0.994994994994995	9.01500513200999e-08\\
0.995995995995996	2.98089158055984e-08\\
0.996996996996997	7.13803735595041e-09\\
0.997997997997998	9.48516521977438e-10\\
0.998998998998999	2.99097903312574e-11\\
1	0\\
};
\addlegendentry{posterior};

\end{axis}
\end{tikzpicture}%
  \caption{An example of Bayesian updating for coin flipping.  Figure
    produced by \texttt{plot\_beta\_example.m}.}
  \label{coin_flipping}
\end{figure}

\section*{Hypothesis testing}

We often wish to use our observed data to draw conclusions about the
plausibility of various hypotheses.  For example, we might wish to
know whether the parameter $\theta$ is less than $\nicefrac{1}{2}$.
The Bayesian method allows us to compute this value directly from the
posterior distribution:
\begin{equation*}
  \Pr(\theta < \nicefrac{1}{2} \given x, n, \alpha, \beta)
  =
  \int_{0}^{\nicefrac{1}{2}} p(\theta \given x, n, \alpha, \beta) \intd{\theta}.
\end{equation*}
For the example in Figure \ref{coin_flipping}, this probability is
approximately 15\%.

There is a sharp contrast between the simplicity of this approach and
the frequentist method.  The classical approach to hypothesis testing
uses the likelihood as a way of generating fake datasets of the same
size as the observations.  The likelihood then serves as a so-called
``null hypothesis'' that allows us to generate hypothetical datasets
under some condition.

From these, we compute \emph{statistics,} which, like estimators, can
be any function of the hypothesized data.  We then identify some
critical set $C$ for this statistic which contains some large portion
$(1 - \alpha)$ of the values corresponding to the datasets generated
by our null hypothesis.  If the statistic computed from the observed
data falls outside this set, we reject the null hypothesis with
``confidence'' $\alpha$.  Note that the ``rejection'' of the null
hypothesis in classical hypothesis testing is purely a statement about
the observed data (that it looks ``unusual''), and not about the
plausibility of alternative hypotheses!

\end{document}
