\documentclass{article}

\usepackage[T1]{fontenc}
\usepackage[osf]{libertine}
\usepackage[scaled=0.8]{beramono}
\usepackage[margin=1.5in]{geometry}
\usepackage{url}
\usepackage{booktabs}
\usepackage{amsmath}
\usepackage{amssymb}
\usepackage{nicefrac}
\usepackage{microtype}
\usepackage{bm}

\usepackage{sectsty}
\sectionfont{\large}
\subsectionfont{\normalsize}

\usepackage{titlesec}
\titlespacing{\section}{0pt}{10pt plus 2pt minus 2pt}{0pt plus 2pt minus 0pt}
\titlespacing{\subsection}{0pt}{5pt plus 2pt minus 2pt}{0pt plus 2pt minus 0pt}

\setlength{\parindent}{0pt}
\setlength{\parskip}{1ex}

\newcommand{\acro}[1]{\textsc{\MakeLowercase{#1}}}
\newcommand{\given}{\mid}
\newcommand{\mc}[1]{\mathcal{#1}}
\newcommand{\data}{\mc{D}}
\newcommand{\intd}[1]{\,\mathrm{d}{#1}}
\newcommand{\mat}[1]{\bm{\mathrm{#1}}}
\renewcommand{\vec}[1]{\bm{\mathrm{#1}}}
\newcommand{\trans}{^\top}
\newcommand{\inv}{^{-1}}

\DeclareMathOperator{\var}{var}
\DeclareMathOperator{\tr}{tr}

\usepackage{pgfplots}
\pgfplotsset{
  compat=newest,
  plot coordinates/math parser=false,
  tick label style={font=\footnotesize, /pgf/number format/fixed},
  label style={font=\small},
  legend style={font=\small},
  every axis/.append style={
    tick align=outside,
    clip mode=individual,
    scaled ticks=false,
    thick,
    tick style={semithick, black}
  }
}

\pgfkeys{/pgf/number format/.cd, set thousands separator={\,}}

\usepgfplotslibrary{external}
\tikzexternalize[prefix=tikz/]

\newlength\figurewidth
\newlength\figureheight

\setlength{\figurewidth}{8cm}
\setlength{\figureheight}{6cm}

\begin{document}

{\large \textbf{CSE 515T (Spring 2015) Assignment 2 solutions}} \\

\begin{enumerate}

\item
  (Curse of dimensionality.)
  Consider a $d$-dimensional, zero-mean, spherical multivariate
  Gaussian distribution:
  \begin{equation*}
    p(\vec{x}) = \mc{N}(\vec{x}; \vec{0}, \mat{I}_d).
  \end{equation*}
  Equivalently, each entry of $\vec{x}$ is drawn iid from a univariate
  standard normal distribution.

  In familiar small dimensions ($d \leq 3$), ``most'' of the vectors
  drawn from a multivariate Gaussian distribution will lie near the
  mean.  For example, the famous 68--95--99.7 rule for $d = 1$
  indicates that large deviations from the mean are unusual.  Here we
  will consider the behavior in larger dimensions.
  \begin{itemize}
  \item Draw 10\,000 samples from $p(\vec{x})$ for each dimension in
    $d \in \{1, 5, 10, 50, 100\},$ and compute the length of each
    vector drawn: $y_d = \sqrt{\vec{x}\trans \vec{x}} = (\sum_i^d
    x_i^2)^{\nicefrac{1}{2}}$.  Estimate the distribution of each
    $y_d$ using either a histogram or a kernel density estimate (in
    \acro{MATLAB}, \texttt{hist} and \texttt{ksdensity},
    respectively).  Plot your estimates.  (Please do not hand in your
    raw samples!)  Summarize the behavior of this distribution as $d$
    increases.
  \item
    The true distribution of $y_d^2$ is a chi-square distribution with
    $d$ degrees of freedom (the distribution of $y_d$ itself is the
    less-commonly seen chi distribution).  Use this fact to compute
    the probability that $y_d < 5$ for each of the dimensions in the
    last part.
  \item
    For $d = 1\,000$, compute the 5th and 95th percentiles of $y_d$.
    Is the mean $\vec{x} = \vec{0}$ a representative summary of the
    distribution in high dimensions?  This behavior has been called
    ``the curse of dimensionality.''
  \end{itemize}

\end{enumerate}

\subsection*{Solution}

Kernel density estimates of the empirical distributions of $y$ (using
10\,000 samples each) are shown in Figure \ref{problem_1}. As the
dimension increases, we see that the bulk of the probability mass
actually lives in a thin ``shell'' centered around the origin: all
samples are approximately the same length, with no vectors near the
mean.  This is somewhat unintuitive.

The compute the probability that $y_d < 5$, we can evaluate the
corresponding $\chi^2$ \acro{CDF} at $y_d^2 = 25$:\footnote{Computed
  with \texttt{chi2cdf(25, [1, 5, 10, 50, 100])} in \acro{MATLAB}.}
\begin{align*}
  \Pr(y < 5 \given d = \phantom{00}1) &\approx 1.0000 \\
  \Pr(y < 5 \given d = \phantom{00}5) &\approx 0.9999 \\
  \Pr(y < 5 \given d = \phantom{0}10) &\approx 0.9947 \\
  \Pr(y < 5 \given d = \phantom{0}50) &\approx 0.0012 \\
  \Pr(y < 5 \given d = 100) &\approx 0.0000,
\end{align*}
so the probability of being within a distance of five standard
deviations form he mean decreases from near certainty to near
impossibility, another surprising result.

For the last part, we invert the $\chi^2$ \acro{CDF} and take the
square root.\footnote{Computed with \texttt{sqrt(chi2inv([0.05, 0.95],
    1000))} in \acro{MATLAB}.}  The 5th percentile is 30.46 and the
95th percentile is 32.78.  Again, we see that most of the mass lies in
a narrow shell centered around the mean.

Whether the mean is a representative summary is a much more
complicated question with no definitive answer.  In some sense, it's a
very odd summary: in dimensions higher than 10 or so, we can't imagine
seeing a vector anywhere near the mean!  On the other hand, if we want
to choose another single point to summarize the distribution, there's
no clear better alternative.  By definition, the mean minimizes the
average squared distance from the chosen point to the vector
$\vec{x}$.  It just so happens that the \emph{minimum} squared
distance to the mean is relatively high in large dimension.  This is
the ``curse of dimensionality:'' all points are ``far away from the
mean'' (and also each other!).

\begin{figure}
  \centering
  % This file was created by matlab2tikz.
% Minimal pgfplots version: 1.3
%
\tikzsetnextfilename{problem_1}
\definecolor{mycolor1}{rgb}{0.65098,0.80784,0.89020}%
\definecolor{mycolor2}{rgb}{0.20000,0.62745,0.17255}%
\definecolor{mycolor3}{rgb}{0.07843,0.07843,0.07843}%
\definecolor{mycolor4}{rgb}{0.12157,0.47059,0.70588}%
\definecolor{mycolor5}{rgb}{0.69804,0.87451,0.54118}%
%
\begin{tikzpicture}

\begin{axis}[%
width=0.95092\figurewidth,
height=\figureheight,
at={(0\figurewidth,0\figureheight)},
scale only axis,
xmin=0,
xmax=15,
xlabel={$y$},
ymin=0,
ymax=0.8,
ylabel={$p(y \given d)$},
axis x line*=bottom,
axis y line*=left,
legend style={legend cell align=left,align=left,draw=white!15!black},
legend style={draw=none}
]
\addplot [color=mycolor1,solid]
  table[row sep=crcr]{%
0	0.39331849048811\\
0.015015015015015	0.442139924872031\\
0.03003003003003	0.489865266852357\\
0.045045045045045	0.535407411584238\\
0.0600600600600601	0.577831198021182\\
0.0750750750750751	0.616400845122858\\
0.0900900900900901	0.650626588420932\\
0.105105105105105	0.680261737465101\\
0.12012012012012	0.705303524636115\\
0.135135135135135	0.725945616690583\\
0.15015015015015	0.742543141406836\\
0.165165165165165	0.755553513592699\\
0.18018018018018	0.765477388533125\\
0.195195195195195	0.772822071043983\\
0.21021021021021	0.778058599168201\\
0.225225225225225	0.78159819980899\\
0.24024024024024	0.783771239165282\\
0.255255255255255	0.78482867685734\\
0.27027027027027	0.784951399761494\\
0.285285285285285	0.784260763916417\\
0.3003003003003	0.782833030702803\\
0.315315315315315	0.78072486705031\\
0.33033033033033	0.777975656418451\\
0.345345345345345	0.77463518677239\\
0.36036036036036	0.770754807284952\\
0.375375375375375	0.766406765198463\\
0.39039039039039	0.761665543038423\\
0.405405405405405	0.756611991424914\\
0.42042042042042	0.751334735200599\\
0.435435435435435	0.745901949416561\\
0.45045045045045	0.740360177795554\\
0.465465465465465	0.734732714500846\\
0.48048048048048	0.729022129311828\\
0.495495495495495	0.723211302895336\\
0.510510510510511	0.717248184233479\\
0.525525525525526	0.711087476271227\\
0.540540540540541	0.704669020003307\\
0.555555555555556	0.697944970451687\\
0.570570570570571	0.690877993346359\\
0.585585585585586	0.6834469816983\\
0.600600600600601	0.675659768312945\\
0.615615615615616	0.667552713702777\\
0.630630630630631	0.659180231074469\\
0.645645645645646	0.650622888703933\\
0.660660660660661	0.64197966559825\\
0.675675675675676	0.633342702129373\\
0.690690690690691	0.624819899993591\\
0.705705705705706	0.616498570945249\\
0.720720720720721	0.608446893241786\\
0.735735735735736	0.600711759183954\\
0.750750750750751	0.593309341989367\\
0.765765765765766	0.586227011145132\\
0.780780780780781	0.579419338410037\\
0.795795795795796	0.572824999962985\\
0.810810810810811	0.566360688655458\\
0.825825825825826	0.559949549981113\\
0.840840840840841	0.553515529291616\\
0.855855855855856	0.546994948056384\\
0.870870870870871	0.540336933810352\\
0.885885885885886	0.533524102461939\\
0.900900900900901	0.526547037613031\\
0.915915915915916	0.519423517090204\\
0.930930930930931	0.512178425865227\\
0.945945945945946	0.504844445597397\\
0.960960960960961	0.497461480173211\\
0.975975975975976	0.490054746796354\\
0.990990990990991	0.482645873450366\\
1.00600600600601	0.475255499385808\\
1.02102102102102	0.467889978728206\\
1.03603603603604	0.460551700941375\\
1.05105105105105	0.453236845510686\\
1.06606606606607	0.445942709263427\\
1.08108108108108	0.438658546612484\\
1.0960960960961	0.431379096826897\\
1.11111111111111	0.424092982629543\\
1.12612612612613	0.416788007466407\\
1.14114114114114	0.409456225818678\\
1.15615615615616	0.402076156551649\\
1.17117117117117	0.394642275341676\\
1.18618618618619	0.387143748430231\\
1.2012012012012	0.379576696500856\\
1.21621621621622	0.371940836550384\\
1.23123123123123	0.364252401693204\\
1.24624624624625	0.356527200193968\\
1.26126126126126	0.348800985894459\\
1.27627627627628	0.341112433707433\\
1.29129129129129	0.333498932960206\\
1.30630630630631	0.32600921498125\\
1.32132132132132	0.318687140862426\\
1.33633633633634	0.311575315363633\\
1.35135135135135	0.304710993601901\\
1.36636636636637	0.298117838151438\\
1.38138138138138	0.291816494562928\\
1.3963963963964	0.285819392200494\\
1.41141141141141	0.280125175852875\\
1.42642642642643	0.274726422517079\\
1.44144144144144	0.269608096371822\\
1.45645645645646	0.264741298113756\\
1.47147147147147	0.260090118781881\\
1.48648648648649	0.2556144300126\\
1.5015015015015	0.251265691082419\\
1.51651651651652	0.246993997024692\\
1.53153153153153	0.242753196635653\\
1.54654654654655	0.238493890482589\\
1.56156156156156	0.234182432010806\\
1.57657657657658	0.229798124922777\\
1.59159159159159	0.225322461606179\\
1.60660660660661	0.220759020648082\\
1.62162162162162	0.216121450005774\\
1.63663663663664	0.211432309160033\\
1.65165165165165	0.206718700034329\\
1.66666666666667	0.202008357717453\\
1.68168168168168	0.197325956627634\\
1.6966966966967	0.192689756778878\\
1.71171171171171	0.18811330829021\\
1.72672672672673	0.183600088424388\\
1.74174174174174	0.17914391198068\\
1.75675675675676	0.1747421549407\\
1.77177177177177	0.170381120074064\\
1.78678678678679	0.166055936253649\\
1.8018018018018	0.161759743057737\\
1.81681681681682	0.157486757720537\\
1.83183183183183	0.153238249396644\\
1.84684684684685	0.149014068342566\\
1.86186186186186	0.144813967309616\\
1.87687687687688	0.140640438502671\\
1.89189189189189	0.13649045327367\\
1.90690690690691	0.132362767227169\\
1.92192192192192	0.128258051020939\\
1.93693693693694	0.124180825258712\\
1.95195195195195	0.120133936259073\\
1.96696696696697	0.116131378326269\\
1.98198198198198	0.112187754763062\\
1.996996996997	0.108326411450157\\
2.01201201201201	0.104572703097358\\
2.02702702702703	0.100951523843351\\
2.04204204204204	0.0974929875764708\\
2.05705705705706	0.0942185202122292\\
2.07207207207207	0.0911410589200802\\
2.08708708708709	0.0882691370993639\\
2.1021021021021	0.0855988297186902\\
2.11711711711712	0.0831181201549093\\
2.13213213213213	0.0808076981158156\\
2.14714714714715	0.0786389098791937\\
2.16216216216216	0.0765813354720695\\
2.17717717717718	0.0746009003028845\\
2.19219219219219	0.0726691801692081\\
2.20720720720721	0.0707601978922442\\
2.22222222222222	0.0688535792485291\\
2.23723723723724	0.0669395788885889\\
2.25225225225225	0.0650086892663629\\
2.26726726726727	0.0630635483460715\\
2.28228228228228	0.061108940187004\\
2.2972972972973	0.0591500515699764\\
2.31231231231231	0.0571980109800666\\
2.32732732732733	0.0552580751831493\\
2.34234234234234	0.0533333627548002\\
2.35735735735736	0.0514275955146683\\
2.37237237237237	0.0495386230507448\\
2.38738738738739	0.0476623832317251\\
2.4024024024024	0.0457928610387335\\
2.41741741741742	0.0439236816054789\\
2.43243243243243	0.0420532719130918\\
2.44744744744745	0.040179193349971\\
2.46246246246246	0.0383061696158116\\
2.47747747747748	0.0364419733765776\\
2.49249249249249	0.0346016585641481\\
2.50750750750751	0.0328039944250264\\
2.52252252252252	0.0310692274810652\\
2.53753753753754	0.02942059084521\\
2.55255255255255	0.0278773167505643\\
2.56756756756757	0.0264551323826573\\
2.58258258258258	0.0251652326281853\\
2.5975975975976	0.0240111544162037\\
2.61261261261261	0.0229868890839685\\
2.62762762762763	0.0220821505802056\\
2.64264264264264	0.0212794092372411\\
2.65765765765766	0.020558714748472\\
2.67267267267267	0.0198967294177705\\
2.68768768768769	0.0192719670049864\\
2.7027027027027	0.0186648155045379\\
2.71771771771772	0.0180615334675199\\
2.73273273273273	0.017452377928143\\
2.74774774774775	0.0168319608687239\\
2.76276276276276	0.016201227255078\\
2.77777777777778	0.0155669043924579\\
2.79279279279279	0.0149359235457499\\
2.80780780780781	0.0143194580961327\\
2.82282282282282	0.0137270080823048\\
2.83783783783784	0.0131689158622329\\
2.85285285285285	0.0126519191226309\\
2.86786786786787	0.0121799915409022\\
2.88288288288288	0.0117530856840747\\
2.8978978978979	0.0113678039764087\\
2.91291291291291	0.0110188015562928\\
2.92792792792793	0.0106984915425175\\
2.94294294294294	0.0103982747494705\\
2.95795795795796	0.0101081559264508\\
2.97297297297297	0.00982054891122765\\
2.98798798798799	0.00952814340227287\\
3.003003003003	0.00922485235628208\\
3.01801801801802	0.00890641240041586\\
3.03303303303303	0.00857083907006301\\
3.04804804804805	0.00821701404039693\\
3.06306306306306	0.00784557451584293\\
3.07807807807808	0.00745839140184556\\
3.09309309309309	0.00705996922194125\\
3.10810810810811	0.00665500886756732\\
3.12312312312312	0.00624848463157592\\
3.13813813813814	0.00584762182743768\\
3.15315315315315	0.00545906650070711\\
3.16816816816817	0.0050885664452531\\
3.18318318318318	0.00474148554661392\\
3.1981981981982	0.00442304747746498\\
3.21321321321321	0.00413581397010822\\
3.22822822822823	0.00388163902396993\\
3.24324324324324	0.0036608590540516\\
3.25825825825826	0.0034726092360237\\
3.27327327327327	0.00331561695380179\\
3.28828828828829	0.0031872771063355\\
3.3033033033033	0.00308521173765427\\
3.31831831831832	0.00300642243197225\\
3.33333333333333	0.00294821641385668\\
3.34834834834835	0.00290818189535772\\
3.36336336336336	0.00288340448491932\\
3.37837837837838	0.00287199749516396\\
3.39339339339339	0.00287131392277066\\
3.40840840840841	0.00287886115985157\\
3.42342342342342	0.00289141976584691\\
3.43843843843844	0.00290627376439788\\
3.45345345345345	0.00291898581205817\\
3.46846846846847	0.00292495635036548\\
3.48348348348348	0.00292079252147399\\
3.4984984984985	0.00290193393843023\\
3.51351351351351	0.00286514316739426\\
3.52852852852853	0.00280800646586591\\
3.54354354354354	0.0027299982536379\\
3.55855855855856	0.00263169097987509\\
3.57357357357357	0.0025160349069103\\
3.58858858858859	0.00238652449064803\\
3.6036036036036	0.00224774638897388\\
3.61861861861862	0.0021048243768362\\
3.63363363363363	0.00196255529555741\\
3.64864864864865	0.00182449241014937\\
3.66366366366366	0.00169308659770327\\
3.67867867867868	0.00156967067298211\\
3.69369369369369	0.0014538770633547\\
3.70870870870871	0.00134445525401639\\
3.72372372372372	0.00123936044414797\\
3.73873873873874	0.00113709439591553\\
3.75375375375375	0.00103570896073734\\
3.76876876876877	0.000934270388407025\\
3.78378378378378	0.000832587164098025\\
3.7987987987988	0.000731505769767071\\
3.81381381381381	0.000632553864705259\\
3.82882882882883	0.000537365450257662\\
3.84384384384384	0.000447882415548473\\
3.85885885885886	0.000365755679402704\\
3.87387387387387	0.000292604268742107\\
3.88888888888889	0.000228976292154818\\
3.9039039039039	0.00017534134649726\\
3.91891891891892	0.00013115755099041\\
3.93393393393393	9.59865322511699e-05\\
3.94894894894895	6.85525660500404e-05\\
3.96396396396396	4.78000677771938e-05\\
3.97897897897898	3.25613752718919e-05\\
3.99399399399399	2.17219431116181e-05\\
4.00900900900901	1.41484304412086e-05\\
4.02402402402402	8.997680876089e-06\\
4.03903903903904	5.58681985341391e-06\\
4.05405405405405	3.3869555892811e-06\\
4.06906906906907	2.00476764485749e-06\\
4.08408408408408	1.15858317303062e-06\\
4.0990990990991	6.53729234517337e-07\\
4.11411411411411	2.49806586802737e-07\\
4.12912912912913	1.35679193352996e-07\\
4.14414414414414	7.19337291191247e-08\\
4.15915915915916	3.72273268374408e-08\\
4.17417417417417	1.88062014543571e-08\\
4.18918918918919	9.27364081001879e-09\\
4.2042042042042	4.46384814577455e-09\\
4.21921921921922	2.09738669301634e-09\\
4.23423423423423	9.61961208515951e-10\\
4.24924924924925	4.30671928532469e-10\\
4.26426426426426	1.88211226875912e-10\\
4.27927927927928	8.02887027819003e-11\\
4.29429429429429	3.3432843186588e-11\\
4.30930930930931	1.35894580180008e-11\\
4.32432432432432	5.39188866584995e-12\\
4.33933933933934	2.08828443125661e-12\\
4.35435435435435	7.89493102503513e-13\\
4.36936936936937	2.91351316133049e-13\\
4.38438438438438	1.04953175497518e-13\\
4.3993993993994	3.69049088532817e-14\\
4.41441441441441	1.26672588609584e-14\\
4.42942942942943	4.24415375602714e-15\\
4.44444444444444	1.38806411642443e-15\\
4.45945945945946	4.43136884625675e-16\\
4.47447447447447	1.38094451032484e-16\\
4.48948948948949	4.20072750586204e-17\\
4.5045045045045	1.24733396352103e-17\\
4.51951951951952	3.61535493279893e-18\\
4.53453453453453	1.02289036297777e-18\\
4.54954954954955	2.82499178172473e-19\\
4.56456456456456	7.6157949996971e-20\\
4.57957957957958	2.00411785218187e-20\\
4.59459459459459	5.14803182202843e-21\\
4.60960960960961	1.29083027792017e-21\\
4.62462462462462	3.15941750072921e-22\\
4.63963963963964	7.54839902760154e-23\\
4.65465465465465	1.76040517003777e-23\\
4.66966966966967	4.00756315846396e-24\\
4.68468468468468	8.90549512848404e-25\\
4.6996996996997	1.93172686698589e-25\\
4.71471471471471	4.09018851268267e-26\\
4.72972972972973	8.45377887814184e-27\\
4.74474474474474	1.70556548246484e-27\\
4.75975975975976	3.35889066221087e-28\\
4.77477477477477	6.45703723399219e-29\\
4.78978978978979	1.21165971625018e-29\\
4.8048048048048	2.21941237148479e-30\\
4.81981981981982	3.96830725193221e-31\\
4.83483483483483	6.92600073921038e-32\\
4.84984984984985	1.17996664585052e-32\\
4.86486486486486	1.96230668678722e-33\\
4.87987987987988	3.18547345985152e-34\\
4.89489489489489	5.04767114744218e-35\\
4.90990990990991	7.80760925206533e-36\\
4.92492492492492	1.1788404830586e-36\\
4.93993993993994	1.73740860524318e-37\\
4.95495495495495	2.49953285383239e-38\\
4.96996996996997	3.51015008004286e-39\\
4.98498498498498	4.81174359004376e-40\\
5	6.43856704110272e-41\\
5.01501501501502	8.40980493705997e-42\\
5.03003003003003	1.07224141535341e-42\\
5.04504504504505	1.33447114607539e-43\\
5.06006006006006	1.62119685575583e-44\\
5.07507507507508	1.92252622388657e-45\\
5.09009009009009	2.22545464062781e-46\\
5.10510510510511	2.51463627467313e-47\\
5.12012012012012	2.77358551264874e-48\\
5.13513513513514	2.9861931784215e-49\\
5.15015015015015	3.13837035414123e-50\\
5.16516516516517	3.21958904601052e-51\\
5.18018018018018	3.22408619751907e-52\\
5.1951951951952	3.15153984577451e-53\\
5.21021021021021	3.00710723285499e-54\\
5.22522522522523	2.80081860634478e-55\\
5.24024024024024	2.5464256795086e-56\\
5.25525525525526	2.25988834583288e-57\\
5.27027027027027	1.95773053458954e-58\\
5.28528528528529	1.65549853724813e-59\\
5.3003003003003	1.36651575435853e-60\\
5.31531531531532	1.10105861519654e-61\\
5.33033033033033	8.65996607313797e-63\\
5.34534534534535	6.64862589283625e-64\\
5.36036036036036	4.98261759631754e-65\\
5.37537537537538	3.6449634100814e-66\\
5.39039039039039	2.60278768619233e-67\\
5.40540540540541	1.81423803937198e-68\\
5.42042042042042	1.2344109018824e-69\\
5.43543543543544	8.19851476987675e-71\\
5.45045045045045	5.31521177630133e-72\\
5.46546546546547	3.36368963818179e-73\\
5.48048048048048	2.07788360572736e-74\\
5.4954954954955	1.25295787636237e-75\\
5.51051051051051	7.37499413569277e-77\\
5.52552552552553	4.23737444303428e-78\\
5.54054054054054	2.37652266898181e-79\\
5.55555555555556	1.30105912935878e-80\\
5.57057057057057	6.95283722973944e-82\\
5.58558558558559	3.62691222356642e-83\\
5.6006006006006	1.8468090017259e-84\\
5.61561561561562	9.17945453341626e-86\\
5.63063063063063	4.45370781957964e-87\\
5.64564564564565	2.10929111945804e-88\\
5.66066066066066	9.75127222929381e-90\\
5.67567567567568	4.40043887252365e-91\\
5.69069069069069	1.93838776026183e-92\\
5.70570570570571	8.33480186002981e-94\\
5.72072072072072	3.49832260151209e-95\\
5.73573573573574	1.43329104229671e-96\\
5.75075075075075	5.73216749640851e-98\\
5.76576576576577	2.23775900014901e-99\\
5.78078078078078	8.5274210578784e-101\\
5.7957957957958	3.17199181187151e-102\\
5.81081081081081	1.15174507090272e-103\\
5.82582582582583	4.08216562984513e-105\\
5.84084084084084	1.41232561580662e-106\\
5.85585585585586	4.76967738872681e-108\\
5.87087087087087	1.57236413490133e-109\\
5.88588588588589	5.0597284324608e-111\\
5.9009009009009	1.58931960760995e-112\\
5.91591591591592	4.87309894654469e-114\\
5.93093093093093	1.45850919950419e-115\\
5.94594594594595	4.2611131893757e-117\\
5.96096096096096	1.21519765660083e-118\\
5.97597597597598	3.38283434214677e-120\\
5.99099099099099	9.19230642575092e-122\\
6.00600600600601	2.43824987276106e-123\\
6.02102102102102	6.31308856661483e-125\\
6.03603603603604	1.59556870201295e-126\\
6.05105105105105	3.93639879569069e-128\\
6.06606606606607	9.47965706296342e-130\\
6.08108108108108	2.22841522820528e-131\\
6.0960960960961	5.11339760254051e-133\\
6.11111111111111	1.1453361385812e-134\\
6.12612612612613	2.50418447993617e-136\\
6.14114114114114	5.34453136195623e-138\\
6.15615615615616	1.11342993917208e-139\\
6.17117117117117	2.26425906325004e-141\\
6.18618618618619	4.49468529876373e-143\\
6.2012012012012	8.7092833232167e-145\\
6.21621621621622	1.64731063843113e-146\\
6.23123123123123	3.04143501964216e-148\\
6.24624624624625	5.48140051198448e-150\\
6.26126126126126	9.64305146642054e-152\\
6.27627627627628	1.65595064288164e-153\\
6.29129129129129	2.7758132618152e-155\\
6.30630630630631	4.5419574293007e-157\\
6.32132132132132	7.25447230715166e-159\\
6.33633633633634	1.13104152482314e-160\\
6.35135135135135	1.72131846372195e-162\\
6.36636636636637	2.55713619478436e-164\\
6.38138138138138	3.70814276641441e-166\\
6.3963963963964	5.24890833272569e-168\\
6.41141141141141	7.25256258036824e-170\\
6.42642642642643	9.78191616243193e-172\\
6.44144144144144	1.28785314392623e-173\\
6.45645645645646	1.6550789055835e-175\\
6.47147147147147	2.07625656806009e-177\\
6.48648648648649	2.54245508789226e-179\\
6.5015015015015	3.039033695006e-181\\
6.51651651651652	3.54590989447753e-183\\
6.53153153153153	4.03859060358389e-185\\
6.54654654654655	4.48995429604103e-187\\
6.56156156156156	4.8726359474602e-189\\
6.57657657657658	5.16173808832411e-191\\
6.59159159159159	5.33750037339198e-193\\
6.60660660660661	5.38753156160352e-195\\
6.62162162162162	5.30825394472367e-197\\
6.63663663663664	5.10532635970603e-199\\
6.65165165165165	4.79297631735073e-201\\
6.66666666666667	4.39235072941539e-203\\
6.68168168168168	3.92915077248006e-205\\
6.6966966966967	3.43091794922803e-207\\
6.71171171171171	2.9243673371502e-209\\
6.72672672672673	2.43311968228627e-211\\
6.74174174174174	1.976082053781e-213\\
6.75675675675676	1.56659389939213e-215\\
6.77177177177177	1.21232159902627e-217\\
6.78678678678679	9.15775897961603e-220\\
6.8018018018018	6.75259232201901e-222\\
6.81681681681682	4.86028578725201e-224\\
6.83183183183183	3.41478242599358e-226\\
6.84684684684685	2.34193169049041e-228\\
6.86186186186186	1.56781698087725e-230\\
6.87687687687688	1.02453419785425e-232\\
6.89189189189189	6.53532980911034e-235\\
6.90690690690691	4.06928881128323e-237\\
6.92192192192192	2.47331524282702e-239\\
6.93693693693694	1.4674063312243e-241\\
6.95195195195195	8.49828434000948e-244\\
6.96696696696697	4.80421109599022e-246\\
6.98198198198198	2.65108016783333e-248\\
6.996996996997	1.42801771346381e-250\\
7.01201201201201	7.50851898264147e-253\\
7.02702702702703	3.85376232211314e-255\\
7.04204204204204	1.93074769242741e-257\\
7.05705705705706	9.44226266223822e-260\\
7.07207207207207	4.50750881834821e-262\\
7.08708708708709	2.10042436521774e-264\\
7.1021021021021	9.55404901052957e-267\\
7.11711711711712	4.24207006044357e-269\\
7.13213213213213	1.83856150904576e-271\\
7.14714714714715	7.77836727223457e-274\\
7.16216216216216	3.21224514881313e-276\\
7.17717717717718	1.29490787838256e-278\\
7.19219219219219	5.09540864108751e-281\\
7.20720720720721	1.95717256849725e-283\\
7.22222222222222	7.33819377605068e-286\\
7.23723723723724	2.68571043290487e-288\\
7.25225225225225	9.59487125021208e-291\\
7.26726726726727	3.34602415284983e-293\\
7.28228228228228	1.1390136963395e-295\\
7.2972972972973	3.78476375516545e-298\\
7.31231231231231	1.22760475081895e-300\\
7.32732732732733	3.88676514705153e-303\\
7.34234234234234	1.20123504466534e-305\\
7.35735735735736	3.62391199583173e-308\\
7.37237237237237	1.06717894014008e-310\\
7.38738738738739	3.06765704163025e-313\\
7.4024024024024	8.60768386483705e-316\\
7.41741741741742	2.35762691473339e-318\\
7.43243243243243	6.28945567155907e-321\\
7.44744744744745	0\\
7.46246246246246	0\\
7.47747747747748	0\\
7.49249249249249	0\\
7.50750750750751	0\\
7.52252252252252	0\\
7.53753753753754	0\\
7.55255255255255	0\\
7.56756756756757	0\\
7.58258258258258	0\\
7.5975975975976	0\\
7.61261261261261	0\\
7.62762762762763	0\\
7.64264264264264	0\\
7.65765765765766	0\\
7.67267267267267	0\\
7.68768768768769	0\\
7.7027027027027	0\\
7.71771771771772	0\\
7.73273273273273	0\\
7.74774774774775	0\\
7.76276276276276	0\\
7.77777777777778	0\\
7.79279279279279	0\\
7.80780780780781	0\\
7.82282282282282	0\\
7.83783783783784	0\\
7.85285285285285	0\\
7.86786786786787	0\\
7.88288288288288	0\\
7.8978978978979	0\\
7.91291291291291	0\\
7.92792792792793	0\\
7.94294294294294	0\\
7.95795795795796	0\\
7.97297297297297	0\\
7.98798798798799	0\\
8.003003003003	0\\
8.01801801801802	0\\
8.03303303303303	0\\
8.04804804804805	0\\
8.06306306306306	0\\
8.07807807807808	0\\
8.09309309309309	0\\
8.10810810810811	0\\
8.12312312312312	0\\
8.13813813813814	0\\
8.15315315315315	0\\
8.16816816816817	0\\
8.18318318318318	0\\
8.1981981981982	0\\
8.21321321321321	0\\
8.22822822822823	0\\
8.24324324324324	0\\
8.25825825825826	0\\
8.27327327327327	0\\
8.28828828828829	0\\
8.3033033033033	0\\
8.31831831831832	0\\
8.33333333333333	0\\
8.34834834834835	0\\
8.36336336336336	0\\
8.37837837837838	0\\
8.39339339339339	0\\
8.40840840840841	0\\
8.42342342342342	0\\
8.43843843843844	0\\
8.45345345345345	0\\
8.46846846846847	0\\
8.48348348348348	0\\
8.4984984984985	0\\
8.51351351351351	0\\
8.52852852852853	0\\
8.54354354354354	0\\
8.55855855855856	0\\
8.57357357357357	0\\
8.58858858858859	0\\
8.6036036036036	0\\
8.61861861861862	0\\
8.63363363363363	0\\
8.64864864864865	0\\
8.66366366366366	0\\
8.67867867867868	0\\
8.69369369369369	0\\
8.70870870870871	0\\
8.72372372372372	0\\
8.73873873873874	0\\
8.75375375375375	0\\
8.76876876876877	0\\
8.78378378378378	0\\
8.7987987987988	0\\
8.81381381381381	0\\
8.82882882882883	0\\
8.84384384384384	0\\
8.85885885885886	0\\
8.87387387387387	0\\
8.88888888888889	0\\
8.9039039039039	0\\
8.91891891891892	0\\
8.93393393393393	0\\
8.94894894894895	0\\
8.96396396396396	0\\
8.97897897897898	0\\
8.99399399399399	0\\
9.00900900900901	0\\
9.02402402402402	0\\
9.03903903903904	0\\
9.05405405405405	0\\
9.06906906906907	0\\
9.08408408408408	0\\
9.0990990990991	0\\
9.11411411411411	0\\
9.12912912912913	0\\
9.14414414414414	0\\
9.15915915915916	0\\
9.17417417417417	0\\
9.18918918918919	0\\
9.2042042042042	0\\
9.21921921921922	0\\
9.23423423423423	0\\
9.24924924924925	0\\
9.26426426426426	0\\
9.27927927927928	0\\
9.29429429429429	0\\
9.30930930930931	0\\
9.32432432432432	0\\
9.33933933933934	0\\
9.35435435435435	0\\
9.36936936936937	0\\
9.38438438438438	0\\
9.3993993993994	0\\
9.41441441441441	0\\
9.42942942942943	0\\
9.44444444444444	0\\
9.45945945945946	0\\
9.47447447447447	0\\
9.48948948948949	0\\
9.5045045045045	0\\
9.51951951951952	0\\
9.53453453453453	0\\
9.54954954954955	0\\
9.56456456456456	0\\
9.57957957957958	0\\
9.59459459459459	0\\
9.60960960960961	0\\
9.62462462462462	0\\
9.63963963963964	0\\
9.65465465465465	0\\
9.66966966966967	0\\
9.68468468468468	0\\
9.6996996996997	0\\
9.71471471471471	0\\
9.72972972972973	0\\
9.74474474474474	0\\
9.75975975975976	0\\
9.77477477477477	0\\
9.78978978978979	0\\
9.8048048048048	0\\
9.81981981981982	0\\
9.83483483483483	0\\
9.84984984984985	0\\
9.86486486486486	0\\
9.87987987987988	0\\
9.89489489489489	0\\
9.90990990990991	0\\
9.92492492492492	0\\
9.93993993993994	0\\
9.95495495495495	0\\
9.96996996996997	0\\
9.98498498498498	0\\
10	0\\
10.015015015015	0\\
10.03003003003	0\\
10.045045045045	0\\
10.0600600600601	0\\
10.0750750750751	0\\
10.0900900900901	0\\
10.1051051051051	0\\
10.1201201201201	0\\
10.1351351351351	0\\
10.1501501501502	0\\
10.1651651651652	0\\
10.1801801801802	0\\
10.1951951951952	0\\
10.2102102102102	0\\
10.2252252252252	0\\
10.2402402402402	0\\
10.2552552552553	0\\
10.2702702702703	0\\
10.2852852852853	0\\
10.3003003003003	0\\
10.3153153153153	0\\
10.3303303303303	0\\
10.3453453453453	0\\
10.3603603603604	0\\
10.3753753753754	0\\
10.3903903903904	0\\
10.4054054054054	0\\
10.4204204204204	0\\
10.4354354354354	0\\
10.4504504504505	0\\
10.4654654654655	0\\
10.4804804804805	0\\
10.4954954954955	0\\
10.5105105105105	0\\
10.5255255255255	0\\
10.5405405405405	0\\
10.5555555555556	0\\
10.5705705705706	0\\
10.5855855855856	0\\
10.6006006006006	0\\
10.6156156156156	0\\
10.6306306306306	0\\
10.6456456456456	0\\
10.6606606606607	0\\
10.6756756756757	0\\
10.6906906906907	0\\
10.7057057057057	0\\
10.7207207207207	0\\
10.7357357357357	0\\
10.7507507507508	0\\
10.7657657657658	0\\
10.7807807807808	0\\
10.7957957957958	0\\
10.8108108108108	0\\
10.8258258258258	0\\
10.8408408408408	0\\
10.8558558558559	0\\
10.8708708708709	0\\
10.8858858858859	0\\
10.9009009009009	0\\
10.9159159159159	0\\
10.9309309309309	0\\
10.9459459459459	0\\
10.960960960961	0\\
10.975975975976	0\\
10.990990990991	0\\
11.006006006006	0\\
11.021021021021	0\\
11.036036036036	0\\
11.0510510510511	0\\
11.0660660660661	0\\
11.0810810810811	0\\
11.0960960960961	0\\
11.1111111111111	0\\
11.1261261261261	0\\
11.1411411411411	0\\
11.1561561561562	0\\
11.1711711711712	0\\
11.1861861861862	0\\
11.2012012012012	0\\
11.2162162162162	0\\
11.2312312312312	0\\
11.2462462462462	0\\
11.2612612612613	0\\
11.2762762762763	0\\
11.2912912912913	0\\
11.3063063063063	0\\
11.3213213213213	0\\
11.3363363363363	0\\
11.3513513513514	0\\
11.3663663663664	0\\
11.3813813813814	0\\
11.3963963963964	0\\
11.4114114114114	0\\
11.4264264264264	0\\
11.4414414414414	0\\
11.4564564564565	0\\
11.4714714714715	0\\
11.4864864864865	0\\
11.5015015015015	0\\
11.5165165165165	0\\
11.5315315315315	0\\
11.5465465465465	0\\
11.5615615615616	0\\
11.5765765765766	0\\
11.5915915915916	0\\
11.6066066066066	0\\
11.6216216216216	0\\
11.6366366366366	0\\
11.6516516516517	0\\
11.6666666666667	0\\
11.6816816816817	0\\
11.6966966966967	0\\
11.7117117117117	0\\
11.7267267267267	0\\
11.7417417417417	0\\
11.7567567567568	0\\
11.7717717717718	0\\
11.7867867867868	0\\
11.8018018018018	0\\
11.8168168168168	0\\
11.8318318318318	0\\
11.8468468468468	0\\
11.8618618618619	0\\
11.8768768768769	0\\
11.8918918918919	0\\
11.9069069069069	0\\
11.9219219219219	0\\
11.9369369369369	0\\
11.951951951952	0\\
11.966966966967	0\\
11.981981981982	0\\
11.996996996997	0\\
12.012012012012	0\\
12.027027027027	0\\
12.042042042042	0\\
12.0570570570571	0\\
12.0720720720721	0\\
12.0870870870871	0\\
12.1021021021021	0\\
12.1171171171171	0\\
12.1321321321321	0\\
12.1471471471471	0\\
12.1621621621622	0\\
12.1771771771772	0\\
12.1921921921922	0\\
12.2072072072072	0\\
12.2222222222222	0\\
12.2372372372372	0\\
12.2522522522523	0\\
12.2672672672673	0\\
12.2822822822823	0\\
12.2972972972973	0\\
12.3123123123123	0\\
12.3273273273273	0\\
12.3423423423423	0\\
12.3573573573574	0\\
12.3723723723724	0\\
12.3873873873874	0\\
12.4024024024024	0\\
12.4174174174174	0\\
12.4324324324324	0\\
12.4474474474474	0\\
12.4624624624625	0\\
12.4774774774775	0\\
12.4924924924925	0\\
12.5075075075075	0\\
12.5225225225225	0\\
12.5375375375375	0\\
12.5525525525526	0\\
12.5675675675676	0\\
12.5825825825826	0\\
12.5975975975976	0\\
12.6126126126126	0\\
12.6276276276276	0\\
12.6426426426426	0\\
12.6576576576577	0\\
12.6726726726727	0\\
12.6876876876877	0\\
12.7027027027027	0\\
12.7177177177177	0\\
12.7327327327327	0\\
12.7477477477477	0\\
12.7627627627628	0\\
12.7777777777778	0\\
12.7927927927928	0\\
12.8078078078078	0\\
12.8228228228228	0\\
12.8378378378378	0\\
12.8528528528529	0\\
12.8678678678679	0\\
12.8828828828829	0\\
12.8978978978979	0\\
12.9129129129129	0\\
12.9279279279279	0\\
12.9429429429429	0\\
12.957957957958	0\\
12.972972972973	0\\
12.987987987988	0\\
13.003003003003	0\\
13.018018018018	0\\
13.033033033033	0\\
13.048048048048	0\\
13.0630630630631	0\\
13.0780780780781	0\\
13.0930930930931	0\\
13.1081081081081	0\\
13.1231231231231	0\\
13.1381381381381	0\\
13.1531531531532	0\\
13.1681681681682	0\\
13.1831831831832	0\\
13.1981981981982	0\\
13.2132132132132	0\\
13.2282282282282	0\\
13.2432432432432	0\\
13.2582582582583	0\\
13.2732732732733	0\\
13.2882882882883	0\\
13.3033033033033	0\\
13.3183183183183	0\\
13.3333333333333	0\\
13.3483483483483	0\\
13.3633633633634	0\\
13.3783783783784	0\\
13.3933933933934	0\\
13.4084084084084	0\\
13.4234234234234	0\\
13.4384384384384	0\\
13.4534534534535	0\\
13.4684684684685	0\\
13.4834834834835	0\\
13.4984984984985	0\\
13.5135135135135	0\\
13.5285285285285	0\\
13.5435435435435	0\\
13.5585585585586	0\\
13.5735735735736	0\\
13.5885885885886	0\\
13.6036036036036	0\\
13.6186186186186	0\\
13.6336336336336	0\\
13.6486486486486	0\\
13.6636636636637	0\\
13.6786786786787	0\\
13.6936936936937	0\\
13.7087087087087	0\\
13.7237237237237	0\\
13.7387387387387	0\\
13.7537537537538	0\\
13.7687687687688	0\\
13.7837837837838	0\\
13.7987987987988	0\\
13.8138138138138	0\\
13.8288288288288	0\\
13.8438438438438	0\\
13.8588588588589	0\\
13.8738738738739	0\\
13.8888888888889	0\\
13.9039039039039	0\\
13.9189189189189	0\\
13.9339339339339	0\\
13.9489489489489	0\\
13.963963963964	0\\
13.978978978979	0\\
13.993993993994	0\\
14.009009009009	0\\
14.024024024024	0\\
14.039039039039	0\\
14.0540540540541	0\\
14.0690690690691	0\\
14.0840840840841	0\\
14.0990990990991	0\\
14.1141141141141	0\\
14.1291291291291	0\\
14.1441441441441	0\\
14.1591591591592	0\\
14.1741741741742	0\\
14.1891891891892	0\\
14.2042042042042	0\\
14.2192192192192	0\\
14.2342342342342	0\\
14.2492492492492	0\\
14.2642642642643	0\\
14.2792792792793	0\\
14.2942942942943	0\\
14.3093093093093	0\\
14.3243243243243	0\\
14.3393393393393	0\\
14.3543543543544	0\\
14.3693693693694	0\\
14.3843843843844	0\\
14.3993993993994	0\\
14.4144144144144	0\\
14.4294294294294	0\\
14.4444444444444	0\\
14.4594594594595	0\\
14.4744744744745	0\\
14.4894894894895	0\\
14.5045045045045	0\\
14.5195195195195	0\\
14.5345345345345	0\\
14.5495495495495	0\\
14.5645645645646	0\\
14.5795795795796	0\\
14.5945945945946	0\\
14.6096096096096	0\\
14.6246246246246	0\\
14.6396396396396	0\\
14.6546546546547	0\\
14.6696696696697	0\\
14.6846846846847	0\\
14.6996996996997	0\\
14.7147147147147	0\\
14.7297297297297	0\\
14.7447447447447	0\\
14.7597597597598	0\\
14.7747747747748	0\\
14.7897897897898	0\\
14.8048048048048	0\\
14.8198198198198	0\\
14.8348348348348	0\\
14.8498498498498	0\\
14.8648648648649	0\\
14.8798798798799	0\\
14.8948948948949	0\\
14.9099099099099	0\\
14.9249249249249	0\\
14.9399399399399	0\\
14.954954954955	0\\
14.96996996997	0\\
14.984984984985	0\\
15	0\\
};
\addlegendentry{$d = 1$};

\addplot [color=mycolor2,solid]
  table[row sep=crcr]{%
0	3.73062887762674e-05\\
0.015015015015015	5.2850520880674e-05\\
0.03003003003003	7.39091880550532e-05\\
0.045045045045045	0.000101584178729442\\
0.0600600600600601	0.000137681122814274\\
0.0750750750750751	0.000184205755779773\\
0.0900900900900901	0.000243067749518483\\
0.105105105105105	0.000316357580800687\\
0.12012012012012	0.000407225965473064\\
0.135135135135135	0.000517432264850509\\
0.15015015015015	0.000649278287321303\\
0.165165165165165	0.000805310791651178\\
0.18018018018018	0.000987607816052387\\
0.195195195195195	0.00119854216173312\\
0.21021021021021	0.00143929188439279\\
0.225225225225225	0.00171145906241416\\
0.24024024024024	0.00201571565744264\\
0.255255255255255	0.0023533787958585\\
0.27027027027027	0.00272535288253831\\
0.285285285285285	0.00313150951820724\\
0.3003003003003	0.00357326480200382\\
0.315315315315315	0.00405135851555322\\
0.33033033033033	0.00456753164739084\\
0.345345345345345	0.00512270199047833\\
0.36036036036036	0.00571960759143936\\
0.375375375375375	0.00636101506059456\\
0.39039039039039	0.00705104166437239\\
0.405405405405405	0.00779460275531353\\
0.42042042042042	0.0085977554334307\\
0.435435435435435	0.00946594395923427\\
0.45045045045045	0.0104088971810006\\
0.465465465465465	0.0114319984322966\\
0.48048048048048	0.0125460769769706\\
0.495495495495495	0.0137591799510193\\
0.510510510510511	0.0150836668921465\\
0.525525525525526	0.0165273050542166\\
0.540540540540541	0.0181022253460605\\
0.555555555555556	0.019820805708404\\
0.570570570570571	0.0216941316630409\\
0.585585585585586	0.0237333382890649\\
0.600600600600601	0.0259490851568516\\
0.615615615615616	0.0283531571579315\\
0.630630630630631	0.0309562369601162\\
0.645645645645646	0.0337669143655404\\
0.660660660660661	0.0367925854349795\\
0.675675675675676	0.0400396933780995\\
0.690690690690691	0.0435109820016503\\
0.705705705705706	0.0472049449040617\\
0.720720720720721	0.0511187369832666\\
0.735735735735736	0.0552463365860636\\
0.750750750750751	0.0595792435583405\\
0.765765765765766	0.0641040024586522\\
0.780780780780781	0.0688087441943765\\
0.795795795795796	0.0736805423385434\\
0.810810810810811	0.0787053834632795\\
0.825825825825826	0.0838702263053392\\
0.840840840840841	0.0891719262299881\\
0.855855855855856	0.0946033358193208\\
0.870870870870871	0.100169559572885\\
0.885885885885886	0.105876278886046\\
0.900900900900901	0.111736800680279\\
0.915915915915916	0.117767386858794\\
0.930930930930931	0.123989820251242\\
0.945945945945946	0.130425597016901\\
0.960960960960961	0.137099124693482\\
0.975975975975976	0.144021594100601\\
0.990990990990991	0.151211436300509\\
1.00600600600601	0.158676274465364\\
1.02102102102102	0.166415867947716\\
1.03603603603604	0.174429108112427\\
1.05105105105105	0.182697903443893\\
1.06606606606607	0.191205593891426\\
1.08108108108108	0.199929629372186\\
1.0960960960961	0.20884289964929\\
1.11111111111111	0.217920323326409\\
1.12612612612613	0.227131862970782\\
1.14114114114114	0.236456760643241\\
1.15615615615616	0.245873647459482\\
1.17117117117117	0.255365891775979\\
1.18618618618619	0.264920258837785\\
1.2012012012012	0.274522604004335\\
1.21621621621622	0.284163266007837\\
1.23123123123123	0.293831286916582\\
1.24624624624625	0.303514957964992\\
1.26126126126126	0.313202929464975\\
1.27627627627628	0.322878543511187\\
1.29129129129129	0.332524349850946\\
1.30630630630631	0.342114939239241\\
1.32132132132132	0.351629731566188\\
1.33633633633634	0.361040535067312\\
1.35135135135135	0.370320936781018\\
1.36636636636637	0.379446154243144\\
1.38138138138138	0.388392382522346\\
1.3963963963964	0.397142156553312\\
1.41141141141141	0.40567587322378\\
1.42642642642643	0.4139841283913\\
1.44144144144144	0.422060990349642\\
1.45645645645646	0.429904886030139\\
1.47147147147147	0.437517950707701\\
1.48648648648649	0.444903586551676\\
1.5015015015015	0.452068114102149\\
1.51651651651652	0.45901908296601\\
1.53153153153153	0.465759533970637\\
1.54654654654655	0.472290961519962\\
1.56156156156156	0.478611151073957\\
1.57657657657658	0.48471904700502\\
1.59159159159159	0.490605115763205\\
1.60660660660661	0.496262425752077\\
1.62162162162162	0.501684848349521\\
1.63663663663664	0.506864209931794\\
1.65165165165165	0.511801107376648\\
1.66666666666667	0.516498994991292\\
1.68168168168168	0.520965089568492\\
1.6966966966967	0.525218916586845\\
1.71171171171171	0.529278656816944\\
1.72672672672673	0.533167982455025\\
1.74174174174174	0.536912224017264\\
1.75675675675676	0.540535074121939\\
1.77177177177177	0.544054275235327\\
1.78678678678679	0.547486395332669\\
1.8018018018018	0.550824605076296\\
1.81681681681682	0.55406854913666\\
1.83183183183183	0.557195075526331\\
1.84684684684685	0.560173641338534\\
1.86186186186186	0.562965437725319\\
1.87687687687688	0.565520173756553\\
1.89189189189189	0.567795220818931\\
1.90690690690691	0.569749211168862\\
1.92192192192192	0.571336187237077\\
1.93693693693694	0.572522363298939\\
1.95195195195195	0.573289299057408\\
1.96696696696697	0.573625677590814\\
1.98198198198198	0.573537853961602\\
1.996996996997	0.573040802098852\\
2.01201201201201	0.572153739737803\\
2.02702702702703	0.570909019863259\\
2.04204204204204	0.569338503931242\\
2.05705705705706	0.567477030448308\\
2.07207207207207	0.565351926762711\\
2.08708708708709	0.562984188999962\\
2.1021021021021	0.560389152066384\\
2.11711711711712	0.557573792512655\\
2.13213213213213	0.554541369203858\\
2.14714714714715	0.551292175795083\\
2.16216216216216	0.547822484937962\\
2.17717717717718	0.544129745172819\\
2.19219219219219	0.540216457472885\\
2.20720720720721	0.536094373428441\\
2.22222222222222	0.53178020507159\\
2.23723723723724	0.527302098063285\\
2.25225225225225	0.522693752037292\\
2.26726726726727	0.517996431671904\\
2.28228228228228	0.513255761993612\\
2.2972972972973	0.508512058379636\\
2.31231231231231	0.503810295293224\\
2.32732732732733	0.499182966799269\\
2.34234234234234	0.494657813002642\\
2.35735735735736	0.490250540032092\\
2.37237237237237	0.485969777978042\\
2.38738738738739	0.481807231453727\\
2.4024024024024	0.477747555250062\\
2.41741741741742	0.473768841440953\\
2.43243243243243	0.469841909933725\\
2.44744744744745	0.465932559303521\\
2.46246246246246	0.462005015099606\\
2.47747747747748	0.458022896831486\\
2.49249249249249	0.453954553243317\\
2.50750750750751	0.449765667563197\\
2.52252252252252	0.445429704305251\\
2.53753753753754	0.44092673678281\\
2.55255255255255	0.436237046584582\\
2.56756756756757	0.431352990165653\\
2.58258258258258	0.426265105646386\\
2.5975975975976	0.420975784403088\\
2.61261261261261	0.415490784798415\\
2.62762762762763	0.409814797601156\\
2.64264264264264	0.403966278808646\\
2.65765765765766	0.397958738883362\\
2.67267267267267	0.391810781527959\\
2.68768768768769	0.385540517581063\\
2.7027027027027	0.379163232462754\\
2.71771771771772	0.372691785659823\\
2.73273273273273	0.366139298768377\\
2.74774774774775	0.359510392420609\\
2.76276276276276	0.35281176525659\\
2.77777777777778	0.346043338933509\\
2.79279279279279	0.339207118869498\\
2.80780780780781	0.332306328544247\\
2.82282282282282	0.325337170048381\\
2.83783783783784	0.318306679153248\\
2.85285285285285	0.311218780663916\\
2.86786786786787	0.304090377771771\\
2.88288288288288	0.296934702077012\\
2.8978978978979	0.289768625617541\\
2.91291291291291	0.282614212244254\\
2.92792792792793	0.275493140263703\\
2.94294294294294	0.268430033264759\\
2.95795795795796	0.261439267043215\\
2.97297297297297	0.254540060232271\\
2.98798798798799	0.247744113322742\\
3.003003003003	0.241057994302441\\
3.01801801801802	0.234491423998963\\
3.03303303303303	0.228043616656158\\
3.04804804804805	0.221716241304777\\
3.06306306306306	0.215509437030581\\
3.07807807807808	0.209419811305617\\
3.09309309309309	0.203447367486739\\
3.10810810810811	0.197595110121632\\
3.12312312312312	0.19186543848273\\
3.13813813813814	0.1862588681357\\
3.15315315315315	0.180781997071615\\
3.16816816816817	0.175437647274598\\
3.18318318318318	0.170225785269357\\
3.1981981981982	0.165149460706507\\
3.21321321321321	0.160207527491553\\
3.22822822822823	0.155398017856122\\
3.24324324324324	0.150717375485888\\
3.25825825825826	0.146165162842721\\
3.27327327327327	0.141735342268613\\
3.28828828828829	0.137424566679863\\
3.3033033033033	0.133229437612024\\
3.31831831831832	0.12915278870491\\
3.33333333333333	0.125191528341134\\
3.34834834834835	0.121345746122583\\
3.36336336336336	0.117615553871989\\
3.37837837837838	0.113997903006241\\
3.39339339339339	0.110492538300894\\
3.40840840840841	0.107090698520776\\
3.42342342342342	0.103783665887061\\
3.43843843843844	0.100562140875016\\
3.45345345345345	0.0974131044834166\\
3.46846846846847	0.0943196379230629\\
3.48348348348348	0.0912686496720465\\
3.4984984984985	0.0882466341298904\\
3.51351351351351	0.0852426432635931\\
3.52852852852853	0.0822483953107446\\
3.54354354354354	0.0792604570268851\\
3.55855855855856	0.0762799863745805\\
3.57357357357357	0.0733107634797899\\
3.58858858858859	0.0703626278105554\\
3.6036036036036	0.0674459314404412\\
3.61861861861862	0.0645751282069851\\
3.63363363363363	0.0617659763939325\\
3.64864864864865	0.0590338340181209\\
3.66366366366366	0.0563937840235677\\
3.67867867867868	0.0538588347904559\\
3.69369369369369	0.0514409554861182\\
3.70870870870871	0.049150182552605\\
3.72372372372372	0.0469920425882154\\
3.73873873873874	0.0449702061582226\\
3.75375375375375	0.0430858180654385\\
3.76876876876877	0.0413381068464204\\
3.78378378378378	0.0397231040073686\\
3.7987987987988	0.0382324803481281\\
3.81381381381381	0.0368570069083813\\
3.82882882882883	0.0355875014238295\\
3.84384384384384	0.0344114758762365\\
3.85885885885886	0.0333153166912438\\
3.87387387387387	0.0322868391777166\\
3.88888888888889	0.0313140645369403\\
3.9039039039039	0.0303860438701187\\
3.91891891891892	0.0294929608250262\\
3.93393393393393	0.0286293078980699\\
3.94894894894895	0.0277906043625314\\
3.96396396396396	0.0269741515606635\\
3.97897897897898	0.0261798759269464\\
3.99399399399399	0.0254082652081843\\
4.00900900900901	0.0246626515741214\\
4.02402402402402	0.0239435270127824\\
4.03903903903904	0.0232524500101682\\
4.05405405405405	0.0225882615883098\\
4.06906906906907	0.0219486186005998\\
4.08408408408408	0.021328070160324\\
4.0990990990991	0.0207197024756218\\
4.11411411411411	0.0201150603605411\\
4.12912912912913	0.0195046850594732\\
4.14414414414414	0.0188776432876166\\
4.15915915915916	0.0182262499994917\\
4.17417417417417	0.0175408707585813\\
4.18918918918919	0.0168175250902383\\
4.2042042042042	0.0160535348821794\\
4.21921921921922	0.0152500829067194\\
4.23423423423423	0.014410119638616\\
4.24924924924925	0.0135395957954629\\
4.26426426426426	0.0126472875051784\\
4.27927927927928	0.0117425557151807\\
4.29429429429429	0.0108359899802075\\
4.30930930930931	0.00993783173639815\\
4.32432432432432	0.00905770918878407\\
4.33933933933934	0.00820486398292945\\
4.35435435435435	0.00738579482612358\\
4.36936936936937	0.00660724150962239\\
4.38438438438438	0.00587299592154615\\
4.3993993993994	0.00518620973149331\\
4.41441441441441	0.00454970493490857\\
4.42942942942943	0.00396331811491906\\
4.44444444444444	0.00342798325047684\\
4.45945945945946	0.00294355232870828\\
4.47447447447447	0.0025091365980757\\
4.48948948948949	0.00212410002726315\\
4.5045045045045	0.0017860949776557\\
4.51951951951952	0.00149416562102573\\
4.53453453453453	0.00124583444501241\\
4.54954954954955	0.00103936740746553\\
4.56456456456456	0.000872389593554434\\
4.57957957957958	0.000742259924983809\\
4.59459459459459	0.000646080993675036\\
4.60960960960961	0.000581688606341412\\
4.62462462462462	0.000545820343094881\\
4.63963963963964	0.000535316382603733\\
4.65465465465465	0.000548346609743375\\
4.66966966966967	0.000581196966610551\\
4.68468468468468	0.000630496985870127\\
4.6996996996997	0.00069329670307687\\
4.71471471471471	0.000766499979348926\\
4.72972972972973	0.000846773804186389\\
4.74474474474474	0.000931005003520019\\
4.75975975975976	0.00101603652334077\\
4.77477477477477	0.0010985627366523\\
4.78978978978979	0.00117616337633005\\
4.8048048048048	0.00124592738182519\\
4.81981981981982	0.00130567002662357\\
4.83483483483483	0.00135364210947084\\
4.84984984984985	0.00138853627385579\\
4.86486486486486	0.00140908130092343\\
4.87987987987988	0.00141474379319635\\
4.89489489489489	0.00140528066390814\\
4.90990990990991	0.00138139137302832\\
4.92492492492492	0.00134357057015299\\
4.93993993993994	0.00129292385022736\\
4.95495495495495	0.00123086980896987\\
4.96996996996997	0.0011590882449508\\
4.98498498498498	0.00107946103450908\\
5	0.000994006414074872\\
5.01501501501502	0.000904807897432203\\
5.03003003003003	0.000813939903847055\\
5.04504504504505	0.000723393159600207\\
5.06006006006006	0.000635003787873118\\
5.07507507507508	0.000550390462693108\\
5.09009009009009	0.000470903893449142\\
5.10510510510511	0.000397592170610917\\
5.12012012012012	0.000331184217079886\\
5.13513513513514	0.000272091942410765\\
5.15015015015015	0.000220429946742422\\
5.16516516516517	0.000176050037414307\\
5.18018018018018	0.000138586631343898\\
5.1951951951952	0.000107508465271302\\
5.21021021021021	8.21719676367196e-05\\
5.22522522522523	6.18721063451356e-05\\
5.24024024024024	4.58873870067682e-05\\
5.25525525525526	3.3437772891042e-05\\
5.27027027027027	2.40611216147088e-05\\
5.28528528528529	1.70464048234936e-05\\
5.3003003003003	1.18892110572696e-05\\
5.31531531531532	8.08967294788104e-06\\
5.33033033033033	5.37455132470094e-06\\
5.34534534534535	3.58650067955363e-06\\
5.36036036036036	2.35466019663982e-06\\
5.37537537537538	1.52093974170595e-06\\
5.39039039039039	9.66540375573236e-07\\
5.40540540540541	6.04296362003937e-07\\
5.42042042042042	2.83031244414513e-07\\
5.43543543543544	1.72821448345973e-07\\
5.45045045045045	1.03791906412902e-07\\
5.46546546546547	6.13100861601831e-08\\
5.48048048048048	3.56207412630106e-08\\
5.4954954954955	2.03552559738763e-08\\
5.51051051051051	1.14407046039938e-08\\
5.52552552552553	6.32457818216736e-09\\
5.54054054054054	3.43884774314144e-09\\
5.55555555555556	1.8390643199228e-09\\
5.57057057057057	9.67349649356835e-10\\
5.58558558558559	5.00463751903651e-10\\
5.6006006006006	2.54662128835958e-10\\
5.61561561561562	1.27455532550155e-10\\
5.63063063063063	6.27416025922832e-11\\
5.64564564564565	3.03777156187431e-11\\
5.66066066066066	1.44662912874136e-11\\
5.67567567567568	6.77582043096162e-12\\
5.69069069069069	3.12154183248587e-12\\
5.70570570570571	1.41442205500358e-12\\
5.72072072072072	6.30364096015896e-13\\
5.73573573573574	2.76316301334864e-13\\
5.75075075075075	1.19130820789861e-13\\
5.76576576576577	5.05177907840098e-14\\
5.78078078078078	2.10701270749871e-14\\
5.7957957957958	8.64355796303804e-15\\
5.81081081081081	3.48755071271113e-15\\
5.82582582582583	1.38404784386837e-15\\
5.84084084084084	5.40236923709579e-16\\
5.85585585585586	2.0740537791052e-16\\
5.87087087087087	7.831741631093e-17\\
5.88588588588589	2.90870233157207e-17\\
5.9009009009009	1.06253390071489e-17\\
5.91591591591592	3.81758662617701e-18\\
5.93093093093093	1.34907966066256e-18\\
5.94594594594595	4.68909334001945e-19\\
5.96096096096096	1.60303405924149e-19\\
5.97597597597598	5.39012964036428e-20\\
5.99099099099099	1.78261788945896e-20\\
6.00600600600601	5.79855658209919e-21\\
6.02102102102102	1.85517176998018e-21\\
6.03603603603604	5.83782327461024e-22\\
6.05105105105105	1.80684271584911e-22\\
6.06606606606607	5.50037556032123e-23\\
6.08108108108108	1.64689885314976e-23\\
6.0960960960961	4.85002636492327e-24\\
6.11111111111111	1.4048302837867e-24\\
6.12612612612613	4.00226840891833e-25\\
6.14114114114114	1.12147901105877e-25\\
6.15615615615616	3.09085532996994e-26\\
6.17117117117117	8.37854863561138e-27\\
6.18618618618619	2.2338885435954e-27\\
6.2012012012012	5.85810019826848e-28\\
6.21621621621622	1.51096595661662e-28\\
6.23123123123123	3.83314415306666e-29\\
6.24624624624625	9.56441063441009e-30\\
6.26126126126126	2.34727441033382e-30\\
6.27627627627628	5.66594172738258e-31\\
6.29129129129129	1.34518783744317e-31\\
6.30630630630631	3.14120550047714e-32\\
6.32132132132132	7.21460162839478e-33\\
6.33633633633634	1.62978738009212e-33\\
6.35135135135135	3.62119685637677e-34\\
6.36636636636637	7.91363303588596e-35\\
6.38138138138138	1.70099199418815e-35\\
6.3963963963964	3.59609553177553e-36\\
6.41141141141141	7.47760913428963e-37\\
6.42642642642643	1.52931479097996e-37\\
6.44144144144144	3.07633485798301e-38\\
6.45645645645646	6.08657409183929e-39\\
6.47147147147147	1.18444475335856e-39\\
6.48648648648649	2.26704057400642e-40\\
6.5015015015015	4.26782267238923e-41\\
6.51651651651652	7.90234577012569e-42\\
6.53153153153153	1.43915734871814e-42\\
6.54654654654655	2.57788250856106e-43\\
6.56156156156156	4.54172201415685e-44\\
6.57657657657658	7.87010572716525e-45\\
6.59159159159159	1.34135327387591e-45\\
6.60660660660661	2.2485801339964e-46\\
6.62162162162162	3.70745726840965e-47\\
6.63663663663664	6.01238147166659e-48\\
6.65165165165165	9.59001950204551e-49\\
6.66666666666667	1.50450987231122e-49\\
6.68168168168168	2.32152402402007e-50\\
6.6966966966967	3.52333478581069e-51\\
6.71171171171171	5.25941244771755e-52\\
6.72672672672673	7.72188157911024e-53\\
6.74174174174174	1.11509439599065e-53\\
6.75675675675676	1.5838086782462e-54\\
6.77177177177177	2.21256680427216e-55\\
6.78678678678679	3.04013350119568e-56\\
6.8018018018018	4.10857772262264e-57\\
6.81681681681682	5.4612611923176e-58\\
6.83183183183183	7.13997970705378e-59\\
6.84684684684685	9.1812876920533e-60\\
6.86186186186186	1.16121544489049e-60\\
6.87687687687688	1.44452346193771e-61\\
6.89189189189189	1.76741682690125e-62\\
6.90690690690691	2.12694353063808e-63\\
6.92192192192192	2.51753503439446e-64\\
6.93693693693694	2.93087755564164e-65\\
6.95195195195195	3.35600358605893e-66\\
6.96696696696697	3.77963399276641e-67\\
6.98198198198198	4.18677541377272e-68\\
6.996996996997	4.56154720008789e-69\\
7.01201201201201	4.88818090092363e-70\\
7.02702702702703	5.15210796776894e-71\\
7.04204204204204	5.34103261725434e-72\\
7.05705705705706	5.44588032999469e-73\\
7.07207207207207	5.46152022463594e-74\\
7.08708708708709	5.38718125198341e-75\\
7.1021021021021	5.22651520269416e-76\\
7.11711711711712	4.98729935958894e-77\\
7.13213213213213	4.68081250621546e-78\\
7.14714714714715	4.32095399649709e-79\\
7.16216216216216	3.92320162706672e-80\\
7.17717717717718	3.50351679379772e-81\\
7.19219219219219	3.07730376001878e-82\\
7.20720720720721	2.65851503212977e-83\\
7.22222222222222	2.25896999318689e-84\\
7.23723723723724	1.88792350140485e-85\\
7.25225225225225	1.55188992020106e-86\\
7.26726726726727	1.25470035491662e-87\\
7.28228228228228	9.9774995826548e-89\\
7.2972972972973	7.80379773829937e-90\\
7.31231231231231	6.00333916483999e-91\\
7.32732732732733	4.5423682981613e-92\\
7.34234234234234	3.3804490709244e-93\\
7.35735735735736	2.47439505695806e-94\\
7.37237237237237	1.7814197809071e-95\\
7.38738738738739	1.26143852471516e-96\\
7.4024024024024	8.78554078479906e-98\\
7.41741741741742	6.0182953485037e-99\\
7.43243243243243	4.05490873326541e-100\\
7.44744744744745	2.68714597418116e-101\\
7.46246246246246	1.75147532726702e-102\\
7.47747747747748	1.12284388286667e-103\\
7.49249249249249	7.08006622126896e-105\\
7.50750750750751	4.39094352205132e-106\\
7.52252252252252	2.67843413608035e-107\\
7.53753753753754	1.60696597193334e-108\\
7.55255255255255	9.48276479340859e-110\\
7.56756756756757	5.50384082958977e-111\\
7.58258258258258	3.14195052154916e-112\\
7.5975975975976	1.76414958355658e-113\\
7.61261261261261	9.74258178757805e-115\\
7.62762762762763	5.29194477099887e-116\\
7.64264264264264	2.82721694234061e-117\\
7.65765765765766	1.48561253553199e-118\\
7.67267267267267	7.67811415638829e-120\\
7.68768768768769	3.90306840933174e-121\\
7.7027027027027	1.95146313123147e-122\\
7.71771771771772	9.59659411649874e-124\\
7.73273273273273	4.64169384587192e-125\\
7.74774774774775	2.20820020704104e-126\\
7.76276276276276	1.03324423148045e-127\\
7.77777777777778	4.75521460168559e-129\\
7.79279279279279	2.15248355556657e-130\\
7.80780780780781	9.58323530972257e-132\\
7.82282282282282	4.19649829354114e-133\\
7.83783783783784	1.80744270195242e-134\\
7.85285285285285	7.65675260847183e-136\\
7.86786786786787	3.19026883007367e-137\\
7.88288288288288	1.30741220586073e-138\\
7.8978978978979	5.26987578367925e-140\\
7.91291291291291	2.08925192452899e-141\\
7.92792792792793	8.14673954876711e-143\\
7.94294294294294	3.12449218188742e-144\\
7.95795795795796	1.17863040787363e-145\\
7.97297297297297	4.37298944837712e-147\\
7.98798798798799	1.5958122935048e-148\\
8.003003003003	5.7277990390049e-150\\
8.01801801801802	2.02207065714347e-151\\
8.03303303303303	7.02113737975828e-153\\
8.04804804804805	2.39784553642385e-154\\
8.06306306306306	8.05448032550794e-156\\
8.07807807807808	2.66107083702082e-157\\
8.09309309309309	8.6472484056811e-159\\
8.10810810810811	2.76377079501207e-160\\
8.12312312312312	8.68817781156664e-162\\
8.13813813813814	2.6863213688775e-163\\
8.15315315315315	8.16939362085459e-165\\
8.16816816816817	2.44356708183016e-166\\
8.18318318318318	7.1888810711307e-168\\
8.1981981981982	2.08018002077357e-169\\
8.21321321321321	5.92029205733115e-171\\
8.22822822822823	1.65724963271505e-172\\
8.24324324324324	4.56284084052247e-174\\
8.25825825825826	1.23562104832895e-175\\
8.27327327327327	3.29107564232951e-177\\
8.28828828828829	8.62170223579419e-179\\
8.3033033033033	2.22152274499703e-180\\
8.31831831831832	5.63003503503157e-182\\
8.33333333333333	1.40337591773054e-183\\
8.34834834834835	3.44064235128744e-185\\
8.36336336336336	8.29674287236595e-187\\
8.37837837837838	1.96778805483016e-188\\
8.39339339339339	4.59041127753123e-190\\
8.40840840840841	1.0532402917965e-191\\
8.42342342342342	2.3768726890891e-193\\
8.43843843843844	5.27578349927121e-195\\
8.45345345345345	1.15178282680748e-196\\
8.46846846846847	2.47318623928872e-198\\
8.48348348348348	5.2233079702473e-200\\
8.4984984984985	1.0850182256885e-201\\
8.51351351351351	2.21682302039268e-203\\
8.52852852852853	4.45479374256687e-205\\
8.54354354354354	8.80494744617946e-207\\
8.55855855855856	1.71170344316519e-208\\
8.57357357357357	3.27290100391547e-210\\
8.58858858858859	6.1551664716607e-212\\
8.6036036036036	1.13854270815173e-213\\
8.61861861861862	2.07138791437861e-215\\
8.63363363363363	3.70660356195983e-217\\
8.64864864864865	6.52369169526105e-219\\
8.66366366366366	1.12931034637155e-220\\
8.67867867867868	1.92280735434125e-222\\
8.69369369369369	3.22003677006815e-224\\
8.70870870870871	5.3038163328917e-226\\
8.72372372372372	8.59248357577118e-228\\
8.73873873873874	1.36915158621063e-229\\
8.75375375375375	2.14578866115877e-231\\
8.76876876876877	3.30769111844336e-233\\
8.78378378378378	5.01493764145552e-235\\
8.7987987987988	7.47840043643504e-237\\
8.81381381381381	1.09686829635745e-238\\
8.82882882882883	1.58235105408485e-240\\
8.84384384384384	2.24519355252562e-242\\
8.85885885885886	3.13333854345018e-244\\
8.87387387387387	4.30094059997881e-246\\
8.88888888888889	5.80660330643711e-248\\
8.9039039039039	7.71051627996279e-250\\
8.91891891891892	1.00704145520647e-251\\
8.93393393393393	1.29364116870263e-253\\
8.94894894894895	1.63449236495842e-255\\
8.96396396396396	2.0312087249264e-257\\
8.97897897897898	2.48272604024889e-259\\
8.99399399399399	2.98473401069793e-261\\
9.00900900900901	3.52927141189906e-263\\
9.02402402402402	4.10456437141227e-265\\
9.03903903903904	4.69517379697033e-267\\
9.05405405405405	5.28249218770559e-269\\
9.06906906906907	5.8455939254569e-271\\
9.08408408408408	6.36240061679549e-273\\
9.0990990990991	6.81107980612835e-275\\
9.11411411411411	7.17155800374313e-277\\
9.12912912912913	7.42700387729528e-279\\
9.14414414414414	7.56512964238745e-281\\
9.15915915915916	7.57917069182324e-283\\
9.17417417417417	7.46843473362732e-285\\
9.18918918918919	7.23835836833012e-287\\
9.2042042042042	6.90006466757141e-289\\
9.21921921921922	6.46947188451097e-291\\
9.23423423423423	5.96605273139978e-293\\
9.24924924924925	5.41137877995695e-295\\
9.26426426426426	4.82760101469363e-297\\
9.27927927927928	4.23601407479878e-299\\
9.29429429429429	3.65583013084859e-301\\
9.30930930930931	3.10325318869977e-303\\
9.32432432432432	2.59090209177646e-305\\
9.33933933933934	2.12758726384789e-307\\
9.35435435435435	1.71840825793554e-309\\
9.36936936936937	1.3651108207115e-311\\
9.38438438438438	1.06662573055555e-313\\
9.3993993993994	8.19707277409087e-316\\
9.41441441441441	6.19593892611424e-318\\
9.42942942942943	4.60469181924042e-320\\
9.44444444444444	3.4090529563046e-322\\
9.45945945945946	0\\
9.47447447447447	0\\
9.48948948948949	0\\
9.5045045045045	0\\
9.51951951951952	0\\
9.53453453453453	0\\
9.54954954954955	0\\
9.56456456456456	0\\
9.57957957957958	0\\
9.59459459459459	0\\
9.60960960960961	0\\
9.62462462462462	0\\
9.63963963963964	0\\
9.65465465465465	0\\
9.66966966966967	0\\
9.68468468468468	0\\
9.6996996996997	0\\
9.71471471471471	0\\
9.72972972972973	0\\
9.74474474474474	0\\
9.75975975975976	0\\
9.77477477477477	0\\
9.78978978978979	0\\
9.8048048048048	0\\
9.81981981981982	0\\
9.83483483483483	0\\
9.84984984984985	0\\
9.86486486486486	0\\
9.87987987987988	0\\
9.89489489489489	0\\
9.90990990990991	0\\
9.92492492492492	0\\
9.93993993993994	0\\
9.95495495495495	0\\
9.96996996996997	0\\
9.98498498498498	0\\
10	0\\
10.015015015015	0\\
10.03003003003	0\\
10.045045045045	0\\
10.0600600600601	0\\
10.0750750750751	0\\
10.0900900900901	0\\
10.1051051051051	0\\
10.1201201201201	0\\
10.1351351351351	0\\
10.1501501501502	0\\
10.1651651651652	0\\
10.1801801801802	0\\
10.1951951951952	0\\
10.2102102102102	0\\
10.2252252252252	0\\
10.2402402402402	0\\
10.2552552552553	0\\
10.2702702702703	0\\
10.2852852852853	0\\
10.3003003003003	0\\
10.3153153153153	0\\
10.3303303303303	0\\
10.3453453453453	0\\
10.3603603603604	0\\
10.3753753753754	0\\
10.3903903903904	0\\
10.4054054054054	0\\
10.4204204204204	0\\
10.4354354354354	0\\
10.4504504504505	0\\
10.4654654654655	0\\
10.4804804804805	0\\
10.4954954954955	0\\
10.5105105105105	0\\
10.5255255255255	0\\
10.5405405405405	0\\
10.5555555555556	0\\
10.5705705705706	0\\
10.5855855855856	0\\
10.6006006006006	0\\
10.6156156156156	0\\
10.6306306306306	0\\
10.6456456456456	0\\
10.6606606606607	0\\
10.6756756756757	0\\
10.6906906906907	0\\
10.7057057057057	0\\
10.7207207207207	0\\
10.7357357357357	0\\
10.7507507507508	0\\
10.7657657657658	0\\
10.7807807807808	0\\
10.7957957957958	0\\
10.8108108108108	0\\
10.8258258258258	0\\
10.8408408408408	0\\
10.8558558558559	0\\
10.8708708708709	0\\
10.8858858858859	0\\
10.9009009009009	0\\
10.9159159159159	0\\
10.9309309309309	0\\
10.9459459459459	0\\
10.960960960961	0\\
10.975975975976	0\\
10.990990990991	0\\
11.006006006006	0\\
11.021021021021	0\\
11.036036036036	0\\
11.0510510510511	0\\
11.0660660660661	0\\
11.0810810810811	0\\
11.0960960960961	0\\
11.1111111111111	0\\
11.1261261261261	0\\
11.1411411411411	0\\
11.1561561561562	0\\
11.1711711711712	0\\
11.1861861861862	0\\
11.2012012012012	0\\
11.2162162162162	0\\
11.2312312312312	0\\
11.2462462462462	0\\
11.2612612612613	0\\
11.2762762762763	0\\
11.2912912912913	0\\
11.3063063063063	0\\
11.3213213213213	0\\
11.3363363363363	0\\
11.3513513513514	0\\
11.3663663663664	0\\
11.3813813813814	0\\
11.3963963963964	0\\
11.4114114114114	0\\
11.4264264264264	0\\
11.4414414414414	0\\
11.4564564564565	0\\
11.4714714714715	0\\
11.4864864864865	0\\
11.5015015015015	0\\
11.5165165165165	0\\
11.5315315315315	0\\
11.5465465465465	0\\
11.5615615615616	0\\
11.5765765765766	0\\
11.5915915915916	0\\
11.6066066066066	0\\
11.6216216216216	0\\
11.6366366366366	0\\
11.6516516516517	0\\
11.6666666666667	0\\
11.6816816816817	0\\
11.6966966966967	0\\
11.7117117117117	0\\
11.7267267267267	0\\
11.7417417417417	0\\
11.7567567567568	0\\
11.7717717717718	0\\
11.7867867867868	0\\
11.8018018018018	0\\
11.8168168168168	0\\
11.8318318318318	0\\
11.8468468468468	0\\
11.8618618618619	0\\
11.8768768768769	0\\
11.8918918918919	0\\
11.9069069069069	0\\
11.9219219219219	0\\
11.9369369369369	0\\
11.951951951952	0\\
11.966966966967	0\\
11.981981981982	0\\
11.996996996997	0\\
12.012012012012	0\\
12.027027027027	0\\
12.042042042042	0\\
12.0570570570571	0\\
12.0720720720721	0\\
12.0870870870871	0\\
12.1021021021021	0\\
12.1171171171171	0\\
12.1321321321321	0\\
12.1471471471471	0\\
12.1621621621622	0\\
12.1771771771772	0\\
12.1921921921922	0\\
12.2072072072072	0\\
12.2222222222222	0\\
12.2372372372372	0\\
12.2522522522523	0\\
12.2672672672673	0\\
12.2822822822823	0\\
12.2972972972973	0\\
12.3123123123123	0\\
12.3273273273273	0\\
12.3423423423423	0\\
12.3573573573574	0\\
12.3723723723724	0\\
12.3873873873874	0\\
12.4024024024024	0\\
12.4174174174174	0\\
12.4324324324324	0\\
12.4474474474474	0\\
12.4624624624625	0\\
12.4774774774775	0\\
12.4924924924925	0\\
12.5075075075075	0\\
12.5225225225225	0\\
12.5375375375375	0\\
12.5525525525526	0\\
12.5675675675676	0\\
12.5825825825826	0\\
12.5975975975976	0\\
12.6126126126126	0\\
12.6276276276276	0\\
12.6426426426426	0\\
12.6576576576577	0\\
12.6726726726727	0\\
12.6876876876877	0\\
12.7027027027027	0\\
12.7177177177177	0\\
12.7327327327327	0\\
12.7477477477477	0\\
12.7627627627628	0\\
12.7777777777778	0\\
12.7927927927928	0\\
12.8078078078078	0\\
12.8228228228228	0\\
12.8378378378378	0\\
12.8528528528529	0\\
12.8678678678679	0\\
12.8828828828829	0\\
12.8978978978979	0\\
12.9129129129129	0\\
12.9279279279279	0\\
12.9429429429429	0\\
12.957957957958	0\\
12.972972972973	0\\
12.987987987988	0\\
13.003003003003	0\\
13.018018018018	0\\
13.033033033033	0\\
13.048048048048	0\\
13.0630630630631	0\\
13.0780780780781	0\\
13.0930930930931	0\\
13.1081081081081	0\\
13.1231231231231	0\\
13.1381381381381	0\\
13.1531531531532	0\\
13.1681681681682	0\\
13.1831831831832	0\\
13.1981981981982	0\\
13.2132132132132	0\\
13.2282282282282	0\\
13.2432432432432	0\\
13.2582582582583	0\\
13.2732732732733	0\\
13.2882882882883	0\\
13.3033033033033	0\\
13.3183183183183	0\\
13.3333333333333	0\\
13.3483483483483	0\\
13.3633633633634	0\\
13.3783783783784	0\\
13.3933933933934	0\\
13.4084084084084	0\\
13.4234234234234	0\\
13.4384384384384	0\\
13.4534534534535	0\\
13.4684684684685	0\\
13.4834834834835	0\\
13.4984984984985	0\\
13.5135135135135	0\\
13.5285285285285	0\\
13.5435435435435	0\\
13.5585585585586	0\\
13.5735735735736	0\\
13.5885885885886	0\\
13.6036036036036	0\\
13.6186186186186	0\\
13.6336336336336	0\\
13.6486486486486	0\\
13.6636636636637	0\\
13.6786786786787	0\\
13.6936936936937	0\\
13.7087087087087	0\\
13.7237237237237	0\\
13.7387387387387	0\\
13.7537537537538	0\\
13.7687687687688	0\\
13.7837837837838	0\\
13.7987987987988	0\\
13.8138138138138	0\\
13.8288288288288	0\\
13.8438438438438	0\\
13.8588588588589	0\\
13.8738738738739	0\\
13.8888888888889	0\\
13.9039039039039	0\\
13.9189189189189	0\\
13.9339339339339	0\\
13.9489489489489	0\\
13.963963963964	0\\
13.978978978979	0\\
13.993993993994	0\\
14.009009009009	0\\
14.024024024024	0\\
14.039039039039	0\\
14.0540540540541	0\\
14.0690690690691	0\\
14.0840840840841	0\\
14.0990990990991	0\\
14.1141141141141	0\\
14.1291291291291	0\\
14.1441441441441	0\\
14.1591591591592	0\\
14.1741741741742	0\\
14.1891891891892	0\\
14.2042042042042	0\\
14.2192192192192	0\\
14.2342342342342	0\\
14.2492492492492	0\\
14.2642642642643	0\\
14.2792792792793	0\\
14.2942942942943	0\\
14.3093093093093	0\\
14.3243243243243	0\\
14.3393393393393	0\\
14.3543543543544	0\\
14.3693693693694	0\\
14.3843843843844	0\\
14.3993993993994	0\\
14.4144144144144	0\\
14.4294294294294	0\\
14.4444444444444	0\\
14.4594594594595	0\\
14.4744744744745	0\\
14.4894894894895	0\\
14.5045045045045	0\\
14.5195195195195	0\\
14.5345345345345	0\\
14.5495495495495	0\\
14.5645645645646	0\\
14.5795795795796	0\\
14.5945945945946	0\\
14.6096096096096	0\\
14.6246246246246	0\\
14.6396396396396	0\\
14.6546546546547	0\\
14.6696696696697	0\\
14.6846846846847	0\\
14.6996996996997	0\\
14.7147147147147	0\\
14.7297297297297	0\\
14.7447447447447	0\\
14.7597597597598	0\\
14.7747747747748	0\\
14.7897897897898	0\\
14.8048048048048	0\\
14.8198198198198	0\\
14.8348348348348	0\\
14.8498498498498	0\\
14.8648648648649	0\\
14.8798798798799	0\\
14.8948948948949	0\\
14.9099099099099	0\\
14.9249249249249	0\\
14.9399399399399	0\\
14.954954954955	0\\
14.96996996997	0\\
14.984984984985	0\\
15	0\\
};
\addlegendentry{$d = 5$};

\addplot [color=mycolor3,solid]
  table[row sep=crcr]{%
0	1.83660490682502e-21\\
0.015015015015015	5.59847653974567e-21\\
0.03003003003003	1.67973068705988e-20\\
0.045045045045045	4.96049620386203e-20\\
0.0600600600600601	1.44187050581876e-19\\
0.0750750750750751	4.12518154550441e-19\\
0.0900900900900901	1.16165066566608e-18\\
0.105105105105105	3.21976145109659e-18\\
0.12012012012012	8.78390274559503e-18\\
0.135135135135135	2.35866907147251e-17\\
0.15015015015015	6.23393367496006e-17\\
0.165165165165165	1.6217093244335e-16\\
0.18018018018018	4.15240324407519e-16\\
0.195195195195195	1.04650594766837e-15\\
0.21021021021021	2.59596929658514e-15\\
0.225225225225225	6.33830412483678e-15\\
0.24024024024024	1.52321871557145e-14\\
0.255255255255255	3.60302380993377e-14\\
0.27027027027027	8.38856489039964e-14\\
0.285285285285285	1.92231199596544e-13\\
0.3003003003003	4.33586503538605e-13\\
0.315315315315315	9.6259434455198e-13\\
0.33033033033033	2.103422560667e-12\\
0.345345345345345	4.52402942589417e-12\\
0.36036036036036	9.57723151216509e-12\\
0.375375375375375	1.99558495923645e-11\\
0.39039039039039	4.0927585983172e-11\\
0.405405405405405	8.26185766227192e-11\\
0.42042042042042	1.64155318824288e-10\\
0.435435435435435	3.21031672045708e-10\\
0.45045045045045	6.17954464133394e-10\\
0.465465465465465	1.17079467514525e-09\\
0.48048048048048	2.18333649710696e-09\\
0.495495495495495	4.00752538518743e-09\\
0.510510510510511	7.24014994904508e-09\\
0.525525525525526	1.28746227154016e-08\\
0.540540540540541	2.2533939956313e-08\\
0.555555555555556	3.88199905946607e-08\\
0.570570570570571	6.58247760630969e-08\\
0.585585585585586	1.09859856191171e-07\\
0.600600600600601	2.21619364320323e-07\\
0.615615615615616	3.61454701082752e-07\\
0.630630630630631	6.09572694489941e-07\\
0.645645645645646	9.67562799731866e-07\\
0.660660660660661	1.51258695462221e-06\\
0.675675675675676	2.37674021184837e-06\\
0.690690690690691	3.61245339690677e-06\\
0.705705705705706	5.50405381206395e-06\\
0.720720720720721	8.33619520266552e-06\\
0.735735735735736	1.21815810784673e-05\\
0.750750750750751	1.76607635353824e-05\\
0.765765765765766	2.52233744615862e-05\\
0.780780780780781	3.59459029575102e-05\\
0.795795795795796	5.00981610376274e-05\\
0.810810810810811	6.88025549260008e-05\\
0.825825825825826	9.35775762590257e-05\\
0.840840840840841	0.000125616372604196\\
0.855855855855856	0.000166366184082222\\
0.870870870870871	0.000217787591252982\\
0.885885885885886	0.000282259358110959\\
0.900900900900901	0.000361875784036803\\
0.915915915915916	0.000459444519629002\\
0.930930930930931	0.000576762475548234\\
0.945945945945946	0.000717455883594449\\
0.960960960960961	0.000883966525828529\\
0.975975975975976	0.00107932563264689\\
0.990990990990991	0.00130730595509253\\
1.00600600600601	0.0015701658395368\\
1.02102102102102	0.0018709286397402\\
1.03603603603604	0.00221201442335006\\
1.05105105105105	0.00259705841747253\\
1.06606606606607	0.00302722765050214\\
1.08108108108108	0.00350483402973396\\
1.0960960960961	0.00403115720965692\\
1.11111111111111	0.0046072426881343\\
1.12612612612613	0.00523349532860124\\
1.14114114114114	0.00591024953653559\\
1.15615615615616	0.00663660810500989\\
1.17117117117117	0.00741102752023488\\
1.18618618618619	0.00823353992964614\\
1.2012012012012	0.00910249942931327\\
1.21621621621622	0.0100158672993092\\
1.23123123123123	0.0109727570706065\\
1.24624624624625	0.0119720869780121\\
1.26126126126126	0.013013609748589\\
1.27627627627628	0.014096895340197\\
1.29129129129129	0.0152219051510289\\
1.30630630630631	0.0163915036082572\\
1.32132132132132	0.0176046831498142\\
1.33633633633634	0.0188653581983378\\
1.35135135135135	0.0201760299785817\\
1.36636636636637	0.0215382390620751\\
1.38138138138138	0.0229563493691191\\
1.3963963963964	0.0244319498289476\\
1.41141141141141	0.025968853517823\\
1.42642642642643	0.0275675033204049\\
1.44144144144144	0.029229745094628\\
1.45645645645646	0.0309565230862953\\
1.47147147147147	0.0327493490850413\\
1.48648648648649	0.034609168131666\\
1.5015015015015	0.0365378159552574\\
1.51651651651652	0.0385362861667578\\
1.53153153153153	0.0406083723853594\\
1.54654654654655	0.0427604209613733\\
1.56156156156156	0.0449975080415125\\
1.57657657657658	0.0473291822222387\\
1.59159159159159	0.0497640823919709\\
1.60660660660661	0.0523152299408161\\
1.62162162162162	0.0549991421853826\\
1.63663663663664	0.0578307628014611\\
1.65165165165165	0.0608277368565106\\
1.66666666666667	0.0640072727207516\\
1.68168168168168	0.0673864979217986\\
1.6966966966967	0.0709800266694294\\
1.71171171171171	0.0748037685394822\\
1.72672672672673	0.078867530556001\\
1.74174174174174	0.0831818053334236\\
1.75675675675676	0.0877464730016565\\
1.77177177177177	0.0925645434804983\\
1.78678678678679	0.097631583293668\\
1.8018018018018	0.102939997224059\\
1.81681681681682	0.108479629927332\\
1.83183183183183	0.114237418730026\\
1.84684684684685	0.120196503126669\\
1.86186186186186	0.12634039605932\\
1.87687687687688	0.132651668695868\\
1.89189189189189	0.139113050382535\\
1.90690690690691	0.145705943292069\\
1.92192192192192	0.152415046980619\\
1.93693693693694	0.159222094734371\\
1.95195195195195	0.166114099389945\\
1.96696696696697	0.17307902910908\\
1.98198198198198	0.18010261856611\\
1.996996996997	0.187175549949499\\
2.01201201201201	0.194290751690528\\
2.02702702702703	0.201444116189268\\
2.04204204204204	0.208627502911203\\
2.05705705705706	0.215839477350328\\
2.07207207207207	0.22307871711102\\
2.08708708708709	0.230346637071772\\
2.1021021021021	0.237640287978791\\
2.11711711711712	0.244958514455535\\
2.13213213213213	0.252302068211445\\
2.14714714714715	0.259669786295154\\
2.16216216216216	0.267063374375772\\
2.17717717717718	0.274477692271892\\
2.19219219219219	0.28190845379569\\
2.20720720720721	0.289356267859436\\
2.22222222222222	0.296818387752609\\
2.23723723723724	0.304295046445456\\
2.25225225225225	0.311779363500648\\
2.26726726726727	0.319272045085097\\
2.28228228228228	0.326769372750727\\
2.2972972972973	0.334270420004223\\
2.31231231231231	0.341770122523219\\
2.32732732732733	0.349264599873184\\
2.34234234234234	0.356747812372093\\
2.35735735735736	0.364209713286281\\
2.37237237237237	0.371642662909588\\
2.38738738738739	0.379035005302854\\
2.4024024024024	0.386372273039102\\
2.41741741741742	0.393638812573788\\
2.43243243243243	0.400814535288019\\
2.44744744744745	0.407876619250822\\
2.46246246246246	0.414803049233868\\
2.47747747747748	0.421569054976764\\
2.49249249249249	0.428149720248752\\
2.50750750750751	0.434516690851176\\
2.52252252252252	0.440653838399423\\
2.53753753753754	0.446542451149041\\
2.55255255255255	0.452168183904692\\
2.56756756756757	0.457526944068853\\
2.58258258258258	0.462624804520712\\
2.5975975975976	0.467475158234189\\
2.61261261261261	0.47210208288287\\
2.62762762762763	0.476537589432081\\
2.64264264264264	0.480818669655302\\
2.65765765765766	0.484987045724165\\
2.67267267267267	0.489088281651003\\
2.68768768768769	0.493162505140205\\
2.7027027027027	0.497246385555158\\
2.71771771771772	0.501368435454237\\
2.73273273273273	0.505552102963799\\
2.74774774774775	0.509804722444162\\
2.76276276276276	0.514130583305584\\
2.77777777777778	0.518518919452406\\
2.79279279279279	0.522951887894593\\
2.80780780780781	0.527402885241731\\
2.82282282282282	0.531840844417658\\
2.83783783783784	0.536227878638793\\
2.85285285285285	0.540527203138784\\
2.86786786786787	0.544694245990309\\
2.88288288288288	0.548686530853626\\
2.8978978978979	0.552464895118985\\
2.91291291291291	0.555984777119251\\
2.92792792792793	0.559204582130549\\
2.94294294294294	0.562084372501723\\
2.95795795795796	0.564576866974741\\
2.97297297297297	0.566646879371576\\
2.98798798798799	0.56825103309112\\
3.003003003003	0.569356215173806\\
3.01801801801802	0.569928717047999\\
3.03303303303303	0.569936057908425\\
3.04804804804805	0.569363255252982\\
3.06306306306306	0.568202552781548\\
3.07807807807808	0.566455702729863\\
3.09309309309309	0.564137288510063\\
3.10810810810811	0.561277053132017\\
3.12312312312312	0.557929118244867\\
3.13813813813814	0.554151280697191\\
3.15315315315315	0.550018505261974\\
3.16816816816817	0.545612304285092\\
3.18318318318318	0.541019116097155\\
3.1981981981982	0.536326856955038\\
3.21321321321321	0.531615913389528\\
3.22822822822823	0.52695980904685\\
3.24324324324324	0.522413333566152\\
3.25825825825826	0.518017448438315\\
3.27327327327327	0.513794674224588\\
3.28828828828829	0.509751951362865\\
3.3033033033033	0.505875084586645\\
3.31831831831832	0.502142138236293\\
3.33333333333333	0.49851495112148\\
3.34834834834835	0.494954397403401\\
3.36336336336336	0.491415077330806\\
3.37837837837838	0.487853337124164\\
3.39339339339339	0.484233373232537\\
3.40840840840841	0.480518895116878\\
3.42342342342342	0.47668498101012\\
3.43843843843844	0.472708735008465\\
3.45345345345345	0.468573319336378\\
3.46846846846847	0.464268971257926\\
3.48348348348348	0.459784920355622\\
3.4984984984985	0.455113386736624\\
3.51351351351351	0.450249261224646\\
3.52852852852853	0.445180200841069\\
3.54354354354354	0.439900189494167\\
3.55855855855856	0.434404937613117\\
3.57357357357357	0.428682272797951\\
3.58858858858859	0.42272729736988\\
3.6036036036036	0.416540743946475\\
3.61861861861862	0.410116524270624\\
3.63363363363363	0.403464452568875\\
3.64864864864865	0.396593865837565\\
3.66366366366366	0.389520118591314\\
3.67867867867868	0.382258110357586\\
3.69369369369369	0.374830184212498\\
3.70870870870871	0.367258947636302\\
3.72372372372372	0.359569335315644\\
3.73873873873874	0.35178284204952\\
3.75375375375375	0.343924654608432\\
3.76876876876877	0.336018754610837\\
3.78378378378378	0.328080816399763\\
3.7987987987988	0.320132235656588\\
3.81381381381381	0.312194965664536\\
3.82882882882883	0.304282847681469\\
3.84384384384384	0.296417606647672\\
3.85885885885886	0.288615782309863\\
3.87387387387387	0.280899020397904\\
3.88888888888889	0.27328452565944\\
3.9039039039039	0.265795748629874\\
3.91891891891892	0.258445335926361\\
3.93393393393393	0.251251158411377\\
3.94894894894895	0.244226154587861\\
3.96396396396396	0.237376787526564\\
3.97897897897898	0.230710872754374\\
3.99399399399399	0.224231523963807\\
4.00900900900901	0.217935961812366\\
4.02402402402402	0.21182054657427\\
4.03903903903904	0.205882129319605\\
4.05405405405405	0.200119201719791\\
4.06906906906907	0.194525869543087\\
4.08408408408408	0.189102883412274\\
4.0990990990991	0.183850021816763\\
4.11411411411411	0.17876916863423\\
4.12912912912913	0.173863840910697\\
4.14414414414414	0.16913693324565\\
4.15915915915916	0.164590899029673\\
4.17417417417417	0.160226671990917\\
4.18918918918919	0.156042175152944\\
4.2042042042042	0.152032054645065\\
4.21921921921922	0.148189034557672\\
4.23423423423423	0.144502087775464\\
4.24924924924925	0.140957278186425\\
4.26426426426426	0.137542232040575\\
4.27927927927928	0.134241521402375\\
4.29429429429429	0.131040030545949\\
4.30930930930931	0.127922426920941\\
4.32432432432432	0.124875650535057\\
4.33933933933934	0.121891368810598\\
4.35435435435435	0.118955930995973\\
4.36936936936937	0.116059353037382\\
4.38438438438438	0.113193015157749\\
4.3993993993994	0.110348928817787\\
4.41441441441441	0.107518582256127\\
4.42942942942943	0.104693277287821\\
4.44444444444444	0.10186570316599\\
4.45945945945946	0.099031484243828\\
4.47447447447447	0.096181309569995\\
4.48948948948949	0.0933124054177231\\
4.5045045045045	0.0904209646683462\\
4.51951951951952	0.0875088020804437\\
4.53453453453453	0.0845806872670909\\
4.54954954954955	0.0816403762617345\\
4.56456456456456	0.0786962603784789\\
4.57957957957958	0.0757595320800026\\
4.59459459459459	0.0728425011125374\\
4.60960960960961	0.0699562626683384\\
4.62462462462462	0.0671155436422283\\
4.63963963963964	0.0643333518143909\\
4.65465465465465	0.0616208148246652\\
4.66966966966967	0.0589899480906223\\
4.68468468468468	0.0564484122628907\\
4.6996996996997	0.054003851796152\\
4.71471471471471	0.0516606744263053\\
4.72972972972973	0.0494212319848925\\
4.74474474474474	0.0472874054006181\\
4.75975975975976	0.045256493950253\\
4.77477477477477	0.0433271233042389\\
4.78978978978979	0.0414927572878052\\
4.8048048048048	0.0397478265787918\\
4.81981981981982	0.0380851352741066\\
4.83483483483483	0.0364985472352586\\
4.84984984984985	0.0349769712827718\\
4.86486486486486	0.0335127128738406\\
4.87987987987988	0.0320980167489514\\
4.89489489489489	0.0307236900827738\\
4.90990990990991	0.0293824546592703\\
4.92492492492492	0.0280676839559009\\
4.93993993993994	0.0267714684829109\\
4.95495495495495	0.0254910728833459\\
4.96996996996997	0.0242228983999669\\
4.98498498498498	0.0229655473667782\\
5	0.0217176046950515\\
5.01501501501502	0.0204813165457979\\
5.03003003003003	0.0192582355558421\\
5.04504504504505	0.018051778922597\\
5.06006006006006	0.016867675989691\\
5.07507507507508	0.0157111764970654\\
5.09009009009009	0.0145886517171127\\
5.10510510510511	0.0135059482119349\\
5.12012012012012	0.0124700383721636\\
5.13513513513514	0.011487535536038\\
5.15015015015015	0.0105647415788068\\
5.16516516516517	0.00970716330494282\\
5.18018018018018	0.00891758944565534\\
5.1951951951952	0.00820001545123376\\
5.21021021021021	0.00755586437344466\\
5.22522522522523	0.00698486615183666\\
5.24024024024024	0.0064860585640037\\
5.25525525525526	0.00605710756004013\\
5.27027027027027	0.00569254560395072\\
5.28528528528529	0.00538664433440177\\
5.3003003003003	0.00513287993347552\\
5.31531531531532	0.00492315267957159\\
5.33033033033033	0.00474998500458245\\
5.34534534534535	0.00460386008500795\\
5.36036036036036	0.00447687592231513\\
5.37537537537538	0.00436070235650004\\
5.39039039039039	0.00424812817957657\\
5.40540540540541	0.00413259487095901\\
5.42042042042042	0.00400851634656874\\
5.43543543543544	0.00387304695723361\\
5.45045045045045	0.00372345377711397\\
5.46546546546547	0.00355888492208503\\
5.48048048048048	0.00337918644491926\\
5.4954954954955	0.00318572909440018\\
5.51051051051051	0.00298096480998999\\
5.52552552552553	0.00276775907919788\\
5.54054054054054	0.00254926298729473\\
5.55555555555556	0.00232947133491121\\
5.57057057057057	0.00211175001693473\\
5.58558558558559	0.00189877645347572\\
5.6006006006006	0.00169385282423129\\
5.61561561561562	0.00149884263205684\\
5.63063063063063	0.00131600069137658\\
5.64564564564565	0.00114632244211187\\
5.66066066066066	0.00099075668006078\\
5.67567567567568	0.000849659389752182\\
5.69069069069069	0.000722971017316228\\
5.70570570570571	0.000610194437263504\\
5.72072072072072	0.000510746969360792\\
5.73573573573574	0.000423864184869723\\
5.75075075075075	0.000348659027593087\\
5.76576576576577	0.000284082103875023\\
5.78078078078078	0.000229245755750531\\
5.7957957957958	0.000183182746952178\\
5.81081081081081	0.000144826785268769\\
5.82582582582583	0.000113354179367448\\
5.84084084084084	8.77201432176669e-05\\
5.85585585585586	6.70037704827805e-05\\
5.87087087087087	5.06428176172391e-05\\
5.88588588588589	3.77262241439068e-05\\
5.9009009009009	2.77295438325161e-05\\
5.91591591591592	1.99181122571967e-05\\
5.93093093093093	1.43329596195286e-05\\
5.94594594594595	1.01681940812314e-05\\
5.96096096096096	7.00104816635633e-06\\
5.97597597597598	4.83500765663807e-06\\
5.99099099099099	3.21650991994103e-06\\
6.00600600600601	2.08716691050383e-06\\
6.02102102102102	1.38637087780141e-06\\
6.03603603603604	9.06394672961429e-07\\
6.05105105105105	5.83271737705329e-07\\
6.06606606606607	3.69436833832702e-07\\
6.08108108108108	2.30316538757513e-07\\
6.0960960960961	1.4132719141609e-07\\
6.11111111111111	8.53575621286693e-08\\
6.12612612612613	5.07427440016593e-08\\
6.14114114114114	2.96907772009656e-08\\
6.15615615615616	1.70995568249847e-08\\
6.17117117117117	9.69312503546167e-09\\
6.18618618618619	5.40827101434338e-09\\
6.2012012012012	2.9700841174498e-09\\
6.21621621621622	1.60544238213091e-09\\
6.23123123123123	8.54154374634595e-10\\
6.24624624624625	4.47294632474827e-10\\
6.26126126126126	2.30550822147677e-10\\
6.27627627627628	1.16964841123911e-10\\
6.29129129129129	5.84062962067037e-11\\
6.30630630630631	2.87064619578898e-11\\
6.32132132132132	1.38872208968304e-11\\
6.33633633633634	6.6125152902823e-12\\
6.35135135135135	3.09908655783231e-12\\
6.36636636636637	1.42960610704717e-12\\
6.38138138138138	6.49104701232475e-13\\
6.3963963963964	2.90087346752827e-13\\
6.41141141141141	1.276022958829e-13\\
6.42642642642643	5.52463846138078e-14\\
6.44144144144144	2.35431683445904e-14\\
6.45645645645646	9.87510279064658e-15\\
6.47147147147147	4.07693711987012e-15\\
6.48648648648649	1.65669313613486e-15\\
6.5015015015015	6.6262192835607e-16\\
6.51651651651652	2.60858628475373e-16\\
6.53153153153153	1.01078847271041e-16\\
6.54654654654655	3.8550589705219e-17\\
6.56156156156156	1.44716297426027e-17\\
6.57657657657658	5.34711507581589e-18\\
6.59159159159159	1.9446313830837e-18\\
6.60660660660661	6.96098537358077e-19\\
6.62162162162162	2.45256103193913e-19\\
6.63663663663664	8.50520130948866e-20\\
6.65165165165165	2.90312036682711e-20\\
6.66666666666667	9.75351506645871e-21\\
6.68168168168168	3.22532102159087e-21\\
6.6966966966967	1.04978510537834e-21\\
6.71171171171171	3.36312908898152e-22\\
6.72672672672673	1.06047965141793e-22\\
6.74174174174174	3.29137081055549e-23\\
6.75675675675676	1.00546500192407e-23\\
6.77177177177177	3.02324104257327e-24\\
6.78678678678679	8.94734662890445e-25\\
6.8018018018018	2.60634211639047e-25\\
6.81681681681682	7.47281537497658e-26\\
6.83183183183183	2.10888430517487e-26\\
6.84684684684685	5.85783221645019e-27\\
6.86186186186186	1.60153624981115e-27\\
6.87687687687688	4.3097520947143e-28\\
6.89189189189189	1.14151987434099e-28\\
6.90690690690691	2.97598224796972e-29\\
6.92192192192192	7.63647381295382e-30\\
6.93693693693694	1.92872837051343e-30\\
6.95195195195195	4.79473867579927e-31\\
6.96696696696697	1.17320657133365e-31\\
6.98198198198198	2.8255285704646e-32\\
6.996996996997	6.69792978723181e-33\\
7.01201201201201	1.56277749907884e-33\\
7.02702702702703	3.58896629696255e-34\\
7.04204204204204	8.11254804443422e-35\\
7.05705705705706	1.80493204289824e-35\\
7.07207207207207	3.95257482011515e-36\\
7.08708708708709	8.51951782033572e-37\\
7.1021021021021	1.80744713245825e-37\\
7.11711711711712	3.77426041328197e-38\\
7.13213213213213	7.757357244868e-39\\
7.14714714714715	1.56931958182012e-39\\
7.16216216216216	3.12481756327036e-40\\
7.17717717717718	6.12425987886331e-41\\
7.19219219219219	1.18140342966185e-41\\
7.20720720720721	2.24315102607406e-42\\
7.22222222222222	4.19212740751629e-43\\
7.23723723723724	7.71127344039409e-44\\
7.25225225225225	1.39615425718211e-44\\
7.26726726726727	2.48803438370787e-45\\
7.28228228228228	4.36410330228754e-46\\
7.2972972972973	7.53441153382048e-47\\
7.31231231231231	1.28032234481115e-47\\
7.32732732732733	2.14143542923288e-48\\
7.34234234234234	3.52538306222107e-49\\
7.35735735735736	5.712461879921e-50\\
7.37237237237237	9.11078897876037e-51\\
7.38738738738739	1.43022489071752e-51\\
7.4024024024024	2.20987825013523e-52\\
7.41741741741742	3.36084152544271e-53\\
7.43243243243243	5.0308734506078e-54\\
7.44744744744745	7.41232448301147e-55\\
7.46246246246246	1.0749323274669e-55\\
7.47747747747748	1.53434667551868e-56\\
7.49249249249249	2.15566634153799e-57\\
7.50750750750751	2.98095387526162e-58\\
7.52252252252252	4.05737027393124e-59\\
7.53753753753754	5.43562767719974e-60\\
7.55255255255255	7.16754504529521e-61\\
7.56756756756757	9.30265322562472e-62\\
7.58258258258258	1.18838976993027e-62\\
7.5975975975976	1.4942616431337e-63\\
7.61261261261261	1.84931147217361e-64\\
7.62762762762763	2.25273004083452e-65\\
7.64264264264264	2.70099579873747e-66\\
7.65765765765766	3.18753042058286e-67\\
7.67267267267267	3.70254579794871e-68\\
7.68768768768769	4.23313590272372e-69\\
7.7027027027027	4.76364791854137e-70\\
7.71771771771772	5.27633996760923e-71\\
7.73273273273273	5.75230038772785e-72\\
7.74774774774775	6.17256987582139e-73\\
7.76276276276276	6.51937772695178e-74\\
7.77777777777778	6.77738169302081e-75\\
7.79279279279279	6.93479164729557e-76\\
7.80780780780781	6.98426261737808e-77\\
7.82282282282282	6.92346300732091e-78\\
7.83783783783784	6.75525678759414e-79\\
7.85285285285285	6.48747982620234e-80\\
7.86786786786787	6.13233468850659e-81\\
7.88288288288288	5.70546900323294e-82\\
7.8978978978979	5.22483430625676e-83\\
7.91291291291291	4.70944105858158e-84\\
7.92792792792793	4.1781293461657e-85\\
7.94294294294294	3.64846401380717e-86\\
7.95795795795796	3.1358402103682e-87\\
7.97297297297297	2.65285463892298e-88\\
7.98798798798799	2.20896408239171e-89\\
8.003003003003	1.81042078188221e-90\\
8.01801801801802	1.46044790343573e-91\\
8.03303303303303	1.15960020733725e-92\\
8.04804804804805	9.06246141474274e-94\\
8.06306306306306	6.9710745637569e-95\\
8.07807807807808	5.27799478792933e-96\\
8.09309309309309	3.93327088517756e-97\\
8.10810810810811	2.88505720761449e-98\\
8.12312312312312	2.08291080261703e-99\\
8.13813813813814	1.48013928975962e-100\\
8.15315315315315	1.03526176102328e-101\\
8.16816816816817	7.1271094852335e-103\\
8.18318318318318	4.82939089757194e-104\\
8.1981981981982	3.22097207379346e-105\\
8.21321321321321	2.11444908063648e-106\\
8.22822822822823	1.36622797626601e-107\\
8.24324324324324	8.68889979226483e-109\\
8.25825825825826	5.43903776742945e-110\\
8.27327327327327	3.3511589827382e-111\\
8.28828828828829	2.03228054846705e-112\\
8.3033033033033	1.21307586023714e-113\\
8.31831831831832	7.12701925310997e-115\\
8.33333333333333	4.12138868959869e-116\\
8.34834834834835	2.34582110492493e-117\\
8.36336336336336	1.31420124704897e-118\\
8.37837837837838	7.24677090625354e-120\\
8.39339339339339	3.93317129216896e-121\\
8.40840840840841	2.10114904693765e-122\\
8.42342342342342	1.10480729175104e-123\\
8.43843843843844	5.71783836995783e-125\\
8.45345345345345	2.91268077225584e-126\\
8.46846846846847	1.46039243453315e-127\\
8.48348348348348	7.20712295769442e-129\\
8.4984984984985	3.50082143814522e-130\\
8.51351351351351	1.67376183020928e-131\\
8.52852852852853	7.87649552959128e-133\\
8.54354354354354	3.64827925312353e-134\\
8.55855855855856	1.66325480785271e-135\\
8.57357357357357	7.4635452970587e-137\\
8.58858858858859	3.29645559382325e-138\\
8.6036036036036	1.43306192714113e-139\\
8.61861861861862	6.13194650899739e-141\\
8.63363363363363	2.58254221154271e-142\\
8.64864864864865	1.07056285630737e-143\\
8.66366366366366	4.36810012992606e-145\\
8.67867867867868	1.75423863032469e-146\\
8.69369369369369	6.93426488016113e-148\\
8.70870870870871	2.69791246852414e-149\\
8.72372372372372	1.03316801981867e-150\\
8.73873873873874	3.89430287891747e-152\\
8.75375375375375	1.4447881403049e-153\\
8.76876876876877	5.27587238100059e-155\\
8.78378378378378	1.89626953104099e-156\\
8.7987987987988	6.70843974777657e-158\\
8.81381381381381	2.33592369175188e-159\\
8.82882882882883	8.00592377491832e-161\\
8.84384384384384	2.7007222451589e-162\\
8.85885885885886	8.96734882303718e-164\\
8.87387387387387	2.93064947737599e-165\\
8.88888888888889	9.42712709964426e-167\\
8.9039039039039	2.98476758006305e-168\\
8.91891891891892	9.30159330071913e-170\\
8.93393393393393	2.85311873714782e-171\\
8.94894894894895	8.61386411742322e-173\\
8.96396396396396	2.55971644497848e-174\\
8.97897897897898	7.48688861538298e-176\\
8.99399399399399	2.15539344090243e-177\\
9.00900900900901	6.10755537412501e-179\\
9.02402402402402	1.70342862449356e-180\\
9.03903903903904	4.67623297409045e-182\\
9.05405405405405	1.26352549811604e-183\\
9.06906906906907	3.36037350914472e-185\\
9.08408408408408	8.79643637332003e-187\\
9.0990990990991	2.26642655698529e-188\\
9.11411411411411	5.74767504496719e-190\\
9.12912912912913	1.43469119872206e-191\\
9.14414414414414	3.52484750600088e-193\\
9.15915915915916	8.5238913118994e-195\\
9.17417417417417	2.02885563325642e-196\\
9.18918918918919	4.75313400287945e-198\\
9.2042042042042	1.09603555544242e-199\\
9.21921921921922	2.48762492087286e-201\\
9.23423423423423	5.55726141124476e-203\\
9.24924924924925	1.22194715648434e-204\\
9.26426426426426	2.64459818362199e-206\\
9.27927927927928	5.63355627482844e-208\\
9.29429429429429	1.18119404817828e-209\\
9.30930930930931	2.43767349032966e-211\\
9.32432432432432	4.95159927015922e-213\\
9.33933933933934	9.89990706074962e-215\\
9.35435435435435	1.94819496306544e-216\\
9.36936936936937	3.7735437215711e-218\\
9.38438438438438	7.19419205876521e-220\\
9.3993993993994	1.34998934448987e-221\\
9.41441441441441	2.49341360547075e-223\\
9.42942942942943	4.53287748018189e-225\\
9.44444444444444	8.11090486959624e-227\\
9.45945945945946	1.42850020235626e-228\\
9.47447447447447	2.47632120264605e-230\\
9.48948948948949	4.22521990909966e-232\\
9.5045045045045	7.09589744688248e-234\\
9.51951951951952	1.17295409122379e-235\\
9.53453453453453	1.90840422760954e-237\\
9.54954954954955	3.05615514226226e-239\\
9.56456456456456	4.81721572437465e-241\\
9.57957957957958	7.47364500322386e-243\\
9.59459459459459	1.14125976437313e-244\\
9.60960960960961	1.71534791464722e-246\\
9.62462462462462	2.53767267246441e-248\\
9.63963963963964	3.69517199927379e-250\\
9.65465465465465	5.29601717857038e-252\\
9.66966966966967	7.47101801065046e-254\\
9.68468468468468	1.03735129399699e-255\\
9.6996996996997	1.41771062464469e-257\\
9.71471471471471	1.90706285198113e-259\\
9.72972972972973	2.52498085997524e-261\\
9.74474474474474	3.29053750874807e-263\\
9.75975975975976	4.22076596985837e-265\\
9.77477477477477	5.32882392415335e-267\\
9.78978978978979	6.62196894862309e-269\\
9.8048048048048	8.09950692110845e-271\\
9.81981981981982	9.75092224137417e-273\\
9.83483483483483	1.15544286977781e-274\\
9.84984984984985	1.347618399842e-276\\
9.86486486486486	1.54703827478002e-278\\
9.87987987987988	1.74803794824784e-280\\
9.89489489489489	1.94408981883352e-282\\
9.90990990990991	2.12812663201887e-284\\
9.92492492492492	2.29294838545418e-286\\
9.93993993993994	2.43168190130569e-288\\
9.95495495495495	2.53825305397166e-290\\
9.96996996996997	2.60782679781289e-292\\
9.98498498498498	2.63717069804563e-294\\
10	2.62490391615234e-296\\
10.015015015015	2.57160494041965e-298\\
10.03003003003	2.47976635604257e-300\\
10.045045045045	2.35360157984387e-302\\
10.0600600600601	2.19872442933496e-304\\
10.0750750750751	2.02173545807833e-306\\
10.0900900900901	1.82975745661518e-308\\
10.1051051051051	1.62996543452757e-310\\
10.1201201201201	1.42915369082777e-312\\
10.1351351351351	1.2333750169321e-314\\
10.1501501501502	1.04767588569302e-316\\
10.1651651651652	8.75958627450696e-319\\
10.1801801801802	7.20841777282379e-321\\
10.1951951951952	3.95252516672997e-323\\
10.2102102102102	0\\
10.2252252252252	0\\
10.2402402402402	0\\
10.2552552552553	0\\
10.2702702702703	0\\
10.2852852852853	0\\
10.3003003003003	0\\
10.3153153153153	0\\
10.3303303303303	0\\
10.3453453453453	0\\
10.3603603603604	0\\
10.3753753753754	0\\
10.3903903903904	0\\
10.4054054054054	0\\
10.4204204204204	0\\
10.4354354354354	0\\
10.4504504504505	0\\
10.4654654654655	0\\
10.4804804804805	0\\
10.4954954954955	0\\
10.5105105105105	0\\
10.5255255255255	0\\
10.5405405405405	0\\
10.5555555555556	0\\
10.5705705705706	0\\
10.5855855855856	0\\
10.6006006006006	0\\
10.6156156156156	0\\
10.6306306306306	0\\
10.6456456456456	0\\
10.6606606606607	0\\
10.6756756756757	0\\
10.6906906906907	0\\
10.7057057057057	0\\
10.7207207207207	0\\
10.7357357357357	0\\
10.7507507507508	0\\
10.7657657657658	0\\
10.7807807807808	0\\
10.7957957957958	0\\
10.8108108108108	0\\
10.8258258258258	0\\
10.8408408408408	0\\
10.8558558558559	0\\
10.8708708708709	0\\
10.8858858858859	0\\
10.9009009009009	0\\
10.9159159159159	0\\
10.9309309309309	0\\
10.9459459459459	0\\
10.960960960961	0\\
10.975975975976	0\\
10.990990990991	0\\
11.006006006006	0\\
11.021021021021	0\\
11.036036036036	0\\
11.0510510510511	0\\
11.0660660660661	0\\
11.0810810810811	0\\
11.0960960960961	0\\
11.1111111111111	0\\
11.1261261261261	0\\
11.1411411411411	0\\
11.1561561561562	0\\
11.1711711711712	0\\
11.1861861861862	0\\
11.2012012012012	0\\
11.2162162162162	0\\
11.2312312312312	0\\
11.2462462462462	0\\
11.2612612612613	0\\
11.2762762762763	0\\
11.2912912912913	0\\
11.3063063063063	0\\
11.3213213213213	0\\
11.3363363363363	0\\
11.3513513513514	0\\
11.3663663663664	0\\
11.3813813813814	0\\
11.3963963963964	0\\
11.4114114114114	0\\
11.4264264264264	0\\
11.4414414414414	0\\
11.4564564564565	0\\
11.4714714714715	0\\
11.4864864864865	0\\
11.5015015015015	0\\
11.5165165165165	0\\
11.5315315315315	0\\
11.5465465465465	0\\
11.5615615615616	0\\
11.5765765765766	0\\
11.5915915915916	0\\
11.6066066066066	0\\
11.6216216216216	0\\
11.6366366366366	0\\
11.6516516516517	0\\
11.6666666666667	0\\
11.6816816816817	0\\
11.6966966966967	0\\
11.7117117117117	0\\
11.7267267267267	0\\
11.7417417417417	0\\
11.7567567567568	0\\
11.7717717717718	0\\
11.7867867867868	0\\
11.8018018018018	0\\
11.8168168168168	0\\
11.8318318318318	0\\
11.8468468468468	0\\
11.8618618618619	0\\
11.8768768768769	0\\
11.8918918918919	0\\
11.9069069069069	0\\
11.9219219219219	0\\
11.9369369369369	0\\
11.951951951952	0\\
11.966966966967	0\\
11.981981981982	0\\
11.996996996997	0\\
12.012012012012	0\\
12.027027027027	0\\
12.042042042042	0\\
12.0570570570571	0\\
12.0720720720721	0\\
12.0870870870871	0\\
12.1021021021021	0\\
12.1171171171171	0\\
12.1321321321321	0\\
12.1471471471471	0\\
12.1621621621622	0\\
12.1771771771772	0\\
12.1921921921922	0\\
12.2072072072072	0\\
12.2222222222222	0\\
12.2372372372372	0\\
12.2522522522523	0\\
12.2672672672673	0\\
12.2822822822823	0\\
12.2972972972973	0\\
12.3123123123123	0\\
12.3273273273273	0\\
12.3423423423423	0\\
12.3573573573574	0\\
12.3723723723724	0\\
12.3873873873874	0\\
12.4024024024024	0\\
12.4174174174174	0\\
12.4324324324324	0\\
12.4474474474474	0\\
12.4624624624625	0\\
12.4774774774775	0\\
12.4924924924925	0\\
12.5075075075075	0\\
12.5225225225225	0\\
12.5375375375375	0\\
12.5525525525526	0\\
12.5675675675676	0\\
12.5825825825826	0\\
12.5975975975976	0\\
12.6126126126126	0\\
12.6276276276276	0\\
12.6426426426426	0\\
12.6576576576577	0\\
12.6726726726727	0\\
12.6876876876877	0\\
12.7027027027027	0\\
12.7177177177177	0\\
12.7327327327327	0\\
12.7477477477477	0\\
12.7627627627628	0\\
12.7777777777778	0\\
12.7927927927928	0\\
12.8078078078078	0\\
12.8228228228228	0\\
12.8378378378378	0\\
12.8528528528529	0\\
12.8678678678679	0\\
12.8828828828829	0\\
12.8978978978979	0\\
12.9129129129129	0\\
12.9279279279279	0\\
12.9429429429429	0\\
12.957957957958	0\\
12.972972972973	0\\
12.987987987988	0\\
13.003003003003	0\\
13.018018018018	0\\
13.033033033033	0\\
13.048048048048	0\\
13.0630630630631	0\\
13.0780780780781	0\\
13.0930930930931	0\\
13.1081081081081	0\\
13.1231231231231	0\\
13.1381381381381	0\\
13.1531531531532	0\\
13.1681681681682	0\\
13.1831831831832	0\\
13.1981981981982	0\\
13.2132132132132	0\\
13.2282282282282	0\\
13.2432432432432	0\\
13.2582582582583	0\\
13.2732732732733	0\\
13.2882882882883	0\\
13.3033033033033	0\\
13.3183183183183	0\\
13.3333333333333	0\\
13.3483483483483	0\\
13.3633633633634	0\\
13.3783783783784	0\\
13.3933933933934	0\\
13.4084084084084	0\\
13.4234234234234	0\\
13.4384384384384	0\\
13.4534534534535	0\\
13.4684684684685	0\\
13.4834834834835	0\\
13.4984984984985	0\\
13.5135135135135	0\\
13.5285285285285	0\\
13.5435435435435	0\\
13.5585585585586	0\\
13.5735735735736	0\\
13.5885885885886	0\\
13.6036036036036	0\\
13.6186186186186	0\\
13.6336336336336	0\\
13.6486486486486	0\\
13.6636636636637	0\\
13.6786786786787	0\\
13.6936936936937	0\\
13.7087087087087	0\\
13.7237237237237	0\\
13.7387387387387	0\\
13.7537537537538	0\\
13.7687687687688	0\\
13.7837837837838	0\\
13.7987987987988	0\\
13.8138138138138	0\\
13.8288288288288	0\\
13.8438438438438	0\\
13.8588588588589	0\\
13.8738738738739	0\\
13.8888888888889	0\\
13.9039039039039	0\\
13.9189189189189	0\\
13.9339339339339	0\\
13.9489489489489	0\\
13.963963963964	0\\
13.978978978979	0\\
13.993993993994	0\\
14.009009009009	0\\
14.024024024024	0\\
14.039039039039	0\\
14.0540540540541	0\\
14.0690690690691	0\\
14.0840840840841	0\\
14.0990990990991	0\\
14.1141141141141	0\\
14.1291291291291	0\\
14.1441441441441	0\\
14.1591591591592	0\\
14.1741741741742	0\\
14.1891891891892	0\\
14.2042042042042	0\\
14.2192192192192	0\\
14.2342342342342	0\\
14.2492492492492	0\\
14.2642642642643	0\\
14.2792792792793	0\\
14.2942942942943	0\\
14.3093093093093	0\\
14.3243243243243	0\\
14.3393393393393	0\\
14.3543543543544	0\\
14.3693693693694	0\\
14.3843843843844	0\\
14.3993993993994	0\\
14.4144144144144	0\\
14.4294294294294	0\\
14.4444444444444	0\\
14.4594594594595	0\\
14.4744744744745	0\\
14.4894894894895	0\\
14.5045045045045	0\\
14.5195195195195	0\\
14.5345345345345	0\\
14.5495495495495	0\\
14.5645645645646	0\\
14.5795795795796	0\\
14.5945945945946	0\\
14.6096096096096	0\\
14.6246246246246	0\\
14.6396396396396	0\\
14.6546546546547	0\\
14.6696696696697	0\\
14.6846846846847	0\\
14.6996996996997	0\\
14.7147147147147	0\\
14.7297297297297	0\\
14.7447447447447	0\\
14.7597597597598	0\\
14.7747747747748	0\\
14.7897897897898	0\\
14.8048048048048	0\\
14.8198198198198	0\\
14.8348348348348	0\\
14.8498498498498	0\\
14.8648648648649	0\\
14.8798798798799	0\\
14.8948948948949	0\\
14.9099099099099	0\\
14.9249249249249	0\\
14.9399399399399	0\\
14.954954954955	0\\
14.96996996997	0\\
14.984984984985	0\\
15	0\\
};
\addlegendentry{$d = 10$};

\addplot [color=mycolor4,solid]
  table[row sep=crcr]{%
0	0\\
0.015015015015015	0\\
0.03003003003003	0\\
0.045045045045045	0\\
0.0600600600600601	2.96439387504748e-322\\
0.0750750750750751	3.90262453650001e-320\\
0.0900900900900901	5.04065534513073e-318\\
0.105105105105105	6.40858176628387e-316\\
0.12012012012012	8.01603595260665e-314\\
0.135135135135135	9.86458547109408e-312\\
0.15015015015015	1.19431664799916e-309\\
0.165165165165165	1.42259603711091e-307\\
0.18018018018018	1.66711348853172e-305\\
0.195195195195195	1.92207441842708e-303\\
0.21021021021021	2.18020180141597e-301\\
0.225225225225225	2.43301424740578e-299\\
0.24024024024024	2.67124715390769e-297\\
0.255255255255255	2.88539287590315e-295\\
0.27027027027027	3.06631876153677e-293\\
0.285285285285285	3.20590838779211e-291\\
0.3003003003003	3.2976638062302e-289\\
0.315315315315315	3.3372067235754e-287\\
0.33033033033033	3.32262481702035e-285\\
0.345345345345345	3.25462504091136e-283\\
0.36036036036036	3.13647681065075e-281\\
0.375375375375375	2.97375142214298e-279\\
0.39039039039039	2.77388662754907e-277\\
0.405405405405405	2.54562374576642e-275\\
0.42042042042042	2.29837655221867e-273\\
0.435435435435435	2.0415950952803e-271\\
0.45045045045045	1.78418345341976e-269\\
0.465465465465465	1.53401944064589e-267\\
0.48048048048048	1.297608500593e-265\\
0.495495495495495	1.07988612904005e-263\\
0.510510510510511	8.84165787452404e-262\\
0.525525525525526	7.12214664572251e-260\\
0.540540540540541	5.64429338114324e-258\\
0.555555555555556	4.40078052415593e-256\\
0.570570570570571	3.37575797800934e-254\\
0.585585585585586	2.54761828829779e-252\\
0.600600600600601	1.89155457203566e-250\\
0.615615615615616	1.38173530749182e-248\\
0.630630630630631	9.93006971575602e-247\\
0.645645645645646	7.02103603793442e-245\\
0.660660660660661	4.88395404958526e-243\\
0.675675675675676	3.34243829668979e-241\\
0.690690690690691	2.2504879427432e-239\\
0.705705705705706	1.4907727467477e-237\\
0.720720720720721	9.71555604242695e-236\\
0.735735735735736	6.22938740181952e-234\\
0.750750750750751	3.92956504173936e-232\\
0.765765765765766	2.43873766228564e-230\\
0.780780780780781	1.48904269853031e-228\\
0.795795795795796	8.94480093535894e-227\\
0.810810810810811	5.28634704363142e-225\\
0.825825825825826	3.07370457828571e-223\\
0.840840840840841	1.75828811136022e-221\\
0.855855855855856	9.89553815429953e-220\\
0.870870870870871	5.47911300922595e-218\\
0.885885885885886	2.984712887706e-216\\
0.900900900900901	1.59961816031816e-214\\
0.915915915915916	8.43434872951019e-213\\
0.930930930930931	4.37530409969877e-211\\
0.945945945945946	2.23298803385393e-209\\
0.960960960960961	1.12120754203758e-207\\
0.975975975975976	5.53869037167456e-206\\
0.990990990990991	2.69184238424987e-204\\
1.00600600600601	1.28710396696088e-202\\
1.02102102102102	6.05479011267865e-201\\
1.03603603603604	2.80224446273847e-199\\
1.05105105105105	1.2759522105724e-197\\
1.06606606606607	5.715895233538e-196\\
1.08108108108108	2.51915896721549e-194\\
1.0960960960961	1.09231613772328e-192\\
1.11111111111111	4.65974969331304e-191\\
1.12612612612613	1.95568222860299e-189\\
1.14114114114114	8.07524025964154e-188\\
1.15615615615616	3.28045495172334e-186\\
1.17117117117117	1.31109501551956e-184\\
1.18618618618619	5.15532105452098e-183\\
1.2012012012012	1.99433784013313e-181\\
1.21621621621622	7.59037451289795e-180\\
1.23123123123123	2.84216405306989e-178\\
1.24624624624625	1.04702377222846e-176\\
1.26126126126126	3.79476952854333e-175\\
1.27627627627628	1.353118180676e-173\\
1.29129129129129	4.74687194920004e-172\\
1.30630630630631	1.63832766892995e-170\\
1.32132132132132	5.5630822084556e-169\\
1.33633633633634	1.85845329825237e-167\\
1.35135135135135	6.1081443041265e-166\\
1.36636636636637	1.97509681047534e-164\\
1.38138138138138	6.28331668434401e-163\\
1.3963963963964	1.96657707815392e-161\\
1.41141141141141	6.05556205557061e-160\\
1.42642642642643	1.8345071448922e-158\\
1.44144144144144	5.46771436070648e-157\\
1.45645645645646	1.60329593619986e-155\\
1.47147147147147	4.62533323773763e-154\\
1.48648648648649	1.31278566429497e-152\\
1.5015015015015	3.66577737124987e-151\\
1.51651651651652	1.00707034793531e-149\\
1.53153153153153	2.72191772717232e-148\\
1.54654654654655	7.23788438623501e-147\\
1.56156156156156	1.89351936778783e-145\\
1.57657657657658	4.8735937421648e-144\\
1.59159159159159	1.23410001831291e-142\\
1.60660660660661	3.0744883794124e-141\\
1.62162162162162	7.53558226611458e-140\\
1.63663663663664	1.8171142630463e-138\\
1.65165165165165	4.31091164285382e-137\\
1.66666666666667	1.0061841417781e-135\\
1.68168168168168	2.3105069396962e-134\\
1.6966966966967	5.21985631732626e-133\\
1.71171171171171	1.1601957827792e-131\\
1.72672672672673	2.53702927549578e-130\\
1.74174174174174	5.45809558849023e-129\\
1.75675675675676	1.15525606769912e-127\\
1.77177177177177	2.40567437166664e-126\\
1.78678678678679	4.9285241309004e-125\\
1.8018018018018	9.93386824085582e-124\\
1.81681681681682	1.96988723459302e-122\\
1.83183183183183	3.84313640065568e-121\\
1.84684684684685	7.37652288262532e-120\\
1.86186186186186	1.39296130260585e-118\\
1.87687687687688	2.58790227047956e-117\\
1.89189189189189	4.73018529901698e-116\\
1.90690690690691	8.5060885361715e-115\\
1.92192192192192	1.5048842812399e-113\\
1.93693693693694	2.61937558688205e-112\\
1.95195195195195	4.48553146402591e-111\\
1.96696696696697	7.55703618461391e-110\\
1.98198198198198	1.25259466719283e-108\\
1.996996996997	2.04263645011623e-107\\
2.01201201201201	3.27712539123634e-106\\
2.02702702702703	5.17269060618435e-105\\
2.04204204204204	8.03269743744209e-104\\
2.05705705705706	1.227235127095e-102\\
2.07207207207207	1.84465693341531e-101\\
2.08708708708709	2.72787774014685e-100\\
2.1021021021021	3.96876750031491e-99\\
2.11711711711712	5.68077875927248e-98\\
2.13213213213213	7.99984470251093e-97\\
2.14714714714715	1.10834942216897e-95\\
2.16216216216216	1.51075244044426e-94\\
2.17717717717718	2.02596235865955e-93\\
2.19219219219219	2.67295041370469e-92\\
2.20720720720721	3.46953996642588e-91\\
2.22222222222222	4.43072044687182e-90\\
2.23723723723724	5.56670576286144e-89\\
2.25225225225225	6.8808746475755e-88\\
2.26726726726727	8.36778452030049e-87\\
2.28228228228228	1.00114917182454e-85\\
2.2972972972973	1.17844294066981e-84\\
2.31231231231231	1.36470815974047e-83\\
2.32732732732733	1.55486425542003e-82\\
2.34234234234234	1.74287657803463e-81\\
2.35735735735736	1.92203923973801e-80\\
2.37237237237237	2.0853517868345e-79\\
2.38738738738739	2.22596263234846e-78\\
2.4024024024024	2.33764130734624e-77\\
2.41741741741742	2.41523469452986e-76\\
2.43243243243243	2.4550608838361e-75\\
2.44744744744745	2.4551987670965e-74\\
2.46246246246246	2.41564165690708e-73\\
2.47747747747748	2.33829782428664e-72\\
2.49249249249249	2.22683789492747e-71\\
2.50750750750751	2.08640609769892e-70\\
2.52252252252252	1.92322699283568e-69\\
2.53753753753754	1.74414951124125e-68\\
2.55255255255255	1.55617465505231e-67\\
2.56756756756757	1.36601172604979e-66\\
2.58258258258258	1.17970108647136e-65\\
2.5975975975976	1.00233061080219e-64\\
2.61261261261261	8.3786002525476e-64\\
2.62762762762763	6.89054240297843e-63\\
2.64264264264264	5.57515326725839e-62\\
2.65765765765766	4.43794254092568e-61\\
2.67267267267267	3.47558569685262e-60\\
2.68768768768769	2.67790884648418e-59\\
2.7027027027027	2.02994859829174e-58\\
2.71771771771772	1.51389499925194e-57\\
2.73273273273273	1.11077968973228e-56\\
2.74774774774775	8.01828647232183e-56\\
2.76276276276276	5.69451405186675e-55\\
2.77777777777778	3.97881028510161e-54\\
2.79279279279279	2.73508770496395e-53\\
2.80780780780781	1.84974024201327e-52\\
2.82282282282282	1.23075524445277e-51\\
2.83783783783784	8.05664276983786e-51\\
2.85285285285285	5.18869308226074e-50\\
2.86786786786787	3.28763291333991e-49\\
2.88288288288288	2.04941598485701e-48\\
2.8978978978979	1.25689321339562e-47\\
2.91291291291291	7.58382155533278e-47\\
2.92792792792793	4.50193574803143e-46\\
2.94294294294294	2.62925035711741e-45\\
2.95795795795796	1.51072721582108e-44\\
2.97297297297297	8.54007386538168e-44\\
2.98798798798799	4.74961779805217e-43\\
3.003003003003	2.59882575463659e-42\\
3.01801801801802	1.39899809489027e-41\\
3.03303303303303	7.40932332111804e-41\\
3.04804804804805	3.86065890154763e-40\\
3.06306306306306	1.97909107629681e-39\\
3.07807807807808	9.98140301264453e-39\\
3.09309309309309	4.95266398507188e-38\\
3.10810810810811	2.417728887206e-37\\
3.12312312312312	1.16117532167553e-36\\
3.13813813813814	5.48667779951951e-36\\
3.15315315315315	2.55060131864006e-35\\
3.16816816816817	1.16653336456358e-34\\
3.18318318318318	5.24895938145272e-34\\
3.1981981981982	2.32365004650863e-33\\
3.21321321321321	1.01202139445593e-32\\
3.22822822822823	4.33640791200178e-32\\
3.24324324324324	1.82806664659259e-31\\
3.25825825825826	7.58185342296701e-31\\
3.27327327327327	3.09371430666449e-30\\
3.28828828828829	1.24195679986586e-29\\
3.3033033033033	4.90517194842027e-29\\
3.31831831831832	1.90600240666518e-28\\
3.33333333333333	7.28641857000656e-28\\
3.34834834834835	2.74047755356273e-27\\
3.36336336336336	1.01405111077486e-26\\
3.37837837837838	3.69160226131066e-26\\
3.39339339339339	1.32218255873333e-25\\
3.40840840840841	4.65896456128924e-25\\
3.42342342342342	1.61513508909075e-24\\
3.43843843843844	5.50870809876644e-24\\
3.45345345345345	1.84846882423885e-23\\
3.46846846846847	6.10233383912103e-23\\
3.48348348348348	1.98198908169686e-22\\
3.4984984984985	6.33327017154846e-22\\
3.51351351351351	1.99102280630502e-21\\
3.52852852852853	6.1580883690919e-21\\
3.54354354354354	1.87385965700368e-20\\
3.55855855855856	5.60982957929263e-20\\
3.57357357357357	1.6522803472651e-19\\
3.58858858858859	4.78783603257617e-19\\
3.6036036036036	1.36494849128566e-18\\
3.61861861861862	3.82837718298083e-18\\
3.63363363363363	1.05641519170632e-17\\
3.64864864864865	2.86797934365894e-17\\
3.66366366366366	7.66017801889824e-17\\
3.67867867867868	2.01290448920751e-16\\
3.69369369369369	5.20389993158059e-16\\
3.70870870870871	1.32359819839778e-15\\
3.72372372372372	3.31211082731902e-15\\
3.73873873873874	8.15408089716192e-15\\
3.75375375375375	1.97499787916238e-14\\
3.76876876876877	4.70630115128503e-14\\
3.78378378378378	1.10335243130223e-13\\
3.7987987987988	2.5448973019151e-13\\
3.81381381381381	5.77494335386519e-13\\
3.82882882882883	1.2892782562379e-12\\
3.84384384384384	2.83182916611653e-12\\
3.85885885885886	6.11940088372357e-12\\
3.87387387387387	1.3009847976377e-11\\
3.88888888888889	2.72117831223766e-11\\
3.9039039039039	5.59968042717959e-11\\
3.91891891891892	1.13368120489497e-10\\
3.93393393393393	2.25808410130235e-10\\
3.94894894894895	4.4249741886561e-10\\
3.96396396396396	8.53105678811984e-10\\
3.97897897897898	1.6181410908404e-09\\
3.99399399399399	3.01961380604105e-09\\
4.00900900900901	5.54380416799546e-09\\
4.02402402402402	1.00134981363258e-08\\
4.03903903903904	1.77944768471973e-08\\
4.05405405405405	3.11104354427034e-08\\
4.06906906906907	5.35116592735024e-08\\
4.08408408408408	9.05549526587212e-08\\
4.0990990990991	1.84129791632184e-07\\
4.11411411411411	3.04219322096755e-07\\
4.12912912912913	4.94671035091917e-07\\
4.14414414414414	7.93343714315127e-07\\
4.15915915915916	1.2500006400714e-06\\
4.17417417417417	1.93851801087274e-06\\
4.18918918918919	2.95903382365857e-06\\
4.2042042042042	4.44591494378682e-06\\
4.21921921921922	6.57529447555829e-06\\
4.23423423423423	9.57249239973197e-06\\
4.24924924924925	1.37184471578086e-05\\
4.26426426426426	1.94202531214065e-05\\
4.27927927927928	2.69920848223108e-05\\
4.29429429429429	3.70196392682317e-05\\
4.30930930930931	4.99285145617446e-05\\
4.32432432432432	6.63150100073362e-05\\
4.33933933933934	8.67677490100719e-05\\
4.35435435435435	0.000111855706757785\\
4.36936936936937	0.000142099532645251\\
4.38438438438438	0.0001780028166013\\
4.3993993993994	0.000219858678894613\\
4.41441441441441	0.000267798199836811\\
4.42942942942943	0.000321805475839679\\
4.44444444444444	0.00038170165851022\\
4.45945945945946	0.000447187804363241\\
4.47447447447447	0.000517671162971161\\
4.48948948948949	0.000592457721513851\\
4.5045045045045	0.000670906086715845\\
4.51951951951952	0.00075226061001414\\
4.53453453453453	0.000835895648185937\\
4.54954954954955	0.00092149261737145\\
4.56456456456456	0.00100853115588404\\
4.57957957957958	0.00109719698748568\\
4.59459459459459	0.00118723527036143\\
4.60960960960961	0.00127950319937291\\
4.62462462462462	0.00137437842475124\\
4.63963963963964	0.00147213080936569\\
4.65465465465465	0.00157381534799664\\
4.66966966966967	0.00168052510776825\\
4.68468468468468	0.00179298870939959\\
4.6996996996997	0.00191231642649762\\
4.71471471471471	0.00203928496471414\\
4.72972972972973	0.00217525133897681\\
4.74474474474474	0.00232211344915482\\
4.75975975975976	0.00248115391849767\\
4.77477477477477	0.00265401240890645\\
4.78978978978979	0.00284312406822444\\
4.8048048048048	0.00305132982013715\\
4.81981981981982	0.00328126889098824\\
4.83483483483483	0.00353588752112253\\
4.84984984984985	0.00381796290149013\\
4.86486486486486	0.00413092611281068\\
4.87987987987988	0.00447599998146451\\
4.89489489489489	0.00485556038004336\\
4.90990990990991	0.0052693730680851\\
4.92492492492492	0.00571661199915321\\
4.93993993993994	0.00619499297102148\\
4.95495495495495	0.0067009265461182\\
4.96996996996997	0.00722921750541049\\
4.98498498498498	0.00777289227977891\\
5	0.00832467323797119\\
5.01501501501502	0.00887756016156297\\
5.03003003003003	0.00942360474437289\\
5.04504504504505	0.00995650891675675\\
5.06006006006006	0.0104712764267635\\
5.07507507507508	0.0109645510983207\\
5.09009009009009	0.0114355123639295\\
5.10510510510511	0.011886783276155\\
5.12012012012012	0.0123237217434641\\
5.13513513513514	0.0127553349967194\\
5.15015015015015	0.0131921852290956\\
5.16516516516517	0.0136476044178552\\
5.18018018018018	0.0141384218884895\\
5.1951951951952	0.0146795714029093\\
5.21021021021021	0.0152900880566849\\
5.22522522522523	0.0159853027998749\\
5.24024024024024	0.0167828514837831\\
5.25525525525526	0.0176956733400159\\
5.27027027027027	0.0187379511244406\\
5.28528528528529	0.0199202549853342\\
5.3003003003003	0.0212518335337565\\
5.31531531531532	0.022739298297299\\
5.33033033033033	0.0243870577594222\\
5.34534534534535	0.0261961466959566\\
5.36036036036036	0.0281677997545383\\
5.37537537537538	0.0302996550361645\\
5.39039039039039	0.0325882646871005\\
5.40540540540541	0.0350284837590202\\
5.42042042042042	0.0376134929053481\\
5.43543543543544	0.0403330074918695\\
5.45045045045045	0.0431757726522171\\
5.46546546546547	0.0461282797599241\\
5.48048048048048	0.0491773457567897\\
5.4954954954955	0.0523061468529168\\
5.51051051051051	0.0554989185406258\\
5.52552552552553	0.0587407774412262\\
5.54054054054054	0.0620192162280672\\
5.55555555555556	0.0653195655549182\\
5.57057057057057	0.0686343449401203\\
5.58558558558559	0.0719559024276646\\
5.6006006006006	0.0752833493343228\\
5.61561561561562	0.0786171648444675\\
5.63063063063063	0.0819639872053387\\
5.64564564564565	0.0853340730794421\\
5.66066066066066	0.0887349696563406\\
5.67567567567568	0.0921801163611665\\
5.69069069069069	0.0956850939667727\\
5.70570570570571	0.0992600008322346\\
5.72072072072072	0.102918275280908\\
5.73573573573574	0.106673119712793\\
5.75075075075075	0.110530811857322\\
5.76576576576577	0.114499142042232\\
5.78078078078078	0.118585622385183\\
5.7957957957958	0.122793242037805\\
5.81081081081081	0.12712655177884\\
5.82582582582583	0.131588996354431\\
5.84084084084084	0.1361846256863\\
5.85585585585586	0.140918688008577\\
5.87087087087087	0.145797046623319\\
5.88588588588589	0.150825415218489\\
5.9009009009009	0.156012379125276\\
5.91591591591592	0.161368341402013\\
5.93093093093093	0.166899356128922\\
5.94594594594595	0.172616065564524\\
5.96096096096096	0.178526484256716\\
5.97597597597598	0.184635464884833\\
5.99099099099099	0.190947474296766\\
6.00600600600601	0.197461242766998\\
6.02102102102102	0.204170408189131\\
6.03603603603604	0.211070585193025\\
6.05105105105105	0.218140451725026\\
6.06606606606607	0.225361362061319\\
6.08108108108108	0.232706469444873\\
6.0960960960961	0.240142970888645\\
6.11111111111111	0.247639007776056\\
6.12612612612613	0.255159438492252\\
6.14114114114114	0.262666329471141\\
6.15615615615616	0.270135719744139\\
6.17117117117117	0.277539928068027\\
6.18618618618619	0.284861854660673\\
6.2012012012012	0.292096020847834\\
6.21621621621622	0.299244515698547\\
6.23123123123123	0.306323002721386\\
6.24624624624625	0.313352046755854\\
6.26126126126126	0.320365649011274\\
6.27627627627628	0.327400853512547\\
6.29129129129129	0.334498423900241\\
6.30630630630631	0.341689855004832\\
6.32132132132132	0.349015483683901\\
6.33633633633634	0.356496854471409\\
6.35135135135135	0.364148085138591\\
6.36636636636637	0.371974413901484\\
6.38138138138138	0.379963111192665\\
6.3963963963964	0.388094053179177\\
6.41141141141141	0.396333198487606\\
6.42642642642643	0.404637859005938\\
6.44144144144144	0.412952194427736\\
6.45645645645646	0.421227229080972\\
6.47147147147147	0.42940638949084\\
6.48648648648649	0.437435506834864\\
6.5015015015015	0.445266953528342\\
6.51651651651652	0.452863519108805\\
6.53153153153153	0.460191293685819\\
6.54654654654655	0.467232171980472\\
6.56156156156156	0.473973285623368\\
6.57657657657658	0.480420699365387\\
6.59159159159159	0.48658163807918\\
6.60660660660661	0.492479991717753\\
6.62162162162162	0.49813984930545\\
6.63663663663664	0.50359158318354\\
6.65165165165165	0.508871207320522\\
6.66666666666667	0.514005825151396\\
6.68168168168168	0.519025581909887\\
6.6966966966967	0.523950439293809\\
6.71171171171171	0.528795917596856\\
6.72672672672673	0.533564143159164\\
6.74174174174174	0.538251000807518\\
6.75675675675676	0.542834845422253\\
6.77177177177177	0.54729257083308\\
6.78678678678679	0.551584897825424\\
6.8018018018018	0.555672409004296\\
6.81681681681682	0.559509509296743\\
6.83183183183183	0.563045366292237\\
6.84684684684685	0.566238876459038\\
6.86186186186186	0.569044100912046\\
6.87687687687688	0.571426475301734\\
6.89189189189189	0.573352453436549\\
6.90690690690691	0.574805805070393\\
6.92192192192192	0.575764352531336\\
6.93693693693694	0.576222554132493\\
6.95195195195195	0.576177451573029\\
6.96696696696697	0.575631276070188\\
6.98198198198198	0.574592494894103\\
6.996996996997	0.573078124908542\\
7.01201201201201	0.571100614730133\\
7.02702702702703	0.568688852015071\\
7.04204204204204	0.565864966699706\\
7.05705705705706	0.562665292957487\\
7.07207207207207	0.559126872290949\\
7.08708708708709	0.555293700695165\\
7.1021021021021	0.551213972551339\\
7.11711711711712	0.546944274393126\\
7.13213213213213	0.542536514008003\\
7.14714714714715	0.538046543263069\\
7.16216216216216	0.533517240276101\\
7.17717717717718	0.528997205508009\\
7.19219219219219	0.524524690531647\\
7.20720720720721	0.520122191248899\\
7.22222222222222	0.515802073181875\\
7.23723723723724	0.51156588192923\\
7.25225225225225	0.507400913685666\\
7.26726726726727	0.503279862519002\\
7.28228228228228	0.499173071459935\\
7.2972972972973	0.495040949239615\\
7.31231231231231	0.49084097858314\\
7.32732732732733	0.486528297020051\\
7.34234234234234	0.4820635046573\\
7.35735735735736	0.477413822424802\\
7.37237237237237	0.472563367965038\\
7.38738738738739	0.467498219378284\\
7.4024024024024	0.46221931944047\\
7.41741741741742	0.456744295713199\\
7.43243243243243	0.451103630883218\\
7.44744744744745	0.445332318992447\\
7.46246246246246	0.439480734471667\\
7.47747747747748	0.433592385759812\\
7.49249249249249	0.427715724794697\\
7.50750750750751	0.421894093119705\\
7.52252252252252	0.416158112734723\\
7.53753753753754	0.410529710198018\\
7.55255255255255	0.405016003770697\\
7.56756756756757	0.399611764600081\\
7.58258258258258	0.39429529429844\\
7.5975975975976	0.389027906092456\\
7.61261261261261	0.383771719067734\\
7.62762762762763	0.378481466237377\\
7.64264264264264	0.373108208039876\\
7.65765765765766	0.367607848397921\\
7.67267267267267	0.361944928693302\\
7.68768768768769	0.356095680402052\\
7.7027027027027	0.35004933354718\\
7.71771771771772	0.343801513919822\\
7.73273273273273	0.337368327121487\\
7.74774774774775	0.330765993625385\\
7.76276276276276	0.324027016988417\\
7.77777777777778	0.317185024976561\\
7.79279279279279	0.310266699564033\\
7.80780780780781	0.303308127269596\\
7.82282282282282	0.296332043277001\\
7.83783783783784	0.289357753586206\\
7.85285285285285	0.28239929081008\\
7.86786786786787	0.275465059286365\\
7.88288288288288	0.268559970986262\\
7.8978978978979	0.261681884716984\\
7.91291291291291	0.25483039225594\\
7.92792792792793	0.248005803641934\\
7.94294294294294	0.241207179608557\\
7.95795795795796	0.234439384577256\\
7.97297297297297	0.227707015599019\\
7.98798798798799	0.221022016295596\\
8.003003003003	0.214393577198845\\
8.01801801801802	0.207836126065365\\
8.03303303303303	0.201360521996269\\
8.04804804804805	0.19498128798299\\
8.06306306306306	0.188709691642545\\
8.07807807807808	0.182560821764316\\
8.09309309309309	0.176545722277675\\
8.10810810810811	0.170674711110513\\
8.12312312312312	0.164956012488346\\
8.13813813813814	0.159398926593199\\
8.15315315315315	0.154012041713096\\
8.16816816816817	0.148803935643132\\
8.18318318318318	0.143783912311111\\
8.1981981981982	0.138956053328818\\
8.21321321321321	0.134326825624053\\
8.22822822822823	0.129897511659923\\
8.24324324324324	0.125668009941795\\
8.25825825825826	0.121635500281991\\
8.27327327327327	0.117795733540629\\
8.28828828828829	0.114137413554056\\
8.3033033033033	0.110647078021631\\
8.31831831831832	0.107307490350358\\
8.33333333333333	0.104101632149687\\
8.34834834834835	0.101009449044498\\
8.36336336336336	0.0980090034150306\\
8.37837837837838	0.0950782956872589\\
8.39339339339339	0.0921985971579306\\
8.40840840840841	0.0893521045675064\\
8.42342342342342	0.0865260962000692\\
8.43843843843844	0.0837071122379995\\
8.45345345345345	0.0808884650158702\\
8.46846846846847	0.0780679032141084\\
8.48348348348348	0.0752457757851231\\
8.4984984984985	0.0724250195432459\\
8.51351351351351	0.069613367007007\\
8.52852852852853	0.0668207785728853\\
8.54354354354354	0.0640592595828284\\
8.55855855855856	0.0613398146845549\\
8.57357357357357	0.0586744203326834\\
8.58858858858859	0.05607476734237\\
8.6036036036036	0.0535542467039287\\
8.61861861861862	0.0511215401891613\\
8.63363363363363	0.0487862212409662\\
8.64864864864865	0.0465559168026389\\
8.66366366366366	0.0444388236034945\\
8.67867867867868	0.0424389886524192\\
8.69369369369369	0.0405634978185939\\
8.70870870870871	0.038814257929553\\
8.72372372372372	0.0371942123754283\\
8.73873873873874	0.0357028921966419\\
8.75375375375375	0.0343395741146975\\
8.76876876876877	0.033100406528542\\
8.78378378378378	0.0319791604950826\\
8.7987987987988	0.0309655187165838\\
8.81381381381381	0.030049683775762\\
8.82882882882883	0.0292169480957569\\
8.84384384384384	0.0284518297132962\\
8.85885885885886	0.0277373182775626\\
8.87387387387387	0.0270545776634977\\
8.88888888888889	0.0263872908646707\\
8.9039039039039	0.025718672799246\\
8.91891891891892	0.0250329262968859\\
8.93393393393393	0.0243177030532019\\
8.94894894894895	0.0235639979870157\\
8.96396396396396	0.0227641000015499\\
8.97897897897898	0.0219156738931287\\
8.99399399399399	0.0210186384519241\\
9.00900900900901	0.0200761199652805\\
9.02402402402402	0.0190930085686197\\
9.03903903903904	0.0180772920640485\\
9.05405405405405	0.0170399027015665\\
9.06906906906907	0.0159905109534938\\
9.08408408408408	0.0149407929653729\\
9.0990990990991	0.0139034135565147\\
9.11411411411411	0.0128874192505858\\
9.12912912912913	0.0119043379043695\\
9.14414414414414	0.0109617402405921\\
9.15915915915916	0.0100686736443184\\
9.17417417417417	0.00923054621709536\\
9.18918918918919	0.0084505823466636\\
9.2042042042042	0.00773135251483475\\
9.21921921921922	0.00707353158945038\\
9.23423423423423	0.00647631444265441\\
9.24924924924925	0.00593658701529684\\
9.26426426426426	0.00545043634821221\\
9.27927927927928	0.0050131853673302\\
9.29429429429429	0.00461889573099907\\
9.30930930930931	0.0042626286838153\\
9.32432432432432	0.00393772190147117\\
9.33933933933934	0.0036398228076047\\
9.35435435435435	0.00336265911285789\\
9.36936936936937	0.00310299626809559\\
9.38438438438438	0.00285657405570469\\
9.3993993993994	0.0026218703769794\\
9.41441441441441	0.00239687816879131\\
9.42942942942943	0.00218149319812114\\
9.44444444444444	0.00197609085102112\\
9.45945945945946	0.00178157791580195\\
9.47447447447447	0.00159928268286213\\
9.48948948948949	0.0014304880302214\\
9.5045045045045	0.00127680879369568\\
9.51951951951952	0.00113932915033726\\
9.53453453453453	0.00101901278558351\\
9.54954954954955	0.000916491706783091\\
9.56456456456456	0.00083114222368613\\
9.57957957957958	0.000762831512335873\\
9.59459459459459	0.000710767066536767\\
9.60960960960961	0.000673411877004482\\
9.62462462462462	0.000650004408733676\\
9.63963963963964	0.000638919442676489\\
9.65465465465465	0.000638955515305759\\
9.66966966966967	0.000649110065831982\\
9.68468468468468	0.000668239438225025\\
9.6996996996997	0.000694953870087127\\
9.71471471471471	0.000728564918744981\\
9.72972972972973	0.000767760121469858\\
9.74474474474474	0.000811014407684329\\
9.75975975975976	0.000856708419514885\\
9.77477477477477	0.000902960122031142\\
9.78978978978979	0.000947543249415499\\
9.8048048048048	0.000988359463535169\\
9.81981981981982	0.0010229879788057\\
9.83483483483483	0.00104928849868747\\
9.84984984984985	0.00106513885027637\\
9.86486486486486	0.0010688834184881\\
9.87987987987988	0.00105959679579385\\
9.89489489489489	0.00103688890892518\\
9.90990990990991	0.00100110530207192\\
9.92492492492492	0.000953074993489867\\
9.93993993993994	0.000894358583778851\\
9.95495495495495	0.000826994445276585\\
9.96996996996997	0.00075334870802847\\
9.98498498498498	0.000675946926608636\\
10	0.000597305171127309\\
10.015015015015	0.00051977691057807\\
10.03003003003	0.0004454287916467\\
10.045045045045	0.000375953894445368\\
10.0600600600601	0.000312626013576077\\
10.0750750750751	0.000256293692922171\\
10.0900900900901	0.000207408727274234\\
10.1051051051051	0.000166081019170151\\
10.1201201201201	0.000132150199339132\\
10.1351351351351	0.000105181058955692\\
10.1501501501502	8.49070296574057e-05\\
10.1651651651652	7.06826177543466e-05\\
10.1801801801802	6.2003805439326e-05\\
10.1951951951952	5.84056297053348e-05\\
10.2102102102102	5.94798805027603e-05\\
10.2252252252252	6.48750562070983e-05\\
10.2402402402402	7.42810823179528e-05\\
10.2552552552553	8.74024983562837e-05\\
10.2702702702703	0.000103924674354866\\
10.2852852852853	0.000123478120838898\\
10.3003003003003	0.00014560607846232\\
10.3153153153153	0.000169740278124454\\
10.3303303303303	0.0001951050850443\\
10.3453453453453	0.000221091024238375\\
10.3603603603604	0.000246550092796892\\
10.3753753753754	0.00027068356979731\\
10.3903903903904	0.000292530221145362\\
10.4054054054054	0.000311029115959981\\
10.4204204204204	0.000325351501124393\\
10.4354354354354	0.000334831297746673\\
10.4504504504505	0.000339016424929727\\
10.4654654654655	0.000337704537333148\\
10.4804804804805	0.000330959242731209\\
10.4954954954955	0.000319104989593065\\
10.5105105105105	0.000302701196864935\\
10.5255255255255	0.000282498498393567\\
10.5405405405405	0.000259381866819679\\
10.5555555555556	0.0002343066107216\\
10.5705705705706	0.000208233661015748\\
10.5855855855856	0.000182070166651962\\
10.6006006006006	0.00015662032008218\\
10.6156156156156	0.000132549742090704\\
10.6306306306306	0.000110364939427857\\
10.6456456456456	9.04075778207501e-05\\
10.6606606606607	7.28618175979957e-05\\
10.6756756756757	5.77718972615706e-05\\
10.6906906906907	4.5066592484057e-05\\
10.7057057057057	3.45871100815724e-05\\
10.7207207207207	2.6115316446603e-05\\
10.7357357357357	1.93998231806194e-05\\
10.7507507507508	1.41782200638344e-05\\
10.7657657657658	1.01945278475938e-05\\
10.7807807807808	7.21163906464468e-06\\
10.7957957957958	5.01905911667101e-06\\
10.8108108108108	3.4366246578301e-06\\
10.8258258258258	2.31506588404233e-06\\
10.8408408408408	1.53432043975464e-06\\
10.8558558558559	1.00043815951531e-06\\
10.8708708708709	6.41779571480756e-07\\
10.8858858858859	4.05044735295912e-07\\
10.9009009009009	2.51502065876334e-07\\
10.9159159159159	1.5363903979116e-07\\
10.9309309309309	9.23385550240587e-08\\
10.9459459459459	5.4599168196028e-08\\
10.960960960961	3.17621901229822e-08\\
10.975975975976	1.81784277341362e-08\\
10.990990990991	1.02358439784199e-08\\
11.006006006006	5.67038385392087e-09\\
11.021021021021	3.09045712901725e-09\\
11.036036036036	1.65712187372778e-09\\
11.0510510510511	8.74193632643588e-10\\
11.0660660660661	4.53714114875202e-10\\
11.0810810810811	2.31674590426339e-10\\
11.0960960960961	1.16384723724678e-10\\
11.1111111111111	5.75221378400962e-11\\
11.1261261261261	2.79701966193287e-11\\
11.1411411411411	1.33806583678382e-11\\
11.1561561561562	6.29768462294963e-12\\
11.1711711711712	2.91612271705795e-12\\
11.1861861861862	1.32847120840214e-12\\
11.2012012012012	5.95415257526986e-13\\
11.2162162162162	2.625483144002e-13\\
11.2312312312312	1.13899015945468e-13\\
11.2462462462462	4.86129718802109e-14\\
11.2612612612613	2.04129529285917e-14\\
11.2762762762763	8.4329778336683e-15\\
11.2912912912913	3.42750065000274e-15\\
11.3063063063063	1.37055224468822e-15\\
11.3213213213213	5.39181654543358e-16\\
11.3363363363363	2.08687329351607e-16\\
11.3513513513514	7.94654812469763e-17\\
11.3663663663664	2.97702455507499e-17\\
11.3813813813814	1.09725552962061e-17\\
11.3963963963964	3.97882283614217e-18\\
11.4114114114114	1.41945911507778e-18\\
11.4264264264264	4.98210224948669e-19\\
11.4414414414414	1.72037786982643e-19\\
11.4564564564565	5.84462323976877e-20\\
11.4714714714715	1.95348758833805e-20\\
11.4864864864865	6.42371484971605e-21\\
11.5015015015015	2.07818073558246e-21\\
11.5165165165165	6.61457336021379e-22\\
11.5315315315315	2.07129435995838e-22\\
11.5465465465465	6.38121342433528e-23\\
11.5615615615616	1.93413231062105e-23\\
11.5765765765766	5.76753916614074e-24\\
11.5915915915916	1.69206244737349e-24\\
11.6066066066066	4.88386537701982e-25\\
11.6216216216216	1.38685942928121e-25\\
11.6366366366366	3.87456245231769e-26\\
11.6516516516517	1.06496258401119e-26\\
11.6666666666667	2.87983427536778e-27\\
11.6816816816817	7.66164641296252e-28\\
11.6966966966967	2.00538691407085e-28\\
11.7117117117117	5.16411268599057e-29\\
11.7267267267267	1.30832218756113e-29\\
11.7417417417417	3.2610327853595e-30\\
11.7567567567568	7.9968156796383e-31\\
11.7717717717718	1.92930313172006e-31\\
11.7867867867868	4.57936557383225e-32\\
11.8018018018018	1.06937893011378e-32\\
11.8168168168168	2.45685425099961e-33\\
11.8318318318318	5.55326791365813e-34\\
11.8468468468468	1.23492141370193e-34\\
11.8618618618619	2.70178930993027e-35\\
11.8768768768769	5.81547373944577e-36\\
11.8918918918919	1.23151634787149e-36\\
11.9069069069069	2.56576411060453e-37\\
11.9219219219219	5.25914001328126e-38\\
11.9369369369369	1.06055745867395e-38\\
11.951951951952	2.10414243236394e-39\\
11.966966966967	4.10712129251607e-40\\
11.981981981982	7.88717333547884e-41\\
11.996996996997	1.49013873629439e-41\\
12.012012012012	2.76983236610208e-42\\
12.027027027027	5.065259847757e-43\\
12.042042042042	9.11321121058627e-44\\
12.0570570570571	1.6131048771891e-44\\
12.0720720720721	2.80915197537589e-45\\
12.0870870870871	4.81292759784838e-46\\
12.1021021021021	8.1126900572577e-47\\
12.1171171171171	1.34537050505425e-47\\
12.1321321321321	2.19502960840021e-48\\
12.1471471471471	3.52338682124963e-49\\
12.1621621621622	5.5641871738296e-50\\
12.1771771771772	8.64499205365571e-51\\
12.1921921921922	1.32144450212375e-51\\
12.2072072072072	1.98726013597123e-52\\
12.2222222222222	2.94023454289206e-53\\
12.2372372372372	4.27987113359309e-54\\
12.2522522522523	6.1291587797399e-55\\
12.2672672672673	8.63559992866288e-56\\
12.2822822822823	1.19703160830065e-56\\
12.2972972972973	1.63245111557548e-57\\
12.3123123123123	2.190262734477e-58\\
12.3273273273273	2.89117051040033e-59\\
12.3423423423423	3.75467766556884e-60\\
12.3573573573574	4.79725799695067e-61\\
12.3723723723724	6.03024495581227e-62\\
12.3873873873874	7.45758642456256e-63\\
12.4024024024024	9.07367246501536e-64\\
12.4174174174174	1.08614889343961e-64\\
12.4324324324324	1.27913709063405e-65\\
12.4474474474474	1.48206169471739e-66\\
12.4624624624625	1.68941729377228e-67\\
12.4774774774775	1.89465024689745e-68\\
12.4924924924925	2.09046368943859e-69\\
12.5075075075075	2.26922556903808e-70\\
12.5225225225225	2.42345059671963e-71\\
12.5375375375375	2.54631504216425e-72\\
12.5525525525526	2.63215562264414e-73\\
12.5675675675676	2.67690187757673e-74\\
12.5825825825826	2.67839610189063e-75\\
12.5975975975976	2.63656582544899e-76\\
12.6126126126126	2.55342963950477e-77\\
12.6276276276276	2.43293571790016e-78\\
12.6426426426426	2.28065102085126e-79\\
12.6576576576577	2.10333524678772e-80\\
12.6726726726727	1.90844490077236e-81\\
12.6876876876877	1.70361797788775e-82\\
12.7027027027027	1.49618833833213e-83\\
12.7177177177177	1.29277152845556e-84\\
12.7327327327327	1.09895207794535e-85\\
12.7477477477477	9.19088204980928e-87\\
12.7627627627628	7.56235556714088e-88\\
12.7777777777778	6.12179040125148e-89\\
12.7927927927928	4.87552380625628e-90\\
12.8078078078078	3.82019532712657e-91\\
12.8228228228228	2.94490520114112e-92\\
12.8378378378378	2.23346181320438e-93\\
12.8528528528529	1.66650724589901e-94\\
12.8678678678679	1.22336863984103e-95\\
12.8828828828829	8.83545548309646e-97\\
12.8978978978979	6.27801002710065e-98\\
12.9129129129129	4.38870641404575e-99\\
12.9279279279279	3.01837028053925e-100\\
12.9429429429429	2.04234903814015e-101\\
12.957957957958	1.35959286457959e-102\\
12.972972972973	8.90449410055423e-104\\
12.987987987988	5.73761081853013e-105\\
13.003003003003	3.63726032519089e-106\\
13.018018018018	2.26850189508887e-107\\
13.033033033033	1.39195555912307e-108\\
13.048048048048	8.40297469276707e-110\\
13.0630630630631	4.99070849586767e-111\\
13.0780780780781	2.91616984168795e-112\\
13.0930930930931	1.67642792789354e-113\\
13.1081081081081	9.48152997815878e-115\\
13.1231231231231	5.27586191447998e-116\\
13.1381381381381	2.8882174014391e-117\\
13.1531531531532	1.55556370949699e-118\\
13.1681681681682	8.242656232275e-120\\
13.1831831831832	4.29702652740252e-121\\
13.1981981981982	2.20389218238023e-122\\
13.2132132132132	1.11207509899408e-123\\
13.2282282282282	5.52076613922381e-125\\
13.2432432432432	2.69641073489401e-126\\
13.2582582582583	1.29566941344629e-127\\
13.2732732732733	6.12525012883988e-129\\
13.2882882882883	2.84888495695323e-130\\
13.3033033033033	1.30360932463568e-131\\
13.3183183183183	5.86869401918958e-133\\
13.3333333333333	2.59930317925646e-134\\
13.3483483483483	1.1326451545226e-135\\
13.3633633633634	4.85570469989043e-137\\
13.3783783783784	2.04801024768614e-138\\
13.3933933933934	8.49832715737259e-140\\
13.4084084084084	3.46941481703959e-141\\
13.4234234234234	1.39347914779179e-142\\
13.4384384384384	5.50638003611411e-144\\
13.4534534534535	2.14068788917745e-145\\
13.4684684684685	8.18770118705237e-147\\
13.4834834834835	3.08100270363352e-148\\
13.4984984984985	1.14062697117652e-149\\
13.5135135135135	4.15447984417635e-151\\
13.5285285285285	1.4887135398748e-152\\
13.5435435435435	5.24840188374109e-154\\
13.5585585585586	1.82039020267148e-155\\
13.5735735735736	6.211884306955e-157\\
13.5885885885886	2.08546876768897e-158\\
13.6036036036036	6.88819580723874e-160\\
13.6186186186186	2.23835383641144e-161\\
13.6336336336336	7.15605152774334e-163\\
13.6486486486486	2.25081416770681e-164\\
13.6636636636637	6.96509883727885e-166\\
13.6786786786787	2.12049078472416e-167\\
13.6936936936937	6.35136333534549e-169\\
13.7087087087087	1.87162564925056e-170\\
13.7237237237237	5.42615857148483e-172\\
13.7387387387387	1.54770233065631e-173\\
13.7537537537538	4.34314025376695e-175\\
13.7687687687688	1.1990622614074e-176\\
13.7837837837838	3.2568748984592e-178\\
13.7987987987988	8.70325847880735e-180\\
13.8138138138138	2.28814819730359e-181\\
13.8288288288288	5.91844858022555e-183\\
13.8438438438438	1.50609744174491e-184\\
13.8588588588589	3.77068041524412e-186\\
13.8738738738739	9.2876927962305e-188\\
13.8888888888889	2.25069901691046e-189\\
13.9039039039039	5.36597274928704e-191\\
13.9189189189189	1.2586383686548e-192\\
13.9339339339339	2.90452375722671e-194\\
13.9489489489489	6.59432521792529e-196\\
13.963963963964	1.4729474500144e-197\\
13.978978978979	3.23687260021234e-199\\
13.993993993994	6.9981850512963e-201\\
14.009009009009	1.48856121973486e-202\\
14.024024024024	3.11508169727601e-204\\
14.039039039039	6.41347851379384e-206\\
14.0540540540541	1.29909020876663e-207\\
14.0690690690691	2.58884731444029e-209\\
14.0840840840841	5.07568927459231e-211\\
14.0990990990991	9.79050469407328e-213\\
14.1141141141141	1.85796100560159e-214\\
14.1291291291291	3.46888236245566e-216\\
14.1441441441441	6.37182752848005e-218\\
14.1591591591592	1.15148931916651e-219\\
14.1741741741742	2.04728016306231e-221\\
14.1891891891892	3.58109716873802e-223\\
14.2042042042042	6.16277616795207e-225\\
14.2192192192192	1.0434175713176e-226\\
14.2342342342342	1.73804637415823e-228\\
14.2492492492492	2.84830201020937e-230\\
14.2642642642643	4.59232034215241e-232\\
14.2792792792793	7.28450047977714e-234\\
14.2942942942943	1.13681249855933e-235\\
14.3093093093093	1.74541765930508e-237\\
14.3243243243243	2.63652172275728e-239\\
14.3393393393393	3.91818379904829e-241\\
14.3543543543544	5.72874798636001e-243\\
14.3693693693694	8.24054799221948e-245\\
14.3843843843844	1.16620239992299e-246\\
14.3993993993994	1.62372781742084e-248\\
14.4144144144144	2.22420080484972e-250\\
14.4294294294294	2.99747947866028e-252\\
14.4444444444444	3.97429322007756e-254\\
14.4594594594595	5.18423948591164e-256\\
14.4744744744745	6.65321668000527e-258\\
14.4894894894895	8.40039566471768e-260\\
14.5045045045045	1.04349237271178e-261\\
14.5195195195195	1.27526452345116e-263\\
14.5345345345345	1.53331978597691e-265\\
14.5495495495495	1.81378858291849e-267\\
14.5645645645646	2.11087273433435e-269\\
14.5795795795796	2.41690120934862e-271\\
14.5945945945946	2.72255843664927e-273\\
14.6096096096096	3.01728951570094e-275\\
14.6246246246246	3.28986608046399e-277\\
14.6396396396396	3.52907524235866e-279\\
14.6546546546547	3.72447512968895e-281\\
14.6696696696697	3.86714718390807e-283\\
14.6846846846847	3.95037011501088e-285\\
14.6996996996997	3.97014469218874e-287\\
14.7147147147147	3.92551232404488e-289\\
14.7297297297297	3.81863208914631e-291\\
14.7447447447447	3.65460759741367e-293\\
14.7597597597598	3.44108301596393e-295\\
14.7747747747748	3.18765278526815e-297\\
14.7897897897898	2.90514846781772e-299\\
14.8048048048048	2.60487637220672e-301\\
14.8198198198198	2.2978800959713e-303\\
14.8348348348348	1.99429349469441e-305\\
14.8498498498498	1.70283373036186e-307\\
14.8648648648649	1.4304638213222e-309\\
14.8798798798799	1.18223271945638e-311\\
14.8948948948949	9.61281387290625e-314\\
14.9099099099099	7.68987980403962e-316\\
14.9249249249249	6.05214111989214e-318\\
14.9399399399399	4.68818891338759e-320\\
14.954954954955	3.35964639172048e-322\\
14.96996996997	0\\
14.984984984985	0\\
15	0\\
};
\addlegendentry{$d = 50$};

\addplot [color=mycolor5,solid]
  table[row sep=crcr]{%
0	0\\
0.015015015015015	0\\
0.03003003003003	0\\
0.045045045045045	0\\
0.0600600600600601	0\\
0.0750750750750751	0\\
0.0900900900900901	0\\
0.105105105105105	0\\
0.12012012012012	0\\
0.135135135135135	0\\
0.15015015015015	0\\
0.165165165165165	0\\
0.18018018018018	0\\
0.195195195195195	0\\
0.21021021021021	0\\
0.225225225225225	0\\
0.24024024024024	0\\
0.255255255255255	0\\
0.27027027027027	0\\
0.285285285285285	0\\
0.3003003003003	0\\
0.315315315315315	0\\
0.33033033033033	0\\
0.345345345345345	0\\
0.36036036036036	0\\
0.375375375375375	0\\
0.39039039039039	0\\
0.405405405405405	0\\
0.42042042042042	0\\
0.435435435435435	0\\
0.45045045045045	0\\
0.465465465465465	0\\
0.48048048048048	0\\
0.495495495495495	0\\
0.510510510510511	0\\
0.525525525525526	0\\
0.540540540540541	0\\
0.555555555555556	0\\
0.570570570570571	0\\
0.585585585585586	0\\
0.600600600600601	0\\
0.615615615615616	0\\
0.630630630630631	0\\
0.645645645645646	0\\
0.660660660660661	0\\
0.675675675675676	0\\
0.690690690690691	0\\
0.705705705705706	0\\
0.720720720720721	0\\
0.735735735735736	0\\
0.750750750750751	0\\
0.765765765765766	0\\
0.780780780780781	0\\
0.795795795795796	0\\
0.810810810810811	0\\
0.825825825825826	0\\
0.840840840840841	0\\
0.855855855855856	0\\
0.870870870870871	0\\
0.885885885885886	0\\
0.900900900900901	0\\
0.915915915915916	0\\
0.930930930930931	0\\
0.945945945945946	0\\
0.960960960960961	0\\
0.975975975975976	0\\
0.990990990990991	0\\
1.00600600600601	0\\
1.02102102102102	0\\
1.03603603603604	0\\
1.05105105105105	0\\
1.06606606606607	0\\
1.08108108108108	0\\
1.0960960960961	0\\
1.11111111111111	0\\
1.12612612612613	0\\
1.14114114114114	0\\
1.15615615615616	0\\
1.17117117117117	0\\
1.18618618618619	0\\
1.2012012012012	0\\
1.21621621621622	0\\
1.23123123123123	0\\
1.24624624624625	0\\
1.26126126126126	0\\
1.27627627627628	0\\
1.29129129129129	0\\
1.30630630630631	0\\
1.32132132132132	0\\
1.33633633633634	0\\
1.35135135135135	0\\
1.36636636636637	0\\
1.38138138138138	0\\
1.3963963963964	0\\
1.41141141141141	0\\
1.42642642642643	0\\
1.44144144144144	0\\
1.45645645645646	0\\
1.47147147147147	0\\
1.48648648648649	0\\
1.5015015015015	0\\
1.51651651651652	0\\
1.53153153153153	0\\
1.54654654654655	0\\
1.56156156156156	0\\
1.57657657657658	0\\
1.59159159159159	0\\
1.60660660660661	0\\
1.62162162162162	0\\
1.63663663663664	0\\
1.65165165165165	0\\
1.66666666666667	0\\
1.68168168168168	0\\
1.6966966966967	0\\
1.71171171171171	0\\
1.72672672672673	0\\
1.74174174174174	0\\
1.75675675675676	0\\
1.77177177177177	0\\
1.78678678678679	0\\
1.8018018018018	0\\
1.81681681681682	0\\
1.83183183183183	0\\
1.84684684684685	0\\
1.86186186186186	0\\
1.87687687687688	0\\
1.89189189189189	0\\
1.90690690690691	0\\
1.92192192192192	0\\
1.93693693693694	0\\
1.95195195195195	0\\
1.96696696696697	0\\
1.98198198198198	0\\
1.996996996997	0\\
2.01201201201201	0\\
2.02702702702703	0\\
2.04204204204204	0\\
2.05705705705706	0\\
2.07207207207207	0\\
2.08708708708709	0\\
2.1021021021021	0\\
2.11711711711712	0\\
2.13213213213213	0\\
2.14714714714715	0\\
2.16216216216216	0\\
2.17717717717718	0\\
2.19219219219219	0\\
2.20720720720721	0\\
2.22222222222222	0\\
2.23723723723724	0\\
2.25225225225225	0\\
2.26726726726727	0\\
2.28228228228228	0\\
2.2972972972973	0\\
2.31231231231231	0\\
2.32732732732733	0\\
2.34234234234234	0\\
2.35735735735736	0\\
2.37237237237237	0\\
2.38738738738739	0\\
2.4024024024024	0\\
2.41741741741742	0\\
2.43243243243243	0\\
2.44744744744745	0\\
2.46246246246246	0\\
2.47747747747748	0\\
2.49249249249249	0\\
2.50750750750751	0\\
2.52252252252252	0\\
2.53753753753754	0\\
2.55255255255255	0\\
2.56756756756757	0\\
2.58258258258258	0\\
2.5975975975976	0\\
2.61261261261261	0\\
2.62762762762763	0\\
2.64264264264264	0\\
2.65765765765766	0\\
2.67267267267267	0\\
2.68768768768769	0\\
2.7027027027027	0\\
2.71771771771772	0\\
2.73273273273273	0\\
2.74774774774775	0\\
2.76276276276276	0\\
2.77777777777778	0\\
2.79279279279279	0\\
2.80780780780781	0\\
2.82282282282282	0\\
2.83783783783784	0\\
2.85285285285285	2.96439387504748e-322\\
2.86786786786787	3.90756519295842e-320\\
2.88288288288288	5.32571640081176e-318\\
2.8978978978979	7.14201263266167e-316\\
2.91291291291291	9.41951805944216e-314\\
2.92792792792793	1.2218018443546e-311\\
2.94294294294294	1.55860798211696e-309\\
2.95795795795796	1.9554064440358e-307\\
2.97297297297297	2.41268827898568e-305\\
2.98798798798799	2.92771914724972e-303\\
3.003003003003	3.49398984136093e-301\\
3.01801801801802	4.10088781914981e-299\\
3.03303303303303	4.73367205195649e-297\\
3.04804804804805	5.37381202314427e-295\\
3.06306306306306	5.99971748614459e-293\\
3.07807807807808	6.5878417374588e-291\\
3.09309309309309	7.11409303185337e-289\\
3.10810810810811	7.5554432755923e-287\\
3.12312312312312	7.89158765580112e-285\\
3.13813813813814	8.10648991177598e-283\\
3.15315315315315	8.18964986587968e-281\\
3.16816816816817	8.13695377921238e-279\\
3.18318318318318	7.95101164947294e-277\\
3.1981981981982	7.64094294463167e-275\\
3.21321321321321	7.22163526350819e-273\\
3.22822822822823	6.712559832057e-271\\
3.24324324324324	6.13627501793226e-269\\
3.25825825825826	5.51677774744932e-267\\
3.27327327327327	4.87786960545138e-265\\
3.28828828828829	4.24168986534211e-263\\
3.3033033033033	3.6275354291426e-261\\
3.31831831831832	3.05104378592351e-259\\
3.33333333333333	2.52376692330645e-257\\
3.34834834834835	2.05311877428379e-255\\
3.36336336336336	1.64264202980885e-253\\
3.37837837837838	1.29251565977954e-251\\
3.39339339339339	1.00021352470821e-249\\
3.40840840840841	7.61226078425827e-248\\
3.42342342342342	5.69768742093523e-246\\
3.43843843843844	4.19418524337126e-244\\
3.45345345345345	3.03641146285831e-242\\
3.46846846846847	2.16191026172983e-240\\
3.48348348348348	1.51383565532767e-238\\
3.4984984984985	1.04251862667753e-236\\
3.51351351351351	7.06078427471558e-235\\
3.52852852852853	4.7031203130438e-233\\
3.54354354354354	3.08094014442535e-231\\
3.55855855855856	1.98492663595807e-229\\
3.57357357357357	1.25767863835254e-227\\
3.58858858858859	7.83716401451888e-226\\
3.6036036036036	4.80299586992457e-224\\
3.61861861861862	2.8948730087697e-222\\
3.63363363363363	1.71597456857099e-220\\
3.64864864864865	1.00035973674593e-218\\
3.66366366366366	5.73542475330784e-217\\
3.67867867867868	3.23399240343429e-215\\
3.69369369369369	1.79339696646646e-213\\
3.70870870870871	9.78087914115021e-212\\
3.72372372372372	5.24618233242318e-210\\
3.73873873873874	2.76740610621625e-208\\
3.75375375375375	1.43570904186986e-206\\
3.76876876876877	7.32527616682299e-205\\
3.78378378378378	3.67574686704611e-203\\
3.7987987987988	1.81397465733857e-201\\
3.81381381381381	8.80401677199508e-200\\
3.82882882882883	4.20237238030737e-198\\
3.84384384384384	1.97275081727491e-196\\
3.85885885885886	9.10780988867359e-195\\
3.87387387387387	4.13542080167641e-193\\
3.88888888888889	1.84667081433674e-191\\
3.9039039039039	8.11004547161777e-190\\
3.91891891891892	3.50284660512436e-188\\
3.93393393393393	1.48793162922563e-186\\
3.94894894894895	6.2159692657563e-185\\
3.96396396396396	2.5538698881302e-183\\
3.97897897897898	1.03193585601753e-181\\
3.99399399399399	4.10081959970957e-180\\
4.00900900900901	1.60270153369075e-178\\
4.02402402402402	6.16025467274116e-177\\
4.03903903903904	2.32867407073143e-175\\
4.05405405405405	8.65730612067071e-174\\
4.06906906906907	3.16534365649587e-172\\
4.08408408408408	1.13821154699671e-170\\
4.0990990990991	4.02521546894534e-169\\
4.11411411411411	1.39997185654991e-167\\
4.12912912912913	4.78865449715565e-166\\
4.14414414414414	1.61091167297662e-164\\
4.15915915915916	5.32959219858941e-163\\
4.17417417417417	1.73412443213976e-161\\
4.18918918918919	5.54920234105862e-160\\
4.2042042042042	1.74640469755738e-158\\
4.21921921921922	5.40534287092314e-157\\
4.23423423423423	1.64537778712522e-155\\
4.24924924924925	4.92574629068276e-154\\
4.26426426426426	1.45024866478067e-152\\
4.27927927927928	4.19930032853049e-151\\
4.29429429429429	1.1958464644901e-149\\
4.30930930930931	3.34917575110157e-148\\
4.32432432432432	9.22495972417002e-147\\
4.33933933933934	2.49893549374969e-145\\
4.35435435435435	6.65747644954083e-144\\
4.36936936936937	1.74432842649692e-142\\
4.38438438438438	4.49480550585604e-141\\
4.3993993993994	1.13908878586266e-139\\
4.41441441441441	2.83901884151678e-138\\
4.42942942942943	6.95893851847603e-137\\
4.44444444444444	1.67757424756017e-135\\
4.45945945945946	3.97726499663081e-134\\
4.47447447447447	9.27366346898243e-133\\
4.48948948948949	2.12658213831703e-131\\
4.5045045045045	4.79597618795321e-130\\
4.51951951951952	1.0637410052624e-128\\
4.53453453453453	2.32037837729956e-127\\
4.54954954954955	4.97789515355613e-126\\
4.56456456456456	1.05025971304768e-124\\
4.57957957957958	2.17927328294415e-123\\
4.59459459459459	4.44724144992736e-122\\
4.60960960960961	8.92552456043455e-121\\
4.62462462462462	1.7617359483444e-119\\
4.63963963963964	3.41988820550633e-118\\
4.65465465465465	6.52900541651194e-117\\
4.66966966966967	1.22587496158829e-115\\
4.68468468468468	2.26365035784014e-114\\
4.6996996996997	4.11089652417871e-113\\
4.71471471471471	7.34222636615858e-112\\
4.72972972972973	1.2896831361217e-110\\
4.74474474474474	2.22793388304885e-109\\
4.75975975975976	3.78517185420287e-108\\
4.77477477477477	6.32459808331811e-107\\
4.78978978978979	1.03930804347108e-105\\
4.8048048048048	1.67965339090673e-104\\
4.81981981981982	2.66967921325221e-103\\
4.83483483483483	4.1731357780449e-102\\
4.84984984984985	6.4154921337553e-101\\
4.86486486486486	9.69977063103969e-100\\
4.87987987987988	1.44230456766376e-98\\
4.89489489489489	2.10919396199667e-97\\
4.90990990990991	3.03347266788783e-96\\
4.92492492492492	4.29069534729511e-95\\
4.93993993993994	5.96869378768972e-94\\
4.95495495495495	8.16572865979751e-93\\
4.96996996996997	1.09868859589291e-91\\
4.98498498498498	1.453845695904e-90\\
5	1.89202138053151e-89\\
5.01501501501502	2.42157425521378e-88\\
5.03003003003003	3.0481305812131e-87\\
5.04504504504505	3.77340457611722e-86\\
5.06006006006006	4.5940656998553e-85\\
5.07507507507508	5.50078990558552e-84\\
5.09009009009009	6.4776419770507e-83\\
5.10510510510511	7.50192727215198e-82\\
5.12012012012012	8.54462061797602e-81\\
5.13513513513514	9.57142821033116e-80\\
5.15015015015015	1.05444694806156e-78\\
5.16516516516517	1.14244880653998e-77\\
5.18018018018018	1.21734251226493e-76\\
5.1951951951952	1.27571263109605e-75\\
5.21021021021021	1.3147916845154e-74\\
5.22522522522523	1.33267750816119e-73\\
5.24024024024024	1.32848670909766e-72\\
5.25525525525526	1.30242698532435e-71\\
5.27027027027027	1.25578006423003e-70\\
5.28528528528529	1.19079721137321e-69\\
5.3003003003003	1.11051917026479e-68\\
5.31531531531532	1.01854058161465e-67\\
5.33033033033033	9.18744264391589e-67\\
5.34534534534535	8.15032582386111e-66\\
5.36036036036036	7.11081413481747e-65\\
5.37537537537538	6.10137487871506e-64\\
5.39039039039039	5.14873000590459e-63\\
5.40540540540541	4.27303601897081e-62\\
5.42042042042042	3.48768308586428e-61\\
5.43543543543544	2.79963557188961e-60\\
5.45045045045045	2.21019207380483e-59\\
5.46546546546547	1.7160208391132e-58\\
5.48048048048048	1.31032529328841e-57\\
5.4954954954955	9.84010465427328e-57\\
5.51051051051051	7.26748822481503e-56\\
5.52552552552553	5.27877294878785e-55\\
5.54054054054054	3.77090554773281e-54\\
5.55555555555556	2.64924639979503e-53\\
5.57057057057057	1.83047194700341e-52\\
5.58558558558559	1.24384947851924e-51\\
5.6006006006006	8.31259469463064e-51\\
5.61561561561562	5.4634806500554e-50\\
5.63063063063063	3.5315571540177e-49\\
5.64564564564565	2.24505557097908e-48\\
5.66066066066066	1.40362785408349e-47\\
5.67567567567568	8.63059796922897e-47\\
5.69069069069069	5.2190784135219e-46\\
5.70570570570571	3.10392188755682e-45\\
5.72072072072072	1.81548126347786e-44\\
5.73573573573574	1.04432767061944e-43\\
5.75075075075075	5.90807263102973e-43\\
5.76576576576577	3.28714523192274e-42\\
5.78078078078078	1.79868861374491e-41\\
5.7957957957958	9.67959317800744e-41\\
5.81081081081081	5.12297516438951e-40\\
5.82582582582583	2.66656038256367e-39\\
5.84084084084084	1.36503762059741e-38\\
5.85585585585586	6.87229581682522e-38\\
5.87087087087087	3.40269557474754e-37\\
5.88588588588589	1.65694607082168e-36\\
5.9009009009009	7.93519634242074e-36\\
5.91591591591592	3.73741226898918e-35\\
5.93093093093093	1.73120446337016e-34\\
5.94594594594595	7.88659875038005e-34\\
5.96096096096096	3.53341911726804e-33\\
5.97597597597598	1.55691387684538e-32\\
5.99099099099099	6.74680333576289e-32\\
6.00600600600601	2.87538192731583e-31\\
6.02102102102102	1.20519425504275e-30\\
6.03603603603604	4.96801150798342e-30\\
6.05105105105105	2.0140588664815e-29\\
6.06606606606607	8.03018873822559e-29\\
6.08108108108108	3.14878764135688e-28\\
6.0960960960961	1.21429726776482e-27\\
6.11111111111111	4.60543534596823e-27\\
6.12612612612613	1.71783081126046e-26\\
6.14114114114114	6.3016475951899e-26\\
6.15615615615616	2.27348379780178e-25\\
6.17117117117117	8.06665792669048e-25\\
6.18618618618619	2.81487714090834e-24\\
6.2012012012012	9.66027020368273e-24\\
6.21621621621622	3.26049193440841e-23\\
6.23123123123123	1.08228343828368e-22\\
6.24624624624625	3.53315687960048e-22\\
6.26126126126126	1.13435465228446e-21\\
6.27627627627628	3.58177926463987e-21\\
6.29129129129129	1.11227661281057e-20\\
6.30630630630631	3.39696302478291e-20\\
6.32132132132132	1.02031168417989e-19\\
6.33633633633634	3.01397001049763e-19\\
6.35135135135135	8.75606586312821e-19\\
6.36636636636637	2.50174556731638e-18\\
6.38138138138138	7.02977272832414e-18\\
6.3963963963964	1.94268978450746e-17\\
6.41141141141141	5.2799481134951e-17\\
6.42642642642643	1.41130174738358e-16\\
6.44144144144144	3.71000188357145e-16\\
6.45645645645646	9.59162977703179e-16\\
6.47147147147147	2.43879171379746e-15\\
6.48648648648649	6.09847208971118e-15\\
6.5015015015015	1.49979325692082e-14\\
6.51651651651652	3.6274862849782e-14\\
6.53153153153153	8.62867651595291e-14\\
6.54654654654655	2.01858305340615e-13\\
6.56156156156156	4.6442232068494e-13\\
6.57657657657658	1.05085688009312e-12\\
6.59159159159159	2.33850383227945e-12\\
6.60660660660661	5.11795692401658e-12\\
6.62162162162162	1.10158799124264e-11\\
6.63663663663664	2.33187778910088e-11\\
6.65165165165165	4.85463301507589e-11\\
6.66666666666667	9.93964905219883e-11\\
6.68168168168168	2.00147293879224e-10\\
6.6966966966967	3.96362373812937e-10\\
6.71171171171171	7.71967728264144e-10\\
6.72672672672673	1.47866535009007e-09\\
6.74174174174174	2.78550968757869e-09\\
6.75675675675676	5.16063895896393e-09\\
6.77177177177177	9.40299719224187e-09\\
6.78678678678679	1.68497383950007e-08\\
6.8018018018018	2.96950489081291e-08\\
6.81681681681682	5.14681968829883e-08\\
6.83183183183183	8.77319714521972e-08\\
6.84684684684685	1.47103367226601e-07\\
6.86186186186186	2.42544080803059e-07\\
6.87687687687688	3.93303857761606e-07\\
6.89189189189189	6.27244109936041e-07\\
6.90690690690691	9.83824418349089e-07\\
6.92192192192192	1.51765632719965e-06\\
6.93693693693694	2.302538459898e-06\\
6.95195195195195	3.43574928577457e-06\\
6.96696696696697	5.04221720631611e-06\\
6.98198198198198	7.27801046018134e-06\\
6.996996996997	1.03324136625083e-05\\
7.01201201201201	1.44277165385811e-05\\
7.02702702702703	1.98157756159395e-05\\
7.04204204204204	2.67704674551382e-05\\
7.05705705705706	3.55765302581521e-05\\
7.07207207207207	4.65086490709972e-05\\
7.08708708708709	5.98141636442157e-05\\
7.1021021021021	7.56849708661026e-05\\
7.11711711711712	9.42311936321821e-05\\
7.13213213213213	0.000115455477928991\\
7.14714714714715	0.000139231852945361\\
7.16216216216216	0.00016529270967369\\
7.17717717717718	0.000193230511682138\\
7.19219219219219	0.00022249947830531\\
7.20720720720721	0.000252456399595059\\
7.22222222222222	0.00028239401734857\\
7.23723723723724	0.000311595442753896\\
7.25225225225225	0.000339393326004752\\
7.26726726726727	0.000365229288800655\\
7.28228228228228	0.000388783271789605\\
7.2972972972973	0.000409780857678926\\
7.31231231231231	0.000428275423128181\\
7.32732732732733	0.000444545021127735\\
7.34234234234234	0.000459058912644805\\
7.35735735735736	0.000472558835807382\\
7.37237237237237	0.000485816044078899\\
7.38738738738739	0.00049945077196638\\
7.4024024024024	0.000514381398549929\\
7.41741741741742	0.000531407286505581\\
7.43243243243243	0.000551160557213715\\
7.44744744744745	0.000574255641020734\\
7.46246246246246	0.00060125110517518\\
7.47747747747748	0.000632525000021649\\
7.49249249249249	0.000668746434432887\\
7.50750750750751	0.000710151137580601\\
7.52252252252252	0.000757298700320162\\
7.53753753753754	0.000810729484345403\\
7.55255255255255	0.000871057055014938\\
7.56756756756757	0.000939078333561703\\
7.58258258258258	0.0010157094066621\\
7.5975975975976	0.00110123929860665\\
7.61261261261261	0.00119668776970441\\
7.62762762762763	0.00130215439008672\\
7.64264264264264	0.00141835498421296\\
7.65765765765766	0.00154558181937285\\
7.67267267267267	0.00168341686097967\\
7.68768768768769	0.00183193524271359\\
7.7027027027027	0.00199087245529074\\
7.71771771771772	0.00216014098003759\\
7.73273273273273	0.0023399714390759\\
7.74774774774775	0.0025306187535865\\
7.76276276276276	0.00273281050123716\\
7.77777777777778	0.00294775158329395\\
7.79279279279279	0.00317764270082411\\
7.80780780780781	0.00342490084042767\\
7.82282282282282	0.00369288138450945\\
7.83783783783784	0.00398505228181003\\
7.85285285285285	0.00430594133769819\\
7.86786786786787	0.00465907307861684\\
7.88288288288288	0.00504813778000225\\
7.8978978978979	0.00547628030963809\\
7.91291291291291	0.00594600615098929\\
7.92792792792793	0.00645876200611784\\
7.94294294294294	0.00701549561018294\\
7.95795795795796	0.00761390784980378\\
7.97297297297297	0.00825205792655078\\
7.98798798798799	0.0089261499995201\\
8.003003003003	0.00963177368453695\\
8.01801801801802	0.0103650795340282\\
8.03303303303303	0.0111212011476599\\
8.04804804804805	0.0118974642270084\\
8.06306306306306	0.0126916441527726\\
8.07807807807808	0.0135056897913444\\
8.09309309309309	0.0143420700296787\\
8.10810810810811	0.0152079167252184\\
8.12312312312312	0.0161125504123793\\
8.13813813813814	0.0170690560012183\\
8.15315315315315	0.0180900612743875\\
8.16816816816817	0.0191906826366753\\
8.18318318318318	0.0203858021972868\\
8.1981981981982	0.0216879800812605\\
8.21321321321321	0.0231080547656569\\
8.22822822822823	0.0246517049728658\\
8.24324324324324	0.0263221962438966\\
8.25825825825826	0.0281153282081379\\
8.27327327327327	0.0300237034134504\\
8.28828828828829	0.0320346459680589\\
8.3033033033033	0.0341342315979648\\
8.31831831831832	0.0363046554404735\\
8.33333333333333	0.0385277279926228\\
8.34834834834835	0.0407879612523916\\
8.36336336336336	0.0430684315808589\\
8.37837837837838	0.0453583541775803\\
8.39339339339339	0.0476481759766575\\
8.40840840840841	0.0499347736686047\\
8.42342342342342	0.0522177880983257\\
8.43843843843844	0.0545012701342025\\
8.45345345345345	0.0567928735557909\\
8.46846846846847	0.0591021031621269\\
8.48348348348348	0.06144050372421\\
8.4984984984985	0.063820687392078\\
8.51351351351351	0.0662561602973415\\
8.52852852852853	0.0687609260683435\\
8.54354354354354	0.0713468177935308\\
8.55855855855856	0.0740234737120131\\
8.57357357357357	0.0768044418701766\\
8.58858858858859	0.0796975172308446\\
8.6036036036036	0.0827106384758688\\
8.61861861861862	0.0858530069794741\\
8.63363363363363	0.089130018991583\\
8.64864864864865	0.0925487539756008\\
8.66366366366366	0.0961143553340911\\
8.67867867867868	0.0998324005425884\\
8.69369369369369	0.103708115158987\\
8.70870870870871	0.107746137598142\\
8.72372372372372	0.111951848711175\\
8.73873873873874	0.11632415610219\\
8.75375375375375	0.120866492127206\\
8.76876876876877	0.125576117515971\\
8.78378378378378	0.130453728764331\\
8.7987987987988	0.135496686838167\\
8.81381381381381	0.14070108914844\\
8.82882882882883	0.146061018565158\\
8.84384384384384	0.151569204321899\\
8.85885885885886	0.157220012072361\\
8.87387387387387	0.163004375112182\\
8.88888888888889	0.168918338054393\\
8.9039039039039	0.174953793320676\\
8.91891891891892	0.181103779880956\\
8.93393393393393	0.187362316038695\\
8.94894894894895	0.193723978485691\\
8.96396396396396	0.200181226287495\\
8.97897897897898	0.206726701119758\\
8.99399399399399	0.21335379989549\\
9.00900900900901	0.220052645529047\\
9.02402402402402	0.22681506595105\\
9.03903903903904	0.233629749855353\\
9.05405405405405	0.240485747694309\\
9.06906906906907	0.247373921913336\\
9.08408408408408	0.25428451823107\\
9.0990990990991	0.261208982320135\\
9.11411411411411	0.268143803032523\\
9.12912912912913	0.27508578476654\\
9.14414414414414	0.282037299129171\\
9.15915915915916	0.288998947116589\\
9.17417417417417	0.295978777533941\\
9.18918918918919	0.302991374467637\\
9.2042042042042	0.310045812382993\\
9.21921921921922	0.317160966460994\\
9.23423423423423	0.324343154914161\\
9.24924924924925	0.331605474917386\\
9.26426426426426	0.338961157168819\\
9.27927927927928	0.346411298583442\\
9.29429429429429	0.353961474344225\\
9.30930930930931	0.361603864660804\\
9.32432432432432	0.369336774623359\\
9.33933933933934	0.377144911321985\\
9.35435435435435	0.385016804363776\\
9.36936936936937	0.392938961878527\\
9.38438438438438	0.400893711465339\\
9.3993993993994	0.408861494387735\\
9.41441441441441	0.416823291677462\\
9.42942942942943	0.424758351453888\\
9.44444444444444	0.432647493257606\\
9.45945945945946	0.440464934915009\\
9.47447447447447	0.448185053024187\\
9.48948948948949	0.45578469025876\\
9.5045045045045	0.463227888716123\\
9.51951951951952	0.470483096162788\\
9.53453453453453	0.477512676493401\\
9.54954954954955	0.484280138741934\\
9.56456456456456	0.490749145097994\\
9.57957957957958	0.496882203509848\\
9.59459459459459	0.502648954838102\\
9.60960960960961	0.50802354415857\\
9.62462462462462	0.512992030023126\\
9.63963963963964	0.517544220222844\\
9.65465465465465	0.521678321122345\\
9.66966966966967	0.525410984554462\\
9.68468468468468	0.528759477975944\\
9.6996996996997	0.531754853862886\\
9.71471471471471	0.534424460014957\\
9.72972972972973	0.53680498285336\\
9.74474474474474	0.538937616147823\\
9.75975975975976	0.540858302121594\\
9.77477477477477	0.542602981497139\\
9.78978978978979	0.544204994514567\\
9.8048048048048	0.545684468433325\\
9.81981981981982	0.547063177934825\\
9.83483483483483	0.54835665639234\\
9.84984984984985	0.549573800566708\\
9.86486486486486	0.550718072146693\\
9.87987987987988	0.551792269412242\\
9.89489489489489	0.552799362340709\\
9.90990990990991	0.553736956905461\\
9.92492492492492	0.554609163186798\\
9.93993993993994	0.555419037004767\\
9.95495495495495	0.556170374922166\\
9.96996996996997	0.556873540237535\\
9.98498498498498	0.557535987858355\\
10	0.558164651736991\\
10.015015015015	0.558766468323879\\
10.03003003003	0.559335702604645\\
10.045045045045	0.559864026564769\\
10.0600600600601	0.560328101610851\\
10.0750750750751	0.560693699364551\\
10.0900900900901	0.560913124101731\\
10.1051051051051	0.560917870549541\\
10.1201201201201	0.560633027329034\\
10.1351351351351	0.559978509639779\\
10.1501501501502	0.558865781760357\\
10.1651651651652	0.557216422416839\\
10.1801801801802	0.554957863539037\\
10.1951951951952	0.55203347386316\\
10.2102102102102	0.548408888609917\\
10.2252252252252	0.544078944327333\\
10.2402402402402	0.539065724878861\\
10.2552552552553	0.533411145667844\\
10.2702702702703	0.527194651674593\\
10.2852852852853	0.520504988480277\\
10.3003003003003	0.513447821338116\\
10.3153153153153	0.506140152457891\\
10.3303303303303	0.498694216422817\\
10.3453453453453	0.491212800059653\\
10.3603603603604	0.48378243293402\\
10.3753753753754	0.476469824919845\\
10.3903903903904	0.469319618908868\\
10.4054054054054	0.462349518873887\\
10.4204204204204	0.455557034902004\\
10.4354354354354	0.448916181426648\\
10.4504504504505	0.442389748037734\\
10.4654654654655	0.435931718094485\\
10.4804804804805	0.429485317908824\\
10.4954954954955	0.423003452640276\\
10.5105105105105	0.416440479279749\\
10.5255255255255	0.409762849573563\\
10.5405405405405	0.40294866486268\\
10.5555555555556	0.395985049578963\\
10.5705705705706	0.388876595550072\\
10.5855855855856	0.381638315150745\\
10.6006006006006	0.374287365163037\\
10.6156156156156	0.366853067573554\\
10.6306306306306	0.35936167787978\\
10.6456456456456	0.351842515349098\\
10.6606606606607	0.34432335893882\\
10.6756756756757	0.336823081340709\\
10.6906906906907	0.32935749354675\\
10.7057057057057	0.321940917962699\\
10.7207207207207	0.314578326928672\\
10.7357357357357	0.307278151794054\\
10.7507507507508	0.300034761019735\\
10.7657657657658	0.292847436418867\\
10.7807807807808	0.285711439408136\\
10.7957957957958	0.278619757564493\\
10.8108108108108	0.271568756163572\\
10.8258258258258	0.264551244815914\\
10.8408408408408	0.257568323498461\\
10.8558558558559	0.250619698089492\\
10.8708708708709	0.243709972874304\\
10.8858858858859	0.236852829345653\\
10.9009009009009	0.230057052803373\\
10.9159159159159	0.223341791504232\\
10.9309309309309	0.21673022494542\\
10.9459459459459	0.210243420394008\\
10.960960960961	0.203909786226335\\
10.975975975976	0.19774975652591\\
10.990990990991	0.191787056975403\\
11.006006006006	0.186037380791367\\
11.021021021021	0.180516238045228\\
11.036036036036	0.175227127403296\\
11.0510510510511	0.170167505704656\\
11.0660660660661	0.165329700137611\\
11.0810810810811	0.160697529585244\\
11.0960960960961	0.156254408411189\\
11.1111111111111	0.151975587515766\\
11.1261261261261	0.147836243459161\\
11.1411411411411	0.143816350812728\\
11.1561561561562	0.139890717256139\\
11.1711711711712	0.136041918597074\\
11.1861861861862	0.132257673130551\\
11.2012012012012	0.128523274935653\\
11.2162162162162	0.12483483581722\\
11.2312312312312	0.121188917391948\\
11.2462462462462	0.117584556656256\\
11.2612612612613	0.114022910498862\\
11.2762762762763	0.110506605727152\\
11.2912912912913	0.107036992713407\\
11.3063063063063	0.103611678718927\\
11.3213213213213	0.100230707245384\\
11.3363363363363	0.0968912684306485\\
11.3513513513514	0.0935846837954134\\
11.3663663663664	0.0903082861071964\\
11.3813813813814	0.0870510765670081\\
11.3963963963964	0.0838064844617453\\
11.4114114114114	0.0805698951362652\\
11.4264264264264	0.0773364146408948\\
11.4414414414414	0.0741073835869404\\
11.4564564564565	0.0708841358835248\\
11.4714714714715	0.0676750779203386\\
11.4864864864865	0.0644912182506997\\
11.5015015015015	0.061347386209619\\
11.5165165165165	0.0582597573554899\\
11.5315315315315	0.0552470326719185\\
11.5465465465465	0.0523270069308525\\
11.5615615615616	0.0495185661022407\\
11.5765765765766	0.0468380331036081\\
11.5915915915916	0.0442955994858539\\
11.6066066066066	0.0419005072011608\\
11.6216216216216	0.0396580337864786\\
11.6366366366366	0.0375647344512481\\
11.6516516516517	0.0356155252337029\\
11.6666666666667	0.03380413363568\\
11.6816816816817	0.0321202914423084\\
11.6966966966967	0.0305506673727377\\
11.7117117117117	0.0290839807688392\\
11.7267267267267	0.0277093802969703\\
11.7417417417417	0.0264148976272189\\
11.7567567567568	0.0251921787812966\\
11.7717717717718	0.0240323852695169\\
11.7867867867868	0.02293065097128\\
11.8018018018018	0.0218823011449144\\
11.8168168168168	0.0208815610867\\
11.8318318318318	0.019926631269134\\
11.8468468468468	0.0190139684053668\\
11.8618618618619	0.0181399048265214\\
11.8768768768769	0.0172999800819944\\
11.8918918918919	0.0164908956256253\\
11.9069069069069	0.0157084049861726\\
11.9219219219219	0.0149475795316073\\
11.9369369369369	0.0142049911073305\\
11.951951951952	0.0134769398961851\\
11.966966966967	0.0127631771366197\\
11.981981981982	0.0120630238806451\\
11.996996996997	0.0113772421757559\\
12.012012012012	0.0107088225225852\\
12.027027027027	0.0100621293331671\\
12.042042042042	0.00944159506711196\\
12.0570570570571	0.00885174855739605\\
12.0720720720721	0.00829834969087353\\
12.0870870870871	0.0077853462505741\\
12.1021021021021	0.0073160847569764\\
12.1171171171171	0.00689233305516391\\
12.1321321321321	0.00651393794777902\\
12.1471471471471	0.00617949439331709\\
12.1621621621622	0.00588515693659371\\
12.1771771771772	0.00562537520371465\\
12.1921921921922	0.00539500374956728\\
12.2072072072072	0.00518757692464622\\
12.2222222222222	0.00499738363746525\\
12.2372372372372	0.00481840010669674\\
12.2522522522523	0.00464644114122137\\
12.2672672672673	0.00447785140335204\\
12.2822822822823	0.00431105471309294\\
12.2972972972973	0.00414538910485396\\
12.3123123123123	0.00398185281911451\\
12.3273273273273	0.00382117979685497\\
12.3423423423423	0.00366625403807134\\
12.3573573573574	0.00351907418909916\\
12.3723723723724	0.00338132153086848\\
12.3873873873874	0.00325514615109333\\
12.4024024024024	0.00314186058174462\\
12.4174174174174	0.00304187984699856\\
12.4324324324324	0.00295476140079874\\
12.4474474474474	0.00287958694021984\\
12.4624624624625	0.00281478220652229\\
12.4774774774775	0.00275866953561571\\
12.4924924924925	0.00270854324690457\\
12.5075075075075	0.00266144987262049\\
12.5225225225225	0.00261517342527909\\
12.5375375375375	0.00256726661174626\\
12.5525525525526	0.00251503752206798\\
12.5675675675676	0.00245667698240538\\
12.5825825825826	0.0023904681206445\\
12.5975975975976	0.0023152623239509\\
12.6126126126126	0.00223031281762476\\
12.6276276276276	0.00213563610087322\\
12.6426426426426	0.00203134282551013\\
12.6576576576577	0.0019183714980584\\
12.6726726726727	0.00179760522541503\\
12.6876876876877	0.00167083430284028\\
12.7027027027027	0.00153980791823807\\
12.7177177177177	0.00140610793593931\\
12.7327327327327	0.00127200704630558\\
12.7477477477477	0.00113961495072947\\
12.7627627627628	0.00101062397746076\\
12.7777777777778	0.000886961506887272\\
12.7927927927928	0.000770000059402064\\
12.8078078078078	0.000661018366031565\\
12.8228228228228	0.000560987884747888\\
12.8378378378378	0.000470562548223162\\
12.8528528528529	0.000390085261710686\\
12.8678678678679	0.000319505024258024\\
12.8828828828829	0.000258778307429077\\
12.8978978978979	0.00020749920686098\\
12.9129129129129	0.000165142679130335\\
12.9279279279279	0.00013094838310786\\
12.9429429429429	0.000104233171779511\\
12.957957957958	8.43156271128716e-05\\
12.972972972973	7.04875769048688e-05\\
12.987987987988	6.22332590693617e-05\\
13.003003003003	5.92244015580157e-05\\
13.018018018018	6.08853087788441e-05\\
13.033033033033	6.68879195674475e-05\\
13.048048048048	7.68590048648536e-05\\
13.0630630630631	9.0717344839072e-05\\
13.0780780780781	0.000107865625371991\\
13.0930930930931	0.000128286869112498\\
13.1081081081081	0.000151261322343174\\
13.1231231231231	0.00017619038831789\\
13.1381381381381	0.000202337516719049\\
13.1531531531532	0.000228758457711935\\
13.1681681681682	0.000254676024221194\\
13.1831831831832	0.000278937970098126\\
13.1981981981982	0.000300433731857479\\
13.2132132132132	0.000318401765572823\\
13.2282282282282	0.000331868680384022\\
13.2432432432432	0.000340189649611601\\
13.2582582582583	0.000342957219656434\\
13.2732732732733	0.000340034380116435\\
13.2882882882883	0.000331565806248603\\
13.3033033033033	0.000317965989349111\\
13.3183183183183	0.000299885610927106\\
13.3333333333333	0.000278159958804098\\
13.3483483483483	0.000253745076958562\\
13.3633633633634	0.00022764843184225\\
13.3783783783784	0.000200861049389497\\
13.3933933933934	0.000174297363540907\\
13.4084084084084	0.000148747588908794\\
13.4234234234234	0.000124845553204384\\
13.4384384384384	0.000103052907103673\\
13.4534534534535	8.36587639293272e-05\\
13.4684684684685	6.67923410650838e-05\\
13.4834834834835	5.2445220320814e-05\\
13.4984984984985	4.04994559378082e-05\\
13.5135135135135	3.07578865100408e-05\\
13.5285285285285	2.29735345337226e-05\\
13.5435435435435	1.6875753430882e-05\\
13.5585585585586	1.2191651825816e-05\\
13.5735735735736	8.6621551415553e-06\\
13.5885885885886	6.05275911630607e-06\\
13.6036036036036	4.15953563716691e-06\\
13.6186186186186	2.81125560823728e-06\\
13.6336336336336	1.86861506761227e-06\\
13.6486486486486	1.22152788201014e-06\\
13.6636636636637	7.85327758901957e-07\\
13.6786786786787	4.96549488971099e-07\\
13.6936936936937	3.0877217441513e-07\\
13.7087087087087	1.88832959354988e-07\\
13.7237237237237	1.13574666151427e-07\\
13.7387387387387	6.71814177625956e-08\\
13.7537537537538	3.90823695429029e-08\\
13.7687687687688	2.23602489936866e-08\\
13.7837837837838	1.25816155381738e-08\\
13.7987987987988	6.96241989046064e-09\\
13.8138138138138	3.7892044496989e-09\\
13.8288288288288	2.02814916175932e-09\\
13.8438438438438	1.06761771734782e-09\\
13.8588588588589	5.52707908430236e-10\\
13.8738738738739	2.81410055223011e-10\\
13.8888888888889	1.40911867417796e-10\\
13.9039039039039	6.93936219340081e-11\\
13.9189189189189	3.36089981877715e-11\\
13.9339339339339	1.60086830103339e-11\\
13.9489489489489	7.49928173208683e-12\\
13.963963963964	3.45499767912571e-12\\
13.978978978979	1.56545243632954e-12\\
13.993993993994	6.97583211470796e-13\\
14.009009009009	3.05714606826334e-13\\
14.024024024024	1.31765096767945e-13\\
14.039039039039	5.5853270374483e-14\\
14.0540540540541	2.32841750614669e-14\\
14.0690690690691	9.54634511232053e-15\\
14.0840840840841	3.84926163935859e-15\\
14.0990990990991	1.52644710314405e-15\\
14.1141141141141	5.95319506029709e-16\\
14.1291291291291	2.28340260787883e-16\\
14.1441441441441	8.61348494633837e-17\\
14.1591591591592	3.19550375098635e-17\\
14.1741741741742	1.16590668011635e-17\\
14.1891891891892	4.18362043715239e-18\\
14.2042042042042	1.47640255280003e-18\\
14.2192192192192	5.12414399335662e-19\\
14.2342342342342	1.74904864973979e-19\\
14.2492492492492	5.87146504960168e-20\\
14.2642642642643	1.93845220270227e-20\\
14.2792792792793	6.29401451341446e-21\\
14.2942942942943	2.00985353245374e-21\\
14.3093093093093	6.31197260430937e-22\\
14.3243243243243	1.9495295721721e-22\\
14.3393393393393	5.92186599705491e-23\\
14.3543543543544	1.7690957959338e-23\\
14.3693693693694	5.19766340298362e-24\\
14.3843843843844	1.50185836847074e-24\\
14.3993993993994	4.26789604387192e-25\\
14.4144144144144	1.19278649264879e-25\\
14.4294294294294	3.27850325148882e-26\\
14.4444444444444	8.862424559439e-27\\
14.4594594594595	2.35609886588376e-27\\
14.4744744744745	6.16025224196233e-28\\
14.4894894894895	1.58404497082487e-28\\
14.5045045045045	4.00590406887862e-29\\
14.5195195195195	9.96317156759408e-30\\
14.5345345345345	2.43701778992776e-30\\
14.5495495495495	5.86251296658004e-31\\
14.5645645645646	1.38698881679107e-31\\
14.5795795795796	3.22720184250203e-32\\
14.5945945945946	7.38487799152558e-33\\
14.6096096096096	1.66197522929838e-33\\
14.6246246246246	3.67849182297584e-34\\
14.6396396396396	8.00717037896796e-35\\
14.6546546546547	1.71416400243324e-35\\
14.6696696696697	3.60902340024624e-36\\
14.6846846846847	7.47293236423415e-37\\
14.6996996996997	1.52179597644209e-37\\
14.7147147147147	3.04779560012472e-38\\
14.7297297297297	6.00315103903663e-39\\
14.7447447447447	1.16288489447902e-39\\
14.7597597597598	2.21543092856462e-40\\
14.7747747747748	4.15091390713726e-41\\
14.7897897897898	7.64879926971034e-42\\
14.8048048048048	1.38613914355051e-42\\
14.8198198198198	2.47049749747874e-43\\
14.8348348348348	4.33038026701561e-44\\
14.8498498498498	7.46503207571443e-45\\
14.8648648648649	1.26561422514985e-45\\
14.8798798798799	2.11025551156551e-46\\
14.8948948948949	3.4604514879826e-47\\
14.9099099099099	5.58077500201138e-48\\
14.9249249249249	8.85156959304902e-49\\
14.9399399399399	1.38073422218805e-49\\
14.954954954955	2.11818513765359e-50\\
14.96996996997	3.1958160244091e-51\\
14.984984984985	4.74202293307236e-52\\
15	6.9200551587926e-53\\
};
\addlegendentry{$d = 100$};

\end{axis}
\end{tikzpicture}%
  \caption{Empirical distributions of lengths $y$ as a function of the
    dimension $d$.}
  \label{problem_1}
\end{figure}

\clearpage
\begin{enumerate}
\setcounter{enumi}{1}
\item
  (Bayesian linear regression.)
  Consider the following data:
  \begin{align*}
    \vec{x}
    &=
    [-2.26, -1.31, -0.43, 0.32, 0.34, 0.54, 0.86, 1.83, 2.77, 3.58]\trans; \\
    \vec{y}
    &=
    [1.03, 0.70, -0.68, -1.36, -1.74, -1.01, 0.24, 1.55, 1.68, 1.53]\trans.
  \end{align*}
  Fix the noise variance at $\sigma^2 = 0.5^2$.
  \begin{itemize}
  \item
    Perform Bayesian linear regression for these data using the
    polynomial basis functions $\phi_k(x) = [1, x, x^2, \dotsc
      x^k]\trans$ for $k \in \{1, 2, 3\}$, in each case using the
    parameter prior $p(\vec{w}) = \mc{N}(\vec{w}; \vec{0}, \mat{I})$.
    Evaluate and plot the posterior means $\mathbb{E}[\vec{y}_\ast
      \given \mat{X}_\ast, \data, \sigma^2]$ on the interval $x_\ast
    \in [-4, 4]$ for each model.  Also plot the posterior mean
    plus-or-minus two times the posterior standard deviation:
    \begin{equation*}
      \mathbb{E}[\vec{y}_\ast \given \mat{X}_\ast, \data, \sigma^2] \pm
      2 \sqrt{\var[\vec{y}_\ast \given \mat{X}_\ast, \data, \sigma^2]}.
    \end{equation*}
    This is a pointwise 95\% credible interval for the regression
    function.  Where is the pointwise uncertainty the largest?
  \item
    Compute the marginal likelihood of the data for each of the basis
    expansions above: $p(\vec{y} \given \mat{X}, k, \sigma^2)$.  Which
    model explains the data the best?
  \end{itemize}
\end{enumerate}

\subsection*{Solution}

Given the feature expansions $\mat{\Phi} = \phi(\mat{X})$ and
$\mat{\Phi}_\ast = \phi(\mat{X}_\ast)$, we may compute the posterior
distribution of $\vec{y}_\ast$:
\begin{equation*}
  p(\vec{y}_\ast \given \vec{X}_\ast, \data)
  =
  \mc{N}(\vec{y}_\ast;
  \mat{\Phi}_\ast \vec{\mu}_{\vec{w}\given\data},
  \mat{X}_\ast \mat{\Sigma}_{\vec{w}\given\data} \mat{X}_\ast\trans + \sigma^2 \mat{I}),
\end{equation*}
where
\begin{align*}
  \vec{\mu}_{\vec{w}\given\data}
  &=
  \vec{\mu}
  +
  \mat{\Sigma}
  \mat{\Phi}\trans
  (\mat{\Phi}\mat{\Sigma}\mat{\Phi}\trans + \sigma^2 \mat{I})\inv
  (\vec{y} - \mat{\Phi}\vec{\mu});
  \\
  \mat{\Sigma}_{\vec{w}\given\data}
  &=
  \mat{\Sigma}
  -
  \mat{\Sigma}
  \mat{\Phi}\trans
  (\mat{\Phi}\mat{\Sigma}\mat{\Phi}\trans + \sigma^2 \mat{I})\inv
  \mat{\Phi}
  \mat{\Sigma}.
\end{align*}
The diagonal of the posterior covariance for $\vec{y}\ast$,
$\mat{\Phi}_\ast\mat{\Sigma}\mat{\Phi}_\ast + \sigma^2\mat{I}$,
gives the desired variance for plotting the credible interval.

Plugging in the prior $p(\vec{w}) = \mc{N}(\vec{w}; \vec{0}, \mat{I})$
gives the result, plotted below.  For all three models, the pointwise
uncertainty is maximized on the extreme ranges of the domain at $x =
-4$ and $x = 4$; we have few observations near either of these
locations.  The pointwise uncertainty tends to be especially large at
$x = -4$.

\begin{figure}
  \centering
  % This file was created by matlab2tikz.
% Minimal pgfplots version: 1.3
%
\tikzsetnextfilename{order_1_expansion}
\definecolor{mycolor1}{rgb}{0.65098,0.80784,0.89020}%
\definecolor{mycolor2}{rgb}{0.12157,0.47059,0.70588}%
%
\begin{tikzpicture}

\begin{axis}[%
width=0.95092\figurewidth,
height=\figureheight,
at={(0\figurewidth,0\figureheight)},
scale only axis,
xmin=-4,
xmax=4,
xlabel={$x$},
ymin=-15,
ymax=15,
axis x line*=bottom,
axis y line*=left,
legend style={legend cell align=left,align=left,draw=white!15!black},
legend style={legend columns=-1, draw=none}, reverse legend
]

\addplot[area legend,solid,fill=mycolor1,opacity=3.000000e-01,draw=none]
table[row sep=crcr] {%
x	y\\
-4	-12.6851318914754\\
-3.99199199199199	-12.6611881266474\\
-3.98398398398398	-12.6372472357903\\
-3.97597597597598	-12.613309235137\\
-3.96796796796797	-12.5893741410409\\
-3.95995995995996	-12.5654419699766\\
-3.95195195195195	-12.5415127385413\\
-3.94394394394394	-12.5175864634558\\
-3.93593593593594	-12.4936631615657\\
-3.92792792792793	-12.4697428498424\\
-3.91991991991992	-12.4458255453846\\
-3.91191191191191	-12.4219112654189\\
-3.9039039039039	-12.3980000273017\\
-3.8958958958959	-12.3740918485196\\
-3.88788788788789	-12.3501867466913\\
-3.87987987987988	-12.3262847395685\\
-3.87187187187187	-12.302385845037\\
-3.86386386386386	-12.2784900811183\\
-3.85585585585586	-12.2545974659706\\
-3.84784784784785	-12.2307080178901\\
-3.83983983983984	-12.2068217553124\\
-3.83183183183183	-12.1829386968137\\
-3.82382382382382	-12.1590588611121\\
-3.81581581581582	-12.1351822670692\\
-3.80780780780781	-12.111308933691\\
-3.7997997997998	-12.0874388801296\\
-3.79179179179179	-12.0635721256844\\
-3.78378378378378	-12.0397086898036\\
-3.77577577577578	-12.0158485920856\\
-3.76776776776777	-11.9919918522804\\
-3.75975975975976	-11.9681384902908\\
-3.75175175175175	-11.9442885261745\\
-3.74374374374374	-11.9204419801446\\
-3.73573573573574	-11.8965988725722\\
-3.72772772772773	-11.8727592239869\\
-3.71971971971972	-11.8489230550789\\
-3.71171171171171	-11.8250903867006\\
-3.7037037037037	-11.8012612398675\\
-3.6956956956957	-11.7774356357607\\
-3.68768768768769	-11.7536135957276\\
-3.67967967967968	-11.7297951412843\\
-3.67167167167167	-11.7059802941165\\
-3.66366366366366	-11.6821690760816\\
-3.65565565565566	-11.6583615092104\\
-3.64764764764765	-11.6345576157084\\
-3.63963963963964	-11.6107574179578\\
-3.63163163163163	-11.5869609385193\\
-3.62362362362362	-11.5631682001335\\
-3.61561561561562	-11.5393792257229\\
-3.60760760760761	-11.5155940383937\\
-3.5995995995996	-11.4918126614374\\
-3.59159159159159	-11.468035118333\\
-3.58358358358358	-11.4442614327481\\
-3.57557557557558	-11.4204916285419\\
-3.56756756756757	-11.3967257297659\\
-3.55955955955956	-11.3729637606666\\
-3.55155155155155	-11.3492057456871\\
-3.54354354354354	-11.3254517094693\\
-3.53553553553554	-11.3017016768553\\
-3.52752752752753	-11.2779556728903\\
-3.51951951951952	-11.2542137228237\\
-3.51151151151151	-11.2304758521117\\
-3.5035035035035	-11.2067420864193\\
-3.4954954954955	-11.1830124516222\\
-3.48748748748749	-11.1592869738091\\
-3.47947947947948	-11.1355656792838\\
-3.47147147147147	-11.1118485945672\\
-3.46346346346346	-11.0881357463998\\
-3.45545545545546	-11.0644271617436\\
-3.44744744744745	-11.0407228677846\\
-3.43943943943944	-11.0170228919349\\
-3.43143143143143	-10.993327261835\\
-3.42342342342342	-10.9696360053563\\
-3.41541541541542	-10.9459491506032\\
-3.40740740740741	-10.9222667259158\\
-3.3993993993994	-10.898588759872\\
-3.39139139139139	-10.8749152812899\\
-3.38338338338338	-10.8512463192307\\
-3.37537537537538	-10.8275819030008\\
-3.36736736736737	-10.8039220621544\\
-3.35935935935936	-10.7802668264961\\
-3.35135135135135	-10.7566162260835\\
-3.34334334334334	-10.7329702912299\\
-3.33533533533534	-10.7093290525068\\
-3.32732732732733	-10.6856925407464\\
-3.31931931931932	-10.6620607870449\\
-3.31131131131131	-10.6384338227647\\
-3.3033033033033	-10.6148116795374\\
-3.2952952952953	-10.5911943892666\\
-3.28728728728729	-10.5675819841307\\
-3.27927927927928	-10.5439744965858\\
-3.27127127127127	-10.5203719593688\\
-3.26326326326326	-10.4967744054999\\
-3.25525525525526	-10.4731818682861\\
-3.24724724724725	-10.4495943813239\\
-3.23923923923924	-10.4260119785026\\
-3.23123123123123	-10.4024346940071\\
-3.22322322322322	-10.3788625623213\\
-3.21521521521522	-10.3552956182311\\
-3.20720720720721	-10.3317338968279\\
-3.1991991991992	-10.3081774335114\\
-3.19119119119119	-10.2846262639933\\
-3.18318318318318	-10.2610804243006\\
-3.17517517517518	-10.2375399507787\\
-3.16716716716717	-10.2140048800951\\
-3.15915915915916	-10.1904752492427\\
-3.15115115115115	-10.1669510955434\\
-3.14314314314314	-10.1434324566516\\
-3.13513513513514	-10.119919370558\\
-3.12712712712713	-10.0964118755929\\
-3.11911911911912	-10.07291001043\\
-3.11111111111111	-10.0494138140903\\
-3.1031031031031	-10.0259233259458\\
-3.0950950950951	-10.0024385857232\\
-3.08708708708709	-9.97895963350792\\
-3.07907907907908	-9.95548650974803\\
-3.07107107107107	-9.93201925525813\\
-3.06306306306306	-9.90855791122343\\
-3.05505505505506	-9.88510251920385\\
-3.04704704704705	-9.8616531211381\\
-3.03903903903904	-9.83820975934791\\
-3.03103103103103	-9.81477247654225\\
-3.02302302302302	-9.7913413158216\\
-3.01501501501502	-9.76791632068233\\
-3.00700700700701	-9.74449753502108\\
-2.998998998999	-9.7210850031392\\
-2.99099099099099	-9.69767876974733\\
-2.98298298298298	-9.6742788799699\\
-2.97497497497497	-9.6508853793498\\
-2.96696696696697	-9.62749831385309\\
-2.95895895895896	-9.60411772987369\\
-2.95095095095095	-9.58074367423827\\
-2.94294294294294	-9.55737619421107\\
-2.93493493493493	-9.53401533749888\\
-2.92692692692693	-9.51066115225602\\
-2.91891891891892	-9.48731368708943\\
-2.91091091091091	-9.46397299106377\\
-2.9029029029029	-9.44063911370668\\
-2.89489489489489	-9.417312105014\\
-2.88688688688689	-9.39399201545516\\
-2.87887887887888	-9.37067889597852\\
-2.87087087087087	-9.34737279801694\\
-2.86286286286286	-9.32407377349328\\
-2.85485485485485	-9.30078187482601\\
-2.84684684684685	-9.27749715493499\\
-2.83883883883884	-9.25421966724715\\
-2.83083083083083	-9.23094946570242\\
-2.82282282282282	-9.20768660475965\\
-2.81481481481481	-9.18443113940257\\
-2.80680680680681	-9.16118312514593\\
-2.7987987987988	-9.13794261804166\\
-2.79079079079079	-9.11470967468512\\
-2.78278278278278	-9.09148435222143\\
-2.77477477477477	-9.06826670835191\\
-2.76676676676677	-9.04505680134057\\
-2.75875875875876	-9.02185469002071\\
-2.75075075075075	-8.99866043380161\\
-2.74274274274274	-8.9754740926753\\
-2.73473473473473	-8.95229572722343\\
-2.72672672672673	-8.9291253986242\\
-2.71871871871872	-8.90596316865945\\
-2.71071071071071	-8.88280909972177\\
-2.7027027027027	-8.85966325482179\\
-2.69469469469469	-8.83652569759545\\
-2.68668668668669	-8.81339649231151\\
-2.67867867867868	-8.79027570387904\\
-2.67067067067067	-8.76716339785512\\
-2.66266266266266	-8.74405964045251\\
-2.65465465465465	-8.72096449854757\\
-2.64664664664665	-8.69787803968819\\
-2.63863863863864	-8.67480033210188\\
-2.63063063063063	-8.65173144470391\\
-2.62262262262262	-8.62867144710564\\
-2.61461461461461	-8.6056204096229\\
-2.60660660660661	-8.58257840328456\\
-2.5985985985986	-8.55954549984106\\
-2.59059059059059	-8.5365217717733\\
-2.58258258258258	-8.51350729230142\\
-2.57457457457457	-8.49050213539383\\
-2.56656656656657	-8.46750637577634\\
-2.55855855855856	-8.44452008894139\\
-2.55055055055055	-8.42154335115747\\
-2.54254254254254	-8.39857623947856\\
-2.53453453453453	-8.37561883175383\\
-2.52652652652653	-8.35267120663738\\
-2.51851851851852	-8.32973344359816\\
-2.51051051051051	-8.30680562292999\\
-2.5025025025025	-8.2838878257618\\
-2.49449449449449	-8.2609801340679\\
-2.48648648648649	-8.23808263067849\\
-2.47847847847848	-8.21519539929026\\
-2.47047047047047	-8.19231852447716\\
-2.46246246246246	-8.16945209170131\\
-2.45445445445445	-8.14659618732408\\
-2.44644644644645	-8.12375089861732\\
-2.43843843843844	-8.10091631377471\\
-2.43043043043043	-8.07809252192336\\
-2.42242242242242	-8.05527961313544\\
-2.41441441441441	-8.0324776784401\\
-2.40640640640641	-8.00968680983549\\
-2.3983983983984	-7.98690710030099\\
-2.39039039039039	-7.96413864380952\\
-2.38238238238238	-7.94138153534015\\
-2.37437437437437	-7.91863587089084\\
-2.36636636636637	-7.89590174749128\\
-2.35835835835836	-7.87317926321605\\
-2.35035035035035	-7.85046851719785\\
-2.34234234234234	-7.82776960964098\\
-2.33433433433433	-7.80508264183502\\
-2.32632632632633	-7.78240771616864\\
-2.31831831831832	-7.75974493614364\\
-2.31031031031031	-7.73709440638924\\
-2.3023023023023	-7.71445623267649\\
-2.29429429429429	-7.69183052193293\\
-2.28628628628629	-7.66921738225744\\
-2.27827827827828	-7.64661692293534\\
-2.27027027027027	-7.62402925445363\\
-2.26226226226226	-7.60145448851655\\
-2.25425425425425	-7.57889273806126\\
-2.24624624624625	-7.55634411727379\\
-2.23823823823824	-7.53380874160524\\
-2.23023023023023	-7.51128672778817\\
-2.22222222222222	-7.48877819385324\\
-2.21421421421421	-7.46628325914609\\
-2.20620620620621	-7.44380204434447\\
-2.1981981981982	-7.42133467147556\\
-2.19019019019019	-7.39888126393364\\
-2.18218218218218	-7.37644194649791\\
-2.17417417417417	-7.35401684535062\\
-2.16616616616617	-7.33160608809543\\
-2.15815815815816	-7.3092098037761\\
-2.15015015015015	-7.28682812289533\\
-2.14214214214214	-7.26446117743395\\
-2.13413413413413	-7.2421091008704\\
-2.12612612612613	-7.21977202820044\\
-2.11811811811812	-7.19745009595714\\
-2.11011011011011	-7.17514344223118\\
-2.1021021021021	-7.15285220669147\\
-2.09409409409409	-7.13057653060595\\
-2.08608608608609	-7.10831655686285\\
-2.07807807807808	-7.0860724299921\\
-2.07007007007007	-7.06384429618717\\
-2.06206206206206	-7.04163230332711\\
-2.05405405405405	-7.01943660099897\\
-2.04604604604605	-6.99725734052058\\
-2.03803803803804	-6.97509467496351\\
-2.03003003003003	-6.95294875917652\\
-2.02202202202202	-6.93081974980925\\
-2.01401401401401	-6.90870780533625\\
-2.00600600600601	-6.88661308608139\\
-1.997997997998	-6.86453575424258\\
-1.98998998998999	-6.84247597391685\\
-1.98198198198198	-6.82043391112582\\
-1.97397397397397	-6.79840973384144\\
-1.96596596596597	-6.77640361201222\\
-1.95795795795796	-6.75441571758971\\
-1.94994994994995	-6.73244622455544\\
-1.94194194194194	-6.71049530894817\\
-1.93393393393393	-6.68856314889159\\
-1.92592592592593	-6.66664992462236\\
-1.91791791791792	-6.64475581851853\\
-1.90990990990991	-6.62288101512846\\
-1.9019019019019	-6.60102570119998\\
-1.89389389389389	-6.5791900657101\\
-1.88588588588589	-6.55737429989507\\
-1.87787787787788	-6.53557859728085\\
-1.86986986986987	-6.51380315371404\\
-1.86186186186186	-6.49204816739319\\
-1.85385385385385	-6.47031383890058\\
-1.84584584584585	-6.44860037123442\\
-1.83783783783784	-6.4269079698415\\
-1.82982982982983	-6.40523684265029\\
-1.82182182182182	-6.38358720010444\\
-1.81381381381381	-6.36195925519685\\
-1.80580580580581	-6.34035322350408\\
-1.7977977977978	-6.3187693232213\\
-1.78978978978979	-6.29720777519767\\
-1.78178178178178	-6.27566880297224\\
-1.77377377377377	-6.2541526328103\\
-1.76576576576577	-6.23265949374019\\
-1.75775775775776	-6.21118961759069\\
-1.74974974974975	-6.18974323902879\\
-1.74174174174174	-6.16832059559803\\
-1.73373373373373	-6.14692192775736\\
-1.72572572572573	-6.12554747892042\\
-1.71771771771772	-6.10419749549542\\
-1.70970970970971	-6.08287222692552\\
-1.7017017017017	-6.06157192572967\\
-1.69369369369369	-6.04029684754406\\
-1.68568568568569	-6.01904725116402\\
-1.67767767767768	-5.9978233985865\\
-1.66966966966967	-5.97662555505311\\
-1.66166166166166	-5.95545398909359\\
-1.65365365365365	-5.93430897256997\\
-1.64564564564565	-5.91319078072113\\
-1.63763763763764	-5.89209969220805\\
-1.62962962962963	-5.87103598915945\\
-1.62162162162162	-5.84999995721817\\
-1.61361361361361	-5.82899188558793\\
-1.60560560560561	-5.80801206708075\\
-1.5975975975976	-5.78706079816495\\
-1.58958958958959	-5.76613837901357\\
-1.58158158158158	-5.74524511355354\\
-1.57357357357357	-5.72438130951529\\
-1.56556556556557	-5.70354727848294\\
-1.55755755755756	-5.6827433359451\\
-1.54954954954955	-5.66196980134617\\
-1.54154154154154	-5.6412269981383\\
-1.53353353353353	-5.62051525383381\\
-1.52552552552553	-5.59983490005819\\
-1.51751751751752	-5.57918627260379\\
-1.50950950950951	-5.55856971148387\\
-1.5015015015015	-5.53798556098733\\
-1.49349349349349	-5.51743416973401\\
-1.48548548548549	-5.49691589073045\\
-1.47747747747748	-5.47643108142629\\
-1.46946946946947	-5.45598010377113\\
-1.46146146146146	-5.43556332427198\\
-1.45345345345345	-5.41518111405125\\
-1.44544544544545	-5.39483384890521\\
-1.43743743743744	-5.37452190936301\\
-1.42942942942943	-5.35424568074623\\
-1.42142142142142	-5.33400555322887\\
-1.41341341341341	-5.31380192189786\\
-1.40540540540541	-5.29363518681409\\
-1.3973973973974	-5.27350575307389\\
-1.38938938938939	-5.2534140308709\\
-1.38138138138138	-5.2333604355585\\
-1.37337337337337	-5.21334538771262\\
-1.36536536536537	-5.19336931319496\\
-1.35735735735736	-5.17343264321662\\
-1.34934934934935	-5.15353581440214\\
-1.34134134134134	-5.13367926885387\\
-1.33333333333333	-5.11386345421677\\
-1.32532532532533	-5.09408882374339\\
-1.31731731731732	-5.07435583635933\\
-1.30930930930931	-5.0546649567289\\
-1.3013013013013	-5.03501665532102\\
-1.29329329329329	-5.0154114084754\\
-1.28528528528529	-4.99584969846896\\
-1.27727727727728	-4.97633201358231\\
-1.26926926926927	-4.95685884816656\\
-1.26126126126126	-4.93743070271007\\
-1.25325325325325	-4.91804808390543\\
-1.24524524524525	-4.8987115047164\\
-1.23723723723724	-4.87942148444493\\
-1.22922922922923	-4.8601785487981\\
-1.22122122122122	-4.84098322995503\\
-1.21321321321321	-4.82183606663371\\
-1.20520520520521	-4.80273760415761\\
-1.1971971971972	-4.78368839452216\\
-1.18918918918919	-4.76468899646097\\
-1.18118118118118	-4.7457399755117\\
-1.17317317317317	-4.7268419040817\\
-1.16516516516517	-4.70799536151311\\
-1.15715715715716	-4.68920093414759\\
-1.14914914914915	-4.67045921539051\\
-1.14114114114114	-4.65177080577453\\
-1.13313313313313	-4.63313631302257\\
-1.12512512512513	-4.61455635210999\\
-1.11711711711712	-4.59603154532612\\
-1.10910910910911	-4.57756252233473\\
-1.1011011011011	-4.55914992023378\\
-1.09309309309309	-4.54079438361393\\
-1.08508508508509	-4.52249656461617\\
-1.07707707707708	-4.50425712298807\\
-1.06906906906907	-4.48607672613889\\
-1.06106106106106	-4.46795604919323\\
-1.05305305305305	-4.4498957750433\\
-1.04504504504505	-4.43189659439957\\
-1.03703703703704	-4.41395920583977\\
-1.02902902902903	-4.39608431585613\\
-1.02102102102102	-4.37827263890075\\
-1.01301301301301	-4.36052489742895\\
-1.00500500500501	-4.34284182194057\\
-0.996996996996997	-4.32522415101897\\
-0.988988988988989	-4.30767263136775\\
-0.980980980980981	-4.29018801784496\\
-0.972972972972973	-4.27277107349471\\
-0.964964964964965	-4.25542256957602\\
-0.956956956956957	-4.23814328558888\\
-0.948948948948949	-4.22093400929712\\
-0.940940940940941	-4.2037955367483\\
-0.932932932932933	-4.1867286722902\\
-0.924924924924925	-4.16973422858376\\
-0.916916916916917	-4.15281302661258\\
-0.908908908908909	-4.13596589568848\\
-0.900900900900901	-4.11919367345315\\
-0.892892892892893	-4.10249720587573\\
-0.884884884884885	-4.08587734724606\\
-0.876876876876877	-4.06933496016344\\
-0.868868868868869	-4.05287091552081\\
-0.860860860860861	-4.03648609248406\\
-0.852852852852853	-4.02018137846635\\
-0.844844844844845	-4.00395766909726\\
-0.836836836836837	-3.98781586818648\\
-0.828828828828829	-3.97175688768199\\
-0.820820820820821	-3.95578164762243\\
-0.812812812812813	-3.93989107608355\\
-0.804804804804805	-3.92408610911837\\
-0.796796796796797	-3.9083676906911\\
-0.788788788788789	-3.89273677260441\\
-0.780780780780781	-3.8771943144199\\
-0.772772772772773	-3.86174128337161\\
-0.764764764764765	-3.84637865427234\\
-0.756756756756757	-3.83110740941254\\
-0.748748748748749	-3.81592853845159\\
-0.740740740740741	-3.80084303830126\\
-0.732732732732733	-3.78585191300118\\
-0.724724724724725	-3.77095617358597\\
-0.716716716716717	-3.75615683794407\\
-0.708708708708709	-3.74145493066785\\
-0.700700700700701	-3.72685148289487\\
-0.692692692692693	-3.71234753214012\\
-0.684684684684685	-3.69794412211907\\
-0.676676676676677	-3.68364230256126\\
-0.668668668668669	-3.66944312901425\\
-0.660660660660661	-3.6553476626379\\
-0.652652652652653	-3.64135696998855\\
-0.644644644644645	-3.62747212279323\\
-0.636636636636636	-3.61369419771345\\
-0.628628628628629	-3.60002427609867\\
-0.62062062062062	-3.5864634437291\\
-0.612612612612613	-3.5730127905479\\
-0.604604604604605	-3.55967341038244\\
-0.596596596596596	-3.54644640065469\\
-0.588588588588589	-3.53333286208057\\
-0.58058058058058	-3.52033389835816\\
-0.572572572572573	-3.50745061584465\\
-0.564564564564565	-3.49468412322217\\
-0.556556556556556	-3.48203553115216\\
-0.548548548548549	-3.46950595191857\\
-0.54054054054054	-3.45709649905952\\
-0.532532532532533	-3.44480828698787\\
-0.524524524524525	-3.43264243060024\\
-0.516516516516516	-3.42060004487493\\
-0.508508508508509	-3.40868224445858\\
-0.5005005005005	-3.39689014324172\\
-0.492492492492492	-3.3852248539233\\
-0.484484484484485	-3.37368748756437\\
-0.476476476476476	-3.362279153131\\
-0.468468468468469	-3.35100095702672\\
-0.46046046046046	-3.33985400261452\\
-0.452452452452452	-3.3288393897288\\
-0.444444444444445	-3.31795821417746\\
-0.436436436436436	-3.30721156723433\\
-0.428428428428429	-3.2966005351224\\
-0.42042042042042	-3.28612619848806\\
-0.412412412412412	-3.27578963186673\\
-0.404404404404405	-3.26559190314036\\
-0.396396396396396	-3.25553407298707\\
-0.388388388388389	-3.24561719432347\\
-0.38038038038038	-3.23584231174012\\
-0.372372372372372	-3.22621046093066\\
-0.364364364364364	-3.21672266811498\\
-0.356356356356356	-3.20737994945722\\
-0.348348348348348	-3.19818331047898\\
-0.34034034034034	-3.18913374546853\\
-0.332332332332332	-3.18023223688644\\
-0.324324324324324	-3.17147975476846\\
-0.316316316316316	-3.16287725612634\\
-0.308308308308308	-3.15442568434714\\
-0.3003003003003	-3.14612596859194\\
-0.292292292292292	-3.13797902319456\\
-0.284284284284284	-3.12998574706123\\
-0.276276276276276	-3.12214702307178\\
-0.268268268268268	-3.11446371748339\\
-0.26026026026026	-3.10693667933755\\
-0.252252252252252	-3.09956673987116\\
-0.244244244244244	-3.09235471193262\\
-0.236236236236236	-3.08530138940375\\
-0.228228228228228	-3.07840754662837\\
-0.22022022022022	-3.07167393784855\\
-0.212212212212212	-3.06510129664926\\
-0.204204204204204	-3.0586903354124\\
-0.196196196196196	-3.05244174478107\\
-0.188188188188188	-3.04635619313498\\
-0.18018018018018	-3.04043432607781\\
-0.172172172172172	-3.03467676593748\\
-0.164164164164164	-3.02908411128014\\
-0.156156156156156	-3.02365693643872\\
-0.148148148148148	-3.01839579105692\\
-0.14014014014014	-3.01330119964938\\
-0.132132132132132	-3.00837366117891\\
-0.124124124124124	-3.0036136486515\\
-0.116116116116116	-2.99902160872985\\
-0.108108108108108	-2.99459796136615\\
-0.1001001001001	-2.99034309945489\\
-0.0920920920920922	-2.98625738850618\\
-0.084084084084084	-2.98234116634033\\
-0.0760760760760761	-2.97859474280429\\
-0.0680680680680679	-2.97501839951041\\
-0.06006006006006	-2.97161238959811\\
-0.0520520520520522	-2.96837693751886\\
-0.0440440440440439	-2.96531223884499\\
-0.0360360360360361	-2.96241846010255\\
-0.0280280280280278	-2.95969573862876\\
-0.02002002002002	-2.95714418245416\\
-0.0120120120120122	-2.95476387020973\\
-0.00400400400400391	-2.95255485105927\\
0.00400400400400436	-2.95051714465706\\
0.0120120120120122	-2.9486507411309\\
0.02002002002002	-2.94695560109065\\
0.0280280280280278	-2.94543165566213\\
0.0360360360360357	-2.94407880654641\\
0.0440440440440444	-2.94289692610433\\
0.0520520520520522	-2.9418858574661\\
0.06006006006006	-2.94104541466568\\
0.0680680680680679	-2.94037538279983\\
0.0760760760760757	-2.93987551821138\\
0.0840840840840844	-2.93954554869634\\
0.0920920920920922	-2.93938517373465\\
0.1001001001001	-2.93939406474378\\
0.108108108108108	-2.93957186535505\\
0.116116116116116	-2.93991819171181\\
0.124124124124124	-2.94043263278914\\
0.132132132132132	-2.94111475073437\\
0.14014014014014	-2.94196408122768\\
0.148148148148148	-2.94298013386231\\
0.156156156156156	-2.94416239254347\\
0.164164164164164	-2.94551031590534\\
0.172172172172172	-2.94702333774526\\
0.18018018018018	-2.94870086747456\\
0.188188188188188	-2.95054229058493\\
0.196196196196196	-2.95254696912971\\
0.204204204204204	-2.95471424221918\\
0.212212212212212	-2.95704342652899\\
0.22022022022022	-2.95953381682093\\
0.228228228228228	-2.96218468647502\\
0.236236236236236	-2.96499528803225\\
0.244244244244245	-2.96796485374684\\
0.252252252252252	-2.97109259614735\\
0.26026026026026	-2.97437770860569\\
0.268268268268268	-2.97781936591299\\
0.276276276276277	-2.98141672486179\\
0.284284284284285	-2.98516892483329\\
0.292292292292292	-2.98907508838917\\
0.3003003003003	-2.99313432186686\\
0.308308308308308	-2.99734571597754\\
0.316316316316317	-3.00170834640613\\
0.324324324324325	-3.0062212744123\\
0.332332332332332	-3.01088354743185\\
0.34034034034034	-3.01569419967767\\
0.348348348348348	-3.02065225273952\\
0.356356356356357	-3.0257567161819\\
0.364364364364365	-3.03100658813941\\
0.372372372372372	-3.03640085590876\\
0.38038038038038	-3.04193849653703\\
0.388388388388388	-3.04761847740525\\
0.396396396396397	-3.05343975680706\\
0.404404404404405	-3.05940128452157\\
0.412412412412412	-3.06550200238007\\
0.42042042042042	-3.07174084482606\\
0.428428428428428	-3.07811673946806\\
0.436436436436437	-3.08462860762475\\
0.444444444444445	-3.09127536486212\\
0.452452452452452	-3.09805592152207\\
0.46046046046046	-3.10496918324227\\
0.468468468468468	-3.11201405146674\\
0.476476476476477	-3.11918942394701\\
0.484484484484485	-3.12649419523351\\
0.492492492492492	-3.13392725715688\\
0.5005005005005	-3.14148749929901\\
0.508508508508508	-3.14917380945366\\
0.516516516516517	-3.15698507407631\\
0.524524524524525	-3.16492017872329\\
0.532532532532533	-3.17297800847984\\
0.54054054054054	-3.18115744837717\\
0.548548548548548	-3.18945738379831\\
0.556556556556557	-3.1978767008727\\
0.564564564564565	-3.20641428685951\\
0.572572572572573	-3.21506903051966\\
0.58058058058058	-3.22383982247639\\
0.588588588588588	-3.23272555556462\\
0.596596596596597	-3.24172512516893\\
0.604604604604605	-3.25083742955027\\
0.612612612612613	-3.26006137016146\\
0.62062062062062	-3.26939585195154\\
0.628628628628628	-3.27883978365904\\
0.636636636636637	-3.28839207809427\\
0.644644644644645	-3.29805165241076\\
0.652652652652653	-3.30781742836597\\
0.66066066066066	-3.31768833257131\\
0.668668668668668	-3.32766329673175\\
0.676676676676677	-3.3377412578751\\
0.684684684684685	-3.34792115857103\\
0.692692692692693	-3.35820194714019\\
0.7007007007007	-3.36858257785342\\
0.708708708708708	-3.37906201112131\\
0.716716716716717	-3.38963921367431\\
0.724724724724725	-3.40031315873347\\
0.732732732732733	-3.41108282617218\\
0.74074074074074	-3.42194720266893\\
0.748748748748748	-3.4329052818514\\
0.756756756756757	-3.44395606443203\\
0.764764764764765	-3.45509855833525\\
0.772772772772773	-3.46633177881674\\
0.780780780780781	-3.47765474857464\\
0.788788788788789	-3.48906649785315\\
0.796796796796797	-3.50056606453872\\
0.804804804804805	-3.51215249424886\\
0.812812812812813	-3.52382484041392\\
0.820820820820821	-3.5355821643521\\
0.828828828828829	-3.5474235353377\\
0.836836836836837	-3.55934803066298\\
0.844844844844845	-3.57135473569381\\
0.852852852852853	-3.58344274391919\\
0.860860860860861	-3.59561115699504\\
0.868868868868869	-3.60785908478224\\
0.876876876876877	-3.62018564537934\\
0.884884884884885	-3.63258996514985\\
0.892892892892893	-3.64507117874463\\
0.900900900900901	-3.65762842911931\\
0.908908908908909	-3.67026086754696\\
0.916916916916917	-3.68296765362636\\
0.924924924924925	-3.69574795528582\\
0.932932932932933	-3.70860094878287\\
0.940940940940941	-3.72152581869997\\
0.948948948948949	-3.73452175793628\\
0.956956956956957	-3.74758796769589\\
0.964964964964965	-3.76072365747241\\
0.972972972972973	-3.7739280450303\\
0.980980980980981	-3.78720035638286\\
0.988988988988989	-3.80053982576728\\
0.996996996996997	-3.81394569561667\\
1.00500500500501	-3.82741721652937\\
1.01301301301301	-3.84095364723556\\
1.02102102102102	-3.85455425456139\\
1.02902902902903	-3.8682183133907\\
1.03703703703704	-3.88194510662445\\
1.04504504504505	-3.89573392513808\\
1.05305305305305	-3.90958406773677\\
1.06106106106106	-3.92349484110882\\
1.06906906906907	-3.93746555977722\\
1.07707707707708	-3.95149554604949\\
1.08508508508509	-3.96558412996602\\
1.09309309309309	-3.9797306492468\\
1.1011011011011	-3.99393444923681\\
1.10910910910911	-4.00819488285013\\
1.11711711711712	-4.02251131051275\\
1.12512512512513	-4.03688310010437\\
1.13313313313313	-4.05130962689904\\
1.14114114114114	-4.06579027350488\\
1.14914914914915	-4.08032442980296\\
1.15715715715716	-4.09491149288528\\
1.16516516516517	-4.10955086699207\\
1.17317317317317	-4.12424196344842\\
1.18118118118118	-4.13898420060022\\
1.18918918918919	-4.15377700374972\\
1.1971971971972	-4.16861980509042\\
1.20520520520521	-4.18351204364167\\
1.21321321321321	-4.19845316518289\\
1.22122122122122	-4.21344262218739\\
1.22922922922923	-4.22847987375603\\
1.23723723723724	-4.24356438555063\\
1.24524524524525	-4.2586956297272\\
1.25325325325325	-4.27387308486906\\
1.26126126126126	-4.28909623591991\\
1.26926926926927	-4.3043645741168\\
1.27727727727728	-4.31967759692321\\
1.28528528528529	-4.33503480796205\\
1.29329329329329	-4.35043571694882\\
1.3013013013013	-4.36587983962485\\
1.30930930930931	-4.38136669769067\\
1.31731731731732	-4.39689581873958\\
1.32532532532533	-4.41246673619142\\
1.33333333333333	-4.42807898922653\\
1.34134134134134	-4.44373212272002\\
1.34934934934935	-4.45942568717623\\
1.35735735735736	-4.47515923866364\\
1.36536536536537	-4.49093233874996\\
1.37337337337337	-4.5067445544376\\
1.38138138138138	-4.52259545809958\\
1.38938938938939	-4.5384846274157\\
1.3973973973974	-4.55441164530925\\
1.40540540540541	-4.57037609988396\\
1.41341341341341	-4.58637758436157\\
1.42142142142142	-4.60241569701969\\
1.42942942942943	-4.61849004113016\\
1.43743743743744	-4.63460022489799\\
1.44544544544545	-4.65074586140055\\
1.45345345345345	-4.66692656852749\\
1.46146146146146	-4.68314196892101\\
1.46946946946947	-4.69939168991665\\
1.47747747747748	-4.71567536348468\\
1.48548548548549	-4.73199262617193\\
1.49349349349349	-4.74834311904415\\
1.5015015015015	-4.764726487629\\
1.50950950950951	-4.78114238185945\\
1.51751751751752	-4.7975904560178\\
1.52552552552553	-4.81407036868029\\
1.53353353353353	-4.83058178266212\\
1.54154154154154	-4.8471243649632\\
1.54954954954955	-4.86369778671432\\
1.55755755755756	-4.88030172312398\\
1.56556556556557	-4.89693585342571\\
1.57357357357357	-4.91359986082596\\
1.58158158158158	-4.93029343245264\\
1.58958958958959	-4.94701625930411\\
1.5975975975976	-4.96376803619879\\
1.60560560560561	-4.98054846172535\\
1.61361361361361	-4.99735723819344\\
1.62162162162162	-5.01419407158498\\
1.62962962962963	-5.031058671506\\
1.63763763763764	-5.04795075113912\\
1.64564564564565	-5.06487002719644\\
1.65365365365365	-5.08181621987317\\
1.66166166166166	-5.09878905280164\\
1.66966966966967	-5.11578825300598\\
1.67767767767768	-5.13281355085729\\
1.68568568568569	-5.14986468002941\\
1.69369369369369	-5.16694137745514\\
1.7017017017017	-5.18404338328313\\
1.70970970970971	-5.20117044083517\\
1.71771771771772	-5.21832229656411\\
1.72572572572573	-5.2354987000123\\
1.73373373373373	-5.25269940377048\\
1.74174174174174	-5.26992416343727\\
1.74974974974975	-5.28717273757917\\
1.75775775775776	-5.30444488769103\\
1.76576576576577	-5.32174037815706\\
1.77377377377377	-5.33905897621236\\
1.78178178178178	-5.3564004519049\\
1.78978978978979	-5.37376457805803\\
1.7977977977978	-5.39115113023351\\
1.80580580580581	-5.40855988669496\\
1.81381381381381	-5.42599062837182\\
1.82182182182182	-5.44344313882384\\
1.82982982982983	-5.46091720420594\\
1.83783783783784	-5.47841261323364\\
1.84584584584585	-5.49592915714888\\
1.85385385385385	-5.51346662968633\\
1.86186186186186	-5.53102482704014\\
1.86986986986987	-5.54860354783119\\
1.87787787787788	-5.56620259307469\\
1.88588588588589	-5.58382176614827\\
1.89389389389389	-5.60146087276056\\
1.9019019019019	-5.61911972092008\\
1.90990990990991	-5.63679812090463\\
1.91791791791792	-5.65449588523111\\
1.92592592592593	-5.6722128286257\\
1.93393393393393	-5.68994876799451\\
1.94194194194194	-5.70770352239454\\
1.94994994994995	-5.72547691300517\\
1.95795795795796	-5.74326876309994\\
1.96596596596597	-5.76107889801877\\
1.97397397397397	-5.77890714514054\\
1.98198198198198	-5.79675333385603\\
1.98998998998999	-5.81461729554135\\
1.997997997998	-5.83249886353157\\
2.00600600600601	-5.85039787309482\\
2.01401401401401	-5.8683141614068\\
2.02202202202202	-5.88624756752547\\
2.03003003003003	-5.90419793236629\\
2.03803803803804	-5.92216509867766\\
2.04604604604605	-5.94014891101682\\
2.05405405405405	-5.95814921572597\\
2.06206206206206	-5.9761658609088\\
2.07007007007007	-5.99419869640736\\
2.07807807807808	-6.01224757377918\\
2.08608608608609	-6.03031234627482\\
2.09409409409409	-6.0483928688156\\
2.1021021021021	-6.06648899797175\\
2.11011011011011	-6.08460059194086\\
2.11811811811812	-6.10272751052656\\
2.12612612612613	-6.12086961511757\\
2.13413413413413	-6.13902676866699\\
2.14214214214214	-6.15719883567198\\
2.15015015015015	-6.17538568215357\\
2.15815815815816	-6.1935871756369\\
2.16616616616617	-6.21180318513167\\
2.17417417417417	-6.23003358111288\\
2.18218218218218	-6.24827823550179\\
2.19019019019019	-6.26653702164726\\
2.1981981981982	-6.28480981430725\\
2.20620620620621	-6.30309648963064\\
2.21421421421421	-6.32139692513923\\
2.22222222222222	-6.33971099971014\\
2.23023023023023	-6.35803859355827\\
2.23823823823824	-6.37637958821917\\
2.24624624624625	-6.39473386653205\\
2.25425425425425	-6.41310131262306\\
2.26226226226226	-6.43148181188883\\
2.27027027027027	-6.4498752509802\\
2.27827827827828	-6.4682815177862\\
2.28628628628629	-6.48670050141826\\
2.29429429429429	-6.50513209219462\\
2.3023023023023	-6.52357618162499\\
2.31031031031031	-6.5420326623954\\
2.31831831831832	-6.56050142835325\\
2.32632632632633	-6.57898237449265\\
2.33433433433433	-6.59747539693982\\
2.34234234234234	-6.61598039293887\\
2.35035035035035	-6.63449726083761\\
2.35835835835836	-6.65302590007369\\
2.36636636636637	-6.67156621116086\\
2.37437437437437	-6.69011809567546\\
2.38238238238238	-6.70868145624304\\
2.39039039039039	-6.72725619652527\\
2.3983983983984	-6.74584222120691\\
2.40640640640641	-6.76443943598308\\
2.41441441441441	-6.78304774754661\\
2.42242242242242	-6.8016670635756\\
2.43043043043043	-6.8202972927212\\
2.43843843843844	-6.83893834459548\\
2.44644644644645	-6.85759012975953\\
2.45445445445445	-6.8762525597117\\
2.46246246246246	-6.89492554687597\\
2.47047047047047	-6.91360900459058\\
2.47847847847848	-6.93230284709669\\
2.48648648648649	-6.9510069895273\\
2.49449449449449	-6.96972134789626\\
2.5025025025025	-6.98844583908749\\
2.51051051051051	-7.00718038084425\\
2.51851851851852	-7.02592489175869\\
2.52652652652653	-7.04467929126143\\
2.53453453453453	-7.06344349961133\\
2.54254254254254	-7.08221743788543\\
2.55055055055055	-7.10100102796897\\
2.55855855855856	-7.11979419254556\\
2.56656656656657	-7.13859685508754\\
2.57457457457457	-7.15740893984638\\
2.58258258258258	-7.17623037184331\\
2.59059059059059	-7.19506107685999\\
2.5985985985986	-7.21390098142937\\
2.60660660660661	-7.23275001282662\\
2.61461461461461	-7.25160809906024\\
2.62262262262262	-7.27047516886327\\
2.63063063063063	-7.28935115168458\\
2.63863863863864	-7.30823597768034\\
2.64664664664665	-7.32712957770558\\
2.65465465465465	-7.34603188330583\\
2.66266266266266	-7.36494282670893\\
2.67067067067067	-7.38386234081691\\
2.67867867867868	-7.40279035919802\\
2.68668668668669	-7.42172681607882\\
2.69469469469469	-7.44067164633637\\
2.7027027027027	-7.45962478549063\\
2.71071071071071	-7.47858616969679\\
2.71871871871872	-7.49755573573789\\
2.72672672672673	-7.51653342101736\\
2.73473473473473	-7.53551916355182\\
2.74274274274274	-7.55451290196384\\
2.75075075075075	-7.57351457547489\\
2.75875875875876	-7.59252412389838\\
2.76676676676677	-7.61154148763269\\
2.77477477477477	-7.63056660765444\\
2.78278278278278	-7.64959942551172\\
2.79079079079079	-7.66863988331753\\
2.7987987987988	-7.68768792374316\\
2.80680680680681	-7.70674349001183\\
2.81481481481481	-7.72580652589225\\
2.82282282282282	-7.74487697569238\\
2.83083083083083	-7.76395478425322\\
2.83883883883884	-7.7830398969427\\
2.84684684684685	-7.80213225964962\\
2.85485485485485	-7.82123181877774\\
2.86286286286286	-7.84033852123983\\
2.87087087087087	-7.85945231445194\\
2.87887887887888	-7.87857314632763\\
2.88688688688689	-7.89770096527232\\
2.89489489489489	-7.91683572017772\\
2.9029029029029	-7.93597736041634\\
2.91091091091091	-7.95512583583602\\
2.91891891891892	-7.97428109675458\\
2.92692692692693	-7.99344309395455\\
2.93493493493493	-8.01261177867791\\
2.94294294294294	-8.03178710262094\\
2.95095095095095	-8.05096901792913\\
2.95895895895896	-8.07015747719218\\
2.96696696696697	-8.08935243343897\\
2.97497497497497	-8.10855384013274\\
2.98298298298298	-8.1277616511662\\
2.99099099099099	-8.1469758208568\\
2.998998998999	-8.16619630394195\\
3.00700700700701	-8.18542305557444\\
3.01501501501502	-8.20465603131782\\
3.02302302302302	-8.22389518714185\\
3.03103103103103	-8.24314047941804\\
3.03903903903904	-8.26239186491522\\
3.04704704704705	-8.28164930079518\\
3.05505505505506	-8.30091274460838\\
3.06306306306306	-8.32018215428965\\
3.07107107107107	-8.33945748815405\\
3.07907907907908	-8.35873870489267\\
3.08708708708709	-8.37802576356854\\
3.0950950950951	-8.39731862361264\\
3.1031031031031	-8.41661724481982\\
3.11111111111111	-8.43592158734494\\
3.11911911911912	-8.45523161169894\\
3.12712712712713	-8.47454727874495\\
3.13513513513514	-8.49386854969459\\
3.14314314314314	-8.51319538610414\\
3.15115115115115	-8.53252774987087\\
3.15915915915916	-8.55186560322938\\
3.16716716716717	-8.57120890874798\\
3.17517517517518	-8.59055762932511\\
3.18318318318318	-8.60991172818588\\
3.19119119119119	-8.6292711688785\\
3.1991991991992	-8.64863591527092\\
3.20720720720721	-8.66800593154737\\
3.21521521521522	-8.68738118220507\\
3.22322322322322	-8.70676163205089\\
3.23123123123123	-8.72614724619806\\
3.23923923923924	-8.74553799006297\\
3.24724724724725	-8.76493382936196\\
3.25525525525526	-8.78433473010818\\
3.26326326326326	-8.80374065860848\\
3.27127127127127	-8.82315158146029\\
3.27927927927928	-8.84256746554866\\
3.28728728728729	-8.86198827804318\\
3.2952952952953	-8.88141398639506\\
3.3033033033033	-8.90084455833419\\
3.31131131131131	-8.92027996186625\\
3.31931931931932	-8.93972016526982\\
3.32732732732733	-8.9591651370936\\
3.33533533533534	-8.9786148461536\\
3.34334334334334	-8.99806926153037\\
3.35135135135135	-9.01752835256631\\
3.35935935935936	-9.03699208886291\\
3.36736736736737	-9.05646044027818\\
3.37537537537538	-9.07593337692395\\
3.38338338338338	-9.09541086916331\\
3.39139139139139	-9.11489288760805\\
3.3993993993994	-9.13437940311607\\
3.40740740740741	-9.15387038678895\\
3.41541541541542	-9.17336580996943\\
3.42342342342342	-9.19286564423895\\
3.43143143143143	-9.2123698614153\\
3.43943943943944	-9.23187843355014\\
3.44744744744745	-9.2513913329267\\
3.45545545545546	-9.27090853205746\\
3.46346346346346	-9.29043000368177\\
3.47147147147147	-9.30995572076364\\
3.47947947947948	-9.32948565648948\\
3.48748748748749	-9.34901978426583\\
3.4954954954955	-9.36855807771722\\
3.5035035035035	-9.38810051068394\\
3.51151151151151	-9.40764705721993\\
3.51951951951952	-9.42719769159065\\
3.52752752752753	-9.44675238827095\\
3.53553553553554	-9.46631112194304\\
3.54354354354354	-9.48587386749441\\
3.55155155155155	-9.50544060001582\\
3.55955955955956	-9.52501129479927\\
3.56756756756757	-9.54458592733604\\
3.57557557557558	-9.56416447331473\\
3.58358358358358	-9.58374690861934\\
3.59159159159159	-9.6033332093273\\
3.5995995995996	-9.62292335170764\\
3.60760760760761	-9.64251731221909\\
3.61561561561562	-9.66211506750825\\
3.62362362362362	-9.68171659440773\\
3.63163163163163	-9.70132186993438\\
3.63963963963964	-9.72093087128746\\
3.64764764764765	-9.74054357584692\\
3.65565565565566	-9.76015996117165\\
3.66366366366366	-9.7797800049977\\
3.67167167167167	-9.79940368523664\\
3.67967967967968	-9.81903097997381\\
3.68768768768769	-9.83866186746672\\
3.6956956956957	-9.85829632614335\\
3.7037037037037	-9.87793433460051\\
3.71171171171171	-9.89757587160229\\
3.71971971971972	-9.91722091607838\\
3.72772772772773	-9.93686944712256\\
3.73573573573574	-9.95652144399112\\
3.74374374374374	-9.9761768861013\\
3.75175175175175	-9.99583575302978\\
3.75975975975976	-10.0154980245112\\
3.76776776776777	-10.0351636804365\\
3.77577577577578	-10.0548327008518\\
3.78378378378378	-10.0745050659566\\
3.79179179179179	-10.0941807561024\\
3.7997997997998	-10.1138597517915\\
3.80780780780781	-10.1335420336753\\
3.81581581581582	-10.153227582553\\
3.82382382382382	-10.1729163793704\\
3.83183183183183	-10.1926084052181\\
3.83983983983984	-10.2123036413308\\
3.84784784784785	-10.2320020690853\\
3.85585585585586	-10.2517036699995\\
3.86386386386386	-10.2714084257313\\
3.87187187187187	-10.2911163180769\\
3.87987987987988	-10.3108273289695\\
3.88788788788789	-10.3305414404787\\
3.8958958958959	-10.3502586348083\\
3.9039039039039	-10.3699788942958\\
3.91191191191191	-10.3897022014109\\
3.91991991991992	-10.4094285387543\\
3.92792792792793	-10.4291578890564\\
3.93593593593594	-10.4488902351763\\
3.94394394394394	-10.4686255601006\\
3.95195195195195	-10.4883638469421\\
3.95995995995996	-10.5081050789388\\
3.96796796796797	-10.5278492394527\\
3.97597597597598	-10.5475963119685\\
3.98398398398398	-10.567346280093\\
3.99199199199199	-10.5870991275535\\
4	-10.6068548381968\\
4	12.7597524978374\\
3.99199199199199	12.7358068202395\\
3.98398398398398	12.7118640058244\\
3.97597597597598	12.6879240707453\\
3.96796796796797	12.6639870312748\\
3.95995995995996	12.6400529038064\\
3.95195195195195	12.6161217048551\\
3.94394394394394	12.592193451059\\
3.93593593593594	12.5682681591801\\
3.92792792792793	12.5443458461056\\
3.91991991991992	12.5204265288489\\
3.91191191191191	12.4965102245509\\
3.9039039039039	12.4725969504812\\
3.8958958958959	12.448686724039\\
3.88788788788789	12.4247795627548\\
3.87987987987988	12.4008754842911\\
3.87187187187187	12.3769745064438\\
3.86386386386386	12.3530766471437\\
3.85585585585586	12.3291819244573\\
3.84784784784785	12.3052903565884\\
3.83983983983984	12.2814019618794\\
3.83183183183183	12.2575167588121\\
3.82382382382382	12.2336347660097\\
3.81581581581582	12.2097560022378\\
3.80780780780781	12.1858804864055\\
3.7997997997998	12.1620082375671\\
3.79179179179179	12.1381392749234\\
3.78378378378378	12.114273617823\\
3.77577577577578	12.0904112857636\\
3.76776776776777	12.0665522983937\\
3.75975975975976	12.0426966755137\\
3.75175175175175	12.0188444370778\\
3.74374374374374	11.9949956031947\\
3.73573573573574	11.9711501941299\\
3.72772772772773	11.9473082303067\\
3.71971971971972	11.923469732308\\
3.71171171171171	11.8996347208773\\
3.7037037037037	11.8758032169209\\
3.6956956956957	11.8519752415091\\
3.68768768768769	11.8281508158779\\
3.67967967967968	11.8043299614304\\
3.67167167167167	11.7805126997386\\
3.66366366366366	11.7566990525451\\
3.65565565565566	11.7328890417644\\
3.64764764764765	11.7090826894851\\
3.63963963963964	11.685280017971\\
3.63163163163163	11.6614810496634\\
3.62362362362362	11.6376858071821\\
3.61561561561562	11.613894313328\\
3.60760760760761	11.5901065910843\\
3.5995995995996	11.5663226636182\\
3.59159159159159	11.5425425542833\\
3.58358358358358	11.5187662866207\\
3.57557557557558	11.4949938843615\\
3.56756756756757	11.4712253714282\\
3.55955955955956	11.4474607719369\\
3.55155155155155	11.4237001101988\\
3.54354354354354	11.3999434107228\\
3.53553553553554	11.3761906982168\\
3.52752752752753	11.3524419975902\\
3.51951951951952	11.3286973339553\\
3.51151151151151	11.3049567326299\\
3.5035035035035	11.2812202191393\\
3.4954954954955	11.257487819218\\
3.48748748748749	11.233759558812\\
3.47947947947948	11.2100354640811\\
3.47147147147147	11.1863155614007\\
3.46346346346346	11.1625998773642\\
3.45545545545546	11.1388884387853\\
3.44744744744745	11.1151812726999\\
3.43943943943944	11.0914784063688\\
3.43143143143143	11.0677798672793\\
3.42342342342342	11.0440856831484\\
3.41541541541542	11.0203958819242\\
3.40740740740741	10.9967104917892\\
3.3993993993994	10.9730295411617\\
3.39139139139139	10.9493530586991\\
3.38338338338338	10.9256810732997\\
3.37537537537538	10.9020136141058\\
3.36736736736737	10.8783507105054\\
3.35935935935936	10.8546923921355\\
3.35135135135135	10.8310386888843\\
3.34334334334334	10.8073896308938\\
3.33533533533534	10.7837452485624\\
3.32732732732733	10.7601055725478\\
3.31931931931932	10.7364706337694\\
3.31131131131131	10.7128404634113\\
3.3033033033033	10.6892150929246\\
3.2952952952953	10.6655945540309\\
3.28728728728729	10.6419788787244\\
3.27927927927928	10.6183680992753\\
3.27127127127127	10.5947622482323\\
3.26326326326326	10.5711613584259\\
3.25525525525526	10.547565462971\\
3.24724724724725	10.5239745952702\\
3.23923923923924	10.5003887890166\\
3.23123123123123	10.4768080781971\\
3.22322322322322	10.4532324970953\\
3.21521521521522	10.4296620802949\\
3.20720720720721	10.4060968626826\\
3.1991991991992	10.3825368794516\\
3.19119119119119	10.3589821661045\\
3.18318318318318	10.3354327584573\\
3.17517517517518	10.311888692642\\
3.16716716716717	10.2883500051102\\
3.15915915915916	10.264816732637\\
3.15115115115115	10.2412889123239\\
3.14314314314314	10.2177665816026\\
3.13513513513514	10.1942497782384\\
3.12712712712713	10.1707385403342\\
3.11911911911912	10.1472329063336\\
3.11111111111111	10.123732915025\\
3.1031031031031	10.1002386055453\\
3.0950950950951	10.0767500173835\\
3.08708708708709	10.0532671903848\\
3.07907907907908	10.0297901647543\\
3.07107107107107	10.0063189810611\\
3.06306306306306	9.9828536802421\\
3.05505505505506	9.95939430360623\\
3.04704704704705	9.93594089283843\\
3.03903903903904	9.91249349000387\\
3.03103103103103	9.88905213755209\\
3.02302302302302	9.86561687832131\\
3.01501501501502	9.84218775554268\\
3.00700700700701	9.8187648128447\\
2.998998998999	9.79534809425761\\
2.99099099099099	9.77193764421786\\
2.98298298298298	9.74853350757266\\
2.97497497497497	9.7251357295846\\
2.96696696696697	9.70174435593623\\
2.95895895895896	9.67835943273484\\
2.95095095095095	9.6549810065172\\
2.94294294294294	9.63160912425441\\
2.93493493493493	9.60824383335678\\
2.92692692692693	9.58488518167882\\
2.91891891891892	9.56153321752425\\
2.91091091091091	9.53818798965109\\
2.9029029029029	9.51484954727681\\
2.89489489489489	9.49151794008359\\
2.88688688688689	9.46819321822359\\
2.87887887887888	9.4448754323243\\
2.87087087087087	9.42156463349401\\
2.86286286286286	9.3982608733273\\
2.85485485485485	9.37496420391061\\
2.84684684684685	9.3516746778279\\
2.83883883883884	9.32839234816637\\
2.83083083083083	9.3051172685223\\
2.82282282282282	9.28184949300686\\
2.81481481481481	9.25858907625213\\
2.80680680680681	9.23533607341711\\
2.7987987987988	9.21209054019384\\
2.79079079079079	9.18885253281361\\
2.78278278278278	9.16562210805321\\
2.77477477477477	9.14239932324132\\
2.76676676676677	9.11918423626498\\
2.75875875875876	9.09597690557607\\
2.75075075075075	9.07277739019798\\
2.74274274274274	9.04958574973232\\
2.73473473473473	9.0264020443657\\
2.72672672672673	9.00322633487665\\
2.71871871871872	8.98005868264258\\
2.71071071071071	8.95689914964688\\
2.7027027027027	8.93374779848612\\
2.69469469469469	8.91060469237726\\
2.68668668668669	8.88746989516511\\
2.67867867867868	8.86434347132972\\
2.67067067067067	8.841225485994\\
2.66266266266266	8.81811600493142\\
2.65465465465465	8.79501509457372\\
2.64664664664665	8.77192282201888\\
2.63863863863864	8.74883925503904\\
2.63063063063063	8.72576446208867\\
2.62262262262262	8.70269851231276\\
2.61461461461461	8.67964147555514\\
2.60660660660661	8.65659342236692\\
2.5985985985986	8.63355442401507\\
2.59059059059059	8.61052455249109\\
2.58258258258258	8.58750388051981\\
2.57457457457457	8.56449248156828\\
2.56656656656657	8.54149042985484\\
2.55855855855856	8.51849780035826\\
2.55055055055055	8.49551466882707\\
2.54254254254254	8.47254111178894\\
2.53453453453453	8.44957720656024\\
2.52652652652653	8.42662303125574\\
2.51851851851852	8.4036786647984\\
2.51051051051051	8.38074418692936\\
2.5025025025025	8.357819678218\\
2.49449449449449	8.33490522007217\\
2.48648648648649	8.31200089474861\\
2.47847847847848	8.2891067853634\\
2.47047047047047	8.26622297590269\\
2.46246246246246	8.24334955123348\\
2.45445445445445	8.22048659711461\\
2.44644644644645	8.19763420020785\\
2.43843843843844	8.17479244808919\\
2.43043043043043	8.15196142926031\\
2.42242242242242	8.12914123316011\\
2.41441441441441	8.10633195017652\\
2.40640640640641	8.0835336716584\\
2.3983983983984	8.06074648992763\\
2.39039039039039	8.03797049829139\\
2.38238238238238	8.01520579105456\\
2.37437437437437	7.99245246353238\\
2.36636636636637	7.96971061206319\\
2.35835835835836	7.94698033402141\\
2.35035035035035	7.92426172783073\\
2.34234234234234	7.90155489297739\\
2.33433433433433	7.87885993002375\\
2.32632632632633	7.85617694062197\\
2.31831831831832	7.83350602752798\\
2.31031031031031	7.81084729461552\\
2.3023023023023	7.78820084689052\\
2.29429429429429	7.76556679050555\\
2.28628628628629	7.74294523277459\\
2.27827827827828	7.72033628218794\\
2.27027027027027	7.69774004842734\\
2.26226226226226	7.67515664238137\\
2.25425425425425	7.652586176161\\
2.24624624624625	7.63002876311539\\
2.23823823823824	7.60748451784791\\
2.23023023023023	7.58495355623241\\
2.22222222222222	7.56243599542968\\
2.21421421421421	7.53993195390417\\
2.20620620620621	7.51744155144098\\
2.1981981981982	7.494964909163\\
2.19019019019019	7.4725021495484\\
2.18218218218218	7.45005339644833\\
2.17417417417417	7.42761877510482\\
2.16616616616617	7.40519841216902\\
2.15815815815816	7.38279243571964\\
2.15015015015015	7.36040097528171\\
2.14214214214214	7.33802416184552\\
2.13413413413413	7.31566212788594\\
2.12612612612613	7.29331500738191\\
2.11811811811812	7.27098293583631\\
2.11011011011011	7.24866605029601\\
2.1021021021021	7.2263644893723\\
2.09409409409409	7.20407839326155\\
2.08608608608609	7.18180790376617\\
2.07807807807808	7.15955316431594\\
2.07007007007007	7.13731431998951\\
2.06206206206206	7.11509151753635\\
2.05405405405405	7.09288490539892\\
2.04604604604605	7.07069463373518\\
2.03803803803804	7.04852085444142\\
2.03003003003003	7.02636372117544\\
2.02202202202202	7.00422338938003\\
2.01401401401401	6.98210001630676\\
2.00600600600601	6.95999376104018\\
1.997997997998	6.93790478452233\\
1.98998998998999	6.91583324957751\\
1.98198198198198	6.8937793209376\\
1.97397397397397	6.8717431652675\\
1.96596596596597	6.84972495119114\\
1.95795795795796	6.82772484931771\\
1.94994994994995	6.80574303226833\\
1.94194194194194	6.7837796747031\\
1.93393393393393	6.76183495334847\\
1.92592592592593	6.73990904702507\\
1.91791791791792	6.71800213667588\\
1.90990990990991	6.6961144053948\\
1.9019019019019	6.67424603845565\\
1.89389389389389	6.65239722334154\\
1.88588588588589	6.63056814977465\\
1.87787787787788	6.60875900974646\\
1.86986986986987	6.58696999754837\\
1.86186186186186	6.56520130980272\\
1.85385385385385	6.5434531454943\\
1.84584584584585	6.52172570600226\\
1.83783783783784	6.50001919513242\\
1.82982982982983	6.47833381915012\\
1.82182182182182	6.45666978681342\\
1.81381381381381	6.4350273094068\\
1.80580580580581	6.41340660077534\\
1.7977977977978	6.39180787735929\\
1.78978978978979	6.37023135822922\\
1.78178178178178	6.34867726512148\\
1.77377377377377	6.32714582247435\\
1.76576576576577	6.30563725746445\\
1.75775775775776	6.28415180004381\\
1.74974974974975	6.26268968297735\\
1.74174174174174	6.24125114188086\\
1.73373373373373	6.21983641525946\\
1.72572572572573	6.19844574454669\\
1.71771771771772	6.1770793741439\\
1.70970970970971	6.15573755146036\\
1.7017017017017	6.13442052695372\\
1.69369369369369	6.11312855417113\\
1.68568568568569	6.0918618897908\\
1.67767767767768	6.07062079366409\\
1.66966966966967	6.04940552885817\\
1.66166166166166	6.02821636169924\\
1.65365365365365	6.00705356181617\\
1.64564564564565	5.98591740218484\\
1.63763763763764	5.96480815917291\\
1.62962962962963	5.9437261125852\\
1.62162162162162	5.92267154570958\\
1.61361361361361	5.90164474536344\\
1.60560560560561	5.88064600194075\\
1.5975975975976	5.85967560945959\\
1.58958958958959	5.83873386561031\\
1.58158158158158	5.81782107180425\\
1.57357357357357	5.79693753322297\\
1.56556556556557	5.77608355886811\\
1.55755755755756	5.75525946161179\\
1.54954954954955	5.73446555824753\\
1.54154154154154	5.7137021695418\\
1.53353353353353	5.69296962028612\\
1.52552552552553	5.67226823934969\\
1.51751751751752	5.65159835973261\\
1.50950950950951	5.63096031861966\\
1.5015015015015	5.61035445743461\\
1.49349349349349	5.58978112189516\\
1.48548548548549	5.56924066206834\\
1.47747747747748	5.5487334324265\\
1.46946946946947	5.52825979190387\\
1.46146146146146	5.50782010395363\\
1.45345345345345	5.48741473660551\\
1.44544544544545	5.46704406252397\\
1.43743743743744	5.4467084590668\\
1.42942942942943	5.42640830834438\\
1.42142142142142	5.4061439972793\\
1.41341341341341	5.38591591766659\\
1.40540540540541	5.36572446623439\\
1.3973973973974	5.34557004470507\\
1.38938938938939	5.32545305985693\\
1.38138138138138	5.3053739235862\\
1.37337337337337	5.28533305296962\\
1.36536536536537	5.26533087032738\\
1.35735735735736	5.24536780328647\\
1.34934934934935	5.22544428484446\\
1.34134134134134	5.20556075343364\\
1.33333333333333	5.18571765298556\\
1.32532532532533	5.16591543299585\\
1.31731731731732	5.14615454858941\\
1.30930930930931	5.1264354605859\\
1.3013013013013	5.10675863556548\\
1.29329329329329	5.08712454593485\\
1.28528528528529	5.06753366999349\\
1.27727727727728	5.04798649200005\\
1.26926926926927	5.02848350223904\\
1.26126126126126	5.00902519708754\\
1.25325325325325	4.9896120790821\\
1.24524524524525	4.97024465698564\\
1.23723723723724	4.95092344585447\\
1.22922922922923	4.93164896710527\\
1.22122122122122	4.91242174858203\\
1.21321321321321	4.89324232462293\\
1.20520520520521	4.87411123612711\\
1.1971971971972	4.85502903062126\\
1.18918918918919	4.83599626232596\\
1.18118118118118	4.81701349222187\\
1.17317317317317	4.79808128811546\\
1.16516516516517	4.77920022470452\\
1.15715715715716	4.76037088364313\\
1.14914914914915	4.74159385360621\\
1.14114114114114	4.72286973035353\\
1.13313313313313	4.70419911679308\\
1.12512512512513	4.68558262304382\\
1.11711711711712	4.6670208664976\\
1.10910910910911	4.64851447188038\\
1.1011011011011	4.63006407131246\\
1.09309309309309	4.61167030436785\\
1.08508508508509	4.59333381813248\\
1.07707707707708	4.57505526726135\\
1.06906906906907	4.55683531403447\\
1.06106106106106	4.53867462841148\\
1.05305305305305	4.52057388808483\\
1.04504504504505	4.50253377853154\\
1.03703703703704	4.48455499306331\\
1.02902902902903	4.46663823287496\\
1.02102102102102	4.44878420709105\\
1.01301301301301	4.43099363281062\\
1.00500500500501	4.41326723514983\\
0.996996996996997	4.39560574728253\\
0.988988988988989	4.37800991047854\\
0.980980980980981	4.36048047413953\\
0.972972972972973	4.34301819583237\\
0.964964964964965	4.32562384131988\\
0.956956956956957	4.30829818458875\\
0.948948948948949	4.29104200787455\\
0.940940940940941	4.27385610168364\\
0.932932932932933	4.25674126481194\\
0.924924924924925	4.23969830436029\\
0.916916916916917	4.22272803574623\\
0.908908908908909	4.20583128271223\\
0.900900900900901	4.18900887732998\\
0.892892892892893	4.17226166000071\\
0.884884884884885	4.15559047945132\\
0.876876876876877	4.13899619272621\\
0.868868868868869	4.12247966517452\\
0.860860860860861	4.10604177043271\\
0.852852852852853	4.08968339040227\\
0.844844844844845	4.07340541522229\\
0.836836836836837	4.05720874323686\\
0.828828828828829	4.04109428095698\\
0.820820820820821	4.02506294301678\\
0.812812812812813	4.009115652124\\
0.804804804804805	3.99325333900434\\
0.796796796796797	3.97747694233961\\
0.788788788788789	3.96178740869944\\
0.780780780780781	3.94618569246632\\
0.772772772772773	3.93067275575383\\
0.764764764764765	3.91524956831774\\
0.756756756756757	3.89991710745991\\
0.748748748748748	3.88467635792469\\
0.74074074074074	3.86952831178762\\
0.732732732732733	3.85447396833627\\
0.724724724724725	3.83951433394296\\
0.716716716716717	3.8246504219292\\
0.708708708708708	3.80988325242161\\
0.7007007007007	3.79521385219911\\
0.692692692692693	3.78064325453128\\
0.684684684684685	3.76617249900752\\
0.676676676676677	3.75180263135699\\
0.668668668668668	3.73753470325905\\
0.66066066066066	3.723369772144\\
0.652652652652653	3.70930890098407\\
0.644644644644645	3.69535315807426\\
0.636636636636637	3.68150361680317\\
0.628628628628628	3.66776135541334\\
0.62062062062062	3.65412745675125\\
0.612612612612613	3.64060300800657\\
0.604604604604605	3.62718910044077\\
0.596596596596597	3.61388682910483\\
0.588588588588588	3.60069729254592\\
0.58058058058058	3.58762159250309\\
0.572572572572573	3.57466083359177\\
0.564564564564565	3.56181612297702\\
0.556556556556557	3.5490885700356\\
0.548548548548548	3.53647928600662\\
0.54054054054054	3.52398938363088\\
0.532532532532533	3.51161997677895\\
0.524524524524525	3.4993721800678\\
0.516516516516517	3.48724710846622\\
0.508508508508508	3.47524587688897\\
0.5005005005005	3.46336959977972\\
0.492492492492492	3.45161939068299\\
0.484484484484485	3.43999636180503\\
0.476476476476477	3.42850162356393\\
0.468468468468468	3.41713628412906\\
0.46046046046046	3.40590144894999\\
0.452452452452452	3.39479822027519\\
0.444444444444445	3.38382769666064\\
0.436436436436437	3.37299097246867\\
0.428428428428428	3.36228913735738\\
0.42042042042042	3.35172327576078\\
0.412412412412412	3.34129446636019\\
0.404404404404405	3.33100378154709\\
0.396396396396397	3.32085228687799\\
0.388388388388388	3.31084104052158\\
0.38038038038038	3.30097109269875\\
0.372372372372372	3.29124348511589\\
0.364364364364365	3.28165925039194\\
0.356356356356357	3.27221941147983\\
0.348348348348348	3.26292498108285\\
0.34034034034034	3.2537769610664\\
0.332332332332332	3.24477634186598\\
0.324324324324325	3.23592410189183\\
0.316316316316317	3.22722120693106\\
0.308308308308308	3.21866860954787\\
0.3003003003003	3.21026724848259\\
0.292292292292292	3.20201804805031\\
0.284284284284285	3.19392191753983\\
0.276276276276277	3.18597975061372\\
0.268268268268268	3.17819242471033\\
0.26026026026026	3.17056080044842\\
0.252252252252252	3.16308572103549\\
0.244244244244245	3.15576801168038\\
0.236236236236236	3.14860847901119\\
0.228228228228228	3.14160791049937\\
0.22022022022022	3.13476707389067\\
0.212212212212212	3.12808671664413\\
0.204204204204204	3.12156756537972\\
0.196196196196196	3.11521032533566\\
0.188188188188188	3.10901567983628\\
0.18018018018018	3.10298428977131\\
0.172172172172172	3.09711679308741\\
0.164164164164164	3.09141380429288\\
0.156156156156156	3.08587591397642\\
0.148148148148148	3.08050368834066\\
0.14014014014014	3.07529766875143\\
0.132132132132132	3.07025837130352\\
0.124124124124124	3.0653862864037\\
0.116116116116116	3.06068187837176\\
0.108108108108108	3.0561455850604\\
0.1001001001001	3.05177781749454\\
0.0920920920920922	3.04757895953081\\
0.0840840840840844	3.0435493675379\\
0.0760760760760757	3.03968937009833\\
0.0680680680680679	3.03599926773219\\
0.06006006006006	3.03247933264344\\
0.0520520520520522	3.02912980848926\\
0.0440440440440444	3.02595091017289\\
0.0360360360360357	3.02294282366037\\
0.0280280280280278	3.02010570582149\\
0.02002002002002	3.01743968429542\\
0.0120120120120122	3.01494485738107\\
0.00400400400400436	3.01262129395263\\
-0.00400400400400391	3.01046903340024\\
-0.0120120120120122	3.00848808559609\\
-0.02002002002002	3.00667843088593\\
-0.0280280280280278	3.00504002010593\\
-0.0360360360360361	3.00357277462512\\
-0.0440440440440439	3.00227658641296\\
-0.0520520520520522	3.00115131813223\\
-0.06006006006006	3.00019680325688\\
-0.0680680680680679	2.99941284621458\\
-0.0760760760760761	2.99879922255386\\
-0.084084084084084	2.9983556791353\\
-0.0920920920920922	2.99808193434655\\
-0.1001001001001	2.99797767834067\\
-0.108108108108108	2.99804257329733\\
-0.116116116116116	2.99827625370642\\
-0.124124124124124	2.99867832667348\\
-0.132132132132132	2.99924837224629\\
-0.14014014014014	2.99998594376216\\
-0.148148148148148	3.0008905682151\\
-0.156156156156156	3.0019617466423\\
-0.164164164164164	3.00319895452912\\
-0.172172172172172	3.00460164223186\\
-0.18018018018018	3.00616923541759\\
-0.188188188188188	3.00790113552016\\
-0.196196196196196	3.00979672021165\\
-0.204204204204204	3.01185534388838\\
-0.212212212212212	3.01407633817065\\
-0.22022022022022	3.01645901241534\\
-0.228228228228228	3.01900265424056\\
-0.236236236236236	3.02170653006134\\
-0.244244244244244	3.02456988563562\\
-0.252252252252252	3.02759194661955\\
-0.26026026026026	3.03077191913134\\
-0.268268268268268	3.03410899032259\\
-0.276276276276276	3.03760232895638\\
-0.284284284284284	3.04125108599122\\
-0.292292292292292	3.04505439516996\\
-0.3003003003003	3.04901137361273\\
-0.308308308308308	3.05312112241334\\
-0.316316316316316	3.05738272723794\\
-0.324324324324324	3.06179525892546\\
-0.332332332332332	3.06635777408884\\
-0.34034034034034	3.07106931571634\\
-0.348348348348348	3.07592891377218\\
-0.356356356356356	3.08093558579582\\
-0.364364364364364	3.08608833749898\\
-0.372372372372372	3.09138616336007\\
-0.38038038038038	3.09682804721493\\
-0.388388388388389	3.10241296284367\\
-0.396396396396396	3.10813987455268\\
-0.404404404404405	3.11400773775137\\
-0.412412412412412	3.12001549952314\\
-0.42042042042042	3.12616209918987\\
-0.428428428428429	3.13244646886961\\
-0.436436436436436	3.13886753402694\\
-0.444444444444445	3.14542421401547\\
-0.452452452452452	3.15211542261221\\
-0.46046046046046	3.15894006854333\\
-0.468468468468469	3.16589705600093\\
-0.476476476476476	3.17298528515062\\
-0.484484484484485	3.18020365262939\\
-0.492492492492492	3.18755105203372\\
-0.5005005005005	3.19502637439754\\
-0.508508508508509	3.2026285086598\\
-0.516516516516516	3.21035634212155\\
-0.524524524524525	3.21820876089226\\
-0.532532532532533	3.22618465032529\\
-0.54054054054054	3.23428289544234\\
-0.548548548548549	3.24250238134679\\
-0.556556556556556	3.25084199362579\\
-0.564564564564565	3.25930061874119\\
-0.572572572572573	3.26787714440907\\
-0.58058058058058	3.27657045996798\\
-0.588588588588589	3.2853794567358\\
-0.596596596596596	3.29430302835531\\
-0.604604604604605	3.30334007112846\\
-0.612612612612613	3.31248948433933\\
-0.62062062062062	3.32175017056593\\
-0.628628628628629	3.3311210359809\\
-0.636636636636636	3.34060099064108\\
-0.644644644644645	3.35018894876626\\
-0.652652652652653	3.35988382900698\\
-0.660660660660661	3.36968455470173\\
-0.668668668668669	3.37959005412348\\
-0.676676676676677	3.38959926071589\\
-0.684684684684685	3.39971111331911\\
-0.692692692692693	3.40992455638555\\
-0.700700700700701	3.4202385401857\\
-0.708708708708709	3.43065202100409\\
-0.716716716716717	3.44116396132571\\
-0.724724724724725	3.45177333001301\\
-0.732732732732733	3.46247910247362\\
-0.740740740740741	3.47328026081911\\
-0.748748748748749	3.48417579401483\\
-0.756756756756757	3.49516469802118\\
-0.764764764764765	3.50624597592638\\
-0.772772772772773	3.51741863807105\\
-0.780780780780781	3.52868170216474\\
-0.788788788788789	3.54003419339466\\
-0.796796796796797	3.55147514452675\\
-0.804804804804805	3.56300359599941\\
-0.812812812812813	3.57461859600999\\
-0.820820820820821	3.58631920059428\\
-0.828828828828829	3.59810447369924\\
-0.836836836836837	3.60997348724913\\
-0.844844844844845	3.62192532120532\\
-0.852852852852853	3.63395906361981\\
-0.860860860860861	3.64607381068291\\
-0.868868868868869	3.65826866676506\\
-0.876876876876877	3.6705427444531\\
-0.884884884884885	3.68289516458112\\
-0.892892892892893	3.69532505625619\\
-0.900900900900901	3.70783155687901\\
-0.908908908908909	3.72041381215973\\
-0.916916916916917	3.73307097612924\\
-0.924924924924925	3.74580221114582\\
-0.932932932932933	3.75860668789766\\
-0.940940940940941	3.77148358540117\\
-0.948948948948949	3.78443209099538\\
-0.956956956956957	3.79745140033254\\
-0.964964964964965	3.81054071736509\\
-0.972972972972973	3.82369925432917\\
-0.980980980980981	3.83692623172483\\
-0.988988988988989	3.85022087829301\\
-0.996996996996997	3.86358243098964\\
-1.00500500500501	3.87701013495664\\
-1.01301301301301	3.89050324349042\\
-1.02102102102102	3.90406101800762\\
-1.02902902902903	3.9176827280084\\
-1.03703703703704	3.93136765103744\\
-1.04504504504505	3.94511507264265\\
-1.05305305305305	3.95892428633178\\
-1.06106106106106	3.9727945935271\\
-1.06906906906907	3.98672530351816\\
-1.07707707707708	4.00071573341275\\
-1.08508508508509	4.01476520808625\\
-1.09309309309309	4.02887306012941\\
-1.1011011011011	4.04303862979465\\
-1.10910910910911	4.05726126494101\\
-1.11711711711712	4.0715403209778\\
-1.12512512512513	4.08587516080708\\
-1.13313313313313	4.10026515476505\\
-1.14114114114114	4.11470968056242\\
-1.14914914914915	4.12920812322379\\
-1.15715715715716	4.14375987502627\\
-1.16516516516517	4.15836433543719\\
-1.17317317317317	4.17302091105119\\
-1.18118118118118	4.18772901552659\\
-1.18918918918919	4.20248806952125\\
-1.1971971971972	4.21729750062785\\
-1.20520520520521	4.2321567433087\\
-1.21321321321321	4.2470652388302\\
-1.22122122122122	4.26202243519693\\
-1.22922922922923	4.27702778708539\\
-1.23723723723724	4.29208075577763\\
-1.24524524524525	4.3071808090945\\
-1.25325325325325	4.32232742132892\\
-1.26126126126126	4.33752007317896\\
-1.26926926926927	4.35275825168085\\
-1.27727727727728	4.36804145014201\\
-1.28528528528529	4.38336916807405\\
-1.29329329329329	4.3987409111259\\
-1.3013013013013	4.41415619101692\\
-1.30930930930931	4.4296145254702\\
-1.31731731731732	4.44511543814603\\
-1.32532532532533	4.46065845857549\\
-1.33333333333333	4.47624312209427\\
-1.34134134134134	4.49186896977678\\
-1.34934934934935	4.50753554837044\\
-1.35735735735736	4.52324241023032\\
-1.36536536536537	4.53898911325406\\
-1.37337337337337	4.55477522081713\\
-1.38138138138138	4.5706003017084\\
-1.38938938938939	4.5864639300662\\
-1.3973973973974	4.6023656853146\\
-1.40540540540541	4.6183051521002\\
-1.41341341341341	4.63428192022937\\
-1.42142142142142	4.65029558460578\\
-1.42942942942943	4.66634574516855\\
-1.43743743743744	4.68243200683072\\
-1.44544544544545	4.69855397941832\\
-1.45345345345345	4.71471127760977\\
-1.46146146146146	4.7309035208759\\
-1.46946946946947	4.74713033342045\\
-1.47747747747748	4.76339134412101\\
-1.48548548548549	4.77968618647057\\
-1.49349349349349	4.79601449851953\\
-1.5015015015015	4.81237592281825\\
-1.50950950950951	4.82877010636019\\
-1.51751751751752	4.84519670052551\\
-1.52552552552553	4.86165536102532\\
-1.53353353353353	4.87814574784633\\
-1.54154154154154	4.89466752519623\\
-1.54954954954955	4.9112203614495\\
-1.55755755755756	4.92780392909382\\
-1.56556556556557	4.94441790467707\\
-1.57357357357357	4.96106196875482\\
-1.58158158158158	4.97773580583847\\
-1.58958958958959	4.9944391043439\\
-1.5975975975976	5.01117155654067\\
-1.60560560560561	5.02793285850188\\
-1.61361361361361	5.04472271005446\\
-1.62162162162162	5.0615408147301\\
-1.62962962962963	5.07838687971678\\
-1.63763763763764	5.09526061581078\\
-1.64564564564565	5.11216173736927\\
-1.65365365365365	5.1290899622635\\
-1.66166166166166	5.14604501183253\\
-1.66966966966967	5.16302661083744\\
-1.67767767767768	5.18003448741624\\
-1.68568568568569	5.19706837303915\\
-1.69369369369369	5.2141280024646\\
-1.7017017017017	5.23121311369561\\
-1.70970970970971	5.24832344793686\\
-1.71771771771772	5.26545874955216\\
-1.72572572572573	5.28261876602256\\
-1.73373373373373	5.2998032479049\\
-1.74174174174174	5.31701194879098\\
-1.74974974974975	5.33424462526713\\
-1.75775775775776	5.35150103687444\\
-1.76576576576577	5.36878094606934\\
-1.77377377377377	5.38608411818485\\
-1.78178178178178	5.40341032139219\\
-1.78978978978979	5.42075932666302\\
-1.7977977977978	5.43813090773205\\
-1.80580580580581	5.45552484106023\\
-1.81381381381381	5.4729409057984\\
-1.82182182182182	5.49037888375139\\
-1.82982982982983	5.50783855934264\\
-1.83783783783784	5.52531971957926\\
-1.84584584584585	5.54282215401758\\
-1.85385385385385	5.56034565472914\\
-1.86186186186186	5.57789001626715\\
-1.86986986986987	5.5954550356334\\
-1.87787787787788	5.61304051224561\\
-1.88588588588589	5.63064624790523\\
-1.89389389389389	5.64827204676566\\
-1.9019019019019	5.66591771530094\\
-1.90990990990991	5.68358306227482\\
-1.91791791791792	5.70126789871029\\
-1.92592592592593	5.71897203785952\\
-1.93393393393393	5.73669529517415\\
-1.94194194194194	5.75443748827613\\
-1.94994994994995	5.7721984369288\\
-1.95795795795796	5.78997796300847\\
-1.96596596596597	5.80777589047638\\
-1.97397397397397	5.82559204535101\\
-1.98198198198198	5.84342625568078\\
-1.98998998998999	5.86127835151722\\
-1.997997997998	5.87914816488835\\
-2.00600600600601	5.89703552977257\\
-2.01401401401401	5.91494028207283\\
-2.02202202202202	5.93286225959123\\
-2.03003003003003	5.9508013020039\\
-2.03803803803804	5.96875725083628\\
-2.04604604604605	5.98672994943875\\
-2.05405405405405	6.00471924296255\\
-2.06206206206206	6.02272497833608\\
-2.07007007007007	6.04074700424155\\
-2.07807807807808	6.05878517109188\\
-2.08608608608609	6.07683933100802\\
-2.09409409409409	6.09490933779653\\
-2.1021021021021	6.11299504692744\\
-2.11011011011011	6.13109631551256\\
-2.11811811811812	6.14921300228392\\
-2.12612612612613	6.16734496757262\\
-2.13413413413413	6.18549207328799\\
-2.14214214214214	6.20365418289693\\
-2.15015015015015	6.22183116140371\\
-2.15815815815816	6.24002287532989\\
-2.16616616616617	6.25822919269462\\
-2.17417417417417	6.2764499829952\\
-2.18218218218218	6.2946851171879\\
-2.19019019019019	6.31293446766903\\
-2.1981981981982	6.33119790825635\\
-2.20620620620621	6.34947531417066\\
-2.21421421421421	6.36776656201769\\
-2.22222222222222	6.38607152977023\\
-2.23023023023023	6.40439009675056\\
-2.23823823823824	6.42272214361303\\
-2.24624624624625	6.44106755232698\\
-2.25425425425425	6.45942620615985\\
-2.26226226226226	6.47779798966055\\
-2.27027027027027	6.49618278864303\\
-2.27827827827828	6.51458049017013\\
-2.28628628628629	6.53299098253764\\
-2.29429429429429	6.55141415525853\\
-2.3023023023023	6.56984989904749\\
-2.31031031031031	6.58829810580565\\
-2.31831831831832	6.60675866860544\\
-2.32632632632633	6.62523148167584\\
-2.33433433433433	6.64371644038763\\
-2.34234234234234	6.66221344123899\\
-2.35035035035035	6.68072238184125\\
-2.35835835835836	6.69924316090485\\
-2.36636636636637	6.71777567822549\\
-2.37437437437437	6.73631983467045\\
-2.38238238238238	6.75487553216516\\
-2.39039039039039	6.77344267367993\\
-2.3983983983984	6.7920211632168\\
-2.40640640640641	6.81061090579671\\
-2.41441441441441	6.82921180744671\\
-2.42242242242242	6.84782377518745\\
-2.43043043043043	6.86644671702077\\
-2.43843843843844	6.88508054191753\\
-2.44644644644645	6.90372515980553\\
-2.45445445445445	6.9223804815577\\
-2.46246246246246	6.94104641898033\\
-2.47047047047047	6.95972288480158\\
-2.47847847847848	6.97840979266008\\
-2.48648648648649	6.99710705709371\\
-2.49449449449449	7.01581459352852\\
-2.5025025025025	7.03453231826782\\
-2.51051051051051	7.05326014848141\\
-2.51851851851852	7.07199800219498\\
-2.52652652652653	7.09074579827961\\
-2.53453453453453	7.10950345644146\\
-2.54254254254254	7.12827089721159\\
-2.55055055055055	7.14704804193589\\
-2.55855855855856	7.16583481276522\\
-2.56656656656657	7.18463113264556\\
-2.57457457457457	7.20343692530846\\
-2.58258258258258	7.22225211526145\\
-2.59059059059059	7.24107662777873\\
-2.5985985985986	7.25991038889189\\
-2.60660660660661	7.27875332538079\\
-2.61461461461461	7.29760536476454\\
-2.62262262262262	7.31646643529267\\
-2.63063063063063	7.33533646593634\\
-2.63863863863864	7.35421538637971\\
-2.64664664664665	7.37310312701143\\
-2.65465465465465	7.39199961891621\\
-2.66266266266266	7.41090479386654\\
-2.67067067067067	7.42981858431456\\
-2.67867867867868	7.44874092338388\\
-2.68668668668669	7.46767174486175\\
-2.69469469469469	7.48661098319109\\
-2.7027027027027	7.50555857346283\\
-2.71071071071071	7.52451445140822\\
-2.71871871871872	7.54347855339129\\
-2.72672672672673	7.56245081640145\\
-2.73473473473473	7.58143117804607\\
-2.74274274274274	7.60041957654335\\
-2.75075075075075	7.61941595071506\\
-2.75875875875876	7.63842023997956\\
-2.76676676676677	7.65743238434482\\
-2.77477477477477	7.67645232440156\\
-2.78278278278278	7.69548000131648\\
-2.79079079079079	7.71451535682557\\
-2.7987987987988	7.73355833322751\\
-2.80680680680681	7.75260887337718\\
-2.81481481481481	7.77166692067922\\
-2.82282282282282	7.79073241908171\\
-2.83083083083083	7.80980531306988\\
-2.83883883883884	7.82888554766\\
-2.84684684684685	7.84797306839324\\
-2.85485485485485	7.86706782132967\\
-2.86286286286286	7.88616975304234\\
-2.87087087087087	7.9052788106114\\
-2.87887887887888	7.92439494161838\\
-2.88688688688689	7.94351809414042\\
-2.89489489489489	7.96264821674467\\
-2.9029029029029	7.98178525848274\\
-2.91091091091091	8.00092916888523\\
-2.91891891891892	8.02007989795629\\
-2.92692692692693	8.03923739616829\\
-2.93493493493493	8.05840161445655\\
-2.94294294294294	8.07757250421414\\
-2.95095095095095	8.09675001728673\\
-2.95895895895896	8.11593410596756\\
-2.96696696696697	8.13512472299236\\
-2.97497497497497	8.15432182153448\\
-2.98298298298298	8.17352535519997\\
-2.99099099099099	8.1927352780228\\
-2.998998998999	8.21195154446007\\
-3.00700700700701	8.23117410938735\\
-3.01501501501502	8.25040292809401\\
-3.02302302302302	8.26963795627868\\
-3.03103103103103	8.28887915004473\\
-3.03903903903904	8.30812646589579\\
-3.04704704704705	8.32737986073138\\
-3.05505505505506	8.34663929184252\\
-3.06306306306306	8.36590471690751\\
-3.07107107107107	8.38517609398761\\
-3.07907907907908	8.40445338152291\\
-3.08708708708709	8.4237365383282\\
-3.0950950950951	8.44302552358886\\
-3.1031031031031	8.46232029685686\\
-3.11111111111111	8.48162081804678\\
-3.11911911911912	8.50092704743187\\
-3.12712712712713	8.52023894564018\\
-3.13513513513514	8.53955647365073\\
-3.14314314314314	8.55887959278974\\
-3.15115115115115	8.57820826472688\\
-3.15915915915916	8.59754245147157\\
-3.16716716716717	8.61688211536939\\
-3.17517517517518	8.63622721909842\\
-3.18318318318318	8.65557772566572\\
-3.19119119119119	8.67493359840383\\
-3.1991991991992	8.69429480096727\\
-3.20720720720721	8.71366129732916\\
-3.21521521521522	8.7330330517778\\
-3.22322322322322	8.75241002891337\\
-3.23123123123123	8.7717921936446\\
-3.23923923923924	8.79117951118551\\
-3.24724724724725	8.81057194705223\\
-3.25525525525526	8.82996946705979\\
-3.26326326326326	8.84937203731899\\
-3.27127127127127	8.86877962423328\\
-3.27927927927928	8.88819219449575\\
-3.28728728728729	8.90760971508602\\
-3.2952952952953	8.92703215326733\\
-3.3033033033033	8.94645947658355\\
-3.31131131131131	8.96589165285624\\
-3.31931931931932	8.98532865018184\\
-3.32732732732733	9.00477043692873\\
-3.33533533533534	9.02421698173447\\
-3.34334334334334	9.04366825350304\\
-3.35135135135135	9.06312422140203\\
-3.35935935935936	9.08258485485999\\
-3.36736736736737	9.10205012356368\\
-3.37537537537538	9.12151999745551\\
-3.38338338338338	9.14099444673083\\
-3.39139139139139	9.16047344183543\\
-3.3993993993994	9.1799569534629\\
-3.40740740740741	9.19944495255216\\
-3.41541541541542	9.21893741028496\\
-3.42342342342342	9.23843429808341\\
-3.43143143143143	9.25793558760752\\
-3.43943943943944	9.27744125075281\\
-3.44744744744745	9.29695125964795\\
-3.45545545545546	9.31646558665236\\
-3.46346346346346	9.33598420435395\\
-3.47147147147147	9.35550708556676\\
-3.47947947947948	9.37503420332873\\
-3.48748748748749	9.39456553089945\\
-3.4954954954955	9.41410104175792\\
-3.5035035035035	9.4336407096004\\
-3.51151151151151	9.45318450833821\\
-3.51951951951952	9.47273241209559\\
-3.52752752752753	9.49228439520761\\
-3.53553553553554	9.51184043221807\\
-3.54354354354354	9.5314004978774\\
-3.55155155155155	9.55096456714066\\
-3.55955955955956	9.57053261516552\\
-3.56756756756757	9.5901046173102\\
-3.57557557557558	9.60968054913161\\
-3.58358358358358	9.62926038638327\\
-3.59159159159159	9.64884410501348\\
-3.5995995995996	9.66843168116338\\
-3.60760760760761	9.68802309116505\\
-3.61561561561562	9.70761831153966\\
-3.62362362362362	9.72721731899563\\
-3.63163163163163	9.74682009042684\\
-3.63963963963964	9.76642660291075\\
-3.64764764764765	9.7860368337067\\
-3.65565565565566	9.80565076025411\\
-3.66366366366366	9.82526836017075\\
-3.67167167167167	9.84488961125101\\
-3.67967967967968	9.86451449146424\\
-3.68768768768769	9.88414297895299\\
-3.6956956956957	9.90377505203144\\
-3.7037037037037	9.92341068918367\\
-3.71171171171171	9.94304986906211\\
-3.71971971971972	9.96269257048589\\
-3.72772772772773	9.98233877243923\\
-3.73573573573574	10.0019884540699\\
-3.74374374374374	10.0216415946878\\
-3.75175175175175	10.041298173763\\
-3.75975975975976	10.0609581709248\\
-3.76776776776777	10.0806215659597\\
-3.77577577577578	10.1002883388104\\
-3.78378378378378	10.1199584695737\\
-3.79179179179179	10.1396319384999\\
-3.7997997997998	10.1593087259905\\
-3.80780780780781	10.1789888125973\\
-3.81581581581582	10.1986721790209\\
-3.82382382382382	10.2183588061093\\
-3.83183183183183	10.2380486748562\\
-3.83983983983984	10.2577417664003\\
-3.84784784784785	10.2774380620235\\
-3.85585585585586	10.2971375431494\\
-3.86386386386386	10.3168401913425\\
-3.87187187187187	10.3365459883066\\
-3.87987987987988	10.3562549158835\\
-3.88788788788789	10.3759669560517\\
-3.8958958958959	10.3956820909254\\
-3.9039039039039	10.4154003027528\\
-3.91191191191191	10.4351215739155\\
-3.91991991991992	10.4548458869265\\
-3.92792792792793	10.4745732244298\\
-3.93593593593594	10.4943035691984\\
-3.94394394394394	10.514036904134\\
-3.95195195195195	10.5337732122649\\
-3.95995995995996	10.5535124767456\\
-3.96796796796797	10.5732546808553\\
-3.97597597597598	10.5929998079968\\
-3.98398398398398	10.6127478416954\\
-3.99199199199199	10.632498765598\\
-4	10.6522525634714\\
}--cycle;

\addlegendentry{$\pm 2\sigma$};

\addplot [color=mycolor2,solid]
  table[row sep=crcr]{%
-4	-1.01643966400201\\
-3.99199199199199	-1.01434468052471\\
-3.98398398398398	-1.01224969704741\\
-3.97597597597598	-1.01015471357011\\
-3.96796796796797	-1.00805973009281\\
-3.95995995995996	-1.00596474661551\\
-3.95195195195195	-1.00386976313822\\
-3.94394394394394	-1.00177477966092\\
-3.93593593593594	-0.999679796183616\\
-3.92792792792793	-0.997584812706316\\
-3.91991991991992	-0.995489829229017\\
-3.91191191191191	-0.993394845751717\\
-3.9039039039039	-0.991299862274417\\
-3.8958958958959	-0.989204878797118\\
-3.88788788788789	-0.987109895319818\\
-3.87987987987988	-0.985014911842519\\
-3.87187187187187	-0.982919928365219\\
-3.86386386386386	-0.980824944887919\\
-3.85585585585586	-0.97872996141062\\
-3.84784784784785	-0.97663497793332\\
-3.83983983983984	-0.974539994456021\\
-3.83183183183183	-0.972445010978721\\
-3.82382382382382	-0.970350027501422\\
-3.81581581581582	-0.968255044024122\\
-3.80780780780781	-0.966160060546822\\
-3.7997997997998	-0.964065077069523\\
-3.79179179179179	-0.961970093592223\\
-3.78378378378378	-0.959875110114924\\
-3.77577577577578	-0.957780126637624\\
-3.76776776776777	-0.955685143160324\\
-3.75975975975976	-0.953590159683025\\
-3.75175175175175	-0.951495176205725\\
-3.74374374374374	-0.949400192728426\\
-3.73573573573574	-0.947305209251126\\
-3.72772772772773	-0.945210225773826\\
-3.71971971971972	-0.943115242296527\\
-3.71171171171171	-0.941020258819227\\
-3.7037037037037	-0.938925275341928\\
-3.6956956956957	-0.936830291864628\\
-3.68768768768769	-0.934735308387328\\
-3.67967967967968	-0.932640324910029\\
-3.67167167167167	-0.930545341432729\\
-3.66366366366366	-0.92845035795543\\
-3.65565565565566	-0.92635537447813\\
-3.64764764764765	-0.92426039100083\\
-3.63963963963964	-0.922165407523531\\
-3.63163163163163	-0.920070424046231\\
-3.62362362362362	-0.917975440568932\\
-3.61561561561562	-0.915880457091632\\
-3.60760760760761	-0.913785473614332\\
-3.5995995995996	-0.911690490137033\\
-3.59159159159159	-0.909595506659733\\
-3.58358358358358	-0.907500523182434\\
-3.57557557557558	-0.905405539705134\\
-3.56756756756757	-0.903310556227835\\
-3.55955955955956	-0.901215572750535\\
-3.55155155155155	-0.899120589273235\\
-3.54354354354354	-0.897025605795936\\
-3.53553553553554	-0.894930622318636\\
-3.52752752752753	-0.892835638841337\\
-3.51951951951952	-0.890740655364037\\
-3.51151151151151	-0.888645671886737\\
-3.5035035035035	-0.886550688409438\\
-3.4954954954955	-0.884455704932138\\
-3.48748748748749	-0.882360721454839\\
-3.47947947947948	-0.880265737977539\\
-3.47147147147147	-0.87817075450024\\
-3.46346346346346	-0.87607577102294\\
-3.45545545545546	-0.87398078754564\\
-3.44744744744745	-0.871885804068341\\
-3.43943943943944	-0.869790820591041\\
-3.43143143143143	-0.867695837113742\\
-3.42342342342342	-0.865600853636442\\
-3.41541541541542	-0.863505870159142\\
-3.40740740740741	-0.861410886681843\\
-3.3993993993994	-0.859315903204543\\
-3.39139139139139	-0.857220919727244\\
-3.38338338338338	-0.855125936249944\\
-3.37537537537538	-0.853030952772644\\
-3.36736736736737	-0.850935969295345\\
-3.35935935935936	-0.848840985818045\\
-3.35135135135135	-0.846746002340746\\
-3.34334334334334	-0.844651018863446\\
-3.33533533533534	-0.842556035386146\\
-3.32732732732733	-0.840461051908847\\
-3.31931931931932	-0.838366068431547\\
-3.31131131131131	-0.836271084954248\\
-3.3033033033033	-0.834176101476948\\
-3.2952952952953	-0.832081117999649\\
-3.28728728728729	-0.829986134522349\\
-3.27927927927928	-0.827891151045049\\
-3.27127127127127	-0.82579616756775\\
-3.26326326326326	-0.82370118409045\\
-3.25525525525526	-0.821606200613151\\
-3.24724724724725	-0.819511217135851\\
-3.23923923923924	-0.817416233658551\\
-3.23123123123123	-0.815321250181252\\
-3.22322322322322	-0.813226266703952\\
-3.21521521521522	-0.811131283226653\\
-3.20720720720721	-0.809036299749353\\
-3.1991991991992	-0.806941316272053\\
-3.19119119119119	-0.804846332794754\\
-3.18318318318318	-0.802751349317454\\
-3.17517517517518	-0.800656365840155\\
-3.16716716716717	-0.798561382362855\\
-3.15915915915916	-0.796466398885555\\
-3.15115115115115	-0.794371415408256\\
-3.14314314314314	-0.792276431930956\\
-3.13513513513514	-0.790181448453657\\
-3.12712712712713	-0.788086464976357\\
-3.11911911911912	-0.785991481499058\\
-3.11111111111111	-0.783896498021758\\
-3.1031031031031	-0.781801514544458\\
-3.0950950950951	-0.779706531067159\\
-3.08708708708709	-0.777611547589859\\
-3.07907907907908	-0.77551656411256\\
-3.07107107107107	-0.77342158063526\\
-3.06306306306306	-0.77132659715796\\
-3.05505505505506	-0.769231613680661\\
-3.04704704704705	-0.767136630203361\\
-3.03903903903904	-0.765041646726062\\
-3.03103103103103	-0.762946663248762\\
-3.02302302302302	-0.760851679771462\\
-3.01501501501502	-0.758756696294163\\
-3.00700700700701	-0.756661712816863\\
-2.998998998999	-0.754566729339564\\
-2.99099099099099	-0.752471745862264\\
-2.98298298298298	-0.750376762384965\\
-2.97497497497497	-0.748281778907665\\
-2.96696696696697	-0.746186795430365\\
-2.95895895895896	-0.744091811953066\\
-2.95095095095095	-0.741996828475766\\
-2.94294294294294	-0.739901844998467\\
-2.93493493493493	-0.737806861521167\\
-2.92692692692693	-0.735711878043867\\
-2.91891891891892	-0.733616894566568\\
-2.91091091091091	-0.731521911089268\\
-2.9029029029029	-0.729426927611969\\
-2.89489489489489	-0.727331944134669\\
-2.88688688688689	-0.72523696065737\\
-2.87887887887888	-0.72314197718007\\
-2.87087087087087	-0.72104699370277\\
-2.86286286286286	-0.718952010225471\\
-2.85485485485485	-0.716857026748171\\
-2.84684684684685	-0.714762043270872\\
-2.83883883883884	-0.712667059793572\\
-2.83083083083083	-0.710572076316272\\
-2.82282282282282	-0.708477092838973\\
-2.81481481481481	-0.706382109361673\\
-2.80680680680681	-0.704287125884374\\
-2.7987987987988	-0.702192142407074\\
-2.79079079079079	-0.700097158929774\\
-2.78278278278278	-0.698002175452475\\
-2.77477477477477	-0.695907191975175\\
-2.76676676676677	-0.693812208497876\\
-2.75875875875876	-0.691717225020576\\
-2.75075075075075	-0.689622241543276\\
-2.74274274274274	-0.687527258065977\\
-2.73473473473473	-0.685432274588677\\
-2.72672672672673	-0.683337291111378\\
-2.71871871871872	-0.681242307634078\\
-2.71071071071071	-0.679147324156778\\
-2.7027027027027	-0.677052340679479\\
-2.69469469469469	-0.674957357202179\\
-2.68668668668669	-0.67286237372488\\
-2.67867867867868	-0.67076739024758\\
-2.67067067067067	-0.668672406770281\\
-2.66266266266266	-0.666577423292981\\
-2.65465465465465	-0.664482439815681\\
-2.64664664664665	-0.662387456338382\\
-2.63863863863864	-0.660292472861082\\
-2.63063063063063	-0.658197489383783\\
-2.62262262262262	-0.656102505906483\\
-2.61461461461461	-0.654007522429183\\
-2.60660660660661	-0.651912538951884\\
-2.5985985985986	-0.649817555474584\\
-2.59059059059059	-0.647722571997285\\
-2.58258258258258	-0.645627588519985\\
-2.57457457457457	-0.643532605042686\\
-2.56656656656657	-0.641437621565386\\
-2.55855855855856	-0.639342638088086\\
-2.55055055055055	-0.637247654610787\\
-2.54254254254254	-0.635152671133487\\
-2.53453453453453	-0.633057687656188\\
-2.52652652652653	-0.630962704178888\\
-2.51851851851852	-0.628867720701588\\
-2.51051051051051	-0.626772737224289\\
-2.5025025025025	-0.624677753746989\\
-2.49449449449449	-0.62258277026969\\
-2.48648648648649	-0.62048778679239\\
-2.47847847847848	-0.61839280331509\\
-2.47047047047047	-0.616297819837791\\
-2.46246246246246	-0.614202836360491\\
-2.45445445445445	-0.612107852883192\\
-2.44644644644645	-0.610012869405892\\
-2.43843843843844	-0.607917885928592\\
-2.43043043043043	-0.605822902451293\\
-2.42242242242242	-0.603727918973993\\
-2.41441441441441	-0.601632935496694\\
-2.40640640640641	-0.599537952019394\\
-2.3983983983984	-0.597442968542094\\
-2.39039039039039	-0.595347985064795\\
-2.38238238238238	-0.593253001587495\\
-2.37437437437437	-0.591158018110196\\
-2.36636636636637	-0.589063034632896\\
-2.35835835835836	-0.586968051155596\\
-2.35035035035035	-0.584873067678297\\
-2.34234234234234	-0.582778084200997\\
-2.33433433433433	-0.580683100723698\\
-2.32632632632633	-0.578588117246398\\
-2.31831831831832	-0.576493133769098\\
-2.31031031031031	-0.574398150291799\\
-2.3023023023023	-0.572303166814499\\
-2.29429429429429	-0.5702081833372\\
-2.28628628628629	-0.5681131998599\\
-2.27827827827828	-0.566018216382601\\
-2.27027027027027	-0.563923232905301\\
-2.26226226226226	-0.561828249428001\\
-2.25425425425425	-0.559733265950702\\
-2.24624624624625	-0.557638282473402\\
-2.23823823823824	-0.555543298996103\\
-2.23023023023023	-0.553448315518803\\
-2.22222222222222	-0.551353332041503\\
-2.21421421421421	-0.549258348564204\\
-2.20620620620621	-0.547163365086904\\
-2.1981981981982	-0.545068381609605\\
-2.19019019019019	-0.542973398132305\\
-2.18218218218218	-0.540878414655006\\
-2.17417417417417	-0.538783431177706\\
-2.16616616616617	-0.536688447700406\\
-2.15815815815816	-0.534593464223107\\
-2.15015015015015	-0.532498480745807\\
-2.14214214214214	-0.530403497268508\\
-2.13413413413413	-0.528308513791208\\
-2.12612612612613	-0.526213530313908\\
-2.11811811811812	-0.524118546836609\\
-2.11011011011011	-0.522023563359309\\
-2.1021021021021	-0.51992857988201\\
-2.09409409409409	-0.51783359640471\\
-2.08608608608609	-0.51573861292741\\
-2.07807807807808	-0.513643629450111\\
-2.07007007007007	-0.511548645972811\\
-2.06206206206206	-0.509453662495512\\
-2.05405405405405	-0.507358679018212\\
-2.04604604604605	-0.505263695540913\\
-2.03803803803804	-0.503168712063613\\
-2.03003003003003	-0.501073728586313\\
-2.02202202202202	-0.498978745109014\\
-2.01401401401401	-0.496883761631714\\
-2.00600600600601	-0.494788778154415\\
-1.997997997998	-0.492693794677115\\
-1.98998998998999	-0.490598811199815\\
-1.98198198198198	-0.488503827722516\\
-1.97397397397397	-0.486408844245216\\
-1.96596596596597	-0.484313860767917\\
-1.95795795795796	-0.482218877290617\\
-1.94994994994995	-0.480123893813317\\
-1.94194194194194	-0.478028910336018\\
-1.93393393393393	-0.475933926858718\\
-1.92592592592593	-0.473838943381419\\
-1.91791791791792	-0.471743959904119\\
-1.90990990990991	-0.469648976426819\\
-1.9019019019019	-0.46755399294952\\
-1.89389389389389	-0.46545900947222\\
-1.88588588588589	-0.463364025994921\\
-1.87787787787788	-0.461269042517621\\
-1.86986986986987	-0.459174059040322\\
-1.86186186186186	-0.457079075563022\\
-1.85385385385385	-0.454984092085722\\
-1.84584584584585	-0.452889108608423\\
-1.83783783783784	-0.450794125131123\\
-1.82982982982983	-0.448699141653824\\
-1.82182182182182	-0.446604158176524\\
-1.81381381381381	-0.444509174699224\\
-1.80580580580581	-0.442414191221925\\
-1.7977977977978	-0.440319207744625\\
-1.78978978978979	-0.438224224267326\\
-1.78178178178178	-0.436129240790026\\
-1.77377377377377	-0.434034257312726\\
-1.76576576576577	-0.431939273835427\\
-1.75775775775776	-0.429844290358127\\
-1.74974974974975	-0.427749306880828\\
-1.74174174174174	-0.425654323403528\\
-1.73373373373373	-0.423559339926229\\
-1.72572572572573	-0.421464356448929\\
-1.71771771771772	-0.419369372971629\\
-1.70970970970971	-0.41727438949433\\
-1.7017017017017	-0.41517940601703\\
-1.69369369369369	-0.413084422539731\\
-1.68568568568569	-0.410989439062431\\
-1.67767767767768	-0.408894455585131\\
-1.66966966966967	-0.406799472107832\\
-1.66166166166166	-0.404704488630532\\
-1.65365365365365	-0.402609505153233\\
-1.64564564564565	-0.400514521675933\\
-1.63763763763764	-0.398419538198633\\
-1.62962962962963	-0.396324554721334\\
-1.62162162162162	-0.394229571244034\\
-1.61361361361361	-0.392134587766735\\
-1.60560560560561	-0.390039604289435\\
-1.5975975975976	-0.387944620812135\\
-1.58958958958959	-0.385849637334836\\
-1.58158158158158	-0.383754653857536\\
-1.57357357357357	-0.381659670380237\\
-1.56556556556557	-0.379564686902937\\
-1.55755755755756	-0.377469703425638\\
-1.54954954954955	-0.375374719948338\\
-1.54154154154154	-0.373279736471038\\
-1.53353353353353	-0.371184752993739\\
-1.52552552552553	-0.369089769516439\\
-1.51751751751752	-0.36699478603914\\
-1.50950950950951	-0.36489980256184\\
-1.5015015015015	-0.36280481908454\\
-1.49349349349349	-0.360709835607241\\
-1.48548548548549	-0.358614852129941\\
-1.47747747747748	-0.356519868652642\\
-1.46946946946947	-0.354424885175342\\
-1.46146146146146	-0.352329901698042\\
-1.45345345345345	-0.350234918220743\\
-1.44544544544545	-0.348139934743443\\
-1.43743743743744	-0.346044951266144\\
-1.42942942942943	-0.343949967788844\\
-1.42142142142142	-0.341854984311544\\
-1.41341341341341	-0.339760000834245\\
-1.40540540540541	-0.337665017356945\\
-1.3973973973974	-0.335570033879646\\
-1.38938938938939	-0.333475050402346\\
-1.38138138138138	-0.331380066925047\\
-1.37337337337337	-0.329285083447747\\
-1.36536536536537	-0.327190099970447\\
-1.35735735735736	-0.325095116493148\\
-1.34934934934935	-0.323000133015848\\
-1.34134134134134	-0.320905149538549\\
-1.33333333333333	-0.318810166061249\\
-1.32532532532533	-0.316715182583949\\
-1.31731731731732	-0.31462019910665\\
-1.30930930930931	-0.31252521562935\\
-1.3013013013013	-0.310430232152051\\
-1.29329329329329	-0.308335248674751\\
-1.28528528528529	-0.306240265197451\\
-1.27727727727728	-0.304145281720152\\
-1.26926926926927	-0.302050298242852\\
-1.26126126126126	-0.299955314765553\\
-1.25325325325325	-0.297860331288253\\
-1.24524524524525	-0.295765347810954\\
-1.23723723723724	-0.293670364333654\\
-1.22922922922923	-0.291575380856354\\
-1.22122122122122	-0.289480397379055\\
-1.21321321321321	-0.287385413901755\\
-1.20520520520521	-0.285290430424456\\
-1.1971971971972	-0.283195446947156\\
-1.18918918918919	-0.281100463469856\\
-1.18118118118118	-0.279005479992557\\
-1.17317317317317	-0.276910496515257\\
-1.16516516516517	-0.274815513037958\\
-1.15715715715716	-0.272720529560658\\
-1.14914914914915	-0.270625546083358\\
-1.14114114114114	-0.268530562606059\\
-1.13313313313313	-0.266435579128759\\
-1.12512512512513	-0.26434059565146\\
-1.11711711711712	-0.26224561217416\\
-1.10910910910911	-0.26015062869686\\
-1.1011011011011	-0.258055645219561\\
-1.09309309309309	-0.255960661742261\\
-1.08508508508509	-0.253865678264962\\
-1.07707707707708	-0.251770694787662\\
-1.06906906906907	-0.249675711310363\\
-1.06106106106106	-0.247580727833063\\
-1.05305305305305	-0.245485744355763\\
-1.04504504504505	-0.243390760878464\\
-1.03703703703704	-0.241295777401164\\
-1.02902902902903	-0.239200793923865\\
-1.02102102102102	-0.237105810446565\\
-1.01301301301301	-0.235010826969265\\
-1.00500500500501	-0.232915843491966\\
-0.996996996996997	-0.230820860014666\\
-0.988988988988989	-0.228725876537367\\
-0.980980980980981	-0.226630893060067\\
-0.972972972972973	-0.224535909582767\\
-0.964964964964965	-0.222440926105468\\
-0.956956956956957	-0.220345942628168\\
-0.948948948948949	-0.218250959150869\\
-0.940940940940941	-0.216155975673569\\
-0.932932932932933	-0.214060992196269\\
-0.924924924924925	-0.21196600871897\\
-0.916916916916917	-0.20987102524167\\
-0.908908908908909	-0.207776041764371\\
-0.900900900900901	-0.205681058287071\\
-0.892892892892893	-0.203586074809772\\
-0.884884884884885	-0.201491091332472\\
-0.876876876876877	-0.199396107855172\\
-0.868868868868869	-0.197301124377873\\
-0.860860860860861	-0.195206140900573\\
-0.852852852852853	-0.193111157423274\\
-0.844844844844845	-0.191016173945974\\
-0.836836836836837	-0.188921190468674\\
-0.828828828828829	-0.186826206991375\\
-0.820820820820821	-0.184731223514075\\
-0.812812812812813	-0.182636240036776\\
-0.804804804804805	-0.180541256559476\\
-0.796796796796797	-0.178446273082176\\
-0.788788788788789	-0.176351289604877\\
-0.780780780780781	-0.174256306127577\\
-0.772772772772773	-0.172161322650278\\
-0.764764764764765	-0.170066339172978\\
-0.756756756756757	-0.167971355695678\\
-0.748748748748749	-0.165876372218379\\
-0.740740740740741	-0.163781388741079\\
-0.732732732732733	-0.16168640526378\\
-0.724724724724725	-0.15959142178648\\
-0.716716716716717	-0.157496438309181\\
-0.708708708708709	-0.155401454831881\\
-0.700700700700701	-0.153306471354581\\
-0.692692692692693	-0.151211487877282\\
-0.684684684684685	-0.149116504399982\\
-0.676676676676677	-0.147021520922683\\
-0.668668668668669	-0.144926537445383\\
-0.660660660660661	-0.142831553968084\\
-0.652652652652653	-0.140736570490784\\
-0.644644644644645	-0.138641587013484\\
-0.636636636636636	-0.136546603536185\\
-0.628628628628629	-0.134451620058885\\
-0.62062062062062	-0.132356636581585\\
-0.612612612612613	-0.130261653104286\\
-0.604604604604605	-0.128166669626986\\
-0.596596596596596	-0.126071686149687\\
-0.588588588588589	-0.123976702672387\\
-0.58058058058058	-0.121881719195087\\
-0.572572572572573	-0.119786735717788\\
-0.564564564564565	-0.117691752240488\\
-0.556556556556556	-0.115596768763189\\
-0.548548548548549	-0.113501785285889\\
-0.54054054054054	-0.11140680180859\\
-0.532532532532533	-0.10931181833129\\
-0.524524524524525	-0.10721683485399\\
-0.516516516516516	-0.105121851376691\\
-0.508508508508509	-0.103026867899391\\
-0.5005005005005	-0.100931884422092\\
-0.492492492492492	-0.0988369009447921\\
-0.484484484484485	-0.0967419174674925\\
-0.476476476476476	-0.0946469339901928\\
-0.468468468468469	-0.0925519505128933\\
-0.46046046046046	-0.0904569670355936\\
-0.452452452452452	-0.0883619835582941\\
-0.444444444444445	-0.0862670000809946\\
-0.436436436436436	-0.0841720166036949\\
-0.428428428428429	-0.0820770331263954\\
-0.42042042042042	-0.0799820496490957\\
-0.412412412412412	-0.0778870661717962\\
-0.404404404404405	-0.0757920826944966\\
-0.396396396396396	-0.073697099217197\\
-0.388388388388389	-0.0716021157398974\\
-0.38038038038038	-0.0695071322625977\\
-0.372372372372372	-0.0674121487852982\\
-0.364364364364364	-0.0653171653079986\\
-0.356356356356356	-0.063222181830699\\
-0.348348348348348	-0.0611271983533995\\
-0.34034034034034	-0.0590322148760998\\
-0.332332332332332	-0.0569372313988003\\
-0.324324324324324	-0.0548422479215006\\
-0.316316316316316	-0.0527472644442011\\
-0.308308308308308	-0.0506522809669015\\
-0.3003003003003	-0.0485572974896019\\
-0.292292292292292	-0.0464623140123023\\
-0.284284284284284	-0.0443673305350026\\
-0.276276276276276	-0.0422723470577031\\
-0.268268268268268	-0.0401773635804036\\
-0.26026026026026	-0.0380823801031039\\
-0.252252252252252	-0.0359873966258044\\
-0.244244244244244	-0.0338924131485047\\
-0.236236236236236	-0.0317974296712052\\
-0.228228228228228	-0.0297024461939056\\
-0.22022022022022	-0.027607462716606\\
-0.212212212212212	-0.0255124792393064\\
-0.204204204204204	-0.0234174957620068\\
-0.196196196196196	-0.0213225122847072\\
-0.188188188188188	-0.0192275288074077\\
-0.18018018018018	-0.017132545330108\\
-0.172172172172172	-0.0150375618528085\\
-0.164164164164164	-0.0129425783755088\\
-0.156156156156156	-0.0108475948982093\\
-0.148148148148148	-0.00875261142090973\\
-0.14014014014014	-0.00665762794361007\\
-0.132132132132132	-0.00456264446631053\\
-0.124124124124124	-0.00246766098901087\\
-0.116116116116116	-0.000372677511711324\\
-0.108108108108108	0.00172230596558834\\
-0.1001001001001	0.00381728944288788\\
-0.0920920920920922	0.00591227292018742\\
-0.084084084084084	0.00800725639748708\\
-0.0760760760760761	0.0101022398747866\\
-0.0680680680680679	0.0121972233520863\\
-0.06006006006006	0.0142922068293858\\
-0.0520520520520522	0.0163871903066854\\
-0.0440440440440439	0.018482173783985\\
-0.0360360360360361	0.0205771572612846\\
-0.0280280280280278	0.0226721407385842\\
-0.02002002002002	0.0247671242158838\\
-0.0120120120120122	0.0268621076931833\\
-0.00400400400400391	0.028957091170483\\
0.00400400400400436	0.0310520746477826\\
0.0120120120120122	0.0331470581250822\\
0.02002002002002	0.0352420416023817\\
0.0280280280280278	0.0373370250796813\\
0.0360360360360357	0.0394320085569808\\
0.0440440440440444	0.0415269920342806\\
0.0520520520520522	0.0436219755115801\\
0.06006006006006	0.0457169589888797\\
0.0680680680680679	0.0478119424661792\\
0.0760760760760757	0.0499069259434787\\
0.0840840840840844	0.0520019094207785\\
0.0920920920920922	0.0540968928980781\\
0.1001001001001	0.0561918763753776\\
0.108108108108108	0.0582868598526771\\
0.116116116116116	0.0603818433299767\\
0.124124124124124	0.0624768268072765\\
0.132132132132132	0.064571810284576\\
0.14014014014014	0.0666667937618756\\
0.148148148148148	0.0687617772391751\\
0.156156156156156	0.0708567607164746\\
0.164164164164164	0.0729517441937744\\
0.172172172172172	0.075046727671074\\
0.18018018018018	0.0771417111483735\\
0.188188188188188	0.079236694625673\\
0.196196196196196	0.0813316781029726\\
0.204204204204204	0.0834266615802724\\
0.212212212212212	0.0855216450575719\\
0.22022022022022	0.0876166285348715\\
0.228228228228228	0.089711612012171\\
0.236236236236236	0.0918065954894705\\
0.244244244244245	0.0939015789667703\\
0.252252252252252	0.0959965624440699\\
0.26026026026026	0.0980915459213694\\
0.268268268268268	0.100186529398669\\
0.276276276276277	0.102281512875969\\
0.284284284284285	0.104376496353268\\
0.292292292292292	0.106471479830568\\
0.3003003003003	0.108566463307867\\
0.308308308308308	0.110661446785167\\
0.316316316316317	0.112756430262467\\
0.324324324324325	0.114851413739766\\
0.332332332332332	0.116946397217066\\
0.34034034034034	0.119041380694365\\
0.348348348348348	0.121136364171665\\
0.356356356356357	0.123231347648965\\
0.364364364364365	0.125326331126264\\
0.372372372372372	0.127421314603564\\
0.38038038038038	0.129516298080863\\
0.388388388388388	0.131611281558163\\
0.396396396396397	0.133706265035463\\
0.404404404404405	0.135801248512762\\
0.412412412412412	0.137896231990062\\
0.42042042042042	0.139991215467361\\
0.428428428428428	0.142086198944661\\
0.436436436436437	0.14418118242196\\
0.444444444444445	0.14627616589926\\
0.452452452452452	0.14837114937656\\
0.46046046046046	0.150466132853859\\
0.468468468468468	0.152561116331159\\
0.476476476476477	0.154656099808458\\
0.484484484484485	0.156751083285758\\
0.492492492492492	0.158846066763058\\
0.5005005005005	0.160941050240357\\
0.508508508508508	0.163036033717657\\
0.516516516516517	0.165131017194956\\
0.524524524524525	0.167226000672256\\
0.532532532532533	0.169320984149555\\
0.54054054054054	0.171415967626855\\
0.548548548548548	0.173510951104155\\
0.556556556556557	0.175605934581454\\
0.564564564564565	0.177700918058754\\
0.572572572572573	0.179795901536053\\
0.58058058058058	0.181890885013353\\
0.588588588588588	0.183985868490653\\
0.596596596596597	0.186080851967952\\
0.604604604604605	0.188175835445252\\
0.612612612612613	0.190270818922551\\
0.62062062062062	0.192365802399851\\
0.628628628628628	0.19446078587715\\
0.636636636636637	0.19655576935445\\
0.644644644644645	0.19865075283175\\
0.652652652652653	0.200745736309049\\
0.66066066066066	0.202840719786349\\
0.668668668668668	0.204935703263648\\
0.676676676676677	0.207030686740948\\
0.684684684684685	0.209125670218248\\
0.692692692692693	0.211220653695547\\
0.7007007007007	0.213315637172847\\
0.708708708708708	0.215410620650146\\
0.716716716716717	0.217505604127446\\
0.724724724724725	0.219600587604746\\
0.732732732732733	0.221695571082045\\
0.74074074074074	0.223790554559345\\
0.748748748748748	0.225885538036644\\
0.756756756756757	0.227980521513944\\
0.764764764764765	0.230075504991244\\
0.772772772772773	0.232170488468543\\
0.780780780780781	0.234265471945843\\
0.788788788788789	0.236360455423142\\
0.796796796796797	0.238455438900442\\
0.804804804804805	0.240550422377742\\
0.812812812812813	0.242645405855041\\
0.820820820820821	0.244740389332341\\
0.828828828828829	0.24683537280964\\
0.836836836836837	0.24893035628694\\
0.844844844844845	0.25102533976424\\
0.852852852852853	0.253120323241539\\
0.860860860860861	0.255215306718839\\
0.868868868868869	0.257310290196138\\
0.876876876876877	0.259405273673438\\
0.884884884884885	0.261500257150737\\
0.892892892892893	0.263595240628037\\
0.900900900900901	0.265690224105337\\
0.908908908908909	0.267785207582636\\
0.916916916916917	0.269880191059936\\
0.924924924924925	0.271975174537235\\
0.932932932932933	0.274070158014535\\
0.940940940940941	0.276165141491834\\
0.948948948948949	0.278260124969134\\
0.956956956956957	0.280355108446434\\
0.964964964964965	0.282450091923733\\
0.972972972972973	0.284545075401033\\
0.980980980980981	0.286640058878332\\
0.988988988988989	0.288735042355632\\
0.996996996996997	0.290830025832932\\
1.00500500500501	0.292925009310231\\
1.01301301301301	0.295019992787531\\
1.02102102102102	0.29711497626483\\
1.02902902902903	0.29920995974213\\
1.03703703703704	0.30130494321943\\
1.04504504504505	0.303399926696729\\
1.05305305305305	0.305494910174029\\
1.06106106106106	0.307589893651328\\
1.06906906906907	0.309684877128628\\
1.07707707707708	0.311779860605928\\
1.08508508508509	0.313874844083227\\
1.09309309309309	0.315969827560527\\
1.1011011011011	0.318064811037826\\
1.10910910910911	0.320159794515126\\
1.11711711711712	0.322254777992426\\
1.12512512512513	0.324349761469725\\
1.13313313313313	0.326444744947025\\
1.14114114114114	0.328539728424324\\
1.14914914914915	0.330634711901624\\
1.15715715715716	0.332729695378923\\
1.16516516516517	0.334824678856223\\
1.17317317317317	0.336919662333523\\
1.18118118118118	0.339014645810822\\
1.18918918918919	0.341109629288122\\
1.1971971971972	0.343204612765421\\
1.20520520520521	0.345299596242721\\
1.21321321321321	0.347394579720021\\
1.22122122122122	0.34948956319732\\
1.22922922922923	0.35158454667462\\
1.23723723723724	0.353679530151919\\
1.24524524524525	0.355774513629219\\
1.25325325325325	0.357869497106518\\
1.26126126126126	0.359964480583818\\
1.26926926926927	0.362059464061118\\
1.27727727727728	0.364154447538417\\
1.28528528528529	0.366249431015717\\
1.29329329329329	0.368344414493016\\
1.3013013013013	0.370439397970316\\
1.30930930930931	0.372534381447616\\
1.31731731731732	0.374629364924915\\
1.32532532532533	0.376724348402215\\
1.33333333333333	0.378819331879514\\
1.34134134134134	0.380914315356814\\
1.34934934934935	0.383009298834114\\
1.35735735735736	0.385104282311413\\
1.36536536536537	0.387199265788713\\
1.37337337337337	0.389294249266012\\
1.38138138138138	0.391389232743312\\
1.38938938938939	0.393484216220612\\
1.3973973973974	0.395579199697911\\
1.40540540540541	0.397674183175211\\
1.41341341341341	0.39976916665251\\
1.42142142142142	0.40186415012981\\
1.42942942942943	0.40395913360711\\
1.43743743743744	0.406054117084409\\
1.44544544544545	0.408149100561709\\
1.45345345345345	0.410244084039008\\
1.46146146146146	0.412339067516308\\
1.46946946946947	0.414434050993608\\
1.47747747747748	0.416529034470907\\
1.48548548548549	0.418624017948207\\
1.49349349349349	0.420719001425506\\
1.5015015015015	0.422813984902806\\
1.50950950950951	0.424908968380106\\
1.51751751751752	0.427003951857405\\
1.52552552552553	0.429098935334705\\
1.53353353353353	0.431193918812004\\
1.54154154154154	0.433288902289304\\
1.54954954954955	0.435383885766603\\
1.55755755755756	0.437478869243903\\
1.56556556556557	0.439573852721203\\
1.57357357357357	0.441668836198502\\
1.58158158158158	0.443763819675802\\
1.58958958958959	0.445858803153101\\
1.5975975975976	0.447953786630401\\
1.60560560560561	0.450048770107701\\
1.61361361361361	0.452143753585\\
1.62162162162162	0.4542387370623\\
1.62962962962963	0.456333720539599\\
1.63763763763764	0.458428704016899\\
1.64564564564565	0.460523687494198\\
1.65365365365365	0.462618670971498\\
1.66166166166166	0.464713654448798\\
1.66966966966967	0.466808637926097\\
1.67767767767768	0.468903621403397\\
1.68568568568569	0.470998604880696\\
1.69369369369369	0.473093588357996\\
1.7017017017017	0.475188571835296\\
1.70970970970971	0.477283555312595\\
1.71771771771772	0.479378538789895\\
1.72572572572573	0.481473522267194\\
1.73373373373373	0.483568505744494\\
1.74174174174174	0.485663489221794\\
1.74974974974975	0.487758472699093\\
1.75775775775776	0.489853456176393\\
1.76576576576577	0.491948439653692\\
1.77377377377377	0.494043423130992\\
1.78178178178178	0.496138406608292\\
1.78978978978979	0.498233390085591\\
1.7977977977978	0.500328373562891\\
1.80580580580581	0.50242335704019\\
1.81381381381381	0.50451834051749\\
1.82182182182182	0.50661332399479\\
1.82982982982983	0.508708307472089\\
1.83783783783784	0.510803290949389\\
1.84584584584585	0.512898274426688\\
1.85385385385385	0.514993257903988\\
1.86186186186186	0.517088241381288\\
1.86986986986987	0.519183224858587\\
1.87787787787788	0.521278208335887\\
1.88588588588589	0.523373191813186\\
1.89389389389389	0.525468175290486\\
1.9019019019019	0.527563158767785\\
1.90990990990991	0.529658142245085\\
1.91791791791792	0.531753125722385\\
1.92592592592593	0.533848109199684\\
1.93393393393393	0.535943092676984\\
1.94194194194194	0.538038076154283\\
1.94994994994995	0.540133059631583\\
1.95795795795796	0.542228043108883\\
1.96596596596597	0.544323026586182\\
1.97397397397397	0.546418010063482\\
1.98198198198198	0.548512993540781\\
1.98998998998999	0.550607977018081\\
1.997997997998	0.55270296049538\\
2.00600600600601	0.55479794397268\\
2.01401401401401	0.55689292744998\\
2.02202202202202	0.558987910927279\\
2.03003003003003	0.561082894404579\\
2.03803803803804	0.563177877881878\\
2.04604604604605	0.565272861359178\\
2.05405405405405	0.567367844836478\\
2.06206206206206	0.569462828313777\\
2.07007007007007	0.571557811791077\\
2.07807807807808	0.573652795268376\\
2.08608608608609	0.575747778745676\\
2.09409409409409	0.577842762222976\\
2.1021021021021	0.579937745700275\\
2.11011011011011	0.582032729177575\\
2.11811811811812	0.584127712654874\\
2.12612612612613	0.586222696132174\\
2.13413413413413	0.588317679609474\\
2.14214214214214	0.590412663086773\\
2.15015015015015	0.592507646564073\\
2.15815815815816	0.594602630041372\\
2.16616616616617	0.596697613518672\\
2.17417417417417	0.598792596995972\\
2.18218218218218	0.600887580473271\\
2.19019019019019	0.602982563950571\\
2.1981981981982	0.60507754742787\\
2.20620620620621	0.60717253090517\\
2.21421421421421	0.60926751438247\\
2.22222222222222	0.611362497859769\\
2.23023023023023	0.613457481337069\\
2.23823823823824	0.615552464814368\\
2.24624624624625	0.617647448291668\\
2.25425425425425	0.619742431768967\\
2.26226226226226	0.621837415246267\\
2.27027027027027	0.623932398723567\\
2.27827827827828	0.626027382200866\\
2.28628628628629	0.628122365678166\\
2.29429429429429	0.630217349155465\\
2.3023023023023	0.632312332632765\\
2.31031031031031	0.634407316110065\\
2.31831831831832	0.636502299587364\\
2.32632632632633	0.638597283064664\\
2.33433433433433	0.640692266541963\\
2.34234234234234	0.642787250019263\\
2.35035035035035	0.644882233496562\\
2.35835835835836	0.646977216973862\\
2.36636636636637	0.649072200451162\\
2.37437437437437	0.651167183928461\\
2.38238238238238	0.653262167405761\\
2.39039039039039	0.65535715088306\\
2.3983983983984	0.65745213436036\\
2.40640640640641	0.65954711783766\\
2.41441441441441	0.661642101314959\\
2.42242242242242	0.663737084792259\\
2.43043043043043	0.665832068269558\\
2.43843843843844	0.667927051746858\\
2.44644644644645	0.670022035224158\\
2.45445445445445	0.672117018701457\\
2.46246246246246	0.674212002178757\\
2.47047047047047	0.676306985656056\\
2.47847847847848	0.678401969133356\\
2.48648648648649	0.680496952610656\\
2.49449449449449	0.682591936087955\\
2.5025025025025	0.684686919565255\\
2.51051051051051	0.686781903042554\\
2.51851851851852	0.688876886519854\\
2.52652652652653	0.690971869997154\\
2.53453453453453	0.693066853474453\\
2.54254254254254	0.695161836951753\\
2.55055055055055	0.697256820429052\\
2.55855855855856	0.699351803906352\\
2.56656656656657	0.701446787383652\\
2.57457457457457	0.703541770860951\\
2.58258258258258	0.705636754338251\\
2.59059059059059	0.70773173781555\\
2.5985985985986	0.70982672129285\\
2.60660660660661	0.711921704770149\\
2.61461461461461	0.714016688247449\\
2.62262262262262	0.716111671724749\\
2.63063063063063	0.718206655202048\\
2.63863863863864	0.720301638679348\\
2.64664664664665	0.722396622156647\\
2.65465465465465	0.724491605633947\\
2.66266266266266	0.726586589111246\\
2.67067067067067	0.728681572588546\\
2.67867867867868	0.730776556065846\\
2.68668668668669	0.732871539543145\\
2.69469469469469	0.734966523020445\\
2.7027027027027	0.737061506497744\\
2.71071071071071	0.739156489975044\\
2.71871871871872	0.741251473452343\\
2.72672672672673	0.743346456929643\\
2.73473473473473	0.745441440406943\\
2.74274274274274	0.747536423884242\\
2.75075075075075	0.749631407361542\\
2.75875875875876	0.751726390838842\\
2.76676676676677	0.753821374316141\\
2.77477477477477	0.755916357793441\\
2.78278278278278	0.75801134127074\\
2.79079079079079	0.76010632474804\\
2.7987987987988	0.76220130822534\\
2.80680680680681	0.764296291702639\\
2.81481481481481	0.766391275179939\\
2.82282282282282	0.768486258657238\\
2.83083083083083	0.770581242134538\\
2.83883883883884	0.772676225611838\\
2.84684684684685	0.774771209089137\\
2.85485485485485	0.776866192566437\\
2.86286286286286	0.778961176043736\\
2.87087087087087	0.781056159521036\\
2.87887887887888	0.783151142998336\\
2.88688688688689	0.785246126475635\\
2.89489489489489	0.787341109952935\\
2.9029029029029	0.789436093430234\\
2.91091091091091	0.791531076907534\\
2.91891891891892	0.793626060384833\\
2.92692692692693	0.795721043862133\\
2.93493493493493	0.797816027339433\\
2.94294294294294	0.799911010816732\\
2.95095095095095	0.802005994294032\\
2.95895895895896	0.804100977771331\\
2.96696696696697	0.806195961248631\\
2.97497497497497	0.80829094472593\\
2.98298298298298	0.81038592820323\\
2.99099099099099	0.81248091168053\\
2.998998998999	0.814575895157829\\
3.00700700700701	0.816670878635129\\
3.01501501501502	0.818765862112428\\
3.02302302302302	0.820860845589728\\
3.03103103103103	0.822955829067027\\
3.03903903903904	0.825050812544327\\
3.04704704704705	0.827145796021627\\
3.05505505505506	0.829240779498926\\
3.06306306306306	0.831335762976226\\
3.07107107107107	0.833430746453525\\
3.07907907907908	0.835525729930825\\
3.08708708708709	0.837620713408125\\
3.0950950950951	0.839715696885424\\
3.1031031031031	0.841810680362724\\
3.11111111111111	0.843905663840023\\
3.11911911911912	0.846000647317323\\
3.12712712712713	0.848095630794623\\
3.13513513513514	0.850190614271922\\
3.14314314314314	0.852285597749222\\
3.15115115115115	0.854380581226521\\
3.15915915915916	0.856475564703821\\
3.16716716716717	0.858570548181121\\
3.17517517517518	0.86066553165842\\
3.18318318318318	0.86276051513572\\
3.19119119119119	0.864855498613019\\
3.1991991991992	0.866950482090319\\
3.20720720720721	0.869045465567619\\
3.21521521521522	0.871140449044918\\
3.22322322322322	0.873235432522218\\
3.23123123123123	0.875330415999517\\
3.23923923923924	0.877425399476817\\
3.24724724724725	0.879520382954117\\
3.25525525525526	0.881615366431416\\
3.26326326326326	0.883710349908716\\
3.27127127127127	0.885805333386015\\
3.27927927927928	0.887900316863315\\
3.28728728728729	0.889995300340614\\
3.2952952952953	0.892090283817914\\
3.3033033033033	0.894185267295214\\
3.31131131131131	0.896280250772513\\
3.31931931931932	0.898375234249813\\
3.32732732732733	0.900470217727112\\
3.33533533533534	0.902565201204412\\
3.34334334334334	0.904660184681711\\
3.35135135135135	0.906755168159011\\
3.35935935935936	0.908850151636311\\
3.36736736736737	0.91094513511361\\
3.37537537537538	0.91304011859091\\
3.38338338338338	0.915135102068209\\
3.39139139139139	0.917230085545509\\
3.3993993993994	0.919325069022809\\
3.40740740740741	0.921420052500108\\
3.41541541541542	0.923515035977408\\
3.42342342342342	0.925610019454707\\
3.43143143143143	0.927705002932007\\
3.43943943943944	0.929799986409307\\
3.44744744744745	0.931894969886606\\
3.45545545545546	0.933989953363906\\
3.46346346346346	0.936084936841205\\
3.47147147147147	0.938179920318505\\
3.47947947947948	0.940274903795805\\
3.48748748748749	0.942369887273104\\
3.4954954954955	0.944464870750404\\
3.5035035035035	0.946559854227703\\
3.51151151151151	0.948654837705003\\
3.51951951951952	0.950749821182303\\
3.52752752752753	0.952844804659602\\
3.53553553553554	0.954939788136902\\
3.54354354354354	0.957034771614201\\
3.55155155155155	0.959129755091501\\
3.55955955955956	0.961224738568801\\
3.56756756756757	0.9633197220461\\
3.57557557557558	0.9654147055234\\
3.58358358358358	0.967509689000699\\
3.59159159159159	0.969604672477999\\
3.5995995995996	0.971699655955299\\
3.60760760760761	0.973794639432598\\
3.61561561561562	0.975889622909898\\
3.62362362362362	0.977984606387197\\
3.63163163163163	0.980079589864497\\
3.63963963963964	0.982174573341796\\
3.64764764764765	0.984269556819096\\
3.65565565565566	0.986364540296396\\
3.66366366366366	0.988459523773695\\
3.67167167167167	0.990554507250995\\
3.67967967967968	0.992649490728294\\
3.68768768768769	0.994744474205594\\
3.6956956956957	0.996839457682894\\
3.7037037037037	0.998934441160193\\
3.71171171171171	1.00102942463749\\
3.71971971971972	1.00312440811479\\
3.72772772772773	1.00521939159209\\
3.73573573573574	1.00731437506939\\
3.74374374374374	1.00940935854669\\
3.75175175175175	1.01150434202399\\
3.75975975975976	1.01359932550129\\
3.76776776776777	1.01569430897859\\
3.77577577577578	1.01778929245589\\
3.78378378378378	1.01988427593319\\
3.79179179179179	1.02197925941049\\
3.7997997997998	1.02407424288779\\
3.80780780780781	1.02616922636509\\
3.81581581581582	1.02826420984239\\
3.82382382382382	1.03035919331969\\
3.83183183183183	1.03245417679699\\
3.83983983983984	1.03454916027429\\
3.84784784784785	1.03664414375159\\
3.85585585585586	1.03873912722889\\
3.86386386386386	1.04083411070618\\
3.87187187187187	1.04292909418348\\
3.87987987987988	1.04502407766078\\
3.88788788788789	1.04711906113808\\
3.8958958958959	1.04921404461538\\
3.9039039039039	1.05130902809268\\
3.91191191191191	1.05340401156998\\
3.91991991991992	1.05549899504728\\
3.92792792792793	1.05759397852458\\
3.93593593593594	1.05968896200188\\
3.94394394394394	1.06178394547918\\
3.95195195195195	1.06387892895648\\
3.95995995995996	1.06597391243378\\
3.96796796796797	1.06806889591108\\
3.97597597597598	1.07016387938838\\
3.98398398398398	1.07225886286568\\
3.99199199199199	1.07435384634298\\
4	1.07644882982028\\
};
\addlegendentry{$\mu(x)$};

\addplot [color=black,only marks,mark=*,mark options={solid}]
  table[row sep=crcr]{%
-2.26	1.03\\
-1.31	0.7\\
-0.43	-0.68\\
0.32	-1.36\\
0.34	-1.74\\
0.54	-1.01\\
0.86	0.24\\
1.83	1.55\\
2.77	1.68\\
3.58	1.53\\
};
\addlegendentry{observations};

\end{axis}
\end{tikzpicture}%
  \caption{Posterior for $k = 1$.}
  \label{order_1_expansion}
\end{figure}

\begin{figure}
  \centering
  % This file was created by matlab2tikz.
% Minimal pgfplots version: 1.3
%
\tikzsetnextfilename{order_2_expansion}
\definecolor{mycolor1}{rgb}{0.65098,0.80784,0.89020}%
\definecolor{mycolor2}{rgb}{0.12157,0.47059,0.70588}%
%
\begin{tikzpicture}

\begin{axis}[%
width=0.95092\figurewidth,
height=\figureheight,
at={(0\figurewidth,0\figureheight)},
scale only axis,
xmin=-4,
xmax=4,
xlabel={$x$},
ymin=-2,
ymax=7,
axis x line*=bottom,
axis y line*=left,
legend style={legend cell align=left,align=left,draw=white!15!black},
legend style={legend columns=-1, draw=none}, reverse legend
]

\addplot[area legend,solid,fill=mycolor1,opacity=3.000000e-01,draw=none]
table[row sep=crcr] {%
x	y\\
-4	1.59190357764963\\
-3.99199199199199	1.58136325463587\\
-3.98398398398398	1.57083948301664\\
-3.97597597597598	1.56033224635293\\
-3.96796796796797	1.5498415281523\\
-3.95995995995996	1.5393673118693\\
-3.95195195195195	1.52890958090579\\
-3.94394394394394	1.51846831861148\\
-3.93593593593594	1.50804350828426\\
-3.92792792792793	1.49763513317075\\
-3.91991991991992	1.48724317646671\\
-3.91191191191191	1.47686762131751\\
-3.9039039039039	1.46650845081869\\
-3.8958958958959	1.4561656480164\\
-3.88788788788789	1.445839195908\\
-3.87987987987988	1.43552907744253\\
-3.87187187187187	1.42523527552131\\
-3.86386386386386	1.41495777299854\\
-3.85585585585586	1.40469655268182\\
-3.84784784784785	1.39445159733281\\
-3.83983983983984	1.38422288966787\\
-3.83183183183183	1.37401041235863\\
-3.82382382382382	1.36381414803274\\
-3.81581581581582	1.35363407927445\\
-3.80780780780781	1.34347018862539\\
-3.7997997997998	1.33332245858523\\
-3.79179179179179	1.32319087161245\\
-3.78378378378378	1.31307541012504\\
-3.77577577577578	1.3029760565013\\
-3.76776776776777	1.29289279308064\\
-3.75975975975976	1.28282560216435\\
-3.75175175175175	1.27277446601644\\
-3.74374374374374	1.26273936686451\\
-3.73573573573574	1.2527202869006\\
-3.72772772772773	1.24271720828204\\
-3.71971971971972	1.23273011313244\\
-3.71171171171171	1.22275898354252\\
-3.7037037037037	1.21280380157116\\
-3.6956956956957	1.20286454924628\\
-3.68768768768769	1.1929412085659\\
-3.67967967967968	1.18303376149912\\
-3.67167167167167	1.17314218998716\\
-3.66366366366366	1.16326647594445\\
-3.65565565565566	1.15340660125966\\
-3.64764764764765	1.14356254779687\\
-3.63963963963964	1.13373429739664\\
-3.63163163163163	1.12392183187721\\
-3.62362362362362	1.11412513303565\\
-3.61561561561562	1.10434418264906\\
-3.60760760760761	1.09457896247584\\
-3.5995995995996	1.08482945425686\\
-3.59159159159159	1.07509563971684\\
-3.58358358358358	1.06537750056556\\
-3.57557557557558	1.05567501849924\\
-3.56756756756757	1.04598817520187\\
-3.55955955955956	1.03631695234663\\
-3.55155155155155	1.02666133159723\\
-3.54354354354354	1.01702129460941\\
-3.53553553553554	1.00739682303236\\
-3.52752752752753	0.997787898510216\\
-3.51951951951952	0.988194502683598\\
-3.51151151151151	0.978616617191127\\
-3.5035035035035	0.969054223671012\\
-3.4954954954955	0.959507303762646\\
-3.48748748748749	0.949975839108238\\
-3.47947947947948	0.94045981135447\\
-3.47147147147147	0.930959202154191\\
-3.46346346346346	0.921473993168127\\
-3.45545545545546	0.912004166066633\\
-3.44744744744745	0.902549702531472\\
-3.43943943943944	0.893110584257622\\
-3.43143143143143	0.883686792955112\\
-3.42342342342342	0.874278310350901\\
-3.41541541541542	0.86488511819077\\
-3.40740740740741	0.855507198241256\\
-3.3993993993994	0.846144532291623\\
-3.39139139139139	0.83679710215585\\
-3.38338338338338	0.827464889674664\\
-3.37537537537538	0.818147876717599\\
-3.36736736736737	0.808846045185093\\
-3.35935935935936	0.799559377010604\\
-3.35135135135135	0.790287854162784\\
-3.34334334334334	0.781031458647658\\
-3.33533533533534	0.771790172510858\\
-3.32732732732733	0.76256397783988\\
-3.31931931931932	0.753352856766375\\
-3.31131131131131	0.744156791468483\\
-3.3033033033033	0.734975764173187\\
-3.2952952952953	0.725809757158714\\
-3.28728728728729	0.716658752756966\\
-3.27927927927928	0.707522733355983\\
-3.27127127127127	0.698401681402441\\
-3.26326326326326	0.689295579404199\\
-3.25525525525526	0.680204409932855\\
-3.24724724724725	0.671128155626362\\
-3.23923923923924	0.662066799191665\\
-3.23123123123123	0.65302032340738\\
-3.22322322322322	0.643988711126507\\
-3.21521521521522	0.634971945279178\\
-3.20720720720721	0.625970008875444\\
-3.1991991991992	0.616982885008089\\
-3.19119119119119	0.608010556855494\\
-3.18318318318318	0.599053007684529\\
-3.17517517517518	0.590110220853477\\
-3.16716716716717	0.581182179815003\\
-3.15915915915916	0.572268868119156\\
-3.15115115115115	0.563370269416406\\
-3.14314314314314	0.554486367460716\\
-3.13513513513514	0.545617146112658\\
-3.12712712712713	0.536762589342552\\
-3.11911911911912	0.527922681233655\\
-3.11111111111111	0.519097405985379\\
-3.1031031031031	0.510286747916544\\
-3.0950950950951	0.501490691468674\\
-3.08708708708709	0.492709221209319\\
-3.07907907907908	0.48394232183542\\
-3.07107107107107	0.475189978176714\\
-3.06306306306306	0.466452175199161\\
-3.05505505505506	0.457728898008419\\
-3.04704704704705	0.449020131853348\\
-3.03903903903904	0.440325862129553\\
-3.03103103103103	0.431646074382961\\
-3.02302302302302	0.422980754313428\\
-3.01501501501502	0.414329887778389\\
-3.00700700700701	0.405693460796539\\
-2.998998998999	0.39707145955154\\
-2.99099099099099	0.388463870395781\\
-2.98298298298298	0.37987067985415\\
-2.97497497497497	0.371291874627856\\
-2.96696696696697	0.362727441598272\\
-2.95895895895896	0.354177367830822\\
-2.95095095095095	0.345641640578888\\
-2.94294294294294	0.337120247287763\\
-2.93493493493493	0.328613175598623\\
-2.92692692692693	0.320120413352531\\
-2.91891891891892	0.311641948594489\\
-2.91091091091091	0.303177769577493\\
-2.9029029029029	0.294727864766643\\
-2.89489489489489	0.286292222843262\\
-2.88688688688689	0.277870832709061\\
-2.87887887887888	0.269463683490319\\
-2.87087087087087	0.2610707645421\\
-2.86286286286286	0.252692065452492\\
-2.85485485485485	0.244327576046871\\
-2.84684684684685	0.235977286392197\\
-2.83883883883884	0.227641186801331\\
-2.83083083083083	0.219319267837378\\
-2.82282282282282	0.211011520318054\\
-2.81481481481481	0.202717935320075\\
-2.80680680680681	0.194438504183566\\
-2.7987987987988	0.186173218516505\\
-2.79079079079079	0.177922070199166\\
-2.78278278278278	0.169685051388603\\
-2.77477477477477	0.161462154523141\\
-2.76676676676677	0.15325337232689\\
-2.75875875875876	0.145058697814273\\
-2.75075075075075	0.136878124294579\\
-2.74274274274274	0.128711645376524\\
-2.73473473473473	0.120559254972825\\
-2.72672672672673	0.112420947304799\\
-2.71871871871872	0.104296716906966\\
-2.71071071071071	0.0961865586316613\\
-2.7027027027027	0.088090467653674\\
-2.69469469469469	0.0800084394748746\\
-2.68668668668669	0.071940469928867\\
-2.67867867867868	0.0638865551856427\\
-2.67067067067067	0.0558466917562417\\
-2.66266266266266	0.0478208764974197\\
-2.65465465465465	0.0398091066163166\\
-2.64664664664665	0.0318113796751336\\
-2.63863863863864	0.0238276935958059\\
-2.63063063063063	0.0158580466646836\\
-2.62262262262262	0.00790243753720343\\
-2.61461461461461	-3.91347574348888e-05\\
-2.60660660660661	-0.0079666708115993\\
-2.5985985985986	-0.0158801708345577\\
-2.59059059059059	-0.0237796346478178\\
-2.58258258258258	-0.0316650616804728\\
-2.57457457457457	-0.039536450964559\\
-2.56656656656657	-0.04739380113042\\
-2.55855855855856	-0.055237110402081\\
-2.55055055055055	-0.0630663765926422\\
-2.54254254254254	-0.0708815970996777\\
-2.53453453453453	-0.0786827689006599\\
-2.52652652652653	-0.0864698885483943\\
-2.51851851851852	-0.0942429521664725\\
-2.51051051051051	-0.102001955444752\\
-2.5025025025025	-0.109746893634851\\
-2.49449449449449	-0.11747776154567\\
-2.48648648648649	-0.12519455353894\\
-2.47847847847848	-0.132897263524792\\
-2.47047047047047	-0.140585884957357\\
-2.46246246246246	-0.148260410830401\\
-2.45445445445445	-0.155920833672983\\
-2.44644644644645	-0.163567145545152\\
-2.43843843843844	-0.171199338033671\\
-2.43043043043043	-0.178817402247795\\
-2.42242242242242	-0.186421328815058\\
-2.41441441441441	-0.194011107877134\\
-2.40640640640641	-0.201586729085709\\
-2.3983983983984	-0.209148181598414\\
-2.39039039039039	-0.216695454074798\\
-2.38238238238238	-0.224228534672346\\
-2.37437437437437	-0.231747411042549\\
-2.36636636636637	-0.239252070327015\\
-2.35835835835836	-0.246742499153645\\
-2.35035035035035	-0.254218683632851\\
-2.34234234234234	-0.261680609353834\\
-2.33433433433433	-0.269128261380918\\
-2.32632632632633	-0.276561624249943\\
-2.31831831831832	-0.283980681964715\\
-2.31031031031031	-0.291385417993525\\
-2.3023023023023	-0.298775815265722\\
-2.29429429429429	-0.306151856168358\\
-2.28628628628629	-0.313513522542902\\
-2.27827827827828	-0.320860795682012\\
-2.27027027027027	-0.32819365632639\\
-2.26226226226226	-0.335512084661699\\
-2.25425425425425	-0.342816060315561\\
-2.24624624624625	-0.350105562354629\\
-2.23823823823824	-0.357380569281727\\
-2.23023023023023	-0.364641059033082\\
-2.22222222222222	-0.371887008975628\\
-2.21421421421421	-0.379118395904394\\
-2.20620620620621	-0.386335196039975\\
-2.1981981981982	-0.393537385026087\\
-2.19019019019019	-0.40072493792721\\
-2.18218218218218	-0.407897829226319\\
-2.17417417417417	-0.415056032822703\\
-2.16616616616617	-0.422199522029875\\
-2.15815815815816	-0.429328269573576\\
-2.15015015015015	-0.43644224758987\\
-2.14214214214214	-0.443541427623339\\
-2.13413413413413	-0.450625780625368\\
-2.12612612612613	-0.457695276952533\\
-2.11811811811812	-0.464749886365085\\
-2.11011011011011	-0.471789578025537\\
-2.1021021021021	-0.478814320497353\\
-2.09409409409409	-0.485824081743733\\
-2.08608608608609	-0.492818829126513\\
-2.07807807807808	-0.499798529405163\\
-2.07007007007007	-0.50676314873589\\
-2.06206206206206	-0.513712652670856\\
-2.05405405405405	-0.520647006157496\\
-2.04604604604605	-0.527566173537952\\
-2.03803803803804	-0.534470118548613\\
-2.03003003003003	-0.541358804319767\\
-2.02202202202202	-0.548232193375367\\
-2.01401401401401	-0.555090247632912\\
-2.00600600600601	-0.561932928403433\\
-1.997997997998	-0.568760196391604\\
-1.98998998998999	-0.57557201169596\\
-1.98198198198198	-0.582368333809237\\
-1.97397397397397	-0.589149121618827\\
-1.96596596596597	-0.595914333407345\\
-1.95795795795796	-0.602663926853322\\
-1.94994994994995	-0.609397859032007\\
-1.94194194194194	-0.616116086416297\\
-1.93393393393393	-0.622818564877775\\
-1.92592592592593	-0.62950524968788\\
-1.91791791791792	-0.636176095519185\\
-1.90990990990991	-0.642831056446797\\
-1.9019019019019	-0.649470085949885\\
-1.89389389389389	-0.656093136913318\\
-1.88588588588589	-0.662700161629427\\
-1.87787787787788	-0.669291111799886\\
-1.86986986986987	-0.675865938537716\\
-1.86186186186186	-0.682424592369406\\
-1.85385385385385	-0.688967023237152\\
-1.84584584584585	-0.69549318050122\\
-1.83783783783784	-0.702003012942426\\
-1.82982982982983	-0.708496468764734\\
-1.82182182182182	-0.714973495597976\\
-1.81381381381381	-0.721434040500682\\
-1.80580580580581	-0.727878049963042\\
-1.7977977977978	-0.734305469909971\\
-1.78978978978979	-0.740716245704298\\
-1.78178178178178	-0.747110322150068\\
-1.77377377377377	-0.753487643495965\\
-1.76576576576577	-0.759848153438842\\
-1.75775775775776	-0.76619179512737\\
-1.74974974974975	-0.772518511165797\\
-1.74174174174174	-0.778828243617823\\
-1.73373373373373	-0.785120934010582\\
-1.72572572572573	-0.791396523338737\\
-1.71771771771772	-0.797654952068686\\
-1.70970970970971	-0.803896160142867\\
-1.7017017017017	-0.810120086984186\\
-1.69369369369369	-0.816326671500532\\
-1.68568568568569	-0.822515852089417\\
-1.67767767767768	-0.828687566642699\\
-1.66966966966967	-0.834841752551422\\
-1.66166166166166	-0.840978346710753\\
-1.65365365365365	-0.847097285525015\\
-1.64564564564565	-0.853198504912819\\
-1.63763763763764	-0.859281940312297\\
-1.62962962962963	-0.865347526686425\\
-1.62162162162162	-0.871395198528442\\
-1.61361361361361	-0.877424889867358\\
-1.60560560560561	-0.883436534273557\\
-1.5975975975976	-0.889430064864484\\
-1.58958958958959	-0.89540541431042\\
-1.58158158158158	-0.901362514840347\\
-1.57357357357357	-0.907301298247887\\
-1.56556556556557	-0.913221695897329\\
-1.55755755755756	-0.919123638729739\\
-1.54954954954955	-0.925007057269135\\
-1.54154154154154	-0.930871881628755\\
-1.53353353353353	-0.936718041517383\\
-1.52552552552553	-0.942545466245759\\
-1.51751751751752	-0.948354084733055\\
-1.50950950950951	-0.954143825513414\\
-1.5015015015015	-0.959914616742564\\
-1.49349349349349	-0.965666386204493\\
-1.48548548548549	-0.971399061318181\\
-1.47747747747748	-0.977112569144399\\
-1.46946946946947	-0.982806836392563\\
-1.46146146146146	-0.988481789427642\\
-1.45345345345345	-0.994137354277123\\
-1.44544544544545	-0.999773456638028\\
-1.43743743743744	-1.00539002188397\\
-1.42942942942943	-1.01098697507229\\
-1.42142142142142	-1.01656424095116\\
-1.41341341341341	-1.02212174396686\\
-1.40540540540541	-1.02765940827095\\
-1.3973973973974	-1.03317715772758\\
-1.38938938938939	-1.03867491592081\\
-1.38138138138138	-1.04415260616194\\
-1.37337337337337	-1.04961015149692\\
-1.36536536536537	-1.05504747471373\\
-1.35735735735736	-1.06046449834985\\
-1.34934934934935	-1.06586114469971\\
-1.34134134134134	-1.07123733582216\\
-1.33333333333333	-1.07659299354803\\
-1.32532532532533	-1.08192803948759\\
-1.31731731731732	-1.08724239503818\\
-1.30930930930931	-1.09253598139168\\
-1.3013013013013	-1.09780871954216\\
-1.29329329329329	-1.10306053029341\\
-1.28528528528529	-1.10829133426656\\
-1.27727727727728	-1.11350105190768\\
-1.26926926926927	-1.11868960349535\\
-1.26126126126126	-1.1238569091483\\
-1.25325325325325	-1.12900288883301\\
-1.24524524524525	-1.13412746237129\\
-1.23723723723724	-1.1392305494479\\
-1.22922922922923	-1.14431206961814\\
-1.22122122122122	-1.14937194231542\\
-1.21321321321321	-1.15441008685885\\
-1.20520520520521	-1.15942642246079\\
-1.1971971971972	-1.16442086823443\\
-1.18918918918919	-1.16939334320127\\
-1.18118118118118	-1.1743437662987\\
-1.17317317317317	-1.17927205638744\\
-1.16516516516517	-1.18417813225906\\
-1.15715715715716	-1.18906191264339\\
-1.14914914914915	-1.19392331621601\\
-1.14114114114114	-1.19876226160559\\
-1.13313313313313	-1.20357866740128\\
-1.12512512512513	-1.2083724521601\\
-1.11711711711712	-1.21314353441419\\
-1.10910910910911	-1.21789183267813\\
-1.1011011011011	-1.22261726545619\\
-1.09309309309309	-1.2273197512495\\
-1.08508508508509	-1.23199920856328\\
-1.07707707707708	-1.23665555591392\\
-1.06906906906907	-1.24128871183612\\
-1.06106106106106	-1.2458985948899\\
-1.05305305305305	-1.25048512366766\\
-1.04504504504505	-1.25504821680108\\
-1.03703703703704	-1.25958779296811\\
-1.02902902902903	-1.26410377089977\\
-1.02102102102102	-1.26859606938705\\
-1.01301301301301	-1.27306460728763\\
-1.00500500500501	-1.2775093035326\\
-0.996996996996997	-1.28193007713319\\
-0.988988988988989	-1.28632684718732\\
-0.980980980980981	-1.29069953288623\\
-0.972972972972973	-1.29504805352094\\
-0.964964964964965	-1.29937232848872\\
-0.956956956956957	-1.30367227729951\\
-0.948948948948949	-1.30794781958225\\
-0.940940940940941	-1.31219887509114\\
-0.932932932932933	-1.31642536371191\\
-0.924924924924925	-1.32062720546792\\
-0.916916916916917	-1.32480432052634\\
-0.908908908908909	-1.32895662920412\\
-0.900900900900901	-1.33308405197401\\
-0.892892892892893	-1.33718650947047\\
-0.884884884884885	-1.34126392249551\\
-0.876876876876877	-1.34531621202448\\
-0.868868868868869	-1.34934329921181\\
-0.860860860860861	-1.35334510539666\\
-0.852852852852853	-1.35732155210849\\
-0.844844844844845	-1.36127256107262\\
-0.836836836836837	-1.36519805421568\\
-0.828828828828829	-1.36909795367099\\
-0.820820820820821	-1.37297218178389\\
-0.812812812812813	-1.37682066111701\\
-0.804804804804805	-1.38064331445545\\
-0.796796796796797	-1.38444006481191\\
-0.788788788788789	-1.3882108354317\\
-0.780780780780781	-1.39195554979781\\
-0.772772772772773	-1.39567413163575\\
-0.764764764764765	-1.39936650491841\\
-0.756756756756757	-1.40303259387088\\
-0.748748748748749	-1.4066723229751\\
-0.740740740740741	-1.41028561697454\\
-0.732732732732733	-1.41387240087876\\
-0.724724724724725	-1.41743259996791\\
-0.716716716716717	-1.42096613979714\\
-0.708708708708709	-1.42447294620099\\
-0.700700700700701	-1.42795294529768\\
-0.692692692692693	-1.43140606349332\\
-0.684684684684685	-1.43483222748607\\
-0.676676676676677	-1.43823136427023\\
-0.668668668668669	-1.44160340114025\\
-0.660660660660661	-1.44494826569471\\
-0.652652652652653	-1.44826588584012\\
-0.644644644644645	-1.45155618979485\\
-0.636636636636636	-1.45481910609278\\
-0.628628628628629	-1.45805456358699\\
-0.62062062062062	-1.46126249145344\\
-0.612612612612613	-1.46444281919443\\
-0.604604604604605	-1.46759547664212\\
-0.596596596596596	-1.47072039396194\\
-0.588588588588589	-1.4738175016559\\
-0.58058058058058	-1.47688673056594\\
-0.572572572572573	-1.47992801187707\\
-0.564564564564565	-1.48294127712056\\
-0.556556556556556	-1.48592645817703\\
-0.548548548548549	-1.48888348727945\\
-0.54054054054054	-1.49181229701612\\
-0.532532532532533	-1.49471282033354\\
-0.524524524524525	-1.49758499053929\\
-0.516516516516516	-1.50042874130474\\
-0.508508508508509	-1.50324400666783\\
-0.5005005005005	-1.50603072103567\\
-0.492492492492492	-1.50878881918714\\
-0.484484484484485	-1.51151823627545\\
-0.476476476476476	-1.51421890783058\\
-0.468468468468469	-1.51689076976171\\
-0.46046046046046	-1.51953375835956\\
-0.452452452452452	-1.52214781029872\\
-0.444444444444445	-1.52473286263985\\
-0.436436436436436	-1.52728885283192\\
-0.428428428428429	-1.52981571871428\\
-0.42042042042042	-1.53231339851878\\
-0.412412412412412	-1.53478183087178\\
-0.404404404404405	-1.53722095479611\\
-0.396396396396396	-1.53963070971301\\
-0.388388388388389	-1.54201103544397\\
-0.38038038038038	-1.54436187221252\\
-0.372372372372372	-1.54668316064607\\
-0.364364364364364	-1.54897484177754\\
-0.356356356356356	-1.55123685704706\\
-0.348348348348348	-1.55346914830358\\
-0.34034034034034	-1.55567165780645\\
-0.332332332332332	-1.55784432822687\\
-0.324324324324324	-1.55998710264947\\
-0.316316316316316	-1.56209992457364\\
-0.308308308308308	-1.56418273791494\\
-0.3003003003003	-1.56623548700646\\
-0.292292292292292	-1.56825811660009\\
-0.284284284284284	-1.57025057186776\\
-0.276276276276276	-1.57221279840264\\
-0.268268268268268	-1.57414474222033\\
-0.26026026026026	-1.57604634975997\\
-0.252252252252252	-1.57791756788533\\
-0.244244244244244	-1.5797583438858\\
-0.236236236236236	-1.58156862547748\\
-0.228228228228228	-1.58334836080408\\
-0.22022022022022	-1.58509749843784\\
-0.212212212212212	-1.5868159873805\\
-0.204204204204204	-1.58850377706403\\
-0.196196196196196	-1.59016081735157\\
-0.188188188188188	-1.59178705853813\\
-0.18018018018018	-1.59338245135137\\
-0.172172172172172	-1.59494694695232\\
-0.164164164164164	-1.59648049693604\\
-0.156156156156156	-1.59798305333228\\
-0.148148148148148	-1.59945456860612\\
-0.14014014014014	-1.60089499565853\\
-0.132132132132132	-1.60230428782692\\
-0.124124124124124	-1.6036823988857\\
-0.116116116116116	-1.60502928304676\\
-0.108108108108108	-1.60634489495994\\
-0.1001001001001	-1.60762918971348\\
-0.0920920920920922	-1.60888212283441\\
-0.084084084084084	-1.61010365028899\\
-0.0760760760760761	-1.61129372848304\\
-0.0680680680680679	-1.61245231426225\\
-0.06006006006006	-1.61357936491258\\
-0.0520520520520522	-1.61467483816047\\
-0.0440440440440439	-1.61573869217313\\
-0.0360360360360361	-1.61677088555882\\
-0.0280280280280278	-1.61777137736703\\
-0.02002002002002	-1.61874012708868\\
-0.0120120120120122	-1.61967709465636\\
-0.00400400400400391	-1.62058224044441\\
0.00400400400400436	-1.62145552526912\\
0.0120120120120122	-1.62229691038883\\
0.02002002002002	-1.62310635750404\\
0.0280280280280278	-1.62388382875748\\
0.0360360360360357	-1.62462928673422\\
0.0440440440440444	-1.62534269446165\\
0.0520520520520522	-1.6260240154096\\
0.06006006006006	-1.62667321349029\\
0.0680680680680679	-1.62729025305838\\
0.0760760760760757	-1.62787509891094\\
0.0840840840840844	-1.62842771628743\\
0.0920920920920922	-1.62894807086965\\
0.1001001001001	-1.62943612878173\\
0.108108108108108	-1.62989185659\\
0.116116116116116	-1.63031522130298\\
0.124124124124124	-1.63070619037126\\
0.132132132132132	-1.63106473168741\\
0.14014014014014	-1.63139081358586\\
0.148148148148148	-1.6316844048428\\
0.156156156156156	-1.63194547467603\\
0.164164164164164	-1.63217399274484\\
0.172172172172172	-1.63236992914985\\
0.18018018018018	-1.63253325443288\\
0.188188188188188	-1.63266393957675\\
0.196196196196196	-1.63276195600512\\
0.204204204204204	-1.63282727558235\\
0.212212212212212	-1.63285987061325\\
0.22022022022022	-1.63285971384294\\
0.228228228228228	-1.63282677845664\\
0.236236236236236	-1.63276103807946\\
0.244244244244245	-1.63266246677619\\
0.252252252252252	-1.63253103905111\\
0.26026026026026	-1.63236672984775\\
0.268268268268268	-1.63216951454867\\
0.276276276276277	-1.63193936897526\\
0.284284284284285	-1.63167626938748\\
0.292292292292292	-1.63138019248365\\
0.3003003003003	-1.63105111540022\\
0.308308308308308	-1.63068901571152\\
0.316316316316317	-1.63029387142953\\
0.324324324324325	-1.62986566100363\\
0.332332332332332	-1.62940436332039\\
0.34034034034034	-1.62890995770329\\
0.348348348348348	-1.6283824239125\\
0.356356356356357	-1.62782174214464\\
0.364364364364365	-1.62722789303252\\
0.372372372372372	-1.62660085764491\\
0.38038038038038	-1.62594061748631\\
0.388388388388388	-1.62524715449666\\
0.396396396396397	-1.62452045105117\\
0.404404404404405	-1.62376048996001\\
0.412412412412412	-1.62296725446813\\
0.42042042042042	-1.62214072825498\\
0.428428428428428	-1.62128089543429\\
0.436436436436437	-1.62038774055386\\
0.444444444444445	-1.61946124859529\\
0.452452452452452	-1.61850140497378\\
0.46046046046046	-1.6175081955379\\
0.468468468468468	-1.61648160656935\\
0.476476476476477	-1.61542162478276\\
0.484484484484485	-1.61432823732549\\
0.492492492492492	-1.61320143177736\\
0.5005005005005	-1.61204119615051\\
0.508508508508508	-1.61084751888914\\
0.516516516516517	-1.60962038886935\\
0.524524524524525	-1.60835979539892\\
0.532532532532533	-1.60706572821711\\
0.54054054054054	-1.6057381774945\\
0.548548548548548	-1.60437713383279\\
0.556556556556557	-1.60298258826463\\
0.564564564564565	-1.60155453225341\\
0.572572572572573	-1.60009295769317\\
0.58058058058058	-1.59859785690833\\
0.588588588588588	-1.59706922265364\\
0.596596596596597	-1.59550704811395\\
0.604604604604605	-1.59391132690408\\
0.612612612612613	-1.59228205306871\\
0.62062062062062	-1.59061922108219\\
0.628628628628628	-1.58892282584846\\
0.636636636636637	-1.5871928627009\\
0.644644644644645	-1.58542932740219\\
0.652652652652653	-1.58363221614425\\
0.66066066066066	-1.58180152554808\\
0.668668668668668	-1.57993725266368\\
0.676676676676677	-1.57803939496996\\
0.684684684684685	-1.57610795037463\\
0.692692692692693	-1.57414291721416\\
0.7007007007007	-1.57214429425363\\
0.708708708708708	-1.57011208068672\\
0.716716716716717	-1.56804627613563\\
0.724724724724725	-1.56594688065101\\
0.732732732732733	-1.5638138947119\\
0.74074074074074	-1.56164731922571\\
0.748748748748748	-1.55944715552816\\
0.756756756756757	-1.55721340538323\\
0.764764764764765	-1.55494607098318\\
0.772772772772773	-1.55264515494848\\
0.780780780780781	-1.5503106603278\\
0.788788788788789	-1.54794259059803\\
0.796796796796797	-1.54554094966425\\
0.804804804804805	-1.54310574185972\\
0.812812812812813	-1.54063697194592\\
0.820820820820821	-1.53813464511254\\
0.828828828828829	-1.53559876697751\\
0.836836836836837	-1.533029343587\\
0.844844844844845	-1.53042638141549\\
0.852852852852853	-1.52778988736576\\
0.860860860860861	-1.52511986876899\\
0.868868868868869	-1.52241633338473\\
0.876876876876877	-1.51967928940102\\
0.884884884884885	-1.51690874543439\\
0.892892892892893	-1.51410471052997\\
0.900900900900901	-1.51126719416154\\
0.908908908908909	-1.50839620623156\\
0.916916916916917	-1.50549175707132\\
0.924924924924925	-1.50255385744096\\
0.932932932932933	-1.49958251852957\\
0.940940940940941	-1.4965777519553\\
0.948948948948949	-1.49353956976543\\
0.956956956956957	-1.49046798443646\\
0.964964964964965	-1.48736300887423\\
0.972972972972973	-1.48422465641401\\
0.980980980980981	-1.48105294082063\\
0.988988988988989	-1.47784787628856\\
0.996996996996997	-1.47460947744203\\
1.00500500500501	-1.47133775933514\\
1.01301301301301	-1.46803273745203\\
1.02102102102102	-1.46469442770692\\
1.02902902902903	-1.4613228464443\\
1.03703703703704	-1.457918010439\\
1.04504504504505	-1.45447993689638\\
1.05305305305305	-1.45100864345239\\
1.06106106106106	-1.44750414817375\\
1.06906906906907	-1.44396646955805\\
1.07707707707708	-1.44039562653392\\
1.08508508508509	-1.43679163846111\\
1.09309309309309	-1.43315452513065\\
1.1011011011011	-1.42948430676499\\
1.10910910910911	-1.4257810040181\\
1.11711711711712	-1.42204463797566\\
1.12512512512513	-1.4182752301551\\
1.13313313313313	-1.41447280250583\\
1.14114114114114	-1.41063737740927\\
1.14914914914915	-1.40676897767907\\
1.15715715715716	-1.40286762656116\\
1.16516516516517	-1.39893334773389\\
1.17317317317317	-1.39496616530819\\
1.18118118118118	-1.39096610382761\\
1.18918918918919	-1.38693318826852\\
1.1971971971972	-1.38286744404012\\
1.20520520520521	-1.37876889698465\\
1.21321321321321	-1.37463757337739\\
1.22122122122122	-1.37047349992684\\
1.22922922922923	-1.36627670377474\\
1.23723723723724	-1.36204721249619\\
1.24524524524525	-1.35778505409975\\
1.25325325325325	-1.35349025702745\\
1.26126126126126	-1.34916285015491\\
1.26926926926927	-1.34480286279138\\
1.27727727727728	-1.34041032467977\\
1.28528528528529	-1.33598526599674\\
1.29329329329329	-1.33152771735268\\
1.3013013013013	-1.32703770979179\\
1.30930930930931	-1.32251527479203\\
1.31731731731732	-1.3179604442652\\
1.32532532532533	-1.31337325055688\\
1.33333333333333	-1.30875372644645\\
1.34134134134134	-1.30410190514706\\
1.34934934934935	-1.29941782030555\\
1.35735735735736	-1.29470150600249\\
1.36536536536537	-1.28995299675202\\
1.37337337337337	-1.28517232750185\\
1.38138138138138	-1.28035953363316\\
1.38938938938939	-1.27551465096045\\
1.3973973973974	-1.27063771573148\\
1.40540540540541	-1.26572876462714\\
1.41341341341341	-1.26078783476125\\
1.42142142142142	-1.25581496368047\\
1.42942942942943	-1.25081018936407\\
1.43743743743744	-1.24577355022375\\
1.44544544544545	-1.24070508510345\\
1.45345345345345	-1.23560483327908\\
1.46146146146146	-1.23047283445831\\
1.46946946946947	-1.22530912878029\\
1.47747747747748	-1.22011375681533\\
1.48548548548549	-1.21488675956466\\
1.49349349349349	-1.20962817846004\\
1.5015015015015	-1.20433805536344\\
1.50950950950951	-1.19901643256666\\
1.51751751751752	-1.19366335279097\\
1.52552552552553	-1.18827885918662\\
1.53353353353353	-1.18286299533246\\
1.54154154154154	-1.17741580523548\\
1.54954954954955	-1.17193733333027\\
1.55755755755756	-1.16642762447855\\
1.56556556556557	-1.16088672396862\\
1.57357357357357	-1.15531467751477\\
1.58158158158158	-1.14971153125671\\
1.58958958958959	-1.14407733175895\\
1.5975975975976	-1.13841212601013\\
1.60560560560561	-1.13271596142236\\
1.61361361361361	-1.12698888583048\\
1.62162162162162	-1.12123094749136\\
1.62962962962963	-1.1154421950831\\
1.63763763763764	-1.10962267770425\\
1.64564564564565	-1.10377244487294\\
1.65365365365365	-1.09789154652606\\
1.66166166166166	-1.09198003301831\\
1.66966966966967	-1.08603795512132\\
1.67767767767768	-1.08006536402261\\
1.68568568568569	-1.07406231132466\\
1.69369369369369	-1.06802884904382\\
1.7017017017017	-1.06196502960924\\
1.70970970970971	-1.05587090586176\\
1.71771771771772	-1.04974653105276\\
1.72572572572573	-1.04359195884297\\
1.73373373373373	-1.03740724330123\\
1.74174174174174	-1.03119243890321\\
1.74974974974975	-1.02494760053013\\
1.75775775775776	-1.01867278346737\\
1.76576576576577	-1.0123680434031\\
1.77377377377377	-1.00603343642685\\
1.78178178178178	-0.999669019027991\\
1.78978978978979	-0.993274848094252\\
1.7977977977978	-0.986850980910128\\
1.80580580580581	-0.980397475155275\\
1.81381381381381	-0.973914388902851\\
1.82182182182182	-0.967401780617808\\
1.82982982982983	-0.960859709155141\\
1.83783783783784	-0.954288233758086\\
1.84584584584585	-0.947687414056279\\
1.85385385385385	-0.941057310063849\\
1.86186186186186	-0.934397982177481\\
1.86986986986987	-0.927709491174418\\
1.87787787787788	-0.920991898210411\\
1.88588588588589	-0.914245264817624\\
1.89389389389389	-0.907469652902486\\
1.9019019019019	-0.900665124743492\\
1.90990990990991	-0.89383174298894\\
1.91791791791792	-0.886969570654632\\
1.92592592592593	-0.880078671121512\\
1.93393393393393	-0.873159108133245\\
1.94194194194194	-0.866210945793751\\
1.94994994994995	-0.859234248564679\\
1.95795795795796	-0.852229081262819\\
1.96596596596597	-0.845195509057473\\
1.97397397397397	-0.83813359746775\\
1.98198198198198	-0.83104341235982\\
1.98998998998999	-0.823925019944098\\
1.997997997998	-0.81677848677238\\
2.00600600600601	-0.809603879734908\\
2.01401401401401	-0.802401266057391\\
2.02202202202202	-0.795170713297952\\
2.03003003003003	-0.787912289344024\\
2.03803803803804	-0.780626062409182\\
2.04604604604605	-0.773312101029914\\
2.05405405405405	-0.765970474062335\\
2.06206206206206	-0.758601250678834\\
2.07007007007007	-0.751204500364661\\
2.07807807807808	-0.743780292914456\\
2.08608608608609	-0.73632869842871\\
2.09409409409409	-0.728849787310164\\
2.1021021021021	-0.721343630260153\\
2.11011011011011	-0.713810298274873\\
2.11811811811812	-0.7062498626416\\
2.12612612612613	-0.69866239493483\\
2.13413413413413	-0.691047967012371\\
2.14214214214214	-0.683406651011361\\
2.15015015015015	-0.675738519344223\\
2.15815815815816	-0.668043644694557\\
2.16616616616617	-0.660322100012974\\
2.17417417417417	-0.652573958512853\\
2.18218218218218	-0.64479929366605\\
2.19019019019019	-0.636998179198523\\
2.1981981981982	-0.629170689085915\\
2.20620620620621	-0.621316897549053\\
2.21421421421421	-0.613436879049395\\
2.22222222222222	-0.605530708284413\\
2.23023023023023	-0.597598460182902\\
2.23823823823824	-0.589640209900234\\
2.24624624624625	-0.58165603281355\\
2.25425425425425	-0.573646004516879\\
2.26226226226226	-0.565610200816205\\
2.27027027027027	-0.557548697724463\\
2.27827827827828	-0.549461571456472\\
2.28628628628629	-0.541348898423815\\
2.29429429429429	-0.533210755229647\\
2.3023023023023	-0.525047218663445\\
2.31031031031031	-0.516858365695696\\
2.31831831831832	-0.508644273472526\\
2.32632632632633	-0.500405019310267\\
2.33433433433433	-0.492140680689966\\
2.34234234234234	-0.48385133525183\\
2.35035035035035	-0.475537060789617\\
2.35835835835836	-0.467197935244964\\
2.36636636636637	-0.458834036701665\\
2.37437437437437	-0.450445443379883\\
2.38238238238238	-0.442032233630313\\
2.39039039039039	-0.433594485928279\\
2.3983983983984	-0.425132278867792\\
2.40640640640641	-0.416645691155538\\
2.41441441441441	-0.408134801604821\\
2.42242242242242	-0.399599689129454\\
2.43043043043043	-0.391040432737591\\
2.43843843843844	-0.382457111525518\\
2.44644644644645	-0.373849804671386\\
2.45445445445445	-0.365218591428899\\
2.46246246246246	-0.356563551120952\\
2.47047047047047	-0.347884763133226\\
2.47847847847848	-0.339182306907731\\
2.48648648648649	-0.330456261936306\\
2.49449449449449	-0.321706707754084\\
2.5025025025025	-0.312933723932899\\
2.51051051051051	-0.304137390074662\\
2.51851851851852	-0.295317785804697\\
2.52652652652653	-0.286474990765031\\
2.53453453453453	-0.277609084607655\\
2.54254254254254	-0.26872014698774\\
2.55055055055055	-0.259808257556826\\
2.55855855855856	-0.250873495955973\\
2.56656656656657	-0.241915941808873\\
2.57457457457457	-0.232935674714953\\
2.58258258258258	-0.223932774242419\\
2.59059059059059	-0.214907319921294\\
2.5985985985986	-0.205859391236422\\
2.60660660660661	-0.196789067620445\\
2.61461461461461	-0.187696428446763\\
2.62262262262262	-0.178581553022463\\
2.63063063063063	-0.169444520581237\\
2.63863863863864	-0.160285410276275\\
2.64664664664665	-0.151104301173145\\
2.65465465465465	-0.141901272242654\\
2.66266266266266	-0.132676402353703\\
2.67067067067067	-0.123429770266115\\
2.67867867867868	-0.114161454623474\\
2.68668668668669	-0.104871533945936\\
2.69469469469469	-0.0955600866230475\\
2.7027027027027	-0.0862271909065511\\
2.71071071071071	-0.0768729249031894\\
2.71871871871872	-0.0674973665675136\\
2.72672672672673	-0.0581005936946848\\
2.73473473473473	-0.0486826839132861\\
2.74274274274274	-0.0392437146781299\\
2.75075075075075	-0.0297837632630817\\
2.75875875875876	-0.0203029067538831\\
2.76676676676677	-0.0108012220409959\\
2.77477477477477	-0.00127878581244567\\
2.78278278278278	0.00826432545331079\\
2.79079079079079	0.0178280354945042\\
2.7987987987988	0.0274122682731566\\
2.80680680680681	0.0370169479821583\\
2.81481481481481	0.0466419990523312\\
2.82282282282282	0.0562873461594613\\
2.83083083083083	0.065952914231308\\
2.83883883883884	0.0756386284545858\\
2.84684684684685	0.08534441428191\\
2.85485485485485	0.095070197438722\\
2.86286286286286	0.104815903930169\\
2.87087087087087	0.114581460047954\\
2.87887887887888	0.124366792377148\\
2.88688688688689	0.134171827802951\\
2.89489489489489	0.143996493517433\\
2.9029029029029	0.153840717026211\\
2.91091091091091	0.163704426155088\\
2.91891891891892	0.173587549056644\\
2.92692692692693	0.183490014216772\\
2.93493493493493	0.193411750461177\\
2.94294294294294	0.203352686961799\\
2.95095095095095	0.213312753243203\\
2.95895895895896	0.223291879188893\\
2.96696696696697	0.233289995047575\\
2.97497497497497	0.243307031439365\\
2.98298298298298	0.25334291936192\\
2.99099099099099	0.263397590196515\\
2.998998998999	0.273470975714051\\
3.00700700700701	0.283563008080984\\
3.01501501501502	0.293673619865204\\
3.02302302302302	0.30380274404183\\
3.03103103103103	0.313950313998923\\
3.03903903903904	0.324116263543144\\
3.04704704704705	0.334300526905314\\
3.05505505505506	0.344503038745917\\
3.06306306306306	0.3547237341605\\
3.07107107107107	0.364962548685011\\
3.07907907907908	0.375219418301043\\
3.08708708708709	0.385494279440989\\
3.0950950950951	0.395787068993129\\
3.1031031031031	0.40609772430661\\
3.11111111111111	0.416426183196344\\
3.11911911911912	0.426772383947825\\
3.12712712712713	0.437136265321832\\
3.13513513513514	0.447517766559063\\
3.14314314314314	0.457916827384664\\
3.15115115115115	0.468333388012656\\
3.15915915915916	0.47876738915028\\
3.16716716716717	0.48921877200223\\
3.17517517517518	0.499687478274797\\
3.18318318318318	0.510173450179911\\
3.19119119119119	0.520676630439081\\
3.1991991991992	0.531196962287229\\
3.20720720720721	0.541734389476427\\
3.21521521521522	0.552288856279533\\
3.22322322322322	0.562860307493718\\
3.23123123123123	0.57344868844388\\
3.23923923923924	0.584053944985972\\
3.24724724724725	0.594676023510203\\
3.25525525525526	0.605314870944142\\
3.26326326326326	0.615970434755721\\
3.27127127127127	0.626642662956111\\
3.27927927927928	0.637331504102501\\
3.28728728728729	0.648036907300773\\
3.2952952952953	0.658758822208053\\
3.3033033033033	0.669497199035165\\
3.31131131131131	0.680251988548968\\
3.31931931931932	0.691023142074573\\
3.32732732732733	0.701810611497478\\
3.33533533533534	0.712614349265559\\
3.34334334334334	0.723434308390968\\
3.35135135135135	0.734270442451919\\
3.35935935935936	0.745122705594355\\
3.36736736736737	0.755991052533514\\
3.37537537537538	0.766875438555371\\
3.38338338338338	0.777775819517986\\
3.39139139139139	0.788692151852721\\
3.3993993993994	0.799624392565362\\
3.40740740740741	0.810572499237127\\
3.41541541541542	0.821536430025558\\
3.42342342342342	0.832516143665309\\
3.43143143143143	0.843511599468824\\
3.43943943943944	0.854522757326902\\
3.44744744744745	0.865549577709156\\
3.45545545545546	0.87659202166437\\
3.46346346346346	0.887650050820737\\
3.47147147147147	0.898723627385997\\
3.47947947947948	0.909812714147468\\
3.48748748748749	0.920917274471974\\
3.4954954954955	0.932037272305659\\
3.5035035035035	0.94317267217371\\
3.51151151151151	0.954323439179969\\
3.51951951951952	0.965489539006433\\
3.52752752752753	0.976670937912679\\
3.53553553553554	0.987867602735162\\
3.54354354354354	0.999079500886424\\
3.55155155155155	1.01030660035421\\
3.55955955955956	1.02154886970047\\
3.56756756756757	1.03280627806028\\
3.57557557557558	1.04407879514066\\
3.58358358358358	1.05536639121931\\
3.59159159159159	1.06666903714322\\
3.5995995995996	1.07798670432721\\
3.60760760760761	1.08931936475241\\
3.61561561561562	1.10066699096459\\
3.62362362362362	1.11202955607242\\
3.63163163163163	1.12340703374568\\
3.63963963963964	1.13479939821335\\
3.64764764764765	1.14620662426159\\
3.65565565565566	1.15762868723172\\
3.66366366366366	1.16906556301802\\
3.67167167167167	1.1805172280655\\
3.67967967967968	1.19198365936763\\
3.68768768768769	1.20346483446389\\
3.6956956956957	1.21496073143731\\
3.7037037037037	1.22647132891199\\
3.71171171171171	1.23799660605037\\
3.71971971971972	1.24953654255065\\
3.72772772772773	1.26109111864395\\
3.73573573573574	1.27266031509153\\
3.74374374374374	1.28424411318187\\
3.75175175175175	1.29584249472769\\
3.75975975975976	1.30745544206296\\
3.76776776776777	1.31908293803978\\
3.77577577577578	1.33072496602523\\
3.78378378378378	1.34238150989816\\
3.79179179179179	1.35405255404592\\
3.7997997997998	1.36573808336104\\
3.80780780780781	1.37743808323781\\
3.81581581581582	1.38915253956889\\
3.82382382382382	1.40088143874179\\
3.83183183183183	1.41262476763531\\
3.83983983983984	1.42438251361601\\
3.84784784784785	1.43615466453451\\
3.85585585585586	1.44794120872184\\
3.86386386386386	1.45974213498571\\
3.87187187187187	1.47155743260672\\
3.87987987987988	1.48338709133455\\
3.88788788788789	1.49523110138413\\
3.8958958958959	1.50708945343175\\
3.9039039039039	1.51896213861108\\
3.91191191191191	1.53084914850927\\
3.91991991991992	1.54275047516292\\
3.92792792792793	1.55466611105405\\
3.93593593593594	1.56659604910607\\
3.94394394394394	1.57854028267965\\
3.95195195195195	1.59049880556863\\
3.95995995995996	1.60247161199587\\
3.96796796796797	1.6144586966091\\
3.97597597597598	1.62646005447667\\
3.98398398398398	1.63847568108343\\
3.99199199199199	1.65050557232641\\
4	1.66254972451062\\
4	4.56803680790085\\
3.99199199199199	4.54919648607063\\
3.98398398398398	4.53040702179246\\
3.97597597597598	4.51166841137135\\
3.96796796796797	4.49298065070433\\
3.95995995995996	4.47434373527624\\
3.95195195195195	4.45575766015546\\
3.94394394394394	4.43722241998969\\
3.93593593593594	4.41873800900179\\
3.92792792792793	4.40030442098562\\
3.91991991991992	4.38192164930185\\
3.91191191191191	4.36358968687386\\
3.9039039039039	4.3453085261837\\
3.8958958958959	4.32707815926796\\
3.88788788788789	4.30889857771379\\
3.87987987987988	4.29076977265486\\
3.87187187187187	4.27269173476746\\
3.86386386386386	4.25466445426652\\
3.85585585585586	4.23668792090171\\
3.84784784784785	4.21876212395366\\
3.83983983983984	4.20088705223005\\
3.83183183183183	4.18306269406192\\
3.82382382382382	4.1652890372999\\
3.81581581581582	4.14756606931052\\
3.80780780780781	4.12989377697262\\
3.7997997997998	4.11227214667368\\
3.79179179179179	4.09470116430637\\
3.78378378378378	4.07718081526499\\
3.77577577577578	4.05971108444205\\
3.76776776776777	4.04229195622491\\
3.75975975975976	4.02492341449242\\
3.75175175175175	4.00760544261166\\
3.74374374374374	3.99033802343473\\
3.73573573573574	3.9731211392956\\
3.72772772772773	3.955954772007\\
3.71971971971972	3.9388389028574\\
3.71171171171171	3.92177351260805\\
3.7037037037037	3.90475858149009\\
3.6956956956957	3.88779408920169\\
3.68768768768769	3.87088001490533\\
3.67967967967968	3.85401633722508\\
3.67167167167167	3.83720303424399\\
3.66366366366366	3.82044008350153\\
3.65565565565566	3.80372746199116\\
3.64764764764765	3.7870651461579\\
3.63963963963964	3.77045311189604\\
3.63163163163163	3.75389133454688\\
3.62362362362362	3.7373797888966\\
3.61561561561562	3.72091844917417\\
3.60760760760761	3.70450728904936\\
3.5995995995996	3.68814628163086\\
3.59159159159159	3.67183539946443\\
3.58358358358358	3.65557461453119\\
3.57557557557558	3.63936389824597\\
3.56756756756757	3.62320322145577\\
3.55955955955956	3.60709255443828\\
3.55155155155155	3.59103186690052\\
3.54354354354354	3.57502112797756\\
3.53553553553554	3.55906030623136\\
3.52752752752753	3.54314936964966\\
3.51951951951952	3.527288285645\\
3.51151151151151	3.51147702105385\\
3.5035035035035	3.49571554213576\\
3.4954954954955	3.48000381457275\\
3.48748748748749	3.46434180346866\\
3.47947947947948	3.44872947334866\\
3.47147147147147	3.43316678815891\\
3.46346346346346	3.41765371126623\\
3.45545545545546	3.40219020545793\\
3.44744744744745	3.38677623294177\\
3.43943943943944	3.37141175534592\\
3.43143143143143	3.35609673371918\\
3.42342342342342	3.34083112853115\\
3.41541541541542	3.32561489967264\\
3.40740740740741	3.3104480064561\\
3.3993993993994	3.29533040761616\\
3.39139139139139	3.28026206131038\\
3.38338338338338	3.26524292511998\\
3.37537537537538	3.25027295605073\\
3.36736736736737	3.23535211053401\\
3.35935935935936	3.22048034442787\\
3.35135135135135	3.20565761301829\\
3.34334334334334	3.1908838710205\\
3.33533533533534	3.17615907258045\\
3.32732732732733	3.16148317127635\\
3.31931931931932	3.14685612012036\\
3.31131131131131	3.13227787156035\\
3.3033033033033	3.11774837748181\\
3.2952952952953	3.10326758920987\\
3.28728728728729	3.08883545751137\\
3.27927927927928	3.07445193259715\\
3.27127127127127	3.06011696412432\\
3.26326326326326	3.04583050119877\\
3.25525525525526	3.03159249237769\\
3.24724724724725	3.01740288567226\\
3.23923923923924	3.00326162855039\\
3.23123123123123	2.98916866793967\\
3.22322322322322	2.97512395023029\\
3.21521521521522	2.96112742127822\\
3.20720720720721	2.94717902640835\\
3.1991991991992	2.93327871041786\\
3.19119119119119	2.91942641757959\\
3.18318318318318	2.90562209164562\\
3.17517517517518	2.89186567585088\\
3.16716716716717	2.87815711291687\\
3.15915915915916	2.86449634505553\\
3.15115115115115	2.85088331397314\\
3.14314314314314	2.8373179608744\\
3.13513513513514	2.82380022646654\\
3.12712712712713	2.8103300509636\\
3.11911911911912	2.79690737409071\\
3.11111111111111	2.78353213508858\\
3.1031031031031	2.77020427271798\\
3.0950950950951	2.75692372526441\\
3.08708708708709	2.74369043054278\\
3.07907907907908	2.73050432590223\\
3.07107107107107	2.71736534823105\\
3.06306306306306	2.70427343396163\\
3.05505505505506	2.69122851907556\\
3.04704704704705	2.67823053910879\\
3.03903903903904	2.66527942915687\\
3.03103103103103	2.65237512388028\\
3.02302302302302	2.63951755750984\\
3.01501501501502	2.62670666385221\\
3.00700700700701	2.61394237629546\\
2.998998998999	2.6012246278147\\
2.99099099099099	2.58855335097783\\
2.98298298298298	2.57592847795129\\
2.97497497497497	2.563349940506\\
2.96696696696697	2.55081767002322\\
2.95895895895896	2.53833159750061\\
2.95095095095095	2.52589165355829\\
2.94294294294294	2.51349776844496\\
2.93493493493493	2.50114987204413\\
2.92692692692693	2.48884789388037\\
2.91891891891892	2.47659176312561\\
2.91091091091091	2.46438140860555\\
2.9029029029029	2.4522167588061\\
2.89489489489489	2.44009774187983\\
2.88688688688689	2.42802428565254\\
2.87887887887888	2.41599631762986\\
2.87087087087087	2.40401376500384\\
2.86286286286286	2.3920765546597\\
2.85485485485485	2.3801846131825\\
2.84684684684685	2.36833786686394\\
2.83883883883884	2.35653624170918\\
2.83083083083083	2.34477966344365\\
2.82282282282282	2.33306805751997\\
2.81481481481481	2.32140134912485\\
2.80680680680681	2.30977946318606\\
2.7987987987988	2.29820232437937\\
2.79079079079079	2.28666985713562\\
2.78278278278278	2.27518198564769\\
2.77477477477477	2.2637386338776\\
2.76676676676677	2.25233972556358\\
2.75875875875876	2.24098518422718\\
2.75075075075075	2.22967493318037\\
2.74274274274274	2.2184088955327\\
2.73473473473473	2.20718699419841\\
2.72672672672673	2.19600915190364\\
2.71871871871872	2.18487529119358\\
2.71071071071071	2.17378533443965\\
2.7027027027027	2.16273920384669\\
2.69469469469469	2.15173682146014\\
2.68668668668669	2.14077810917327\\
2.67867867867868	2.12986298873432\\
2.67067067067067	2.11899138175376\\
2.66266266266266	2.10816320971142\\
2.65465465465465	2.09737839396373\\
2.64664664664665	2.08663685575086\\
2.63863863863864	2.0759385162039\\
2.63063063063063	2.06528329635206\\
2.62262262262262	2.05467111712976\\
2.61461461461461	2.04410189938382\\
2.60660660660661	2.03357556388054\\
2.5985985985986	2.02309203131283\\
2.59059059059059	2.0126512223073\\
2.58258258258258	2.00225305743131\\
2.57457457457457	1.9918974572\\
2.56656656656657	1.98158434208336\\
2.55855855855856	1.97131363251317\\
2.55055055055055	1.96108524889003\\
2.54254254254254	1.95089911159022\\
2.53453453453453	1.94075514097269\\
2.52652652652653	1.93065325738591\\
2.51851851851852	1.92059338117469\\
2.51051051051051	1.91057543268706\\
2.5025025025025	1.90059933228097\\
2.49449449449449	1.89066500033112\\
2.48648648648649	1.88077235723558\\
2.47847847847848	1.87092132342252\\
2.47047047047047	1.86111181935682\\
2.46246246246246	1.85134376554663\\
2.45445445445445	1.84161708254993\\
2.44644644644645	1.83193169098106\\
2.43843843843844	1.82228751151711\\
2.43043043043043	1.81268446490439\\
2.42242242242242	1.80312247196473\\
2.41441441441441	1.79360145360186\\
2.40640640640641	1.78412133080762\\
2.3983983983984	1.77468202466819\\
2.39039039039039	1.76528345637028\\
2.38238238238238	1.7559255472072\\
2.37437437437437	1.74660821858493\\
2.36636636636637	1.73733139202816\\
2.35835835835836	1.72809498918618\\
2.35035035035035	1.71889893183883\\
2.34234234234234	1.70974314190233\\
2.33433433433433	1.70062754143503\\
2.32632632632633	1.69155205264317\\
2.31831831831832	1.68251659788655\\
2.31031031031031	1.67352109968413\\
2.3023023023023	1.66456548071956\\
2.29429429429429	1.65564966384673\\
2.28628628628629	1.64677357209514\\
2.27827827827828	1.63793712867532\\
2.27027027027027	1.62914025698412\\
2.26226226226226	1.62038288060994\\
2.25425425425425	1.61166492333798\\
2.24624624624625	1.6029863091553\\
2.23823823823824	1.59434696225591\\
2.23023023023023	1.58574680704578\\
2.22222222222222	1.57718576814778\\
2.21421421421421	1.56866377040653\\
2.20620620620621	1.56018073889323\\
2.1981981981982	1.55173659891042\\
2.19019019019019	1.54333127599663\\
2.18218218218218	1.53496469593104\\
2.17417417417417	1.52663678473801\\
2.16616616616617	1.51834746869158\\
2.15815815815816	1.51009667431989\\
2.15015015015015	1.50188432840956\\
2.14214214214214	1.49371035800999\\
2.13413413413413	1.48557469043757\\
2.12612612612613	1.47747725327987\\
2.11811811811812	1.46941797439977\\
2.11011011011011	1.46139678193945\\
2.1021021021021	1.45341360432442\\
2.09409409409409	1.4454683702674\\
2.08608608608609	1.43756100877219\\
2.07807807807808	1.42969144913747\\
2.07007007007007	1.42185962096048\\
2.06206206206206	1.41406545414074\\
2.05405405405405	1.40630887888361\\
2.04604604604605	1.39858982570384\\
2.03803803803804	1.39090822542904\\
2.03003003003003	1.38326400920309\\
2.02202202202202	1.37565710848951\\
2.01401401401401	1.36808745507472\\
2.00600600600601	1.36055498107128\\
1.997997997998	1.35305961892108\\
1.98998998998999	1.34560130139841\\
1.98198198198198	1.33817996161303\\
1.97397397397397	1.33079553301313\\
1.96596596596597	1.3234479493883\\
1.95795795795796	1.31613714487238\\
1.94994994994995	1.30886305394625\\
1.94194194194194	1.30162561144061\\
1.93393393393393	1.29442475253868\\
1.92592592592593	1.2872604127788\\
1.91791791791792	1.28013252805705\\
1.90990990990991	1.27304103462977\\
1.9019019019019	1.26598586911601\\
1.89389389389389	1.25896696849998\\
1.88588588588589	1.25198427013337\\
1.87787787787788	1.24503771173769\\
1.86986986986987	1.23812723140651\\
1.86186186186186	1.23125276760767\\
1.85385385385385	1.22441425918541\\
1.84584584584585	1.21761164536249\\
1.83783783783784	1.21084486574223\\
1.82982982982983	1.2041138603105\\
1.82182182182182	1.19741856943766\\
1.81381381381381	1.19075893388048\\
1.80580580580581	1.18413489478396\\
1.7977977977978	1.17754639368314\\
1.78978978978979	1.17099337250488\\
1.78178178178178	1.16447577356952\\
1.77377377377377	1.15799353959255\\
1.76576576576577	1.15154661368626\\
1.75775775775776	1.14513493936126\\
1.74974974974975	1.13875846052803\\
1.74174174174174	1.13241712149841\\
1.73373373373373	1.12611086698701\\
1.72572572572573	1.11983964211261\\
1.71771771771772	1.11360339239953\\
1.70970970970971	1.10740206377895\\
1.7017017017017	1.10123560259012\\
1.69369369369369	1.09510395558168\\
1.68568568568569	1.08900706991278\\
1.67767767767768	1.08294489315427\\
1.66966966966967	1.07691737328979\\
1.66166166166166	1.07092445871688\\
1.65365365365365	1.06496609824801\\
1.64564564564565	1.05904224111155\\
1.63763763763764	1.05315283695279\\
1.62962962962963	1.04729783583486\\
1.62162162162162	1.04147718823962\\
1.61361361361361	1.03569084506852\\
1.60560560560561	1.02993875764346\\
1.5975975975976	1.02422087770757\\
1.58958958958959	1.01853715742601\\
1.58158158158158	1.01288754938666\\
1.57357357357357	1.0072720066009\\
1.56556556556557	1.00169048250421\\
1.55755755755756	0.996142930956887\\
1.54954954954955	0.990629306244625\\
1.54154154154154	0.985149563079134\\
1.53353353353353	0.979703656598696\\
1.52552552552553	0.97429154236871\\
1.51751751751752	0.968913176382201\\
1.50950950950951	0.963568515060318\\
1.5015015015015	0.95825751525279\\
1.49349349349349	0.95298013423837\\
1.48548548548549	0.947736329725253\\
1.47747747747748	0.942526059851468\\
1.46946946946947	0.937349283185246\\
1.46146146146146	0.932205958725375\\
1.45345345345345	0.927096045901525\\
1.44544544544545	0.922019504574556\\
1.43743743743744	0.916976295036801\\
1.42942942942943	0.91196637801234\\
1.42142142142142	0.906989714657245\\
1.41341341341341	0.902046266559807\\
1.40540540540541	0.897135995740755\\
1.3973973973974	0.892258864653442\\
1.38938938938939	0.88741483618403\\
1.38138138138138	0.882603873651645\\
1.37337337337337	0.877825940808525\\
1.36536536536537	0.873081001840153\\
1.35735735735736	0.868369021365365\\
1.34934934934935	0.863689964436455\\
1.34134134134134	0.859043796539263\\
1.33333333333333	0.854430483593245\\
1.32532532532533	0.849849991951536\\
1.31731731731732	0.845302288400998\\
1.30930930930931	0.840787340162254\\
1.3013013013013	0.836305114889717\\
1.29329329329329	0.8318555806716\\
1.28528528528529	0.827438706029922\\
1.27727727727728	0.8230544599205\\
1.26926926926927	0.818702811732932\\
1.26126126126126	0.814383731290572\\
1.25325325325325	0.810097188850496\\
1.24524524524525	0.805843155103462\\
1.23723723723724	0.801621601173852\\
1.22922922922923	0.797432498619622\\
1.22122122122122	0.793275819432231\\
1.21321321321321	0.789151536036573\\
1.20520520520521	0.785059621290895\\
1.1971971971972	0.781000048486717\\
1.18918918918919	0.776972791348737\\
1.18118118118118	0.772977824034741\\
1.17317317317317	0.769015121135504\\
1.16516516516517	0.765084657674677\\
1.15715715715716	0.76118640910869\\
1.14914914914915	0.757320351326633\\
1.14114114114114	0.753486460650142\\
1.13313313313313	0.749684713833284\\
1.12512512512513	0.74591508806243\\
1.11711711711712	0.742177560956134\\
1.10910910910911	0.738472110565011\\
1.1011011011011	0.734798715371603\\
1.09309309309309	0.731157354290256\\
1.08508508508509	0.727548006666985\\
1.07707707707708	0.723970652279348\\
1.06906906906907	0.720425271336309\\
1.06106106106106	0.716911844478111\\
1.05305305305305	0.713430352776143\\
1.04504504504505	0.709980777732804\\
1.03703703703704	0.706563101281381\\
1.02902902902903	0.703177305785908\\
1.02102102102102	0.699823374041043\\
1.01301301301301	0.696501289271943\\
1.00500500500501	0.693211035134126\\
0.996996996996997	0.689952595713359\\
0.988988988988989	0.686725955525527\\
0.980980980980981	0.683531099516512\\
0.972972972972973	0.680368013062082\\
0.964964964964965	0.677236681967767\\
0.956956956956957	0.67413709246875\\
0.948948948948949	0.671069231229755\\
0.940940940940941	0.668033085344942\\
0.932932932932933	0.665028642337803\\
0.924924924924925	0.662055890161063\\
0.916916916916917	0.659114817196579\\
0.908908908908909	0.656205412255253\\
0.900900900900901	0.653327664576943\\
0.892892892892893	0.650481563830374\\
0.884884884884885	0.647667100113064\\
0.876876876876877	0.644884263951246\\
0.868868868868869	0.642133046299796\\
0.860860860860861	0.639413438542168\\
0.852852852852853	0.636725432490336\\
0.844844844844845	0.634069020384733\\
0.836836836836837	0.631444194894199\\
0.828828828828829	0.628850949115943\\
0.820820820820821	0.626289276575494\\
0.812812812812813	0.623759171226668\\
0.804804804804805	0.621260627451543\\
0.796796796796797	0.618793640060428\\
0.788788788788789	0.616358204291849\\
0.780780780780781	0.613954315812534\\
0.772772772772773	0.611581970717408\\
0.764764764764765	0.60924116552959\\
0.756756756756757	0.606931897200397\\
0.748748748748748	0.604654163109358\\
0.74074074074074	0.60240796106423\\
0.732732732732733	0.600193289301016\\
0.724724724724725	0.598010146484001\\
0.716716716716717	0.595858531705781\\
0.708708708708708	0.593738444487306\\
0.7007007007007	0.59164988477793\\
0.692692692692693	0.589592852955459\\
0.684684684684685	0.587567349826215\\
0.676676676676677	0.585573376625097\\
0.668668668668668	0.583610935015657\\
0.66066066066066	0.581680027090176\\
0.652652652652653	0.579780655369749\\
0.644644644644645	0.57791282280437\\
0.636636636636637	0.576076532773035\\
0.628628628628628	0.574271789083839\\
0.62062062062062	0.572498595974086\\
0.612612612612613	0.570756958110403\\
0.604604604604605	0.569046880588857\\
0.596596596596597	0.567368368935085\\
0.588588588588588	0.565721429104419\\
0.58058058058058	0.56410606748203\\
0.572572572572573	0.562522290883063\\
0.564564564564565	0.560970106552791\\
0.556556556556557	0.559449522166765\\
0.548548548548548	0.557960545830972\\
0.54054054054054	0.556503186082001\\
0.532532532532533	0.555077451887211\\
0.524524524524525	0.553683352644903\\
0.516516516516517	0.5523208981845\\
0.508508508508508	0.550990098766733\\
0.5005005005005	0.549690965083822\\
0.492492492492492	0.548423508259676\\
0.484484484484485	0.547187739850085\\
0.476476476476477	0.545983671842921\\
0.468468468468468	0.544811316658347\\
0.46046046046046	0.54367068714902\\
0.452452452452452	0.542561796600307\\
0.444444444444445	0.5414846587305\\
0.436436436436437	0.540439287691032\\
0.428428428428428	0.539425698066707\\
0.42042042042042	0.538443904875918\\
0.412412412412412	0.537493923570876\\
0.404404404404405	0.536575770037844\\
0.396396396396397	0.535689460597366\\
0.388388388388388	0.534835012004504\\
0.38038038038038	0.534012441449074\\
0.372372372372372	0.533221766555884\\
0.364364364364365	0.532463005384975\\
0.356356356356357	0.53173617643186\\
0.348348348348348	0.531041298627767\\
0.34034034034034	0.530378391339882\\
0.332332332332332	0.52974747437159\\
0.324324324324325	0.529148567962718\\
0.316316316316317	0.528581692789781\\
0.308308308308308	0.52804686996622\\
0.3003003003003	0.527544121042647\\
0.292292292292292	0.527073468007082\\
0.284284284284285	0.526634933285194\\
0.276276276276277	0.526228539740541\\
0.268268268268268	0.5258543106748\\
0.26026026026026	0.525512269828005\\
0.252252252252252	0.525202441378775\\
0.244244244244245	0.524924849944542\\
0.236236236236236	0.524679520581777\\
0.228228228228228	0.524466478786207\\
0.22022022022022	0.524285750493035\\
0.212212212212212	0.524137362077152\\
0.204204204204204	0.524021340353342\\
0.196196196196196	0.523937712576486\\
0.188188188188188	0.52388650644176\\
0.18018018018018	0.523867750084822\\
0.172172172172172	0.523881472082003\\
0.164164164164164	0.523927701450476\\
0.156156156156156	0.524006467648436\\
0.148148148148148	0.524117800575258\\
0.14014014014014	0.524261730571651\\
0.132132132132132	0.524438288419811\\
0.124124124124124	0.524647505343553\\
0.116116116116116	0.524889413008441\\
0.108108108108108	0.525164043521908\\
0.1001001001001	0.525471429433368\\
0.0920920920920922	0.525811603734306\\
0.0840840840840844	0.526184599858374\\
0.0760760760760757	0.526590451681458\\
0.0680680680680679	0.527029193521749\\
0.06006006006006	0.527500860139789\\
0.0520520520520522	0.528005486738509\\
0.0440440440440444	0.528543108963253\\
0.0360360360360357	0.529113762901792\\
0.0280280280280278	0.529717485084311\\
0.02002002002002	0.530354312483397\\
0.0120120120120122	0.531024282514\\
0.00400400400400436	0.531727433033381\\
-0.00400400400400391	0.532463802341043\\
-0.0120120120120122	0.533233429178647\\
-0.02002002002002	0.534036352729906\\
-0.0280280280280278	0.534872612620463\\
-0.0360360360360361	0.535742248917754\\
-0.0440440440440439	0.536645302130839\\
-0.0520520520520522	0.537581813210228\\
-0.06006006006006	0.538551823547676\\
-0.0680680680680679	0.539555374975962\\
-0.0760760760760761	0.540592509768639\\
-0.084084084084084	0.541663270639773\\
-0.0920920920920922	0.542767700743646\\
-0.1001001001001	0.543905843674447\\
-0.108108108108108	0.545077743465928\\
-0.116116116116116	0.546283444591043\\
-0.124124124124124	0.547522991961557\\
-0.132132132132132	0.54879643092763\\
-0.14014014014014	0.550103807277375\\
-0.148148148148148	0.551445167236386\\
-0.156156156156156	0.552820557467241\\
-0.164164164164164	0.554230025068973\\
-0.172172172172172	0.555673617576513\\
-0.18018018018018	0.557151382960103\\
-0.188188188188188	0.558663369624678\\
-0.196196196196196	0.560209626409215\\
-0.204204204204204	0.561790202586052\\
-0.212212212212212	0.563405147860173\\
-0.22022022022022	0.56505451236846\\
-0.228228228228228	0.566738346678909\\
-0.236236236236236	0.568456701789814\\
-0.244244244244244	0.570209629128912\\
-0.252252252252252	0.571997180552494\\
-0.26026026026026	0.57381940834448\\
-0.268268268268268	0.575676365215454\\
-0.276276276276276	0.577568104301659\\
-0.284284284284284	0.579494679163959\\
-0.292292292292292	0.581456143786756\\
-0.3003003003003	0.583452552576867\\
-0.308308308308308	0.585483960362361\\
-0.316316316316316	0.587550422391356\\
-0.324324324324324	0.589651994330772\\
-0.332332332332332	0.591788732265034\\
-0.34034034034034	0.593960692694747\\
-0.348348348348348	0.596167932535306\\
-0.356356356356356	0.598410509115482\\
-0.364364364364364	0.600688480175941\\
-0.372372372372372	0.603001903867733\\
-0.38038038038038	0.605350838750726\\
-0.388388388388389	0.607735343791992\\
-0.396396396396396	0.610155478364142\\
-0.404404404404405	0.612611302243624\\
-0.412412412412412	0.615102875608949\\
-0.42042042042042	0.617630259038891\\
-0.428428428428429	0.620193513510612\\
-0.436436436436436	0.622792700397756\\
-0.444444444444445	0.625427881468472\\
-0.452452452452452	0.628099118883399\\
-0.46046046046046	0.630806475193585\\
-0.468468468468469	0.633550013338355\\
-0.476476476476476	0.636329796643132\\
-0.484484484484485	0.639145888817185\\
-0.492492492492492	0.641998353951337\\
-0.5005005005005	0.644887256515608\\
-0.508508508508509	0.647812661356796\\
-0.516516516516516	0.650774633696012\\
-0.524524524524525	0.653773239126142\\
-0.532532532532533	0.656808543609263\\
-0.54054054054054	0.659880613473984\\
-0.548548548548549	0.662989515412743\\
-0.556556556556556	0.666135316479026\\
-0.564564564564565	0.66931808408454\\
-0.572572572572573	0.672537885996312\\
-0.58058058058058	0.675794790333727\\
-0.588588588588589	0.679088865565515\\
-0.596596596596596	0.682420180506655\\
-0.604604604604605	0.685788804315228\\
-0.612612612612613	0.689194806489204\\
-0.62062062062062	0.69263825686316\\
-0.628628628628629	0.696119225604937\\
-0.636636636636636	0.699637783212225\\
-0.644644644644645	0.703194000509091\\
-0.652652652652653	0.706787948642429\\
-0.660660660660661	0.710419699078357\\
-0.668668668668669	0.714089323598534\\
-0.676676676676677	0.717796894296415\\
-0.684684684684685	0.721542483573442\\
-0.692692692692693	0.72532616413516\\
-0.700700700700701	0.72914800898727\\
-0.708708708708709	0.733008091431608\\
-0.716716716716717	0.736906485062064\\
-0.724724724724725	0.740843263760424\\
-0.732732732732733	0.744818501692147\\
-0.740740740740741	0.748832273302072\\
-0.748748748748749	0.752884653310057\\
-0.756756756756757	0.756975716706545\\
-0.764764764764765	0.761105538748069\\
-0.772772772772773	0.765274194952675\\
-0.780780780780781	0.769481761095291\\
-0.788788788788789	0.773728313203011\\
-0.796796796796797	0.778013927550323\\
-0.804804804804805	0.782338680654262\\
-0.812812812812813	0.786702649269491\\
-0.820820820820821	0.791105910383318\\
-0.828828828828829	0.795548541210647\\
-0.836836836836837	0.800030619188849\\
-0.844844844844845	0.804552221972581\\
-0.852852852852853	0.809113427428519\\
-0.860860860860861	0.813714313630041\\
-0.868868868868869	0.818354958851828\\
-0.876876876876877	0.823035441564408\\
-0.884884884884885	0.827755840428629\\
-0.892892892892893	0.832516234290064\\
-0.900900900900901	0.837316702173358\\
-0.908908908908909	0.842157323276498\\
-0.916916916916917	0.84703817696503\\
-0.924924924924925	0.851959342766205\\
-0.932932932932933	0.856920900363061\\
-0.940940940940941	0.86192292958845\\
-0.948948948948949	0.866965510418989\\
-0.956956956956957	0.872048722968965\\
-0.964964964964965	0.877172647484165\\
-0.972972972972973	0.882337364335658\\
-0.980980980980981	0.887542954013508\\
-0.988988988988989	0.892789497120435\\
-0.996996996996997	0.898077074365413\\
-1.00500500500501	0.90340576655722\\
-1.01301301301301	0.908775654597919\\
-1.02102102102102	0.914186819476301\\
-1.02902902902903	0.919639342261255\\
-1.03703703703704	0.925133304095104\\
-1.04504504504505	0.930668786186875\\
-1.05305305305305	0.936245869805528\\
-1.06106106106106	0.941864636273129\\
-1.06906906906907	0.94752516695798\\
-1.07707707707708	0.953227543267699\\
-1.08508508508509	0.958971846642252\\
-1.09309309309309	0.964758158546951\\
-1.1011011011011	0.970586560465393\\
-1.10910910910911	0.976457133892373\\
-1.11711711711712	0.982369960326746\\
-1.12512512512513	0.988325121264253\\
-1.13313313313313	0.994322698190313\\
-1.14114114114114	1.00036277257278\\
-1.14914914914915	1.00644542585464\\
-1.15715715715716	1.01257073944674\\
-1.16516516516517	1.0187387947204\\
-1.17317317317317	1.02494967300006\\
-1.18118118118118	1.03120345555588\\
-1.18918918918919	1.03750022359629\\
-1.1971971971972	1.04384005826056\\
-1.20520520520521	1.05022304061133\\
-1.21321321321321	1.05664925162706\\
-1.22122122122122	1.06311877219459\\
-1.22922922922923	1.06963168310156\\
-1.23723723723724	1.07618806502884\\
-1.24524524524525	1.08278799854302\\
-1.25325325325325	1.08943156408882\\
-1.26126126126126	1.09611884198147\\
-1.26926926926927	1.10284991239916\\
-1.27727727727728	1.10962485537541\\
-1.28528528528529	1.1164437507915\\
-1.29329329329329	1.12330667836883\\
-1.3013013013013	1.13021371766134\\
-1.30930930930931	1.1371649480479\\
-1.31731731731732	1.14416044872472\\
-1.32532532532533	1.15120029869773\\
-1.33333333333333	1.15828457677505\\
-1.34134134134134	1.16541336155934\\
-1.34934934934935	1.17258673144033\\
-1.35735735735736	1.1798047645872\\
-1.36536536536537	1.18706753894108\\
-1.37337337337337	1.19437513220755\\
-1.38138138138138	1.20172762184913\\
-1.38938938938939	1.20912508507784\\
-1.3973973973974	1.21656759884774\\
-1.40540540540541	1.22405523984751\\
-1.41341341341341	1.23158808449311\\
-1.42142142142142	1.23916620892037\\
-1.42942942942943	1.24678968897774\\
-1.43743743743744	1.25445860021895\\
-1.44544544544545	1.26217301789581\\
-1.45345345345345	1.26993301695099\\
-1.46146146146146	1.27773867201087\\
-1.46946946946947	1.28559005737843\\
-1.47747747747748	1.29348724702619\\
-1.48548548548549	1.30143031458918\\
-1.49349349349349	1.30941933335798\\
-1.5015015015015	1.31745437627181\\
-1.50950950950951	1.32553551591171\\
-1.51751751751752	1.33366282449368\\
-1.52552552552553	1.34183637386199\\
-1.53353353353353	1.35005623548249\\
-1.54154154154154	1.35832248043603\\
-1.54954954954955	1.36663517941186\\
-1.55755755755756	1.37499440270119\\
-1.56556556556557	1.38340022019079\\
-1.57357357357357	1.39185270135663\\
-1.58158158158158	1.40035191525766\\
-1.58958958958959	1.40889793052958\\
-1.5975975975976	1.41749081537877\\
-1.60560560560561	1.42613063757625\\
-1.61361361361361	1.43481746445174\\
-1.62162162162162	1.44355136288779\\
-1.62962962962963	1.45233239931402\\
-1.63763763763764	1.46116063970142\\
-1.64564564564565	1.47003614955675\\
-1.65365365365365	1.47895899391703\\
-1.66166166166166	1.48792923734414\\
-1.66966966966967	1.49694694391946\\
-1.67767767767768	1.50601217723866\\
-1.68568568568569	1.51512500040659\\
-1.69369369369369	1.5242854760322\\
-1.7017017017017	1.53349366622362\\
-1.70970970970971	1.54274963258335\\
-1.71771771771772	1.5520534362035\\
-1.72572572572573	1.56140513766116\\
-1.73373373373373	1.57080479701389\\
-1.74174174174174	1.5802524737953\\
-1.74974974974975	1.58974822701073\\
-1.75775775775776	1.59929211513303\\
-1.76576576576577	1.60888419609852\\
-1.77377377377377	1.61852452730293\\
-1.78178178178178	1.6282131655976\\
-1.78978978978979	1.63795016728568\\
-1.7977977977978	1.64773558811849\\
-1.80580580580581	1.65756948329197\\
-1.81381381381381	1.6674519074433\\
-1.82182182182182	1.67738291464757\\
-1.82982982982983	1.68736255841458\\
-1.83783783783784	1.6973908916858\\
-1.84584584584585	1.70746796683141\\
-1.85385385385385	1.71759383564743\\
-1.86186186186186	1.72776854935306\\
-1.86986986986987	1.73799215858803\\
-1.87787787787788	1.74826471341013\\
-1.88588588588589	1.75858626329288\\
-1.89389389389389	1.76895685712327\\
-1.9019019019019	1.77937654319961\\
-1.90990990990991	1.78984536922957\\
-1.91791791791792	1.80036338232829\\
-1.92592592592593	1.81093062901661\\
-1.93393393393393	1.82154715521939\\
-1.94194194194194	1.83221300626409\\
-1.94994994994995	1.84292822687926\\
-1.95795795795796	1.8536928611933\\
-1.96596596596597	1.86450695273334\\
-1.97397397397397	1.87537054442412\\
-1.98198198198198	1.88628367858711\\
-1.98998998998999	1.89724639693968\\
-1.997997997998	1.90825874059446\\
-2.00600600600601	1.91932075005871\\
-2.01401401401401	1.93043246523388\\
-2.02202202202202	1.94159392541532\\
-2.03003003003003	1.95280516929197\\
-2.03803803803804	1.96406623494635\\
-2.04604604604605	1.97537715985451\\
-2.05405405405405	1.98673798088615\\
-2.06206206206206	1.99814873430489\\
-2.07007007007007	2.00960945576858\\
-2.07807807807808	2.02112018032979\\
-2.08608608608609	2.03268094243636\\
-2.09409409409409	2.04429177593207\\
-2.1021021021021	2.05595271405747\\
-2.11011011011011	2.06766378945072\\
-2.11811811811812	2.0794250341486\\
-2.12612612612613	2.09123647958767\\
-2.13413413413413	2.1030981566054\\
-2.14214214214214	2.11501009544155\\
-2.15015015015015	2.12697232573954\\
-2.15815815815816	2.13898487654799\\
-2.16616616616617	2.1510477763223\\
-2.17417417417417	2.16316105292643\\
-2.18218218218218	2.17532473363463\\
-2.19019019019019	2.18753884513338\\
-2.1981981981982	2.19980341352339\\
-2.20620620620621	2.2121184643217\\
-2.21421421421421	2.22448402246382\\
-2.22222222222222	2.23690011230604\\
-2.23023023023023	2.24936675762775\\
-2.23823823823824	2.26188398163394\\
-2.24624624624625	2.27445180695766\\
-2.25425425425425	2.28707025566269\\
-2.26226226226226	2.29973934924621\\
-2.27027027027027	2.31245910864156\\
-2.27827827827828	2.32522955422113\\
-2.28628628628629	2.33805070579924\\
-2.29429429429429	2.3509225826352\\
-2.3023023023023	2.36384520343635\\
-2.31031031031031	2.37681858636121\\
-2.31831831831832	2.38984274902274\\
-2.32632632632633	2.40291770849159\\
-2.33433433433433	2.41604348129947\\
-2.34234234234234	2.42922008344257\\
-2.35035035035035	2.44244753038505\\
-2.35835835835836	2.45572583706259\\
-2.36636636636637	2.46905501788598\\
-2.37437437437437	2.48243508674482\\
-2.38238238238238	2.4958660570112\\
-2.39039039039039	2.50934794154352\\
-2.3983983983984	2.52288075269028\\
-2.40640640640641	2.536464502294\\
-2.41441441441441	2.55009920169513\\
-2.42242242242242	2.56378486173604\\
-2.43043043043043	2.57752149276504\\
-2.43843843843844	2.59130910464046\\
-2.44644644644645	2.60514770673476\\
-2.45445445445445	2.6190373079387\\
-2.46246246246246	2.6329779166655\\
-2.47047047047047	2.64696954085513\\
-2.47847847847848	2.66101218797851\\
-2.48648648648649	2.67510586504188\\
-2.49449449449449	2.68925057859112\\
-2.5025025025025	2.70344633471608\\
-2.51051051051051	2.71769313905505\\
-2.51851851851852	2.73199099679912\\
-2.52652652652653	2.74633991269667\\
-2.53453453453453	2.76073989105784\\
-2.54254254254254	2.77519093575905\\
-2.55055055055055	2.78969305024748\\
-2.55855855855856	2.80424623754567\\
-2.56656656656657	2.81885050025603\\
-2.57457457457457	2.83350584056548\\
-2.58258258258258	2.84821226024998\\
-2.59059059059059	2.86296976067919\\
-2.5985985985986	2.87777834282108\\
-2.60660660660661	2.89263800724655\\
-2.61461461461461	2.9075487541341\\
-2.62262262262262	2.92251058327445\\
-2.63063063063063	2.93752349407524\\
-2.63863863863864	2.95258748556566\\
-2.64664664664665	2.96770255640117\\
-2.65465465465465	2.98286870486809\\
-2.66266266266266	2.99808592888838\\
-2.67067067067067	3.01335422602423\\
-2.67867867867868	3.02867359348278\\
-2.68668668668669	3.04404402812078\\
-2.69469469469469	3.05946552644929\\
-2.7027027027027	3.07493808463828\\
-2.71071071071071	3.09046169852136\\
-2.71871871871872	3.10603636360041\\
-2.72672672672673	3.12166207505021\\
-2.73473473473473	3.13733882772309\\
-2.74274274274274	3.15306661615358\\
-2.75075075075075	3.168845434563\\
-2.75875875875876	3.18467527686406\\
-2.76676676676677	3.20055613666548\\
-2.77477477477477	3.21648800727654\\
-2.78278278278278	3.23247088171167\\
-2.79079079079079	3.24850475269498\\
-2.7987987987988	3.26458961266479\\
-2.80680680680681	3.28072545377816\\
-2.81481481481481	3.29691226791537\\
-2.82282282282282	3.31315004668438\\
-2.83083083083083	3.32943878142533\\
-2.83883883883884	3.34577846321493\\
-2.84684684684685	3.3621690828709\\
-2.85485485485485	3.37861063095634\\
-2.86286286286286	3.39510309778411\\
-2.87087087087087	3.41164647342118\\
-2.87887887887888	3.42824074769291\\
-2.88688688688689	3.44488591018741\\
-2.89489489489489	3.46158195025972\\
-2.9029029029029	3.47832885703613\\
-2.91091091091091	3.49512661941836\\
-2.91891891891892	3.51197522608772\\
-2.92692692692693	3.52887466550931\\
-2.93493493493493	3.54582492593614\\
-2.94294294294294	3.56282599541319\\
-2.95095095095095	3.57987786178154\\
-2.95895895895896	3.59698051268237\\
-2.96696696696697	3.61413393556095\\
-2.97497497497497	3.63133811767068\\
-2.98298298298298	3.64859304607699\\
-2.99099099099099	3.66589870766123\\
-2.998998998999	3.68325508912463\\
-3.00700700700701	3.70066217699207\\
-3.01501501501502	3.71811995761594\\
-3.02302302302302	3.7356284171799\\
-3.03103103103103	3.75318754170264\\
-3.03903903903904	3.77079731704161\\
-3.04704704704705	3.78845772889665\\
-3.05505505505506	3.8061687628137\\
-3.06306306306306	3.82393040418835\\
-3.07107107107107	3.84174263826948\\
-3.07907907907908	3.85960545016273\\
-3.08708708708709	3.87751882483407\\
-3.0950950950951	3.89548274711324\\
-3.1031031031031	3.91349720169716\\
-3.11111111111111	3.93156217315341\\
-3.11911911911912	3.94967764592349\\
-3.12712712712713	3.96784360432624\\
-3.13513513513514	3.98606003256105\\
-3.14314314314314	4.00432691471119\\
-3.15115115115115	4.02264423474698\\
-3.15915915915916	4.04101197652899\\
-3.16716716716717	4.05943012381119\\
-3.17517517517518	4.07789866024403\\
-3.18318318318318	4.09641756937758\\
-3.19119119119119	4.1149868346645\\
-3.1991991991992	4.13360643946307\\
-3.20720720720721	4.15227636704015\\
-3.21521521521522	4.17099660057414\\
-3.22322322322322	4.18976712315781\\
-3.23123123123123	4.20858791780122\\
-3.23923923923924	4.2274589674345\\
-3.24724724724725	4.24638025491065\\
-3.25525525525526	4.26535176300827\\
-3.26326326326326	4.28437347443433\\
-3.27127127127127	4.30344537182677\\
-3.27927927927928	4.3225674377572\\
-3.28728728728729	4.34173965473346\\
-3.2952952952953	4.36096200520223\\
-3.3033033033033	4.38023447155156\\
-3.31131131131131	4.39955703611335\\
-3.31931931931932	4.41892968116582\\
-3.32732732732733	4.43835238893596\\
-3.33533533533534	4.45782514160191\\
-3.34334334334334	4.47734792129531\\
-3.35135135135135	4.49692071010367\\
-3.35935935935936	4.51654349007261\\
-3.36736736736737	4.53621624320817\\
-3.37537537537538	4.55593895147899\\
-3.38338338338338	4.57571159681853\\
-3.39139139139139	4.59553416112723\\
-3.3993993993994	4.61540662627462\\
-3.40740740740741	4.63532897410144\\
-3.41541541541542	4.65530118642165\\
-3.42342342342342	4.67532324502452\\
-3.43143143143143	4.6953951316766\\
-3.43943943943944	4.71551682812366\\
-3.44744744744745	4.73568831609265\\
-3.45545545545546	4.75590957729362\\
-3.46346346346346	4.77618059342153\\
-3.47147147147147	4.79650134615815\\
-3.47947947947948	4.81687181717384\\
-3.48748748748749	4.83729198812932\\
-3.4954954954955	4.85776184067744\\
-3.5035035035035	4.87828135646488\\
-3.51151151151151	4.89885051713386\\
-3.51951951951952	4.91946930432375\\
-3.52752752752753	4.94013769967278\\
-3.53553553553554	4.96085568481957\\
-3.54354354354354	4.98162324140473\\
-3.55155155155155	5.00244035107239\\
-3.55955955955956	5.02330699547177\\
-3.56756756756757	5.04422315625857\\
-3.57557557557558	5.06518881509654\\
-3.58358358358358	5.08620395365883\\
-3.59159159159159	5.10726855362943\\
-3.5995995995996	5.12838259670458\\
-3.60760760760761	5.14954606459406\\
-3.61561561561562	5.17075893902256\\
-3.62362362362362	5.19202120173098\\
-3.63163163163163	5.21333283447771\\
-3.63963963963964	5.23469381903985\\
-3.64764764764765	5.25610413721448\\
-3.65565565565566	5.27756377081981\\
-3.66366366366366	5.29907270169644\\
-3.67167167167167	5.32063091170842\\
-3.67967967967968	5.34223838274443\\
-3.68768768768769	5.3638950967189\\
-3.6956956956957	5.38560103557305\\
-3.7037037037037	5.40735618127599\\
-3.71171171171171	5.42916051582572\\
-3.71971971971972	5.45101402125018\\
-3.72772772772773	5.47291667960822\\
-3.73573573573574	5.4948684729906\\
-3.74374374374374	5.5168693835209\\
-3.75175175175175	5.53891939335647\\
-3.75975975975976	5.56101848468933\\
-3.76776776776777	5.58316663974709\\
-3.77577577577578	5.60536384079376\\
-3.78378378378378	5.62761007013064\\
-3.79179179179179	5.64990531009712\\
-3.7997997997998	5.67224954307151\\
-3.80780780780781	5.69464275147181\\
-3.81581581581582	5.71708491775649\\
-3.82382382382382	5.73957602442522\\
-3.83183183183183	5.76211605401961\\
-3.83983983983984	5.78470498912395\\
-3.84784784784785	5.80734281236586\\
-3.85585585585586	5.830029506417\\
-3.86386386386386	5.85276505399369\\
-3.87187187187187	5.87554943785761\\
-3.87987987987988	5.89838264081637\\
-3.88788788788789	5.92126464572416\\
-3.8958958958959	5.94419543548229\\
-3.9039039039039	5.96717499303982\\
-3.91191191191191	5.99020330139409\\
-3.91991991991992	6.01328034359127\\
-3.92792792792793	6.03640610272688\\
-3.93593593593594	6.05958056194631\\
-3.94394394394394	6.08280370444532\\
-3.95195195195195	6.1060755134705\\
-3.95995995995996	6.12939597231977\\
-3.96796796796797	6.15276506434282\\
-3.97597597597598	6.17618277294154\\
-3.98398398398398	6.19964908157044\\
-3.99199199199199	6.22316397373711\\
-4	6.24672743300253\\
}--cycle;

\addlegendentry{$\pm 2\sigma$};

\addplot [color=mycolor2,solid]
  table[row sep=crcr]{%
-4	3.91931550532608\\
-3.99199199199199	3.90226361418649\\
-3.98398398398398	3.88524428229354\\
-3.97597597597598	3.86825750964723\\
-3.96796796796797	3.85130329624756\\
-3.95995995995996	3.83438164209453\\
-3.95195195195195	3.81749254718815\\
-3.94394394394394	3.8006360115284\\
-3.93593593593594	3.78381203511529\\
-3.92792792792793	3.76702061794882\\
-3.91991991991992	3.75026176002899\\
-3.91191191191191	3.7335354613558\\
-3.9039039039039	3.71684172192925\\
-3.8958958958959	3.70018054174935\\
-3.88788788788789	3.68355192081608\\
-3.87987987987988	3.66695585912945\\
-3.87187187187187	3.65039235668946\\
-3.86386386386386	3.63386141349611\\
-3.85585585585586	3.61736302954941\\
-3.84784784784785	3.60089720484934\\
-3.83983983983984	3.58446393939591\\
-3.83183183183183	3.56806323318912\\
-3.82382382382382	3.55169508622898\\
-3.81581581581582	3.53535949851547\\
-3.80780780780781	3.5190564700486\\
-3.7997997997998	3.50278600082837\\
-3.79179179179179	3.48654809085479\\
-3.78378378378378	3.47034274012784\\
-3.77577577577578	3.45416994864753\\
-3.76776776776777	3.43802971641387\\
-3.75975975975976	3.42192204342684\\
-3.75175175175175	3.40584692968645\\
-3.74374374374374	3.38980437519271\\
-3.73573573573574	3.3737943799456\\
-3.72772772772773	3.35781694394513\\
-3.71971971971972	3.34187206719131\\
-3.71171171171171	3.32595974968412\\
-3.7037037037037	3.31007999142357\\
-3.6956956956957	3.29423279240967\\
-3.68768768768769	3.2784181526424\\
-3.67967967967968	3.26263607212178\\
-3.67167167167167	3.24688655084779\\
-3.66366366366366	3.23116958882044\\
-3.65565565565566	3.21548518603974\\
-3.64764764764765	3.19983334250567\\
-3.63963963963964	3.18421405821825\\
-3.63163163163163	3.16862733317746\\
-3.62362362362362	3.15307316738332\\
-3.61561561561562	3.13755156083581\\
-3.60760760760761	3.12206251353495\\
-3.5995995995996	3.10660602548072\\
-3.59159159159159	3.09118209667314\\
-3.58358358358358	3.07579072711219\\
-3.57557557557558	3.06043191679789\\
-3.56756756756757	3.04510566573022\\
-3.55955955955956	3.0298119739092\\
-3.55155155155155	3.01455084133481\\
-3.54354354354354	2.99932226800707\\
-3.53553553553554	2.98412625392596\\
-3.52752752752753	2.9689627990915\\
-3.51951951951952	2.95383190350368\\
-3.51151151151151	2.93873356716249\\
-3.5035035035035	2.92366779006795\\
-3.4954954954955	2.90863457222004\\
-3.48748748748749	2.89363391361878\\
-3.47947947947948	2.87866581426416\\
-3.47147147147147	2.86373027415617\\
-3.46346346346346	2.84882729329483\\
-3.45545545545546	2.83395687168012\\
-3.44744744744745	2.81911900931206\\
-3.43943943943944	2.80431370619064\\
-3.43143143143143	2.78954096231585\\
-3.42342342342342	2.77480077768771\\
-3.41541541541542	2.76009315230621\\
-3.40740740740741	2.74541808617135\\
-3.3993993993994	2.73077557928312\\
-3.39139139139139	2.71616563164154\\
-3.38338338338338	2.7015882432466\\
-3.37537537537538	2.68704341409829\\
-3.36736736736737	2.67253114419663\\
-3.35935935935936	2.65805143354161\\
-3.35135135135135	2.64360428213323\\
-3.34334334334334	2.62918968997148\\
-3.33533533533534	2.61480765705638\\
-3.32732732732733	2.60045818338792\\
-3.31931931931932	2.5861412689661\\
-3.31131131131131	2.57185691379092\\
-3.3033033033033	2.55760511786237\\
-3.2952952952953	2.54338588118047\\
-3.28728728728729	2.52919920374521\\
-3.27927927927928	2.51504508555659\\
-3.27127127127127	2.50092352661461\\
-3.26326326326326	2.48683452691927\\
-3.25525525525526	2.47277808647056\\
-3.24724724724725	2.4587542052685\\
-3.23923923923924	2.44476288331308\\
-3.23123123123123	2.4308041206043\\
-3.22322322322322	2.41687791714216\\
-3.21521521521522	2.40298427292666\\
-3.20720720720721	2.3891231879578\\
-3.1991991991992	2.37529466223558\\
-3.19119119119119	2.36149869576\\
-3.18318318318318	2.34773528853106\\
-3.17517517517518	2.33400444054876\\
-3.16716716716717	2.3203061518131\\
-3.15915915915916	2.30664042232407\\
-3.15115115115115	2.29300725208169\\
-3.14314314314314	2.27940664108595\\
-3.13513513513514	2.26583858933685\\
-3.12712712712713	2.25230309683439\\
-3.11911911911912	2.23880016357857\\
-3.11111111111111	2.22532978956939\\
-3.1031031031031	2.21189197480685\\
-3.0950950950951	2.19848671929096\\
-3.08708708708709	2.1851140230217\\
-3.07907907907908	2.17177388599908\\
-3.07107107107107	2.1584663082231\\
-3.06306306306306	2.14519128969376\\
-3.05505505505506	2.13194883041106\\
-3.04704704704705	2.118738930375\\
-3.03903903903904	2.10556158958558\\
-3.03103103103103	2.0924168080428\\
-3.02302302302302	2.07930458574666\\
-3.01501501501502	2.06622492269716\\
-3.00700700700701	2.0531778188943\\
-2.998998998999	2.04016327433809\\
-2.99099099099099	2.02718128902851\\
-2.98298298298298	2.01423186296557\\
-2.97497497497497	2.00131499614927\\
-2.96696696696697	1.98843068857961\\
-2.95895895895896	1.97557894025659\\
-2.95095095095095	1.96275975118022\\
-2.94294294294294	1.94997312135048\\
-2.93493493493493	1.93721905076738\\
-2.92692692692693	1.92449753943092\\
-2.91891891891892	1.9118085873411\\
-2.91091091091091	1.89915219449793\\
-2.9029029029029	1.88652836090139\\
-2.89489489489489	1.87393708655149\\
-2.88688688688689	1.86137837144823\\
-2.87887887887888	1.84885221559162\\
-2.87087087087087	1.83635861898164\\
-2.86286286286286	1.8238975816183\\
-2.85485485485485	1.8114691035016\\
-2.84684684684685	1.79907318463155\\
-2.83883883883884	1.78670982500813\\
-2.83083083083083	1.77437902463135\\
-2.82282282282282	1.76208078350122\\
-2.81481481481481	1.74981510161772\\
-2.80680680680681	1.73758197898086\\
-2.7987987987988	1.72538141559065\\
-2.79079079079079	1.71321341144707\\
-2.78278278278278	1.70107796655013\\
-2.77477477477477	1.68897508089984\\
-2.76676676676677	1.67690475449618\\
-2.75875875875876	1.66486698733917\\
-2.75075075075075	1.65286177942879\\
-2.74274274274274	1.64088913076505\\
-2.73473473473473	1.62894904134796\\
-2.72672672672673	1.6170415111775\\
-2.71871871871872	1.60516654025369\\
-2.71071071071071	1.59332412857651\\
-2.7027027027027	1.58151427614598\\
-2.69469469469469	1.56973698296208\\
-2.68668668668669	1.55799224902483\\
-2.67867867867868	1.54628007433421\\
-2.67067067067067	1.53460045889023\\
-2.66266266266266	1.5229534026929\\
-2.65465465465465	1.51133890574221\\
-2.64664664664665	1.49975696803815\\
-2.63863863863864	1.48820758958074\\
-2.63063063063063	1.47669077036996\\
-2.62262262262262	1.46520651040583\\
-2.61461461461461	1.45375480968833\\
-2.60660660660661	1.44233566821748\\
-2.5985985985986	1.43094908599326\\
-2.59059059059059	1.41959506301569\\
-2.58258258258258	1.40827359928475\\
-2.57457457457457	1.39698469480046\\
-2.56656656656657	1.38572834956281\\
-2.55855855855856	1.37450456357179\\
-2.55055055055055	1.36331333682742\\
-2.54254254254254	1.35215466932968\\
-2.53453453453453	1.34102856107859\\
-2.52652652652653	1.32993501207414\\
-2.51851851851852	1.31887402231632\\
-2.51051051051051	1.30784559180515\\
-2.5025025025025	1.29684972054062\\
-2.49449449449449	1.28588640852272\\
-2.48648648648649	1.27495565575147\\
-2.47847847847848	1.26405746222686\\
-2.47047047047047	1.25319182794888\\
-2.46246246246246	1.24235875291755\\
-2.45445445445445	1.23155823713286\\
-2.44644644644645	1.22079028059481\\
-2.43843843843844	1.21005488330339\\
-2.43043043043043	1.19935204525862\\
-2.42242242242242	1.18868176646049\\
-2.41441441441441	1.178044046909\\
-2.40640640640641	1.16743888660414\\
-2.3983983983984	1.15686628554593\\
-2.39039039039039	1.14632624373436\\
-2.38238238238238	1.13581876116943\\
-2.37437437437437	1.12534383785114\\
-2.36636636636637	1.11490147377948\\
-2.35835835835836	1.10449166895447\\
-2.35035035035035	1.0941144233761\\
-2.34234234234234	1.08376973704437\\
-2.33433433433433	1.07345760995928\\
-2.32632632632633	1.06317804212083\\
-2.31831831831832	1.05293103352901\\
-2.31031031031031	1.04271658418384\\
-2.3023023023023	1.03253469408531\\
-2.29429429429429	1.02238536323342\\
-2.28628628628629	1.01226859162817\\
-2.27827827827828	1.00218437926956\\
-2.27027027027027	0.992132726157588\\
-2.26226226226226	0.982113632292257\\
-2.25425425425425	0.972127097673567\\
-2.24624624624625	0.962173122301516\\
-2.23823823823824	0.952251706176105\\
-2.23023023023023	0.942362849297335\\
-2.22222222222222	0.932506551665205\\
-2.21421421421421	0.922682813279715\\
-2.20620620620621	0.912891634140864\\
-2.1981981981982	0.903133014248654\\
-2.19019019019019	0.893406953603084\\
-2.18218218218218	0.883713452204154\\
-2.17417417417417	0.874052510051865\\
-2.16616616616617	0.864424127146214\\
-2.15815815815816	0.854828303487204\\
-2.15015015015015	0.845265039074835\\
-2.14214214214214	0.835734333909106\\
-2.13413413413413	0.826236187990017\\
-2.12612612612613	0.816770601317567\\
-2.11811811811812	0.807337573891758\\
-2.11011011011011	0.797937105712589\\
-2.1021021021021	0.78856919678006\\
-2.09409409409409	0.779233847094171\\
-2.08608608608609	0.769931056654922\\
-2.07807807807808	0.760660825462313\\
-2.07007007007007	0.751423153516344\\
-2.06206206206206	0.742218040817016\\
-2.05405405405405	0.733045487364327\\
-2.04604604604605	0.723905493158279\\
-2.03803803803804	0.71479805819887\\
-2.03003003003003	0.705723182486102\\
-2.02202202202202	0.696680866019974\\
-2.01401401401401	0.687671108800486\\
-2.00600600600601	0.678693910827638\\
-1.997997997998	0.66974927210143\\
-1.98998998998999	0.660837192621862\\
-1.98198198198198	0.651957672388935\\
-1.97397397397397	0.643110711402647\\
-1.96596596596597	0.634296309662999\\
-1.95795795795796	0.625514467169991\\
-1.94994994994995	0.616765183923624\\
-1.94194194194194	0.608048459923897\\
-1.93393393393393	0.599364295170809\\
-1.92592592592593	0.590712689664362\\
-1.91791791791792	0.582093643404555\\
-1.90990990990991	0.573507156391388\\
-1.9019019019019	0.564953228624862\\
-1.89389389389389	0.556431860104974\\
-1.88588588588589	0.547943050831728\\
-1.87787787787788	0.539486800805121\\
-1.86986986986987	0.531063110025154\\
-1.86186186186186	0.522671978491828\\
-1.85385385385385	0.514313406205142\\
-1.84584584584585	0.505987393165095\\
-1.83783783783784	0.497693939371689\\
-1.82982982982983	0.489433044824923\\
-1.82182182182182	0.481204709524797\\
-1.81381381381381	0.473008933471311\\
-1.80580580580581	0.464845716664465\\
-1.7977977977978	0.456715059104259\\
-1.78978978978979	0.448616960790694\\
-1.78178178178178	0.440551421723768\\
-1.77377377377377	0.432518441903482\\
-1.76576576576577	0.424518021329837\\
-1.75775775775776	0.416550160002832\\
-1.74974974974975	0.408614857922466\\
-1.74174174174174	0.400712115088741\\
-1.73373373373373	0.392841931501656\\
-1.72572572572573	0.385004307161211\\
-1.71771771771772	0.377199242067406\\
-1.70970970970971	0.369426736220241\\
-1.7017017017017	0.361686789619716\\
-1.69369369369369	0.353979402265832\\
-1.68568568568569	0.346304574158587\\
-1.67767767767768	0.338662305297983\\
-1.66966966966967	0.331052595684018\\
-1.66166166166166	0.323475445316694\\
-1.65365365365365	0.31593085419601\\
-1.64564564564565	0.308418822321966\\
-1.63763763763764	0.300939349694561\\
-1.62962962962963	0.293492436313798\\
-1.62162162162162	0.286078082179673\\
-1.61361361361361	0.27869628729219\\
-1.60560560560561	0.271347051651346\\
-1.5975975975976	0.264030375257142\\
-1.58958958958959	0.256746258109579\\
-1.58158158158158	0.249494700208655\\
-1.57357357357357	0.242275701554372\\
-1.56556556556557	0.235089262146728\\
-1.55755755755756	0.227935381985725\\
-1.54954954954955	0.220814061071362\\
-1.54154154154154	0.213725299403639\\
-1.53353353353353	0.206669096982556\\
-1.52552552552553	0.199645453808113\\
-1.51751751751752	0.19265436988031\\
-1.50950950950951	0.185695845199148\\
-1.5015015015015	0.178769879764625\\
-1.49349349349349	0.171876473576743\\
-1.48548548548549	0.1650156266355\\
-1.47747747747748	0.158187338940898\\
-1.46946946946947	0.151391610492936\\
-1.46146146146146	0.144628441291613\\
-1.45345345345345	0.137897831336931\\
-1.44544544544545	0.131199780628889\\
-1.43743743743744	0.124534289167487\\
-1.42942942942943	0.117901356952725\\
-1.42142142142142	0.111300983984603\\
-1.41341341341341	0.104733170263122\\
-1.40540540540541	0.09819791578828\\
-1.3973973973974	0.0916952205600787\\
-1.38938938938939	0.0852250845785174\\
-1.38138138138138	0.0787875078435958\\
-1.37337337337337	0.0723824903553148\\
-1.36536536536537	0.0660100321136734\\
-1.35735735735736	0.0596701331186724\\
-1.34934934934935	0.0533627933703112\\
-1.34134134134134	0.0470880128685904\\
-1.33333333333333	0.0408457916135097\\
-1.32532532532533	0.0346361296050688\\
-1.31731731731732	0.0284590268432683\\
-1.30930930930931	0.0223144833281075\\
-1.3013013013013	0.0162024990595872\\
-1.29329329329329	0.0101230740377068\\
-1.28528528528529	0.00407620826246635\\
-1.27727727727728	-0.00193809826613367\\
-1.26926926926927	-0.007919845548094\\
-1.26126126126126	-0.013869033583414\\
-1.25325325325325	-0.0197856623720938\\
-1.24524524524525	-0.0256697319141339\\
-1.23723723723724	-0.0315212422095336\\
-1.22922922922923	-0.0373401932582935\\
-1.22122122122122	-0.0431265850604129\\
-1.21321321321321	-0.0488804176158924\\
-1.20520520520521	-0.054601690924732\\
-1.1971971971972	-0.0602904049869313\\
-1.18918918918919	-0.0659465598024908\\
-1.18118118118118	-0.0715701553714099\\
-1.17317317317317	-0.0771611916936888\\
-1.16516516516517	-0.0827196687693281\\
-1.15715715715716	-0.0882455865983268\\
-1.14914914914915	-0.0937389451806859\\
-1.14114114114114	-0.0991997445164046\\
-1.13313313313313	-0.104627984605483\\
-1.12512512512513	-0.110023665447922\\
-1.11711711711712	-0.11538678704372\\
-1.10910910910911	-0.120717349392879\\
-1.1011011011011	-0.126015352495397\\
-1.09309309309309	-0.131280796351276\\
-1.08508508508509	-0.136513680960514\\
-1.07707707707708	-0.141714006323112\\
-1.06906906906907	-0.14688177243907\\
-1.06106106106106	-0.152016979308388\\
-1.05305305305305	-0.157119626931066\\
-1.04504504504505	-0.162189715307103\\
-1.03703703703704	-0.167227244436501\\
-1.02902902902903	-0.172232214319258\\
-1.02102102102102	-0.177204624955376\\
-1.01301301301301	-0.182144476344853\\
-1.00500500500501	-0.187051768487691\\
-0.996996996996997	-0.191926501383888\\
-0.988988988988989	-0.196768675033445\\
-0.980980980980981	-0.201578289436362\\
-0.972972972972973	-0.206355344592639\\
-0.964964964964965	-0.211099840502276\\
-0.956956956956957	-0.215811777165273\\
-0.948948948948949	-0.220491154581629\\
-0.940940940940941	-0.225137972751346\\
-0.932932932932933	-0.229752231674423\\
-0.924924924924925	-0.234333931350859\\
-0.916916916916917	-0.238883071780655\\
-0.908908908908909	-0.243399652963812\\
-0.900900900900901	-0.247883674900328\\
-0.892892892892893	-0.252335137590204\\
-0.884884884884885	-0.25675404103344\\
-0.876876876876877	-0.261140385230036\\
-0.868868868868869	-0.265494170179992\\
-0.860860860860861	-0.269815395883308\\
-0.852852852852853	-0.274104062339983\\
-0.844844844844845	-0.278360169550019\\
-0.836836836836837	-0.282583717513415\\
-0.828828828828829	-0.28677470623017\\
-0.820820820820821	-0.290933135700285\\
-0.812812812812813	-0.295059005923761\\
-0.804804804804805	-0.299152316900596\\
-0.796796796796797	-0.303213068630791\\
-0.788788788788789	-0.307241261114346\\
-0.780780780780781	-0.311236894351261\\
-0.772772772772773	-0.315199968341536\\
-0.764764764764765	-0.31913048308517\\
-0.756756756756757	-0.323028438582165\\
-0.748748748748749	-0.32689383483252\\
-0.740740740740741	-0.330726671836234\\
-0.732732732732733	-0.334526949593309\\
-0.724724724724725	-0.338294668103743\\
-0.716716716716717	-0.342029827367537\\
-0.708708708708709	-0.345732427384691\\
-0.700700700700701	-0.349402468155205\\
-0.692692692692693	-0.353039949679079\\
-0.684684684684685	-0.356644871956313\\
-0.676676676676677	-0.360217234986907\\
-0.668668668668669	-0.363757038770861\\
-0.660660660660661	-0.367264283308174\\
-0.652652652652653	-0.370738968598848\\
-0.644644644644645	-0.374181094642881\\
-0.636636636636636	-0.377590661440275\\
-0.628628628628629	-0.380967668991028\\
-0.62062062062062	-0.384312117295141\\
-0.612612612612613	-0.387624006352614\\
-0.604604604604605	-0.390903336163447\\
-0.596596596596596	-0.39415010672764\\
-0.588588588588589	-0.397364318045193\\
-0.58058058058058	-0.400545970116106\\
-0.572572572572573	-0.403695062940379\\
-0.564564564564565	-0.406811596518011\\
-0.556556556556556	-0.409895570849004\\
-0.548548548548549	-0.412946985933356\\
-0.54054054054054	-0.415965841771069\\
-0.532532532532533	-0.418952138362141\\
-0.524524524524525	-0.421905875706573\\
-0.516516516516516	-0.424827053804365\\
-0.508508508508509	-0.427715672655517\\
-0.5005005005005	-0.430571732260029\\
-0.492492492492492	-0.433395232617901\\
-0.484484484484485	-0.436186173729133\\
-0.476476476476476	-0.438944555593724\\
-0.468468468468469	-0.441670378211676\\
-0.46046046046046	-0.444363641582987\\
-0.452452452452452	-0.447024345707659\\
-0.444444444444445	-0.44965249058569\\
-0.436436436436436	-0.452248076217081\\
-0.428428428428429	-0.454811102601832\\
-0.42042042042042	-0.457341569739944\\
-0.412412412412412	-0.459839477631415\\
-0.404404404404405	-0.462304826276245\\
-0.396396396396396	-0.464737615674436\\
-0.388388388388389	-0.467137845825987\\
-0.38038038038038	-0.469505516730898\\
-0.372372372372372	-0.471840628389168\\
-0.364364364364364	-0.474143180800799\\
-0.356356356356356	-0.476413173965789\\
-0.348348348348348	-0.478650607884139\\
-0.34034034034034	-0.48085548255585\\
-0.332332332332332	-0.48302779798092\\
-0.324324324324324	-0.48516755415935\\
-0.316316316316316	-0.48727475109114\\
-0.308308308308308	-0.489349388776289\\
-0.3003003003003	-0.491391467214799\\
-0.292292292292292	-0.493400986406669\\
-0.284284284284284	-0.495377946351899\\
-0.276276276276276	-0.497322347050488\\
-0.268268268268268	-0.499234188502437\\
-0.26026026026026	-0.501113470707747\\
-0.252252252252252	-0.502960193666416\\
-0.244244244244244	-0.504774357378445\\
-0.236236236236236	-0.506555961843834\\
-0.228228228228228	-0.508305007062583\\
-0.22022022022022	-0.510021493034692\\
-0.212212212212212	-0.511705419760161\\
-0.204204204204204	-0.51335678723899\\
-0.196196196196196	-0.514975595471179\\
-0.188188188188188	-0.516561844456727\\
-0.18018018018018	-0.518115534195636\\
-0.172172172172172	-0.519636664687904\\
-0.164164164164164	-0.521125235933532\\
-0.156156156156156	-0.522581247932521\\
-0.148148148148148	-0.524004700684869\\
-0.14014014014014	-0.525395594190577\\
-0.132132132132132	-0.526753928449645\\
-0.124124124124124	-0.528079703462073\\
-0.116116116116116	-0.529372919227861\\
-0.108108108108108	-0.530633575747008\\
-0.1001001001001	-0.531861673019516\\
-0.0920920920920922	-0.533057211045384\\
-0.084084084084084	-0.534220189824611\\
-0.0760760760760761	-0.535350609357198\\
-0.0680680680680679	-0.536448469643146\\
-0.06006006006006	-0.537513770682453\\
-0.0520520520520522	-0.53854651247512\\
-0.0440440440440439	-0.539546695021147\\
-0.0360360360360361	-0.540514318320534\\
-0.0280280280280278	-0.541449382373281\\
-0.02002002002002	-0.542351887179388\\
-0.0120120120120122	-0.543221832738854\\
-0.00400400400400391	-0.544059219051681\\
0.00400400400400436	-0.544864046117868\\
0.0120120120120122	-0.545636313937414\\
0.02002002002002	-0.54637602251032\\
0.0280280280280278	-0.547083171836587\\
0.0360360360360357	-0.547757761916213\\
0.0440440440440444	-0.548399792749199\\
0.0520520520520522	-0.549009264335545\\
0.06006006006006	-0.549586176675251\\
0.0680680680680679	-0.550130529768317\\
0.0760760760760757	-0.550642323614742\\
0.0840840840840844	-0.551121558214528\\
0.0920920920920922	-0.551568233567674\\
0.1001001001001	-0.551982349674179\\
0.108108108108108	-0.552363906534045\\
0.116116116116116	-0.55271290414727\\
0.124124124124124	-0.553029342513855\\
0.132132132132132	-0.5533132216338\\
0.14014014014014	-0.553564541507105\\
0.148148148148148	-0.55378330213377\\
0.156156156156156	-0.553969503513795\\
0.164164164164164	-0.55412314564718\\
0.172172172172172	-0.554244228533925\\
0.18018018018018	-0.554332752174029\\
0.188188188188188	-0.554388716567494\\
0.196196196196196	-0.554412121714318\\
0.204204204204204	-0.554402967614503\\
0.212212212212212	-0.554361254268047\\
0.22022022022022	-0.554286981674951\\
0.228228228228228	-0.554180149835215\\
0.236236236236236	-0.55404075874884\\
0.244244244244245	-0.553868808415824\\
0.252252252252252	-0.553664298836167\\
0.26026026026026	-0.553427230009871\\
0.268268268268268	-0.553157601936935\\
0.276276276276277	-0.552855414617358\\
0.284284284284285	-0.552520668051142\\
0.292292292292292	-0.552153362238285\\
0.3003003003003	-0.551753497178789\\
0.308308308308308	-0.551321072872652\\
0.316316316316317	-0.550856089319875\\
0.324324324324325	-0.550358546520458\\
0.332332332332332	-0.549828444474401\\
0.34034034034034	-0.549265783181704\\
0.348348348348348	-0.548670562642367\\
0.356356356356357	-0.54804278285639\\
0.364364364364365	-0.547382443823773\\
0.372372372372372	-0.546689545544515\\
0.38038038038038	-0.545964088018618\\
0.388388388388388	-0.54520607124608\\
0.396396396396397	-0.544415495226902\\
0.404404404404405	-0.543592359961085\\
0.412412412412412	-0.542736665448627\\
0.42042042042042	-0.541848411689529\\
0.428428428428428	-0.540927598683791\\
0.436436436436437	-0.539974226431413\\
0.444444444444445	-0.538988294932395\\
0.452452452452452	-0.537969804186736\\
0.46046046046046	-0.536918754194438\\
0.468468468468468	-0.5358351449555\\
0.476476476476477	-0.534718976469921\\
0.484484484484485	-0.533570248737702\\
0.492492492492492	-0.532388961758844\\
0.5005005005005	-0.531175115533345\\
0.508508508508508	-0.529928710061206\\
0.516516516516517	-0.528649745342427\\
0.524524524524525	-0.527338221377008\\
0.532532532532533	-0.525994138164949\\
0.54054054054054	-0.52461749570625\\
0.548548548548548	-0.52320829400091\\
0.556556556556557	-0.521766533048931\\
0.564564564564565	-0.520292212850312\\
0.572572572572573	-0.518785333405052\\
0.58058058058058	-0.517245894713152\\
0.588588588588588	-0.515673896774613\\
0.596596596596597	-0.514069339589433\\
0.604604604604605	-0.512432223157613\\
0.612612612612613	-0.510762547479153\\
0.62062062062062	-0.509060312554053\\
0.628628628628628	-0.507325518382313\\
0.636636636636637	-0.505558164963932\\
0.644644644644645	-0.503758252298912\\
0.652652652652653	-0.501925780387252\\
0.66066066066066	-0.500060749228951\\
0.668668668668668	-0.498163158824011\\
0.676676676676677	-0.49623300917243\\
0.684684684684685	-0.494270300274209\\
0.692692692692693	-0.492275032129348\\
0.7007007007007	-0.490247204737848\\
0.708708708708708	-0.488186818099707\\
0.716716716716717	-0.486093872214925\\
0.724724724724725	-0.483968367083504\\
0.732732732732733	-0.481810302705443\\
0.74074074074074	-0.479619679080742\\
0.748748748748748	-0.4773964962094\\
0.756756756756757	-0.475140754091419\\
0.764764764764765	-0.472852452726797\\
0.772772772772773	-0.470531592115535\\
0.780780780780781	-0.468178172257634\\
0.788788788788789	-0.465792193153092\\
0.796796796796797	-0.46337365480191\\
0.804804804804805	-0.460922557204088\\
0.812812812812813	-0.458438900359626\\
0.820820820820821	-0.455922684268523\\
0.828828828828829	-0.453373908930781\\
0.836836836836837	-0.450792574346399\\
0.844844844844845	-0.448178680515376\\
0.852852852852853	-0.445532227437714\\
0.860860860860861	-0.442853215113411\\
0.868868868868869	-0.440141643542468\\
0.876876876876877	-0.437397512724885\\
0.884884884884885	-0.434620822660663\\
0.892892892892893	-0.4318115733498\\
0.900900900900901	-0.428969764792297\\
0.908908908908909	-0.426095396988153\\
0.916916916916917	-0.42318846993737\\
0.924924924924925	-0.420248983639947\\
0.932932932932933	-0.417276938095884\\
0.940940940940941	-0.41427233330518\\
0.948948948948949	-0.411235169267836\\
0.956956956956957	-0.408165445983853\\
0.964964964964965	-0.405063163453229\\
0.972972972972973	-0.401928321675965\\
0.980980980980981	-0.398760920652061\\
0.988988988988989	-0.395560960381517\\
0.996996996996997	-0.392328440864333\\
1.00500500500501	-0.389063362100509\\
1.01301301301301	-0.385765724090045\\
1.02102102102102	-0.382435526832941\\
1.02902902902903	-0.379072770329196\\
1.03703703703704	-0.375677454578811\\
1.04504504504505	-0.372249579581787\\
1.05305305305305	-0.368789145338122\\
1.06106106106106	-0.365296151847817\\
1.06906906906907	-0.361770599110872\\
1.07707707707708	-0.358212487127287\\
1.08508508508509	-0.354621815897063\\
1.09309309309309	-0.350998585420197\\
1.1011011011011	-0.347342795696692\\
1.10910910910911	-0.343654446726547\\
1.11711711711712	-0.339933538509761\\
1.12512512512513	-0.336180071046336\\
1.13313313313313	-0.332394044336271\\
1.14114114114114	-0.328575458379565\\
1.14914914914915	-0.324724313176219\\
1.15715715715716	-0.320840608726233\\
1.16516516516517	-0.316924345029607\\
1.17317317317317	-0.312975522086342\\
1.18118118118118	-0.308994139896436\\
1.18918918918919	-0.304980198459889\\
1.1971971971972	-0.300933697776703\\
1.20520520520521	-0.296854637846877\\
1.21321321321321	-0.29274301867041\\
1.22122122122122	-0.288598840247304\\
1.22922922922923	-0.284422102577557\\
1.23723723723724	-0.280212805661171\\
1.24524524524525	-0.275970949498144\\
1.25325325325325	-0.271696534088477\\
1.26126126126126	-0.26738955943217\\
1.26926926926927	-0.263050025529223\\
1.27727727727728	-0.258677932379636\\
1.28528528528529	-0.254273279983409\\
1.29329329329329	-0.249836068340542\\
1.3013013013013	-0.245366297451034\\
1.30930930930931	-0.240863967314887\\
1.31731731731732	-0.236329077932099\\
1.32532532532533	-0.231761629302672\\
1.33333333333333	-0.227161621426604\\
1.34134134134134	-0.222529054303896\\
1.34934934934935	-0.217863927934548\\
1.35735735735736	-0.21316624231856\\
1.36536536536537	-0.208435997455932\\
1.37337337337337	-0.203673193346664\\
1.38138138138138	-0.198877829990756\\
1.38938938938939	-0.194049907388208\\
1.3973973973974	-0.189189425539019\\
1.40540540540541	-0.184296384443191\\
1.41341341341341	-0.179370784100723\\
1.42142142142142	-0.174412624511614\\
1.42942942942943	-0.169421905675865\\
1.43743743743744	-0.164398627593476\\
1.44544544544545	-0.159342790264447\\
1.45345345345345	-0.154254393688779\\
1.46146146146146	-0.149133437866469\\
1.46946946946947	-0.14397992279752\\
1.47747747747748	-0.138793848481931\\
1.48548548548549	-0.133575214919702\\
1.49349349349349	-0.128324022110832\\
1.5015015015015	-0.123040270055323\\
1.50950950950951	-0.117723958753173\\
1.51751751751752	-0.112375088204384\\
1.52552552552553	-0.106993658408954\\
1.53353353353353	-0.101579669366884\\
1.54154154154154	-0.0961331210781738\\
1.54954954954955	-0.090654013542824\\
1.55755755755756	-0.0851423467608339\\
1.56556556556557	-0.0795981207322038\\
1.57357357357357	-0.0740213354569338\\
1.58158158158158	-0.0684119909350229\\
1.58958958958959	-0.0627700871664727\\
1.5975975975976	-0.0570956241512822\\
1.60560560560561	-0.0513886018894518\\
1.61361361361361	-0.0456490203809812\\
1.62162162162162	-0.0398768796258699\\
1.62962962962963	-0.0340721796241191\\
1.63763763763764	-0.0282349203757284\\
1.64564564564565	-0.0223651018806974\\
1.65365365365365	-0.0164627241390266\\
1.66166166166166	-0.0105277871507149\\
1.66966966966967	-0.00456029091576371\\
1.67767767767768	0.00143976456582762\\
1.68568568568569	0.00747237929405897\\
1.69369369369369	0.0135375532689304\\
1.7017017017017	0.0196352864904425\\
1.70970970970971	0.0257655789585941\\
1.71771771771772	0.0319284306733858\\
1.72572572572573	0.0381238416348174\\
1.73373373373373	0.0443518118428893\\
1.74174174174174	0.0506123412976018\\
1.74974974974975	0.0569054299989539\\
1.75775775775776	0.0632310779469459\\
1.76576576576577	0.0695892851415781\\
1.77377377377377	0.0759800515828511\\
1.78178178178178	0.0824033772707635\\
1.78978978978979	0.0888592622053158\\
1.7977977977978	0.0953477063865084\\
1.80580580580581	0.101868709814341\\
1.81381381381381	0.108422272488814\\
1.82182182182182	0.115008394409927\\
1.82982982982983	0.12162707557768\\
1.83783783783784	0.128278315992073\\
1.84584584584585	0.134962115653106\\
1.85385385385385	0.14167847456078\\
1.86186186186186	0.148427392715093\\
1.86986986986987	0.155208870116046\\
1.87787787787788	0.16202290676364\\
1.88588588588589	0.168869502657873\\
1.89389389389389	0.175748657798747\\
1.9019019019019	0.182660372186261\\
1.90990990990991	0.189604645820415\\
1.91791791791792	0.196581478701208\\
1.92592592592593	0.203590870828642\\
1.93393393393393	0.210632822202717\\
1.94194194194194	0.217707332823431\\
1.94994994994995	0.224814402690785\\
1.95795795795796	0.231954031804779\\
1.96596596596597	0.239126220165414\\
1.97397397397397	0.246330967772689\\
1.98198198198198	0.253568274626603\\
1.98998998998999	0.260838140727158\\
1.997997997998	0.268140566074352\\
2.00600600600601	0.275475550668187\\
2.01401401401401	0.282843094508663\\
2.02202202202202	0.290243197595778\\
2.03003003003003	0.297675859929533\\
2.03803803803804	0.305141081509928\\
2.04604604604605	0.312638862336963\\
2.05405405405405	0.320169202410639\\
2.06206206206206	0.327732101730954\\
2.07007007007007	0.33532756029791\\
2.07807807807808	0.342955578111505\\
2.08608608608609	0.350616155171741\\
2.09409409409409	0.358309291478617\\
2.1021021021021	0.366034987032133\\
2.11011011011011	0.373793241832289\\
2.11811811811812	0.381584055879085\\
2.12612612612613	0.389407429172521\\
2.13413413413413	0.397263361712598\\
2.14214214214214	0.405151853499314\\
2.15015015015015	0.41307290453267\\
2.15815815815816	0.421026514812666\\
2.16616616616617	0.429012684339303\\
2.17417417417417	0.43703141311258\\
2.18218218218218	0.445082701132497\\
2.19019019019019	0.453166548399053\\
2.1981981981982	0.46128295491225\\
2.20620620620621	0.469431920672087\\
2.21421421421421	0.477613445678565\\
2.22222222222222	0.485827529931682\\
2.23023023023023	0.494074173431439\\
2.23823823823824	0.502353376177836\\
2.24624624624625	0.510665138170873\\
2.25425425425425	0.519009459410552\\
2.26226226226226	0.527386339896869\\
2.27027027027027	0.535795779629826\\
2.27827827827828	0.544237778609424\\
2.28628628628629	0.552712336835663\\
2.29429429429429	0.56121945430854\\
2.3023023023023	0.569759131028059\\
2.31031031031031	0.578331366994216\\
2.31831831831832	0.586936162207015\\
2.32632632632633	0.595573516666453\\
2.33433433433433	0.604243430372532\\
2.34234234234234	0.61294590332525\\
2.35035035035035	0.621680935524608\\
2.35835835835836	0.630448526970607\\
2.36636636636637	0.639248677663246\\
2.37437437437437	0.648081387602525\\
2.38238238238238	0.656946656788444\\
2.39039039039039	0.665844485221002\\
2.3983983983984	0.674774872900201\\
2.40640640640641	0.683737819826041\\
2.41441441441441	0.69273332599852\\
2.42242242242242	0.701761391417639\\
2.43043043043043	0.710822016083399\\
2.43843843843844	0.719915199995798\\
2.44644644644645	0.729040943154839\\
2.45445445445445	0.738199245560518\\
2.46246246246246	0.747390107212837\\
2.47047047047047	0.756613528111797\\
2.47847847847848	0.765869508257397\\
2.48648648648649	0.775158047649638\\
2.49449449449449	0.784479146288517\\
2.5025025025025	0.793832804174037\\
2.51051051051051	0.803219021306197\\
2.51851851851852	0.812637797684998\\
2.52652652652653	0.822089133310439\\
2.53453453453453	0.831573028182519\\
2.54254254254254	0.84108948230124\\
2.55055055055055	0.8506384956666\\
2.55855855855856	0.860220068278601\\
2.56656656656657	0.869834200137243\\
2.57457457457457	0.879480891242523\\
2.58258258258258	0.889160141594444\\
2.59059059059059	0.898871951193005\\
2.5985985985986	0.908616320038206\\
2.60660660660661	0.918393248130048\\
2.61461461461461	0.928202735468529\\
2.62262262262262	0.93804478205365\\
2.63063063063063	0.947919387885412\\
2.63863863863864	0.957826552963813\\
2.64664664664665	0.967766277288856\\
2.65465465465465	0.977738560860538\\
2.66266266266266	0.987743403678859\\
2.67067067067067	0.997780805743821\\
2.67867867867868	1.00785076705542\\
2.68668668668669	1.01795328761367\\
2.69469469469469	1.02808836741855\\
2.7027027027027	1.03825600647007\\
2.71071071071071	1.04845620476823\\
2.71871871871872	1.05868896231303\\
2.72672672672673	1.06895427910448\\
2.73473473473473	1.07925215514256\\
2.74274274274274	1.08958259042728\\
2.75075075075075	1.09994558495865\\
2.75875875875876	1.11034113873665\\
2.76676676676677	1.12076925176129\\
2.77477477477477	1.13122992403257\\
2.78278278278278	1.1417231555505\\
2.79079079079079	1.15224894631506\\
2.7987987987988	1.16280729632627\\
2.80680680680681	1.17339820558411\\
2.81481481481481	1.18402167408859\\
2.82282282282282	1.19467770183972\\
2.83083083083083	1.20536628883748\\
2.83883883883884	1.21608743508188\\
2.84684684684685	1.22684114057293\\
2.85485485485485	1.23762740531061\\
2.86286286286286	1.24844622929493\\
2.87087087087087	1.2592976125259\\
2.87887887887888	1.2701815550035\\
2.88688688688689	1.28109805672775\\
2.89489489489489	1.29204711769863\\
2.9029029029029	1.30302873791616\\
2.91091091091091	1.31404291738032\\
2.91891891891892	1.32508965609113\\
2.92692692692693	1.33616895404857\\
2.93493493493493	1.34728081125266\\
2.94294294294294	1.35842522770338\\
2.95095095095095	1.36960220340074\\
2.95895895895896	1.38081173834475\\
2.96696696696697	1.3920538325354\\
2.97497497497497	1.40332848597268\\
2.98298298298298	1.41463569865661\\
2.99099099099099	1.42597547058717\\
2.998998998999	1.43734780176438\\
3.00700700700701	1.44875269218822\\
3.01501501501502	1.46019014185871\\
3.02302302302302	1.47166015077583\\
3.03103103103103	1.4831627189396\\
3.03903903903904	1.49469784635001\\
3.04704704704705	1.50626553300705\\
3.05505505505506	1.51786577891074\\
3.06306306306306	1.52949858406106\\
3.07107107107107	1.54116394845803\\
3.07907907907908	1.55286187210164\\
3.08708708708709	1.56459235499188\\
3.0950950950951	1.57635539712877\\
3.1031031031031	1.5881509985123\\
3.11111111111111	1.59997915914246\\
3.11911911911912	1.61183987901927\\
3.12712712712713	1.62373315814272\\
3.13513513513514	1.6356589965128\\
3.14314314314314	1.64761739412953\\
3.15115115115115	1.6596083509929\\
3.15915915915916	1.6716318671029\\
3.16716716716717	1.68368794245955\\
3.17517517517518	1.69577657706284\\
3.18318318318318	1.70789777091277\\
3.19119119119119	1.72005152400933\\
3.1991991991992	1.73223783635254\\
3.20720720720721	1.74445670794239\\
3.21521521521522	1.75670813877888\\
3.22322322322322	1.768992128862\\
3.23123123123123	1.78130867819177\\
3.23923923923924	1.79365778676818\\
3.24724724724725	1.80603945459123\\
3.25525525525526	1.81845368166092\\
3.26326326326326	1.83090046797725\\
3.27127127127127	1.84337981354022\\
3.27927927927928	1.85589171834982\\
3.28728728728729	1.86843618240607\\
3.2952952952953	1.88101320570896\\
3.3033033033033	1.89362278825849\\
3.31131131131131	1.90626493005466\\
3.31931931931932	1.91893963109747\\
3.32732732732733	1.93164689138692\\
3.33533533533534	1.944386710923\\
3.34334334334334	1.95715908970573\\
3.35135135135135	1.9699640277351\\
3.35935935935936	1.98280152501111\\
3.36736736736737	1.99567158153376\\
3.37537537537538	2.00857419730305\\
3.38338338338338	2.02150937231898\\
3.39139139139139	2.03447710658155\\
3.3993993993994	2.04747740009076\\
3.40740740740741	2.06051025284661\\
3.41541541541542	2.0735756648491\\
3.42342342342342	2.08667363609823\\
3.43143143143143	2.099804166594\\
3.43943943943944	2.11296725633641\\
3.44744744744745	2.12616290532546\\
3.45545545545546	2.13939111356115\\
3.46346346346346	2.15265188104348\\
3.47147147147147	2.16594520777245\\
3.47947947947948	2.17927109374806\\
3.48748748748749	2.19262953897031\\
3.4954954954955	2.2060205434392\\
3.5035035035035	2.21944410715474\\
3.51151151151151	2.23290023011691\\
3.51951951951952	2.24638891232572\\
3.52752752752753	2.25991015378117\\
3.53553553553554	2.27346395448326\\
3.54354354354354	2.28705031443199\\
3.55155155155155	2.30066923362736\\
3.55955955955956	2.31432071206937\\
3.56756756756757	2.32800474975803\\
3.57557557557558	2.34172134669332\\
3.58358358358358	2.35547050287525\\
3.59159159159159	2.36925221830382\\
3.5995995995996	2.38306649297903\\
3.60760760760761	2.39691332690089\\
3.61561561561562	2.41079272006938\\
3.62362362362362	2.42470467248451\\
3.63163163163163	2.43864918414628\\
3.63963963963964	2.45262625505469\\
3.64764764764765	2.46663588520975\\
3.65565565565566	2.48067807461144\\
3.66366366366366	2.49475282325977\\
3.67167167167167	2.50886013115474\\
3.67967967967968	2.52299999829636\\
3.68768768768769	2.53717242468461\\
3.6956956956957	2.5513774103195\\
3.7037037037037	2.56561495520104\\
3.71171171171171	2.57988505932921\\
3.71971971971972	2.59418772270402\\
3.72772772772773	2.60852294532548\\
3.73573573573574	2.62289072719357\\
3.74374374374374	2.6372910683083\\
3.75175175175175	2.65172396866968\\
3.75975975975976	2.66618942827769\\
3.76776776776777	2.68068744713234\\
3.77577577577578	2.69521802523364\\
3.78378378378378	2.70978116258157\\
3.79179179179179	2.72437685917615\\
3.7997997997998	2.73900511501736\\
3.80780780780781	2.75366593010521\\
3.81581581581582	2.76835930443971\\
3.82382382382382	2.78308523802084\\
3.83183183183183	2.79784373084862\\
3.83983983983984	2.81263478292303\\
3.84784784784785	2.82745839424409\\
3.85585585585586	2.84231456481178\\
3.86386386386386	2.85720329462612\\
3.87187187187187	2.87212458368709\\
3.87987987987988	2.8870784319947\\
3.88788788788789	2.90206483954896\\
3.8958958958959	2.91708380634985\\
3.9039039039039	2.93213533239739\\
3.91191191191191	2.94721941769157\\
3.91991991991992	2.96233606223238\\
3.92792792792793	2.97748526601984\\
3.93593593593594	2.99266702905393\\
3.94394394394394	3.00788135133467\\
3.95195195195195	3.02312823286204\\
3.95995995995996	3.03840767363606\\
3.96796796796797	3.05371967365671\\
3.97597597597598	3.06906423292401\\
3.98398398398398	3.08444135143795\\
3.99199199199199	3.09985102919852\\
4	3.11529326620574\\
};
\addlegendentry{$\mu(x)$};

\addplot [color=black,only marks,mark=*,mark options={solid}]
  table[row sep=crcr]{%
-2.26	1.03\\
-1.31	0.7\\
-0.43	-0.68\\
0.32	-1.36\\
0.34	-1.74\\
0.54	-1.01\\
0.86	0.24\\
1.83	1.55\\
2.77	1.68\\
3.58	1.53\\
};
\addlegendentry{observations};

\end{axis}
\end{tikzpicture}%
  \caption{Posterior for $k = 2$.}
  \label{order_2_expansion}
\end{figure}

\begin{figure}
  \centering
  % This file was created by matlab2tikz.
% Minimal pgfplots version: 1.3
%
\tikzsetnextfilename{order_3_expansion}
\definecolor{mycolor1}{rgb}{0.65098,0.80784,0.89020}%
\definecolor{mycolor2}{rgb}{0.12157,0.47059,0.70588}%
%
\begin{tikzpicture}

\begin{axis}[%
width=0.95092\figurewidth,
height=\figureheight,
at={(0\figurewidth,0\figureheight)},
scale only axis,
xmin=-4,
xmax=4,
xlabel={$x$},
ymin=-4,
ymax=16,
axis x line*=bottom,
axis y line*=left,
legend style={legend cell align=left,align=left,draw=white!15!black},
legend style={legend columns=-1, draw=none}, reverse legend
]

\addplot[area legend,solid,fill=mycolor1,opacity=3.000000e-01,draw=none]
table[row sep=crcr] {%
x	y\\
-4	4.94861707222749\\
-3.99199199199199	4.92086069244375\\
-3.98398398398398	4.89316991654148\\
-3.97597597597598	4.8655446074282\\
-3.96796796796797	4.83798462723902\\
-3.95995995995996	4.81048983732765\\
-3.95195195195195	4.78306009825722\\
-3.94394394394394	4.75569526979108\\
-3.93593593593594	4.72839521088348\\
-3.92792792792793	4.70115977967017\\
-3.91991991991992	4.67398883345899\\
-3.91191191191191	4.64688222872022\\
-3.9039039039039	4.61983982107705\\
-3.8958958958959	4.59286146529586\\
-3.88788788788789	4.56594701527637\\
-3.87987987987988	4.5390963240419\\
-3.87187187187187	4.51230924372934\\
-3.86386386386386	4.48558562557919\\
-3.85585585585586	4.45892531992554\\
-3.84784784784785	4.43232817618583\\
-3.83983983983984	4.40579404285073\\
-3.83183183183183	4.37932276747384\\
-3.82382382382382	4.35291419666134\\
-3.81581581581582	4.32656817606163\\
-3.80780780780781	4.30028455035485\\
-3.7997997997998	4.27406316324238\\
-3.79179179179179	4.24790385743629\\
-3.78378378378378	4.22180647464873\\
-3.77577577577578	4.19577085558125\\
-3.76776776776777	4.16979683991413\\
-3.75975975975976	4.14388426629562\\
-3.75175175175175	4.11803297233118\\
-3.74374374374374	4.09224279457264\\
-3.73573573573574	4.0665135685074\\
-3.72772772772773	4.04084512854753\\
-3.71971971971972	4.01523730801894\\
-3.71171171171171	3.98968993915043\\
-3.7037037037037	3.9642028530628\\
-3.6956956956957	3.93877587975796\\
-3.68768768768769	3.913408848108\\
-3.67967967967968	3.8881015858443\\
-3.67167167167167	3.86285391954662\\
-3.66366366366366	3.83766567463227\\
-3.65565565565566	3.8125366753452\\
-3.64764764764765	3.78746674474527\\
-3.63963963963964	3.76245570469742\\
-3.63163163163163	3.73750337586097\\
-3.62362362362362	3.71260957767898\\
-3.61561561561562	3.6877741283676\\
-3.60760760760761	3.66299684490558\\
-3.5995995995996	3.63827754302384\\
-3.59159159159159	3.61361603719509\\
-3.58358358358358	3.58901214062361\\
-3.57557557557558	3.56446566523509\\
-3.56756756756757	3.53997642166668\\
-3.55955955955956	3.51554421925707\\
-3.55155155155155	3.49116886603678\\
-3.54354354354354	3.46685016871861\\
-3.53553553553554	3.44258793268827\\
-3.52752752752753	3.41838196199515\\
-3.51951951951952	3.39423205934334\\
-3.51151151151151	3.37013802608286\\
-3.5035035035035	3.3460996622011\\
-3.4954954954955	3.32211676631454\\
-3.48748748748749	3.29818913566067\\
-3.47947947947948	3.27431656609029\\
-3.47147147147147	3.25049885206005\\
-3.46346346346346	3.22673578662526\\
-3.45545545545546	3.20302716143315\\
-3.44744744744745	3.17937276671642\\
-3.43943943943944	3.15577239128714\\
-3.43143143143143	3.1322258225311\\
-3.42342342342342	3.10873284640262\\
-3.41541541541542	3.08529324741965\\
-3.40740740740741	3.06190680865952\\
-3.3993993993994	3.0385733117551\\
-3.39139139139139	3.0152925368914\\
-3.38338338338338	2.99206426280289\\
-3.37537537537538	2.96888826677126\\
-3.36736736736737	2.94576432462379\\
-3.35935935935936	2.92269221073244\\
-3.35135135135135	2.89967169801347\\
-3.34334334334334	2.87670255792787\\
-3.33533533533534	2.85378456048244\\
-3.32732732732733	2.83091747423163\\
-3.31931931931932	2.80810106628024\\
-3.31131131131131	2.78533510228683\\
-3.3033033033033	2.76261934646811\\
-3.2952952952953	2.73995356160412\\
-3.28728728728729	2.71733750904448\\
-3.27927927927928	2.69477094871545\\
-3.27127127127127	2.67225363912816\\
-3.26326326326326	2.64978533738781\\
-3.25525525525526	2.62736579920401\\
-3.24724724724725	2.60499477890227\\
-3.23923923923924	2.58267202943662\\
-3.23123123123123	2.56039730240359\\
-3.22322322322322	2.53817034805734\\
-3.21521521521522	2.51599091532623\\
-3.20720720720721	2.49385875183072\\
-3.1991991991992	2.47177360390271\\
-3.19119119119119	2.44973521660642\\
-3.18318318318318	2.42774333376073\\
-3.17517517517518	2.40579769796316\\
-3.16716716716717	2.38389805061554\\
-3.15915915915916	2.36204413195132\\
-3.15115115115115	2.3402356810647\\
-3.14314314314314	2.31847243594153\\
-3.13513513513514	2.29675413349217\\
-3.12712712712713	2.27508050958626\\
-3.11911911911912	2.25345129908946\\
-3.11111111111111	2.23186623590236\\
-3.1031031031031	2.21032505300143\\
-3.0950950950951	2.18882748248222\\
-3.08708708708709	2.16737325560483\\
-3.07907907907908	2.1459621028417\\
-3.07107107107107	2.12459375392781\\
-3.06306306306306	2.10326793791332\\
-3.05505505505506	2.08198438321878\\
-3.04704704704705	2.06074281769294\\
-3.03903903903904	2.03954296867322\\
-3.03103103103103	2.01838456304891\\
-3.02302302302302	1.99726732732722\\
-3.01501501501502	1.97619098770215\\
-3.00700700700701	1.9551552701264\\
-2.998998998999	1.93415990038618\\
-2.99099099099099	1.91320460417919\\
-2.98298298298298	1.89228910719573\\
-2.97497497497497	1.87141313520303\\
-2.96696696696697	1.85057641413283\\
-2.95895895895896	1.82977867017233\\
-2.95095095095095	1.80901962985862\\
-2.94294294294294	1.78829902017645\\
-2.93493493493493	1.76761656865962\\
-2.92692692692693	1.74697200349596\\
-2.91891891891892	1.72636505363595\\
-2.91091091091091	1.70579544890498\\
-2.9029029029029	1.68526292011946\\
-2.89489489489489	1.66476719920665\\
-2.88688688688689	1.64430801932837\\
-2.87887887887888	1.62388511500859\\
-2.87087087087087	1.60349822226493\\
-2.86286286286286	1.58314707874411\\
-2.85485485485485	1.56283142386141\\
-2.84684684684685	1.54255099894408\\
-2.83883883883884	1.52230554737888\\
-2.83083083083083	1.5020948147635\\
-2.82282282282282	1.4819185490622\\
-2.81481481481481	1.46177650076533\\
-2.80680680680681	1.44166842305306\\
-2.7987987987988	1.42159407196298\\
-2.79079079079079	1.40155320656183\\
-2.78278278278278	1.38154558912116\\
-2.77477477477477	1.36157098529693\\
-2.76676676676677	1.34162916431307\\
-2.75875875875876	1.32171989914882\\
-2.75075075075075	1.30184296672998\\
-2.74274274274274	1.28199814812373\\
-2.73473473473473	1.26218522873724\\
-2.72672672672673	1.24240399851974\\
-2.71871871871872	1.2226542521681\\
-2.71071071071071	1.20293578933576\\
-2.7027027027027	1.18324841484482\\
-2.69469469469469	1.16359193890126\\
-2.68668668668669	1.14396617731317\\
-2.67867867867868	1.12437095171163\\
-2.67067067067067	1.10480608977433\\
-2.66266266266266	1.08527142545158\\
-2.65465465465465	1.06576679919455\\
-2.64664664664665	1.0462920581856\\
-2.63863863863864	1.02684705657037\\
-2.63063063063063	1.00743165569151\\
-2.62262262262262	0.988045724323696\\
-2.61461461461461	0.968689138909808\\
-2.60660660660661	0.949361783797846\\
-2.5985985985986	0.930063551478424\\
-2.59059059059059	0.910794342822483\\
-2.58258258258258	0.891554067318929\\
-2.57457457457457	0.872342643311887\\
-2.56656656656657	0.853159998237211\\
-2.55855855855856	0.834006068857939\\
-2.55055055055055	0.814880801498313\\
-2.54254254254254	0.795784152276\\
-2.53453453453453	0.776716087332151\\
-2.52652652652653	0.757676583058897\\
-2.51851851851852	0.738665626323892\\
-2.51051051051051	0.7196832146915\\
-2.5025025025025	0.700729356640218\\
-2.49449449449449	0.681804071775894\\
-2.48648648648649	0.662907391040357\\
-2.47847847847848	0.644039356914989\\
-2.47047047047047	0.625200023618826\\
-2.46246246246246	0.606389457300745\\
-2.45445445445445	0.58760773622531\\
-2.44644644644645	0.568854950951817\\
-2.43843843843844	0.550131204506124\\
-2.43043043043043	0.531436612544794\\
-2.42242242242242	0.512771303511176\\
-2.41441441441441	0.494135418782913\\
-2.40640640640641	0.475529112810534\\
-2.3983983983984	0.456952553246658\\
-2.39039039039039	0.438405921065426\\
-2.38238238238238	0.41988941067175\\
-2.37437437437437	0.401403230000004\\
-2.36636636636637	0.382947600601774\\
-2.35835835835836	0.364522757722307\\
-2.35035035035035	0.346128950365317\\
-2.34234234234234	0.32776644134581\\
-2.33433433433433	0.309435507330636\\
-2.32632632632633	0.291136438866454\\
-2.31831831831832	0.272869540394873\\
-2.31031031031031	0.254635130254475\\
-2.3023023023023	0.236433540669539\\
-2.29429429429429	0.218265117725243\\
-2.28628628628629	0.20013022132918\\
-2.27827827827828	0.182029225159029\\
-2.27027027027027	0.163962516596265\\
-2.26226226226226	0.145930496645819\\
-2.25425425425425	0.127933579841625\\
-2.24624624624625	0.109972194138003\\
-2.23823823823824	0.0920467807869012\\
-2.23023023023023	0.0741577942009919\\
-2.22222222222222	0.0563057018027033\\
-2.21421421421421	0.0384909838592562\\
-2.20620620620621	0.0207141333038379\\
-2.1981981981982	0.00297565554304557\\
-2.19019019019019	-0.0147239317492029\\
-2.18218218218218	-0.0323840988511039\\
-2.17417417417417	-0.0500043042254879\\
-2.16616616616617	-0.06758399476517\\
-2.15815815815816	-0.0851226060508228\\
-2.15015015015015	-0.102619562621065\\
-2.14214214214214	-0.120074278254394\\
-2.13413413413413	-0.137486156262595\\
-2.12612612612613	-0.154854589795221\\
-2.11811811811812	-0.172178962154719\\
-2.11011011011011	-0.189458647121752\\
-2.1021021021021	-0.206693009290251\\
-2.09409409409409	-0.2238814044117\\
-2.08608608608609	-0.241023179748147\\
-2.07807807807808	-0.258117674433429\\
-2.07007007007007	-0.27516421984205\\
-2.06206206206206	-0.292162139965185\\
-2.05405405405405	-0.309110751793226\\
-2.04604604604605	-0.326009365704292\\
-2.03803803803804	-0.342857285858151\\
-2.03003003003003	-0.35965381059493\\
-2.02202202202202	-0.37639823283804\\
-2.01401401401401	-0.393089840500715\\
-2.00600600600601	-0.409727916895561\\
-1.997997997998	-0.426311741146535\\
-1.98998998998999	-0.442840588602736\\
-1.98198198198198	-0.459313731253446\\
-1.97397397397397	-0.475730438143801\\
-1.96596596596597	-0.492089975790545\\
-1.95795795795796	-0.508391608597282\\
-1.94994994994995	-0.524634599268669\\
-1.94194194194194	-0.540818209223002\\
-1.93393393393393	-0.556941699002659\\
-1.92592592592593	-0.573004328681873\\
-1.91791791791792	-0.589005358271348\\
-1.90990990990991	-0.60494404811919\\
-1.9019019019019	-0.620819659307725\\
-1.89389389389389	-0.636631454045707\\
-1.88588588588589	-0.652378696055494\\
-1.87787787787788	-0.668060650954797\\
-1.86986986986987	-0.683676586632549\\
-1.86186186186186	-0.69922577361858\\
-1.85385385385385	-0.714707485446684\\
-1.84584584584585	-0.730120999010788\\
-1.83783783783784	-0.745465594913875\\
-1.82982982982983	-0.760740557809386\\
-1.82182182182182	-0.775945176734831\\
-1.81381381381381	-0.79107874543734\\
-1.80580580580581	-0.806140562690963\\
-1.7977977977978	-0.821129932605473\\
-1.78978978978979	-0.836046164926508\\
-1.78178178178178	-0.850888575326897\\
-1.77377377377377	-0.865656485689004\\
-1.76576576576577	-0.880349224377994\\
-1.75775775775776	-0.894966126505891\\
-1.74974974974975	-0.909506534186369\\
-1.74174174174174	-0.923969796780204\\
-1.73373373373373	-0.938355271131334\\
-1.72572572572573	-0.952662321793519\\
-1.71771771771772	-0.966890321247565\\
-1.70970970970971	-0.98103865010914\\
-1.7017017017017	-0.995106697327204\\
-1.69369369369369	-1.00909386037306\\
-1.68568568568569	-1.02299954542015\\
-1.67767767767768	-1.03682316751454\\
-1.66966966966967	-1.05056415073633\\
-1.66166166166166	-1.06422192835191\\
-1.65365365365365	-1.07779594295737\\
-1.64564564564565	-1.09128564661293\\
-1.63763763763764	-1.10469050096872\\
-1.62962962962963	-1.11800997738196\\
-1.62162162162162	-1.13124355702569\\
-1.61361361361361	-1.14439073098918\\
-1.60560560560561	-1.15745100037022\\
-1.5975975975976	-1.1704238763594\\
-1.58958958958959	-1.18330888031665\\
-1.58158158158158	-1.19610554384006\\
-1.57357357357357	-1.20881340882729\\
-1.56556556556557	-1.22143202752972\\
-1.55755755755756	-1.23396096259951\\
-1.54954954954955	-1.24639978712975\\
-1.54154154154154	-1.25874808468793\\
-1.53353353353353	-1.2710054493429\\
-1.52552552552553	-1.28317148568555\\
-1.51751751751752	-1.29524580884329\\
-1.50950950950951	-1.30722804448876\\
-1.5015015015015	-1.31911782884268\\
-1.49349349349349	-1.33091480867127\\
-1.48548548548549	-1.34261864127832\\
-1.47747747747748	-1.35422899449215\\
-1.46946946946947	-1.3657455466476\\
-1.46146146146146	-1.3771679865633\\
-1.45345345345345	-1.38849601351444\\
-1.44544544544545	-1.39972933720108\\
-1.43743743743744	-1.41086767771233\\
-1.42942942942943	-1.4219107654866\\
-1.42142142142142	-1.43285834126792\\
-1.41341341341341	-1.44371015605868\\
-1.40540540540541	-1.45446597106895\\
-1.3973973973974	-1.46512555766243\\
-1.38938938938939	-1.4756886972993\\
-1.38138138138138	-1.48615518147614\\
-1.37337337337337	-1.49652481166299\\
-1.36536536536537	-1.50679739923781\\
-1.35735735735736	-1.5169727654184\\
-1.34934934934935	-1.52705074119202\\
-1.34134134134134	-1.53703116724273\\
-1.33333333333333	-1.54691389387677\\
-1.32532532532533	-1.55669878094587\\
-1.31731731731732	-1.56638569776889\\
-1.30930930930931	-1.57597452305174\\
-1.3013013013013	-1.58546514480567\\
-1.29329329329329	-1.59485746026427\\
-1.28528528528529	-1.60415137579907\\
-1.27727727727728	-1.61334680683392\\
-1.26926926926927	-1.62244367775834\\
-1.26126126126126	-1.63144192183983\\
-1.25325325325325	-1.64034148113526\\
-1.24524524524525	-1.64914230640155\\
-1.23723723723724	-1.65784435700554\\
-1.22922922922923	-1.66644760083331\\
-1.22122122122122	-1.67495201419895\\
-1.21321321321321	-1.68335758175285\\
-1.20520520520521	-1.69166429638963\\
-1.1971971971972	-1.69987215915581\\
-1.18918918918919	-1.70798117915715\\
-1.18118118118118	-1.71599137346593\\
-1.17317317317317	-1.72390276702806\\
-1.16516516516517	-1.73171539257018\\
-1.15715715715716	-1.73942929050678\\
-1.14914914914915	-1.74704450884737\\
-1.14114114114114	-1.75456110310381\\
-1.13313313313313	-1.76197913619781\\
-1.12512512512513	-1.76929867836859\\
-1.11711711711712	-1.77651980708093\\
-1.10910910910911	-1.78364260693341\\
-1.1011011011011	-1.79066716956703\\
-1.09309309309309	-1.79759359357429\\
-1.08508508508509	-1.80442198440856\\
-1.07707707707708	-1.81115245429402\\
-1.06906906906907	-1.81778512213599\\
-1.06106106106106	-1.82432011343183\\
-1.05305305305305	-1.83075756018233\\
-1.04504504504505	-1.83709760080372\\
-1.03703703703704	-1.84334038004017\\
-1.02902902902903	-1.849486048877\\
-1.02102102102102	-1.85553476445439\\
-1.01301301301301	-1.86148668998184\\
-1.00500500500501	-1.86734199465324\\
-0.996996996996997	-1.87310085356254\\
-0.988988988988989	-1.87876344762021\\
-0.980980980980981	-1.88432996347035\\
-0.972972972972973	-1.88980059340842\\
-0.964964964964965	-1.89517553529984\\
-0.956956956956957	-1.90045499249914\\
-0.948948948948949	-1.90563917376992\\
-0.940940940940941	-1.91072829320555\\
-0.932932932932933	-1.91572257015054\\
-0.924924924924925	-1.9206222291227\\
-0.916916916916917	-1.92542749973599\\
-0.908908908908909	-1.93013861662411\\
-0.900900900900901	-1.9347558193649\\
-0.892892892892893	-1.93927935240534\\
-0.884884884884885	-1.94370946498743\\
-0.876876876876877	-1.94804641107468\\
-0.868868868868869	-1.95229044927942\\
-0.860860860860861	-1.95644184279074\\
-0.852852852852853	-1.96050085930329\\
-0.844844844844845	-1.9644677709466\\
-0.836836836836837	-1.96834285421534\\
-0.828828828828829	-1.97212638990005\\
-0.820820820820821	-1.97581866301877\\
-0.812812812812813	-1.97941996274919\\
-0.804804804804805	-1.98293058236166\\
-0.796796796796797	-1.98635081915267\\
-0.788788788788789	-1.98968097437918\\
-0.780780780780781	-1.9929213531935\\
-0.772772772772773	-1.99607226457887\\
-0.764764764764765	-1.99913402128564\\
-0.756756756756757	-2.00210693976808\\
-0.748748748748749	-2.00499134012188\\
-0.740740740740741	-2.00778754602217\\
-0.732732732732733	-2.0104958846622\\
-0.724724724724725	-2.01311668669256\\
-0.716716716716717	-2.01565028616101\\
-0.708708708708709	-2.01809702045291\\
-0.700700700700701	-2.02045723023212\\
-0.692692692692693	-2.0227312593825\\
-0.684684684684685	-2.02491945494997\\
-0.676676676676677	-2.02702216708501\\
-0.668668668668669	-2.02903974898576\\
-0.660660660660661	-2.03097255684155\\
-0.652652652652653	-2.03282094977696\\
-0.644644644644645	-2.03458528979634\\
-0.636636636636636	-2.03626594172879\\
-0.628628628628629	-2.03786327317359\\
-0.62062062062062	-2.03937765444611\\
-0.612612612612613	-2.04080945852408\\
-0.604604604604605	-2.04215906099433\\
-0.596596596596596	-2.04342683999993\\
-0.588588588588589	-2.0446131761877\\
-0.58058058058058	-2.04571845265613\\
-0.572572572572573	-2.04674305490365\\
-0.564564564564565	-2.04768737077727\\
-0.556556556556556	-2.04855179042156\\
-0.548548548548549	-2.04933670622799\\
-0.54054054054054	-2.05004251278458\\
-0.532532532532533	-2.05066960682582\\
-0.524524524524525	-2.05121838718301\\
-0.516516516516516	-2.05168925473477\\
-0.508508508508509	-2.0520826123579\\
-0.5005005005005	-2.05239886487849\\
-0.492492492492492	-2.05263841902331\\
-0.484484484484485	-2.05280168337145\\
-0.476476476476476	-2.05288906830614\\
-0.468468468468469	-2.05290098596692\\
-0.46046046046046	-2.05283785020191\\
-0.452452452452452	-2.05270007652039\\
-0.444444444444445	-2.0524880820455\\
-0.436436436436436	-2.05220228546721\\
-0.428428428428429	-2.0518431069954\\
-0.42042042042042	-2.05141096831322\\
-0.412412412412412	-2.05090629253047\\
-0.404404404404405	-2.05032950413728\\
-0.396396396396396	-2.04968102895789\\
-0.388388388388389	-2.04896129410452\\
-0.38038038038038	-2.04817072793147\\
-0.372372372372372	-2.04730975998928\\
-0.364364364364364	-2.04637882097906\\
-0.356356356356356	-2.04537834270689\\
-0.348348348348348	-2.04430875803833\\
-0.34034034034034	-2.04317050085312\\
-0.332332332332332	-2.04196400599984\\
-0.324324324324324	-2.04068970925075\\
-0.316316316316316	-2.03934804725673\\
-0.308308308308308	-2.03793945750221\\
-0.3003003003003	-2.03646437826027\\
-0.292292292292292	-2.03492324854777\\
-0.284284284284284	-2.0333165080805\\
-0.276276276276276	-2.03164459722853\\
-0.268268268268268	-2.02990795697144\\
-0.26026026026026	-2.02810702885375\\
-0.252252252252252	-2.02624225494034\\
-0.244244244244244	-2.02431407777192\\
-0.236236236236236	-2.02232294032057\\
-0.228228228228228	-2.0202692859453\\
-0.22022022022022	-2.01815355834767\\
-0.212212212212212	-2.01597620152749\\
-0.204204204204204	-2.01373765973847\\
-0.196196196196196	-2.01143837744398\\
-0.188188188188188	-2.00907879927287\\
-0.18018018018018	-2.00665936997527\\
-0.172172172172172	-2.00418053437846\\
-0.164164164164164	-2.00164273734281\\
-0.156156156156156	-1.9990464237177\\
-0.148148148148148	-1.99639203829756\\
-0.14014014014014	-1.9936800257779\\
-0.132132132132132	-1.99091083071142\\
-0.124124124124124	-1.98808489746415\\
-0.116116116116116	-1.98520267017168\\
-0.108108108108108	-1.98226459269542\\
-0.1001001001001	-1.97927110857891\\
-0.0920920920920922	-1.97622266100425\\
-0.084084084084084	-1.97311969274854\\
-0.0760760760760761	-1.96996264614043\\
-0.0680680680680679	-1.96675196301674\\
-0.06006006006006	-1.96348808467918\\
-0.0520520520520522	-1.96017145185115\\
-0.0440440440440439	-1.95680250463463\\
-0.0360360360360361	-1.95338168246717\\
-0.0280280280280278	-1.94990942407906\\
-0.02002002002002	-1.94638616745049\\
-0.0120120120120122	-1.94281234976897\\
-0.00400400400400391	-1.9391884073868\\
0.00400400400400436	-1.93551477577872\\
0.0120120120120122	-1.93179188949967\\
0.02002002002002	-1.92802018214279\\
0.0280280280280278	-1.92420008629747\\
0.0360360360360357	-1.9203320335077\\
0.0440440440440444	-1.91641645423055\\
0.0520520520520522	-1.91245377779482\\
0.06006006006006	-1.90844443235995\\
0.0680680680680679	-1.90438884487515\\
0.0760760760760757	-1.90028744103868\\
0.0840840840840844	-1.89614064525752\\
0.0920920920920922	-1.89194888060712\\
0.1001001001001	-1.88771256879154\\
0.108108108108108	-1.88343213010384\\
0.116116116116116	-1.87910798338668\\
0.124124124124124	-1.87474054599334\\
0.132132132132132	-1.87033023374899\\
0.14014014014014	-1.86587746091224\\
0.148148148148148	-1.86138264013714\\
0.156156156156156	-1.85684618243543\\
0.164164164164164	-1.8522684971392\\
0.172172172172172	-1.84764999186391\\
0.18018018018018	-1.84299107247185\\
0.188188188188188	-1.83829214303591\\
0.196196196196196	-1.83355360580385\\
0.204204204204204	-1.82877586116298\\
0.212212212212212	-1.82395930760524\\
0.22022022022022	-1.81910434169283\\
0.228228228228228	-1.81421135802418\\
0.236236236236236	-1.80928074920055\\
0.244244244244245	-1.80431290579305\\
0.252252252252252	-1.79930821631013\\
0.26026026026026	-1.79426706716572\\
0.268268268268268	-1.78918984264779\\
0.276276276276277	-1.78407692488755\\
0.284284284284285	-1.77892869382913\\
0.292292292292292	-1.77374552719995\\
0.3003003003003	-1.7685278004816\\
0.308308308308308	-1.76327588688131\\
0.316316316316317	-1.75799015730417\\
0.324324324324325	-1.75267098032578\\
0.332332332332332	-1.74731872216578\\
0.34034034034034	-1.74193374666179\\
0.348348348348348	-1.73651641524425\\
0.356356356356357	-1.73106708691175\\
0.364364364364365	-1.72558611820722\\
0.372372372372372	-1.72007386319467\\
0.38038038038038	-1.71453067343673\\
0.388388388388388	-1.70895689797296\\
0.396396396396397	-1.70335288329876\\
0.404404404404405	-1.69771897334518\\
0.412412412412412	-1.69205550945935\\
0.42042042042042	-1.6863628303858\\
0.428428428428428	-1.68064127224845\\
0.436436436436437	-1.67489116853344\\
0.444444444444445	-1.66911285007279\\
0.452452452452452	-1.66330664502873\\
0.46046046046046	-1.65747287887902\\
0.468468468468468	-1.65161187440289\\
0.476476476476477	-1.64572395166799\\
0.484484484484485	-1.63980942801803\\
0.492492492492492	-1.6338686180613\\
0.5005005005005	-1.62790183366008\\
0.508508508508508	-1.62190938392079\\
0.516516516516517	-1.61589157518512\\
0.524524524524525	-1.60984871102186\\
0.532532532532533	-1.60378109221974\\
0.54054054054054	-1.59768901678104\\
0.548548548548548	-1.59157277991607\\
0.556556556556557	-1.58543267403853\\
0.564564564564565	-1.57926898876175\\
0.572572572572573	-1.57308201089578\\
0.58058058058058	-1.56687202444533\\
0.588588588588588	-1.5606393106086\\
0.596596596596597	-1.55438414777702\\
0.604604604604605	-1.54810681153573\\
0.612612612612613	-1.54180757466509\\
0.62062062062062	-1.53548670714293\\
0.628628628628628	-1.52914447614767\\
0.636636636636637	-1.5227811460624\\
0.644644644644645	-1.51639697847967\\
0.652652652652653	-1.50999223220726\\
0.66066066066066	-1.5035671632747\\
0.668668668668668	-1.49712202494067\\
0.676676676676677	-1.49065706770126\\
0.684684684684685	-1.48417253929901\\
0.692692692692693	-1.47766868473279\\
0.7007007007007	-1.47114574626856\\
0.708708708708708	-1.46460396345078\\
0.716716716716717	-1.45804357311482\\
0.724724724724725	-1.45146480940001\\
0.732732732732733	-1.44486790376358\\
0.74074074074074	-1.43825308499527\\
0.748748748748748	-1.43162057923281\\
0.756756756756757	-1.42497060997808\\
0.764764764764765	-1.4183033981141\\
0.772772772772773	-1.41161916192266\\
0.780780780780781	-1.40491811710272\\
0.788788788788789	-1.39820047678957\\
0.796796796796797	-1.39146645157464\\
0.804804804804805	-1.38471624952599\\
0.812812812812813	-1.37795007620951\\
0.820820820820821	-1.37116813471082\\
0.828828828828829	-1.36437062565776\\
0.836836836836837	-1.35755774724355\\
0.844844844844845	-1.35072969525062\\
0.852852852852853	-1.34388666307497\\
0.860860860860861	-1.33702884175123\\
0.868868868868869	-1.33015641997824\\
0.876876876876877	-1.32326958414527\\
0.884884884884885	-1.3163685183587\\
0.892892892892893	-1.30945340446941\\
0.900900900900901	-1.30252442210053\\
0.908908908908909	-1.29558174867585\\
0.916916916916917	-1.28862555944866\\
0.924924924924925	-1.28165602753112\\
0.932932932932933	-1.27467332392409\\
0.940940940940941	-1.26767761754741\\
0.948948948948949	-1.2606690752707\\
0.956956956956957	-1.25364786194451\\
0.964964964964965	-1.24661414043196\\
0.972972972972973	-1.23956807164075\\
0.980980980980981	-1.23250981455554\\
0.988988988988989	-1.22543952627081\\
0.996996996996997	-1.21835736202398\\
1.00500500500501	-1.21126347522893\\
1.01301301301301	-1.20415801750985\\
1.02102102102102	-1.19704113873542\\
1.02902902902903	-1.18991298705331\\
1.03703703703704	-1.18277370892493\\
1.04504504504505	-1.17562344916053\\
1.05305305305305	-1.16846235095456\\
1.06106106106106	-1.16129055592119\\
1.06906906906907	-1.15410820413026\\
1.07707707707708	-1.14691543414327\\
1.08508508508509	-1.13971238304973\\
1.09309309309309	-1.13249918650366\\
1.1011011011011	-1.12527597876027\\
1.10910910910911	-1.11804289271289\\
1.11711711711712	-1.11080005992999\\
1.12512512512513	-1.10354761069246\\
1.13313313313313	-1.09628567403093\\
1.14114114114114	-1.08901437776333\\
1.14914914914915	-1.08173384853254\\
1.15715715715716	-1.07444421184414\\
1.16516516516517	-1.06714559210431\\
1.17317317317317	-1.05983811265781\\
1.18118118118118	-1.05252189582608\\
1.18918918918919	-1.04519706294535\\
1.1971971971972	-1.03786373440494\\
1.20520520520521	-1.03052202968552\\
1.21321321321321	-1.0231720673975\\
1.22122122122122	-1.01581396531944\\
1.22922922922923	-1.0084478404365\\
1.23723723723724	-1.00107380897896\\
1.24524524524525	-0.993691986460734\\
1.25325325325325	-0.986302487717931\\
1.26126126126126	-0.978905426947438\\
1.26926926926927	-0.971500917745506\\
1.27727727727728	-0.964089073146341\\
1.28528528528529	-0.956670005660715\\
1.29329329329329	-0.949243827314569\\
1.3013013013013	-0.941810649687601\\
1.30930930930931	-0.934370583951869\\
1.31731731731732	-0.926923740910356\\
1.32532532532533	-0.91947023103554\\
1.33333333333333	-0.912010164507937\\
1.34134134134134	-0.904543651254629\\
1.34934934934935	-0.897070800987764\\
1.35735735735736	-0.889591723243038\\
1.36536536536537	-0.882106527418151\\
1.37337337337337	-0.874615322811234\\
1.38138138138138	-0.867118218659253\\
1.38938938938939	-0.859615324176384\\
1.3973973973974	-0.852106748592363\\
1.40540540540541	-0.844592601190807\\
1.41341341341341	-0.837072991347509\\
1.42142142142142	-0.82954802856871\\
1.42942942942943	-0.822017822529352\\
1.43743743743744	-0.814482483111292\\
1.44544544544545	-0.806942120441518\\
1.45345345345345	-0.799396844930331\\
1.46146146146146	-0.79184676730952\\
1.46946946946947	-0.784291998670521\\
1.47747747747748	-0.776732650502567\\
1.48548548548549	-0.769168834730836\\
1.49349349349349	-0.761600663754591\\
1.5015015015015	-0.754028250485327\\
1.50950950950951	-0.746451708384926\\
1.51751751751752	-0.738871151503816\\
1.52552552552553	-0.73128669451915\\
1.53353353353353	-0.723698452773013\\
1.54154154154154	-0.716106542310644\\
1.54954954954955	-0.7085110799187\\
1.55755755755756	-0.700912183163554\\
1.56556556556557	-0.69330997042964\\
1.57357357357357	-0.685704560957849\\
1.58158158158158	-0.678096074883982\\
1.58958958958959	-0.670484633277276\\
1.5975975975976	-0.662870358178992\\
1.60560560560561	-0.655253372641089\\
1.61361361361361	-0.64763380076499\\
1.62162162162162	-0.640011767740438\\
1.62962962962963	-0.632387399884462\\
1.63763763763764	-0.624760824680452\\
1.64564564564565	-0.617132170817361\\
1.65365365365365	-0.609501568229032\\
1.66166166166166	-0.601869148133673\\
1.66966966966967	-0.594235043073481\\
1.67767767767768	-0.586599386954413\\
1.68568568568569	-0.57896231508615\\
1.69369369369369	-0.571323964222215\\
1.7017017017017	-0.563684472600302\\
1.70970970970971	-0.556043979982798\\
1.71771771771772	-0.548402627697509\\
1.72572572572573	-0.540760558678619\\
1.73373373373373	-0.533117917507882\\
1.74174174174174	-0.52547485045605\\
1.74974974974975	-0.517831505524568\\
1.75775775775776	-0.510188032487527\\
1.76576576576577	-0.502544582933907\\
1.77377377377377	-0.494901310310104\\
1.78178178178178	-0.487258369962769\\
1.78978978978979	-0.479615919181956\\
1.7977977977978	-0.471974117244601\\
1.80580580580581	-0.464333125458353\\
1.81381381381381	-0.456693107205745\\
1.82182182182182	-0.449054227988746\\
1.82982982982983	-0.441416655473685\\
1.83783783783784	-0.433780559536577\\
1.84584584584585	-0.426146112308857\\
1.85385385385385	-0.418513488223526\\
1.86186186186186	-0.410882864061754\\
1.86986986986987	-0.403254418999906\\
1.87787787787788	-0.395628334657057\\
1.88588588588589	-0.388004795142968\\
1.89389389389389	-0.380383987106567\\
1.9019019019019	-0.372766099784928\\
1.90990990990991	-0.365151325052777\\
1.91791791791792	-0.357539857472531\\
1.92592592592593	-0.349931894344893\\
1.93393393393393	-0.34232763576001\\
1.94194194194194	-0.334727284649213\\
1.94994994994995	-0.327131046837352\\
1.95795795795796	-0.319539131095739\\
1.96596596596597	-0.311951749195732\\
1.97397397397397	-0.304369115962938\\
1.98198198198198	-0.296791449332098\\
1.98998998998999	-0.28921897040262\\
1.997997997998	-0.281651903494818\\
2.00600600600601	-0.274090476206845\\
2.01401401401401	-0.266534919472346\\
2.02202202202202	-0.258985467618848\\
2.03003003003003	-0.251442358426893\\
2.03803803803804	-0.243905833189941\\
2.04604604604605	-0.236376136775047\\
2.05405405405405	-0.228853517684338\\
2.06206206206206	-0.2213382281173\\
2.07007007007007	-0.213830524033884\\
2.07807807807808	-0.206330665218457\\
2.08608608608609	-0.198838915344607\\
2.09409409409409	-0.191355542040812\\
2.1021021021021	-0.18388081695701\\
2.11011011011011	-0.176415015832038\\
2.11811811811812	-0.168958418562012\\
2.12612612612613	-0.161511309269614\\
2.13413413413413	-0.15407397637433\\
2.14214214214214	-0.146646712663635\\
2.15015015015015	-0.139229815365145\\
2.15815815815816	-0.131823586219751\\
2.16616616616617	-0.124428331555738\\
2.17417417417417	-0.117044362363916\\
2.18218218218218	-0.109671994373757\\
2.19019019019019	-0.102311548130561\\
2.1981981981982	-0.0949633490736617\\
2.20620620620621	-0.0876277276156741\\
2.21421421421421	-0.080305019222795\\
2.22222222222222	-0.0729955644961822\\
2.23023023023023	-0.065699709254385\\
2.23823823823824	-0.0584178046168764\\
2.24624624624625	-0.0511502070886569\\
2.25425425425425	-0.0438972786459604\\
2.26226226226226	-0.0366593868230543\\
2.27027027027027	-0.0294369048001406\\
2.27827827827828	-0.0222302114923694\\
2.28628628628629	-0.0150396916399491\\
2.29429429429429	-0.00786573589938167\\
2.3023023023023	-0.000708740935787766\\
2.31031031031031	0.00643089048364365\\
2.31831831831832	0.0135527493951118\\
2.32632632632633	0.0206564205425488\\
2.33433433433433	0.0277414822808761\\
2.34234234234234	0.0348075064781697\\
2.35035035035035	0.0418540584167373\\
2.35835835835836	0.0488806966931317\\
2.36636636636637	0.0558869731171023\\
2.37437437437437	0.0628724326094916\\
2.38238238238238	0.069836613099115\\
2.39039039039039	0.0767790454186121\\
2.3983983983984	0.083699253199313\\
2.40640640640641	0.0905967527651284\\
2.41441441441441	0.097471053025487\\
2.42242242242242	0.104321655367357\\
2.43043043043043	0.111148053546367\\
2.43843843843844	0.117949733577072\\
2.44644644644645	0.124726173622378\\
2.45445445445445	0.13147684388219\\
2.46246246246246	0.138201206481296\\
2.47047047047047	0.144898715356555\\
2.47847847847848	0.151568816143409\\
2.48648648648649	0.15821094606179\\
2.49449449449449	0.164824533801466\\
2.5025025025025	0.17140899940689\\
2.51051051051051	0.17796375416159\\
2.51851851851852	0.184488200472196\\
2.52652652652653	0.190981731752149\\
2.53453453453453	0.197443732305174\\
2.54254254254254	0.20387357720859\\
2.55055055055055	0.210270632196541\\
2.55855855855856	0.216634253543227\\
2.56656656656657	0.222963787946238\\
2.57457457457457	0.229258572410065\\
2.58258258258258	0.235517934129915\\
2.59059059059059	0.241741190375911\\
2.5985985985986	0.247927648377795\\
2.60660660660661	0.254076605210258\\
2.61461461461461	0.260187347679008\\
2.62262262262262	0.2662591522077\\
2.63063063063063	0.272291284725873\\
2.63863863863864	0.278283000558021\\
2.64664664664665	0.284233544313944\\
2.65465465465465	0.290142149780531\\
2.66266266266266	0.296008039815132\\
2.67067067067067	0.301830426240677\\
2.67867867867868	0.307608509742713\\
2.68668668668669	0.313341479768531\\
2.69469469469469	0.319028514428573\\
2.7027027027027	0.324668780400296\\
2.71071071071071	0.330261432834685\\
2.71871871871872	0.335805615265635\\
2.72672672672673	0.341300459522377\\
2.73473473473473	0.346745085645198\\
2.74274274274274	0.35213860180464\\
2.75075075075075	0.357480104224426\\
2.75875875875876	0.362768677108343\\
2.76676676676677	0.368003392571299\\
2.77477477477477	0.373183310574836\\
2.78278278278278	0.37830747886731\\
2.79079079079079	0.383374932929015\\
2.7987987987988	0.388384695922515\\
2.80680680680681	0.393335778648422\\
2.81481481481481	0.39822717950694\\
2.82282282282282	0.403057884465387\\
2.83083083083083	0.407826867032037\\
2.83883883883884	0.412533088236518\\
2.84684684684685	0.417175496617076\\
2.85485485485485	0.421753028214991\\
2.86286286286286	0.426264606576433\\
2.87087087087087	0.430709142762051\\
2.87887887887888	0.435085535364596\\
2.88688688688689	0.439392670534867\\
2.89489489489489	0.443629422016282\\
2.9029029029029	0.447794651188359\\
2.91091091091091	0.451887207119411\\
2.91891891891892	0.455905926628746\\
2.92692692692693	0.459849634358638\\
2.93493493493493	0.463717142856397\\
2.94294294294294	0.467507252666761\\
2.95095095095095	0.471218752434938\\
2.95895895895896	0.474850419020521\\
2.96696696696697	0.478401017622557\\
2.97497497497497	0.481869301916021\\
2.98298298298298	0.48525401419991\\
2.99099099099099	0.48855388555722\\
2.998998998999	0.491767636026988\\
3.00700700700701	0.494893974788621\\
3.01501501501502	0.497931600358698\\
3.02302302302302	0.500879200800395\\
3.03103103103103	0.503735453945714\\
3.03903903903904	0.506499027630631\\
3.04704704704705	0.509168579943288\\
3.05505505505506	0.511742759485309\\
3.06306306306306	0.514220205646325\\
3.07107107107107	0.516599548891753\\
3.07907907907908	0.518879411063835\\
3.08708708708709	0.521058405695958\\
3.0950950950951	0.523135138340193\\
3.1031031031031	0.525108206908034\\
3.11111111111111	0.5269762020242\\
3.11911911911912	0.528737707393423\\
3.12712712712713	0.530391300180035\\
3.13513513513514	0.531935551400205\\
3.14314314314314	0.533369026326577\\
3.15115115115115	0.534690284905089\\
3.15915915915916	0.535897882183657\\
3.16716716716717	0.536990368752435\\
3.17517517517518	0.53796629119528\\
3.18318318318318	0.538824192552033\\
3.19119119119119	0.5395626127912\\
3.1991991991992	0.540180089292557\\
3.20720720720721	0.540675157339213\\
3.21521521521522	0.541046350618562\\
3.22322322322322	0.541292201731587\\
3.23123123123123	0.541411242709917\\
3.23923923923924	0.541402005539991\\
3.24724724724725	0.541263022693683\\
3.25525525525526	0.540992827664673\\
3.26326326326326	0.540589955509876\\
3.27127127127127	0.540052943395135\\
3.27927927927928	0.539380331144446\\
3.28728728728729	0.538570661791868\\
3.2952952952953	0.537622482135335\\
3.3033033033033	0.536534343291506\\
3.31131131131131	0.535304801250779\\
3.31931931931932	0.533932417431621\\
3.32732732732733	0.532415759233305\\
3.33533533533534	0.530753400586139\\
3.34334334334334	0.528943922498309\\
3.35135135135135	0.526985913598394\\
3.35935935935936	0.524877970672658\\
3.36736736736737	0.522618699196175\\
3.37537537537538	0.520206713856906\\
3.38338338338338	0.5176406390718\\
3.39139139139139	0.514919109494052\\
3.3993993993994	0.512040770510575\\
3.40740740740741	0.50900427872892\\
3.41541541541542	0.505808302452692\\
3.42342342342342	0.502451522144709\\
3.43143143143143	0.498932630877089\\
3.43943943943944	0.495250334767468\\
3.44744744744745	0.49140335340064\\
3.45545545545546	0.48739042023491\\
3.46346346346346	0.483210282992457\\
3.47147147147147	0.478861704033124\\
3.47947947947948	0.474343460710999\\
3.48748748748749	0.469654345713258\\
3.4954954954955	0.464793167380761\\
3.5035035035035	0.459758750009944\\
3.51151151151151	0.454549934135583\\
3.51951951951952	0.449165576794089\\
3.52752752752753	0.443604551766973\\
3.53553553553554	0.437865749804283\\
3.54354354354354	0.431948078827742\\
3.55155155155155	0.425850464113438\\
3.55955955955956	0.419571848453983\\
3.56756756756757	0.413111192300035\\
3.57557557557558	0.406467473881215\\
3.58358358358358	0.39963968930642\\
3.59159159159159	0.392626852643656\\
3.5995995995996	0.385427995979508\\
3.60760760760761	0.378042169458432\\
3.61561561561562	0.370468441302105\\
3.62362362362362	0.362705897809112\\
3.63163163163163	0.354753643335273\\
3.63963963963964	0.346610800254983\\
3.64764764764765	0.33827650890394\\
3.65565565565566	0.329749927503731\\
3.66366366366366	0.321030232068696\\
3.67167167167167	0.312116616295607\\
3.67967967967968	0.303008291436679\\
3.68768768768769	0.293704486156457\\
3.6956956956957	0.284204446373181\\
3.7037037037037	0.274507435085221\\
3.71171171171171	0.264612732183201\\
3.71971971971972	0.254519634248463\\
3.72772772772773	0.24422745433853\\
3.73573573573574	0.233735521760198\\
3.74374374374374	0.223043181830995\\
3.75175175175175	0.212149795629639\\
3.75975975975976	0.201054739736211\\
3.76776776776777	0.189757405962718\\
3.77577577577578	0.178257201074763\\
3.78378378378378	0.166553546504969\\
3.79179179179179	0.154645878058914\\
3.7997997997998	0.14253364561416\\
3.80780780780781	0.130216312813145\\
3.81581581581582	0.117693356750544\\
3.82382382382382	0.104964267655766\\
3.83183183183183	0.092028548571238\\
3.83983983983984	0.0788857150270674\\
3.84784784784785	0.0655352947127557\\
3.85585585585586	0.0519768271464944\\
3.86386386386386	0.0382098633426518\\
3.87187187187187	0.0242339654780164\\
3.87987987987988	0.0100487065573085\\
3.88788788788789	-0.00434632992148876\\
3.8958958958959	-0.018951550301501\\
3.9039039039039	-0.0337673511005556\\
3.91191191191191	-0.0487941193435184\\
3.91991991991992	-0.0640322328929797\\
3.92792792792793	-0.0794820607778273\\
3.93593593593594	-0.0951439635193616\\
3.94394394394394	-0.111018293454528\\
3.95195195195195	-0.127105395055985\\
3.95995995995996	-0.143405605248633\\
3.96796796796797	-0.1599192537223\\
3.97597597597598	-0.176646663240344\\
3.98398398398398	-0.193588149943846\\
3.99199199199199	-0.210744023651211\\
4	-0.228114588152922\\
4	3.45931686208106\\
3.99199199199199	3.45220933773197\\
3.98398398398398	3.44514129530075\\
3.97597597597598	3.43811300154884\\
3.96796796796797	3.43112471720975\\
3.95995995995996	3.42417669669429\\
3.95195195195195	3.41726918779103\\
3.94394394394394	3.41040243136204\\
3.93593593593594	3.40357666103434\\
3.92792792792793	3.39679210288718\\
3.91991991991992	3.39004897513552\\
3.91191191191191	3.38334748780998\\
3.9039039039039	3.37668784243357\\
3.8958958958959	3.37007023169562\\
3.88788788788789	3.36349483912317\\
3.87987987987988	3.35696183875031\\
3.87187187187187	3.35047139478581\\
3.86386386386386	3.34402366127958\\
3.85585585585586	3.33761878178825\\
3.84784784784785	3.3312568890405\\
3.83983983983984	3.32493810460263\\
3.83183183183183	3.31866253854474\\
3.82382382382382	3.31243028910822\\
3.81581581581582	3.30624144237512\\
3.80780780780781	3.30009607193976\\
3.7997997997998	3.29399423858345\\
3.79179179179179	3.28793598995279\\
3.78378378378378	3.28192136024213\\
3.77577577577578	3.27595036988094\\
3.76776776776777	3.2700230252267\\
3.75975975975976	3.26413931826396\\
3.75175175175175	3.25829922631022\\
3.74374374374374	3.2525027117294\\
3.73573573573574	3.24674972165349\\
3.72772772772773	3.24104018771312\\
3.71971971971972	3.23537402577773\\
3.71171171171171	3.22975113570604\\
3.7037037037037	3.22417140110746\\
3.6956956956957	3.21863468911525\\
3.68768768768769	3.21314085017195\\
3.67967967967968	3.20768971782784\\
3.67167167167167	3.20228110855307\\
3.66366366366366	3.19691482156409\\
3.65565565565566	3.19159063866503\\
3.64764764764765	3.18630832410457\\
3.63963963963964	3.18106762444897\\
3.63163163163163	3.17586826847171\\
3.62362362362362	3.17070996706041\\
3.61561561561562	3.16559241314138\\
3.60760760760761	3.16051528162234\\
3.5995995995996	3.15547822935379\\
3.59159159159159	3.15048089510933\\
3.58358358358358	3.1455228995853\\
3.57557557557558	3.14060384542022\\
3.56756756756757	3.13572331723399\\
3.55955955955956	3.13088088168743\\
3.55155155155155	3.12607608756206\\
3.54354354354354	3.12130846586046\\
3.53553553553554	3.11657752992714\\
3.52752752752753	3.11188277559011\\
3.51951951951952	3.107223681323\\
3.51151151151151	3.10259970842776\\
3.5035035035035	3.09801030123783\\
3.4954954954955	3.09345488734152\\
3.48748748748749	3.08893287782551\\
3.47947947947948	3.08444366753815\\
3.47147147147147	3.07998663537223\\
3.46346346346346	3.07556114456681\\
3.45545545545546	3.0711665430279\\
3.44744744744745	3.06680216366725\\
3.43943943943944	3.06246732475896\\
3.43143143143143	3.05816133031324\\
3.42342342342342	3.05388347046679\\
3.41541541541542	3.04963302188916\\
3.40740740740741	3.04540924820437\\
3.3993993993994	3.04121140042716\\
3.39139139139139	3.03703871741305\\
3.38338338338338	3.03289042632148\\
3.37537537537538	3.0287657430913\\
3.36736736736737	3.0246638729276\\
3.35935935935936	3.02058401079924\\
3.35135135135135	3.01652534194609\\
3.34334334334334	3.01248704239515\\
3.33533533533534	3.00846827948457\\
3.32732732732733	3.00446821239486\\
3.31931931931932	3.0004859926861\\
3.31131131131131	2.99652076484053\\
3.3033033033033	2.99257166680931\\
3.2952952952953	2.98863783056284\\
3.28728728728729	2.98471838264341\\
3.27927927927928	2.9808124447196\\
3.27127127127127	2.97691913414125\\
3.26326326326326	2.97303756449433\\
3.25525525525526	2.96916684615475\\
3.24724724724725	2.96530608684026\\
3.23923923923924	2.96145439215968\\
3.23123123123123	2.95761086615862\\
3.22322322322322	2.95377461186084\\
3.21521521521522	2.94994473180471\\
3.20720720720721	2.94612032857376\\
3.1991991991992	2.94230050532088\\
3.19119119119119	2.93848436628538\\
3.18318318318318	2.93467101730228\\
3.17517517517518	2.93085956630326\\
3.16716716716717	2.92704912380875\\
3.15915915915916	2.92323880341049\\
3.15115115115115	2.91942772224425\\
3.14314314314314	2.9156150014521\\
3.13513513513514	2.91179976663386\\
3.12712712712713	2.90798114828738\\
3.11911911911912	2.90415828223721\\
3.11111111111111	2.90033031005144\\
3.1031031031031	2.89649637944631\\
3.0950950950951	2.89265564467845\\
3.08708708708709	2.88880726692451\\
3.07907907907908	2.88495041464788\\
3.07107107107107	2.88108426395254\\
3.06306306306306	2.8772079989238\\
3.05505505505506	2.8733208119558\\
3.04704704704705	2.86942190406588\\
3.03903903903904	2.86551048519557\\
3.03103103103103	2.86158577449841\\
3.02302302302302	2.85764700061445\\
3.01501501501502	2.85369340193158\\
3.00700700700701	2.84972422683371\\
2.998998998999	2.84573873393593\\
2.99099099099099	2.84173619230673\\
2.98298298298298	2.83771588167742\\
2.97497497497497	2.83367709263895\\
2.96696696696697	2.82961912682623\\
2.95895895895896	2.82554129709017\\
2.95095095095095	2.82144292765765\\
2.94294294294294	2.81732335427963\\
2.93493493493493	2.81318192436761\\
2.92692692692693	2.80901799711872\\
2.91891891891892	2.80483094362959\\
2.91091091091091	2.80062014699945\\
2.9029029029029	2.7963850024225\\
2.89489489489489	2.79212491726993\\
2.88688688688689	2.78783931116198\\
2.87887887887888	2.78352761603008\\
2.87087087087087	2.77918927616955\\
2.86286286286286	2.7748237482831\\
2.85485485485485	2.77043050151541\\
2.84684684684685	2.76600901747901\\
2.83883883883884	2.76155879027201\\
2.83083083083083	2.75707932648757\\
2.82282282282282	2.75257014521588\\
2.81481481481481	2.74803077803845\\
2.80680680680681	2.74346076901548\\
2.7987987987988	2.73885967466619\\
2.79079079079079	2.7342270639427\\
2.78278278278278	2.72956251819754\\
2.77477477477477	2.72486563114517\\
2.76676676676677	2.7201360088178\\
2.75875875875876	2.7153732695157\\
2.75075075075075	2.71057704375232\\
2.74274274274274	2.70574697419447\\
2.73473473473473	2.70088271559787\\
2.72672672672673	2.69598393473813\\
2.71871871871872	2.69105031033772\\
2.71071071071071	2.68608153298882\\
2.7027027027027	2.68107730507259\\
2.69469469469469	2.67603734067482\\
2.68668668668669	2.67096136549842\\
2.67867867867868	2.66584911677274\\
2.67067067067067	2.66070034316015\\
2.66266266266266	2.65551480465984\\
2.65465465465465	2.65029227250927\\
2.64664664664665	2.64503252908329\\
2.63863863863864	2.63973536779115\\
2.63063063063063	2.63440059297165\\
2.62262262262262	2.6290280197865\\
2.61461461461461	2.62361747411212\\
2.60660660660661	2.61816879242993\\
2.5985985985986	2.61268182171552\\
2.59059059059059	2.60715641932651\\
2.58258258258258	2.60159245288948\\
2.57457457457457	2.5959898001861\\
2.56656656656657	2.5903483490384\\
2.55855855855856	2.5846679971935\\
2.55055055055055	2.57894865220779\\
2.54254254254254	2.57319023133078\\
2.53453453453453	2.56739266138858\\
2.52652652652653	2.56155587866725\\
2.51851851851852	2.555679828796\\
2.51051051051051	2.54976446663049\\
2.5025025025025	2.54380975613606\\
2.49449449449449	2.53781567027124\\
2.48648648648649	2.53178219087149\\
2.47847847847848	2.52570930853316\\
2.47047047047047	2.51959702249793\\
2.46246246246246	2.51344534053765\\
2.45445445445445	2.50725427883966\\
2.44644644644645	2.50102386189274\\
2.43843843843844	2.49475412237359\\
2.43043043043043	2.48844510103401\\
2.42242242242242	2.48209684658883\\
2.41441441441441	2.47570941560452\\
2.40640640640641	2.4692828723886\\
2.3983983983984	2.46281728887997\\
2.39039039039039	2.45631274453995\\
2.38238238238238	2.44976932624438\\
2.37437437437437	2.44318712817649\\
2.36636636636637	2.43656625172083\\
2.35835835835836	2.42990680535812\\
2.35035035035035	2.42320890456113\\
2.34234234234234	2.41647267169151\\
2.33433433433433	2.40969823589772\\
2.32632632632633	2.40288573301399\\
2.31831831831832	2.39603530546029\\
2.31031031031031	2.38914710214347\\
2.3023023023023	2.38222127835936\\
2.29429429429429	2.37525799569607\\
2.28628628628629	2.36825742193833\\
2.27827827827828	2.36121973097292\\
2.27027027027027	2.35414510269526\\
2.26226226226226	2.34703372291704\\
2.25425425425425	2.33988578327503\\
2.24624624624625	2.33270148114093\\
2.23823823823824	2.32548101953239\\
2.23023023023023	2.31822460702508\\
2.22222222222222	2.31093245766592\\
2.21421421421421	2.30360479088734\\
2.20620620620621	2.29624183142271\\
2.1981981981982	2.28884380922277\\
2.19019019019019	2.28141095937324\\
2.18218218218218	2.27394352201342\\
2.17417417417417	2.26644174225587\\
2.16616616616617	2.25890587010723\\
2.15815815815816	2.25133616038991\\
2.15015015015015	2.24373287266502\\
2.14214214214214	2.2360962711562\\
2.13413413413413	2.22842662467445\\
2.12612612612613	2.22072420654407\\
2.11811811811812	2.2129892945295\\
2.11011011011011	2.20522217076316\\
2.1021021021021	2.19742312167428\\
2.09409409409409	2.18959243791866\\
2.08608608608609	2.18173041430936\\
2.07807807807808	2.17383734974837\\
2.07007007007007	2.1659135471591\\
2.06206206206206	2.1579593134199\\
2.05405405405405	2.1499749592983\\
2.04604604604605	2.14196079938625\\
2.03803803803804	2.13391715203619\\
2.03003003003003	2.1258443392979\\
2.02202202202202	2.11774268685622\\
2.01401401401401	2.10961252396963\\
2.00600600600601	2.10145418340947\\
1.997997997998	2.09326800140014\\
1.98998998998999	2.0850543175599\\
1.98198198198198	2.07681347484251\\
1.97397397397397	2.06854581947957\\
1.96596596596597	2.06025170092357\\
1.95795795795796	2.05193147179169\\
1.94994994994995	2.04358548781024\\
1.94194194194194	2.03521410775975\\
1.93393393393393	2.02681769342084\\
1.92592592592593	2.01839660952056\\
1.91791791791792	2.00995122367949\\
1.90990990990991	2.0014819063594\\
1.9019019019019	1.9929890308115\\
1.89389389389389	1.98447297302527\\
1.88588588588589	1.9759341116779\\
1.87787787787788	1.96737282808423\\
1.86986986986987	1.95878950614724\\
1.86186186186186	1.95018453230908\\
1.85385385385385	1.94155829550258\\
1.84584584584585	1.9329111871033\\
1.83783783783784	1.92424360088197\\
1.82982982982983	1.9155559329575\\
1.82182182182182	1.90684858175037\\
1.81381381381381	1.89812194793647\\
1.80580580580581	1.88937643440138\\
1.7977977977978	1.88061244619505\\
1.78978978978979	1.87183039048685\\
1.78178178178178	1.86303067652105\\
1.77377377377377	1.85421371557262\\
1.76576576576577	1.84537992090341\\
1.75775775775776	1.83652970771869\\
1.74974974974975	1.82766349312397\\
1.74174174174174	1.81878169608218\\
1.73373373373373	1.80988473737114\\
1.72572572572573	1.80097303954132\\
1.71771771771772	1.79204702687387\\
1.70970970970971	1.78310712533895\\
1.7017017017017	1.77415376255429\\
1.69369369369369	1.76518736774399\\
1.68568568568569	1.75620837169757\\
1.67767767767768	1.74721720672926\\
1.66966966966967	1.73821430663744\\
1.66166166166166	1.72920010666434\\
1.65365365365365	1.72017504345591\\
1.64564564564565	1.71113955502187\\
1.63763763763764	1.70209408069591\\
1.62962962962963	1.69303906109611\\
1.62162162162162	1.68397493808543\\
1.61361361361361	1.67490215473238\\
1.60560560560561	1.66582115527185\\
1.5975975975976	1.656732385066\\
1.58958958958959	1.64763629056532\\
1.58158158158158	1.63853331926977\\
1.57357357357357	1.62942391968998\\
1.56556556556557	1.62030854130865\\
1.55755755755756	1.61118763454187\\
1.54954954954955	1.60206165070066\\
1.54154154154154	1.59293104195251\\
1.53353353353353	1.58379626128294\\
1.52552552552553	1.57465776245722\\
1.51751751751752	1.56551599998201\\
1.50950950950951	1.55637142906714\\
1.5015015015015	1.54722450558737\\
1.49349349349349	1.53807568604418\\
1.48548548548549	1.5289254275276\\
1.47747747747748	1.51977418767804\\
1.46946946946947	1.51062242464816\\
1.46146146146146	1.50147059706468\\
1.45345345345345	1.49231916399028\\
1.44544544544545	1.48316858488543\\
1.43743743743744	1.47401931957027\\
1.42942942942943	1.46487182818639\\
1.42142142142142	1.45572657115872\\
1.41341341341341	1.44658400915732\\
1.40540540540541	1.43744460305915\\
1.3973973973974	1.42830881390988\\
1.38938938938939	1.41917710288563\\
1.38138138138138	1.41004993125469\\
1.37337337337337	1.40092776033924\\
1.36536536536537	1.39181105147701\\
1.35735735735736	1.38270026598295\\
1.34934934934935	1.37359586511084\\
1.34134134134134	1.36449831001488\\
1.33333333333333	1.35540806171129\\
1.32532532532533	1.34632558103984\\
1.31731731731732	1.33725132862535\\
1.30930930930931	1.32818576483921\\
1.3013013013013	1.31912934976087\\
1.29329329329329	1.31008254313924\\
1.28528528528529	1.30104580435418\\
1.27727727727728	1.29201959237791\\
1.26926926926927	1.28300436573638\\
1.26126126126126	1.27400058247075\\
1.25325325325325	1.26500870009871\\
1.24524524524525	1.25602917557593\\
1.23723723723724	1.24706246525742\\
1.22922922922923	1.238109024859\\
1.22122122122122	1.22916930941865\\
1.21321321321321	1.220243773258\\
1.20520520520521	1.21133286994381\\
1.1971971971972	1.20243705224943\\
1.18918918918919	1.19355677211636\\
1.18118118118118	1.18469248061584\\
1.17317317317317	1.17584462791047\\
1.16516516516517	1.1670136632159\\
1.15715715715716	1.15820003476263\\
1.14914914914915	1.1494041897578\\
1.14114114114114	1.14062657434715\\
1.13313313313313	1.13186763357699\\
1.12512512512513	1.12312781135637\\
1.11711711711712	1.11440755041926\\
1.10910910910911	1.10570729228694\\
1.1011011011011	1.09702747723044\\
1.09309309309309	1.08836854423319\\
1.08508508508509	1.07973093095378\\
1.07707707707708	1.07111507368891\\
1.06906906906907	1.06252140733646\\
1.06106106106106	1.05395036535885\\
1.05305305305305	1.04540237974646\\
1.04504504504505	1.03687788098139\\
1.03703703703704	1.02837729800136\\
1.02902902902903	1.01990105816385\\
1.02102102102102	1.01144958721051\\
1.01301301301301	1.00302330923183\\
1.00500500500501	0.994622646632071\\
0.996996996996997	0.986248020094452\\
0.988988988988989	0.977899848546693\\
0.980980980980981	0.969578549126825\\
0.972972972972973	0.961284537149342\\
0.964964964964965	0.953018226071684\\
0.956956956956957	0.944780027461079\\
0.948948948948949	0.936570350961747\\
0.940940940940941	0.928389604262484\\
0.932932932932933	0.920238193064648\\
0.924924924924925	0.912116521050538\\
0.916916916916917	0.904024989852205\\
0.908908908908909	0.895963999020709\\
0.900900900900901	0.887933945995803\\
0.892892892892893	0.879935226076107\\
0.884884884884885	0.871968232389747\\
0.876876876876877	0.864033355865488\\
0.868868868868869	0.856130985204383\\
0.860860860860861	0.848261506851935\\
0.852852852852853	0.840425304970805\\
0.844844844844845	0.832622761414057\\
0.836836836836837	0.824854255698975\\
0.828828828828829	0.817120164981457\\
0.820820820820821	0.809420864031\\
0.812812812812813	0.801756725206286\\
0.804804804804805	0.794128118431388\\
0.796796796796797	0.786535411172604\\
0.788788788788789	0.77897896841594\\
0.780780780780781	0.771459152645244\\
0.772772772772773	0.763976323821012\\
0.764764764764765	0.756530839359868\\
0.756756756756757	0.749123054114745\\
0.748748748748748	0.741753320355764\\
0.74074074074074	0.734421987751834\\
0.732732732732733	0.727129403352972\\
0.724724724724725	0.719875911573361\\
0.716716716716717	0.712661854175161\\
0.708708708708708	0.705487570253071\\
0.7007007007007	0.698353396219664\\
0.692692692692693	0.691259665791489\\
0.684684684684685	0.684206709975971\\
0.676676676676677	0.677194857059088\\
0.668668668668668	0.670224432593868\\
0.66066066066066	0.663295759389682\\
0.652652652652653	0.656409157502357\\
0.644644644644645	0.649564944225114\\
0.636636636636637	0.642763434080334\\
0.628628628628628	0.636004938812159\\
0.62062062062062	0.629289767379936\\
0.612612612612613	0.622618225952502\\
0.604604604604605	0.615990617903327\\
0.596596596596597	0.609407243806501\\
0.588588588588588	0.602868401433586\\
0.58058058058058	0.596374385751332\\
0.572572572572573	0.589925488920238\\
0.564564564564565	0.583522000294004\\
0.556556556556557	0.577164206419826\\
0.548548548548548	0.570852391039573\\
0.54054054054054	0.564586835091829\\
0.532532532532533	0.558367816714797\\
0.524524524524525	0.552195611250076\\
0.516516516516517	0.546070491247305\\
0.508508508508508	0.539992726469667\\
0.5005005005005	0.533962583900263\\
0.492492492492492	0.527980327749335\\
0.484484484484485	0.522046219462358\\
0.476476476476477	0.516160517728977\\
0.468468468468468	0.510323478492798\\
0.46046046046046	0.504535354962028\\
0.452452452452452	0.498796397620938\\
0.444444444444445	0.493106854242186\\
0.436436436436437	0.487466969899947\\
0.428428428428428	0.481876986983877\\
0.42042042042042	0.476337145213891\\
0.412412412412412	0.470847681655746\\
0.404404404404405	0.465408830737428\\
0.396396396396397	0.46002082426633\\
0.388388388388388	0.454683891447218\\
0.38038038038038	0.44939825890097\\
0.372372372372372	0.444164150684079\\
0.364364364364365	0.438981788308917\\
0.356356356356357	0.433851390764746\\
0.348348348348348	0.428773174539461\\
0.34034034034034	0.423747353642067\\
0.332332332332332	0.418774139625865\\
0.324324324324325	0.413853741612343\\
0.316316316316317	0.408986366315765\\
0.308308308308308	0.404172218068432\\
0.3003003003003	0.399411498846626\\
0.292292292292292	0.394704408297195\\
0.284284284284285	0.390051143764795\\
0.276276276276277	0.385451900319758\\
0.268268268268268	0.380906870786583\\
0.26026026026026	0.376416245773033\\
0.252252252252252	0.371980213699819\\
0.244244244244245	0.367598960830877\\
0.236236236236236	0.363272671304198\\
0.228228228228228	0.359001527163223\\
0.22022022022022	0.354785708388771\\
0.212212212212212	0.350625392931495\\
0.204204204204204	0.346520756744853\\
0.196196196196196	0.342471973818576\\
0.188188188188188	0.338479216212622\\
0.18018018018018	0.334542654091601\\
0.172172172172172	0.33066245575966\\
0.164164164164164	0.326838787695814\\
0.156156156156156	0.323071814589698\\
0.148148148148148	0.319361699377751\\
0.14014014014014	0.315708603279792\\
0.132132132132132	0.312112685835993\\
0.124124124124124	0.308574104944225\\
0.116116116116116	0.305093016897771\\
0.108108108108108	0.301669576423387\\
0.1001001001001	0.298303936719704\\
0.0920920920920922	0.294996249495954\\
0.0840840840840844	0.291746665011007\\
0.0760760760760757	0.288555332112707\\
0.0680680680680679	0.285422398277506\\
0.06006006006006	0.282348009650356\\
0.0520520520520522	0.279332311084884\\
0.0440440440440444	0.276375446183804\\
0.0360360360360357	0.273477557339583\\
0.0280280280280278	0.270638785775324\\
0.02002002002002	0.267859271585884\\
0.0120120120120122	0.26513915377918\\
0.00400400400400436	0.262478570317716\\
-0.00400400400400391	0.259877658160282\\
-0.0120120120120122	0.257336553303835\\
-0.02002002002002	0.254855390825552\\
-0.0280280280280278	0.252434304925045\\
-0.0360360360360361	0.250073428966718\\
-0.0440440440440439	0.247772895522275\\
-0.0520520520520522	0.245532836413355\\
-0.06006006006006	0.243353382754307\\
-0.0680680680680679	0.241234664995062\\
-0.0760760760760761	0.239176812964137\\
-0.084084084084084	0.237179955911735\\
-0.0920920920920922	0.23524422255294\\
-0.1001001001001	0.23336974111101\\
-0.108108108108108	0.231556639360758\\
-0.116116116116116	0.229805044672\\
-0.124124124124124	0.228115084053098\\
-0.132132132132132	0.226486884194559\\
-0.14014014014014	0.224920571512707\\
-0.148148148148148	0.223416272193413\\
-0.156156156156156	0.221974112235891\\
-0.164164164164164	0.220594217496541\\
-0.172172172172172	0.219276713732848\\
-0.18018018018018	0.218021726647333\\
-0.188188188188188	0.216829381931548\\
-0.196196196196196	0.215699805310116\\
-0.204204204204204	0.214633122584821\\
-0.212212212212212	0.213629459678731\\
-0.22022022022022	0.212688942680372\\
-0.228228228228228	0.211811697887943\\
-0.236236236236236	0.210997851853562\\
-0.244244244244244	0.21024753142757\\
-0.252252252252252	0.209560863802864\\
-0.26026026026026	0.208937976559277\\
-0.268268268268268	0.20837899770801\\
-0.276276276276276	0.207884055736097\\
-0.284284284284284	0.207453279650929\\
-0.292292292292292	0.207086799024821\\
-0.3003003003003	0.206784744039641\\
-0.308308308308308	0.206547245531483\\
-0.316316316316316	0.206374435035412\\
-0.324324324324324	0.206266444830258\\
-0.332332332332332	0.206223407983493\\
-0.34034034034034	0.206245458396164\\
-0.348348348348348	0.206332730847907\\
-0.356356356356356	0.206485361042049\\
-0.364364364364364	0.206703485650781\\
-0.372372372372372	0.206987242360432\\
-0.38038038038038	0.207336769916837\\
-0.388388388388389	0.207752208170808\\
-0.396396396396396	0.208233698123708\\
-0.404404404404405	0.208781381973152\\
-0.412412412412412	0.209395403158818\\
-0.42042042042042	0.210075906408392\\
-0.428428428428429	0.210823037783651\\
-0.436436436436436	0.21163694472669\\
-0.444444444444445	0.212517776106293\\
-0.452452452452452	0.213465682264476\\
-0.46046046046046	0.214480815063185\\
-0.468468468468469	0.215563327931183\\
-0.476476476476476	0.216713375911111\\
-0.484484484484485	0.217931115706749\\
-0.492492492492492	0.219216705730483\\
-0.5005005005005	0.220570306150972\\
-0.508508508508509	0.221992078941053\\
-0.516516516516516	0.223482187925867\\
-0.524524524524525	0.225040798831227\\
-0.532532532532533	0.226668079332244\\
-0.54054054054054	0.228364199102209\\
-0.548548548548549	0.230129329861745\\
-0.556556556556556	0.231963645428251\\
-0.564564564564565	0.23386732176563\\
-0.572572572572573	0.235840537034325\\
-0.58058058058058	0.237883471641671\\
-0.588588588588589	0.239996308292567\\
-0.596596596596596	0.242179232040496\\
-0.604604604604605	0.244432430338882\\
-0.612612612612613	0.246756093092811\\
-0.62062062062062	0.249150412711128\\
-0.628628628628629	0.251615584158909\\
-0.636636636636636	0.25415180501033\\
-0.644644644644645	0.256759275501945\\
-0.652652652652653	0.259438198586375\\
-0.660660660660661	0.262188779986433\\
-0.668668668668669	0.26501122824968\\
-0.676676676676677	0.267905754803449\\
-0.684684684684685	0.270872574010315\\
-0.692692692692693	0.273911903224057\\
-0.700700700700701	0.277023962846093\\
-0.708708708708709	0.280208976382427\\
-0.716716716716717	0.283467170501098\\
-0.724724724724725	0.286798775090157\\
-0.732732732732733	0.29020402331617\\
-0.740740740740741	0.293683151683274\\
-0.748748748748749	0.297236400092785\\
-0.756756756756757	0.300864011903373\\
-0.764764764764765	0.304566233991817\\
-0.772772772772773	0.308343316814346\\
-0.780780780780781	0.312195514468582\\
-0.788788788788789	0.316123084756091\\
-0.796796796796797	0.320126289245552\\
-0.804804804804805	0.324205393336564\\
-0.812812812812813	0.328360666324081\\
-0.820820820820821	0.332592381463502\\
-0.828828828828829	0.336900816036418\\
-0.836836836836837	0.341286251417023\\
-0.844844844844845	0.345748973139208\\
-0.852852852852853	0.35028927096432\\
-0.860860860860861	0.354907438949631\\
-0.868868868868869	0.35960377551749\\
-0.876876876876877	0.364378583525186\\
-0.884884884884885	0.369232170335518\\
-0.892892892892893	0.374164847888077\\
-0.900900900900901	0.379176932771256\\
-0.908908908908909	0.384268746294979\\
-0.916916916916917	0.389440614564158\\
-0.924924924924925	0.394692868552884\\
-0.932932932932933	0.40002584417935\\
-0.940940940940941	0.40543988238151\\
-0.948948948948949	0.410935329193473\\
-0.956956956956957	0.416512535822635\\
-0.964964964964965	0.422171858727544\\
-0.972972972972973	0.427913659696496\\
-0.980980980980981	0.43373830592687\\
-0.988988988988989	0.439646170105179\\
-0.996996996996997	0.445637630487848\\
-1.00500500500501	0.451713070982707\\
-1.01301301301301	0.457872881231192\\
-1.02102102102102	0.464117456691248\\
-1.02902902902903	0.470447198720916\\
-1.03703703703704	0.476862514662604\\
-1.04504504504505	0.483363817928027\\
-1.05305305305305	0.489951528083791\\
-1.06106106106106	0.496626070937626\\
-1.06906906906907	0.503387878625225\\
-1.07707707707708	0.510237389697698\\
-1.08508508508509	0.517175049209598\\
-1.09309309309309	0.524201308807513\\
-1.1011011011011	0.531316626819182\\
-1.10910910910911	0.538521468343135\\
-1.11711711711712	0.5458163053388\\
-1.12512512512513	0.553201616717066\\
-1.13313313313313	0.560677888431269\\
-1.14114114114114	0.568245613568554\\
-1.14914914914915	0.575905292441592\\
-1.15715715715716	0.583657432680597\\
-1.16516516516517	0.59150254932562\\
-1.17317317317317	0.599441164919053\\
-1.18118118118118	0.607473809598322\\
-1.18918918918919	0.615601021188695\\
-1.1971971971972	0.623823345296173\\
-1.20520520520521	0.63214133540039\\
-1.21321321321321	0.640555552947485\\
-1.22122122122122	0.649066567442868\\
-1.22922922922923	0.657674956543814\\
-1.23723723723724	0.666381306151846\\
-1.24524524524525	0.675186210504791\\
-1.25325325325325	0.684090272268477\\
-1.26126126126126	0.693094102627968\\
-1.26926926926927	0.702198321378267\\
-1.27727727727728	0.711403557014398\\
-1.28528528528529	0.720710446820783\\
-1.29329329329329	0.730119636959816\\
-1.3013013013013	0.739631782559546\\
-1.30930930930931	0.749247547800364\\
-1.31731731731732	0.758967606000592\\
-1.32532532532533	0.768792639700868\\
-1.33333333333333	0.778723340747219\\
-1.34134134134134	0.788760410372692\\
-1.34934934934935	0.79890455927745\\
-1.35735735735736	0.809156507707185\\
-1.36536536536537	0.819516985529731\\
-1.37337337337337	0.829986732309749\\
-1.38138138138138	0.840566497381341\\
-1.38938938938939	0.851257039918465\\
-1.3973973973974	0.862059129002984\\
-1.40540540540541	0.872973543690237\\
-1.41341341341341	0.884001073071948\\
-1.42142142142142	0.89514251633633\\
-1.42942942942943	0.906398682825234\\
-1.43743743743744	0.917770392088163\\
-1.44544544544545	0.929258473933\\
-1.45345345345345	0.940863768473263\\
-1.46146146146146	0.952587126171733\\
-1.46946946946947	0.964429407880262\\
-1.47747747747748	0.976391484875587\\
-1.48548548548549	0.988474238890976\\
-1.49349349349349	1.0006785621435\\
-1.5015015015015	1.01300535735675\\
-1.50950950950951	1.02545553777885\\
-1.51751751751752	1.03803002719548\\
-1.52552552552553	1.05072975993784\\
-1.53353353353353	1.06355568088521\\
-1.54154154154154	1.07650874546206\\
-1.54954954954955	1.08958991962945\\
-1.55755755755756	1.10280017987042\\
-1.56556556556557	1.11614051316938\\
-1.57357357357357	1.12961191698517\\
-1.58158158158158	1.14321539921753\\
-1.58958958958959	1.156951978167\\
-1.5975975975976	1.17082268248781\\
-1.60560560560561	1.1848285511338\\
-1.61361361361361	1.19897063329695\\
-1.62162162162162	1.21324998833857\\
-1.62962962962963	1.22766768571279\\
-1.63763763763764	1.24222480488224\\
-1.64564564564565	1.25692243522581\\
-1.65365365365365	1.27176167593817\\
-1.66166166166166	1.28674363592111\\
-1.66966966966967	1.3018694336663\\
-1.67767767767768	1.31714019712961\\
-1.68568568568569	1.33255706359651\\
-1.69369369369369	1.34812117953884\\
-1.7017017017017	1.36383370046243\\
-1.70970970970971	1.37969579074575\\
-1.71771771771772	1.39570862346942\\
-1.72572572572573	1.41187338023638\\
-1.73373373373373	1.42819125098287\\
-1.74174174174174	1.44466343378\\
-1.74974974974975	1.46129113462592\\
-1.75775775775776	1.4780755672286\\
-1.76576576576577	1.49501795277918\\
-1.77377377377377	1.51211951971589\\
-1.78178178178178	1.52938150347862\\
-1.78978978978979	1.54680514625412\\
-1.7977977977978	1.56439169671193\\
-1.80580580580581	1.58214240973113\\
-1.81381381381381	1.600058546118\\
-1.82182182182182	1.61814137231467\\
-1.82982982982983	1.636392160099\\
-1.83783783783784	1.65481218627578\\
-1.84584584584585	1.67340273235941\\
-1.85385385385385	1.69216508424835\\
-1.86186186186186	1.71110053189153\\
-1.86986986986987	1.73021036894694\\
-1.87787787787788	1.74949589243268\\
-1.88588588588589	1.76895840237086\\
-1.89389389389389	1.78859920142443\\
-1.9019019019019	1.80841959452761\\
-1.90990990990991	1.82842088850993\\
-1.91791791791792	1.84860439171457\\
-1.92592592592593	1.8689714136111\\
-1.93393393393393	1.88952326440332\\
-1.94194194194194	1.91026125463246\\
-1.94994994994995	1.93118669477617\\
-1.95795795795796	1.952300894844\\
-1.96596596596597	1.97360516396956\\
-1.97397397397397	1.9951008100001\\
-1.98198198198198	2.01678913908393\\
-1.98998998998999	2.03867145525622\\
-1.997997997998	2.06074906002374\\
-2.00600600600601	2.08302325194912\\
-2.01401401401401	2.10549532623517\\
-2.02202202202202	2.12816657430984\\
-2.03003003003003	2.15103828341244\\
-2.03803803803804	2.17411173618165\\
-2.04604604604605	2.19738821024597\\
-2.05405405405405	2.22086897781717\\
-2.06206206206206	2.2445553052874\\
-2.07007007007007	2.26844845283046\\
-2.07807807807808	2.29254967400786\\
-2.08608608608609	2.31686021538033\\
-2.09409409409409	2.34138131612529\\
-2.1021021021021	2.36611420766081\\
-2.11011011011011	2.39106011327674\\
-2.11811811811812	2.41622024777352\\
-2.12612612612613	2.44159581710913\\
-2.13413413413413	2.46718801805481\\
-2.14214214214214	2.49299803786003\\
-2.15015015015015	2.51902705392714\\
-2.15815815815816	2.54527623349627\\
-2.16616616616617	2.57174673334084\\
-2.17417417417417	2.59843969947413\\
-2.18218218218218	2.62535626686738\\
-2.19019019019019	2.65249755917969\\
-2.1981981981982	2.67986468850014\\
-2.20620620620621	2.70745875510244\\
-2.21421421421421	2.73528084721242\\
-2.22222222222222	2.76333204078859\\
-2.23023023023023	2.79161339931605\\
-2.23823823823824	2.82012597361393\\
-2.24624624624625	2.84887080165656\\
-2.25425425425425	2.87784890840853\\
-2.26226226226226	2.9070613056737\\
-2.27027027027027	2.9365089919583\\
-2.27827827827828	2.96619295234817\\
-2.28628628628629	2.99611415840016\\
-2.29429429429429	3.02627356804765\\
-2.3023023023023	3.05667212552023\\
-2.31031031031031	3.0873107612774\\
-2.31831831831832	3.11819039195625\\
-2.32632632632633	3.14931192033298\\
-2.33433433433433	3.18067623529807\\
-2.34234234234234	3.21228421184505\\
-2.35035035035035	3.24413671107247\\
-2.35835835835836	3.27623458019911\\
-2.36636636636637	3.30857865259189\\
-2.37437437437437	3.34116974780642\\
-2.38238238238238	3.37400867163986\\
-2.39039039039039	3.40709621619571\\
-2.3983983983984	3.44043315996025\\
-2.40640640640641	3.47402026789031\\
-2.41441441441441	3.50785829151194\\
-2.42242242242242	3.54194796902967\\
-2.43043043043043	3.57629002544593\\
-2.43843843843844	3.61088517269028\\
-2.44644644644645	3.64573410975798\\
-2.45445445445445	3.68083752285752\\
-2.46246246246246	3.71619608556663\\
-2.47047047047047	3.75181045899655\\
-2.47847847847848	3.78768129196374\\
-2.48648648648649	3.823809221169\\
-2.49449449449449	3.86019487138326\\
-2.5025025025025	3.89683885563982\\
-2.51051051051051	3.93374177543242\\
-2.51851851851852	3.97090422091882\\
-2.52652652652653	4.00832677112943\\
-2.53453453453453	4.04600999418051\\
-2.54254254254254	4.08395444749164\\
-2.55055055055055	4.12216067800686\\
-2.55855855855856	4.16062922241923\\
-2.56656656656657	4.19936060739832\\
-2.57457457457457	4.23835534982029\\
-2.58258258258258	4.27761395700008\\
-2.59059059059059	4.31713692692547\\
-2.5985985985986	4.35692474849248\\
-2.60660660660661	4.39697790174194\\
-2.61461461461461	4.43729685809669\\
-2.62262262262262	4.47788208059926\\
-2.63063063063063	4.51873402414957\\
-2.63863863863864	4.55985313574239\\
-2.64664664664665	4.60123985470432\\
-2.65465465465465	4.64289461292992\\
-2.66266266266266	4.68481783511674\\
-2.67067067067067	4.72700993899904\\
-2.67867867867868	4.76947133557992\\
-2.68668668668669	4.81220242936158\\
-2.69469469469469	4.85520361857363\\
-2.7027027027027	4.89847529539908\\
-2.71071071071071	4.94201784619789\\
-2.71871871871872	4.98583165172798\\
-2.72672672672673	5.02991708736336\\
-2.73473473473473	5.07427452330937\\
-2.74274274274274	5.11890432481478\\
-2.75075075075075	5.16380685238076\\
-2.75875875875876	5.20898246196637\\
-2.76676676676677	5.25443150519071\\
-2.77477477477477	5.30015432953148\\
-2.78278278278278	5.34615127851984\\
-2.79079079079079	5.39242269193162\\
-2.7987987987988	5.43896890597471\\
-2.80680680680681	5.48579025347256\\
-2.81481481481481	5.53288706404381\\
-2.82282282282282	5.58025966427797\\
-2.83083083083083	5.62790837790712\\
-2.83883883883884	5.67583352597353\\
-2.84684684684685	5.72403542699335\\
-2.85485485485485	5.7725143971162\\
-2.86286286286286	5.82127075028074\\
-2.87087087087087	5.87030479836613\\
-2.87887887887888	5.91961685133957\\
-2.88688688688689	5.96920721739968\\
-2.89489489489489	6.01907620311599\\
-2.9029029029029	6.06922411356439\\
-2.91091091091091	6.1196512524586\\
-2.91891891891892	6.1703579222778\\
-2.92692692692693	6.22134442439031\\
-2.93493493493493	6.27261105917343\\
-2.94294294294294	6.32415812612954\\
-2.95095095095095	6.37598592399839\\
-2.95895895895896	6.42809475086569\\
-2.96696696696697	6.48048490426812\\
-2.97497497497497	6.53315668129463\\
-2.98298298298298	6.58611037868437\\
-2.99099099099099	6.63934629292099\\
-2.998998998999	6.69286472032362\\
-3.00700700700701	6.74666595713446\\
-3.01501501501502	6.80075029960314\\
-3.02302302302302	6.85511804406778\\
-3.03103103103103	6.90976948703298\\
-3.03903903903904	6.96470492524465\\
-3.04704704704705	7.01992465576192\\
-3.05505505505506	7.07542897602599\\
-3.06306306306306	7.13121818392618\\
-3.07107107107107	7.18729257786316\\
-3.07907907907908	7.24365245680938\\
-3.08708708708709	7.30029812036693\\
-3.0950950950951	7.35722986882269\\
-3.1031031031031	7.414448003201\\
-3.11111111111111	7.47195282531388\\
-3.11911911911912	7.52974463780879\\
-3.12712712712713	7.58782374421412\\
-3.13513513513514	7.64619044898235\\
-3.14314314314314	7.70484505753107\\
-3.15115115115115	7.76378787628182\\
-3.15915915915916	7.82301921269687\\
-3.16716716716717	7.88253937531399\\
-3.17517517517518	7.94234867377928\\
-3.18318318318318	8.00244741887811\\
-3.19119119119119	8.0628359225642\\
-3.1991991991992	8.12351449798701\\
-3.20720720720721	8.18448345951732\\
-3.21521521521522	8.24574312277125\\
-3.22322322322322	8.30729380463262\\
-3.23123123123123	8.3691358232738\\
-3.23923923923924	8.43126949817505\\
-3.24724724724725	8.49369515014246\\
-3.25525525525526	8.55641310132445\\
-3.26326326326326	8.61942367522698\\
-3.27127127127127	8.68272719672746\\
-3.27927927927928	8.74632399208741\\
-3.28728728728729	8.81021438896394\\
-3.2952952952953	8.87439871642008\\
-3.3033033033033	8.93887730493404\\
-3.31131131131131	9.0036504864073\\
-3.31931931931932	9.06871859417185\\
-3.32732732732733	9.13408196299629\\
-3.33533533533534	9.1997409290911\\
-3.34334334334334	9.26569583011298\\
-3.35135135135135	9.3319470051683\\
-3.35935935935936	9.39849479481577\\
-3.36736736736737	9.46533954106827\\
-3.37537537537538	9.53248158739401\\
-3.38338338338338	9.59992127871683\\
-3.39139139139139	9.66765896141593\\
-3.3993993993994	9.73569498332492\\
-3.40740740740741	9.80402969373014\\
-3.41541541541542	9.87266344336857\\
-3.42342342342342	9.94159658442496\\
-3.43143143143143	10.0108294705285\\
-3.43943943943944	10.0803624567492\\
-3.44744744744745	10.1501958995931\\
-3.45545545545546	10.2203301569981\\
-3.46346346346346	10.290765588328\\
-3.47147147147147	10.3615025543675\\
-3.47947947947948	10.4325414173157\\
-3.48748748748749	10.5038825407799\\
-3.4954954954955	10.5755262897685\\
-3.5035035035035	10.6474730306844\\
-3.51151151151151	10.7197231313169\\
-3.51951951951952	10.7922769608345\\
-3.52752752752753	10.8651348897763\\
-3.53553553553554	10.9382972900443\\
-3.54354354354354	11.0117645348947\\
-3.55155155155155	11.0855369989285\\
-3.55955955955956	11.1596150580835\\
-3.56756756756757	11.2339990896244\\
-3.57557557557558	11.3086894721336\\
-3.58358358358358	11.3836865855017\\
-3.59159159159159	11.4589908109177\\
-3.5995995995996	11.5346025308593\\
-3.60760760760761	11.6105221290827\\
-3.61561561561562	11.6867499906124\\
-3.62362362362362	11.7632865017314\\
-3.63163163163163	11.8401320499702\\
-3.63963963963964	11.917287024097\\
-3.64764764764765	11.9947518141066\\
-3.65565565565566	12.0725268112104\\
-3.66366366366366	12.1506124078251\\
-3.67167167167167	12.2290089975626\\
-3.67967967967968	12.3077169752187\\
-3.68768768768769	12.3867367367626\\
-3.6956956956957	12.4660686793261\\
-3.7037037037037	12.5457132011923\\
-3.71171171171171	12.6256707017853\\
-3.71971971971972	12.705941581659\\
-3.72772772772773	12.7865262424859\\
-3.73573573573574	12.8674250870469\\
-3.74374374374374	12.9486385192197\\
-3.75175175175175	13.0301669439684\\
-3.75975975975976	13.1120107673322\\
-3.76776776776777	13.194170396415\\
-3.77577577577578	13.2766462393739\\
-3.78378378378378	13.3594387054093\\
-3.79179179179179	13.4425482047533\\
-3.7997997997998	13.5259751486593\\
-3.80780780780781	13.6097199493915\\
-3.81581581581582	13.6937830202138\\
-3.82382382382382	13.7781647753795\\
-3.83183183183183	13.8628656301207\\
-3.83983983983984	13.9478860006376\\
-3.84784784784785	14.0332263040883\\
-3.85585585585586	14.1188869585785\\
-3.86386386386386	14.2048683831505\\
-3.87187187187187	14.2911709977739\\
-3.87987987987988	14.3777952233345\\
-3.88788788788789	14.4647414816249\\
-3.8958958958959	14.5520101953337\\
-3.9039039039039	14.6396017880363\\
-3.91191191191191	14.7275166841843\\
-3.91991991991992	14.8157553090959\\
-3.92792792792793	14.9043180889463\\
-3.93593593593594	14.9932054507576\\
-3.94394394394394	15.0824178223896\\
-3.95195195195195	15.17195563253\\
-3.95995995995996	15.2618193106849\\
-3.96796796796797	15.3520092871696\\
-3.97597597597598	15.4425259930991\\
-3.98398398398398	15.533369860379\\
-3.99199199199199	15.6245413216964\\
-4	15.7160408105107\\
}--cycle;

\addlegendentry{$\pm 2\sigma$};

\addplot [color=mycolor2,solid]
  table[row sep=crcr]{%
-4	10.3323289413691\\
-3.99199199199199	10.2727010070701\\
-3.98398398398398	10.2132698884602\\
-3.97597597597598	10.1540353002636\\
-3.96796796796797	10.0949969572043\\
-3.95995995995996	10.0361545740063\\
-3.95195195195195	9.97750786539361\\
-3.94394394394394	9.91905654609035\\
-3.93593593593594	9.86080033082054\\
-3.92792792792793	9.80273893430822\\
-3.91991991991992	9.74487207127745\\
-3.91191191191191	9.68719945645226\\
-3.9039039039039	9.62972080455669\\
-3.8958958958959	9.5724358303148\\
-3.88788788788789	9.51534424845062\\
-3.87987987987988	9.45844577368821\\
-3.87187187187187	9.40174012075161\\
-3.86386386386386	9.34522700436485\\
-3.85585585585586	9.288906139252\\
-3.84784784784785	9.23277724013708\\
-3.83983983983984	9.17684002174415\\
-3.83183183183183	9.12109419879725\\
-3.82382382382382	9.06553948602043\\
-3.81581581581582	9.01017559813773\\
-3.80780780780781	8.95500224987319\\
-3.7997997997998	8.90001915595086\\
-3.79179179179179	8.84522603109479\\
-3.78378378378378	8.79062259002901\\
-3.77577577577578	8.73620854747758\\
-3.76776776776777	8.68198361816454\\
-3.75975975975976	8.62794751681393\\
-3.75175175175175	8.5740999581498\\
-3.74374374374374	8.52044065689619\\
-3.73573573573574	8.46696932777715\\
-3.72772772772773	8.41368568551672\\
-3.71971971971972	8.36058944483895\\
-3.71171171171171	8.30768032046787\\
-3.7037037037037	8.25495802712755\\
-3.6956956956957	8.20242227954201\\
-3.68768768768769	8.15007279243532\\
-3.67967967967968	8.0979092805315\\
-3.67167167167167	8.04593145855461\\
-3.66366366366366	7.99413904122869\\
-3.65565565565566	7.94253174327778\\
-3.64764764764765	7.89110927942593\\
-3.63963963963964	7.83987136439718\\
-3.63163163163163	7.78881771291559\\
-3.62362362362362	7.73794803970518\\
-3.61561561561562	7.68726205949002\\
-3.60760760760761	7.63675948699414\\
-3.5995995995996	7.58644003694159\\
-3.59159159159159	7.5363034240564\\
-3.58358358358358	7.48634936306264\\
-3.57557557557558	7.43657756868433\\
-3.56756756756757	7.38698775564554\\
-3.55955955955956	7.33757963867029\\
-3.55155155155155	7.28835293248264\\
-3.54354354354354	7.23930735180663\\
-3.53553553553554	7.19044261136631\\
-3.52752752752753	7.14175842588572\\
-3.51951951951952	7.0932545100889\\
-3.51151151151151	7.0449305786999\\
-3.5035035035035	6.99678634644277\\
-3.4954954954955	6.94882152804154\\
-3.48748748748749	6.90103583822027\\
-3.47947947947948	6.85342899170299\\
-3.47147147147147	6.80600070321376\\
-3.46346346346346	6.75875068747662\\
-3.45545545545546	6.71167865921561\\
-3.44744744744745	6.66478433315478\\
-3.43943943943944	6.61806742401817\\
-3.43143143143143	6.57152764652982\\
-3.42342342342342	6.52516471541379\\
-3.41541541541542	6.47897834539411\\
-3.40740740740741	6.43296825119483\\
-3.3993993993994	6.38713414754001\\
-3.39139139139139	6.34147574915366\\
-3.38338338338338	6.29599277075986\\
-3.37537537537538	6.25068492708263\\
-3.36736736736737	6.20555193284603\\
-3.35935935935936	6.1605935027741\\
-3.35135135135135	6.11580935159088\\
-3.34334334334334	6.07119919402042\\
-3.33533533533534	6.02676274478677\\
-3.32732732732733	5.98249971861396\\
-3.31931931931932	5.93840983022605\\
-3.31131131131131	5.89449279434707\\
-3.3033033033033	5.85074832570107\\
-3.2952952952953	5.8071761390121\\
-3.28728728728729	5.76377594900421\\
-3.27927927927928	5.72054747040143\\
-3.27127127127127	5.67749041792781\\
-3.26326326326326	5.63460450630739\\
-3.25525525525526	5.59188945026423\\
-3.24724724724725	5.54934496452236\\
-3.23923923923924	5.50697076380584\\
-3.23123123123123	5.46476656283869\\
-3.22322322322322	5.42273207634498\\
-3.21521521521522	5.38086701904874\\
-3.20720720720721	5.33917110567402\\
-3.1991991991992	5.29764405094486\\
-3.19119119119119	5.25628556958531\\
-3.18318318318318	5.21509537631942\\
-3.17517517517518	5.17407318587122\\
-3.16716716716717	5.13321871296476\\
-3.15915915915916	5.09253167232409\\
-3.15115115115115	5.05201177867326\\
-3.14314314314314	5.0116587467363\\
-3.13513513513514	4.97147229123726\\
-3.12712712712713	4.93145212690019\\
-3.11911911911912	4.89159796844912\\
-3.11111111111111	4.85190953060812\\
-3.1031031031031	4.81238652810121\\
-3.0950950950951	4.77302867565245\\
-3.08708708708709	4.73383568798588\\
-3.07907907907908	4.69480727982554\\
-3.07107107107107	4.65594316589548\\
-3.06306306306306	4.61724306091975\\
-3.05505505505506	4.57870667962238\\
-3.04704704704705	4.54033373672743\\
-3.03903903903904	4.50212394695894\\
-3.03103103103103	4.46407702504094\\
-3.02302302302302	4.4261926856975\\
-3.01501501501502	4.38847064365265\\
-3.00700700700701	4.35091061363043\\
-2.998998998999	4.3135123103549\\
-2.99099099099099	4.27627544855009\\
-2.98298298298298	4.23919974294005\\
-2.97497497497497	4.20228490824883\\
-2.96696696696697	4.16553065920047\\
-2.95895895895896	4.12893671051901\\
-2.95095095095095	4.09250277692851\\
-2.94294294294294	4.056228573153\\
-2.93493493493493	4.02011381391652\\
-2.92692692692693	3.98415821394314\\
-2.91891891891892	3.94836148795688\\
-2.91091091091091	3.91272335068179\\
-2.9029029029029	3.87724351684192\\
-2.89489489489489	3.84192170116132\\
-2.88688688688689	3.80675761836402\\
-2.87887887887888	3.77175098317408\\
-2.87087087087087	3.73690151031553\\
-2.86286286286286	3.70220891451242\\
-2.85485485485485	3.6676729104888\\
-2.84684684684685	3.63329321296872\\
-2.83883883883884	3.5990695366762\\
-2.83083083083083	3.56500159633531\\
-2.82282282282282	3.53108910667009\\
-2.81481481481481	3.49733178240457\\
-2.80680680680681	3.46372933826281\\
-2.7987987987988	3.43028148896885\\
-2.79079079079079	3.39698794924673\\
-2.78278278278278	3.3638484338205\\
-2.77477477477477	3.33086265741421\\
-2.76676676676677	3.29803033475189\\
-2.75875875875876	3.2653511805576\\
-2.75075075075075	3.23282490955537\\
-2.74274274274274	3.20045123646926\\
-2.73473473473473	3.1682298760233\\
-2.72672672672673	3.13616054294155\\
-2.71871871871872	3.10424295194804\\
-2.71071071071071	3.07247681776682\\
-2.7027027027027	3.04086185512195\\
-2.69469469469469	3.00939777873745\\
-2.68668668668669	2.97808430333738\\
-2.67867867867868	2.94692114364577\\
-2.67067067067067	2.91590801438668\\
-2.66266266266266	2.88504463028416\\
-2.65465465465465	2.85433070606223\\
-2.64664664664665	2.82376595644496\\
-2.63863863863864	2.79335009615638\\
-2.63063063063063	2.76308283992054\\
-2.62262262262262	2.73296390246148\\
-2.61461461461461	2.70299299850325\\
-2.60660660660661	2.67316984276989\\
-2.5985985985986	2.64349414998545\\
-2.59059059059059	2.61396563487398\\
-2.58258258258258	2.5845840121595\\
-2.57457457457457	2.55534899656609\\
-2.56656656656657	2.52626030281776\\
-2.55855855855856	2.49731764563858\\
-2.55055055055055	2.46852073975259\\
-2.54254254254254	2.43986929988382\\
-2.53453453453453	2.41136304075633\\
-2.52652652652653	2.38300167709416\\
-2.51851851851852	2.35478492362136\\
-2.51051051051051	2.32671249506196\\
-2.5025025025025	2.29878410614002\\
-2.49449449449449	2.27099947157958\\
-2.48648648648649	2.24335830610468\\
-2.47847847847848	2.21586032443937\\
-2.47047047047047	2.18850524130769\\
-2.46246246246246	2.16129277143369\\
-2.45445445445445	2.13422262954141\\
-2.44644644644645	2.1072945303549\\
-2.43843843843844	2.0805081885982\\
-2.43043043043043	2.05386331899536\\
-2.42242242242242	2.02735963627042\\
-2.41441441441441	2.00099685514743\\
-2.40640640640641	1.97477469035042\\
-2.3983983983984	1.94869285660345\\
-2.39039039039039	1.92275106863057\\
-2.38238238238238	1.8969490411558\\
-2.37437437437437	1.87128648890321\\
-2.36636636636637	1.84576312659683\\
-2.35835835835836	1.82037866896071\\
-2.35035035035035	1.7951328307189\\
-2.34234234234234	1.77002532659543\\
-2.33433433433433	1.74505587131435\\
-2.32632632632633	1.72022417959972\\
-2.31831831831832	1.69552996617556\\
-2.31031031031031	1.67097294576594\\
-2.3023023023023	1.64655283309488\\
-2.29429429429429	1.62226934288645\\
-2.28628628628629	1.59812218986467\\
-2.27827827827828	1.5741110887536\\
-2.27027027027027	1.55023575427728\\
-2.26226226226226	1.52649590115976\\
-2.25425425425425	1.50289124412508\\
-2.24624624624625	1.47942149789728\\
-2.23823823823824	1.45608637720041\\
-2.23023023023023	1.43288559675852\\
-2.22222222222222	1.40981887129565\\
-2.21421421421421	1.38688591553584\\
-2.20620620620621	1.36408644420314\\
-2.1981981981982	1.34142017202159\\
-2.19019019019019	1.31888681371524\\
-2.18218218218218	1.29648608400814\\
-2.17417417417417	1.27421769762432\\
-2.16616616616617	1.25208136928783\\
-2.15815815815816	1.23007681372272\\
-2.15015015015015	1.20820374565304\\
-2.14214214214214	1.18646187980282\\
-2.13413413413413	1.16485093089611\\
-2.12612612612613	1.14337061365695\\
-2.11811811811812	1.1220206428094\\
-2.11011011011011	1.10080073307749\\
-2.1021021021021	1.07971059918528\\
-2.09409409409409	1.0587499558568\\
-2.08608608608609	1.03791851781609\\
-2.07807807807808	1.01721599978721\\
-2.07007007007007	0.996642116494204\\
-2.06206206206206	0.976196582661109\\
-2.05405405405405	0.955879113011971\\
-2.04604604604605	0.935689422270837\\
-2.03803803803804	0.91562722516175\\
-2.03003003003003	0.895692236408757\\
-2.02202202202202	0.875884170735901\\
-2.01401401401401	0.856202742867226\\
-2.00600600600601	0.836647667526779\\
-1.997997997998	0.817218659438602\\
-1.98998998998999	0.797915433326742\\
-1.98198198198198	0.778737703915243\\
-1.97397397397397	0.75968518592815\\
-1.96596596596597	0.740757594089507\\
-1.95795795795796	0.721954643123358\\
-1.94994994994995	0.70327604775375\\
-1.94194194194194	0.684721522704727\\
-1.93393393393393	0.666290782700332\\
-1.92592592592593	0.647983542464611\\
-1.91791791791792	0.629799516721609\\
-1.90990990990991	0.611738420195371\\
-1.9019019019019	0.59379996760994\\
-1.89389389389389	0.575983873689362\\
-1.88588588588589	0.558289853157682\\
-1.87787787787788	0.540717620738944\\
-1.86986986986987	0.523266891157193\\
-1.86186186186186	0.505937379136473\\
-1.85385385385385	0.488728799400831\\
-1.84584584584585	0.47164086667431\\
-1.83783783783784	0.454673295680953\\
-1.82982982982983	0.437825801144808\\
-1.82182182182182	0.421098097789918\\
-1.81381381381381	0.404489900340328\\
-1.80580580580581	0.388000923520083\\
-1.7977977977978	0.371630882053227\\
-1.78978978978979	0.355379490663806\\
-1.78178178178178	0.339246464075862\\
-1.77377377377377	0.323231517013444\\
-1.76576576576577	0.307334364200593\\
-1.75775775775776	0.291554720361354\\
-1.74974974974975	0.275892300219775\\
-1.74174174174174	0.260346818499897\\
-1.73373373373373	0.244917989925767\\
-1.72572572572573	0.229605529221429\\
-1.71771771771772	0.214409151110927\\
-1.70970970970971	0.199328570318307\\
-1.7017017017017	0.184363501567611\\
-1.69369369369369	0.169513659582888\\
-1.68568568568569	0.154778759088181\\
-1.67767767767768	0.140158514807532\\
-1.66966966966967	0.12565264146499\\
-1.66166166166166	0.111260853784596\\
-1.65365365365365	0.0969828664903977\\
-1.64564564564565	0.0828183943064382\\
-1.63763763763764	0.0687671519567619\\
-1.62962962962963	0.0548288541654148\\
-1.62162162162162	0.0410032156564403\\
-1.61361361361361	0.0272899511538847\\
-1.60560560560561	0.0136887753817908\\
-1.5975975975976	0.000199403064205461\\
-1.58958958958959	-0.013178451074828\\
-1.58158158158158	-0.0264450723112651\\
-1.57357357357357	-0.0396007459210603\\
-1.56556556556557	-0.0526457571801697\\
-1.55755755755756	-0.0655803913645475\\
-1.54954954954955	-0.0784049337501495\\
-1.54154154154154	-0.0911196696129316\\
-1.53353353353353	-0.103724884228848\\
-1.52552552552553	-0.116220862873855\\
-1.51751751751752	-0.128607890823907\\
-1.50950950950951	-0.140886253354959\\
-1.5015015015015	-0.153056235742968\\
-1.49349349349349	-0.165118123263888\\
-1.48548548548549	-0.177072201193674\\
-1.47747747747748	-0.188918754808282\\
-1.46946946946947	-0.200658069383667\\
-1.46146146146146	-0.212290430195785\\
-1.45345345345345	-0.22381612252059\\
-1.44544544544545	-0.235235431634039\\
-1.43743743743744	-0.246548642812084\\
-1.42942942942943	-0.257756041330684\\
-1.42142142142142	-0.268857912465793\\
-1.41341341341341	-0.279854541493365\\
-1.40540540540541	-0.290746213689357\\
-1.3973973973974	-0.301533214329723\\
-1.38938938938939	-0.312215828690419\\
-1.38138138138138	-0.322794342047401\\
-1.37337337337337	-0.333269039676623\\
-1.36536536536537	-0.34364020685404\\
-1.35735735735736	-0.353908128855609\\
-1.34934934934935	-0.364073090957284\\
-1.34134134134134	-0.37413537843502\\
-1.33333333333333	-0.384095276564773\\
-1.32532532532533	-0.393953070622499\\
-1.31731731731732	-0.403709045884151\\
-1.30930930930931	-0.413363487625687\\
-1.3013013013013	-0.42291668112306\\
-1.29329329329329	-0.432368911652226\\
-1.28528528528529	-0.441720464489141\\
-1.27727727727728	-0.45097162490976\\
-1.26926926926927	-0.460122678190038\\
-1.26126126126126	-0.469173909605931\\
-1.25325325325325	-0.478125604433392\\
-1.24524524524525	-0.486978047948379\\
-1.23723723723724	-0.495731525426846\\
-1.22922922922923	-0.504386322144749\\
-1.22122122122122	-0.512942723378042\\
-1.21321321321321	-0.521401014402682\\
-1.20520520520521	-0.529761480494623\\
-1.1971971971972	-0.538024406929819\\
-1.18918918918919	-0.546190078984229\\
-1.18118118118118	-0.554258781933805\\
-1.17317317317317	-0.562230801054504\\
-1.16516516516517	-0.57010642162228\\
-1.15715715715716	-0.57788592891309\\
-1.14914914914915	-0.585569608202888\\
-1.14114114114114	-0.593157744767629\\
-1.13313313313313	-0.600650623883269\\
-1.12512512512513	-0.608048530825764\\
-1.11711711711712	-0.615351750871067\\
-1.10910910910911	-0.622560569295136\\
-1.1011011011011	-0.629675271373924\\
-1.09309309309309	-0.636696142383388\\
-1.08508508508509	-0.643623467599482\\
-1.07707707707708	-0.650457532298162\\
-1.06906906906907	-0.657198621755383\\
-1.06106106106106	-0.663847021247101\\
-1.05305305305305	-0.67040301604927\\
-1.04504504504505	-0.676866891437846\\
-1.03703703703704	-0.683238932688784\\
-1.02902902902903	-0.68951942507804\\
-1.02102102102102	-0.695708653881569\\
-1.01301301301301	-0.701806904375325\\
-1.00500500500501	-0.707814461835265\\
-0.996996996996997	-0.713731611537344\\
-0.988988988988989	-0.719558638757516\\
-0.980980980980981	-0.725295828771738\\
-0.972972972972973	-0.730943466855964\\
-0.964964964964965	-0.73650183828615\\
-0.956956956956957	-0.741971228338251\\
-0.948948948948949	-0.747351922288222\\
-0.940940940940941	-0.752644205412019\\
-0.932932932932933	-0.757848362985597\\
-0.924924924924925	-0.762964680284911\\
-0.916916916916917	-0.767993442585916\\
-0.908908908908909	-0.772934935164568\\
-0.900900900900901	-0.777789443296821\\
-0.892892892892893	-0.782557252258633\\
-0.884884884884885	-0.787238647325956\\
-0.876876876876877	-0.791833913774748\\
-0.868868868868869	-0.796343336880963\\
-0.860860860860861	-0.800767201920556\\
-0.852852852852853	-0.805105794169483\\
-0.844844844844845	-0.809359398903698\\
-0.836836836836837	-0.813528301399159\\
-0.828828828828829	-0.817612786931818\\
-0.820820820820821	-0.821613140777632\\
-0.812812812812813	-0.825529648212557\\
-0.804804804804805	-0.829362594512546\\
-0.796796796796797	-0.833112264953557\\
-0.788788788788789	-0.836778944811543\\
-0.780780780780781	-0.84036291936246\\
-0.772772772772773	-0.843864473882264\\
-0.764764764764765	-0.84728389364691\\
-0.756756756756757	-0.850621463932353\\
-0.748748748748749	-0.853877470014548\\
-0.740740740740741	-0.85705219716945\\
-0.732732732732733	-0.860145930673016\\
-0.724724724724725	-0.8631589558012\\
-0.716716716716717	-0.866091557829957\\
-0.708708708708709	-0.868944022035243\\
-0.700700700700701	-0.871716633693012\\
-0.692692692692693	-0.874409678079222\\
-0.684684684684685	-0.877023440469825\\
-0.676676676676677	-0.879558206140779\\
-0.668668668668669	-0.882014260368038\\
-0.660660660660661	-0.884391888427557\\
-0.652652652652653	-0.886691375595292\\
-0.644644644644645	-0.888913007147197\\
-0.636636636636636	-0.891057068359229\\
-0.628628628628629	-0.893123844507343\\
-0.62062062062062	-0.895113620867493\\
-0.612612612612613	-0.897026682715636\\
-0.604604604604605	-0.898863315327726\\
-0.596596596596596	-0.900623803979718\\
-0.588588588588589	-0.902308433947569\\
-0.58058058058058	-0.903917490507232\\
-0.572572572572573	-0.905451258934664\\
-0.564564564564565	-0.90691002450582\\
-0.556556556556556	-0.908294072496655\\
-0.548548548548549	-0.909603688183125\\
-0.54054054054054	-0.910839156841183\\
-0.532532532532533	-0.912000763746787\\
-0.524524524524525	-0.913088794175891\\
-0.516516516516516	-0.914103533404451\\
-0.508508508508509	-0.915045266708421\\
-0.5005005005005	-0.915914279363757\\
-0.492492492492492	-0.916710856646415\\
-0.484484484484485	-0.917435283832349\\
-0.476476476476476	-0.918087846197515\\
-0.468468468468469	-0.918668829017868\\
-0.46046046046046	-0.919178517569364\\
-0.452452452452452	-0.919617197127957\\
-0.444444444444445	-0.919985152969603\\
-0.436436436436436	-0.920282670370258\\
-0.428428428428429	-0.920510034605876\\
-0.42042042042042	-0.920667530952413\\
-0.412412412412412	-0.920755444685824\\
-0.404404404404405	-0.920774061082064\\
-0.396396396396396	-0.920723665417089\\
-0.388388388388389	-0.920604542966854\\
-0.38038038038038	-0.920416979007314\\
-0.372372372372372	-0.920161258814425\\
-0.364364364364364	-0.919837667664142\\
-0.356356356356356	-0.919446490832419\\
-0.348348348348348	-0.918988013595213\\
-0.34034034034034	-0.918462521228479\\
-0.332332332332332	-0.917870299008172\\
-0.324324324324324	-0.917211632210246\\
-0.316316316316316	-0.916486806110659\\
-0.308308308308308	-0.915696105985364\\
-0.3003003003003	-0.914839817110317\\
-0.292292292292292	-0.913918224761473\\
-0.284284284284284	-0.912931614214788\\
-0.276276276276276	-0.911880270746217\\
-0.268268268268268	-0.910764479631715\\
-0.26026026026026	-0.909584526147237\\
-0.252252252252252	-0.908340695568739\\
-0.244244244244244	-0.907033273172176\\
-0.236236236236236	-0.905662544233504\\
-0.228228228228228	-0.904228794028677\\
-0.22022022022022	-0.902732307833651\\
-0.212212212212212	-0.901173370924381\\
-0.204204204204204	-0.899552268576822\\
-0.196196196196196	-0.89786928606693\\
-0.188188188188188	-0.89612470867066\\
-0.18018018018018	-0.894318821663967\\
-0.172172172172172	-0.892451910322807\\
-0.164164164164164	-0.890524259923135\\
-0.156156156156156	-0.888536155740906\\
-0.148148148148148	-0.886487883052075\\
-0.14014014014014	-0.884379727132597\\
-0.132132132132132	-0.882211973258429\\
-0.124124124124124	-0.879984906705525\\
-0.116116116116116	-0.87769881274984\\
-0.108108108108108	-0.87535397666733\\
-0.1001001001001	-0.872950683733951\\
-0.0920920920920922	-0.870489219225656\\
-0.084084084084084	-0.867969868418402\\
-0.0760760760760761	-0.865392916588145\\
-0.0680680680680679	-0.862758649010838\\
-0.06006006006006	-0.860067350962437\\
-0.0520520520520522	-0.857319307718899\\
-0.0440440440440439	-0.854514804556177\\
-0.0360360360360361	-0.851654126750228\\
-0.0280280280280278	-0.848737559577006\\
-0.02002002002002	-0.845765388312468\\
-0.0120120120120122	-0.842737898232567\\
-0.00400400400400391	-0.83965537461326\\
0.00400400400400436	-0.836518102730501\\
0.0120120120120122	-0.833326367860247\\
0.02002002002002	-0.830080455278451\\
0.0280280280280278	-0.826780650261071\\
0.0360360360360357	-0.82342723808406\\
0.0440440440440444	-0.820020504023374\\
0.0520520520520522	-0.816560733354969\\
0.06006006006006	-0.813048211354799\\
0.0680680680680679	-0.80948322329882\\
0.0760760760760757	-0.805866054462988\\
0.0840840840840844	-0.802196990123257\\
0.0920920920920922	-0.798476315555582\\
0.1001001001001	-0.79470431603592\\
0.108108108108108	-0.790881276840226\\
0.116116116116116	-0.787007483244453\\
0.124124124124124	-0.783083220524559\\
0.132132132132132	-0.779108773956498\\
0.14014014014014	-0.775084428816225\\
0.148148148148148	-0.771010470379696\\
0.156156156156156	-0.766887183922867\\
0.164164164164164	-0.762714854721691\\
0.172172172172172	-0.758493768052125\\
0.18018018018018	-0.754224209190124\\
0.188188188188188	-0.749906463411644\\
0.196196196196196	-0.745540815992638\\
0.204204204204204	-0.741127552209063\\
0.212212212212212	-0.736666957336875\\
0.22022022022022	-0.732159316652027\\
0.228228228228228	-0.727604915430477\\
0.236236236236236	-0.723004038948178\\
0.244244244244245	-0.718356972481085\\
0.252252252252252	-0.713664001305156\\
0.26026026026026	-0.708925410696344\\
0.268268268268268	-0.704141485930606\\
0.276276276276277	-0.699312512283895\\
0.284284284284285	-0.694438775032168\\
0.292292292292292	-0.68952055945138\\
0.3003003003003	-0.684558150817486\\
0.308308308308308	-0.679551834406441\\
0.316316316316317	-0.674501895494201\\
0.324324324324325	-0.66940861935672\\
0.332332332332332	-0.664272291269956\\
0.34034034034034	-0.659093196509861\\
0.348348348348348	-0.653871620352393\\
0.356356356356357	-0.648607848073504\\
0.364364364364365	-0.643302164949153\\
0.372372372372372	-0.637954856255294\\
0.38038038038038	-0.632566207267881\\
0.388388388388388	-0.627136503262871\\
0.396396396396397	-0.621666029516217\\
0.404404404404405	-0.616155071303877\\
0.412412412412412	-0.610603913901804\\
0.42042042042042	-0.605012842585955\\
0.428428428428428	-0.599382142632285\\
0.436436436436437	-0.593712099316748\\
0.444444444444445	-0.5880029979153\\
0.452452452452452	-0.582255123703897\\
0.46046046046046	-0.576468761958494\\
0.468468468468468	-0.570644197955046\\
0.476476476476477	-0.564781716969507\\
0.484484484484485	-0.558881604277835\\
0.492492492492492	-0.552944145155983\\
0.5005005005005	-0.546969624879908\\
0.508508508508508	-0.540958328725563\\
0.516516516516517	-0.534910541968905\\
0.524524524524525	-0.52882654988589\\
0.532532532532533	-0.522706637752471\\
0.54054054054054	-0.516551090844605\\
0.548548548548548	-0.510360194438247\\
0.556556556556557	-0.504134233809352\\
0.564564564564565	-0.497873494233875\\
0.572572572572573	-0.491578260987772\\
0.58058058058058	-0.485248819346998\\
0.588588588588588	-0.478885454587508\\
0.596596596596597	-0.472488451985257\\
0.604604604604605	-0.466058096816202\\
0.612612612612613	-0.459594674356296\\
0.62062062062062	-0.453098469881496\\
0.628628628628628	-0.446569768667756\\
0.636636636636637	-0.440008855991032\\
0.644644644644645	-0.433416017127279\\
0.652652652652653	-0.426791537352453\\
0.66066066066066	-0.420135701942508\\
0.668668668668668	-0.413448796173401\\
0.676676676676677	-0.406731105321085\\
0.684684684684685	-0.399982914661517\\
0.692692692692693	-0.393204509470653\\
0.7007007007007	-0.386396175024446\\
0.708708708708708	-0.379558196598853\\
0.716716716716717	-0.372690859469827\\
0.724724724724725	-0.365794448913327\\
0.732732732732733	-0.358869250205306\\
0.74074074074074	-0.351915548621719\\
0.748748748748748	-0.344933629438522\\
0.756756756756757	-0.337923777931669\\
0.764764764764765	-0.330886279377117\\
0.772772772772773	-0.323821419050821\\
0.780780780780781	-0.316729482228737\\
0.788788788788789	-0.309610754186817\\
0.796796796796797	-0.30246552020102\\
0.804804804804805	-0.295294065547299\\
0.812812812812813	-0.288096675501611\\
0.820820820820821	-0.28087363533991\\
0.828828828828829	-0.27362523033815\\
0.836836836836837	-0.26635174577229\\
0.844844844844845	-0.259053466918282\\
0.852852852852853	-0.251730679052083\\
0.860860860860861	-0.244383667449648\\
0.868868868868869	-0.237012717386931\\
0.876876876876877	-0.229618114139889\\
0.884884884884885	-0.222200142984477\\
0.892892892892893	-0.214759089196649\\
0.900900900900901	-0.207295238052362\\
0.908908908908909	-0.199808874827569\\
0.916916916916917	-0.192300284798228\\
0.924924924924925	-0.184769753240293\\
0.932932932932933	-0.177217565429719\\
0.940940940940941	-0.169644006642461\\
0.948948948948949	-0.162049362154475\\
0.956956956956957	-0.154433917241716\\
0.964964964964965	-0.14679795718014\\
0.972972972972973	-0.139141767245702\\
0.980980980980981	-0.131465632714356\\
0.988988988988989	-0.123769838862058\\
0.996996996996997	-0.116054670964764\\
1.00500500500501	-0.108320414298429\\
1.01301301301301	-0.100567354139008\\
1.02102102102102	-0.0927957757624568\\
1.02902902902903	-0.085005964444729\\
1.03703703703704	-0.077198205461782\\
1.04504504504505	-0.0693727840895701\\
1.05305305305305	-0.0615299856040489\\
1.06106106106106	-0.0536700952811733\\
1.06906906906907	-0.0457933983968978\\
1.07707707707708	-0.0379001802271795\\
1.08508508508509	-0.0299907260479729\\
1.09309309309309	-0.022065321135233\\
1.1011011011011	-0.0141242507649154\\
1.10910910910911	-0.00616780021297431\\
1.11711711711712	0.00180374524463317\\
1.12512512512513	0.0097901003319526\\
1.13313313313313	0.0177909797730288\\
1.14114114114114	0.0258060982919064\\
1.14914914914915	0.033835170612631\\
1.15715715715716	0.0418779114592458\\
1.16516516516517	0.049934035555796\\
1.17317317317317	0.0580032576263266\\
1.18118118118118	0.0660852923948822\\
1.18918918918919	0.0741798545855084\\
1.1971971971972	0.0822866589222484\\
1.20520520520521	0.0904054201291473\\
1.21321321321321	0.0985358529302502\\
1.22122122122122	0.106677672049602\\
1.22922922922923	0.114830592211248\\
1.23723723723724	0.122994328139231\\
1.24524524524525	0.131168594557597\\
1.25325325325325	0.13935310619039\\
1.26126126126126	0.147547577761656\\
1.26926926926927	0.155751723995439\\
1.27727727727728	0.163965259615783\\
1.28528528528529	0.172187899346734\\
1.29329329329329	0.180419357912335\\
1.3013013013013	0.188659350036634\\
1.30930930930931	0.196907590443672\\
1.31731731731732	0.205163793857495\\
1.32532532532533	0.213427675002148\\
1.33333333333333	0.221698948601676\\
1.34134134134134	0.229977329380125\\
1.34934934934935	0.238262532061536\\
1.35735735735736	0.246554271369956\\
1.36536536536537	0.254852262029431\\
1.37337337337337	0.263156218764003\\
1.38138138138138	0.271465856297719\\
1.38938938938939	0.279780889354622\\
1.3973973973974	0.288101032658758\\
1.40540540540541	0.29642600093417\\
1.41341341341341	0.304755508904905\\
1.42142142142142	0.313089271295007\\
1.42942942942943	0.32142700282852\\
1.43743743743744	0.329768418229488\\
1.44544544544545	0.338113232221958\\
1.45345345345345	0.346461159529973\\
1.46146146146146	0.354811914877579\\
1.46946946946947	0.363165212988818\\
1.47747747747748	0.371520768587738\\
1.48548548548549	0.379878296398382\\
1.49349349349349	0.388237511144795\\
1.5015015015015	0.396598127551023\\
1.50950950950951	0.404959860341109\\
1.51751751751752	0.413322424239097\\
1.52552552552553	0.421685533969034\\
1.53353353353353	0.430048904254963\\
1.54154154154154	0.438412249820931\\
1.54954954954955	0.44677528539098\\
1.55755755755756	0.455137725689156\\
1.56556556556557	0.463499285439503\\
1.57357357357357	0.471859679366067\\
1.58158158158158	0.480218622192892\\
1.58958958958959	0.488575828644022\\
1.5975975975976	0.496931013443503\\
1.60560560560561	0.505283891315379\\
1.61361361361361	0.513634176983695\\
1.62162162162162	0.521981585172496\\
1.62962962962963	0.530325830605826\\
1.63763763763764	0.53866662800773\\
1.64564564564565	0.547003692102253\\
1.65365365365365	0.555336737613438\\
1.66166166166166	0.563665479265334\\
1.66966966966967	0.571989631781981\\
1.67767767767768	0.580308909887426\\
1.68568568568569	0.588623028305712\\
1.69369369369369	0.596931701760887\\
1.7017017017017	0.605234644976994\\
1.70970970970971	0.613531572678076\\
1.71771771771772	0.62182219958818\\
1.72572572572573	0.63010624043135\\
1.73373373373373	0.63838340993163\\
1.74174174174174	0.646653422813066\\
1.74974974974975	0.654915993799702\\
1.75775775775776	0.663170837615583\\
1.76576576576577	0.671417668984753\\
1.77377377377377	0.679656202631258\\
1.78178178178178	0.687886153279142\\
1.78978978978979	0.696107235652449\\
1.7977977977978	0.704319164475225\\
1.80580580580581	0.712521654471514\\
1.81381381381381	0.720714420365361\\
1.82182182182182	0.72889717688081\\
1.82982982982983	0.737069638741907\\
1.83783783783784	0.745231520672695\\
1.84584584584585	0.75338253739722\\
1.85385385385385	0.761522403639528\\
1.86186186186186	0.769650834123661\\
1.86986986986987	0.777767543573665\\
1.87787787787788	0.785872246713584\\
1.88588588588589	0.793964658267464\\
1.89389389389389	0.80204449295935\\
1.9019019019019	0.810111465513284\\
1.90990990990991	0.818165290653313\\
1.91791791791792	0.826205683103481\\
1.92592592592593	0.834232357587834\\
1.93393393393393	0.842245028830416\\
1.94194194194194	0.85024341155527\\
1.94994994994995	0.858227220486442\\
1.95795795795796	0.866196170347978\\
1.96596596596597	0.87414997586392\\
1.97397397397397	0.882088351758316\\
1.98198198198198	0.890011012755208\\
1.98998998998999	0.897917673578642\\
1.997997997998	0.905808048952662\\
2.00600600600601	0.913681853601313\\
2.01401401401401	0.921538802248641\\
2.02202202202202	0.929378609618688\\
2.03003003003003	0.937200990435501\\
2.03803803803804	0.945005659423124\\
2.04604604604605	0.952792331305601\\
2.05405405405405	0.960560720806979\\
2.06206206206206	0.9683105426513\\
2.07007007007007	0.97604151156261\\
2.07807807807808	0.983753342264954\\
2.08608608608609	0.991445749482376\\
2.09409409409409	0.999118447938922\\
2.1021021021021	1.00677115235863\\
2.11011011011011	1.01440357746556\\
2.11811811811812	1.02201543798374\\
2.12612612612613	1.02960644863723\\
2.13413413413413	1.03717632415006\\
2.14214214214214	1.04472477924628\\
2.15015015015015	1.05225152864994\\
2.15815815815816	1.05975628708508\\
2.16616616616617	1.06723876927574\\
2.17417417417417	1.07469868994598\\
2.18218218218218	1.08213576381983\\
2.19019019019019	1.08954970562134\\
2.1981981981982	1.09694023007455\\
2.20620620620621	1.10430705190352\\
2.21421421421421	1.11164988583227\\
2.22222222222222	1.11896844658487\\
2.23023023023023	1.12626244888535\\
2.23823823823824	1.13353160745776\\
2.24624624624625	1.14077563702614\\
2.25425425425425	1.14799425231453\\
2.26226226226226	1.15518716804699\\
2.27027027027027	1.16235409894756\\
2.27827827827828	1.16949475974027\\
2.28628628628629	1.17660886514919\\
2.29429429429429	1.18369612989834\\
2.3023023023023	1.19075626871179\\
2.31031031031031	1.19778899631356\\
2.31831831831832	1.2047940274277\\
2.32632632632633	1.21177107677827\\
2.33433433433433	1.2187198590893\\
2.34234234234234	1.22564008908484\\
2.35035035035035	1.23253148148893\\
2.35835835835836	1.23939375102563\\
2.36636636636637	1.24622661241896\\
2.37437437437437	1.25302978039299\\
2.38238238238238	1.25980296967175\\
2.39039039039039	1.26654589497928\\
2.3983983983984	1.27325827103964\\
2.40640640640641	1.27993981257686\\
2.41441441441441	1.286590234315\\
2.42242242242242	1.29320925097809\\
2.43043043043043	1.29979657729019\\
2.43843843843844	1.30635192797533\\
2.44644644644645	1.31287501775756\\
2.45445445445445	1.31936556136093\\
2.46246246246246	1.32582327350947\\
2.47047047047047	1.33224786892724\\
2.47847847847848	1.33863906233829\\
2.48648648648649	1.34499656846664\\
2.49449449449449	1.35132010203636\\
2.5025025025025	1.35760937777147\\
2.51051051051051	1.36386411039604\\
2.51851851851852	1.3700840146341\\
2.52652652652653	1.3762688052097\\
2.53453453453453	1.38241819684688\\
2.54254254254254	1.38853190426969\\
2.55055055055055	1.39460964220216\\
2.55855855855856	1.40065112536836\\
2.56656656656657	1.40665606849232\\
2.57457457457457	1.41262418629808\\
2.58258258258258	1.4185551935097\\
2.59059059059059	1.42444880485121\\
2.5985985985986	1.43030473504666\\
2.60660660660661	1.4361226988201\\
2.61461461461461	1.44190241089556\\
2.62262262262262	1.4476435859971\\
2.63063063063063	1.45334593884876\\
2.63863863863864	1.45900918417458\\
2.64664664664665	1.46463303669862\\
2.65465465465465	1.4702172111449\\
2.66266266266266	1.47576142223749\\
2.67067067067067	1.48126538470041\\
2.67867867867868	1.48672881325773\\
2.68668668668669	1.49215142263347\\
2.69469469469469	1.4975329275517\\
2.7027027027027	1.50287304273644\\
2.71071071071071	1.50817148291175\\
2.71871871871872	1.51342796280168\\
2.72672672672673	1.51864219713025\\
2.73473473473473	1.52381390062153\\
2.74274274274274	1.52894278799956\\
2.75075075075075	1.53402857398837\\
2.75875875875876	1.53907097331202\\
2.76676676676677	1.54406970069455\\
2.77477477477477	1.54902447086\\
2.78278278278278	1.55393499853242\\
2.79079079079079	1.55880099843586\\
2.7987987987988	1.56362218529435\\
2.80680680680681	1.56839827383195\\
2.81481481481481	1.57312897877269\\
2.82282282282282	1.57781401484063\\
2.83083083083083	1.58245309675981\\
2.83883883883884	1.58704593925426\\
2.84684684684685	1.59159225704804\\
2.85485485485485	1.5960917648652\\
2.86286286286286	1.60054417742977\\
2.87087087087087	1.6049492094658\\
2.87887887887888	1.60930657569734\\
2.88688688688689	1.61361599084842\\
2.89489489489489	1.61787716964311\\
2.9029029029029	1.62208982680543\\
2.91091091091091	1.62625367705943\\
2.91891891891892	1.63036843512917\\
2.92692692692693	1.63443381573868\\
2.93493493493493	1.638449533612\\
2.94294294294294	1.6424153034732\\
2.95095095095095	1.64633084004629\\
2.95895895895896	1.65019585805535\\
2.96696696696697	1.65401007222439\\
2.97497497497497	1.65777319727749\\
2.98298298298298	1.66148494793866\\
2.99099099099099	1.66514503893197\\
2.998998998999	1.66875318498146\\
3.00700700700701	1.67230910081117\\
3.01501501501502	1.67581250114514\\
3.02302302302302	1.67926310070742\\
3.03103103103103	1.68266061422206\\
3.03903903903904	1.6860047564131\\
3.04704704704705	1.68929524200458\\
3.05505505505506	1.69253178572056\\
3.06306306306306	1.69571410228506\\
3.07107107107107	1.69884190642215\\
3.07907907907908	1.70191491285585\\
3.08708708708709	1.70493283631023\\
3.0950950950951	1.70789539150932\\
3.1031031031031	1.71080229317717\\
3.11111111111111	1.71365325603782\\
3.11911911911912	1.71644799481532\\
3.12712712712713	1.71918622423371\\
3.13513513513514	1.72186765901703\\
3.14314314314314	1.72449201388934\\
3.15115115115115	1.72705900357467\\
3.15915915915916	1.72956834279708\\
3.16716716716717	1.73201974628059\\
3.17517517517518	1.73441292874927\\
3.18318318318318	1.73674760492716\\
3.19119119119119	1.73902348953829\\
3.1991991991992	1.74124029730672\\
3.20720720720721	1.74339774295649\\
3.21521521521522	1.74549554121164\\
3.22322322322322	1.74753340679622\\
3.23123123123123	1.74951105443427\\
3.23923923923924	1.75142819884984\\
3.24724724724725	1.75328455476697\\
3.25525525525526	1.75507983690971\\
3.26326326326326	1.7568137600021\\
3.27127127127127	1.75848603876819\\
3.27927927927928	1.76009638793202\\
3.28728728728729	1.76164452221764\\
3.2952952952953	1.76313015634909\\
3.3033033033033	1.76455300505041\\
3.31131131131131	1.76591278304565\\
3.31931931931932	1.76720920505886\\
3.32732732732733	1.76844198581408\\
3.33533533533534	1.76961084003535\\
3.34334334334334	1.77071548244673\\
3.35135135135135	1.77175562777224\\
3.35935935935936	1.77273099073595\\
3.36736736736737	1.77364128606189\\
3.37537537537538	1.7744862284741\\
3.38338338338338	1.77526553269664\\
3.39139139139139	1.77597891345355\\
3.3993993993994	1.77662608546887\\
3.40740740740741	1.77720676346665\\
3.41541541541542	1.77772066217092\\
3.42342342342342	1.77816749630575\\
3.43143143143143	1.77854698059516\\
3.43943943943944	1.77885882976322\\
3.44744744744745	1.77910275853395\\
3.45545545545546	1.77927848163141\\
3.46346346346346	1.77938571377963\\
3.47147147147147	1.77942416970268\\
3.47947947947948	1.77939356412458\\
3.48748748748749	1.77929361176938\\
3.4954954954955	1.77912402736114\\
3.5035035035035	1.77888452562389\\
3.51151151151151	1.77857482128167\\
3.51951951951952	1.77819462905854\\
3.52752752752753	1.77774366367854\\
3.53553553553554	1.77722163986571\\
3.54354354354354	1.7766282723441\\
3.55155155155155	1.77596327583775\\
3.55955955955956	1.77522636507071\\
3.56756756756757	1.77441725476701\\
3.57557557557558	1.77353565965072\\
3.58358358358358	1.77258129444586\\
3.59159159159159	1.77155387387649\\
3.5995995995996	1.77045311266665\\
3.60760760760761	1.76927872554039\\
3.61561561561562	1.76803042722174\\
3.62362362362362	1.76670793243476\\
3.63163163163163	1.76531095590349\\
3.63963963963964	1.76383921235197\\
3.64764764764765	1.76229241650425\\
3.65565565565566	1.76067028308438\\
3.66366366366366	1.75897252681639\\
3.67167167167167	1.75719886242434\\
3.67967967967968	1.75534900463226\\
3.68768768768769	1.7534226681642\\
3.6956956956957	1.75141956774422\\
3.7037037037037	1.74933941809634\\
3.71171171171171	1.74718193394462\\
3.71971971971972	1.7449468300131\\
3.72772772772773	1.74263382102583\\
3.73573573573574	1.74024262170684\\
3.74374374374374	1.7377729467802\\
3.75175175175175	1.73522451096993\\
3.75975975975976	1.73259702900009\\
3.76776776776777	1.72989021559471\\
3.77577577577578	1.72710378547785\\
3.78378378378378	1.72423745337355\\
3.79179179179179	1.72129093400585\\
3.7997997997998	1.71826394209881\\
3.80780780780781	1.71515619237645\\
3.81581581581582	1.71196739956283\\
3.82382382382382	1.70869727838199\\
3.83183183183183	1.70534554355799\\
3.83983983983984	1.70191190981485\\
3.84784784784785	1.69839609187663\\
3.85585585585586	1.69479780446737\\
3.86386386386386	1.69111676231112\\
3.87187187187187	1.68735268013191\\
3.87987987987988	1.68350527265381\\
3.88788788788789	1.67957425460084\\
3.8958958958959	1.67555934069706\\
3.9039039039039	1.67146024566651\\
3.91191191191191	1.66727668423323\\
3.91991991991992	1.66300837112127\\
3.92792792792793	1.65865502105468\\
3.93593593593594	1.65421634875749\\
3.94394394394394	1.64969206895376\\
3.95195195195195	1.64508189636752\\
3.95995995995996	1.64038554572283\\
3.96796796796797	1.63560273174372\\
3.97597597597598	1.63073316915425\\
3.98398398398398	1.62577657267845\\
3.99199199199199	1.62073265704038\\
4	1.61560113696407\\
};
\addlegendentry{$\mu(x)$};

\addplot [color=black,only marks,mark=*,mark options={solid}]
  table[row sep=crcr]{%
-2.26	1.03\\
-1.31	0.7\\
-0.43	-0.68\\
0.32	-1.36\\
0.34	-1.74\\
0.54	-1.01\\
0.86	0.24\\
1.83	1.55\\
2.77	1.68\\
3.58	1.53\\
};
\addlegendentry{observations};

\end{axis}
\end{tikzpicture}%
  \caption{Posterior for $k = 3$.}
  \label{order_3_expansion}
\end{figure}

To compute the marginal likelihood, we must compute
\begin{equation*}
  p(\vec{y} \given \mat{X}, \sigma^2)
  =
  \mc{N}(\vec{y};
  \mat{\Phi}\vec{\mu},
  \mat{\Phi}\mat{\Sigma}\mat{\Phi}\trans + \sigma^2 \mat{I}).
\end{equation*}
Computing the marginal likelihood comes down to plugging in our
observations $\vec{y}$ into this Gaussian \acro{PDF}.  In practice it
is more convenient to compute the logarithm of the marginal
likelihood, due to the large dynamic range of this function.  For our
data, we can compute:
\begin{align*}
  \log p(\vec{y} \given \mat{X}, \sigma^2, k = 1)
  &= -32.9; \\
  \log p(\vec{y} \given \mat{X}, \sigma^2, k = 2)
  &= -22.3; \\
  \log p(\vec{y} \given \mat{X}, \sigma^2, k = 3)
  &= -22.2.
\end{align*}
There is a clear preference for either the quadratic or the cubic
model over the linear model, but there is no clear-cut winner between
those two.

\clearpage
\begin{enumerate}
\setcounter{enumi}{2}
\item
  (Optimal design for Bayesian linear regression.)
  Consider the data from the last problem, and suppose we have
  selected the quadratic model corresponding to $k = 2$ (do not assume
  that this is the answer to the last part of the last question).
  Imagine we are allowed to evaluate the function at a point $x'$ of
  our choosing, giving a new dataset $\data' = \data \cup \bigl\{ (x',
  y') \bigr\}$ and a new posterior for the parameters $p(\vec{w}
  \given \data', \sigma^2) = \mc{N}(\vec{w};
  \vec{\mu}_{\vec{w}\given\data'},
  \mat{\Sigma}_{\vec{w}\given\data'})$.  We hope to select the
  location $x'$ to best improve our current model, under some quality
  measure.

  Assume that we ultimately wish to predict the function at a grid of
  points
  \begin{equation*}
    \vec{x}_\ast = [-4, -3.5, -3, \dotsc, 3.5, 4]\trans.
  \end{equation*}
  We select the squared loss for a set of predictions
  $\hat{\vec{y}}_\ast$ at these points:
  \begin{equation*}
    \ell(\vec{y}_\ast, \hat{\vec{y}}_\ast)
    =
    \sum_i \bigl((y_\ast)_i - (\hat{y}_\ast)_i\bigr)^2;
  \end{equation*}
  therefore, we will predict using the new posterior mean
  $\hat{\vec{y}}_\ast = \mat{X}_\ast \vec{\mu}_{\vec{w}\given\data'}$.
  \begin{itemize}
  \item
    Given a potential observation location $x'$, derive a closed-form
    expression for the expected loss
    $\mathbb{E}\bigl[\ell(\vec{y}_\ast, \hat{\vec{y}}_\ast) \given x',
      \data \bigr]$.  Note: this does not require integration over
    $y'$!  (What is the expected squared deviation from the mean?)
  \item
    Plot the expected loss over the interval $x' \in [-4, 4]$.  Where
    is the optimal location to sample the function?
  \end{itemize}

  Note: this approach of actively selecting where to sample a function
  to maximize some utility function is known as \emph{active learning}
  in machine learning and \emph{optimal experimental design} in
  statistics.  Bayesian decision theory provides a convenient and
  consistent framework for performing active learning with a variety
  of objectives.
\end{enumerate}

\subsection*{Solution}

Imagine for the sake of argument that we have been given a new
observation $(x', y')$ to our dataset $\data$, forming the augmented
dataset $\data'$.  Imagine further that we have computed the updated
posterior $p(\vec{w} \given \data', \sigma^2)$ with this new dataset.

We are compelled to predict the value of the function at the given
test inputs $\vec{x}_\ast$.  Under the given squared loss function
$\ell(\vec{y}_\ast, \hat{\vec{y}}_\ast)$, we will predict the
posterior mean $\hat{\vec{y}}_\ast \vec{\mu}_{\vec{w}\given\data'}$.
(Recall the Bayes estimator under squared loss is the posterior mean).

Let us explicitly compute the expected loss given $\data$':
\begin{align*}
  \mathbb{E}\bigl[
    \ell(\vec{y}_\ast, \hat{\vec{y}}_\ast)
    \given
    \vec{x}_\ast,
    \data'
  \bigr]
  &=
  \mathbb{E}\Bigl[
    \sum_i
    \bigl(
    (y_\ast)_i - (\hat{y}_\ast)_i
    \bigr)^2
    \given \vec{x}_\ast, \data'
  \Bigr]
  \\
  &=
  \sum_i
  \mathbb{E}\Bigl[
    \bigl(
    (y_\ast)_i
    -
    \mathbb{E}\bigl[(y_\ast)_i \given (x_\ast)_i, \data'\bigr]
    \bigr)^2
    \given (x_\ast)_i, \data'
    \Bigr]
  \\
  &=
  \sum_i
  \var\bigl[
    (y_\ast)_i
    \given
    (x_\ast)_i, \data'
  \bigr]
  \\
  &=
  \tr
  \bigl(
  \vec{\Phi}_\ast
  \Sigma_{\vec{w}\given\data'}
  \vec{\Phi}_\ast\trans
  +
  \sigma^2 \mat{I}
  \bigr),
\end{align*}
where, in successive lines, we have: applied the linearity of
expectation, substituted the posterior mean predictions for
$\hat{\vec{y}}_\ast$, applied the definition of variance, and
rewritten the sum of the variances in terms of the trace of the
posterior covariance matrix over $\vec{y}_\ast$ given $\data'$.

The key observation here is that the posterior covariance matrix, and
therefore the expected loss given $\data'$, \emph{does not depend} on
the value of $y'$, only the location of the new input $x'$.  So we may
actually compute the future expected loss as a function of the next
observation location $x'$.  The Bayes action will then be to sample
the function at the point minimizing the trace of the updated
posterior covariance over $\vec{y}_\ast$.

In Figure \ref{problem_3}, we plot the expected loss as a function of
$x'$.  The expected loss is minimized at $x' = -4$, which is the Bayes
action.

Of course, we could have iteratively performed this procedure to
select every observation location!  The result would fall under the
general framework of so-called \emph{active learning.}

\begin{figure}
  \centering
  % This file was created by matlab2tikz.
% Minimal pgfplots version: 1.3
%
\tikzsetnextfilename{problem_3}
\definecolor{mycolor1}{rgb}{0.12157,0.47059,0.70588}%
%
\begin{tikzpicture}

\begin{axis}[%
width=0.95092\figurewidth,
height=\figureheight,
at={(0\figurewidth,0\figureheight)},
scale only axis,
xmin=-4,
xmax=4,
xlabel={$x'$},
ymin=2431.5,
ymax=2434,
ylabel={$\mathbb{E}\bigl[\ell(\vec{y}_\ast, \hat{\vec{y}}_\ast) \given x', \data\bigr]$},
axis x line*=bottom,
axis y line*=left,
legend style={legend cell align=left,align=left,draw=white!15!black},
legend style={draw=none}
]
\addplot [color=mycolor1,solid]
  table[row sep=crcr]{%
-4	2433.90245111984\\
-3.99199199199199	2433.89985359512\\
-3.98398398398398	2433.8972362976\\
-3.97597597597598	2433.89459910934\\
-3.96796796796797	2433.89194191223\\
-3.95995995995996	2433.88926458795\\
-3.95195195195195	2433.88656701802\\
-3.94394394394394	2433.88384908381\\
-3.93593593593594	2433.88111066652\\
-3.92792792792793	2433.87835164723\\
-3.91991991991992	2433.87557190691\\
-3.91191191191191	2433.8727713264\\
-3.9039039039039	2433.86994978646\\
-3.8958958958959	2433.86710716777\\
-3.88788788788789	2433.86424335093\\
-3.87987987987988	2433.8613582165\\
-3.87187187187187	2433.85845164501\\
-3.86386386386386	2433.85552351695\\
-3.85585585585586	2433.85257371281\\
-3.84784784784785	2433.84960211311\\
-3.83983983983984	2433.84660859837\\
-3.83183183183183	2433.84359304916\\
-3.82382382382382	2433.84055534611\\
-3.81581581581582	2433.83749536994\\
-3.80780780780781	2433.83441300145\\
-3.7997997997998	2433.83130812155\\
-3.79179179179179	2433.82818061129\\
-3.78378378378378	2433.82503035186\\
-3.77577577577578	2433.82185722464\\
-3.76776776776777	2433.81866111118\\
-3.75975975975976	2433.81544189322\\
-3.75175175175175	2433.81219945277\\
-3.74374374374374	2433.80893367206\\
-3.73573573573574	2433.8056444336\\
-3.72772772772773	2433.80233162019\\
-3.71971971971972	2433.79899511493\\
-3.71171171171171	2433.79563480127\\
-3.7037037037037	2433.79225056301\\
-3.6956956956957	2433.78884228435\\
-3.68768768768769	2433.78540984986\\
-3.67967967967968	2433.78195314456\\
-3.67167167167167	2433.77847205391\\
-3.66366366366366	2433.77496646387\\
-3.65565565565566	2433.77143626087\\
-3.64764764764765	2433.76788133189\\
-3.63963963963964	2433.76430156446\\
-3.63163163163163	2433.76069684667\\
-3.62362362362362	2433.75706706723\\
-3.61561561561562	2433.7534121155\\
-3.60760760760761	2433.74973188147\\
-3.5995995995996	2433.74602625584\\
-3.59159159159159	2433.74229513001\\
-3.58358358358358	2433.73853839616\\
-3.57557557557558	2433.7347559472\\
-3.56756756756757	2433.73094767687\\
-3.55955955955956	2433.72711347976\\
-3.55155155155155	2433.72325325129\\
-3.54354354354354	2433.71936688782\\
-3.53553553553554	2433.7154542866\\
-3.52752752752753	2433.71151534588\\
-3.51951951951952	2433.70754996486\\
-3.51151151151151	2433.7035580438\\
-3.5035035035035	2433.69953948401\\
-3.4954954954955	2433.6954941879\\
-3.48748748748749	2433.69142205901\\
-3.47947947947948	2433.68732300202\\
-3.47147147147147	2433.68319692283\\
-3.46346346346346	2433.67904372858\\
-3.45545545545546	2433.67486332767\\
-3.44744744744745	2433.67065562979\\
-3.43943943943944	2433.66642054601\\
-3.43143143143143	2433.66215798877\\
-3.42342342342342	2433.6578678719\\
-3.41541541541542	2433.65355011072\\
-3.40740740740741	2433.64920462204\\
-3.3993993993994	2433.6448313242\\
-3.39139139139139	2433.64043013712\\
-3.38338338338338	2433.63600098233\\
-3.37537537537538	2433.63154378303\\
-3.36736736736737	2433.6270584641\\
-3.35935935935936	2433.62254495216\\
-3.35135135135135	2433.61800317562\\
-3.34334334334334	2433.61343306471\\
-3.33533533533534	2433.60883455151\\
-3.32732732732733	2433.60420757003\\
-3.31931931931932	2433.59955205622\\
-3.31131131131131	2433.59486794802\\
-3.3033033033033	2433.59015518541\\
-3.2952952952953	2433.58541371047\\
-3.28728728728729	2433.58064346738\\
-3.27927927927928	2433.57584440251\\
-3.27127127127127	2433.57101646446\\
-3.26326326326326	2433.56615960407\\
-3.25525525525526	2433.5612737745\\
-3.24724724724725	2433.55635893127\\
-3.23923923923924	2433.5514150323\\
-3.23123123123123	2433.54644203798\\
-3.22322322322322	2433.54143991117\\
-3.21521521521522	2433.53640861728\\
-3.20720720720721	2433.53134812434\\
-3.1991991991992	2433.526258403\\
-3.19119119119119	2433.5211394266\\
-3.18318318318318	2433.51599117123\\
-3.17517517517518	2433.51081361577\\
-3.16716716716717	2433.50560674192\\
-3.15915915915916	2433.50037053428\\
-3.15115115115115	2433.49510498038\\
-3.14314314314314	2433.48981007075\\
-3.13513513513514	2433.48448579894\\
-3.12712712712713	2433.47913216159\\
-3.11911911911912	2433.47374915847\\
-3.11111111111111	2433.46833679255\\
-3.1031031031031	2433.46289507002\\
-3.0950950950951	2433.45742400036\\
-3.08708708708709	2433.4519235964\\
-3.07907907907908	2433.44639387432\\
-3.07107107107107	2433.44083485378\\
-3.06306306306306	2433.4352465579\\
-3.05505505505506	2433.42962901335\\
-3.04704704704705	2433.42398225037\\
-3.03903903903904	2433.41830630285\\
-3.03103103103103	2433.41260120836\\
-3.02302302302302	2433.40686700822\\
-3.01501501501502	2433.40110374753\\
-3.00700700700701	2433.39531147521\\
-2.998998998999	2433.38949024409\\
-2.99099099099099	2433.38364011092\\
-2.98298298298298	2433.37776113643\\
-2.97497497497497	2433.37185338539\\
-2.96696696696697	2433.36591692666\\
-2.95895895895896	2433.35995183319\\
-2.95095095095095	2433.35395818215\\
-2.94294294294294	2433.34793605491\\
-2.93493493493493	2433.34188553711\\
-2.92692692692693	2433.33580671871\\
-2.91891891891892	2433.32969969402\\
-2.91091091091091	2433.32356456178\\
-2.9029029029029	2433.31740142516\\
-2.89489489489489	2433.31121039184\\
-2.88688688688689	2433.30499157403\\
-2.87887887887888	2433.29874508853\\
-2.87087087087087	2433.29247105676\\
-2.86286286286286	2433.28616960481\\
-2.85485485485485	2433.27984086348\\
-2.84684684684685	2433.27348496832\\
-2.83883883883884	2433.26710205966\\
-2.83083083083083	2433.26069228267\\
-2.82282282282282	2433.25425578737\\
-2.81481481481481	2433.24779272868\\
-2.80680680680681	2433.24130326649\\
-2.7987987987988	2433.23478756562\\
-2.79079079079079	2433.22824579593\\
-2.78278278278278	2433.2216781323\\
-2.77477477477477	2433.2150847547\\
-2.76676676676677	2433.20846584819\\
-2.75875875875876	2433.20182160297\\
-2.75075075075075	2433.1951522144\\
-2.74274274274274	2433.18845788304\\
-2.73473473473473	2433.18173881466\\
-2.72672672672673	2433.17499522026\\
-2.71871871871872	2433.16822731613\\
-2.71071071071071	2433.16143532383\\
-2.7027027027027	2433.15461947024\\
-2.69469469469469	2433.14777998757\\
-2.68668668668669	2433.14091711336\\
-2.67867867867868	2433.13403109054\\
-2.67067067067067	2433.12712216741\\
-2.66266266266266	2433.12019059765\\
-2.65465465465465	2433.11323664038\\
-2.64664664664665	2433.10626056011\\
-2.63863863863864	2433.09926262678\\
-2.63063063063063	2433.09224311578\\
-2.62262262262262	2433.08520230791\\
-2.61461461461461	2433.07814048945\\
-2.60660660660661	2433.0710579521\\
-2.5985985985986	2433.06395499302\\
-2.59059059059059	2433.0568319148\\
-2.58258258258258	2433.0496890255\\
-2.57457457457457	2433.04252663858\\
-2.56656656656657	2433.03534507296\\
-2.55855855855856	2433.02814465297\\
-2.55055055055055	2433.02092570835\\
-2.54254254254254	2433.01368857424\\
-2.53453453453453	2433.00643359115\\
-2.52652652652653	2432.99916110498\\
-2.51851851851852	2432.99187146695\\
-2.51051051051051	2432.98456503363\\
-2.5025025025025	2432.97724216687\\
-2.49449449449449	2432.9699032338\\
-2.48648648648649	2432.96254860682\\
-2.47847847847848	2432.95517866352\\
-2.47047047047047	2432.9477937867\\
-2.46246246246246	2432.9403943643\\
-2.45445445445445	2432.93298078938\\
-2.44644644644645	2432.92555346009\\
-2.43843843843844	2432.91811277958\\
-2.43043043043043	2432.91065915605\\
-2.42242242242242	2432.90319300259\\
-2.41441441441441	2432.89571473724\\
-2.40640640640641	2432.88822478286\\
-2.3983983983984	2432.88072356711\\
-2.39039039039039	2432.87321152241\\
-2.38238238238238	2432.86568908585\\
-2.37437437437437	2432.85815669916\\
-2.36636636636637	2432.85061480862\\
-2.35835835835836	2432.84306386502\\
-2.35035035035035	2432.83550432358\\
-2.34234234234234	2432.82793664389\\
-2.33433433433433	2432.82036128984\\
-2.32632632632633	2432.81277872951\\
-2.31831831831832	2432.80518943518\\
-2.31031031031031	2432.79759388314\\
-2.3023023023023	2432.7899925537\\
-2.29429429429429	2432.78238593106\\
-2.28628628628629	2432.77477450326\\
-2.27827827827828	2432.76715876205\\
-2.27027027027027	2432.75953920282\\
-2.26226226226226	2432.75191632454\\
-2.25425425425425	2432.7442906296\\
-2.24624624624625	2432.73666262377\\
-2.23823823823824	2432.7290328161\\
-2.23023023023023	2432.72140171876\\
-2.22222222222222	2432.71376984703\\
-2.21421421421421	2432.7061377191\\
-2.20620620620621	2432.69850585603\\
-2.1981981981982	2432.69087478164\\
-2.19019019019019	2432.68324502235\\
-2.18218218218218	2432.6756171071\\
-2.17417417417417	2432.66799156726\\
-2.16616616616617	2432.66036893646\\
-2.15815815815816	2432.65274975052\\
-2.15015015015015	2432.6451345473\\
-2.14214214214214	2432.63752386659\\
-2.13413413413413	2432.62991824998\\
-2.12612612612613	2432.62231824075\\
-2.11811811811812	2432.61472438374\\
-2.11011011011011	2432.6071372252\\
-2.1021021021021	2432.5995573127\\
-2.09409409409409	2432.59198519496\\
-2.08608608608609	2432.58442142173\\
-2.07807807807808	2432.5768665437\\
-2.07007007007007	2432.56932111227\\
-2.06206206206206	2432.56178567953\\
-2.05405405405405	2432.55426079802\\
-2.04604604604605	2432.54674702066\\
-2.03803803803804	2432.53924490059\\
-2.03003003003003	2432.53175499103\\
-2.02202202202202	2432.52427784512\\
-2.01401401401401	2432.51681401582\\
-2.00600600600601	2432.50936405574\\
-1.997997997998	2432.50192851698\\
-1.98998998998999	2432.49450795104\\
-1.98198198198198	2432.48710290863\\
-1.97397397397397	2432.47971393955\\
-1.96596596596597	2432.47234159253\\
-1.95795795795796	2432.4649864151\\
-1.94994994994995	2432.45764895342\\
-1.94194194194194	2432.45032975217\\
-1.93393393393393	2432.44302935439\\
-1.92592592592593	2432.43574830131\\
-1.91791791791792	2432.42848713226\\
-1.90990990990991	2432.42124638446\\
-1.9019019019019	2432.41402659292\\
-1.89389389389389	2432.4068282903\\
-1.88588588588589	2432.39965200673\\
-1.87787787787788	2432.3924982697\\
-1.86986986986987	2432.38536760389\\
-1.86186186186186	2432.37826053105\\
-1.85385385385385	2432.37117756986\\
-1.84584584584585	2432.36411923578\\
-1.83783783783784	2432.35708604088\\
-1.82982982982983	2432.35007849378\\
-1.82182182182182	2432.34309709943\\
-1.81381381381381	2432.33614235903\\
-1.80580580580581	2432.32921476986\\
-1.7977977977978	2432.32231482516\\
-1.78978978978979	2432.31544301402\\
-1.78178178178178	2432.30859982119\\
-1.77377377377377	2432.30178572702\\
-1.76576576576577	2432.29500120727\\
-1.75775775775776	2432.28824673303\\
-1.74974974974975	2432.28152277058\\
-1.74174174174174	2432.27482978125\\
-1.73373373373373	2432.26816822132\\
-1.72572572572573	2432.26153854189\\
-1.71771771771772	2432.25494118877\\
-1.70970970970971	2432.24837660236\\
-1.7017017017017	2432.24184521755\\
-1.69369369369369	2432.23534746358\\
-1.68568568568569	2432.22888376395\\
-1.67767767767768	2432.22245453631\\
-1.66966966966967	2432.21606019238\\
-1.66166166166166	2432.2097011378\\
-1.65365365365365	2432.20337777205\\
-1.64564564564565	2432.1970904884\\
-1.63763763763764	2432.19083967373\\
-1.62962962962963	2432.18462570851\\
-1.62162162162162	2432.17844896667\\
-1.61361361361361	2432.17230981554\\
-1.60560560560561	2432.16620861573\\
-1.5975975975976	2432.16014572109\\
-1.58958958958959	2432.15412147862\\
-1.58158158158158	2432.14813622837\\
-1.57357357357357	2432.14219030339\\
-1.56556556556557	2432.13628402967\\
-1.55755755755756	2432.13041772602\\
-1.54954954954955	2432.1245917041\\
-1.54154154154154	2432.11880626824\\
-1.53353353353353	2432.1130617155\\
-1.52552552552553	2432.10735833553\\
-1.51751751751752	2432.10169641054\\
-1.50950950950951	2432.09607621528\\
-1.5015015015015	2432.09049801696\\
-1.49349349349349	2432.08496207521\\
-1.48548548548549	2432.07946864206\\
-1.47747747747748	2432.0740179619\\
-1.46946946946947	2432.0686102714\\
-1.46146146146146	2432.06324579956\\
-1.45345345345345	2432.0579247676\\
-1.44544544544545	2432.052647389\\
-1.43743743743744	2432.04741386943\\
-1.42942942942943	2432.04222440677\\
-1.42142142142142	2432.03707919106\\
-1.41341341341341	2432.03197840451\\
-1.40540540540541	2432.02692222149\\
-1.3973973973974	2432.02191080851\\
-1.38938938938939	2432.01694432421\\
-1.38138138138138	2432.01202291938\\
-1.37337337337337	2432.00714673697\\
-1.36536536536537	2432.00231591206\\
-1.35735735735736	2431.99753057188\\
-1.34934934934935	2431.99279083583\\
-1.34134134134134	2431.98809681552\\
-1.33333333333333	2431.9834486147\\
-1.32532532532533	2431.97884632938\\
-1.31731731731732	2431.9742900478\\
-1.30930930930931	2431.96977985043\\
-1.3013013013013	2431.96531581006\\
-1.29329329329329	2431.96089799178\\
-1.28528528528529	2431.95652645304\\
-1.27727727727728	2431.95220124368\\
-1.26926926926927	2431.94792240592\\
-1.26126126126126	2431.9436899745\\
-1.25325325325325	2431.93950397662\\
-1.24524524524525	2431.93536443204\\
-1.23723723723724	2431.93127135312\\
-1.22922922922923	2431.92722474485\\
-1.22122122122122	2431.92322460493\\
-1.21321321321321	2431.9192709238\\
-1.20520520520521	2431.91536368469\\
-1.1971971971972	2431.91150286372\\
-1.18918918918919	2431.90768842989\\
-1.18118118118118	2431.90392034523\\
-1.17317317317317	2431.90019856477\\
-1.16516516516517	2431.89652303668\\
-1.15715715715716	2431.8928937023\\
-1.14914914914915	2431.88931049621\\
-1.14114114114114	2431.88577334631\\
-1.13313313313313	2431.88228217389\\
-1.12512512512513	2431.8788368937\\
-1.11711711711712	2431.87543741405\\
-1.10910910910911	2431.87208363682\\
-1.1011011011011	2431.86877545763\\
-1.09309309309309	2431.86551276585\\
-1.08508508508509	2431.8622954447\\
-1.07707707707708	2431.85912337134\\
-1.06906906906907	2431.85599641695\\
-1.06106106106106	2431.85291444683\\
-1.05305305305305	2431.84987732044\\
-1.04504504504505	2431.84688489152\\
-1.03703703703704	2431.8439370082\\
-1.02902902902903	2431.84103351303\\
-1.02102102102102	2431.83817424311\\
-1.01301301301301	2431.83535903018\\
-1.00500500500501	2431.83258770069\\
-0.996996996996997	2431.8298600759\\
-0.988988988988989	2431.82717597198\\
-0.980980980980981	2431.82453520011\\
-0.972972972972973	2431.82193756653\\
-0.964964964964965	2431.81938287269\\
-0.956956956956957	2431.81687091531\\
-0.948948948948949	2431.81440148647\\
-0.940940940940941	2431.81197437374\\
-0.932932932932933	2431.80958936024\\
-0.924924924924925	2431.80724622474\\
-0.916916916916917	2431.80494474177\\
-0.908908908908909	2431.80268468171\\
-0.900900900900901	2431.80046581087\\
-0.892892892892893	2431.79828789163\\
-0.884884884884885	2431.79615068248\\
-0.876876876876877	2431.79405393814\\
-0.868868868868869	2431.79199740967\\
-0.860860860860861	2431.78998084453\\
-0.852852852852853	2431.78800398673\\
-0.844844844844845	2431.78606657685\\
-0.836836836836837	2431.78416835222\\
-0.828828828828829	2431.78230904693\\
-0.820820820820821	2431.78048839199\\
-0.812812812812813	2431.7787061154\\
-0.804804804804805	2431.77696194221\\
-0.796796796796797	2431.77525559469\\
-0.788788788788789	2431.77358679234\\
-0.780780780780781	2431.77195525204\\
-0.772772772772773	2431.77036068809\\
-0.764764764764765	2431.76880281238\\
-0.756756756756757	2431.76728133438\\
-0.748748748748749	2431.76579596131\\
-0.740740740740741	2431.76434639818\\
-0.732732732732733	2431.76293234791\\
-0.724724724724725	2431.76155351141\\
-0.716716716716717	2431.76020958763\\
-0.708708708708709	2431.75890027371\\
-0.700700700700701	2431.75762526502\\
-0.692692692692693	2431.75638425524\\
-0.684684684684685	2431.75517693647\\
-0.676676676676677	2431.75400299932\\
-0.668668668668669	2431.75286213294\\
-0.660660660660661	2431.75175402516\\
-0.652652652652653	2431.75067836253\\
-0.644644644644645	2431.74963483042\\
-0.636636636636636	2431.74862311308\\
-0.628628628628629	2431.74764289373\\
-0.62062062062062	2431.74669385466\\
-0.612612612612613	2431.74577567724\\
-0.604604604604605	2431.74488804206\\
-0.596596596596596	2431.74403062896\\
-0.588588588588589	2431.74320311713\\
-0.58058058058058	2431.74240518516\\
-0.572572572572573	2431.74163651113\\
-0.564564564564565	2431.74089677266\\
-0.556556556556556	2431.74018564699\\
-0.548548548548549	2431.73950281104\\
-0.54054054054054	2431.73884794148\\
-0.532532532532533	2431.73822071479\\
-0.524524524524525	2431.73762080734\\
-0.516516516516516	2431.73704789543\\
-0.508508508508509	2431.73650165536\\
-0.5005005005005	2431.7359817635\\
-0.492492492492492	2431.73548789632\\
-0.484484484484485	2431.73501973051\\
-0.476476476476476	2431.73457694295\\
-0.468468468468469	2431.73415921085\\
-0.46046046046046	2431.73376621174\\
-0.452452452452452	2431.73339762358\\
-0.444444444444445	2431.73305312476\\
-0.436436436436436	2431.73273239418\\
-0.428428428428429	2431.73243511132\\
-0.42042042042042	2431.73216095623\\
-0.412412412412412	2431.73190960965\\
-0.404404404404405	2431.73168075301\\
-0.396396396396396	2431.73147406848\\
-0.388388388388389	2431.73128923903\\
-0.38038038038038	2431.73112594847\\
-0.372372372372372	2431.73098388151\\
-0.364364364364364	2431.73086272376\\
-0.356356356356356	2431.7307621618\\
-0.348348348348348	2431.73068188323\\
-0.34034034034034	2431.73062157668\\
-0.332332332332332	2431.73058093188\\
-0.324324324324324	2431.73055963968\\
-0.316316316316316	2431.73055739206\\
-0.308308308308308	2431.73057388223\\
-0.3003003003003	2431.73060880461\\
-0.292292292292292	2431.73066185487\\
-0.284284284284284	2431.73073273001\\
-0.276276276276276	2431.7308211283\\
-0.268268268268268	2431.73092674942\\
-0.26026026026026	2431.73104929439\\
-0.252252252252252	2431.73118846568\\
-0.244244244244244	2431.73134396719\\
-0.236236236236236	2431.73151550427\\
-0.228228228228228	2431.73170278379\\
-0.22022022022022	2431.73190551412\\
-0.212212212212212	2431.7321234052\\
-0.204204204204204	2431.73235616852\\
-0.196196196196196	2431.73260351716\\
-0.188188188188188	2431.73286516582\\
-0.18018018018018	2431.73314083084\\
-0.172172172172172	2431.73343023019\\
-0.164164164164164	2431.73373308355\\
-0.156156156156156	2431.73404911227\\
-0.148148148148148	2431.73437803942\\
-0.14014014014014	2431.7347195898\\
-0.132132132132132	2431.73507348993\\
-0.124124124124124	2431.73543946814\\
-0.116116116116116	2431.7358172545\\
-0.108108108108108	2431.73620658088\\
-0.1001001001001	2431.73660718096\\
-0.0920920920920922	2431.73701879024\\
-0.084084084084084	2431.73744114604\\
-0.0760760760760761	2431.73787398754\\
-0.0680680680680679	2431.73831705576\\
-0.06006006006006	2431.73877009359\\
-0.0520520520520522	2431.7392328458\\
-0.0440440440440439	2431.73970505904\\
-0.0360360360360361	2431.74018648187\\
-0.0280280280280278	2431.74067686471\\
-0.02002002002002	2431.74117595995\\
-0.0120120120120122	2431.74168352184\\
-0.00400400400400391	2431.74219930661\\
0.00400400400400436	2431.74272307237\\
0.0120120120120122	2431.7432545792\\
0.02002002002002	2431.74379358911\\
0.0280280280280278	2431.74433986605\\
0.0360360360360357	2431.74489317595\\
0.0440440440440444	2431.74545328667\\
0.0520520520520522	2431.74601996803\\
0.06006006006006	2431.74659299181\\
0.0680680680680679	2431.74717213178\\
0.0760760760760757	2431.74775716365\\
0.0840840840840844	2431.7483478651\\
0.0920920920920922	2431.7489440158\\
0.1001001001001	2431.74954539737\\
0.108108108108108	2431.75015179342\\
0.116116116116116	2431.75076298952\\
0.124124124124124	2431.75137877322\\
0.132132132132132	2431.75199893404\\
0.14014014014014	2431.75262326347\\
0.148148148148148	2431.75325155498\\
0.156156156156156	2431.75388360398\\
0.164164164164164	2431.7545192079\\
0.172172172172172	2431.75515816609\\
0.18018018018018	2431.75580027988\\
0.188188188188188	2431.75644535258\\
0.196196196196196	2431.75709318943\\
0.204204204204204	2431.75774359765\\
0.212212212212212	2431.7583963864\\
0.22022022022022	2431.7590513668\\
0.228228228228228	2431.75970835192\\
0.236236236236236	2431.76036715675\\
0.244244244244245	2431.76102759826\\
0.252252252252252	2431.76168949532\\
0.26026026026026	2431.76235266875\\
0.268268268268268	2431.7630169413\\
0.276276276276277	2431.76368213762\\
0.284284284284285	2431.76434808431\\
0.292292292292292	2431.76501460987\\
0.3003003003003	2431.7656815447\\
0.308308308308308	2431.76634872111\\
0.316316316316317	2431.76701597331\\
0.324324324324325	2431.76768313742\\
0.332332332332332	2431.76835005141\\
0.34034034034034	2431.76901655516\\
0.348348348348348	2431.76968249042\\
0.356356356356357	2431.77034770081\\
0.364364364364365	2431.77101203182\\
0.372372372372372	2431.7716753308\\
0.38038038038038	2431.77233744694\\
0.388388388388388	2431.77299823129\\
0.396396396396397	2431.77365753674\\
0.404404404404405	2431.77431521802\\
0.412412412412412	2431.77497113167\\
0.42042042042042	2431.77562513609\\
0.428428428428428	2431.77627709145\\
0.436436436436437	2431.77692685978\\
0.444444444444445	2431.77757430488\\
0.452452452452452	2431.77821929236\\
0.46046046046046	2431.77886168963\\
0.468468468468468	2431.77950136588\\
0.476476476476477	2431.78013819208\\
0.484484484484485	2431.78077204097\\
0.492492492492492	2431.78140278707\\
0.5005005005005	2431.78203030665\\
0.508508508508508	2431.78265447775\\
0.516516516516517	2431.78327518014\\
0.524524524524525	2431.78389229535\\
0.532532532532533	2431.78450570665\\
0.54054054054054	2431.78511529903\\
0.548548548548548	2431.78572095921\\
0.556556556556557	2431.78632257563\\
0.564564564564565	2431.78692003845\\
0.572572572572573	2431.78751323954\\
0.58058058058058	2431.78810207245\\
0.588588588588588	2431.78868643246\\
0.596596596596597	2431.78926621653\\
0.604604604604605	2431.7898413233\\
0.612612612612613	2431.79041165308\\
0.62062062062062	2431.79097710789\\
0.628628628628628	2431.7915375914\\
0.636636636636637	2431.79209300893\\
0.644644644644645	2431.7926432675\\
0.652652652652653	2431.79318827574\\
0.66066066066066	2431.79372794397\\
0.668668668668668	2431.79426218414\\
0.676676676676677	2431.79479090983\\
0.684684684684685	2431.79531403628\\
0.692692692692693	2431.79583148033\\
0.7007007007007	2431.79634316049\\
0.708708708708708	2431.79684899687\\
0.716716716716717	2431.79734891119\\
0.724724724724725	2431.7978428268\\
0.732732732732733	2431.79833066866\\
0.74074074074074	2431.79881236334\\
0.748748748748748	2431.79928783902\\
0.756756756756757	2431.79975702545\\
0.764764764764765	2431.80021985403\\
0.772772772772773	2431.8006762577\\
0.780780780780781	2431.80112617103\\
0.788788788788789	2431.80156953017\\
0.796796796796797	2431.80200627283\\
0.804804804804805	2431.80243633834\\
0.812812812812813	2431.80285966759\\
0.820820820820821	2431.80327620305\\
0.828828828828829	2431.80368588876\\
0.836836836836837	2431.80408867035\\
0.844844844844845	2431.80448449501\\
0.852852852852853	2431.80487331149\\
0.860860860860861	2431.80525507012\\
0.868868868868869	2431.8056297228\\
0.876876876876877	2431.80599722298\\
0.884884884884885	2431.80635752568\\
0.892892892892893	2431.80671058748\\
0.900900900900901	2431.80705636654\\
0.908908908908909	2431.80739482254\\
0.916916916916917	2431.80772591675\\
0.924924924924925	2431.808049612\\
0.932932932932933	2431.80836587265\\
0.940940940940941	2431.80867466465\\
0.948948948948949	2431.80897595548\\
0.956956956956957	2431.80926971419\\
0.964964964964965	2431.8095559114\\
0.972972972972973	2431.80983451926\\
0.980980980980981	2431.81010551149\\
0.988988988988989	2431.81036886338\\
0.996996996996997	2431.81062455177\\
1.00500500500501	2431.81087255505\\
1.01301301301301	2431.81111285318\\
1.02102102102102	2431.81134542768\\
1.02902902902903	2431.81157026164\\
1.03703703703704	2431.8117873397\\
1.04504504504505	2431.81199664807\\
1.05305305305305	2431.81219817454\\
1.06106106106106	2431.81239190845\\
1.06906906906907	2431.81257784072\\
1.07707707707708	2431.81275596384\\
1.08508508508509	2431.81292627188\\
1.09309309309309	2431.81308876048\\
1.1011011011011	2431.81324342687\\
1.10910910910911	2431.81339026984\\
1.11711711711712	2431.81352928978\\
1.12512512512513	2431.81366048867\\
1.13313313313313	2431.81378387007\\
1.14114114114114	2431.81389943914\\
1.14914914914915	2431.81400720263\\
1.15715715715716	2431.81410716889\\
1.16516516516517	2431.81419934787\\
1.17317317317317	2431.81428375114\\
1.18118118118118	2431.81436039185\\
1.18918918918919	2431.81442928479\\
1.1971971971972	2431.81449044635\\
1.20520520520521	2431.81454389455\\
1.21321321321321	2431.81458964902\\
1.22122122122122	2431.81462773104\\
1.22922922922923	2431.81465816349\\
1.23723723723724	2431.81468097091\\
1.24524524524525	2431.81469617948\\
1.25325325325325	2431.814703817\\
1.26126126126126	2431.81470391294\\
1.26926926926927	2431.81469649842\\
1.27727727727728	2431.81468160622\\
1.28528528528529	2431.81465927075\\
1.29329329329329	2431.81462952814\\
1.3013013013013	2431.81459241614\\
1.30930930930931	2431.81454797421\\
1.31731731731732	2431.81449624346\\
1.32532532532533	2431.81443726671\\
1.33333333333333	2431.81437108846\\
1.34134134134134	2431.8142977549\\
1.34934934934935	2431.81421731392\\
1.35735735735736	2431.81412981511\\
1.36536536536537	2431.81403530979\\
1.37337337337337	2431.81393385095\\
1.38138138138138	2431.81382549335\\
1.38938938938939	2431.81371029343\\
1.3973973973974	2431.81358830938\\
1.40540540540541	2431.81345960112\\
1.41341341341341	2431.81332423029\\
1.42142142142142	2431.81318226029\\
1.42942942942943	2431.81303375626\\
1.43743743743744	2431.8128787851\\
1.44544544544545	2431.81271741544\\
1.45345345345345	2431.8125497177\\
1.46146146146146	2431.81237576405\\
1.46946946946947	2431.81219562843\\
1.47747747747748	2431.81200938654\\
1.48548548548549	2431.8118171159\\
1.49349349349349	2431.81161889575\\
1.5015015015015	2431.81141480718\\
1.50950950950951	2431.81120493301\\
1.51751751751752	2431.81098935791\\
1.52552552552553	2431.8107681683\\
1.53353353353353	2431.81054145243\\
1.54154154154154	2431.81030930034\\
1.54954954954955	2431.81007180389\\
1.55755755755756	2431.80982905674\\
1.56556556556557	2431.80958115438\\
1.57357357357357	2431.8093281941\\
1.58158158158158	2431.80907027501\\
1.58958958958959	2431.80880749806\\
1.5975975975976	2431.80853996601\\
1.60560560560561	2431.80826778346\\
1.61361361361361	2431.80799105682\\
1.62162162162162	2431.80770989434\\
1.62962962962963	2431.80742440612\\
1.63763763763764	2431.80713470406\\
1.64564564564565	2431.80684090192\\
1.65365365365365	2431.8065431153\\
1.66166166166166	2431.80624146161\\
1.66966966966967	2431.80593606012\\
1.67767767767768	2431.80562703192\\
1.68568568568569	2431.80531449995\\
1.69369369369369	2431.80499858899\\
1.7017017017017	2431.80467942564\\
1.70970970970971	2431.80435713833\\
1.71771771771772	2431.80403185734\\
1.72572572572573	2431.80370371478\\
1.73373373373373	2431.80337284458\\
1.74174174174174	2431.80303938248\\
1.74974974974975	2431.80270346608\\
1.75775775775776	2431.80236523478\\
1.76576576576577	2431.80202482979\\
1.77377377377377	2431.80168239414\\
1.78178178178178	2431.80133807267\\
1.78978978978979	2431.80099201202\\
1.7977977977978	2431.80064436063\\
1.80580580580581	2431.80029526873\\
1.81381381381381	2431.79994488833\\
1.82182182182182	2431.79959337324\\
1.82982982982983	2431.79924087902\\
1.83783783783784	2431.79888756301\\
1.84584584584585	2431.7985335843\\
1.85385385385385	2431.79817910372\\
1.86186186186186	2431.79782428387\\
1.86986986986987	2431.79746928905\\
1.87787787787788	2431.79711428528\\
1.88588588588589	2431.79675944033\\
1.89389389389389	2431.79640492361\\
1.9019019019019	2431.79605090626\\
1.90990990990991	2431.79569756109\\
1.91791791791792	2431.79534506256\\
1.92592592592593	2431.79499358677\\
1.93393393393393	2431.79464331149\\
1.94194194194194	2431.79429441609\\
1.94994994994995	2431.79394708153\\
1.95795795795796	2431.79360149039\\
1.96596596596597	2431.79325782681\\
1.97397397397397	2431.79291627649\\
1.98198198198198	2431.79257702666\\
1.98998998998999	2431.79224026608\\
1.997997997998	2431.79190618502\\
2.00600600600601	2431.79157497523\\
2.01401401401401	2431.7912468299\\
2.02202202202202	2431.79092194369\\
2.03003003003003	2431.79060051268\\
2.03803803803804	2431.79028273433\\
2.04604604604605	2431.78996880748\\
2.05405405405405	2431.78965893234\\
2.06206206206206	2431.78935331043\\
2.07007007007007	2431.78905214459\\
2.07807807807808	2431.78875563891\\
2.08608608608609	2431.78846399875\\
2.09409409409409	2431.7881774307\\
2.1021021021021	2431.78789614253\\
2.11011011011011	2431.78762034319\\
2.11811811811812	2431.78735024277\\
2.12612612612613	2431.78708605244\\
2.13413413413413	2431.78682798448\\
2.14214214214214	2431.7865762522\\
2.15015015015015	2431.78633106993\\
2.15815815815816	2431.78609265298\\
2.16616616616617	2431.78586121761\\
2.17417417417417	2431.78563698097\\
2.18218218218218	2431.78542016112\\
2.19019019019019	2431.78521097694\\
2.1981981981982	2431.78500964812\\
2.20620620620621	2431.78481639512\\
2.21421421421421	2431.78463143912\\
2.22222222222222	2431.78445500199\\
2.23023023023023	2431.78428730626\\
2.23823823823824	2431.78412857507\\
2.24624624624625	2431.78397903212\\
2.25425425425425	2431.78383890164\\
2.26226226226226	2431.78370840833\\
2.27027027027027	2431.78358777736\\
2.27827827827828	2431.78347723427\\
2.28628628628629	2431.78337700496\\
2.29429429429429	2431.78328731565\\
2.3023023023023	2431.7832083928\\
2.31031031031031	2431.78314046309\\
2.31831831831832	2431.78308375337\\
2.32632632632633	2431.7830384906\\
2.33433433433433	2431.78300490181\\
2.34234234234234	2431.78298321407\\
2.35035035035035	2431.78297365439\\
2.35835835835836	2431.78297644971\\
2.36636636636637	2431.78299182684\\
2.37437437437437	2431.7830200124\\
2.38238238238238	2431.78306123279\\
2.39039039039039	2431.78311571409\\
2.3983983983984	2431.78318368206\\
2.40640640640641	2431.78326536204\\
2.41441441441441	2431.78336097896\\
2.42242242242242	2431.78347075719\\
2.43043043043043	2431.78359492056\\
2.43843843843844	2431.7837336923\\
2.44644644644645	2431.78388729493\\
2.45445445445445	2431.78405595026\\
2.46246246246246	2431.78423987929\\
2.47047047047047	2431.78443930219\\
2.47847847847848	2431.78465443821\\
2.48648648648649	2431.78488550565\\
2.49449449449449	2431.78513272178\\
2.5025025025025	2431.78539630277\\
2.51051051051051	2431.78567646367\\
2.51851851851852	2431.78597341833\\
2.52652652652653	2431.78628737931\\
2.53453453453453	2431.78661855787\\
2.54254254254254	2431.78696716389\\
2.55055055055055	2431.7873334058\\
2.55855855855856	2431.78771749052\\
2.56656656656657	2431.78811962342\\
2.57457457457457	2431.78854000824\\
2.58258258258258	2431.78897884702\\
2.59059059059059	2431.78943634008\\
2.5985985985986	2431.78991268591\\
2.60660660660661	2431.79040808114\\
2.61461461461461	2431.79092272048\\
2.62262262262262	2431.79145679664\\
2.63063063063063	2431.79201050029\\
2.63863863863864	2431.79258401997\\
2.64664664664665	2431.79317754206\\
2.65465465465465	2431.79379125074\\
2.66266266266266	2431.79442532785\\
2.67067067067067	2431.79507995291\\
2.67867867867868	2431.79575530303\\
2.68668668668669	2431.79645155286\\
2.69469469469469	2431.7971688745\\
2.7027027027027	2431.7979074375\\
2.71071071071071	2431.79866740874\\
2.71871871871872	2431.79944895242\\
2.72672672672673	2431.80025222999\\
2.73473473473473	2431.80107740007\\
2.74274274274274	2431.80192461845\\
2.75075075075075	2431.80279403797\\
2.75875875875876	2431.80368580851\\
2.76676676676677	2431.80460007692\\
2.77477477477477	2431.80553698696\\
2.78278278278278	2431.8064966793\\
2.79079079079079	2431.80747929137\\
2.7987987987988	2431.80848495742\\
2.80680680680681	2431.80951380838\\
2.81481481481481	2431.81056597186\\
2.82282282282282	2431.81164157212\\
2.83083083083083	2431.81274072994\\
2.83883883883884	2431.81386356269\\
2.84684684684685	2431.81501018417\\
2.85485485485485	2431.81618070466\\
2.86286286286286	2431.81737523082\\
2.87087087087087	2431.81859386567\\
2.87887887887888	2431.81983670855\\
2.88688688688689	2431.82110385505\\
2.89489489489489	2431.82239539704\\
2.9029029029029	2431.82371142257\\
2.91091091091091	2431.82505201584\\
2.91891891891892	2431.8264172572\\
2.92692692692693	2431.8278072231\\
2.93493493493493	2431.82922198604\\
2.94294294294294	2431.83066161456\\
2.95095095095095	2431.83212617321\\
2.95895895895896	2431.83361572251\\
2.96696696696697	2431.83513031893\\
2.97497497497497	2431.83667001487\\
2.98298298298298	2431.83823485862\\
2.99099099099099	2431.83982489436\\
2.998998998999	2431.84144016212\\
3.00700700700701	2431.84308069777\\
3.01501501501502	2431.844746533\\
3.02302302302302	2431.8464376953\\
3.03103103103103	2431.84815420795\\
3.03903903903904	2431.84989609001\\
3.04704704704705	2431.8516633563\\
3.05505505505506	2431.85345601737\\
3.06306306306306	2431.85527407954\\
3.07107107107107	2431.85711754486\\
3.07907907907908	2431.85898641111\\
3.08708708708709	2431.86088067179\\
3.0950950950951	2431.86280031612\\
3.1031031031031	2431.86474532904\\
3.11111111111111	2431.86671569124\\
3.11911911911912	2431.8687113791\\
3.12712712712713	2431.87073236475\\
3.13513513513514	2431.87277861604\\
3.14314314314314	2431.87485009657\\
3.15115115115115	2431.87694676568\\
3.15915915915916	2431.87906857848\\
3.16716716716717	2431.88121548586\\
3.17517517517518	2431.88338743445\\
3.18318318318318	2431.88558436674\\
3.19119119119119	2431.88780622098\\
3.1991991991992	2431.8900529313\\
3.20720720720721	2431.89232442765\\
3.21521521521522	2431.89462063586\\
3.22322322322322	2431.89694147766\\
3.23123123123123	2431.89928687072\\
3.23923923923924	2431.90165672861\\
3.24724724724725	2431.90405096092\\
3.25525525525526	2431.90646947322\\
3.26326326326326	2431.9089121671\\
3.27127127127127	2431.91137894024\\
3.27927927927928	2431.91386968639\\
3.28728728728729	2431.91638429544\\
3.2952952952953	2431.91892265345\\
3.3033033033033	2431.92148464268\\
3.31131131131131	2431.92407014162\\
3.31931931931932	2431.92667902504\\
3.32732732732733	2431.92931116404\\
3.33533533533534	2431.93196642609\\
3.34334334334334	2431.93464467504\\
3.35135135135135	2431.9373457712\\
3.35935935935936	2431.94006957139\\
3.36736736736737	2431.94281592896\\
3.37537537537538	2431.94558469384\\
3.38338338338338	2431.94837571261\\
3.39139139139139	2431.95118882854\\
3.3993993993994	2431.95402388164\\
3.40740740740741	2431.9568807087\\
3.41541541541542	2431.95975914337\\
3.42342342342342	2431.96265901619\\
3.43143143143143	2431.96558015466\\
3.43943943943944	2431.96852238326\\
3.44744744744745	2431.97148552359\\
3.45545545545546	2431.97446939432\\
3.46346346346346	2431.97747381132\\
3.47147147147147	2431.98049858772\\
3.47947947947948	2431.98354353391\\
3.48748748748749	2431.98660845766\\
3.4954954954955	2431.98969316417\\
3.5035035035035	2431.9927974561\\
3.51151151151151	2431.99592113365\\
3.51951951951952	2431.99906399466\\
3.52752752752753	2432.00222583459\\
3.53553553553554	2432.00540644667\\
3.54354354354354	2432.00860562191\\
3.55155155155155	2432.01182314917\\
3.55955955955956	2432.01505881525\\
3.56756756756757	2432.01831240493\\
3.57557557557558	2432.02158370104\\
3.58358358358358	2432.02487248454\\
3.59159159159159	2432.02817853456\\
3.5995995995996	2432.03150162849\\
3.60760760760761	2432.03484154202\\
3.61561561561562	2432.03819804926\\
3.62362362362362	2432.04157092272\\
3.63163163163163	2432.04495993345\\
3.63963963963964	2432.04836485107\\
3.64764764764765	2432.05178544385\\
3.65565565565566	2432.05522147878\\
3.66366366366366	2432.0586727216\\
3.67167167167167	2432.06213893691\\
3.67967967967968	2432.06561988823\\
3.68768768768769	2432.06911533802\\
3.6956956956957	2432.07262504782\\
3.7037037037037	2432.07614877823\\
3.71171171171171	2432.07968628905\\
3.71971971971972	2432.0832373393\\
3.72772772772773	2432.08680168729\\
3.73573573573574	2432.0903790907\\
3.74374374374374	2432.09396930662\\
3.75175175175175	2432.09757209164\\
3.75975975975976	2432.10118720189\\
3.76776776776777	2432.1048143931\\
3.77577577577578	2432.10845342068\\
3.78378378378378	2432.11210403978\\
3.79179179179179	2432.11576600533\\
3.7997997997998	2432.11943907209\\
3.80780780780781	2432.12312299478\\
3.81581581581582	2432.12681752805\\
3.82382382382382	2432.13052242658\\
3.83183183183183	2432.13423744514\\
3.83983983983984	2432.13796233865\\
3.84784784784785	2432.14169686222\\
3.85585585585586	2432.14544077119\\
3.86386386386386	2432.14919382123\\
3.87187187187187	2432.15295576837\\
3.87987987987988	2432.15672636904\\
3.88788788788789	2432.16050538013\\
3.8958958958959	2432.16429255906\\
3.9039039039039	2432.16808766381\\
3.91191191191191	2432.17189045298\\
3.91991991991992	2432.17570068582\\
3.92792792792793	2432.17951812232\\
3.93593593593594	2432.1833425232\\
3.94394394394394	2432.18717365001\\
3.95195195195195	2432.19101126514\\
3.95995995995996	2432.19485513188\\
3.96796796796797	2432.19870501446\\
3.97597597597598	2432.20256067811\\
3.98398398398398	2432.20642188905\\
3.99199199199199	2432.21028841461\\
4	2432.2141600232\\
};
\addlegendentry{expected loss};

\addplot [color=black,only marks,mark=*,mark options={solid}]
  table[row sep=crcr]{%
-0.316316316316316	2431.73055739206\\
};
\addlegendentry{Bayes action};

\end{axis}
\end{tikzpicture}%
  \caption{Expected loss for predicting $\vec{y}_\ast$ given $\data'$
    as a function of $x'$.}
  \label{problem_3}
\end{figure}

\clearpage
\begin{enumerate}
\setcounter{enumi}{3}
\item
  (Woodbury matrix identity.)
  The \emph{Woodbury matrix identity} is a very useful result.  Let
  $\mat{A}$ be an $(n \times n)$ matrix, let $\mat{U}$ and $\mat{V}$
  be $(n \times k)$ matrices, and let $\mat{C}$ be a $(k \times k)$
  matrix.  Then:
  \begin{equation*}
    (\mat{A} + \mat{U}\mat{C}\mat{V}\trans)\inv
    =
    \mat{A}\inv
    -
    \mat{A}\inv
    \mat{U}
    (\mat{C}\inv + \mat{V}\trans \mat{A}\inv \mat{U})\inv
    \mat{V}\trans
    \mat{A}\inv.
  \end{equation*}
  This result is useful when you already have the inverse of a matrix
  $\mat{A}$ and want to know the inverse after a rank-$k$ adjustment.
  When $k \ll n$, the Woodbry matrix identity can be considerably
  faster than direct inversion!
  \begin{itemize}
  \item
    Prove this result.
  \item
    Use this result to rewrite the posterior covariance of the weight
    vector $\vec{w}$ in Bayesian linear regression (as written in the
    notes to lecture 5) in a simpler form.
  \end{itemize}
\end{enumerate}

\subsection*{Solution}

The first part of the problem can be completed by multiplying the
right hand side by $(\mat{A} + \mat{U}\mat{C}\mat{V}\trans)$ and
checking you get the identity.

The posterior covariance for $\vec{w}$ was previously given as
\begin{equation*}
  \mat{\Sigma}_{\vec{w}\given\data}
  =
  \mat{\Sigma}
  -
  \mat{\Sigma}
  \mat{X}\trans
  (\mat{X}\mat{\Sigma}\mat{X}\trans + \sigma^2 \mat{I})\inv
  \mat{X}
  \mat{\Sigma}.
\end{equation*}
Taking
\begin{equation*}
  \mat{A} = \mat{\Sigma}\inv
  \qquad
  \mat{U} = \mat{V} = \mat{X}\trans
  \qquad
  \mat{C} = \sigma^2 \mat{I},
\end{equation*}
we may rewrite this as
\begin{equation*}
  \mat{\Sigma}_{\vec{w}\given\data}
  =
  (\mat{\Sigma}\inv
  +
  \sigma^{-2}
  \mat{X}\trans\mat{X})\inv.
\end{equation*}

\clearpage
\begin{enumerate}
\setcounter{enumi}{4}
\item
  (Laplace approximation.)
  Find a Laplace approximation to the Gamma distribution:
  \begin{equation*}
    p(\theta \given \alpha, \beta)
    =
    \frac{1}{Z}
    \theta^{\alpha - 1}
    \exp(-\beta\theta).
  \end{equation*}
  Plot the approximation against the true density for $(\alpha, \beta)
  = (2, \nicefrac{1}{2})$.

  The true value of the normalizing constant is
  \begin{equation*}
    Z = \frac{\Gamma(\alpha)}{\beta^\alpha}.
  \end{equation*}
  If we fix $\beta = 1$, then $Z = \Gamma(\alpha)$, so we may use the
  Laplace approximation to estimate the Gamma function.  Analyze the
  quality of this approximation as a function of $\alpha$.
\end{enumerate}

\subsection*{Solution}

We first define the unnormalized log density $\Psi(\theta)$:
\begin{equation*}
  \Psi(\theta)
  =
  \log p(\theta \given \alpha, \beta)
  =
  (\alpha - 1) \log \theta - \beta\theta.
\end{equation*}
Next we find the maximal value of the distribution, $\hat{\theta}$, by
computing the derivative and setting to zero:
\begin{equation*}
  0
  =
  \frac{d}{d\theta}
  \Psi(\theta)
  =
  \frac{\alpha - 1}{\theta}
  - \beta
  \quad
  \Rightarrow
  \quad
  \hat{\theta} = \frac{\alpha - 1}{\beta}.
\end{equation*}
Next we compute the negative Hessian of $\Psi$ at $\hat{\theta}$.
Note that here we have a one-dimensional density, so the Hessian is
simply equal to the second derivative:
\begin{equation*}
  H
  =
  -\frac{d^2}{d\theta^2}
  \Psi(\theta)
  \biggr\rvert_{\theta = \hat{\theta}}
  =
  \frac{\alpha - 1}{\theta^2}
  \biggr\rvert_{\theta = \hat{\theta}}
  =
  \frac{\beta^2}{\alpha - 1}.
\end{equation*}
Now the Laplace approximation to the gamma distribution is
\begin{equation*}
  p(\theta \given \alpha, \beta)
  \approx
  \mc{N}(\theta; \hat{\theta}, H\inv)
  =
  \mc{N}\biggl(\theta;
  \frac{\alpha - 1}{\beta},
  \frac{\alpha - 1}{\beta^2}
  \biggr).
\end{equation*}
The corresponding estimate for the normalizing constant $Z$
is
\begin{equation*}
  Z
  \approx
  \exp\bigl(\Psi(\hat{\vec{\theta}})\bigr)
  \sqrt{
    \frac{(2\pi)^d}
         {\det \mat{H}}
  }
  =
  p(\hat{\theta} \given \alpha, \beta)
  \sqrt{\frac{2\pi}{H}}
  =
  \sqrt{\frac{2\pi(\alpha - 1)}{\beta^2}}
  \biggl(\frac{\alpha - 1}{\beta}\biggr)^{\alpha - 1}
  \exp\bigl(-(\alpha - 1)\bigr).
\end{equation*}
Plugging in $\beta = 1$ and using the true normalizing constant of the
gamma distribution, we have the approximation
\begin{equation*}
  \Gamma(\alpha)
  \approx
  \sqrt{2\pi}
  (\alpha - 1)^{\alpha - \nicefrac{1}{2}}
  \exp\bigl(-(\alpha - 1)\bigr).
\end{equation*}
Note we also have an approximation to the logarithm
of $\Gamma$:
\begin{equation*}
  \log \Gamma(\alpha)
  \approx
  \textstyle
  \frac{1}{2}\log 2 \pi
  +
  (\alpha - \nicefrac{1}{2})
  \log(\alpha - 1)
  -
  (\alpha - 1).
\end{equation*}

Figure \ref{problem_5} shows the resulting approximation to $Z =
\Gamma(\alpha)$ as a function of $\alpha$.  The approximation
appears to be quite good for $\alpha \geq 2$.

To those who are mathematically inclined, note that $n! = \Gamma(n +
1)$.  With a bit of manipulation, we have actually rediscovered a very
famous result known as \emph{Stirling's approximation:}
\begin{equation*}
  n!
  \approx
  \sqrt{2\pi n}
  \biggl(\frac{n}{e}\biggr)^n.
\end{equation*}

\begin{figure}
  \centering
  % This file was created by matlab2tikz.
% Minimal pgfplots version: 1.3
%
\tikzsetnextfilename{problem_5}
\definecolor{mycolor1}{rgb}{0.12157,0.47059,0.70588}%
\definecolor{mycolor2}{rgb}{0.20000,0.62745,0.17255}%
%
\begin{tikzpicture}

\begin{axis}[%
width=0.95092\figurewidth,
height=\figureheight,
at={(0\figurewidth,0\figureheight)},
scale only axis,
xmin=0,
xmax=5,
xlabel={$\alpha$},
ymin=0,
ymax=25,
axis x line*=bottom,
axis y line*=left,
legend style={at={(0.03,0.97)},anchor=north west,legend cell align=left,align=left,draw=white!15!black},
legend style={draw=none}
]
\addplot [color=mycolor1,solid]
  table[row sep=crcr]{%
0.1	9.51350769866873\\
0.104904904904905	9.05002294019284\\
0.10980980980981	8.62827575484633\\
0.114714714714715	8.24290879982183\\
0.11961961961962	7.88944349407788\\
0.124524524524525	7.56410694890028\\
0.129429429429429	7.26369825071511\\
0.134334334334334	6.98548403816251\\
0.139239239239239	6.72711614989941\\
0.144144144144144	6.4865660860053\\
0.149049049049049	6.26207240987672\\
0.153953953953954	6.05209820466201\\
0.158858858858859	5.85529641113704\\
0.163763763763764	5.67048139461639\\
0.168668668668669	5.4966054729118\\
0.173573573573574	5.33273942399962\\
0.178478478478478	5.17805620781062\\
0.183383383383383	5.03181730037089\\
0.188288288288288	4.89336116393141\\
0.193193193193193	4.76209347347771\\
0.198098098098098	4.63747879520341\\
0.203003003003003	4.51903347137355\\
0.207907907907908	4.40631951235159\\
0.212812812812813	4.29893933329813\\
0.217717717717718	4.19653120233526\\
0.222622622622623	4.09876529044926\\
0.227527527527528	4.00534023232792\\
0.232432432432432	3.91598012265807\\
0.237337337337337	3.83043188488748\\
0.242242242242242	3.74846295965928\\
0.247147147147147	3.6698592685089\\
0.252052052052052	3.59442341532733\\
0.256956956956957	3.52197309382025\\
0.261861861861862	3.45233967395346\\
0.266766766766767	3.38536694434789\\
0.271671671671672	3.32090999091478\\
0.276576576576577	3.25883419481791\\
0.281481481481481	3.19901433520766\\
0.286386386386386	3.14133378416551\\
0.291291291291291	3.08568378299038\\
0.296196196196196	3.03196279039746\\
0.301101101101101	2.9800758944293\\
0.306006006006006	2.92993428093051\\
0.310910910910911	2.88145475233936\\
0.315815815815816	2.83455929132603\\
0.320720720720721	2.78917466447627\\
0.325625625625626	2.74523206179803\\
0.330530530530531	2.70266676832963\\
0.335435435435435	2.6614178645637\\
0.34034034034034	2.62142795277963\\
0.345245245245245	2.58264290670792\\
0.35015015015015	2.54501164223856\\
0.355055055055055	2.50848590713819\\
0.35995995995996	2.47302008796296\\
0.364864864864865	2.43857103254892\\
0.36976976976977	2.40509788663314\\
0.374674674674675	2.372561943311\\
0.37957957957958	2.34092650416808\\
0.384484484484485	2.31015675104435\\
0.389389389389389	2.28021962749294\\
0.394294294294294	2.25108372908923\\
0.399199199199199	2.22271920182907\\
0.404104104104104	2.1950976479288\\
0.409009009009009	2.16819203840549\\
0.413913913913914	2.14197663187519\\
0.418818818818819	2.11642689905915\\
0.423723723723724	2.09151945253557\\
0.428628628628629	2.06723198131662\\
0.433533533533534	2.04354318986854\\
0.438438438438438	2.02043274122684\\
0.443343343343343	1.99788120388934\\
0.448248248248248	1.97587000219778\\
0.453153153153153	1.95438136994352\\
0.458058058058058	1.93339830695565\\
0.462962962962963	1.91290453845034\\
0.467867867867868	1.8928844769387\\
0.472772772772773	1.87332318650729\\
0.477677677677678	1.85420634930088\\
0.482582582582583	1.83552023405041\\
0.487487487487487	1.81725166650248\\
0.492392392392392	1.79938800161735\\
0.497297297297297	1.78191709741369\\
0.502202202202202	1.76482729034711\\
0.507107107107107	1.7481073721189\\
0.512012012012012	1.73174656781887\\
0.516916916916917	1.71573451531369\\
0.521821821821822	1.70006124579878\\
0.526726726726727	1.68471716543791\\
0.531631631631632	1.6696930380201\\
0.536536536536537	1.65497996856886\\
0.541441441441441	1.64056938784334\\
0.546346346346346	1.62645303767513\\
0.551251251251251	1.61262295708896\\
0.556156156156156	1.59907146915855\\
0.561061061061061	1.58579116855288\\
0.565965965965966	1.57277490973082\\
0.570870870870871	1.56001579574514\\
0.575775775775776	1.54750716761958\\
0.580680680680681	1.53524259426503\\
0.585585585585586	1.52321586290328\\
0.590490490490491	1.51142096996866\\
0.595395395395395	1.49985211246017\\
0.6003003003003	1.48850367971825\\
0.605205205205205	1.47737024560195\\
0.61011011011011	1.46644656104428\\
0.615015015015015	1.45572754696424\\
0.61991991991992	1.44520828751603\\
0.624824824824825	1.4348840236568\\
0.62972972972973	1.42475014701555\\
0.634634634634635	1.4148021940469\\
0.63953953953954	1.40503584045442\\
0.644444444444444	1.3954468958692\\
0.649349349349349	1.38603129877\\
0.654254254254254	1.37678511163252\\
0.659159159159159	1.36770451629563\\
0.664064064064064	1.35878580953346\\
0.668968968968969	1.35002539882275\\
0.673873873873874	1.34141979829548\\
0.678778778778779	1.33296562486745\\
0.683683683683684	1.32465959453386\\
0.688588588588589	1.31649851882369\\
0.693493493493494	1.30847930140493\\
0.698398398398398	1.30059893483322\\
0.703303303303303	1.29285449743701\\
0.708208208208208	1.28524315033238\\
0.713113113113113	1.27776213456163\\
0.718018018018018	1.27040876834926\\
0.722922922922923	1.26318044447023\\
0.727827827827828	1.25607462772484\\
0.732732732732733	1.24908885251536\\
0.737637637637638	1.24222072051974\\
0.742542542542543	1.23546789845778\\
0.747447447447448	1.22882811594555\\
0.752352352352352	1.22229916343409\\
0.757257257257257	1.21587889022847\\
0.762162162162162	1.20956520258361\\
0.767067067067067	1.20335606187352\\
0.771971971971972	1.19724948283051\\
0.776876876876877	1.1912435318514\\
0.781781781781782	1.18533632536781\\
0.786686686686687	1.17952602827758\\
0.791591591591592	1.1738108524348\\
0.796496496496497	1.16818905519592\\
0.801401401401401	1.16265893801939\\
0.806306306306306	1.15721884511674\\
0.811211211211211	1.15186716215276\\
0.816116116116116	1.14660231499284\\
0.821021021021021	1.14142276849533\\
0.825925925925926	1.13632702534729\\
0.830830830830831	1.13131362494155\\
0.835735735735736	1.1263811422936\\
0.840640640640641	1.12152818699657\\
0.845545545545546	1.11675340221282\\
0.85045045045045	1.11205546370053\\
0.855355355355355	1.1074330788741\\
0.86026026026026	1.10288498589676\\
0.865165165165165	1.09840995280429\\
0.87007007007007	1.09400677665855\\
0.874974974974975	1.08967428272968\\
0.87987987987988	1.08541132370575\\
0.884784784784785	1.08121677892895\\
0.88968968968969	1.07708955365719\\
0.894594594594595	1.0730285783501\\
0.899499499499499	1.06903280797864\\
0.904404404404404	1.06510122135726\\
0.909309309309309	1.06123282049788\\
0.914214214214214	1.05742662998475\\
0.919119119119119	1.05368169636956\\
0.924024024024024	1.04999708758593\\
0.928928928928929	1.04637189238253\\
0.933833833833834	1.04280521977432\\
0.938738738738739	1.03929619851104\\
0.943643643643644	1.03584397656247\\
0.948548548548549	1.03244772061974\\
0.953453453453453	1.02910661561224\\
0.958358358358358	1.02581986423943\\
0.963263263263263	1.02258668651711\\
0.968168168168168	1.01940631933762\\
0.973073073073073	1.0162780160434\\
0.977977977977978	1.01320104601364\\
0.982882882882883	1.01017469426326\\
0.987787787787788	1.00719826105409\\
0.992692692692693	1.00427106151765\\
0.997597597597598	1.00139242528912\\
1.0025025025025	0.998561696152263\\
1.00740740740741	0.995778231694779\\
1.01231231231231	0.993041402973784\\
1.01721721721722	0.990350594191104\\
1.02212212212212	0.987705202378007\\
1.02702702702703	0.985104637089088\\
1.03193193193193	0.982548320104975\\
1.03683683683684	0.980035685143592\\
1.04174174174174	0.97756617757967\\
1.04664664664665	0.975139254172255\\
1.05155155155155	0.972754382799932\\
1.05645645645646	0.970411042203536\\
1.06136136136136	0.968108721736085\\
1.06626626626627	0.965846921119713\\
1.07117117117117	0.963625150209382\\
1.07607607607608	0.961442928763147\\
1.08098098098098	0.959299786218767\\
1.08588588588589	0.957195261476463\\
1.09079079079079	0.955128902687636\\
1.0956956956957	0.953100267049337\\
1.1006006006006	0.951108920604337\\
1.10550550550551	0.949154438046595\\
1.11041041041041	0.947236402531978\\
1.11531531531532	0.945354405494059\\
1.12022022022022	0.943508046464842\\
1.12512512512513	0.941696932900259\\
1.13003003003003	0.939920680010296\\
1.13493493493494	0.938178910593616\\
1.13983983983984	0.936471254876519\\
1.14474474474474	0.934797350356144\\
1.14964964964965	0.933156841647747\\
1.15455455455455	0.931549380335969\\
1.15945945945946	0.929974624829951\\
1.16436436436436	0.928432240222191\\
1.16926926926927	0.926921898151036\\
1.17417417417417	0.925443276666697\\
1.17907907907908	0.923996060100689\\
1.18398398398398	0.922579938938588\\
1.18888888888889	0.921194609696022\\
1.19379379379379	0.919839774797787\\
1.1986986986987	0.918515142460019\\
1.2036036036036	0.917220426575311\\
1.20850850850851	0.915955346600706\\
1.21341341341341	0.914719627448485\\
1.21831831831832	0.913512999379668\\
1.22322322322322	0.912335197900139\\
1.22812812812813	0.911185963659351\\
1.23303303303303	0.910065042351509\\
1.23793793793794	0.908972184619181\\
1.24284284284284	0.907907145959262\\
1.24774774774775	0.906869686631224\\
1.25265265265265	0.905859571567603\\
1.25755755755756	0.904876570286644\\
1.26246246246246	0.903920456807053\\
1.26736736736737	0.902991009564806\\
1.27227227227227	0.902088011331948\\
1.27717717717718	0.901211249137334\\
1.28208208208208	0.900360514189258\\
1.28698698698699	0.899535601799927\\
1.29189189189189	0.898736311311717\\
1.2967967967968	0.897962446025178\\
1.3017017017017	0.897213813128733\\
1.30660660660661	0.89649022363003\\
1.31151151151151	0.895791492288904\\
1.31641641641642	0.895117437551904\\
1.32132132132132	0.894467881488351\\
1.32622622622623	0.893842649727878\\
1.33113113113113	0.893241571399423\\
1.33603603603604	0.892664479071632\\
1.34094094094094	0.89211120869464\\
1.34584584584585	0.891581599543189\\
1.35075075075075	0.891075494161057\\
1.35565565565566	0.89059273830676\\
1.36056056056056	0.890133180900491\\
1.36546546546547	0.889696673972282\\
1.37037037037037	0.889283072611333\\
1.37527527527528	0.888892234916502\\
1.38018018018018	0.888524021947911\\
1.38508508508509	0.888178297679658\\
1.38998998998999	0.887854928953584\\
1.3948948948949	0.887553785434094\\
1.3997997997998	0.887274739563988\\
1.4047047047047	0.887017666521286\\
1.40960960960961	0.886782444177027\\
1.41451451451451	0.886568953054005\\
1.41941941941942	0.886377076286431\\
1.42432432432432	0.886206699580498\\
1.42922922922923	0.886057711175824\\
1.43413413413413	0.885930001807751\\
1.43903903903904	0.885823464670484\\
1.44394394394394	0.885737995381051\\
1.44884884884885	0.88567349194406\\
1.45375375375375	0.885629854717234\\
1.45865865865866	0.885606986377716\\
1.46356356356356	0.885604791889113\\
1.46846846846847	0.885623178469274\\
1.47337337337337	0.885662055558769\\
1.47827827827828	0.885721334790079\\
1.48318318318318	0.88580092995745\\
1.48808808808809	0.88590075698742\\
1.49299299299299	0.886020733909995\\
1.4978978978979	0.886160780830455\\
1.5028028028028	0.886320819901784\\
1.50770770770771	0.88650077529771\\
1.51261261261261	0.886700573186327\\
1.51751751751752	0.88692014170431\\
1.52242242242242	0.887159410931693\\
1.52732732732733	0.887418312867196\\
1.53223223223223	0.887696781404107\\
1.53713713713714	0.887994752306685\\
1.54204204204204	0.888312163187087\\
1.54694694694695	0.888648953482806\\
1.55185185185185	0.889005064434601\\
1.55675675675676	0.889380439064923\\
1.56166166166166	0.889775022156807\\
1.56656656656657	0.890188760233244\\
1.57147147147147	0.890621601536995\\
1.57637637637638	0.891073496010869\\
1.58128128128128	0.891544395278428\\
1.58618618618619	0.892034252625122\\
1.59109109109109	0.892543022979854\\
1.595995995996	0.893070662896944\\
1.6009009009009	0.893617130538508\\
1.60580580580581	0.894182385657223\\
1.61071071071071	0.894766389579487\\
1.61561561561562	0.895369105188955\\
1.62052052052052	0.895990496910446\\
1.62542542542543	0.896630530694219\\
1.63033033033033	0.897289174000601\\
1.63523523523524	0.89796639578497\\
1.64014014014014	0.898662166483083\\
1.64504504504505	0.899376457996731\\
1.64994994994995	0.900109243679739\\
1.65485485485486	0.900860498324279\\
1.65975975975976	0.901630198147506\\
1.66466466466466	0.902418320778504\\
1.66956956956957	0.90322484524554\\
1.67447447447447	0.904049751963615\\
1.67937937937938	0.904893022722316\\
1.68428428428428	0.905754640673947\\
1.68918918918919	0.906634590321956\\
1.69409409409409	0.90753285750963\\
1.698998998999	0.908449429409073\\
1.7039039039039	0.909384294510442\\
1.70880880880881	0.91033744261146\\
1.71371371371371	0.911308864807173\\
1.71861861861862	0.912298553479976\\
1.72352352352352	0.913306502289881\\
1.72842842842843	0.91433270616503\\
1.73333333333333	0.915377161292453\\
1.73823823823824	0.916439865109063\\
1.74314314314314	0.917520816292881\\
1.74804804804805	0.918620014754493\\
1.75295295295295	0.919737461628732\\
1.75785785785786	0.920873159266581\\
1.76276276276276	0.922027111227293\\
1.76766766766767	0.923199322270726\\
1.77257257257257	0.92438979834989\\
1.77747747747748	0.9255985466037\\
1.78238238238238	0.926825575349931\\
1.78728728728729	0.928070894078378\\
1.79219219219219	0.929334513444211\\
1.7970970970971	0.930616445261524\\
1.802002002002	0.931916702497074\\
1.80690690690691	0.933235299264214\\
1.81181181181181	0.934572250817007\\
1.81671671671672	0.935927573544519\\
1.82162162162162	0.937301284965303\\
1.82652652652653	0.938693403722047\\
1.83143143143143	0.940103949576411\\
1.83633633633634	0.941532943404019\\
1.84124124124124	0.942980407189637\\
1.84614614614615	0.944446364022506\\
1.85105105105105	0.945930838091849\\
1.85595595595596	0.947433854682532\\
1.86086086086086	0.948955440170891\\
1.86576576576577	0.950495622020715\\
1.87067067067067	0.952054428779382\\
1.87557557557558	0.953631890074158\\
1.88048048048048	0.955228036608632\\
1.88538538538539	0.956842900159318\\
1.89029029029029	0.958476513572392\\
1.8951951951952	0.960128910760578\\
1.9001001001001	0.961800126700184\\
1.90500500500501	0.96349019742827\\
1.90990990990991	0.965199160039964\\
1.91481481481482	0.966927052685913\\
1.91971971971972	0.968673914569873\\
1.92462462462462	0.970439785946434\\
1.92952952952953	0.972224708118876\\
1.93443443443443	0.97402872343716\\
1.93933933933934	0.97585187529605\\
1.94424424424424	0.97769420813336\\
1.94914914914915	0.979555767428337\\
1.95405405405405	0.98143659970016\\
1.95895895895896	0.983336752506572\\
1.96386386386386	0.98525627444263\\
1.96876876876877	0.987195215139585\\
1.97367367367367	0.989153625263872\\
1.97857857857858	0.991131556516232\\
1.98348348348348	0.993129061630943\\
1.98838838838839	0.995146194375174\\
1.99329329329329	0.997183009548451\\
1.9981981981982	0.999239562982246\\
2.0031031031031	1.00131591153967\\
2.00800800800801	1.00341211311529\\
2.01291291291291	1.00552822663505\\
2.01781781781782	1.00766431205629\\
2.02272272272272	1.00982043036794\\
2.02762762762763	1.0119966435907\\
2.03253253253253	1.01419301477749\\
2.03743743743744	1.01640960801382\\
2.04234234234234	1.01864648841847\\
2.04724724724725	1.02090372214411\\
2.05215215215215	1.02318137637808\\
2.05705705705706	1.02547951934335\\
2.06196196196196	1.02779822029943\\
2.06686686686687	1.03013754954354\\
2.07177177177177	1.03249757841177\\
2.07667667667668	1.03487837928039\\
2.08158158158158	1.03728002556729\\
2.08648648648649	1.03970259173341\\
2.09139139139139	1.04214615328442\\
2.0962962962963	1.04461078677237\\
2.1012012012012	1.04709656979753\\
2.10610610610611	1.04960358101027\\
2.11101101101101	1.05213190011303\\
2.11591591591592	1.05468160786247\\
2.12082082082082	1.05725278607161\\
2.12572572572573	1.05984551761216\\
2.13063063063063	1.06245988641685\\
2.13553553553554	1.06509597748195\\
2.14044044044044	1.06775387686983\\
2.14534534534535	1.07043367171161\\
2.15025025025025	1.07313545020994\\
2.15515515515516	1.07585930164183\\
2.16006006006006	1.07860531636162\\
2.16496496496497	1.08137358580401\\
2.16986986986987	1.08416420248718\\
2.17477477477478	1.08697726001604\\
2.17967967967968	1.08981285308552\\
2.18458458458458	1.09267107748399\\
2.18948948948949	1.09555203009675\\
2.19439439439439	1.09845580890965\\
2.1992992992993	1.10138251301272\\
2.2042042042042	1.10433224260401\\
2.20910910910911	1.10730509899339\\
2.21401401401401	1.11030118460654\\
2.21891891891892	1.11332060298899\\
2.22382382382382	1.11636345881024\\
2.22872872872873	1.11942985786801\\
2.23363363363363	1.12251990709252\\
2.23853853853854	1.12563371455094\\
2.24344344344344	1.12877138945186\\
2.24834834834835	1.13193304214985\\
2.25325325325325	1.1351187841502\\
2.25815815815816	1.13832872811362\\
2.26306306306306	1.14156298786114\\
2.26796796796797	1.144821678379\\
2.27287287287287	1.14810491582376\\
2.27777777777778	1.15141281752737\\
2.28268268268268	1.15474550200238\\
2.28758758758759	1.15810308894729\\
2.29249249249249	1.16148569925188\\
2.2973973973974	1.16489345500275\\
2.3023023023023	1.16832647948886\\
2.30720720720721	1.17178489720721\\
2.31211211211211	1.17526883386858\\
2.31701701701702	1.17877841640338\\
2.32192192192192	1.1823137729676\\
2.32682682682683	1.18587503294881\\
2.33173173173173	1.18946232697228\\
2.33663663663664	1.19307578690721\\
2.34154154154154	1.196715545873\\
2.34644644644645	1.20038173824565\\
2.35135135135135	1.20407449966426\\
2.35625625625626	1.20779396703757\\
2.36116116116116	1.21154027855064\\
2.36606606606607	1.2153135736716\\
2.37097097097097	1.21911399315853\\
2.37587587587588	1.22294167906634\\
2.38078078078078	1.22679677475389\\
2.38568568568569	1.23067942489101\\
2.39059059059059	1.23458977546584\\
2.3954954954955	1.23852797379205\\
2.4004004004004	1.24249416851628\\
2.40530530530531	1.24648850962568\\
2.41021021021021	1.25051114845546\\
2.41511511511512	1.25456223769659\\
2.42002002002002	1.25864193140361\\
2.42492492492493	1.26275038500252\\
2.42982982982983	1.26688775529871\\
2.43473473473474	1.2710542004851\\
2.43963963963964	1.27524988015027\\
2.44454454454455	1.27947495528677\\
2.44944944944945	1.28372958829944\\
2.45435435435435	1.28801394301391\\
2.45925925925926	1.29232818468519\\
2.46416416416416	1.29667248000627\\
2.46906906906907	1.30104699711693\\
2.47397397397397	1.3054519056126\\
2.47887887887888	1.30988737655332\\
2.48378378378378	1.3143535824728\\
2.48868868868869	1.31885069738761\\
2.49359359359359	1.32337889680639\\
2.4984984984985	1.32793835773932\\
2.5034034034034	1.33252925870748\\
2.50830830830831	1.33715177975254\\
2.51321321321321	1.34180610244635\\
2.51811811811812	1.34649240990077\\
2.52302302302302	1.35121088677757\\
2.52792792792793	1.35596171929842\\
2.53283283283283	1.36074509525495\\
2.53773773773774	1.36556120401903\\
2.54264264264264	1.37041023655305\\
2.54754754754755	1.37529238542035\\
2.55245245245245	1.38020784479576\\
2.55735735735736	1.38515681047623\\
2.56226226226226	1.39013947989161\\
2.56716716716717	1.3951560521155\\
2.57207207207207	1.40020672787623\\
2.57697697697698	1.40529170956793\\
2.58188188188188	1.41041120126178\\
2.58678678678679	1.41556540871725\\
2.59169169169169	1.4207545393936\\
2.5965965965966	1.4259788024614\\
2.6015015015015	1.43123840881417\\
2.60640640640641	1.43653357108017\\
2.61131131131131	1.4418645036343\\
2.61621621621622	1.44723142261011\\
2.62112112112112	1.4526345459119\\
2.62602602602603	1.45807409322701\\
2.63093093093093	1.46355028603817\\
2.63583583583584	1.46906334763599\\
2.64074074074074	1.47461350313155\\
2.64564564564565	1.48020097946919\\
2.65055055055055	1.48582600543931\\
2.65545545545546	1.49148881169138\\
2.66036036036036	1.49718963074705\\
2.66526526526527	1.50292869701337\\
2.67017017017017	1.50870624679615\\
2.67507507507508	1.51452251831347\\
2.67997997997998	1.52037775170926\\
2.68488488488489	1.52627218906709\\
2.68978978978979	1.53220607442401\\
2.6946946946947	1.53817965378457\\
2.6995995995996	1.54419317513498\\
2.7045045045045	1.55024688845735\\
2.70940940940941	1.55634104574413\\
2.71431431431431	1.56247590101263\\
2.71921921921922	1.56865171031972\\
2.72412412412412	1.57486873177664\\
2.72902902902903	1.58112722556394\\
2.73393393393393	1.5874274539466\\
2.73883883883884	1.59376968128926\\
2.74374374374374	1.6001541740716\\
2.74864864864865	1.60658120090384\\
2.75355355355355	1.61305103254242\\
2.75845845845846	1.61956394190583\\
2.76336336336336	1.62612020409049\\
2.76826826826827	1.63272009638697\\
2.77317317317317	1.6393638982961\\
2.77807807807808	1.64605189154551\\
2.78298298298298	1.65278436010606\\
2.78788788788789	1.6595615902086\\
2.79279279279279	1.66638387036083\\
2.7976976976977	1.67325149136427\\
2.8026026026026	1.68016474633145\\
2.80750750750751	1.68712393070324\\
2.81241241241241	1.69412934226627\\
2.81731731731732	1.70118128117062\\
2.82222222222222	1.70828004994761\\
2.82712712712713	1.71542595352772\\
2.83203203203203	1.72261929925876\\
2.83693693693694	1.72986039692408\\
2.84184184184184	1.73714955876112\\
2.84674674674675	1.74448709947991\\
2.85165165165165	1.75187333628196\\
2.85655655655656	1.75930858887912\\
2.86146146146146	1.76679317951277\\
2.86636636636637	1.77432743297308\\
2.87127127127127	1.78191167661847\\
2.87617617617618	1.7895462403953\\
2.88108108108108	1.79723145685762\\
2.88598598598599	1.80496766118722\\
2.89089089089089	1.81275519121379\\
2.8957957957958	1.82059438743526\\
2.9007007007007	1.82848559303835\\
2.90560560560561	1.83642915391929\\
2.91051051051051	1.84442541870475\\
2.91541541541542	1.85247473877289\\
2.92032032032032	1.86057746827464\\
2.92522522522523	1.86873396415524\\
2.93013013013013	1.87694458617582\\
2.93503503503504	1.88520969693528\\
2.93993993993994	1.89352966189237\\
2.94484484484484	1.9019048493879\\
2.94974974974975	1.9103356306672\\
2.95465465465466	1.91882237990274\\
2.95955955955956	1.92736547421701\\
2.96446446446447	1.93596529370553\\
2.96936936936937	1.94462222146013\\
2.97427427427427	1.95333664359237\\
2.97917917917918	1.96210894925724\\
2.98408408408408	1.97093953067701\\
2.98898898898899	1.97982878316535\\
2.99389389389389	1.98877710515156\\
2.9987987987988	1.99778489820517\\
3.0037037037037	2.00685256706059\\
3.00860860860861	2.01598051964212\\
3.01351351351351	2.02516916708908\\
3.01841841841842	2.03441892378122\\
3.02332332332332	2.04373020736434\\
3.02822822822823	2.05310343877612\\
3.03313313313313	2.06253904227219\\
3.03803803803804	2.07203744545243\\
3.04294294294294	2.08159907928753\\
3.04784784784785	2.09122437814571\\
3.05275275275275	2.10091377981977\\
3.05765765765766	2.11066772555429\\
3.06256256256256	2.12048666007315\\
3.06746746746747	2.1303710316072\\
3.07237237237237	2.1403212919223\\
3.07727727727728	2.15033789634746\\
3.08218218218218	2.16042130380337\\
3.08708708708709	2.17057197683105\\
3.09199199199199	2.18079038162087\\
3.0968968968969	2.19107698804174\\
3.1018018018018	2.2014322696706\\
3.10670670670671	2.21185670382215\\
3.11161161161161	2.22235077157886\\
3.11651651651652	2.23291495782124\\
3.12142142142142	2.24354975125835\\
3.12632632632633	2.25425564445861\\
3.13123123123123	2.26503313388088\\
3.13613613613614	2.27588271990579\\
3.14104104104104	2.28680490686735\\
3.14594594594595	2.2978002030849\\
3.15085085085085	2.30886912089522\\
3.15575575575576	2.32001217668503\\
3.16066066066066	2.33122989092374\\
3.16556556556557	2.34252278819648\\
3.17047047047047	2.35389139723741\\
3.17537537537538	2.36533625096336\\
3.18028028028028	2.37685788650774\\
3.18518518518519	2.38845684525475\\
3.19009009009009	2.4001336728739\\
3.194994994995	2.41188891935483\\
3.1998998998999	2.42372313904243\\
3.20480480480481	2.43563689067227\\
3.20970970970971	2.44763073740637\\
3.21461461461461	2.45970524686922\\
3.21951951951952	2.4718609911842\\
3.22442442442442	2.48409854701025\\
3.22932932932933	2.49641849557891\\
3.23423423423423	2.50882142273165\\
3.23913913913914	2.52130791895757\\
3.24404404404404	2.53387857943139\\
3.24894894894895	2.54653400405178\\
3.25385385385385	2.55927479748007\\
3.25875875875876	2.57210156917927\\
3.26366366366366	2.5850149334534\\
3.26856856856857	2.59801550948725\\
3.27347347347347	2.61110392138639\\
3.27837837837838	2.62428079821764\\
3.28328328328328	2.6375467740498\\
3.28818818818819	2.65090248799478\\
3.29309309309309	2.66434858424912\\
3.297997997998	2.67788571213584\\
3.3029029029029	2.69151452614666\\
3.30780780780781	2.70523568598458\\
3.31271271271271	2.7190498566069\\
3.31761761761762	2.73295770826856\\
3.32252252252252	2.74695991656584\\
3.32742742742743	2.76105716248055\\
3.33233233233233	2.77525013242449\\
3.33723723723724	2.78953951828436\\
3.34214214214214	2.8039260174671\\
3.34704704704705	2.81841033294555\\
3.35195195195195	2.83299317330459\\
3.35685685685686	2.84767525278763\\
3.36176176176176	2.86245729134353\\
3.36666666666667	2.87734001467396\\
3.37157157157157	2.89232415428115\\
3.37647647647648	2.90741044751607\\
3.38138138138138	2.92259963762703\\
3.38628628628629	2.93789247380869\\
3.39119119119119	2.95328971125161\\
3.3960960960961	2.96879211119206\\
3.401001001001	2.98440044096243\\
3.40590590590591	3.000115474042\\
3.41081081081081	3.01593799010819\\
3.41571571571572	3.03186877508825\\
3.42062062062062	3.04790862121144\\
3.42552552552553	3.0640583270616\\
3.43043043043043	3.0803186976303\\
3.43533533533534	3.09669054437031\\
3.44024024024024	3.11317468524971\\
3.44514514514515	3.12977194480634\\
3.45005005005005	3.14648315420282\\
3.45495495495496	3.16330915128202\\
3.45985985985986	3.18025078062302\\
3.46476476476477	3.19730889359759\\
3.46966966966967	3.21448434842719\\
3.47457457457458	3.23177801024036\\
3.47947947947948	3.24919075113078\\
3.48438438438439	3.26672345021575\\
3.48928928928929	3.28437699369519\\
3.49419419419419	3.30215227491121\\
3.4990990990991	3.32005019440818\\
3.504004004004	3.3380716599933\\
3.50890890890891	3.3562175867978\\
3.51381381381381	3.37448889733859\\
3.51871871871872	3.39288652158053\\
3.52362362362362	3.4114113969992\\
3.52852852852853	3.43006446864426\\
3.53343343343343	3.44884668920335\\
3.53833833833834	3.46775901906663\\
3.54324324324324	3.48680242639175\\
3.54814814814815	3.5059778871696\\
3.55305305305305	3.52528638529041\\
3.55795795795796	3.54472891261069\\
3.56286286286286	3.56430646902054\\
3.56776776776777	3.58402006251173\\
3.57267267267267	3.60387070924628\\
3.57757757757758	3.6238594336257\\
3.58248248248248	3.64398726836086\\
3.58738738738739	3.66425525454245\\
3.59229229229229	3.68466444171205\\
3.5971971971972	3.70521588793395\\
3.6021021021021	3.72591065986743\\
3.60700700700701	3.74674983283987\\
3.61191191191191	3.76773449092038\\
3.61681681681682	3.78886572699415\\
3.62172172172172	3.81014464283748\\
3.62662662662663	3.83157234919342\\
3.63153153153153	3.85314996584812\\
3.63643643643644	3.8748786217079\\
3.64134134134134	3.89675945487696\\
3.64624624624625	3.91879361273579\\
3.65115115115115	3.94098225202031\\
3.65605605605606	3.96332653890169\\
3.66096096096096	3.98582764906693\\
3.66586586586587	4.0084867678001\\
3.67077077077077	4.03130509006437\\
3.67567567567568	4.05428382058472\\
3.68058058058058	4.07742417393143\\
3.68548548548549	4.10072737460435\\
3.69039039039039	4.12419465711781\\
3.6952952952953	4.14782726608646\\
3.7002002002002	4.1716264563117\\
3.70510510510511	4.19559349286905\\
3.71001001001001	4.21972965119623\\
3.71491491491492	4.24403621718197\\
3.71981981981982	4.26851448725577\\
3.72472472472473	4.29316576847835\\
3.72962962962963	4.31799137863291\\
3.73453453453453	4.34299264631732\\
3.73943943943944	4.36817091103703\\
3.74434434434434	4.39352752329886\\
3.74924924924925	4.41906384470563\\
3.75415415415415	4.44478124805169\\
3.75905905905906	4.47068111741918\\
3.76396396396396	4.49676484827536\\
3.76886886886887	4.52303384757059\\
3.77377377377377	4.54948953383741\\
3.77867867867868	4.57613333729033\\
3.78358358358358	4.60296669992668\\
3.78848848848849	4.62999107562824\\
3.79339339339339	4.65720793026388\\
3.7982982982983	4.68461874179307\\
3.8032032032032	4.71222500037035\\
3.80810810810811	4.74002820845078\\
3.81301301301301	4.76802988089627\\
3.81791791791792	4.79623154508291\\
3.82282282282282	4.82463474100931\\
3.82772772772773	4.85324102140583\\
3.83263263263263	4.88205195184491\\
3.83753753753754	4.91106911085231\\
3.84244244244244	4.94029409001939\\
3.84734734734735	4.96972849411642\\
3.85225225225225	4.99937394120692\\
3.85715715715716	5.02923206276297\\
3.86206206206206	5.05930450378172\\
3.86696696696697	5.08959292290275\\
3.87187187187187	5.12009899252669\\
3.87677677677678	5.15082439893478\\
3.88168168168168	5.18177084240959\\
3.88658658658659	5.21294003735683\\
3.89149149149149	5.24433371242822\\
3.8963963963964	5.27595361064553\\
3.9013013013013	5.30780148952575\\
3.90620620620621	5.33987912120733\\
3.91111111111111	5.37218829257766\\
3.91601601601602	5.40473080540167\\
3.92092092092092	5.43750847645156\\
3.92582582582583	5.47052313763779\\
3.93073073073073	5.50377663614122\\
3.93563563563564	5.53727083454645\\
3.94054054054054	5.57100761097639\\
3.94544544544545	5.60498885922808\\
3.95035035035035	5.63921648890968\\
3.95525525525526	5.67369242557877\\
3.96016016016016	5.70841861088193\\
3.96506506506507	5.74339700269554\\
3.96996996996997	5.77862957526786\\
3.97487487487488	5.81411831936248\\
3.97977977977978	5.84986524240303\\
3.98468468468469	5.88587236861921\\
3.98958958958959	5.92214173919416\\
3.9944944944945	5.95867541241324\\
3.9993993993994	5.99547546381403\\
4.0043043043043	6.03254398633787\\
4.00920920920921	6.0698830904827\\
4.01411411411411	6.10749490445725\\
4.01901901901902	6.1453815743368\\
4.02392392392392	6.18354526422015\\
4.02882882882883	6.22198815638827\\
4.03373373373373	6.26071245146423\\
4.03863863863864	6.29972036857464\\
4.04354354354354	6.33901414551262\\
4.04844844844845	6.3785960389022\\
4.05335335335335	6.41846832436429\\
4.05825825825826	6.45863329668415\\
4.06316316316316	6.49909326998035\\
4.06806806806807	6.53985057787537\\
4.07297297297297	6.58090757366773\\
4.07787787787788	6.62226663050567\\
4.08278278278278	6.66393014156248\\
4.08768768768769	6.70590052021345\\
4.09259259259259	6.74818020021434\\
4.0974974974975	6.79077163588168\\
4.1024024024024	6.83367730227452\\
4.10730730730731	6.87689969537808\\
4.11221221221221	6.92044133228885\\
4.11711711711712	6.96430475140162\\
4.12202202202202	7.00849251259808\\
4.12692692692693	7.05300719743724\\
4.13183183183183	7.09785140934759\\
4.13673673673674	7.143027773821\\
4.14164164164164	7.18853893860847\\
4.14654654654655	7.23438757391762\\
4.15145145145145	7.2805763726121\\
4.15635635635636	7.32710805041273\\
4.16126126126126	7.37398534610057\\
4.16616616616617	7.42121102172188\\
4.17107107107107	7.46878786279492\\
4.17597597597598	7.5167186785187\\
4.18088088088088	7.56500630198365\\
4.18578578578579	7.61365359038431\\
4.19069069069069	7.66266342523379\\
4.1955955955956	7.71203871258051\\
4.2005005005005	7.7617823832267\\
4.20540540540541	7.81189739294902\\
4.21031031031031	7.8623867227213\\
4.21521521521522	7.91325337893918\\
4.22012012012012	7.96450039364698\\
4.22502502502503	8.01613082476662\\
4.22992992992993	8.06814775632865\\
4.23483483483484	8.12055429870546\\
4.23973973973974	8.17335358884665\\
4.24464464464464	8.22654879051654\\
4.24954954954955	8.280143094534\\
4.25445445445445	8.33413971901435\\
4.25935935935936	8.38854190961367\\
4.26426426426426	8.44335293977527\\
4.26916916916917	8.49857611097852\\
4.27407407407407	8.55421475298999\\
4.27897897897898	8.61027222411684\\
4.28388388388388	8.66675191146274\\
4.28878878878879	8.72365723118598\\
4.29369369369369	8.78099162876019\\
4.2985985985986	8.83875857923733\\
4.3035035035035	8.89696158751315\\
4.30840840840841	8.95560418859527\\
4.31331331331331	9.01468994787355\\
4.31821821821822	9.07422246139317\\
4.32312312312312	9.13420535613018\\
4.32802802802803	9.19464229026957\\
4.33293293293293	9.25553695348608\\
4.33783783783784	9.31689306722745\\
4.34274274274274	9.37871438500055\\
4.34764764764765	9.44100469265998\\
4.35255255255255	9.5037678086994\\
4.35745745745746	9.56700758454576\\
4.36236236236236	9.63072790485606\\
4.36726726726727	9.69493268781704\\
4.37217217217217	9.75962588544762\\
4.37707707707708	9.82481148390413\\
4.38198198198198	9.89049350378857\\
4.38688688688689	9.95667600045948\\
4.39179179179179	10.023363064346\\
4.3966966966967	10.0905588212649\\
4.4016016016016	10.15826743274\\
4.40650650650651	10.2264930963256\\
4.41141141141141	10.2952400459319\\
4.41631631631632	10.3645125521544\\
4.42122122122122	10.4343149226056\\
4.42612612612613	10.5046515022504\\
4.43103103103103	10.5755266737442\\
4.43593593593594	10.6469448577747\\
4.44084084084084	10.7189105134065\\
4.44574574574575	10.791428138429\\
4.45065065065065	10.8645022697077\\
4.45555555555556	10.9381374835388\\
4.46046046046046	11.0123383960069\\
4.46536536536536	11.0871096633466\\
4.47027027027027	11.1624559823068\\
4.47517517517517	11.238382090519\\
4.48008008008008	11.3148927668691\\
4.48498498498498	11.3919928318719\\
4.48988988988989	11.4696871480502\\
4.4947947947948	11.5479806203169\\
4.4996996996997	11.6268781963608\\
4.50460460460461	11.7063848670359\\
4.50950950950951	11.7865056667545\\
4.51441441441441	11.8672456738846\\
4.51931931931932	11.9486100111497\\
4.52422422422422	12.0306038460337\\
4.52912912912913	12.1132323911889\\
4.53403403403403	12.1965009048483\\
4.53893893893894	12.2804146912411\\
4.54384384384384	12.3649791010131\\
4.54874874874875	12.4501995316505\\
4.55365365365365	12.5360814279075\\
4.55855855855856	12.6226302822384\\
4.56346346346346	12.7098516352338\\
4.56836836836837	12.7977510760608\\
4.57327327327327	12.886334242907\\
4.57817817817818	12.9756068234294\\
4.58308308308308	13.0655745552071\\
4.58798798798799	13.1562432261985\\
4.59289289289289	13.2476186752027\\
4.5977977977978	13.3397067923252\\
4.6027027027027	13.4325135194488\\
4.60760760760761	13.5260448507075\\
4.61251251251251	13.6203068329662\\
4.61741741741742	13.7153055663047\\
4.62232232232232	13.8110472045056\\
4.62722722722723	13.9075379555479\\
4.63213213213213	14.0047840821045\\
4.63703703703704	14.1027919020444\\
4.64194194194194	14.2015677889408\\
4.64684684684685	14.3011181725823\\
4.65175175175175	14.4014495394906\\
4.65665665665666	14.5025684334418\\
4.66156156156156	14.604481455994\\
4.66646646646647	14.7071952670187\\
4.67137137137137	14.8107165852379\\
4.67627627627628	14.9150521887663\\
4.68118118118118	15.0202089156587\\
4.68608608608609	15.1261936644623\\
4.69099099099099	15.2330133947747\\
4.6958958958959	15.3406751278068\\
4.7008008008008	15.4491859469516\\
4.70570570570571	15.5585529983579\\
4.71061061061061	15.6687834915099\\
4.71551551551552	15.7798846998122\\
4.72042042042042	15.8918639611801\\
4.72532532532533	16.0047286786361\\
4.73023023023023	16.1184863209119\\
4.73513513513513	16.2331444230558\\
4.74004004004004	16.3487105870464\\
4.74494494494494	16.4651924824121\\
4.74984984984985	16.5825978468563\\
4.75475475475475	16.7009344868886\\
4.75965965965966	16.8202102784624\\
4.76456456456456	16.9404331676186\\
4.76946946946947	17.0616111711348\\
4.77437437437437	17.1837523771814\\
4.77927927927928	17.3068649459839\\
4.78418418418418	17.4309571104916\\
4.78908908908909	17.5560371770522\\
4.79399399399399	17.6821135260937\\
4.7988988988989	17.8091946128119\\
4.8038038038038	17.9372889678658\\
4.80870870870871	18.0664051980783\\
4.81361361361361	18.1965519871444\\
4.81851851851852	18.3277380963465\\
4.82342342342342	18.4599723652755\\
4.82832832832833	18.59326371256\\
4.83323323323323	18.727621136602\\
4.83813813813814	18.8630537163197\\
4.84304304304304	18.9995706118973\\
4.84794794794795	19.1371810655423\\
4.85285285285285	19.2758944022504\\
4.85775775775776	19.4157200305767\\
4.86266266266266	19.5566674434157\\
4.86756756756757	19.6987462187879\\
4.87247247247247	19.8419660206341\\
4.87737737737738	19.9863365996178\\
4.88228228228228	20.1318677939349\\
4.88718718718719	20.2785695301317\\
4.89209209209209	20.4264518239301\\
4.896996996997	20.5755247810614\\
4.9019019019019	20.7257985981081\\
4.90680680680681	20.8772835633535\\
4.91171171171171	21.0299900576397\\
4.91661661661662	21.1839285552342\\
4.92152152152152	21.3391096247042\\
4.92642642642643	21.4955439298002\\
4.93133133133133	21.6532422303473\\
4.93623623623624	21.8122153831462\\
4.94114114114114	21.9724743428817\\
4.94604604604605	22.1340301630409\\
4.95095095095095	22.29689399684\\
4.95585585585586	22.4610770981602\\
4.96076076076076	22.6265908224925\\
4.96566566566567	22.7934466278916\\
4.97057057057057	22.9616560759398\\
4.97547547547548	23.131230832719\\
4.98038038038038	23.3021826697933\\
4.98528528528529	23.4745234652005\\
4.99019019019019	23.6482652044534\\
4.9950950950951	23.8234199815513\\
5	24\\
};
\addlegendentry{true $Z$};

\addplot [color=mycolor2,solid]
  table[row sep=crcr]{%
0.1	1.98719427904073\\
0.104904904904905	2.07441769578057\\
0.10980980980981	2.16047035907614\\
0.114714714714715	2.2453266120805\\
0.11961961961962	2.32896133875697\\
0.124524524524525	2.41134997462382\\
0.129429429429429	2.49246851725928\\
0.134334334334334	2.57229353656115\\
0.139239239239239	2.65080218475543\\
0.144144144144144	2.7279722061485\\
0.149049049049049	2.80378194661738\\
0.153953953953954	2.87821036283285\\
0.158858858858859	2.95123703121033\\
0.163763763763764	3.02284215658327\\
0.168668668668669	3.09300658059431\\
0.173573573573574	3.16171178979933\\
0.178478478478478	3.2289399234797\\
0.183383383383383	3.29467378115817\\
0.188288288288288	3.358896829814\\
0.193193193193193	3.42159321079308\\
0.198098098098098	3.4827477464088\\
0.203003003003003	3.54234594622971\\
0.207907907907908	3.60037401305017\\
0.212812812812813	3.65681884854014\\
0.217717717717718	3.71166805857064\\
0.222622622622623	3.7649099582115\\
0.227527527527528	3.81653357639795\\
0.232432432432432	3.86652866026319\\
0.237337337337337	3.91488567913382\\
0.242242242242242	3.96159582818537\\
0.247147147147147	4.00665103175548\\
0.252052052052052	4.05004394631191\\
0.256956956956957	4.09176796307356\\
0.261861861861862	4.13181721028199\\
0.266766766766767	4.17018655512169\\
0.271671671671672	4.20687160528736\\
0.276576576576577	4.24186871019644\\
0.281481481481481	4.27517496184565\\
0.286386386386386	4.30678819531024\\
0.291291291291291	4.33670698888482\\
0.296196196196196	4.36493066386504\\
0.301101101101101	4.39145928396935\\
0.306006006006006	4.4162936544003\\
0.310910910910911	4.43943532054521\\
0.315815815815816	4.46088656631585\\
0.320720720720721	4.4806504121275\\
0.325625625625626	4.49873061251731\\
0.330530530530531	4.5151316534027\\
0.335435435435435	4.52985874898016\\
0.34034034034034	4.54291783826556\\
0.345245245245245	4.55431558127669\\
0.35015015015015	4.56405935485954\\
0.355055055055055	4.57215724815944\\
0.35995995995996	4.57861805773886\\
0.364864864864865	4.58345128234365\\
0.36976976976977	4.58666711731952\\
0.374674674674675	4.58827644868119\\
0.37957957957958	4.58829084683643\\
0.384484484484485	4.58672255996755\\
0.389389389389389	4.58358450707324\\
0.394294294294294	4.57889027067354\\
0.399199199199199	4.57265408918121\\
0.404104104104104	4.56489084894281\\
0.409009009009009	4.55561607595296\\
0.413913913913914	4.54484592724557\\
0.418818818818819	4.53259718196598\\
0.423723723723724	4.51888723212799\\
0.428628628628629	4.50373407306015\\
0.433533533533534	4.48715629354577\\
0.438438438438438	4.46917306566134\\
0.443343343343343	4.44980413431809\\
0.448248248248248	4.42906980651187\\
0.453153153153153	4.40699094028638\\
0.458058058058058	4.38358893341529\\
0.462962962962963	4.3588857118087\\
0.467867867867868	4.33290371764979\\
0.472772772772773	4.30566589726736\\
0.477677677677678	4.27719568875061\\
0.482582582582583	4.24751700931221\\
0.487487487487487	4.21665424240612\\
0.492392392392392	4.18463222460684\\
0.497297297297297	4.1514762322567\\
0.502202202202202	4.11721196788817\\
0.507107107107107	4.08186554642824\\
0.512012012012012	4.04546348119214\\
0.516916916916917	4.00803266967368\\
0.521821821821822	3.96960037913981\\
0.526726726726727	3.93019423203705\\
0.531631631631632	3.88984219121765\\
0.536536536536537	3.84857254499335\\
0.541441441441441	3.80641389202503\\
0.546346346346346	3.76339512605628\\
0.551251251251251	3.71954542049944\\
0.556156156156156	3.67489421288257\\
0.561061061061061	3.62947118916599\\
0.565965965965966	3.58330626793724\\
0.570870870870871	3.53642958449329\\
0.575775775775776	3.48887147481907\\
0.580680680680681	3.44066245947152\\
0.585585585585586	3.39183322737845\\
0.590490490490491	3.34241461956149\\
0.595395395395395	3.29243761279283\\
0.6003003003003	3.24193330319534\\
0.605205205205205	3.19093288979574\\
0.61011011011011	3.1394676580409\\
0.615015015015015	3.08756896328704\\
0.61991991991992	3.03526821427216\\
0.624824824824825	2.98259685658186\\
0.62972972972973	2.92958635611883\\
0.634634634634635	2.87626818258665\\
0.63953953953954	2.82267379299847\\
0.644444444444444	2.76883461522122\\
0.649349349349349	2.71478203156632\\
0.654254254254254	2.66054736243799\\
0.659159159159159	2.60616185005009\\
0.664064064064064	2.55165664222312\\
0.668968968968969	2.49706277627263\\
0.673873873873874	2.44241116300088\\
0.678778778778779	2.38773257080349\\
0.683683683683684	2.33305760990328\\
0.688588588588589	2.27841671672357\\
0.693493493493494	2.22384013841338\\
0.698398398398398	2.16935791753761\\
0.703303303303303	2.11499987694506\\
0.708208208208208	2.06079560482798\\
0.713113113113113	2.00677443998693\\
0.718018018018018	1.95296545731519\\
0.722922922922923	1.89939745351757\\
0.727827827827828	1.8460989330788\\
0.732732732732733	1.79309809449752\\
0.737637637637638	1.74042281680236\\
0.742542542542543	1.68810064636749\\
0.747447447447448	1.63615878404599\\
0.752352352352352	1.58462407264011\\
0.757257257257257	1.53352298472908\\
0.762162162162162	1.48288161087598\\
0.767067067067067	1.43272564823716\\
0.771971971971972	1.38308038959908\\
0.776876876876877	1.33397071286968\\
0.781781781781782	1.28542107105377\\
0.786686686686687	1.23745548274438\\
0.791591591591592	1.19009752316552\\
0.796496496496497	1.14337031580523\\
0.801401401401401	1.09729652468223\\
0.806306306306306	1.05189834729449\\
0.811211211211211	1.00719750830405\\
0.816116116116116	0.963215254019307\\
0.821021021021021	0.91997234774451\\
0.825925925925926	0.877489066075878\\
0.830830830830831	0.835785196235692\\
0.835735735735736	0.794880034549824\\
0.840640640640641	0.754792386191178\\
0.845545545545546	0.715540566332155\\
0.85045045045045	0.677142402874267\\
0.855355355355355	0.639615240953672\\
0.86026026026026	0.602975949459026\\
0.865165165165165	0.567240929844591\\
0.87007007007007	0.532426127579519\\
0.874974974974975	0.498547046646862\\
0.87987987987988	0.465618767597681\\
0.884784784784785	0.43365596978247\\
0.88968968968969	0.402672958532284\\
0.894594594594595	0.372683698256682\\
0.899499499499499	0.343701852680653\\
0.904404404404404	0.315740833780674\\
0.909309309309309	0.288813861433143\\
0.914214214214214	0.262934036404143\\
0.919119119119119	0.238114430158431\\
0.924024024024024	0.214368196154951\\
0.928928928928929	0.191708708992839\\
0.933833833833834	0.170149740240815\\
0.938738738738739	0.149705683457219\\
0.943643643643644	0.130391846517102\\
0.948548548548549	0.112224838178106\\
0.953453453453453	0.0952230901436303\\
0.958358358358358	0.0794075800968016\\
0.963263263263263	0.0648028640534233\\
0.968168168168168	0.0514386066719186\\
0.973073073073073	0.0393519592770321\\
0.977977977977978	0.0285914883355696\\
0.982882882882883	0.0192242255145858\\
0.987787787787788	0.0113499237272344\\
0.992692692692693	0.00513595751758842\\
0.997597597597598	0.000943062350490414\\
1.0025025025025	0.123219600699149\\
1.00740740740741	0.206502732927866\\
1.01231231231231	0.260255694915842\\
1.01721721721722	0.301454840769415\\
1.02212212212212	0.335181366618455\\
1.02702702702703	0.363804069757406\\
1.03193193193193	0.388661650385554\\
1.03683683683684	0.410600468500395\\
1.04174174174174	0.430196846271929\\
1.04664664664665	0.447864566659156\\
1.05155155155155	0.463912755623319\\
1.05645645645646	0.478579606918385\\
1.06136136136136	0.492053259373062\\
1.06626626626627	0.504485358110543\\
1.07117117117117	0.516000215565633\\
1.07607607607608	0.526701204839127\\
1.08098098098098	0.536675345906218\\
1.08588588588589	0.54599667381713\\
1.09079079079079	0.554728763276098\\
1.0956956956957	0.562926654898038\\
1.1006006006006	0.570638348204173\\
1.10550550550551	0.577905975060462\\
1.11041041041041	0.584766733530255\\
1.11531531531532	0.591253639441724\\
1.12022022022022	0.597396137417479\\
1.12512512512513	0.603220602244386\\
1.13003003003003	0.608750753737072\\
1.13493493493494	0.61400800267467\\
1.13983983983984	0.619011741311913\\
1.14474474474474	0.623779588943052\\
1.14964964964965	0.628327600730489\\
1.15455455455455	0.632670446291709\\
1.15945945945946	0.636821563222352\\
1.16436436436436	0.640793289716244\\
1.16926926926927	0.644596979650227\\
1.17417417417417	0.64824310287826\\
1.17907907907908	0.651741332985528\\
1.18398398398398	0.655100624359382\\
1.18888888888889	0.658329280117548\\
1.19379379379379	0.661435012178372\\
1.1986986986987	0.664424994549926\\
1.2036036036036	0.667305910744811\\
1.20850850850851	0.670083996087744\\
1.21341341341341	0.672765075567543\\
1.21831831831832	0.675354597789268\\
1.22322322322322	0.677857665502336\\
1.22812812812813	0.680279063113458\\
1.23303303303303	0.682623281536931\\
1.23793793793794	0.684894540687271\\
1.24284284284284	0.687096809878843\\
1.24774774774775	0.689233826362887\\
1.25265265265265	0.691309112203041\\
1.25755755755756	0.693325989665442\\
1.26246246246246	0.695287595277909\\
1.26736736736737	0.697196892694198\\
1.27227227227227	0.699056684483269\\
1.27717717717718	0.700869622949616\\
1.28208208208208	0.702638220078638\\
1.28698698698699	0.704364856690508\\
1.29189189189189	0.706051790876831\\
1.2967967967968	0.707701165786308\\
1.3017017017017	0.709315016818595\\
1.30660660660661	0.710895278279339\\
1.31151151151151	0.712443789543886\\
1.31641641641642	0.713962300772347\\
1.32132132132132	0.715452478214429\\
1.32622622622623	0.716915909138636\\
1.33113113113113	0.7183541064171\\
1.33603603603604	0.719768512794297\\
1.34094094094094	0.721160504865224\\
1.34584584584585	0.722531396786247\\
1.35075075075075	0.723882443739678\\
1.35565565565566	0.725214845171244\\
1.36056056056056	0.726529747817885\\
1.36546546546547	0.727828248541788\\
1.37037037037037	0.729111396985152\\
1.37527527527528	0.730380198058981\\
1.38018018018018	0.731635614277998\\
1.38508508508509	0.73287856795283\\
1.38998998998999	0.734109943249643\\
1.3948948948949	0.73533058812658\\
1.3997997997998	0.736541316155604\\
1.4047047047047	0.737742908237665\\
1.40960960960961	0.738936114218454\\
1.41451451451451	0.740121654411475\\
1.41941941941942	0.741300221034618\\
1.42432432432432	0.74247247956595\\
1.42922922922923	0.74363907002402\\
1.43413413413413	0.74480060817755\\
1.43903903903904	0.745957686689053\\
1.44394394394394	0.747110876196566\\
1.44884884884885	0.748260726337381\\
1.45375375375375	0.749407766717402\\
1.45865865865866	0.750552507829451\\
1.46356356356356	0.751695441923674\\
1.46846846846847	0.752837043832917\\
1.47337337337337	0.753977771755786\\
1.47827827827828	0.755118067999898\\
1.48318318318318	0.756258359687673\\
1.48808808808809	0.757399059426839\\
1.49299299299299	0.758540565947712\\
1.4978978978979	0.759683264709133\\
1.5028028028028	0.760827528474859\\
1.50770770770771	0.761973717862071\\
1.51261261261261	0.76312218186355\\
1.51751751751752	0.764273258344989\\
1.52242242242242	0.765427274518808\\
1.52732732732733	0.766584547395746\\
1.53223223223223	0.767745384215445\\
1.53713713713714	0.76891008285714\\
1.54204204204204	0.770078932231525\\
1.54694694694695	0.77125221265479\\
1.55185185185185	0.772430196205755\\
1.55675675675676	0.773613147066993\\
1.56166166166166	0.774801321850766\\
1.56656656656657	0.775994969910544\\
1.57147147147147	0.777194333638861\\
1.57637637637638	0.778399648752178\\
1.58128128128128	0.779611144563421\\
1.58618618618619	0.780829044242796\\
1.59109109109109	0.782053565067482\\
1.595995995996	0.783284918660724\\
1.6009009009009	0.784523311220855\\
1.60580580580581	0.785768943740752\\
1.61071071071071	0.787022012218152\\
1.61561561561562	0.788282707857303\\
1.62052052052052	0.789551217262334\\
1.62542542542543	0.790827722622754\\
1.63033033033033	0.792112401891441\\
1.63523523523524	0.793405428955472\\
1.64014014014014	0.794706973800136\\
1.64504504504505	0.796017202666423\\
1.64994994994995	0.797336278202318\\
1.65485485485486	0.798664359608156\\
1.65975975975976	0.800001602776324\\
1.66466466466466	0.801348160425552\\
1.66956956956957	0.80270418223005\\
1.67447447447447	0.804069814943709\\
1.67937937937938	0.805445202519588\\
1.68428428428428	0.806830486224896\\
1.68918918918919	0.808225804751667\\
1.69409409409409	0.809631294323316\\
1.698998998999	0.811047088797245\\
1.7039039039039	0.812473319763691\\
1.70880880880881	0.813910116640952\\
1.71371371371371	0.815357606767166\\
1.71861861861862	0.81681591548877\\
1.72352352352352	0.81828516624581\\
1.72842842842843	0.81976548065419\\
1.73333333333333	0.821256978585035\\
1.73823823823824	0.822759778241253\\
1.74314314314314	0.824273996231429\\
1.74804804804805	0.825799747641161\\
1.75295295295295	0.827337146101938\\
1.75785785785786	0.828886303857654\\
1.76276276276276	0.830447331828882\\
1.76766766766767	0.832020339674957\\
1.77257257257257	0.833605435853995\\
1.77747747747748	0.835202727680912\\
1.78238238238238	0.836812321383518\\
1.78728728728729	0.838434322156782\\
1.79219219219219	0.840068834215324\\
1.7970970970971	0.84171596084421\\
1.802002002002	0.843375804448118\\
1.80690690690691	0.845048466598938\\
1.81181181181181	0.846734048081867\\
1.81671671671672	0.848432648940058\\
1.82162162162162	0.850144368517875\\
1.82652652652653	0.851869305502824\\
1.83143143143143	0.853607557966185\\
1.83633633633634	0.855359223402419\\
1.84124124124124	0.857124398767381\\
1.84614614614615	0.858903180515401\\
1.85105105105105	0.860695664635252\\
1.85595595595596	0.862501946685077\\
1.86086086086086	0.864322121826286\\
1.86576576576577	0.866156284856483\\
1.87067067067067	0.868004530241452\\
1.87557557557558	0.869866952146232\\
1.88048048048048	0.871743644465325\\
1.88538538538539	0.873634700852066\\
1.89029029029029	0.875540214747179\\
1.8951951951952	0.877460279406565\\
1.9001001001001	0.879394987928335\\
1.90500500500501	0.881344433279123\\
1.90990990990991	0.88330870831971\\
1.91481481481482	0.885287905829975\\
1.91971971971972	0.887282118533207\\
1.92462462462462	0.889291439119793\\
1.92952952952953	0.891315960270317\\
1.93443443443443	0.893355774678076\\
1.93933933933934	0.895410975071051\\
1.94424424424424	0.89748165423334\\
1.94914914914915	0.899567905026079\\
1.95405405405405	0.901669820407871\\
1.95895895895896	0.903787493454741\\
1.96386386386386	0.905921017379623\\
1.96876876876877	0.908070485551422\\
1.97367367367367	0.910235991513635\\
1.97857857857858	0.912417629002572\\
1.98348348348348	0.914615491965183\\
1.98838838838839	0.916829674576501\\
1.99329329329329	0.919060271256722\\
1.9981981981982	0.921307376687937\\
2.0031031031031	0.923571085830519\\
2.00800800800801	0.925851493939188\\
2.01291291291291	0.928148696578763\\
2.01781781781782	0.930462789639606\\
2.02272272272272	0.932793869352789\\
2.02762762762763	0.935142032304963\\
2.03253253253253	0.937507375452976\\
2.03743743743744	0.939889996138218\\
2.04234234234234	0.942289992100728\\
2.04724724724725	0.944707461493054\\
2.05215215215215	0.94714250289389\\
2.05705705705706	0.949595215321487\\
2.06196196196196	0.952065698246852\\
2.06686686686687	0.954554051606747\\
2.07177177177177	0.957060375816488\\
2.07667667667668	0.959584771782559\\
2.08158158158158	0.962127340915042\\
2.08648648648649	0.964688185139878\\
2.09139139139139	0.967267406910962\\
2.0962962962963	0.969865109222073\\
2.1012012012012	0.972481395618656\\
2.10610610610611	0.975116370209461\\
2.11101101101101	0.977770137678038\\
2.11591591591592	0.980442803294095\\
2.12082082082082	0.983134472924736\\
2.12572572572573	0.985845253045572\\
2.13063063063063	0.988575250751718\\
2.13553553553554	0.991324573768672\\
2.14044044044044	0.994093330463097\\
2.14534534534535	0.996881629853499\\
2.15025025025025	0.999689581620803\\
2.15515515515516	1.00251729611884\\
2.16006006006006	1.00536488438477\\
2.16496496496497	1.00823245814935\\
2.16986986986987	1.01112012984725\\
2.17477477477478	1.01402801262715\\
2.17967967967968	1.01695622036188\\
2.18458458458458	1.01990486765845\\
2.18948948948949	1.02287406986799\\
2.19439439439439	1.02586394309568\\
2.1992992992993	1.02887460421063\\
2.2042042042042	1.03190617085565\\
2.20910910910911	1.03495876145703\\
2.21401401401401	1.03803249523423\\
2.21891891891892	1.04112749220959\\
2.22382382382382	1.04424387321794\\
2.22872872872873	1.04738175991621\\
2.23363363363363	1.05054127479299\\
2.23853853853854	1.0537225411781\\
2.24344344344344	1.05692568325206\\
2.24834834834835	1.06015082605562\\
2.25325325325325	1.06339809549923\\
2.25815815815816	1.06666761837246\\
2.26306306306306	1.06995952235349\\
2.26796796796797	1.0732739360185\\
2.27287287287287	1.07661098885112\\
2.27777777777778	1.0799708112518\\
2.28268268268268	1.08335353454727\\
2.28758758758759	1.08675929099987\\
2.29249249249249	1.09018821381705\\
2.2973973973974	1.09364043716069\\
2.3023023023023	1.09711609615653\\
2.30720720720721	1.10061532690364\\
2.31211211211211	1.10413826648375\\
2.31701701701702	1.10768505297078\\
2.32192192192192	1.11125582544021\\
2.32682682682683	1.11485072397858\\
2.33173173173173	1.11846988969298\\
2.33663663663664	1.12211346472049\\
2.34154154154154	1.12578159223775\\
2.34644644644645	1.12947441647049\\
2.35135135135135	1.13319208270305\\
2.35625625625626	1.13693473728802\\
2.36116116116116	1.14070252765581\\
2.36606606606607	1.14449560232431\\
2.37097097097097	1.14831411090859\\
2.37587587587588	1.15215820413054\\
2.38078078078078	1.15602803382868\\
2.38568568568569	1.15992375296789\\
2.39059059059059	1.16384551564922\\
2.3954954954955	1.16779347711977\\
2.4004004004004	1.17176779378255\\
2.40530530530531	1.17576862320644\\
2.41021021021021	1.17979612413613\\
2.41511511511512	1.18385045650216\\
2.42002002002002	1.187931781431\\
2.42492492492493	1.19204026125511\\
2.42982982982983	1.19617605952317\\
2.43473473473474	1.20033934101023\\
2.43963963963964	1.20453027172801\\
2.44454454454455	1.20874901893522\\
2.44944944944945	1.21299575114787\\
2.45435435435435	1.2172706381498\\
2.45925925925926	1.22157385100305\\
2.46416416416416	1.22590556205849\\
2.46906906906907	1.23026594496637\\
2.47397397397397	1.234655174687\\
2.47887887887888	1.23907342750147\\
2.48378378378378	1.24352088102243\\
2.48868868868869	1.24799771420496\\
2.49359359359359	1.25250410735747\\
2.4984984984985	1.25704024215271\\
2.5034034034034	1.26160630163878\\
2.50830830830831	1.26620247025029\\
2.51321321321321	1.27082893381955\\
2.51811811811812	1.27548587958781\\
2.52302302302302	1.28017349621665\\
2.52792792792793	1.28489197379933\\
2.53283283283283	1.28964150387233\\
2.53773773773774	1.2944222794269\\
2.54264264264264	1.29923449492069\\
2.54754754754755	1.30407834628951\\
2.55245245245245	1.3089540309591\\
2.55735735735736	1.31386174785702\\
2.56226226226226	1.31880169742465\\
2.56716716716717	1.32377408162921\\
2.57207207207207	1.32877910397591\\
2.57697697697698	1.33381696952017\\
2.58188188188188	1.33888788487994\\
2.58678678678679	1.34399205824812\\
2.59169169169169	1.34912969940499\\
2.5965965965966	1.35430101973086\\
2.6015015015015	1.35950623221871\\
2.60640640640641	1.36474555148696\\
2.61131131131131	1.37001919379233\\
2.61621621621622	1.3753273770428\\
2.62112112112112	1.38067032081069\\
2.62602602602603	1.38604824634577\\
2.63093093093093	1.39146137658854\\
2.63583583583584	1.39690993618356\\
2.64074074074074	1.40239415149294\\
2.64564564564565	1.40791425060987\\
2.65055055055055	1.41347046337227\\
2.65545545545546	1.4190630213766\\
2.66036036036036	1.42469215799167\\
2.66526526526527	1.43035810837269\\
2.67017017017017	1.43606110947529\\
2.67507507507508	1.44180140006977\\
2.67997997997998	1.44757922075538\\
2.68488488488489	1.45339481397476\\
2.68978978978979	1.45924842402846\\
2.6946946946947	1.46514029708959\\
2.6995995995996	1.47107068121861\\
2.7045045045045	1.47703982637817\\
2.70940940940941	1.48304798444817\\
2.71431431431431	1.4890954092408\\
2.71921921921922	1.49518235651588\\
2.72412412412412	1.50130908399614\\
2.72902902902903	1.50747585138274\\
2.73393393393393	1.51368292037091\\
2.73883883883884	1.51993055466563\\
2.74374374374374	1.52621901999754\\
2.74864864864865	1.53254858413892\\
2.75355355355355	1.53891951691981\\
2.75845845845846	1.54533209024426\\
2.76336336336336	1.55178657810673\\
2.76826826826827	1.55828325660861\\
2.77317317317317	1.56482240397486\\
2.77807807807808	1.5714043005708\\
2.78298298298298	1.57802922891908\\
2.78788788788789	1.58469747371669\\
2.79279279279279	1.59140932185221\\
2.7976976976977	1.59816506242314\\
2.8026026026026	1.60496498675341\\
2.80750750750751	1.61180938841099\\
2.81241241241241	1.61869856322571\\
2.81731731731732	1.62563280930715\\
2.82222222222222	1.63261242706276\\
2.82712712712713	1.63963771921607\\
2.83203203203203	1.64670899082505\\
2.83693693693694	1.65382654930068\\
2.84184184184184	1.66099070442563\\
2.84674674674675	1.66820176837308\\
2.85165165165165	1.67546005572574\\
2.85655655655656	1.68276588349501\\
2.86146146146146	1.6901195711403\\
2.86636636636637	1.6975214405885\\
2.87127127127127	1.70497181625363\\
2.87617617617618	1.71247102505666\\
2.88108108108108	1.72001939644548\\
2.88598598598599	1.72761726241504\\
2.89089089089089	1.73526495752766\\
2.8957957957958	1.74296281893353\\
2.9007007007007	1.75071118639133\\
2.90560560560561	1.75851040228912\\
2.91051051051051	1.7663608116653\\
2.91541541541542	1.77426276222979\\
2.92032032032032	1.78221660438543\\
2.92522522522523	1.79022269124949\\
2.93013013013013	1.79828137867543\\
2.93503503503504	1.80639302527476\\
2.93993993993994	1.81455799243919\\
2.94484484484484	1.82277664436285\\
2.94974974974975	1.83104934806485\\
2.95465465465466	1.83937647341184\\
2.95955955955956	1.84775839314096\\
2.96446446446447	1.8561954828828\\
2.96936936936937	1.86468812118474\\
2.97427427427427	1.87323668953431\\
2.97917917917918	1.88184157238288\\
2.98408408408408	1.8905031571695\\
2.98898898898899	1.89922183434492\\
2.99389389389389	1.90799799739589\\
2.9987987987988	1.91683204286957\\
3.0037037037037	1.92572437039824\\
3.00860860860861	1.93467538272416\\
3.01351351351351	1.94368548572466\\
3.01841841841842	1.95275508843746\\
3.02332332332332	1.96188460308618\\
3.02822822822823	1.97107444510609\\
3.03313313313313	1.9803250331701\\
3.03803803803804	1.98963678921487\\
3.04294294294294	1.99901013846731\\
3.04784784784785	2.00844550947117\\
3.05275275275275	2.01794333411388\\
3.05765765765766	2.02750404765373\\
3.06256256256256	2.0371280887471\\
3.06746746746747	2.0468158994761\\
3.07237237237237	2.05656792537634\\
3.07727727727728	2.06638461546496\\
3.08218218218218	2.07626642226893\\
3.08708708708709	2.08621380185357\\
3.09199199199199	2.09622721385133\\
3.0968968968969	2.1063071214908\\
3.1018018018018	2.11645399162597\\
3.10670670670671	2.12666829476575\\
3.11161161161161	2.1369505051038\\
3.11651651651652	2.14730110054848\\
3.12142142142142	2.1577205627532\\
3.12632632632633	2.16820937714697\\
3.13123123123123	2.17876803296518\\
3.13613613613614	2.18939702328075\\
3.14104104104104	2.20009684503543\\
3.14594594594595	2.21086799907143\\
3.15085085085085	2.22171099016335\\
3.15575575575576	2.23262632705035\\
3.16066066066066	2.24361452246857\\
3.16556556556557	2.2546760931839\\
3.17047047047047	2.26581156002498\\
3.17537537537538	2.27702144791654\\
3.18028028028028	2.28830628591294\\
3.18518518518519	2.2996666072321\\
3.19009009009009	2.31110294928966\\
3.194994994995	2.32261585373349\\
3.1998998998999	2.33420586647841\\
3.20480480480481	2.34587353774132\\
3.20970970970971	2.35761942207656\\
3.21461461461461	2.3694440784116\\
3.21951951951952	2.38134807008309\\
3.22442442442442	2.3933319648731\\
3.22932932932933	2.4053963350458\\
3.23423423423423	2.41754175738442\\
3.23913913913914	2.42976881322849\\
3.24404404404404	2.44207808851146\\
3.24894894894895	2.45447017379864\\
3.25385385385385	2.46694566432541\\
3.25875875875876	2.47950516003587\\
3.26366366366366	2.49214926562173\\
3.26856856856857	2.50487859056158\\
3.27347347347347	2.51769374916053\\
3.27837837837838	2.53059536059013\\
3.28328328328328	2.54358404892869\\
3.28818818818819	2.55666044320197\\
3.29309309309309	2.56982517742413\\
3.297997997998	2.58307889063917\\
3.3029029029029	2.59642222696264\\
3.30780780780781	2.6098558356237\\
3.31271271271271	2.62338037100768\\
3.31761761761762	2.63699649269882\\
3.32252252252252	2.65070486552358\\
3.32742742742743	2.66450615959417\\
3.33233233233233	2.67840105035255\\
3.33723723723724	2.69239021861481\\
3.34214214214214	2.7064743506159\\
3.34704704704705	2.72065413805481\\
3.35195195195195	2.73493027814005\\
3.35685685685686	2.74930347363568\\
3.36176176176176	2.76377443290759\\
3.36666666666667	2.77834386997026\\
3.37157157157157	2.79301250453398\\
3.37647647647648	2.80778106205236\\
3.38138138138138	2.82265027377038\\
3.38628628628629	2.83762087677277\\
3.39119119119119	2.85269361403292\\
3.3960960960961	2.86786923446209\\
3.401001001001	2.88314849295915\\
3.40590590590591	2.89853215046075\\
3.41081081081081	2.91402097399189\\
3.41571571571572	2.92961573671695\\
3.42062062062062	2.94531721799121\\
3.42552552552553	2.96112620341278\\
3.43043043043043	2.97704348487497\\
3.43533533533534	2.99306986061922\\
3.44024024024024	3.00920613528838\\
3.44514514514515	3.02545311998051\\
3.45005005005005	3.0418116323032\\
3.45495495495496	3.05828249642831\\
3.45985985985986	3.07486654314718\\
3.46476476476477	3.09156460992641\\
3.46966966966967	3.10837754096406\\
3.47457457457458	3.12530618724638\\
3.47947947947948	3.14235140660503\\
3.48438438438439	3.1595140637748\\
3.48928928928929	3.17679503045187\\
3.49419419419419	3.1941951853526\\
3.4990990990991	3.21171541427274\\
3.504004004004	3.22935661014728\\
3.50890890890891	3.24711967311078\\
3.51381381381381	3.26500551055826\\
3.51871871871872	3.28301503720659\\
3.52362362362362	3.30114917515644\\
3.52852852852853	3.31940885395481\\
3.53343343343343	3.33779501065811\\
3.53833833833834	3.35630858989575\\
3.54324324324324	3.37495054393433\\
3.54814814814815	3.39372183274244\\
3.55305305305305	3.41262342405596\\
3.55795795795796	3.43165629344402\\
3.56286286286286	3.45082142437546\\
3.56776776776777	3.47011980828596\\
3.57267267267267	3.48955244464572\\
3.57757757757758	3.50912034102779\\
3.58248248248248	3.52882451317694\\
3.58738738738739	3.54866598507921\\
3.59229229229229	3.56864578903205\\
3.5971971971972	3.58876496571509\\
3.6021021021021	3.60902456426153\\
3.60700700700701	3.6294256423302\\
3.61191191191191	3.64996926617822\\
3.61681681681682	3.67065651073435\\
3.62172172172172	3.69148845967299\\
3.62662662662663	3.71246620548878\\
3.63153153153153	3.73359084957197\\
3.63643643643644	3.75486350228439\\
3.64134134134134	3.7762852830361\\
3.64624624624625	3.79785732036281\\
3.65115115115115	3.81958075200386\\
3.65605605605606	3.84145672498099\\
3.66096096096096	3.86348639567781\\
3.66586586586587	3.88567092991999\\
3.67077077077077	3.90801150305608\\
3.67567567567568	3.93050930003917\\
3.68058058058058	3.95316551550923\\
3.68548548548549	3.97598135387617\\
3.69039039039039	3.99895802940371\\
3.6952952952953	4.02209676629389\\
3.7002002002002	4.04539879877251\\
3.70510510510511	4.06886537117515\\
3.71001001001001	4.0924977380341\\
3.71491491491492	4.11629716416601\\
3.71981981981982	4.14026492476035\\
3.72472472472473	4.16440230546864\\
3.72962962962963	4.18871060249452\\
3.73453453453453	4.21319112268462\\
3.73943943943944	4.23784518362019\\
3.74434434434434	4.26267411370969\\
3.74924924924925	4.28767925228206\\
3.75415415415415	4.31286194968092\\
3.75905905905906	4.33822356735962\\
3.76396396396396	4.3637654779771\\
3.76886886886887	4.38948906549469\\
3.77377377377377	4.41539572527367\\
3.77867867867868	4.44148686417384\\
3.78358358358358	4.46776390065288\\
3.78848848848849	4.49422826486666\\
3.79339339339339	4.52088139877045\\
3.7982982982983	4.54772475622103\\
3.8032032032032	4.57475980307973\\
3.80810810810811	4.60198801731639\\
3.81301301301301	4.62941088911431\\
3.81791791791792	4.65702992097609\\
3.82282282282282	4.68484662783045\\
3.82772772772773	4.71286253714004\\
3.83263263263263	4.74107918901016\\
3.83753753753754	4.76949813629859\\
3.84244244244244	4.79812094472627\\
3.84734734734735	4.82694919298908\\
3.85225225225225	4.8559844728706\\
3.85715715715716	4.88522838935591\\
3.86206206206206	4.91468256074642\\
3.86696696696697	4.94434861877567\\
3.87187187187187	4.97422820872636\\
3.87677677677678	5.00432298954823\\
3.88168168168168	5.03463463397715\\
3.88658658658659	5.06516482865529\\
3.89149149149149	5.09591527425227\\
3.8963963963964	5.12688768558756\\
3.9013013013013	5.15808379175389\\
3.90620620620621	5.18950533624183\\
3.91111111111111	5.22115407706549\\
3.91601601601602	5.25303178688935\\
3.92092092092092	5.28514025315629\\
3.92582582582583	5.31748127821675\\
3.93073073073073	5.35005667945907\\
3.93563563563564	5.38286828944104\\
3.94054054054054	5.4159179560226\\
3.94544544544545	5.44920754249987\\
3.95035035035035	5.48273892774021\\
3.95525525525526	5.51651400631872\\
3.96016016016016	5.55053468865585\\
3.96506506506507	5.5848029011563\\
3.96996996996997	5.61932058634922\\
3.97487487487488	5.65408970302968\\
3.97977977977978	5.68911222640138\\
3.98468468468469	5.72439014822077\\
3.98958958958959	5.75992547694238\\
3.9944944944945	5.79572023786556\\
3.9993993993994	5.83177647328252\\
4.0043043043043	5.86809624262774\\
4.00920920920921	5.90468162262879\\
4.01411411411411	5.94153470745841\\
4.01901901901902	5.97865760888814\\
4.02392392392392	6.01605245644326\\
4.02882882882883	6.05372139755916\\
4.03373373373373	6.09166659773922\\
4.03863863863864	6.129890240714\\
4.04354354354354	6.16839452860204\\
4.04844844844845	6.20718168207204\\
4.05335335335335	6.24625394050659\\
4.05825825825826	6.28561356216735\\
4.06316316316316	6.32526282436176\\
4.06806806806807	6.36520402361136\\
4.07297297297297	6.40543947582156\\
4.07787787787788	6.44597151645297\\
4.08278278278278	6.48680250069444\\
4.08768768768769	6.52793480363755\\
4.09259259259259	6.56937082045274\\
4.0974974974975	6.61111296656711\\
4.1024024024024	6.65316367784384\\
4.10730730730731	6.69552541076325\\
4.11221221221221	6.7382006426055\\
4.11711711711712	6.78119187163509\\
4.12202202202202	6.82450161728688\\
4.12692692692693	6.86813242035404\\
4.13183183183183	6.91208684317755\\
4.13673673673674	6.95636746983757\\
4.14164164164164	7.00097690634652\\
4.14654654654655	7.04591778084396\\
4.15145145145145	7.09119274379329\\
4.15635635635636	7.13680446818027\\
4.16126126126126	7.18275564971328\\
4.16616616616617	7.22904900702561\\
4.17107107107107	7.27568728187951\\
4.17597597597598	7.32267323937206\\
4.18088088088088	7.37000966814315\\
4.18578578578579	7.41769938058522\\
4.19069069069069	7.46574521305496\\
4.1955955955956	7.51415002608709\\
4.2005005005005	7.56291670460996\\
4.20540540540541	7.61204815816329\\
4.21031031031031	7.66154732111788\\
4.21521521521522	7.71141715289726\\
4.22012012012012	7.76166063820161\\
4.22502502502503	7.8122807872336\\
4.22992992992993	7.86328063592629\\
4.23483483483484	7.91466324617338\\
4.23973973973974	7.96643170606131\\
4.24464464464464	8.01858913010376\\
4.24954954954955	8.07113865947817\\
4.25445445445445	8.12408346226453\\
4.25935935935936	8.17742673368642\\
4.26426426426426	8.2311716963542\\
4.26916916916917	8.2853216005106\\
4.27407407407407	8.33987972427848\\
4.27897897897898	8.3948493739109\\
4.28388388388388	8.45023388404367\\
4.28878878878879	8.50603661795006\\
4.29369369369369	8.562260967798\\
4.2985985985986	8.61891035490973\\
4.3035035035035	8.67598823002366\\
4.30840840840841	8.73349807355894\\
4.31331331331331	8.79144339588231\\
4.31821821821822	8.84982773757755\\
4.32312312312312	8.90865466971739\\
4.32802802802803	8.96792779413798\\
4.33293293293293	9.02765074371595\\
4.33783783783784	9.08782718264799\\
4.34274274274274	9.14846080673317\\
4.34764764764765	9.20955534365781\\
4.35255255255255	9.27111455328297\\
4.35745745745746	9.3331422279348\\
4.36236236236236	9.39564219269742\\
4.36726726726727	9.45861830570874\\
4.37217217217217	9.52207445845885\\
4.37707707707708	9.58601457609136\\
4.38198198198198	9.65044261770751\\
4.38688688688689	9.71536257667312\\
4.39179179179179	9.78077848092848\\
4.3966966966967	9.84669439330106\\
4.4016016016016	9.91311441182124\\
4.40650650650651	9.98004267004097\\
4.41141141141141	10.0474833373554\\
4.41631631631632	10.1154406193276\\
4.42122122122122	10.1839187580163\\
4.42612612612613	10.2529220323067\\
4.43103103103103	10.3224547582443\\
4.43593593593594	10.3925212893721\\
4.44084084084084	10.4631260170708\\
4.44574574574575	10.5342733709024\\
4.45065065065065	10.6059678189566\\
4.45555555555556	10.678213868201\\
4.46046046046046	10.751016064834\\
4.46536536536536	10.824378994642\\
4.47027027027027	10.8983072833589\\
4.47517517517517	10.9728055970292\\
4.48008008008008	11.0478786423755\\
4.48498498498498	11.1235311671683\\
4.48988988988989	11.1997679605995\\
4.4947947947948	11.2765938536603\\
4.4996996996997	11.3540137195212\\
4.50460460460461	11.4320324739174\\
4.50950950950951	11.5106550755359\\
4.51441441441441	11.589886526408\\
4.51931931931932	11.6697318723043\\
4.52422422422422	11.7501962031339\\
4.52912912912913	11.8312846533475\\
4.53403403403403	11.9130024023442\\
4.53893893893894	11.995354674882\\
4.54384384384384	12.0783467414924\\
4.54874874874875	12.1619839188986\\
4.55365365365365	12.2462715704385\\
4.55855855855856	12.3312151064903\\
4.56346346346346	12.4168199849038\\
4.56836836836837	12.5030917114343\\
4.57327327327327	12.5900358401815\\
4.57817817817818	12.6776579740326\\
4.58308308308308	12.7659637651087\\
4.58798798798799	12.8549589152168\\
4.59289289289289	12.944649176305\\
4.5977977977978	13.0350403509223\\
4.6027027027027	13.1261382926832\\
4.60760760760761	13.2179489067361\\
4.61251251251251	13.3104781502368\\
4.61741741741742	13.4037320328258\\
4.62232232232232	13.4977166171106\\
4.62722722722723	13.5924380191526\\
4.63213213213213	13.6879024089586\\
4.63703703703704	13.7841160109765\\
4.64194194194194	13.8810851045967\\
4.64684684684685	13.9788160246574\\
4.65175175175175	14.0773151619551\\
4.65665665665666	14.1765889637597\\
4.66156156156156	14.2766439343351\\
4.66646646646647	14.3774866354644\\
4.67137137137137	14.4791236869794\\
4.67627627627628	14.5815617672969\\
4.68118118118118	14.6848076139583\\
4.68608608608609	14.7888680241755\\
4.69099099099099	14.8937498553816\\
4.6958958958959	14.9994600257868\\
4.7008008008008	15.1060055149399\\
4.70570570570571	15.2133933642949\\
4.71061061061061	15.3216306777833\\
4.71551551551552	15.4307246223916\\
4.72042042042042	15.5406824287444\\
4.72532532532533	15.6515113916932\\
4.73023023023023	15.7632188709107\\
4.73513513513513	15.875812291491\\
4.74004004004004	15.9892991445554\\
4.74494494494494	16.1036869878638\\
4.74984984984985	16.2189834464325\\
4.75475475475475	16.3351962131576\\
4.75965965965966	16.4523330494439\\
4.76456456456456	16.5704017858413\\
4.76946946946947	16.6894103226852\\
4.77437437437437	16.8093666307453\\
4.77927927927928	16.9302787518787\\
4.78418418418418	17.0521547996907\\
4.78908908908909	17.175002960201\\
4.79399399399399	17.2988314925168\\
4.7988988988989	17.4236487295122\\
4.8038038038038	17.5494630785145\\
4.80870870870871	17.6762830219962\\
4.81361361361361	17.8041171182749\\
4.81851851851852	17.9329740022189\\
4.82342342342342	18.0628623859601\\
4.82832832832833	18.1937910596136\\
4.83323323323323	18.3257688920045\\
4.83813813813814	18.4588048314012\\
4.84304304304304	18.5929079062565\\
4.84794794794795	18.7280872259551\\
4.85285285285285	18.8643519815686\\
4.85775775775776	19.0017114466183\\
4.86266266266266	19.1401749778442\\
4.86756756756757	19.279752015983\\
4.87247247247247	19.4204520865517\\
4.87737737737738	19.5622848006408\\
4.88228228228228	19.7052598557137\\
4.88718718718719	19.8493870364142\\
4.89209209209209	19.9946762153822\\
4.896996996997	20.1411373540769\\
4.9019019019019	20.2887805036083\\
4.90680680680681	20.4376158055761\\
4.91171171171171	20.5876534929175\\
4.91661661661662	20.7389038907627\\
4.92152152152152	20.8913774172992\\
4.92642642642643	21.0450845846435\\
4.93133133133133	21.2000359997224\\
4.93623623623624	21.3562423651624\\
4.94114114114114	21.5137144801871\\
4.94604604604605	21.6724632415245\\
4.95095095095095	21.8324996443221\\
4.95585585585586	21.9938347830713\\
4.96076076076076	22.1564798525407\\
4.96566566566567	22.3204461487189\\
4.97057057057057	22.4857450697655\\
4.97547547547548	22.652388116972\\
4.98038038038038	22.8203868957326\\
4.98528528528529	22.9897531165227\\
4.99019019019019	23.1604985958891\\
4.9950950950951	23.3326352574479\\
5	23.5061751328933\\
};
\addlegendentry{approximate $Z$};

\end{axis}
\end{tikzpicture}%
  \caption{Laplace approximation to $Z = \Gamma(\alpha)$ as a function
    of $\alpha$.}
  \label{problem_5}
\end{figure}

\end{document}
