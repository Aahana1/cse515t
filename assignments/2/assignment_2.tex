\documentclass{article}

\usepackage[T1]{fontenc}
\usepackage[osf]{libertine}
\usepackage[scaled=0.8]{beramono}
\usepackage[margin=1.5in]{geometry}
\usepackage{url}
\usepackage{booktabs}
\usepackage{amsmath}
\usepackage{amssymb}
\usepackage{nicefrac}
\usepackage{microtype}
\usepackage{bm}

\usepackage{sectsty}
\sectionfont{\large}
\subsectionfont{\normalsize}

\usepackage{titlesec}
\titlespacing{\section}{0pt}{10pt plus 2pt minus 2pt}{0pt plus 2pt minus 0pt}
\titlespacing{\subsection}{0pt}{5pt plus 2pt minus 2pt}{0pt plus 2pt minus 0pt}

\setlength{\parindent}{0pt}
\setlength{\parskip}{1ex}

\newcommand{\acro}[1]{\textsc{\MakeLowercase{#1}}}
\newcommand{\given}{\mid}
\newcommand{\mc}[1]{\mathcal{#1}}
\newcommand{\data}{\mc{D}}
\newcommand{\intd}[1]{\,\mathrm{d}{#1}}
\newcommand{\mat}[1]{\bm{\mathrm{#1}}}
\renewcommand{\vec}[1]{\bm{\mathrm{#1}}}
\newcommand{\trans}{^\top}
\newcommand{\inv}{^{-1}}

\DeclareMathOperator{\var}{var}

\begin{document}

{\large \textbf{CSE 515T (Spring 2015) Assignment 2}} \\
Due Monday, 16 February 2015 \\

\begin{enumerate}

\item
  (Curse of dimensionality.)
  Consider a $d$-dimensional, zero-mean, spherical multivariate
  Gaussian distribution:
  \begin{equation*}
    p(\vec{x}) = \mc{N}(\vec{x}; \vec{0}, \mat{I}_d).
  \end{equation*}
  Equivalently, each entry of $\vec{x}$ is drawn iid from a univariate
  standard normal distribution.

  In familiar small dimensions ($d \leq 3$), ``most'' of the vectors
  drawn from a multivariate Gaussian distribution will lie near the
  mean.  For example, the famous 68--95--99.7 rule for $d = 1$
  indicates that large deviations from the mean are unusual.  Here we
  will consider the behavior in larger dimensions.
  \begin{itemize}
  \item Draw 10\,000 samples from $p(\vec{x})$ for each dimension in
    $d \in \{1, 5, 10, 50, 100\},$ and compute the length of each
    vector drawn: $y_d = \sqrt{\vec{x}\trans \vec{x}} = (\sum_i^d
    x_i^2)^{\nicefrac{1}{2}}$.  Estimate the distribution of each
    $y_d$ using either a histogram or a kernel density estimate (in
    \acro{MATLAB}, \texttt{hist} and \texttt{ksdensity},
    respectively).  Plot your estimates.  (Please do not hand in your
    raw samples!)  Summarize the behavior of this distribution as $d$
    increases.
  \item
    The true distribution of $y_d^2$ is a chi-square distribution with
    $d$ degrees of freedom (the distribution of $y_d$ itself is the
    less-commonly seen chi distribution).  Use this fact to compute
    the probability that $y_d < 5$ for each of the dimensions in the
    last part.
  \item
    For $d = 1\,000$, compute the 5th and 95th percentiles of $y_d$.
    Is the mean $\vec{x} = \vec{0}$ a representative summary of the
    distribution in high dimensions?  This behavior has been called
    ``the curse of dimensionality.''
  \end{itemize}

\item
  (Bayesian linear regression.)
  Consider the following data:
  \begin{align*}
    \vec{x}
    &=
    [-2.26, -1.31, -0.43, 0.32, 0.34, 0.54, 0.86, 1.83, 2.77, 3.58]\trans; \\
    \vec{y}
    &=
    [1.03, 0.70, -0.68, -1.36, -1.74, -1.01, 0.24, 1.55, 1.68, 1.53]\trans.
  \end{align*}
  Fix the noise variance at $\sigma^2 = 0.5^2$.
  \begin{itemize}
  \item
    Perform Bayesian linear regression for these data using the
    polynomial basis functions $\phi_k(x) = [1, x, x^2, \dotsc
      x^k]\trans$ for $k \in \{1, 2, 3\}$, in each case using the
    parameter prior $p(\vec{w}) = \mc{N}(\vec{w}; \vec{0}, \mat{I})$.
    Evaluate and plot the posterior means $\mathbb{E}[\vec{y}_\ast
      \given \mat{X}_\ast, \data, \sigma^2]$ on the interval $x_\ast
    \in [-4, 4]$ for each model.  Also plot the posterior mean
    plus-or-minus two times the posterior standard deviation:
    \begin{equation*}
      \mathbb{E}[\vec{y}_\ast \given \mat{X}_\ast, \data, \sigma^2] \pm
      2 \sqrt{\var[\vec{y}_\ast \given \mat{X}_\ast, \data, \sigma^2]}.
    \end{equation*}
    This is a pointwise 95\% credible interval for the regression
    function.  Where is the pointwise uncertainty the largest?
  \item
    Compute the marginal likelihood of the data for each of the basis
    expansions above: $p(\vec{y} \given \mat{X}, k, \sigma^2)$.  Which
    model explains the data the best?
  \end{itemize}

\item
  (Optimal design for Bayesian linear regression.)
  Consider the data from the last problem, and suppose we have
  selected the quadratic model corresponding to $k = 2$ (do not assume
  that this is the answer to the last part of the last question).
  Imagine we are allowed to evaluate the function at a point $x'$ of
  our choosing, giving a new dataset $\data' = \data \cup \bigl\{ (x',
  y') \bigr\}$ and a new posterior for the parameters $p(\vec{w}
  \given \data', \sigma^2) = \mc{N}(\vec{w};
  \vec{\mu}_{\vec{w}\given\data'},
  \mat{\Sigma}_{\vec{w}\given\data'})$.  We hope to select the
  location $x'$ to best improve our current model, under some quality
  measure.

  Assume that we ultimately wish to predict the function at a grid of
  points
  \begin{equation*}
    \vec{x}_\ast = [-4, -3.5, -3, \dotsc, 3.5, 4]\trans.
  \end{equation*}
  We select the squared loss for a set of predictions
  $\hat{\vec{y}}_\ast$:
  \begin{equation*}
    \ell(\vec{y}_\ast, \hat{\vec{y}}_\ast)
    =
    \sum_i \bigl((y_\ast)_i - (\hat{y}_\ast)_i\bigr)^2;
  \end{equation*}
  therefore, we will predict the posterior mean $\hat{\vec{y}}_\ast =
  \mat{X}_\ast \vec{\mu}_{\vec{w}\given\data'}$.
  \begin{itemize}
  \item
    Given a potential observation location $x'$, derive a closed-form
    expression for the expected loss
    $\mathbb{E}\bigl[\ell(\vec{y}_\ast, \hat{\vec{y}}_\ast) \given x',
      \data \bigr]$.  Note: this does not require integration over
    $y'$!  (What is the expected squared deviation from the mean?)
  \item
    Plot the expected loss over the interval $x' \in [-4, 4]$.  Where
    is the optimal location to sample the function?
  \end{itemize}

\item
  (Woodbury matrix identity.)
  The \emph{Woodbury matrix identity} is a very useful result.  Let
  $\mat{A}$ be an $(n \times n)$ matrix, let $\mat{U}$ and $\mat{V}$
  be $(n \times k)$ matrices, and let $\mat{C}$ be a $(k \times k)$
  matrix.  Then:
  \begin{equation*}
    (\mat{A} + \mat{U}\mat{C}\mat{V}\trans)\inv
    =
    \mat{A}\inv
    -
    \mat{A}\inv
    \mat{U}
    (\mat{C} + \mat{V}\trans \mat{A}\inv \mat{U})\inv
    \mat{V}\trans
    \mat{A}\inv.
  \end{equation*}
  This result is useful when you already have the inverse of a matrix
  $\mat{A}$ and want to know the inverse after a rank-$k$ adjustment.
  When $k \ll n$, the Woodbry matrix identity can be considerably
  faster than direct inversion!
  \begin{itemize}
  \item
    Prove this result.
  \item
    Use this result to rewrite the posterior covariance of the weight
    vector $\vec{w}$ in Bayesian linear regression (as written in the
    notes to lecture 5) in a simpler form.
  \end{itemize}

\item
  (Laplace approximation.)
  Find a Laplace approximation to the Gamma distribution:
  \begin{equation*}
    p(\theta \given \alpha, \beta)
    =
    \frac{1}{Z}
    \theta^{\alpha - 1}
    \exp(-\beta\theta).
  \end{equation*}
  The true value of the normalizing constant is
  \begin{equation*}
    Z = \frac{\Gamma(\alpha)}{\beta^\alpha}.
  \end{equation*}
  If we fix $\beta = 1$, then $Z = \Gamma(\alpha)$, so we may use the
  Laplace approximation to estimate the Gamma function.  Analyze the
  quality of this approximation as a function of $\alpha$.

\end{enumerate}

\end{document}
