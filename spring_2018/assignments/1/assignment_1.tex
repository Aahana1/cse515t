\documentclass{article}

\usepackage[T1]{fontenc}
\usepackage[osf]{libertine}
\usepackage[scaled=0.8]{beramono}
\usepackage[margin=1.5in]{geometry}
\usepackage{url}
\usepackage{booktabs}
\usepackage{amsmath}
\usepackage{nicefrac}
\usepackage{microtype}
\usepackage{bm}

\usepackage{sectsty}
\sectionfont{\large}
\subsectionfont{\normalsize}

\usepackage{titlesec}
\titlespacing{\section}{0pt}{10pt plus 2pt minus 2pt}{0pt plus 2pt minus 0pt}
\titlespacing{\subsection}{0pt}{5pt plus 2pt minus 2pt}{0pt plus 2pt minus 0pt}

\setlength{\parindent}{0pt}
\setlength{\parskip}{1ex}

\newcommand{\acro}[1]{\textsc{\MakeLowercase{#1}}}
\newcommand{\given}{\mid}
\newcommand{\mc}[1]{\mathcal{#1}}
\newcommand{\data}{\mc{D}}
\newcommand{\intd}[1]{\,\mathrm{d}{#1}}

\begin{document}

{\large \textbf{CSE 515T (Spring 2018) Assignment 1}} \\
Due Wednesday, 7 February 2018 \\

\begin{enumerate}

\item
  Your doctor supposes that you may have a rare disease, Bayesian
  syndrome (also known as \acro{BS}), which occurs in every one out of
  100 members of the population. The doctor proposes you take a
  super-accurate (and expensive!) test to be sure.  The scanner is
  fairly reliable; 95\% of all Bayesians are identified as Bayesians,
  and 95\% of non-Bayesians are identified as such.

  The test comes back positive and the doctor tells you it's nearly
  certain you're a Bayesian given that result, due to the accuracy of
  the test. How do you respond?

\item
  Suppose $k$ has a Poisson distribution with unknown rate parameter
  $\theta$
  \[
    \Pr(k \given \theta) = \frac{\lambda^k e^{-\lambda}}{k!},
    \qquad
    k = 0, 1, 2, \dotsc
  \]

  Let the prior for $\theta$ be a gamma distribution:
  \[
    p(\theta \given \alpha, \beta)
    =
    \frac{\beta^\alpha}
         {\Gamma(\alpha)}
    \theta^{\alpha - 1}e^{-\beta\theta},
    \qquad \theta > 0
  \]
  where $\Gamma$ is the gamma function.  Show that, given an
  observation $k$, the posterior $p(\theta \given k, \alpha, \beta)$
  is a gamma distribution with updated parameters $(\alpha', \beta')$.

  Hints:
  \begin{itemize}
  \item For nonnegative integer $k$, $k! = \Gamma(k + 1)$.
  \item $\Gamma(x + 1) = x\Gamma(x)$ for all $x > 0$.
  \item For common distributions (such as the gamma distribution),
    Wikipedia has useful properties in an infobox.
  \item Beware there are two common parameterizations of the gamma distribution.
  \end{itemize}

\item
  Suppose that in the last question, we received a sample of $n$
  observations $\{k_1, k_2, \dotsc, k_n\}$. What is the posterior
  $p(\theta \given k_1, k_2, \dotsc, k_n, \alpha, \beta)$? What is the
  posterior mean? The posterior mode?

  In light of this and the previous question, can you give an
  interpretation of the prior parameters $\alpha$ and $\beta$?
  What happens in the limit as $n \to \infty$?

\item
  (Scenario quoted from Morey, et al.)  A 10-meter-long research
  submersible with several people on board has lost contact with its
  surface support vessel. The submersible has a rescue hatch exactly
  halfway along its length, to which the support vessel will drop a
  rescue line. Because the rescuers only get one rescue attempt, it is
  crucial that when the line is dropped to the craft in the deep water
  that the line be as close as possible to this hatch. The researchers
  on the support vessel do not know where the submersible is, but they
  do know that it forms distinctive bubbles. These bubbles could form
  anywhere along the craft's length, independently, with equal
  probability, and float to the surface where they can be seen by the
  support vessel.

  We wish to perform inference about the location of the rescue
  hatch given observed bubbles; call this location $\theta$.

  A common ``trick'' when wishing to express absolute prior ignorance
  of a parameter is to use a so-called \emph{uninformative} prior. In
  this case, we will consider the uninformative ``prior'' $p(\theta) =
  1$. This prior does not normalize, but we will see that it does not
  lead to major problems.

  (a) Suppose the researchers observe the locations of exactly two
  bubbles, $x_1$ and $x_2$. Write down an appropriate likelihood for
  these data given $\theta$ and derive the posterior distribution for
  the location of the hatch, $p(\theta \given x_1, x_2)$, using the
  uninformative prior described above.

  (b) Now find a 50\% Bayesian credible interval for $\theta$ given
  $(x_1, x_2)$. Plot the width of this interval as a function of
  $\lvert x_1 - x_2 \rvert$. Is this the relationship you would
  expect?

\item

  We are going to consider a Bayesian hypothesis test that a coin is
  exactly fair. Let $\theta$ be the ``heads probability'' of a coin
  \[
    \Pr(\text{H}) = \theta.
  \]
  Let us place a somewhat atypical prior on $\theta$ that includes
  nontrivial prior probability that the coin is exactly fair (that is,
  $\theta = \nicefrac{1}{2}$):
  \[
    \Pr(\theta = \nicefrac{1}{2}) = 0.9
    \qquad
    p(\theta \given \theta \neq \nicefrac{1}{2}) = \mc{U}[0, 1].
  \]
  This can be interpreted as a prior that is a mixture between a uniform
  prior on the interval $[0, 1]$ with weight 0.1 (this is also a special
  of the beta distribution with $\alpha = \beta = 1$) and a ``point
  mass'' (Dirac delta function) at $\theta = \nicefrac{1}{2}$ with
  weight 0.9.

  Suppose we flip a coin $n = 10$ times and observe $x = 8$ ``heads.''
  What is the posterior distribution $p(\theta \given x, n)$?  What is
  the posterior probability that the coin is exactly fair?

  Note there is no probability density function corresponding to the
  prior. It will help to work in cases.

\end{enumerate}

\end{document}
