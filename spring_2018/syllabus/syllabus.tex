\documentclass{article}

\usepackage[T1]{fontenc}
\usepackage[osf]{libertine}
\usepackage[scaled=0.8]{beramono}
\usepackage[margin=1.5in]{geometry}
\usepackage{url}
\usepackage{booktabs}
\usepackage{microtype}

\usepackage{sectsty}
\sectionfont{\large}
\subsectionfont{\normalsize}

\usepackage{titlesec}
\titlespacing{\section}{0pt}{10pt plus 2pt minus 2pt}{0pt plus 2pt minus 0pt}
\titlespacing{\subsection}{0pt}{5pt plus 2pt minus 2pt}{0pt plus 2pt minus 0pt}

\setlength{\parindent}{0pt}
\setlength{\parskip}{1ex}

\newcommand{\acro}[1]{\textsc{\MakeLowercase{#1}}}

\begin{document}

{\large \textbf{CSE 515T: Bayesian Methods in Machine Learning (Spring 2018)}} \\[1ex]

\begin{tabular}{rl}
               Instructor & Professor Roman Garnett                                 \\
%               \acro{TA} & Gustavo Malkomes                                        \\
            Time/Location & Monday/Wednesday 4--5:30pm, Cupples I 215               \\
   Office Hours (Garnett) & Wednesday 5:30--6:30pm, Jolley 504                      \\
% Office Hours (Malkomes) & Friday 3--4pm, Jolley 517                               \\
               \acro{URL} & \url{http://cse.wustl.edu/~garnett/cse515t/}            \\
                   GitHub & \url{https://github.com/rmgarnett/cse515t/spring_2018}  \\
     Piazza message board & \url{https://piazza.com/wustl/spring2018/cse515t/home/}
\end{tabular}

\section*{Course Description}

This course will cover modern machine learning techniques from a Bayesian
probabilistic perspective. Bayesian probability allows us to model and reason
about all types of uncertainty. The result is a powerful, consistent framework
for approaching many problems that arise in machine learning, including
parameter estimation, model comparison, and decision making. We will begin with
a high-level introduction to Bayesian inference, then proceed to cover
more-advanced topics.

\section*{Prerequisites}

We will make heavy use of mathematics in this course.  You should have a good
grasp of multivariable calculus (integration, partial derivation, maximization,
etc.), probability (conditional probability, expectations, etc.), and linear
algebra (solving linear systems, eigendecompositions, etc.).

Please note that this is not an introduction to machine learning; the
\acro{CSE 417T/517A} courses fill that role.  I will assume prior
familiarity with the main concepts of machine learning: supervised and
unsupervised learning, classification, regression, clustering, etc.

\section*{Book}

There is no required book. For each lecture, I will provide a list of related
materials, including book chapters, videos, papers, code, etc.\ on the course
webpage.  These are to give you different viewpoints on the subject.  Hopefully
you can find one that suits you.

Although no book will be required, the following books are highly
aligned with this course:
\begin{itemize}
\item \emph{Pattern Recognition and Machine Learning} by Christopher M.\ Bishop.
  Covers many machine-learning topics thoroughly.  Very Bayesian.  Can also be
  very mathematical and take some effort to read.
\item \emph{Bayesian Reasoning and Machine Learning} by David Barber.  Geared
  (as much as a machine-learning book could be) towards computer scientists.
  Lots of material on graphical models.  Freely available
  online.\footnote{\url{http://www.cs.ucl.ac.uk/staff/d.barber/brml/}, link also
    on course webpage.}
\item \emph{Gaussian Processes for Machine Learning} by Carl Rasmussen and
  Christopher Williams.  Excellent reference for Gaussian processes.  Freely
  available online.\footnote{\url{http://www.gaussianprocess.org/gpml/}, link
    also on course webpage.}
\end{itemize}

\clearpage
The following books are good resources for Bayesian statistics:
\begin{itemize}
\item \emph{Statistical Decision Theory and Bayesian Analysis} by James Berger.
  An old book (1980, advertises ``with 23 illustrations'' on the title page),
  but nonetheless an excellent introduction to Bayesian methods.  Very clear.
  Provides convincing philosophical arguments for the Bayesian viewpoint.
\item \emph{The Bayesian Choice: From Decision-Theoretic Foundations to
  Computational Implementation} by Christian Robert.  Another fairly technical
  resource with passionate arguments for the Bayesian perspective.
\end{itemize}

\section*{Assignments}

There will be a small number of assignments throughout the semester,
with two weeks available to complete each one.

The assignments will form 30\% of your grade, and each will have two types of
questions: traditional ``pencil-and-paper'' questions, and programming exercises
meant to give more insight into applying the techniques we will discuss on
actual data.  The former \emph{will not be corrected.}  If you make a reasonable
attempt to answer a question, I will give you full credit.  After each
assignment, I will provide solutions online.

The programming exercises will require you to implement some of the theoretical
ideas we discuss in class.  The point of these exercises is both to lead to a
better understanding by forcing a different viewpoint (that of the designer),
and also to enable interaction.  I encourage you to play with the data,
parameters, etc. associated with these exercises to see how the results change.
The point of the exercises is \emph{not} for me to judge your programming
skills, so \emph{please do not hand in your code.}  Rather, you should convey
your answers via plots, tables, and/or discussion, as appropriate.  As I don't
need to read your code, feel free to use any language you'd like, but note that
if I provide you with my own code, I will do so in \acro{MATLAB.}

\subsection*{Late policy}

Assignments will be due during class on the dates specified on the course
homepage.  I will allow you to turn in your assignment up to one class late with
no penalty.

\subsection*{Collaboration policy}

Please feel free to collaborate on the paper-and-pencil questions!  This is a
good way to gain a deeper understanding of the material.  Of course, you will be
expected to write up your answers separately.  Also feel free to collaborate on
a high level on the programming exercises, but please write your own code and
produce your own results.

\section*{Midterm}

There will be a take-home midterm on a date to be determined later (probably
just before or just after Spring Break).This will count for 30\% of your grade.

\section*{Project}

In the second half of the semester, you will complete a project, which will
comprise 30\% of your final grade. The goal of the project will be to apply
Bayesian techniques to a real dataset in a nontrivial way.  I will compile a
list of datasets on the course webpage, but you should of course feel free to
find your own that is aligned with your interests.  The project should reach
beyond the scope of the homework problems.  I will judge the success of a
project based on the methodological approach rather than the quantitative
details of the final outcome.  This is an exercise in applying theoretical ideas
in practice, and even the most carefully constructed models or techniques can
fail on a particular problem.  Note that I would expect you to think about
\emph{why} your method might have failed (or succeeded!).

You can complete this project in groups of one, two, or three people.  Of
course, I will expect more out of larger groups.

There will be four components to this project:
\begin{itemize}
\item A project proposal, due \textbf{Monday, 5 March.}  This should be an
  approximately one page document describing your idea.  I will read this and
  give feedback/suggestions.
\item A status report, due \textbf{Monday, 2 April.}  I expect this to be one
  or two pages, updating me on the progress of your project, including data
  processing, implementation, experimental design decisions, etc.
\item A 15-minute presentation describing the project.  These will be held in
  class during the last class sessions, beginning on \textbf{Monday, 16 April}.
  The presentation should briefly explain the idea, the data, and the results of
  your investigation.
\item A final report, due \textbf{Friday, 27 April.}  This should be an
  approximately four-page document explaining the idea, experimental setup,
  results, and your interpretation of them.
\end{itemize}

\section*{Grading}

Your final grade will consist of the following weighted components:
\begin{center}
  \begin{tabular}{lc}
    \toprule
    component                    &   \% \\
    \midrule
    assignments                  & 30\% \\
    midterm                      & 30\% \\
    project proposal             & 10\% \\
    project status report        & 10\% \\
    project presentation         & 10\% \\
    project final report         & 10\% \\
    \midrule
    final project total          & 40\% \\
    \bottomrule
  \end{tabular}
\end{center}

\section*{Topics}

An outline of the topics I expect to cover is below; this is subject to change,
more likely by deletion than addition.  If there is a particular topic you would
like me to spend more time on (or don't care about at all!), please let me know.

I will keep the course webpage updated with lecture-specific information and
resources.

\begin{itemize}
\item \textbf{Introduction to the Bayesian method:} review of probability,
  Bayes' theorem, Bayesian inference, Bayesian parameter estimation, Bayesian
  decision theory, Bayesian model selection.
\item \textbf{Approximate inference:} the Laplace approximation, variational
  Bayes, expectation propagation.
\item \textbf{Sampling methods:} rejection sampling, importance sampling, Markov
  chain Monte Carlo.
\item \textbf{Parametric models:} Bayesian linear regression, logistic
  regression, general linear models, basis expansions, mixture models, latent
  Dirichlet allocation.
\item \textbf{Nonparametric models:} Gaussian proesses for regression and
  classification.
\item \textbf{Bayesian numerical analysis:} Bayesian optimization, Bayesian
  quadrature.
\end{itemize}

\end{document}
