\documentclass{article}

\usepackage[T1]{fontenc}
\usepackage[osf]{libertine}
\usepackage[scaled=0.8]{beramono}
\usepackage[margin=1.5in]{geometry}
\usepackage{url}
\usepackage{booktabs}
\usepackage{amsmath}
\usepackage{amsfonts}
\usepackage{nicefrac}
\usepackage{microtype}
\usepackage{bm}

\usepackage{sectsty}
\sectionfont{\large}
\subsectionfont{\normalsize}

\usepackage{titlesec}
\titlespacing{\section}{0pt}{10pt plus 2pt minus 2pt}{0pt plus 2pt minus 0pt}
\titlespacing{\subsection}{0pt}{5pt plus 2pt minus 2pt}{0pt plus 2pt minus 0pt}

\setlength{\parindent}{0pt}
\setlength{\parskip}{1ex}

\newcommand{\acro}[1]{\textsc{\MakeLowercase{#1}}}
\newcommand{\given}{\mid}
\newcommand{\mc}[1]{\mathcal{#1}}
\newcommand{\data}{\mc{D}}
\newcommand{\model}{\mc{M}}
\newcommand{\R}{\mathbb{R}}
\newcommand{\trans}{^\top}
\newcommand{\intd}[1]{\,\mathrm{d}{#1}}
\newcommand{\mat}[1]{\bm{\mathrm{#1}}}
\renewcommand{\vec}[1]{\bm{\mathrm{#1}}}

\DeclareMathOperator{\var}{var}

\begin{document}

{\large \textbf{CSE 515T (Spring 2019) Midterm}}
\begin{itemize}
\item
  There are two ways to hand in this midterm.  \textbf{Late submissions will not
    be accepted!}  I do not recommend cutting it too close.
  \begin{itemize}
  \item Physically in class.  The due date for this option is \textbf{5:30
    \acro{PM}, Wednesday, 24 April (the last day of class).}
  \item Electronically on Piazza as a private message to the instructors. The
    due date for this option is \textbf{23:59 \acro{CDT} Monday, 29 April} (the
    midnight of the Monday at the start of reading week).
  \end{itemize}
\item
  Please do not discuss the questions with other members of the class.
\item
  Please post any questions as a \emph{private message to the instructors} on
  Piazza.
\item
  Any corrections will be posted by the instructors on Piazza. This document
  will also be kept up-to-date on the course webpage and in GitHub.
\end{itemize}

\clearpage

In this question we will consider a discrete parameter $\theta \in \{1, 2\}$
with a uniform prior distribution: $\Pr(\theta = 1) = \Pr(\theta = 2) =
\nicefrac{1}{2}$.

A series of observations $\{x_1, x_2, \dotsc\} \in \R$ will be generated
\emph{independently and identically distributed} (iid) according to the
following distributions conditional on the unknown parameter $\theta$:
\[
  p(x \given \theta = 1) = \mc{N}(x; 0, 1^2);
  \qquad
  p(x \given \theta = 2) = \mc{N}(x; 1, 2^2).
\]

\begin{enumerate}
\item
  \begin{enumerate}
  \item
    Plot the probability density function of the marginal prior predictive
    distribution for the first value $p(x_1)$ over the interval $x_1 \in [-5,
      7]$, and report the value of this pdf at $x_1 = 0$ and $x_1 = 1$ to three
    decimal places.
  \item
    Assume we make an observation $x_1 = 0$. What is the posterior $\Pr(\theta
    \given x_1)$?
  \item
    What is the posterior mean and variance of $p(x_2 \given x_1 = 0)$?
  \item
    Are there any possible value(s) for $x_1$ that would not change our prior
    belief about $\theta$, so that $\Pr(\theta \given x_1) = \Pr(\theta)$? If
    so, what is it/are they?
  \item
    What would be the observation $x_1$ that would maximize the posterior
    probability $\Pr(\theta = 1 \given x_1)$? Report its value to three decimal
    places.
  \end{enumerate}
\end{enumerate}

\clearpage

Consider the following dataset $\data = (\vec{x}, \vec{y})$:
\begin{align*}
  \vec{x} &= [-1.30, 0.293, -0.974, -1.44, -1.12, -0.151, 1.08, -0.292, 2.36, -0.04]; \\
  \vec{y} &= [-1.68, 0.563, -0.954, -0.924, -0.905, -0.833, 0.952, -0.124, 0.838, 0.198].
\end{align*}
We are going to perform regression with these data assuming the observation
model $y = f(x) + \varepsilon$, where $f\colon \R \to \R$ is a latent function
we wish to estimate. Assume that the noise $\varepsilon$ is generated iid with
distribution $p(\varepsilon) = \mc{N}(\varepsilon; 0, \nicefrac{1}{3}^2)$.

Consider two Gaussian process models for $f$ with zero prior mean.  Model
$\model_1$ has a linear covariance:
\[
  p(f \given \model_1) = \mc{GP}(f; 0, K_1);
  \qquad
  K_1(x, x') = x\trans x'.
\]
Model $\model_2$ has a squared exponential covariance with unit length scale and
output scale:
\[
  p(f \given \model_2) = \mc{GP}(f; 0, K_2);
  \qquad
  K_2(x, x') = \exp\bigl(-\tfrac{1}{2}(x - x')^2\bigr).
\]

\begin{enumerate}\setcounter{enumi}{1}
\item
  \begin{enumerate}
  \item What is the log model evidence for each model, $\log p(\vec{y} \given \vec{x}, \model_i)$?
  \item What is the model posterior, $\Pr(\model \given \data)$? Report each
    value to three decimal places.
  \item
    Can you find a kernel with higher model evidence given the data above,
    keeping the mean function and noise variance fixed?  (I will award an extra
    credit point to the person who provides the kernel with the highest
    evidence.)
  \end{enumerate}
\end{enumerate}

In the following assume the model posterior is actually uniform $\Pr(\model_1
\given \data) = \Pr(\model_2 \given \data) = \nicefrac{1}{2}$, even though it is
not.
\begin{enumerate}\setcounter{enumi}{2}
\item
  \begin{enumerate}
  \item Plot the pdf of the predictive distribution at $x = 4$, $p\bigl(f(x)
    \given, x = 4, \data\bigr)$, over the range $-4 \leq f(x) \leq 4$.
  \item
    Plot the model-marginal mean function $\mathbb{E}[f \given \data]$ over the
    range $-4 \leq x \leq 4$.
  \end{enumerate}
\end{enumerate}
We will adopt the second model $\model_2$ \emph{only} for the next
question. That is, the prior on $f$ is only
\[
  p(f) = \mc{GP}(f; 0, K);
  \qquad
  K(x, x') = \exp\bigl(-\tfrac{1}{2}(x - x')^2\bigr).
\]
We will consider a decision problem where we can augment our dataset $\data$
with a single random measurement at a point $x^\ast$ of our choosing, receiving
a measurement $y^\ast = f(x^\ast) + \varepsilon$.

Define a loss function $\ell(x^\ast, y^\ast, \data)$ by the posterior predictive
standard deviation of $f(0)$:
\[
  \ell(x^\ast, y^\ast, \data) =
  \sqrt{\var\bigl[f(x) \given x = 0, \data, x^\ast, y^\ast\bigr]}.
\]

\begin{enumerate}\setcounter{enumi}{3}
\item
  \begin{enumerate}
  \item
    Plot the expected loss $\mathbb{E}[\ell \given x^\ast, \data]$ over the
    range $-4 \leq x^\ast \leq 4$. Report its value at $x^\ast = 1$ to three
    decimal places.
  \item
    What is the optimal final measurement location? Report its value to three
    decimal places.
  \end{enumerate}
\end{enumerate}

\clearpage

Let $f\colon \R \to \R$ have an arbitrary Gaussian process prior:
\[
  p(f) = \mc{GP}(f; \mu, K),
\]
where $K$ is differentiable everywhere.

Let $x \in \R$ be an arbitrary point. We will consider inference of the
derivative of $f$ at $x$, $f'(x)$.

Given $h > 0$, consider the value
\[
  a_h = \frac{f(x + h) - f(x)}{h}.
\]
\begin{enumerate}\setcounter{enumi}{3}
\item
  \begin{enumerate}
  \item
    What is the variance of $a_h$? Take the limit as $h \to 0$ and interpret the
    result.
  \item
    Let $x' \in \R$ be another arbitrary point. What is the covariance between
    $a_h$ and $f(x')$? Take the limit as $h \to 0$ and interpret the result.
  \item
    Using the above, find the joint distribution between $f'(x)$ and $f(x')$.
  \end{enumerate}
\end{enumerate}

\clearpage

Let $\theta \in \R^d$ be a $d$-dimensional random variable, and let $p(\theta)$
be an arbitrary prior distribution.

Suppose we discover information $\data$ that informs us every entry of $\theta$
is actually positive, without providing any additional information. That is,
$\theta_i \geq 0; \forall i$.
\begin{enumerate}\setcounter{enumi}{4}
\item
  Describe a rejection sampling procedure for sampling from the posterior
  distribution $p(\theta \given \data)$.
\end{enumerate}

\end{document}
