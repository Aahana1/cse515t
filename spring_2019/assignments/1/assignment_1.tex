\documentclass{article}

\usepackage[T1]{fontenc}
\usepackage[osf]{libertine}
\usepackage[scaled=0.8]{beramono}
\usepackage[margin=1.5in]{geometry}
\usepackage{url}
\usepackage{booktabs}
\usepackage{amsmath}
\usepackage{amsfonts}
\usepackage{nicefrac}
\usepackage{microtype}
\usepackage{bm}

\usepackage{sectsty}
\sectionfont{\large}
\subsectionfont{\normalsize}

\usepackage{titlesec}
\titlespacing{\section}{0pt}{10pt plus 2pt minus 2pt}{0pt plus 2pt minus 0pt}
\titlespacing{\subsection}{0pt}{5pt plus 2pt minus 2pt}{0pt plus 2pt minus 0pt}

\setlength{\parindent}{0pt}
\setlength{\parskip}{1ex}

\newcommand{\acro}[1]{\textsc{\MakeLowercase{#1}}}
\newcommand{\given}{\mid}
\newcommand{\mc}[1]{\mathcal{#1}}
\newcommand{\data}{\mc{D}}
\newcommand{\intd}[1]{\,\mathrm{d}{#1}}

\begin{document}

{\large \textbf{CSE 515T (Spring 2019) Assignment 1}} \\
Due Wednesday, 6 February 2019 \\

\begin{enumerate}

\item
  (Barber, originally adopted from David Spiegelhalter)
  A secret government agency has developed a scanner which determines
  whether a person is a terrorist. The scanner is fairly reliable;
  95\% of all scanned terrorists are identified as terrorists, and
  95\% of all upstanding citizens are identified as such. An informant
  tells the agency that exactly one passenger of 100 aboard an
  aeroplane in which you are seated is a terrorist. The police haul
  off the plane the first person for which the scanner tests
  positive. What is the probability that this person is a terrorist?

  Additionally, if the police were to scan all passengers, how many
  positive detections should we expect?

\item
  Suppose $k$ has a geometric distribution with unknown success
  probability $\theta$
  \begin{equation*}
    \Pr(k \given \theta) = (1 - \theta)^{k - 1} \theta,
    \qquad
    k = 1, 2, \dotsc
  \end{equation*}
  The geometric distribution is appropriate for modeling the number of
  independent Bernoilli trials required, each with success probability
  $\theta$, before observing the first ``success.''

  Let the prior for $\theta$ be a beta distribution:
  \begin{equation*}
    p(\theta \given \alpha, \beta)
    =
    \frac{\theta^{\alpha - 1} (1 - \theta)^{\beta - 1}}
         {B(\alpha, \beta)}
    \qquad 0 < \theta < 1
  \end{equation*}
  where $B$ is the beta function.  Show that, given an observation
  $k$, the posterior $p(\theta \given k, \alpha, \beta)$ is a beta
  distribution with updated parameters $(\alpha', \beta')$.

  Hint: it may help to look up properties of the Beta function, e.g., on
  Wikipedia.

\item
  Suppose that in the last question, we received a sample of $n$
  observations $\{k_1, k_2, \dotsc, k_n\}$. What is the posterior
  $p(\theta \given k_1, k_2, \dotsc, k_n, \alpha, \beta)$? What is the
  posterior mean? The posterior mode?

  In light of this and the previous question, can you give an
  interpretation of the prior parameters $\alpha$ and $\beta$?
  What happens in the limit as $n \to \infty$?

\item
  (Scenario quoted from Morey, et al.)  A 10-meter-long research
  submersible with several people on board has lost contact with its
  surface support vessel. The submersible has a rescue hatch exactly
  halfway along its length, to which the support vessel will drop a
  rescue line. Because the rescuers only get one rescue attempt, it is
  crucial that when the line is dropped to the craft in the deep water
  that the line be as close as possible to this hatch. The researchers
  on the support vessel do not know where the submersible is, but they
  do know that it forms distinctive bubbles. These bubbles could form
  anywhere along the craft's length, independently, with equal
  probability, and float to the surface where they can be seen by the
  support vessel.

  We wish to perform inference about the location of the rescue
  hatch given observed bubbles; call this location $\theta$.

  A common ``trick'' when wishing to express absolute prior ignorance
  of a parameter is to use a so-called \emph{uninformative} prior. In
  this case, we will consider the uninformative ``prior'' $p(\theta) =
  1$. This prior does not normalize, but we will see that it does not
  lead to major problems.

  (a) Suppose the researchers observe the locations of exactly two
  bubbles, $x_1$ and $x_2$. Write down an appropriate likelihood for
  these data given $\theta$ and derive the posterior distribution for
  the location of the hatch, $p(\theta \given x_1, x_2)$, using the
  uninformative prior described above.

  (b) Now find a 50\% Bayesian credible interval for $\theta$ given
  $(x_1, x_2)$. Plot the width of this interval as a function of
  $\lvert x_1 - x_2 \rvert$. Is this the relationship you would
  expect?

\item
  (Maximum-likelihood estimation.)  Suppose you flip a coin with
  unknown bias $\theta$; $\Pr(x = \text{H} \given \theta) = \theta$,
  three times and observe the outcome HHH.  What is the maximum
  likelihood estimator for $\theta$?  Do you think this is a good
  estimator?  Would you want to use it to make predictions?

  Consider a Bayesian analysis of $\theta$ with a beta prior $p(\theta
  \given \alpha, \beta) = \mc{B}(\theta; \alpha, \beta)$.  What is the
  posterior mean of $\theta$?  What is the posterior mode?  Consider
  $(\alpha, \beta) = (50, 50)$.  Plot the posterior density in this
  case.  Is the maximum likelihood estimator a good summary of the
  distribution?

\item
  (Effect of weird priors.)  Let us consider the following set of
  observations. We flip a coin independently $n = 1\,000$ times and
  observe $x = 900$ successes. Call the unknown bias of the coin
  $\theta \in (0, 1)$.

  For each of the prior distributions $p(\theta)$ below, please:
  \begin{itemize}
  \item
    plot the prior distribution $p(\theta)$ over the range $0 < \theta < 1$
  \item
    plot the posterior distribution given the above data, $p(\theta
    \given \data)$, over the range $0 < \theta < 1$
  \item
    report the posterior mean,
    $\mathbb{E}[\theta \given \data] = \int \theta\,p(\theta\given\data)\intd\theta.$
  \end{itemize}

  (a) A uniform prior on $\theta$, which can be realized by selecting
  the beta distribution with $\alpha = \beta = 1$:
  \[
    p(\theta) = \mathcal{B}(\theta; \alpha = 1, \beta = 1).
  \]

  (b) A prior with extreme bias toward small values of $\theta$:
  \[
    p(\theta) = \mathcal{B}(\theta; \alpha = 1, \beta = 100).
  \]

  (c) A prior that has no support on values greater than $\theta =
  \nicefrac{1}{2}$:
  \[
    p(\theta) = \begin{cases} 2 & \theta <    \nicefrac{1}{2}; \\
                              0 & \theta \geq \nicefrac{1}{2}. \end{cases}
  \]

\item
  (Optimal \emph{Price is Right} bidding.)
  Suppose you have a standard normal belief about an unknown parameter
  $\theta$, $p(\theta) = \mc{N}(\theta; 0, 1^2)$.  You are asked to
  give a point estimate $\hat{\theta}$ of $\theta$, but are told that
  there is a heavy penalty for guessing too high.  The loss function is
  \begin{equation*}
    \ell(\hat{\theta}, \theta; c)
    =
    \begin{cases}
      (\theta - \hat{\theta})^2 & \hat{\theta}  <   \theta; \\
      c                         & \hat{\theta} \geq \theta
    \end{cases},
  \end{equation*}
  where $c > 0$ is a constant cost for overestimating.  What is the
  Bayesian estimator in this case?  How does it change as a function
  of $c$? Plot the optimal action as a function of $c$ for $0 < c <
  10$. Hint: you may need to minimize certain expressions you
  encounter numerically as an analytic solution may not be available.

\end{enumerate}

\end{document}
