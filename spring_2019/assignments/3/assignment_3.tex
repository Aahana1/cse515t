\documentclass{article}

\usepackage[T1]{fontenc}
\usepackage[osf]{libertine}
\usepackage[scaled=0.8]{beramono}
\usepackage[margin=1.5in]{geometry}
\usepackage{url}
\usepackage{booktabs}
\usepackage{amsmath}
\usepackage{amsfonts}
\usepackage{nicefrac}
\usepackage{microtype}
\usepackage{bm}

\usepackage{sectsty}
\sectionfont{\large}
\subsectionfont{\normalsize}

\usepackage{titlesec}
\titlespacing{\section}{0pt}{10pt plus 2pt minus 2pt}{0pt plus 2pt minus 0pt}
\titlespacing{\subsection}{0pt}{5pt plus 2pt minus 2pt}{0pt plus 2pt minus 0pt}

\setlength{\parindent}{0pt}
\setlength{\parskip}{1ex}

\newcommand{\acro}[1]{\textsc{\MakeLowercase{#1}}}
\newcommand{\given}{\mid}
\newcommand{\mc}[1]{\mathcal{#1}}
\newcommand{\data}{\mc{D}}
\newcommand{\model}{\mc{M}}
\newcommand{\trans}{^\top}
\newcommand{\intd}[1]{\,\mathrm{d}{#1}}
\newcommand{\mat}[1]{\bm{\mathrm{#1}}}
\renewcommand{\vec}[1]{\bm{\mathrm{#1}}}

\begin{document}

{\large \textbf{CSE 515T (Spring 2019) Assignment 3}} \\
Due Wednesday, 10 April 2019 \\

\begin{enumerate}
\item
  (Sampling from a multivariate Gaussian distribution).  Assume you can sample
  from a univariate standard normal distribution $\mc{N}(0, 1^2)$.  In the last
  assignment, this allowed you to sample trivially from a multivariate standard
  normal distribution $\mc{N}(\vec{0}, \mat{I})$.  Use the closure of the
  Gaussian distribution to affine transformations to derive a procedure to
  sample from an arbitrary multivariate Gaussian distribution
  \[
    p(\vec{x}) = \mc{N}(\vec{x}; \vec{\mu}, \mat{\Sigma})
  \]
  using an oracle that can draw a sample from the univariate standard normal
  distribution.

  Hint: the \emph{Cholesky decomposition} may be helpful.
\end{enumerate}

\clearpage
Now we will consider a series of questions relating to an application of
Bayesian inference to numerical analysis, specifically quadrature.

We are going to consider the function
\[f(x) = \exp(-x^2)\]
and its definite integral
\[Z = \int_{-\infty}^\infty f(x) \intd x.\]
The function $f$ has no elementary antiderivative, so the calculation of $Z$ is
not straightforward. There is a famous method for computing $Z$ with the trick
of considering $Z^2$ instead, rewriting the resulting 2$d$ integral in polar
coordinates, and making a convenient substitution.\footnote{This is commonly
  credited to Gauss, but the idea goes back at least to Poisson.} If you haven't
seen this, it's beautiful and worth checking out. The result is
\[Z = \sqrt{\pi}.\]

We will consider modeling $f$ with a Gaussian process prior distribution:
\[p(f) = \mc{GP}(f; \mu, K),\]
and conditioning on the following set of data $\data = (\vec{x}, \vec{y})$:
\begin{align*}
  \vec{x} &= [-2.5, -1.5, -0.5, 0.5, 1.5, 2.5]\trans; \\
  \vec{y} &= \exp(-\vec{x}^2) \\
          &= [0.0019305, 0.1054, 0.7788, 0.7788, 0.1054, 0.0019305]\trans.
\end{align*}
We will fix the prior mean function $\mu$ to be identically zero; $\mu(x) = 0$.

\begin{enumerate}
\setcounter{enumi}{1}
\item
  First, let us consider the question of model, specifically kernel, selection.

  Consider the following three choices for the covariance function $K$:
  \begin{align*}
    K_1(x, x') &= \exp\bigl(-\lvert x - x' \rvert^2\bigr) \\
    K_2(x, x') &= \exp\bigl(-\lvert x - x' \rvert\bigr) \\
    K_3(x, x') &= \bigl(1 + \sqrt{3} \lvert x - x' \rvert\bigr)
                  \exp\bigl(-\sqrt{3} \lvert x - x' \rvert\bigr).
  \end{align*}
  Note that I am not parameterizing any of these kernels; please consider them
  to be fixed.

  Each kernel defines a Gaussian process model for the data in a natural way:
  \[p(f \given \model_i) = \mc{GP}(f; \mu, K_i).\]
  Consider a uniform prior distribution over these models:
  \[\Pr(\model_i) = \nicefrac{1}{3} \qquad i = 1, 2, 3.\]

  \begin{enumerate}
  \item
    Compute the log marginal likelihood for each model given the data $\data$
    above.
  \item
    Compute the model posterior $\Pr(\model \given \data)$. Is there strong
    evidence for one model being the best?
  \end{enumerate}

\item
  Now we will consider integration.

  Perform Bayesian quadrature to estimate the definite integral $\int_{-5}^5
  f(x) \intd{x}$, using the model $\model_1$ from question 1.  What is the
  predictive mean and standard deviation, $p(Z \given \data, \model_1)$?  (You
  may give a numeric answer.)  How does this compare with the true answer?

\end{enumerate}

\end{document}
