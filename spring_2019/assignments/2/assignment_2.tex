\documentclass{article}

\usepackage[T1]{fontenc}
\usepackage[osf]{libertine}
\usepackage[scaled=0.8]{beramono}
\usepackage[margin=1.5in]{geometry}
\usepackage{url}
\usepackage{booktabs}
\usepackage{amsmath}
\usepackage{nicefrac}
\usepackage{microtype}
\usepackage{bm}

\usepackage{sectsty}
\sectionfont{\large}
\subsectionfont{\normalsize}

\usepackage{titlesec}
\titlespacing{\section}{0pt}{10pt plus 2pt minus 2pt}{0pt plus 2pt minus 0pt}
\titlespacing{\subsection}{0pt}{5pt plus 2pt minus 2pt}{0pt plus 2pt minus 0pt}

\setlength{\parindent}{0pt}
\setlength{\parskip}{1ex}

\newcommand{\acro}[1]{\textsc{\MakeLowercase{#1}}}
\newcommand{\given}{\mid}
\newcommand{\mc}[1]{\mathcal{#1}}
\newcommand{\data}{\mc{D}}
\newcommand{\intd}[1]{\,\mathrm{d}{#1}}
\newcommand{\mat}[1]{\bm{\mathrm{#1}}}
\renewcommand{\vec}[1]{\bm{\mathrm{#1}}}
\newcommand{\trans}{^\top}

\begin{document}

{\large \textbf{CSE 515T (Spring 2019) Assignment 2}} \\
Due Monday, 25 March 2019 \\

\begin{enumerate}

\item
  (Curse of dimensionality.)
  Consider a $d$-dimensional, zero-mean, spherical multivariate
  Gaussian distribution:
  \begin{equation*}
    p(\vec{x}) = \mc{N}(\vec{x}; \vec{0}, \mat{I}_d).
  \end{equation*}
  Equivalently, each entry of $\vec{x}$ is drawn iid from a univariate
  standard normal distribution.

  In familiar small dimensions ($d \leq 3$), ``most'' of the vectors
  drawn from a multivariate Gaussian distribution will lie near the
  mean.  For example, the famous 68--95--99.7 rule for $d = 1$
  indicates that large deviations from the mean are unusual.  Here we
  will consider the behavior in larger dimensions.
  \begin{itemize}
  \item Draw 10\,000 samples from $p(\vec{x})$ for each dimension in
    $d \in \{1, 5, 10, 50, 100\},$ and compute the length of each
    vector drawn: $y_d = \sqrt{\vec{x}\trans \vec{x}} = (\sum_i^d
    x_i^2)^{\nicefrac{1}{2}}$.  Estimate the distribution of each
    $y_d$ using either a histogram or a kernel density estimate (in
    \acro{MATLAB}, \texttt{hist} and \texttt{ksdensity},
    respectively).  Plot your estimates.  (Please do not hand in your
    raw samples!)  Summarize the behavior of this distribution as $d$
    increases.
  \item
    The true distribution of $y_d^2$ is a chi-square distribution with
    $d$ degrees of freedom (the distribution of $y_d$ itself is the
    less-commonly seen chi distribution).  Use this fact to compute
    the probability that $y_d < 5$ for each of the dimensions in the
    last part.
  \item
    For $d = 1\,000$, compute the 5th and 95th percentiles of $y_d$.
    Is the mean $\vec{x} = \vec{0}$ a representative summary of the
    distribution in high dimensions?  This behavior has been called
    ``the curse of dimensionality.''
  \end{itemize}

\item
  (Laplace approximation.)
  Find a Laplace approximation to the gamma distribution:
  \begin{equation*}
    p(\theta \given \alpha, \beta)
    =
    \frac{1}{Z}
    \theta^{\alpha - 1}
    \exp(-\beta\theta).
  \end{equation*}
  Plot the approximation against the true density for $(\alpha, \beta)
  = (3, 1)$.

  The true value of the normalizing constant is
  \begin{equation*}
    Z = \frac{\Gamma(\alpha)}{\beta^\alpha}.
  \end{equation*}
  If we fix $\beta = 1$, then $Z = \Gamma(\alpha)$, so we may use the
  Laplace approximation to estimate the Gamma function.  Analyze the
  quality of this approximation as a function of $\alpha$.

  Read the Wikipedia article about Stirling's approximation. Do you see
  a connection?

\item
  (Gaussian process regression).
  Consider the following data:
  \begin{align*}
    \vec{x}
    &=
    [-2.26, -1.31, -0.43, 0.32, 0.34, 0.54, 0.86, 1.83, 2.77, 3.58]\trans; \\
    \vec{y}
    &=
    [1.03, 0.70, -0.68, -1.36, -1.74, -1.01, 0.24, 1.55, 1.68, 1.53]\trans.
  \end{align*}
  Fix the observation noise variance at $\sigma^2 = 0.5^2$.

  \begin{itemize}
  \item
    Examining a scatter plot of the data, guess which values of
    $(\lambda, \ell)$ in the squared exponential covariance (if any)
    might explain this data well.
  \item
    Perform Gaussian process regression for these data on the interval
    $x_\ast \in [-4, 4]$ using the squared exponential covariance for
    the same set of hyperparameters $(\lambda, \ell)$ above.  Plot the
    posterior mean and the pointwise 95\% credible interval for each.
    Which predictions look the best?
  \item
    Visualize the log model evidence $\log p(\vec{y} \given \vec{x},
    \lambda, \ell, \sigma^2)$ as a function of $(\lambda, \ell)$.  You
    can choose to make, for example, a heatmap or a contour plot of
    this function.
  \end{itemize}

\end{enumerate}

\end{document}
