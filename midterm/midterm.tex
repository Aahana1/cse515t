\documentclass{article}

\usepackage[T1]{fontenc}
\usepackage[osf]{libertine}
\usepackage[scaled=0.8]{beramono}
\usepackage[margin=1.5in]{geometry}
\usepackage{url}
\usepackage{booktabs}
\usepackage{amsmath}
\usepackage{nicefrac}
\usepackage{microtype}
\usepackage{bm}

\usepackage{sectsty}
\sectionfont{\large}
\subsectionfont{\normalsize}

\usepackage{titlesec}
\titlespacing{\section}{0pt}{10pt plus 2pt minus 2pt}{0pt plus 2pt minus 0pt}
\titlespacing{\subsection}{0pt}{5pt plus 2pt minus 2pt}{0pt plus 2pt minus 0pt}

\setlength{\parindent}{0pt}
\setlength{\parskip}{1ex}

\newcommand{\acro}[1]{\textsc{\MakeLowercase{#1}}}
\newcommand{\given}{\mid}
\newcommand{\mc}[1]{\mathcal{#1}}
\newcommand{\data}{\mc{D}}
\newcommand{\trans}{^\top}
\newcommand{\intd}[1]{\,\mathrm{d}{#1}}
\newcommand{\mat}[1]{\bm{\mathrm{#1}}}
\renewcommand{\vec}[1]{\bm{\mathrm{#1}}}

\begin{document}

{\large \textbf{CSE 515T (Spring 2015) Midterm}}
\begin{itemize}
\item
  There are two ways to hand in this midterm.  \textbf{Late
    submissions will not be accepted!}  I do not recommend cutting it
  too close.
  \begin{itemize}
  \item Physically in my department mailbox in the \acro{CSE} head
    office on the fifth floor of Bryan Hall.  The due date for this
    option is \textbf{5:00 \acro{PM}, Friday, 6 March.}
  \item Electronically on Piazza as a private message to the
    instructors. The due date for this option is \textbf{midnight,
      Saturday, 7 March.}
  \end{itemize}
\item
  Please do not discuss the questions with other members of the class.
\item
  Please post any questions as a \emph{private message to the
    instructors} on Piazza.
\item
  Any corrections will be posted by the instructors on Piazza. This
  document will also be kept up-to-date on the course webpage and in
  GitHub.
\end{itemize}

\clearpage

\begin{enumerate}
\item
  Consider two coins with unknown bias $\theta_1$ and $\theta_2$,
  respectively.  We place independent, identical beta priors on these
  quantities:
  \begin{equation*}
    p(\theta_1) = \mc{B}(\theta_1; 2, 2);
    \qquad
    p(\theta_2) = \mc{B}(\theta_2; 2, 2).
  \end{equation*}
  Imagine someone flips both coins and tells you that \emph{exactly
    one} of the outcomes (but not which) was a ``head.''  Thus the
  observation was either HT or TH, but you are not told which.  The
  below expressions are conditioned on ``H'' to indicate this
  observation.
  \begin{itemize}
  \item
    Give an expression for the posterior of the first coin's bias
    given this observation, $p(\theta_1 \given \text{H})$.  Simplify
    the result as much as you can.  Plot the prior and the posterior
    for $\theta_1$ over the interval $\theta_1 \in (0, 1)$.
  \item
    Give an expression for the joint posterior $p(\theta_1, \theta_2
    \given \text{H})$. Plot the joint prior, the likelihood, and the
    joint posterior as three separate heat maps over the unit square
    $(\theta_1, \theta_2) \in (0, 1)^2$.  Use a grid with at least 100
    values along each of the two $\theta$ axes.
  \item
    Summarize what the observation taught us about the bias of the
    coins.
  \end{itemize}

\item
  Consider the three-dimensional parameter vector $\vec{\theta} =
  [\theta_1, \theta_2, \theta_3]\trans$, with the following joint
  multivariate Gaussian prior:
  \begin{equation*}
    p(\vec{\theta})
    =
    \mc{N}(\vec{\theta}; \vec{\mu}, \mat{\Sigma})
    =
    \mc{N}
    \left(
    \begin{bmatrix}
      \theta_1 \\
      \theta_2 \\
      \theta_3
    \end{bmatrix}
    ;
    \begin{bmatrix}
      0 \\
      1 \\
      2
    \end{bmatrix},
    \begin{bmatrix}
      1 & 2 & 0   \\
      2 & 9 & 0 \\
      0 & 0 & 16
    \end{bmatrix}
    \right).
  \end{equation*}
  We are going to consider a decision problem with action space
  $\mc{A} = \{1, 2, 3\}$.  The result of choosing an action $a \in
  \mc{A}$ will be to observe the exact value of $\theta_a$, the $a$th
  element of $\vec{\theta}$.

  Consider the following loss functions, $\ell_1$ and $\ell_2$:
  \begin{equation*}
    \ell_1(\vec{\theta}, a) =
    \begin{cases}
      1 & \theta_a   >  0 \\
      0 & \theta_a \leq 0
    \end{cases}
    \qquad
    \ell_2(\vec{\theta}, a) = \min(0, \theta_a).
  \end{equation*}
  For each:
  \begin{itemize}
  \item
    Write a generic expression for the expected loss of action $a$ in
    terms of $\vec{\mu}$ and $\mat{\Sigma}$.  Evaluate any integrals
    you encounter.
  \item
    Give a numerical value for the expected loss of each action, using
    the values of $(\vec{\mu}, \mat{\Sigma})$ provided above.
  \item
    State the Bayes action.
  \end{itemize}

\item
  Consider a $d$-dimensional vector $\vec{\theta}$ with an arbitrary
  multivariate Gaussian distribution:
  \begin{equation*}
    p(\vec{\theta})
    =
    \mc{N}(\vec{\theta}; \vec{\mu}, \mat{\Sigma}).
  \end{equation*}
  \begin{itemize}
  \item
    Give a general expression for the distribution of the following
    (scalar) value $\tau$.
    \begin{equation*}
      \tau = \theta_1 + 2\theta_2 + \dotsb d\theta_d
    \end{equation*}
  \item
    Consider again the specific distribution of the three-dimensional
    vector $\vec{\theta}$ from the last problem, as well as the action
    space $\mc{A}$ with the same observation mechanism: after choosing
    $a \in \mc{A}$, we will observe the corresponding value
    $\theta_a$.  Suppose we may select one action and then must
    predict $\tau$ under a squared loss function:
    \begin{equation*}
      \ell(\tau, \hat{\tau}) = (\tau - \hat{\tau})^2.
    \end{equation*}
    Using the distribution from the last problem. what is the expected
    loss of each of the three available actions?  Which is the Bayes
    action?
  \end{itemize}

\item
  Consider the following data:
  \begin{align*}
    \vec{x} &= [0.54, 1.84, -2.26, 0.86, 0.32]\trans; \\
    \vec{y} &= [-1.31, -0.43, 0.34, 3.58, 2.77]\trans.
  \end{align*}
  Consider the Bayesian linear regression model with $\phi(x) = [1,
    x]\trans$.  Use the prior $p(\vec{w}) = \mc{N}(\vec{w}; \vec{0},
  \mat{I})$.

  Plot the posterior probability that the slope of the regression line
  is positive as a function of the standard deviation of the
  observation noise $\sigma$ (the noise variance is then $\sigma^2$).
  Use a grid of at least 100 points in the range $\sigma \in (0.01,
  10)$.

\end{enumerate}

\end{document}
